\mysection{OpenMP}
\label{openmp}
\myindex{OpenMP}

OpenMP est l'un des moyens les plus simple de paralléliser des algorithmes simples.

\myindex{Bitcoin}

À titre d'exemple, essayons de construire un programme pour calculer une \emph{nonce}
cryptographique.

Dans mon exemple simpliste, le \emph{nonce} est un nombre ajouté au texte non chiffré
afin de produire un hash avec quelques caractéristiques spécifiques.

Par exemple, à certaines étapes, le protocole Bitcoin nécessite de trouver de tels
\emph{nonce} dont le hash résultant contient un nombre spécifique de zéros consécutifs.
Ceci est aussi appelé \q{preuve de travail}
\footnote{\href{http://go.yurichev.com/17101}{Wikipédia}}
(i.e., le système prouve qu'il a fait des calculs intensifs et y a passé du temps).

\myindex{SHA512}
Mon exemple n'est en aucun cas lié au Bitcoin, il va essayer d'ajouter des nombres
à la chaîne afin de trouver un nombre tel que le hash de \q{hello, world!\_<number>}
avec l'algorithme SHA512, contiendra au moins 3 octets à zéro.

Limitons notre recherche brute-force dans l'intervalle 0..INT32\_MAX-1 (i.e., \TT{0x7FFFFFFE}
ou 2147483646).

L'algorithme est assez direct:

\lstinputlisting[style=customc]{advanced/700_openmp/openmp_example.c}

La fonction \TT{check\_nonce()} ajoute simplement un nombre à la chaîne, hashe le
résultat avec l'algorithme SHA12 et teste si il y a 3 octets à zéro dans le résultat.

Une partie très importante du code est:

\begin{lstlisting}[style=customc]
	#pragma omp parallel for
	for (i=0; i<INT32_MAX; i++)
		check_nonce (i);
\end{lstlisting}

Oui, c'est simple, sans le \TT{\#pragma} nous appelons \TT{check\_nonce()} pour chaque
nombre de 0 à \TT{INT32\_MAX} (\TT{0x7fffffff} ou 2147483647).
Avec le \TT{\#pragma}, le compilateur ajoute du code particulier qui découpe l'intervalle
de la boucle en des plus petits, afin de les lancer sur tous les c\oe{}urs de \ac{CPU}
disponible\footnote{N.B.: Ceci est intentionnellement l'exemple le plus simple possible,
mais en pratique, l'utilisation de OpenMP peut être plus difficile et plus complexe.}.

L'exemple peut être compilé\footnote{Les fichiers sha512.(c|h) et u64.h peuvent être
pris de la bibliothèque OpenSSL:
\url{http://go.yurichev.com/17324}}
dans MSVC 2012:
% FIXME1: \footnote{other source code files can be downloaded here: ...} 

\begin{lstlisting}
cl openmp_example.c sha512.obj /openmp /O1 /Zi /Faopenmp_example.asm
\end{lstlisting}

Ou dans GCC:

\begin{lstlisting}
gcc -fopenmp 2.c sha512.c -S -masm=intel
\end{lstlisting}

\subsection{MSVC}

Maintenant voici comment MSVC 2012 génère la boucle principale:

\lstinputlisting[caption=MSVC 2012,style=customasmx86]{advanced/700_openmp/MSVC_loop.asm}

Toutes les fonctions préfixées par \TT{vcomp} sont relatives à OpenMP et sont stockées
dans le fichier vcomp*.dll.
Donc, ici un groupe de threads est démarré.

Regardons \TT{\_main\$omp\$1}:

\lstinputlisting[caption=MSVC 2012,style=customasmx86]{advanced/700_openmp/MSVC_1.asm}

Cette fonction va être démarrée $n$ fois en parallèle, où $n$ est le nombre de c\oe{}urs
du \ac{CPU}.\\
\TT{vcomp\_for\_static\_simple\_init()} 
calcul l'intervalle pour la construction for() du thread courant, dépendant du numéro
du thread courant.

Les valeurs de début et de fin sont stockées dans les variables locales \TT{\$T1}
et \TT{\$T2}.
Vous pouvez également remarquer l'argument \TT{7ffffffeh} (ou 2147483646) comme argument
de la fonction\\
\TT{vcomp\_for\_static\_simple\_init()}---ceci est le nombre d'itérations
de la boucle complète,\\
 qui doit éventuellement être divisée.

Puis nous voyons une nouvelle boucle avec un appel à la fonction \TT{check\_nonce()},
qui fait tout le travail.

Ajoutons du code au début de la fonction \TT{check\_nonce()} pour collecter des statistiques
sur les arguments avec lesquels la fonction a été appelée.

Voici ce que nous pouvons voir lorsque nous le lançons:

\begin{lstlisting}
threads=4
...
checked=2800000
checked=3000000
checked=3200000
checked=3300000
found (thread 3): [hello, world!_1611446522]. seconds spent=3
__min[0]=0x00000000 __max[0]=0x1fffffff
__min[1]=0x20000000 __max[1]=0x3fffffff
__min[2]=0x40000000 __max[2]=0x5fffffff
__min[3]=0x60000000 __max[3]=0x7ffffffe
\end{lstlisting}

Oui, le résultat est correct, les 3 premiers octets sont des zéros:

\begin{lstlisting}
C:\...\sha512sum test
000000f4a8fac5a4ed38794da4c1e39f54279ad5d9bb3c5465cdf57adaf60403
df6e3fe6019f5764fc9975e505a7395fed780fee50eb38dd4c0279cb114672e2 *test
\end{lstlisting}

Le temps de traitement est $\approx2..3$ secondes sur un Intel Xeon E3-1220 3.10 GHz 4-core.
Dans le gestionnaire de taches nous voyons 5 threads:
1 thread principale + 4 autres.
Il n'y a pas d'optimisations faites afin de garder cet exemple aussi petit et clair
que possible.
Mais probablement qu'on pourrait le rendre plus rapide.
Mon \ac{CPU} a 4 c\oe{}urs, c'est pourquoi OpenMP a démarré exactement 4 threads.

En regardant la table des statistiques, nous voyons clairement comment la boucle
a été découpée en 4 parties égales.
Ooui bon, presque égales, si nous ne tenons pas compte du dernier bit.

Il y a aussi des pragmas pour les \glslink{atomic operation}{operations atomiques.}.

Voyons comment ce code est compilé:

\begin{lstlisting}[style=customc]
	#pragma omp atomic
	checked++;

	#pragma omp critical
	if ((checked % 100000)==0)
		printf ("checked=%d\n", checked);
\end{lstlisting}

\lstinputlisting[caption=MSVC 2012,style=customasmx86]{advanced/700_openmp/MSVC_2.asm}

Il semble que la fonction \TT{vcomp\_atomic\_add\_i4()} dans vcomp*.dll est juste
une minuscule fonction avec l'instruction \TT{LOCK XADD}\footnote{En savoir plus sur le préfixe LOCK: \myref{x86_lock}}
dedans.

\TT{vcomp\_enter\_critsect()} 
appelle éventuellement la fonction de l'\ac{API} win32 \\
\TT{EnterCriticalSection()}
\footnote{Vous pouvez en lire plus sur les sections critiques ici: \myref{critical_sections}}.

\subsection{GCC}

GCC 4.8.1 produit un programme qui montre exactement la même table de statistique,

donc, l'implémentation de GCC divise la boucle en parties de la même manière.

\begin{lstlisting}[caption=GCC 4.8.1,style=customasmx86]
	mov	edi, OFFSET FLAT:main._omp_fn.0
	call	GOMP_parallel_start
	mov	edi, 0
	call	main._omp_fn.0
	call	GOMP_parallel_end
\end{lstlisting}

Contrairement à l'implémentation de MSVC, ce que le code de GCC fait est de démarrer
3 threads et lance la quatrième dans le thread courant. Il y a donc 4 threads au
lieu de 5 dans MSVC.

Voici les fonctions \TT{main.\_omp\_fn.0}: 

\lstinputlisting[caption=GCC 4.8.1,style=customasmx86]{advanced/700_openmp/GCC_1.asm}

Ici nous voyons la division clairement: en appelant
\TT{omp\_get\_num\_threads()} et \TT{omp\_get\_thread\_num()}

nous obtenons le nombre de threads lancées, le nombre courant de threads,
et ainsi détermine l'intervalle de la boucle.
Ensuite nous lançons \TT{check\_nonce()}.

GCC a également inséré l'instruction \TT{LOCK ADD} directement dans le code, contrairement
à MSVC, qui génère un appel qui génère un appel à une fonction DLL séparée:

\lstinputlisting[caption=GCC 4.8.1,style=customasmx86]{advanced/700_openmp/GCC_2.asm}

Les fonctions préfixées par GOMP sont de la bibliothèques GNU OpenMP.
Contrairement à vcomp*.dll, son code source est librement disponible:
\href{http://go.yurichev.com/17102}{GitHub}.
