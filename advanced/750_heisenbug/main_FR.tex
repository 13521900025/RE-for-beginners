\mysection{Un autre heisenbug}
\label{GlobalArraysOverflowHeisenbug}

Parfois, un tableau (ou tampon) peut déborder à cause d'une \emph{erreur de poteaux et d'intervalles}:

\lstinputlisting[style=customc]{advanced/750_heisenbug/1.c}

Ceci est ce que ce programme a affiché dans mon cas (GCC 5.4 x86 sans optimisation sur Linux):

\begin{lstlisting}
important_var1=1
important_var2=456
important_var3=3
important_var4=4
important_var5=5
\end{lstlisting}

Lorsque ça se produit, \TT{important\_var2} avait été mise par le compilateur juste
après \TT{array1[]}:

\begin{lstlisting}[caption=objdump -x]
0804a040 g     O .bss   00000200              array1
...
0804a240 g     O .bss   00000004              important_var2
0804a244 g     O .bss   00000004              important_var4
...
0804a248 g     O .bss   00000004              important_var1
0804a24c g     O .bss   00000004              important_var3
0804a250 g     O .bss   00000004              important_var5
\end{lstlisting}

D'autres compilateurs peuvent arranger les variables dans un autre ordre, et une autre
variable sera écrasée.
\myindex{Heisenbug}
Ceci est aussi un \textit{heisenbug} (\myref{heisenbug})---bug qui peut se produire
ou passer inaperçu suivant la version du compilateur et les options d'optimisation.

\myindex{Valgrind}
Si toutes les variables et tableaux sont allouées sur la pile locale, la protection
de la pile peut être déclenchée, ou pas.
Toutefois, Valgrind peut trouver ce genre de bugs.

Un exemple connexe dans le livre (jeu Angband): \myref{Angband}.

