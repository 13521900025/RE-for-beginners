\subsection{x64: MSVC 2013 \Optimizing}

\lstinputlisting[caption=MSVC 2013 x64 \Optimizing,style=customasmx86]{\CURPATH/MSVC2013_x64_Ox_FR.asm}

Tout d'abord, MSVC a inliné le code la fonction \strlen{}, car il en a conclus que
ceci était plus rapide que le \strlen{} habituel + le coût de l'appel et du retour.
Ceci est appelé de l'inlining: \myref{inline_code}.

\myindex{x86!\Instructions!OR}
\myindex{\CStandardLibrary!strlen()}
\label{using_OR_instead_of_MOV}
La première instruction de \strlen{} mis en ligne est\\
\TT{OR RAX, 0xFFFFFFFFFFFFFFFF}. 

MSVC utilise souvent \TT{OR} au lieu de \TT{MOV RAX, 0xFFFFFFFFFFFFFFFF}, car l'opcode
résultant est plus court.

Et bien sûr, c'est équivalent: tous les bits sont mis à 1, et un nombre avec tous
les bits mis vaut $-1$ en complément à 2:\myref{sec:signednumbers}.

On peut se demander pourquoi le nombre $-1$ est utilisé dans \strlen{}.
À des fins d'optimisation, bien sûr.
Voici le code que MSVC a généré:

\lstinputlisting[caption=Inlined \strlen{} by MSVC 2013 x64,style=customasmx86]{\CURPATH/strlen_MSVC_FR.asm}

Essayez d'écrite plus court si vous voulez initialiser le compteur à 0!
OK, essayons:

\lstinputlisting[caption=Our version of \strlen{},style=customasmx86]{\CURPATH/my_strlen_FR.asm}

Nous avons échoué. Nous devons utilisé une instruction \INS{JMP} additionnelle!

Donc, ce que le compilateur de MSVC 2013 a fait, c'est de déplacer l'instruction
\TT{INC} avant le chargement du caractère courant.

Si le premier caractère est 0, c'est OK, \RAX contient 0 à ce moment, donc la longueur
de la chaîne est 0.

Le reste de cette fonction semble facile à comprendre.

\subsection{x64: GCC 4.9.1 \NonOptimizing}

\lstinputlisting[style=customasmx86]{\CURPATH/GCC491_x64_O0_FR.asm}

Les commentaires ont été ajoutés par l'auteur du livre.

Après l'exécution de \strlen{}, le contrôle est passé au label L2, et ici deux clauses
sont vérifiées, l'une après l'autre.

\myindex{\CLanguageElements!Short-circuit}
La seconde ne sera jamais vérifiée, si la première (\emph{str\_len==0}) est fausse
(ceci est un \q{short-circuit} (court-circuit)).

Maintenant regardons la forme courte de cette fonction:

\begin{itemize}
\item Première partie de for() (appel à \strlen{})
\item goto L2
\item L5: corps de for(). sauter à la fin, si besoin
\item troisième partie de for() (décrémenter str\_len)
\item L2: 
deuxième partie de for(): vérifier la première clause, puis la seconde. sauter au
début du corps de la boucle ou sortir.
\item L4: // sortir
\item renvoyer s
\end{itemize}

\subsection{x64: GCC 4.9.1 \Optimizing}
\label{string_trim_GCC_x64_O3}

\lstinputlisting[style=customasmx86]{\CURPATH/GCC491_x64_O3_FR.asm}

Maintenant, c'est plus complexe.

Le code avant le début du corps de la boucle est exécuté une seule fois, mais il contient
le test des caractères \CRLF{} aussi!
À quoi sert cette duplication du code?

La façon courante d'implémenter la boucle principale est sans doute ceci:

\begin{itemize}
\item (début de la boucle) tester la présence des caractères \CRLF{}, décider
\item stocker le caractère zéro
\end{itemize}

Mais GCC a décidé d'inverser ces deux étapes.

Bien sûr,  \emph{stocker le caractère zéro} ne peut pas être la première étape, donc
un autre test est nécessaire:

\begin{itemize}
\item traiter le premier caractère. matcher avec \CRLF{}, sortir si le caractère
n'est pas \CRLF{}

\item (début de la boucle) stocker le caractère zéro

\item tester la présence des caractères \CRLF{}, décider
\end{itemize}

Maintenant la boucle principale est très courte, ce qui est bon pour les derniers
\ac{CPU}s.

Le code n'utilise pas la variable str\_len, mais str\_len-1.
Donc c'est plus comme un index dans un buffer.

Apparemment, GCC a remarqué que l'expression str\_len-1 est utilisée deux fois.

Donc, c'est mieux d'allouer une variable qui contient toujours une valeur qui est
plus petite que la longueur actuelle de la chaîne de un, et la décrémente (ceci a
le même effet que de décrémenter la variable str\_len).
