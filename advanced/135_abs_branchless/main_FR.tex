\mysection{Fonction \emph{abs()} sans branchement}
\label{chap:branchless_abs}

Retravaillons un exemple que nous avons vu avant \myref{sec:abs} et demandons-nous,
est-il possible de faire une version sans branchement de la fonction en code x86?

\lstinputlisting[style=customc]{abs.c}

Et la réponse est oui.

\subsection{GCC 4.9.1 x64 \Optimizing}

Nous pouvons le voir si nous compilons en utilisant GCC 4.9 avec optimisation:

\lstinputlisting[caption=GCC 4.9 x64 \Optimizing,style=customasmx86]{\CURPATH/abs_GCC491_x64_O3_FR.asm}

Voici comment ça fonctionne:

Décaler arithmétiquement la valeur en entrée par 31.

Le décalage arithmétique implique l'extension du signe, donc si le \ac{MSB} est 1,
les 32 bits seront tous remplis avec  1, ou avec 0 sinon.
\myindex{x86!\Instructions!SAR}

Autrement dit, l'instruction \TT{SAR REG, 31} donne \TT{0xFFFFFFFF} si le signe
était négatif et 0 s'il était positif.

Après l'exécution de \TT{SAR}, nous avons cette valeur dans \EDX.
\myindex{x86!\Instructions!XOR}

Puis, si la valeur est \TT{0xFFFFFFFF} (i.e., le signe est négatif), la valeur en
entrée est inversée (car \TT{XOR REG, 0xFFFFFFFF} est en effet une opération qui
inverse tous les bits).

Puis, à nouveau, si la valeur est \TT{0xFFFFFFFF} (i.e., le signe est négatif), 1
est ajouté au résultat final (car soustraire $-1$ d'une valeur revient à l'incrémenter).

Inverser tous les bits et incrémenter est exactement la façon dont une valeur en
complément à deux est multipliée par -1.
\myref{sec:signednumbers}.

Nous pouvons observer que les deux dernières instructions font quelque chose si le
signe de la valeur en entrée est négatif.

Autrement (si le signe est positif) elles ne font rien du tout, laissant la valeur
en entrée inchangée.

L'algorithme est expliqué dans \InSqBrackets{\HenryWarren 2-4}.

Il est difficile de dire comment a fait GCC, l'a-t-il déduit lui-même ou un pattern
correspondant parmi ceux connus?

\subsection{GCC 4.9 ARM64 \Optimizing}

GCC 4.9 pour ARM64 génère en gros la même chose, il décide juste d'utiliser des registres
64-bit complets.

Il y a moins d'instructions, car la valeur en entrée peut être décalée en utilisant
un suffixe d'instruction (\q{asr}) au lieu d'une instruction séparée.

\lstinputlisting[caption=GCC 4.9 ARM64 \Optimizing,style=customasmARM]{\CURPATH/abs_GCC49_ARM64_O3_FR.s}
