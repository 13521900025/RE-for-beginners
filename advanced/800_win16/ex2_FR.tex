\subsection{\Example{} \#2}
\label{win16_messagebox}

\begin{lstlisting}[style=customc]
#include <windows.h>

int PASCAL WinMain( HINSTANCE hInstance,
                    HINSTANCE hPrevInstance,
                    LPSTR lpCmdLine,
                    int nCmdShow )
{
	MessageBox (NULL, "hello, world", "caption", MB_YESNOCANCEL);
	return 0;
};
\end{lstlisting}

\begin{lstlisting}[style=customasmx86]
WinMain         proc near
                push    bp
                mov     bp, sp
                xor     ax, ax          ; NULL
                push    ax
                push    ds
                mov     ax, offset aHelloWorld ; 0x18. "hello, world"
                push    ax
                push    ds
                mov     ax, offset aCaption ; 0x10. "caption"
                push    ax
                mov     ax, 3           ; MB\_YESNOCANCEL
                push    ax
                call    MESSAGEBOX
                xor     ax, ax          ; return 0
                pop     bp
                retn    0Ah
WinMain         endp

dseg02:0010 aCaption        db 'caption',0
dseg02:0018 aHelloWorld     db 'hello, world',0
\end{lstlisting}

\myindex{x86!\Instructions!RET}
Quelques points importants ici: la convention d'appel \TT{PASCAL} impose de passer
le premier argument en premier (\TT{MB\_YESNOCANCEL}), et le dernier argument --- en dernier (NULL).
Cette convention demande aussi à l'\glslink{callee}{appelant} de restaurer le \glslink{stack pointer}{pointeur de pile}:
D'où l'instruction \TT{RETN} qui a \TT{0Ah} comme argument, ce qui implique que le
pointeur sera incrémenté de 10 octets lorsque l'on sortira de la fonction.
C'est comme stdcall (\myref{sec:stdcall}), mais les arguments sont passés dans l'ordre
\q{naturel}.

Les pointeurs sont passés par paire: d'abord le segment de données, puis le pointeur
dans le segment.
Il y a seulemnt un segment dans cet exemple, donc \TT{DS} pointe toujours sur le segment
de données de l'exécutable.
