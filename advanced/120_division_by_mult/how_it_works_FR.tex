\subsection{Comment ça marche}

Des mathématiques du niveau de l'école, nous pouvons nous souvenir que la division
par 9 peut être remplacée par la multiplication par $\frac{1}{9}$.
En fait, parfois les compilateurs font cela pour l'arithmétique en virgule flottante,
par exemple, l'instruction \INS{FDIV} en code x86 peut être remplacée par \INS{FMUL}.
Au moins MSVC 6.0 va remplacer la division par 9 par un multiplication par $0.111111...$
et parfois il est difficile d'être sûr de quelle opération il s'agissait dans le
code source.

Mais lorsque nous opérons avec des valeurs entières et des registres CPU entier,
nous ne pouvons pas utiliser de fractions.
Toutefois, nous pouvons retravailler la fraction comme ceci:

% FIXME: equation size
\begin{center}
$result = \frac{x}{9} = x \cdot \frac{1}{9} = x \cdot \frac{1 \cdot MagicNumber}{9 \cdot MagicNumber}$
\end{center}

Avec le fait que la division par $2^n$ est très rapide (en utilisant des décalages),
nous devons maintenant trouver quels $MagicNumber$, pour lesquels l'équation suivante
sera vraie: $2^n = 9 \cdot MagicNumber$.

La division par $2^{32}$ est quelque peu cachée: la partie basse 32-bit du produit
dans EAX n'est pas utilisée (ignorée), seule la partie haute 32-bit du produit
(dans EDX) est utilisée et ensuite décalée de 1 bit additionnel.

Autrement dit, le code assembleur que nous venons de voir multiplie par {\Large $\frac{954437177}{2^{32+1}}$},
ou divise par {\Large $\frac{2^{32+1}}{954437177}$}.
Pour trouver le diviseur, nous avons juste à diviser le numérateur par le dénominateur.
En utilisant Wolfram Alpha, nous obtenons 8.99999999.... comme résultat (qui est
proche de 9).

% TODO ref to https://yurichev.com/blog/signed_division_using_shifts/

En lire plus à ce sujet dans \InSqBrackets{\HenryWarren 10-3}.

Beaucoup de gens manquent la division ``cachée'' par $2^{32}$ or $2^{64}$, lorsque
la partie basse 32-bit (ou la partie 64-bit) du produit n'est pas utilisée.
% TBT
% This is why division by multiplication is difficult to understand at the beginning.

