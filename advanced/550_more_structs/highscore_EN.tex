\subsection{High-score file in \q{Block out} game and primitive serialization}

Many videogames has high-score file, sometimes called \q{Hall of fame}.
Ancient \q{Block out}\footnote{\url{http://www.bestoldgames.net/eng/old-games/blockout.php}} game
(3D tetris from 1989) isn't exception, here is what we see at the end:

\begin{figure}[H]
\centering
\includegraphics[width=0.7\textwidth]{advanced/550_more_structs/blockout/hs.png}
\caption{High score table}
\end{figure}

Now we can see that the file has changed after we added our name is \emph{BLSCORE.DAT}.
\begin{lstlisting}
% xxd -g 1 BLSCORE.DAT

00000000: 0a 00 58 65 6e 69 61 2e 2e 2e 2e 2e 00 df 01 00  ..Xenia.........
00000010: 00 30 33 2d 32 37 2d 32 30 31 38 00 50 61 75 6c  .03-27-2018.Paul
00000020: 2e 2e 2e 2e 2e 2e 00 61 01 00 00 30 33 2d 32 37  .......a...03-27
00000030: 2d 32 30 31 38 00 4a 6f 68 6e 2e 2e 2e 2e 2e 2e  -2018.John......
00000040: 00 46 01 00 00 30 33 2d 32 37 2d 32 30 31 38 00  .F...03-27-2018.
00000050: 4a 61 6d 65 73 2e 2e 2e 2e 2e 00 44 01 00 00 30  James......D...0
00000060: 33 2d 32 37 2d 32 30 31 38 00 43 68 61 72 6c 69  3-27-2018.Charli
00000070: 65 2e 2e 2e 00 ea 00 00 00 30 33 2d 32 37 2d 32  e........03-27-2
00000080: 30 31 38 00 4d 69 6b 65 2e 2e 2e 2e 2e 2e 00 b5  018.Mike........
00000090: 00 00 00 30 33 2d 32 37 2d 32 30 31 38 00 50 68  ...03-27-2018.Ph
000000a0: 69 6c 2e 2e 2e 2e 2e 2e 00 ac 00 00 00 30 33 2d  il...........03-
000000b0: 32 37 2d 32 30 31 38 00 4d 61 72 79 2e 2e 2e 2e  27-2018.Mary....
000000c0: 2e 2e 00 7b 00 00 00 30 33 2d 32 37 2d 32 30 31  ...{...03-27-201
000000d0: 38 00 54 6f 6d 2e 2e 2e 2e 2e 2e 2e 00 77 00 00  8.Tom........w..
000000e0: 00 30 33 2d 32 37 2d 32 30 31 38 00 42 6f 62 2e  .03-27-2018.Bob.
000000f0: 2e 2e 2e 2e 2e 2e 00 77 00 00 00 30 33 2d 32 37  .......w...03-27
00000100: 2d 32 30 31 38 00                                -2018.
\end{lstlisting}

All entries are clearly visible.
The very first byte is probably number of entries.
Second is zero and, in fact, number of entries can be 16-bit value spanning over first two bytes.

Next, after \q{Xenia} name we see 0xDF and 0x01 bytes.
Xenia has score of 479, and this is exactly 0x1DF in hexadecimal radix.
So a high score value is probably 16-bit integer, or maybe 32-bit integer: there are two more zero bytes after.

Now let's think about the fact that both array elements and structure elements are always placed in memory adjacently to each other.
\myindex{\CStandardLibrary!write()}
\myindex{\CStandardLibrary!fwrite()}
\myindex{\CStandardLibrary!read()}
\myindex{\CStandardLibrary!fread()}
That enables us to write the whole array/structure to the file using simple \emph{write()} or \emph{fwrite()} function, 
and then restore it using \emph{read()} or \emph{fread()}, as simple as that.
This is what is called \emph{serialization} nowadays.

\subsubsection{Read}

Now let's write C program to read highscore file:

\begin{lstlisting}[style=customc]
#include <assert.h>
#include <stdio.h>
#include <stdint.h>
#include <string.h>

struct entry
{
	char name[11]; // incl. terminating zero
	uint32_t score;
	char date[11]; // incl. terminating zero
} __attribute__ ((aligned (1),packed));

struct highscore_file
{
	uint8_t count;
	uint8_t unknown;
	struct entry entries[10];
} __attribute__ ((aligned (1), packed));

struct highscore_file file;

int main(int argc, char* argv[])
{
	FILE* f=fopen(argv[1], "rb");
	assert (f!=NULL);
	size_t got=fread(&file, 1, sizeof(struct highscore_file), f);
	assert (got==sizeof(struct highscore_file));
	fclose(f);
	for (int i=0; i<file.count; i++)
	{
		printf ("name=%s score=%d date=%s\n",
				file.entries[i].name,
				file.entries[i].score,
				file.entries[i].date);
	};
};
\end{lstlisting}

We need GCC \emph{((aligned (1),packed))} attribute so that all structure fields will be packed on 1-byte boundary.

Of course it works:

\begin{lstlisting}
name=Xenia..... score=479 date=03-27-2018
name=Paul...... score=353 date=03-27-2018
name=John...... score=326 date=03-27-2018
name=James..... score=324 date=03-27-2018
name=Charlie... score=234 date=03-27-2018
name=Mike...... score=181 date=03-27-2018
name=Phil...... score=172 date=03-27-2018
name=Mary...... score=123 date=03-27-2018
name=Tom....... score=119 date=03-27-2018
name=Bob....... score=119 date=03-27-2018
\end{lstlisting}

(Needless to say, each name is padded with dots, both on screen and in the file, perhaps, for \ae{}sthetical reasons.)

\subsubsection{Write}

Let's check if we right about width of score value. Is it really has 32 bits?

\begin{lstlisting}[style=customc]
int main(int argc, char* argv[])
{
	FILE* f=fopen(argv[1], "rb");
	assert (f!=NULL);
	size_t got=fread(&file, 1, sizeof(struct highscore_file), f);
	assert (got==sizeof(struct highscore_file));
	fclose(f);

	strcpy (file.entries[1].name, "Mallory...");
	file.entries[1].score=12345678;
	strcpy (file.entries[1].date, "08-12-2016");

	f=fopen(argv[1], "wb");
	assert (f!=NULL);
	got=fwrite(&file, 1, sizeof(struct highscore_file), f);
	assert (got==sizeof(struct highscore_file));
	fclose(f);
};
\end{lstlisting}

Let's run Blockout:

\begin{figure}[H]
\centering
\includegraphics[width=0.7\textwidth]{advanced/550_more_structs/blockout/hs345678.png}
\caption{High score table}
\end{figure}

First two digits (1 and 2) are truncated: 12345678 becomes 345678. Perhaps, this is formatting issues... but the number is almost correct.
Now I'm changing it to 999999 and run again:

\begin{figure}[H]
\centering
\includegraphics[width=0.7\textwidth]{advanced/550_more_structs/blockout/hs999999.png}
\caption{High score table}
\end{figure}

Now it's correct. Yes, high score value is 32-bit integer.

\subsubsection{Is it serialization?}

\dots almost.
Serialization like this is highly popular in scientific and engineering software, where efficiency and speed is much more important
than converting into \ac{XML} or \ac{JSON} and back.

One important thing is that you obviously cannot serialize pointers, because each time you load the file into memory,
all the structures may be allocated in different places.

But: if you work on some kind of low-cost \ac{MCU} with simple \ac{OS} on it
and you have your structures allocated at always same
places in memory, perhaps you can save and restore pointers as well.

\subsubsection{Random noise}

When I prepared this example, I had to run \q{Block out} many times and played for it a bit
to fill high-score table with random names.

And when there were just 3 entries in the file, I saw this:

\begin{lstlisting}
00000000: 03 00 54 6f 6d 61 73 2e 2e 2e 2e 2e 00 da 2a 00  ..Tomas.......*.
00000010: 00 30 38 2d 31 32 2d 32 30 31 36 00 43 68 61 72  .08-12-2016.Char
00000020: 6c 69 65 2e 2e 2e 00 8b 1e 00 00 30 38 2d 31 32  lie........08-12
00000030: 2d 32 30 31 36 00 4a 6f 68 6e 2e 2e 2e 2e 2e 2e  -2016.John......
00000040: 00 80 00 00 00 30 38 2d 31 32 2d 32 30 31 36 00  .....08-12-2016.
00000050: 00 00 57 c8 a2 01 06 01 ba f9 47 c7 05 00 f8 4f  ..W.......G....O
00000060: 06 01 06 01 a6 32 00 00 00 00 00 00 00 00 00 00  .....2..........
00000070: 00 00 00 00 00 00 00 00 00 00 00 00 00 00 00 00  ................
00000080: 00 00 00 00 00 00 00 00 00 00 00 00 00 00 00 00  ................
00000090: 00 00 00 00 00 00 00 00 00 00 00 00 00 00 00 00  ................
000000a0: 00 00 00 00 00 00 00 00 00 00 93 c6 a2 01 46 72  ..............Fr
000000b0: 8c f9 f6 c5 05 00 f8 4f 00 02 06 01 a6 32 06 01  .......O.....2..
000000c0: 00 00 98 f9 f2 c0 05 00 f8 4f 00 02 a6 32 a2 f9  .........O...2..
000000d0: 80 c1 a6 32 a6 32 f4 4f aa f9 39 c1 a6 32 06 01  ...2.2.O..9..2..
000000e0: b4 f9 2b c5 a6 32 e1 4f c7 c8 a2 01 82 72 c6 f9  ..+..2.O.....r..
000000f0: 30 c0 05 00 00 00 00 00 00 00 a6 32 d4 f9 76 2d  0..........2..v-
00000100: a6 32 00 00 00 00                                .2....
\end{lstlisting}

The first byte has value of 3, meaning there are 3 entries.
And there are 3 entries present.
But then we see a random noise at the second half of file.

The noise is probably has its origins in uninitialized data.
Perhaps, \q{Block out} allocated memory for 10 entries somewhere in \gls{heap}, where, obviously,
some pseudorandom noise (left from something else) was present.
Then it set first/second byte, fill 3 entries, and then it never touched 7 entries left, so they are written
to the file as is.

When \q{Block out} loads high score file at the next run, it reads number of entries from the first/second byte (3) and
then completely ignores what is after it.

This is common problem.
Not a problem in strict sense: it's not a bug, but information can be exposed outwards.

\myindex{Microsoft Word}
Microsoft Word versions from 1990s has been often left pieces of previously edited texts into the *.doc* files.
It was some kind of amusement back then, to get a \emph{.doc} file from someone,
then open it in a hexadecimal editor and read something else,
what has been edited on that computer before.

\myindex{Heartbleed}
\myindex{OpenSSL}
The problem can be even much more serious: Heartbleed bug\footnote{\url{https://en.wikipedia.org/wiki/Heartbleed}}
in OpenSSL.

\subsubsection{Homework}

\q{Block out} has several polycubes (flat/basic/extended), size of pit can be configured, etc.
And it seems, for each configuration, \q{Block out} has its own high score table.
I've noticed that some information is probably stored in \emph{BLSCORE.IDX} file.
This can be a homework for hardcore \q{Block out} fans---to understand its structure as well.

The \q{Block out} files are here: \url{http://beginners.re/examples/blockout.zip}
(including the binary high score files I've used in this example).
You can use DosBox to run it.

