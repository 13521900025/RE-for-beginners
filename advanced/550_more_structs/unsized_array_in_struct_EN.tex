\subsection{Unsized array in C structure}

In some win32 structures we can find ones with last field defined as an array of one element:

\begin{lstlisting}[style=customc]
typedef struct _SYMBOL_INFO {
  ULONG   SizeOfStruct;
  ULONG   TypeIndex;
  
  ...

  ULONG   MaxNameLen;
  TCHAR   Name[1];
} SYMBOL_INFO, *PSYMBOL_INFO;
\end{lstlisting}

( \url{https://msdn.microsoft.com/en-us/library/windows/desktop/ms680686(v=vs.85).aspx} )

This is a hack, meaning, the last field is array of unknown size,
which is to be calculated at the time of structure allocation.

Why: \emph{Name} field may be short, so why to define it with some kind of \emph{MAX\_NAME}
constant which can be 128, 256, or even bigger?

Why not to use pointer instead? Then you have to allocate two blocks: one for structure and the other one for string.
This may be slower and may require larger memory overhead.
Also, you need dereference pointer (i.e., read address of the string from the structure)---not a big deal, but some
people say this is still surplus cost.

This is also known as \emph{struct hack}: \url{http://c-faq.com/struct/structhack.html}.

Example:

\begin{lstlisting}[style=customc]
#include <stdio.h>

struct st
{
	int a;
	int b;
	char s[];
};

void f (struct st *s)
{
	printf ("%d %d %s\n", s->a, s->b, s->s);
	// f() can't replace s[] with bigger string - size of allocated block is unknown at this point
};

int main()
{
#define STRING "Hello!"
	struct st *s=malloc(sizeof(struct st)+strlen(STRING)+1); // incl. terminating zero
	s->a=1;
	s->b=2;
	strcpy (s->s, STRING);
	f(s);
};
\end{lstlisting}

In short, it works because C has no array boundary checks. Any array is treated as having infinite size.

Problem: after allocation, the whole size of allocated block for structure is unknown (except for memory manager),
so you can't just replace string with larger string.
You would still be able to do so if the field would be declared as something like \emph{s[MAX\_NAME]}.

In other words, you have a structure plus an array (or string) fused together in the single allocated memory block.
Another problem is what you obviously can't declare two such arrays in single structure, or to declare another field
after such array.

Older compilers require to declare array with at least one element: \emph{s[1]}, newer allows to declare it as variable-sized
array: \emph{s[]}.
This is also called \emph{flexible array member}\footnote{\url{https://en.wikipedia.org/wiki/Flexible_array_member}}
in C99 standard.

Read more about it in
GCC documentation\footnote{\url{https://gcc.gnu.org/onlinedocs/gcc/Zero-Length.html}},
MSDN documentation\footnote{\url{https://msdn.microsoft.com/en-us/library/b6fae073.aspx}}.

Dennis Ritchie (one of C creators) called this trick \q{unwarranted chumminess with the C implementation} 
(perhaps, acknowledging hackish nature of the trick).

Like it or not, use it or not:
it is still another demonstration on how structures are stored in memory, that's why I write about it.

