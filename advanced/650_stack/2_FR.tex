\subsection{Renvoyer une chaîne}

Ceci est un bug classique tiré de \RobPikePractice{}:

\begin{lstlisting}[style=customc]
#include <stdio.h>

char* amsg(int n, char* s)
{
        char buf[100];

        sprintf (buf, "error %d: %s\n", n, s) ;

        return buf;
};

int main()
{
        printf ("%s\n", amsg (1234, "something wrong!"));
};
\end{lstlisting}

Il va planter.
Tout d'abord essayons de comprendre pourquoi.

Ceci est l'état de la pile avant le retour de amsg():

\begin{lstlisting}
(lower addresses)

...

[amsg(): 100 bytes]
[RA]                   <- current SP
[two amsg arguments]
[something else]
[main() local variables]

...

(upper addresses)
\end{lstlisting}

Ensuite amsg() rend le contrôle du flux à \main, jusqu'ici, tout va bien.
Mais \printf est appelée depuis \main, qui, en fait, utilise la pile pour ses propres
besoin, zappant le buffer de 100-octet.
Au mieux, du contenu indéterminé sera affiché.

Difficile à croire, mais je sais comment résoudre ce problème:

\begin{lstlisting}[style=customc]
#include <stdio.h>

char* amsg(int n, char* s)
{
        char buf[100];

        sprintf (buf, "error %d: %s\n", n, s) ;

        return buf;
};

char* interim (int n, char* s)
{
        char large_buf[8000];
        // make use of local array.
        // it will be optimized away otherwise, as useless.
        large_buf[0]=0;
        return amsg (n, s);
};

int main()
{
        printf ("%s\n", interim (1234, "something wrong!"));
};
\end{lstlisting}

Cela va fonctionner si il est compilé avec MSVC 2013 sans optimisation et avec l'option
\TT{/GS-} option\footnote{Supprimer la vérification de sécurité du buffer}.
MSVC avertira: ``warning C4172: returning address of local variable or temporary'',
mais le code s'exécutera et le message sera affiché.
Regardons l'état de la pile au moment où amsg() renvoie le contrôle à interim():

\begin{lstlisting}
(lower addresses)

...

[amsg(): 100 bytes]
[RA]                                     <- current SP
[two amsg() arguments]
[interim() stuff, incl. 8000 bytes]
[something else]
[main() local variables]

...

(upper addresses)
\end{lstlisting}

Maintenant, l'état de la pile au moment où interim() rend le contrôle à \main{}:

\begin{lstlisting}
(lower addresses)

...

[amsg(): 100 bytes]
[RA]
[two amsg() arguments]
[interim() stuff, incl. 8000 bytes]
[something else]                         <- current SP
[main() local variables]

...

(upper addresses)
\end{lstlisting}

Donc lorsque \main appelle \printf, elle utilise l'espace de pile où le buffer d'interim()
était alloué, et ne zappe pas les 100 octets contenant le message d'erreur, car 8000
octets (ou peut-être bien moins) sont suffisants pour tout ce que \printf et les autres
fonctions font!

Ça pourrait aussi fonctionner si il y a plusieurs fonctions entre, comme:
\main $\rightarrow$ f1() $\rightarrow$ f2() $\rightarrow$ f3() ... $\rightarrow$ amsg(),
et alors le résultat de amsg() est utilisé dans \main.
La distance entre \ac{SP} dans \main et l'adresse de \TT{buf[]} doit être assez grande.

C'est pourquoi les bugs de ce genre sont dangereux: parfois votre code fonctionne
(et le bug ne se produit pas), parfois non.
\label{heisenbug}
\myindex{Heisenbug}
Ces genres de bug sont par humour appelés heisenbugs ou schrödinbugs\footnote{\url{https://en.wikipedia.org/wiki/Heisenbug}}.
