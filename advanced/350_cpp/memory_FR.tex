\subsection{Mémoire}

Vous pouvez parfois entendre de la part de programmeurs C++ \q{allouer la mémoire sur la pile}
et/ou \q{allouer la mémoire sur le \gls{heap}}.

Allouer un objet \emph{sur la pile}:

\begin{lstlisting}[style=customc]
void f()
{
	...

	Class o=Class(...);

	...
};
\end{lstlisting}

La mémoire de l'objet (ou de la structure) est allouée sur le pile, en utilisant
un simple décalage de \ac{SP}.
La mémoire est allouée jusqu'à la sortie de la fonction, ou, plus précisément, à
la fin du \emph{scope}---\ac{SP} est remis à son état (comme au début de la fonction)
et le destructeur de \emph{Class} est appelé.
De la même manière, la mémoire allouée pour une structure en C est désallouée à la
sortie de la fonction.

Allouer un objet \emph{dans le \gls{heap}}:

\begin{lstlisting}[style=customc]
void f1()
{
	...

	Class *o=new Class(...);

	...
};

void f2()
{
	...

	delete o;

	...
};
\end{lstlisting}

\myindex{\CStandardLibrary!malloc()}
\myindex{\CStandardLibrary!free()}
Ceci est la même chose que d'allouer de la mémoire pour une structure en utilisant
un appel à \emph{malloc()}.
En fait, \emph{new} en C++ est un wrapper pour \emph{malloc()}, et \emph{delete} est un
wrapper pour \emph{free()}.
Puisque le bloc de mémoire a été allouée sur le \gls{heap}, il doit être désalloué
explicitement, en utilisant \emph{delete}.
Le destructeur de classe sera appelé automatiquement juste avant ce moment.

Quelle méthode est la meilleure?
L'allocation \emph{isur la pile} est très rapide, et bon pour les petits, à durée de
vie courte objets, qui seront utilisés seulement dans la fonction courante.

L'allocation \emph{sur le heap} est plus lente, et meilleure pour des objets à longue
durée de vie, qui seront utilisés dans plusieurs fonctions.
Aussi, les objets alloués sur le \gls{heap} sont sujets à la fuite de mémoire, car
ils doivent être libérés explicitement, mais on peut oublier de le faire.

De toutes façons, ceci est une affaire de goût.
