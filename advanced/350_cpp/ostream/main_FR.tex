\subsection{ostream}
\myindex{\Cpp!ostream}

Recommençons avec l'exemple \q{hello world}, mais cette fois nous allons utiliser
\emph{ostream}:

\lstinputlisting[style=customc]{\CURPATH/ostream/1.cpp}

Presque tous les livres sur \Cpp nous disent que l'opérateur \TT{<<} peut-être défini
(\emph{surchargé}) pour tous les types.
C'est ce qui est fait dans \emph{ostream}.
Nous voyons que \TT{operator<<} est appelé pour \emph{ostream}:

\lstinputlisting[caption=MSVC 2012 (listing réduit),style=customasmx86]{\CURPATH/ostream/1.asm}

Modifions l'exemple:

\lstinputlisting[style=customc]{\CURPATH/ostream/2.cpp}

À nouveau, dans chaque livre sur \Cpp nous lisons que le résultat de chaque \TT{operator<<}
dans ostream est transmis au suivant.
En effet:

\lstinputlisting[caption=MSVC 2012,style=customasmx86]{\CURPATH/ostream/2_EN.asm}

Si nous renommions la méthode \TT{operator<<} en \ttf{}, ce code ressemblerait
à ceci:

\begin{lstlisting}[style=customc]
f(f(std::cout, "Hello, "), "world!");
\end{lstlisting}

GCC génère presque le même code que MSVC.

