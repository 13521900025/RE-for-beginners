\subsubsection{Encapsulation}

L'encapsulation consiste à cacher les données dans des sections \IT{private} de la
classe, e.g. de n'autoriser leurs accès que depuis les méthodes de la classe.

Toutefois, y a-t-il des repères dans le code à propos du fait que certains champs
sont privé et d'autres---non?

Non, il n'y a pas de tels repères.

Essayons avec ce simple exemple:

\lstinputlisting[style=customc]{\CURPATH/classes/classes2_0.cpp}

Compilons-le à nouveau dans MSVC 2008 avec les options \Ox et \Obzero puis regardons
le code de la méthode \TT{box::dump()}:

\lstinputlisting[style=customasmx86]{\CURPATH/classes/classes2_1.asm}

Voici l'agencement mémoire de la classe:

\begin{center}
\begin{tabular}{ | l | l | }
\hline
  \tableheader{} \\
\hline
  +0x0 & int color \\
\hline
  +0x4 & int width \\
\hline
  +0x8 & int height \\
\hline
  +0xC & int depth \\
\hline
\end{tabular}
\end{center}

Tous les champs sont privés et ne peuvent être accédés depuis une autre fonction,
mais connaissant cette disposition, pouvons-nous créer le code qui modifie ces champs?

Pour faire ceci, nous ajoutons la fonction \TT{hack\_oop\_encapsulation()}, qui ne
compilerait pas si elle ressemblait à ceci:

\lstinputlisting[style=customc]{\CURPATH/classes/classes2_2_FR.cpp}

Néanmoins, si nous castons le type \IT{box} sur un \IT{pointeur sur un tableau de int},
que nous modifions le tableau de \Tint{}-s que nous avons, nous pourrions réussir.

\lstinputlisting[style=customc]{\CURPATH/classes/classes2_3.cpp}

Le code de cette fonction est très simple---on peut dire que la fonction prend un
pointeur sur un tableau de \Tint{}-s en entrée et écrit \IT{123} dans le second \Tint{}:

\lstinputlisting[style=customasmx86]{\CURPATH/classes/classes2_5.asm}

Regardons comment ça fonctionne:

\lstinputlisting[style=customc]{\CURPATH/classes/classes2_4.cpp}

Lançons-le:

\begin{lstlisting}
this is a box. color=1, width=10, height=20, depth=30
this is a box. color=1, width=123, height=20, depth=30
\end{lstlisting}

Nous voyons que l'encapsulation est juste une protection des champs de la classe
lors de l'étape de compilation.

Le compilateur \Cpp n'autorise pas la génération de code qui modifie directement
les champs protégés, néanmoins, il est possible de le faire avec l'aide de \IT{dirty hacks}.
