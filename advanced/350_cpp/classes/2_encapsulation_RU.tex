\subsubsection{Инкапсуляция}

Инкапсуляция --- это сокрытие данных в \emph{private} секциях класса, например, чтобы разрешить доступ к ним только 
для методов этого класса, но не более.

Однако, маркируется ли как-нибудь в коде тот сам факт, что некоторое поле --- приватное, 
а некоторое другое --- нет?

Нет, никак не маркируется.

Попробуем простой пример:

\lstinputlisting[style=customc]{\CURPATH/classes/classes2_0.cpp}

Снова скомпилируем в MSVC 2008 с опциями \Ox и \Obzero и посмотрим код метода \TT{box::dump()}:

\lstinputlisting[style=customasmx86]{\CURPATH/classes/classes2_1.asm}

Разметка полей в классе выходит такой:

\begin{center}
\begin{tabular}{ | l | l | }
\hline
  \tableheader{} \\
\hline
  +0x0 & int color \\
\hline
  +0x4 & int width \\
\hline
  +0x8 & int height \\
\hline
  +0xC & int depth \\
\hline
\end{tabular}
\end{center}

Все поля приватные и недоступные для модификации из других функций, но, зная эту разметку, 
сможем ли мы создать код модифицирующий эти поля? 

Для этого добавим функцию \TT{hack\_oop\_encapsulation()}, которая если обладает 
приведенным ниже телом, то просто не скомпилируется:

\lstinputlisting[style=customc]{\CURPATH/classes/classes2_2_RU.cpp}

Тем не менее, если преобразовать тип \emph{box} к типу \emph{указатель на массив int}, 
и если модифицировать полученный массив \Tint{}-ов, тогда всё получится.

\lstinputlisting[style=customc]{\CURPATH/classes/classes2_3.cpp}

Код этой функции довольно прост --- можно сказать, функция берет на вход указатель на массив \Tint{}-ов и 
записывает \emph{123} во второй \Tint{}:

\lstinputlisting[style=customasmx86]{\CURPATH/classes/classes2_5.asm}

Проверим, как это работает:

\lstinputlisting[style=customc]{\CURPATH/classes/classes2_4.cpp}

Запускаем:

\begin{lstlisting}
this is a box. color=1, width=10, height=20, depth=30
this is a box. color=1, width=123, height=20, depth=30
\end{lstlisting}

Выходит, инкапсуляция --- это защита полей класса только на стадии компиляции.

Компилятор ЯП \Cpp не позволяет сгенерировать код прямо
модифицирующий защищенные поля, тем не менее, используя \emph{грязные трюки} --- это вполне возможно.


