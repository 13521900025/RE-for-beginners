\subsubsection{Простой пример}

Внутреннее представление классов в \Cpp почти такое же, как и представление структур.

Давайте попробуем простой пример с двумя переменными, двумя конструкторами и одним методом:

\lstinputlisting[style=customc]{\CURPATH/classes/simple_RU.cpp}

\myparagraph{MSVC: x86}

Вот как выглядит \main на ассемблере:

\lstinputlisting[caption=MSVC,style=customasmx86]{\CURPATH/classes/simple_1_msvc.asm}

Вот что происходит. 
Под каждый экземпляр класса $c$ выделяется по 8 байт, столько же, сколько нужно 
для хранения двух переменных.

Для \emph{c1} вызывается конструктор по умолчанию без аргументов \TT{??0c@@QAE@XZ}. \\
Для \emph{c2} вызывается другой конструктор \TT{??0c@@QAE@HH@Z} и передаются два числа в качестве аргументов.

\myindex{thiscall}
\label{thiscall}
\myindex{x86!\Registers!ECX}
А указатель на объект (\ITthis в терминологии \Cpp) передается в регистре \ECX. 
Это называется thiscall~(\myref{thiscall}) --- метод передачи указателя на объект.

В данном случае, MSVC делает это через \ECX. Необходимо помнить, что это не стандартизированный метод, 
и другие компиляторы могут делать это иначе, например, через первый аргумент функции (как GCC).

\label{namemangling}
\myindex{Name mangling}
\myindex{ООП!Полиморфизм}
Почему у имен функций такие странные имена? Это \gls{name mangling}.

В \Cpp, у класса, может иметься несколько методов с одинаковыми именами, 
но аргументами разных типов --- это полиморфизм. 
Ну и конечно, у разных классов могут быть методы с одинаковыми именами.

\myindex{Компоновщик}
\emph{Name mangling} позволяет закодировать имя класса + имя метода + типы всех аргументов метода 
в одной ASCII-строке, которая затем используется как внутреннее имя функции. 
Это все потому что ни компоновщик\footnote{linker}, ни загрузчик DLL \ac{OS}
(мангленные имена могут быть среди экспортов/импортов в DLL), 
ничего не знают о \Cpp или \ac{OOP}.

Далее вызывается два раза \TT{dump()}.

Теперь смотрим на код в конструкторах:

\lstinputlisting[caption=MSVC,style=customasmx86]{\CURPATH/classes/simple_2_msvc.asm}

Конструкторы --- это просто функции, они используют указатель на структуру в \ECX, 
копируют его себе в локальную переменную, хотя это и не обязательно.

Из стандарта \Cpp мы знаем (C++11 12.1) 
что конструкторы не должны возвращать значение.
В реальности, внутри, конструкторы возвращают указатель на созданный объект, т.е., \ITthis.

И еще метод \TT{dump()}:

\lstinputlisting[caption=MSVC,style=customasmx86]{\CURPATH/classes/simple_3_msvc.asm}

Все очень просто, \TT{dump()} берет указатель на структуру состоящую из двух \Tint через \ECX, 
выдергивает оттуда две переменные и передает их в \printf.

А если скомпилировать с оптимизацией (\Ox), то кода будет намного меньше:

\lstinputlisting[caption=MSVC,style=customasmx86]{\CURPATH/classes/simple_4_msvc_Ox.asm}

\myindex{x86!\Instructions!RET}
Вот и все. Единственное о чем еще нужно сказать, это о том, что в функции \main, 
когда вызывался второй конструктор с двумя аргументами, за ним не корректировался стек при помощи 
\TT{add esp, X}. В то же время, в конце конструктора вместо \RET имеется \TT{RET 8}.

\myindex{thiscall}
Это потому что здесь используется thiscall~(\myref{thiscall}), который, вместе с stdcall~(\myref{sec:stdcall})
(все это --- методы передачи аргументов через стек), предлагает вызываемой функции корректировать стек. 
Инструкция \TT{ret X} сначала прибавляет \TT{X} к \ESP, затем передает управление вызывающей функции.

См. также в соответствующем разделе о способах передачи аргументов через стек
~(\myref{sec:callingconventions}).

Еще, кстати, нужно отметить, что именно компилятор решает, когда вызывать конструктор и деструктор --- но это
и так известно из основ языка \Cpp.

\myparagraph{MSVC: x86-64}
\label{simple_CPP_MSVC_x64}

Как мы уже знаем, в x86-64 первые 4 аргумента функции передаются через регистры \RCX, \RDX, \Reg{8}, 
\Reg{9}, а остальные --- через стек.
Тем не менее, указатель на объект \ITthis передается через \RCX, а первый аргумент метода --- в \RDX, итд.
Здесь это видно во внутренностях метода 
\TT{c(int a, int b)}:

\lstinputlisting[caption=\Optimizing MSVC 2012 x64,style=customasmx86]{\CURPATH/classes/simple_MSVC_x64_RU.asm}

Тип \Tint в x64 остается 32-битным
\footnote{Видимо, так решили для упрощения портирования \CCpp{}-кода на x64}, 
поэтому здесь используются 32-битные части регистров.

В методе \TT{dump()} вместо \RET мы видим \TT{JMP printf}, этот \emph{хак} мы рассматривали ранее: \myref{JMP_instead_of_RET}.

\myparagraph{GCC: x86}

В GCC 4.4.1 всё почти так же, за исключением некоторых различий.

\lstinputlisting[caption=GCC 4.4.1,style=customasmx86]{\CURPATH/classes/simple_GCC.asm}

Здесь мы видим, что применяется иной \emph{name mangling} характерный для стандартов GNU
\footnote{Еще о name mangling разных компиляторов:\\
\InSqBrackets{\AgnerFogCC}.}
Во-вторых, указатель на экземпляр передается как первый аргумент функции --- конечно же, скрыто от программиста.

Это первый конструктор:

\begin{lstlisting}[style=customasmx86]
                public _ZN1cC1Ev ; weak
_ZN1cC1Ev       proc near               ; CODE XREF: main+10

arg_0           = dword ptr  8

                push    ebp
                mov     ebp, esp
                mov     eax, [ebp+arg_0]
                mov     dword ptr [eax], 667
                mov     eax, [ebp+arg_0]
                mov     dword ptr [eax+4], 999
                pop     ebp
                retn
_ZN1cC1Ev       endp
\end{lstlisting}

Он просто записывает два числа по указателю, переданному в первом (и единственном) аргументе.

Второй конструктор:

\begin{lstlisting}[style=customasmx86]
                public _ZN1cC1Eii
_ZN1cC1Eii      proc near

arg_0           = dword ptr  8
arg_4           = dword ptr  0Ch
arg_8           = dword ptr  10h

                push    ebp
                mov     ebp, esp
                mov     eax, [ebp+arg_0]
                mov     edx, [ebp+arg_4]
                mov     [eax], edx
                mov     eax, [ebp+arg_0]
                mov     edx, [ebp+arg_8]
                mov     [eax+4], edx
                pop     ebp
                retn
_ZN1cC1Eii      endp
\end{lstlisting}

Это функция, аналог которой мог бы выглядеть так:

\begin{lstlisting}[style=customc]
void ZN1cC1Eii (int *obj, int a, int b)
{
  *obj=a;
  *(obj+1)=b;
};
\end{lstlisting}

\dots что, в общем, предсказуемо.

И функция \TT{dump()}:

\lstinputlisting[style=customasmx86]{\CURPATH/classes/tmp1.asm}

Эта функция \emph{во внутреннем представлении} имеет один аргумент, через который передается указатель на 
объект\footnote{экземпляр класса} (\ITthis).

Это можно переписать на Си:

\begin{lstlisting}[style=customc]
void ZN1c4dumpEv (int *obj)
{
  printf ("%d; %d\n", *obj, *(obj+1));
};
\end{lstlisting}

Таким образом, если брать в учет только эти простые примеры, разница между MSVC и GCC 
в способе кодирования имен функций (\emph{name mangling}) и передаче указателя на экземпляр класса 
(через \ECX или через первый аргумент).

\myparagraph{GCC: x86-64}

Первые 6 аргументов, как мы уже знаем, передаются через 6 регистров \RDI, \RSI, \RDX, \RCX, \Reg{8} и 
\Reg{9} (\SysVABI), а указатель на \ITthis через первый (\RDI) что мы здесь и видим.
Тип \Tint 32-битный и здесь.
\emph{Хак} с \JMP вместо \RET используется и здесь.

\lstinputlisting[caption=GCC 4.4.6 x64,style=customasmx86]{\CURPATH/classes/simple_GCC_x64_RU.s}

