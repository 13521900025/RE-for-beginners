\subsubsection{std::list}
\myindex{\Cpp!STL!std::list}
\label{std_list}
\index{Doubly linked list}

Ceci est la célèbre liste doublement chaînée: chaque élément a deux pointeurs, un
sur l'élément précédent et un sur le suivant.

Ceci implique que l'empreinte mémoire est augmentés de 2 \glslink{word}{mots} pour
chaque élément (8 octets dans un environnement 32-bit ou 16v octets en 64-bit).

\Cpp STL ajoute juste les pointeurs \q{next} et \q{previous} à la structure existante
du type que vous voulez unir dans une liste.

Travaillons sur un exemple avec une simple structure de deux variables que nous voulons
stocker dans une liste.

Bien que le standard \Cpp ne dise pas comme l'implémenter, GCC et MSVC l'implémentent
de manière directe et similaire, donc il n'y a qu'un seul code source pour les deux:

\lstinputlisting[style=customc]{\CURPATH/STL/list/2_FR.cpp}

\myparagraph{GCC}

Commençons avec GCC.

Lorsque nous lançons l'exemple, nous voyons un long dump, travaillons avec par morceaux.

\begin{lstlisting}
* empty list:
ptr=0x0028fe90 _Next=0x0028fe90 _Prev=0x0028fe90 x=3 y=0
\end{lstlisting}

Ici, nous voyons une liste vide.

Bien que ce soit une liste vide, elle a un élément avec des données non initialisées
(\ac{AKA} \emph{dummy node}) dans $x$ et $y$.
Les deux pointeurs \q{next} et \q{prev} pointeurs sur le n\oe{}ud lui-même:

\input{\CURPATH/STL/list/empty_list}

À ce moment, les itérateurs .begin et .end sont égaux l'un à l'autre.

Si nous poussons 3 éléments, la liste interne sera:

\begin{lstlisting}
* 3-elements list:
ptr=0x000349a0 _Next=0x00034988 _Prev=0x0028fe90 x=3 y=4
ptr=0x00034988 _Next=0x00034b40 _Prev=0x000349a0 x=1 y=2
ptr=0x00034b40 _Next=0x0028fe90 _Prev=0x00034988 x=5 y=6
ptr=0x0028fe90 _Next=0x000349a0 _Prev=0x00034b40 x=5 y=6
\end{lstlisting}

Le dernier élément est encore en 0x0028fe90, il ne sera pas déplacé avant l'élimination
de la liste.

Il contient toujours des valeurs aléatoires dans $x$ et $y$ (5 et 6).
Par coïncidence, ces valeurs sont les même que dans le dernier élément, mais ça ne
signifie pas que ça soit significatif.

Voici comment ces 3 éléments sont stockés en mémoire:

\input{\CURPATH/STL/list/GCC_internals}

La variable $l$ pointe toujours sur le premier n\oe{}ud.

Les itérateurs .begin() et .end() ne sont pas des variables, mais des fonctions,
qui renvoient des pointeurs sur le n\oe{}ud correspondant lorsqu'elle sont appelées.

Avoir un élément fictif (\ac{AKA} \emph{n\oe{}ud sentinelle}) est une pratique répandue
dans l'implémentation des listes doublements chaînées.

Sans lui, de nombreuses opérations deviennent légèrement plus complexes et de ce fait,
plus lentes.

L'itérateur est en fait juste un pointeur sur un n\oe{}ud.
list.begin() et list.end() retourne juste des pointeurs.

\begin{lstlisting}
node at .begin:
ptr=0x000349a0 _Next=0x00034988 _Prev=0x0028fe90 x=3 y=4
node at .end:
ptr=0x0028fe90 _Next=0x000349a0 _Prev=0x00034b40 x=5 y=6
\end{lstlisting}

Le fait que le dernier élément ait un pointeur sur le premier et le premier élément
ait un pointeur sur le dernier nous rappelle les listes circulaires.

Ceci est très pratique ici: en ayant un pointeur sur le premier élément de la liste,
i.e., ce qui est dans la variable $l$, il est très facile d'obtenir rapidement
un pointeur sur le dernier, sans devoir traverser toute la liste.

Insérer un élément à la fin de la liste est également rapide, grâce à cette caractéristique.

\TT{operator--} et \TT{operator++} mettent la valeur courante de l'itérateur à la valeur\\
\TT{current\_node->prev} ou \TT{current\_node->next}.

Les itérateurs inverses (.rbegin, .rend) fonctionne de la même façon, mais à l'envers.

\TT{operator*} renvoie un pointeur sur le point dans la structure du n\oe{}ud, où la
structure débute, i.e., un pointeur sur le premier élément de la structure ($x$).

L'insertion et la suppression dans la liste sont triviaux: simplement allouer un nouveau
n\oe{}ud (ou le désallouer) et mettre à jour les pointeurs afin qu'ils soient valides.

C'est pourquoi un itérateur peut devenir invalide après la suppression d'un élément:
il peut toujours pointer sur le n\oe{}ud qui a été déjà désalloué.
Ceci est aussi appelé un \emph{dangling pointer}.

Et bien sûr, l'information sur le n\oe{}ud libéré (sur lequel pointe toujours l'itérateur)
ne peut plus être utilisée.

L'implémentation de GCC (à partir de 4.8.1) ne stocke plus la taille courante de
la liste: ceci implique une méthode .size() lente: il doit traverser toute la liste
pour compter les éléments, car il n'a pas d'autre moyen d'obtenir l'information.

Ceci signifie que cette opération est en $O(n)$, i.e., elle devient constamment plus
lente lorsque la liste grandit.

\lstinputlisting[caption=GCC 4.8.1 -fno-inline-small-functions \Optimizing,style=customasmx86]{\CURPATH/STL/list/GCC_FR.asm}

\lstinputlisting[caption=La sortie complète]{\CURPATH/STL/list/GCC.txt}

\myparagraph{MSVC}
\label{MSVC_std_list}

L'implémentation de MSVC (2012) est la même, mais elle stocke aussi la taille courante
de la liste.

Ceci implique que la méthode .size() est très rapide ($O(1)$): elle doit juste lire
une valeur depuis la mémoire.

D'un autre côté, la variable size doit être mise à jour à chaque insertion/suppression.

L'implémentation de MSVC est aussi légèrement différente dans la façon dont elle
arrange les n\oe{}uds:

\input{\CURPATH/STL/list/MSVC_internals}

GCC a son élément fictif à la fin de la liste, tandis que MSVC l'a au début.

\lstinputlisting[caption=MSVC 2012 \Optimizing\ /Fa2.asm /GS- /Ob1,style=customasmx86]{\CURPATH/STL/list/MSVC_FR.asm}

Contrairement à GCC, le code de MSVC alloue l'élément fictif au début de la fonction
avec l'aide de la fonction \q{Buynode}, qui est aussi utilisée pour allouer le reste
des n\oe{}uds (le code de GCC alloue le premier élément dans la pile locale).

\lstinputlisting[caption=La sortie complète]{\CURPATH/STL/list/MSVC.txt}

\myparagraph{C++11 std::forward\_list}
\myindex{\Cpp!STL!std::forward\_list}

La même chose que std::list, mais simplement chaîné, i.e., ayant seulement le champ
\q{next} dans chaque n\oe{}ud.

Il a une empreinte mémoire plus faible, mais dons n'offre pas la possibilité de
traverser la liste en arrière.
