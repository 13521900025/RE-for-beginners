\mysection{setjmp/longjmp}

\myindex{\CStandardLibrary!setjmp}
\myindex{\CStandardLibrary!longjmp}

Il s'agit d'un mécanisme en C qui est très similaire au mecanisme throw/catch en
C++ ou d'autres \ac{PL}s de haut niveau.
\myindex{zlib}
Voici un exemple tiré de la zlib:

\begin{lstlisting}[style=customc]
...

    /* return if bits() or decode() tries to read past available input */
    if (setjmp(s.env) != 0)             /* if came back here via longjmp(), */
        err = 2;                        /*  then skip decomp(), return error */
    else
	err = decomp(&s); /* decompress */

...

    /* load at least need bits into val */
    val = s->bitbuf;
    while (s->bitcnt < need) {
        if (s->left == 0) {
            s->left = s->infun(s->inhow, &(s->in));
	    if (s->left == 0) longjmp(s->env, 1); /* out of input */

...

        if (s->left == 0) {
            s->left = s->infun(s->inhow, &(s->in));
	    if (s->left == 0) longjmp(s->env, 1); /* out of input */
\end{lstlisting}
( zlib/contrib/blast/blast.c )

L'appel à \TT{setjmp()} sauve les valeurs courantes de \ac{PC}, \ac{SP} et autres
registres dans une structure \TT{env}, puis renvoie 0.

En cas d'erreur, \TT{longjmp()} vous \emph{téléporte} au point juste après l'appel
à \TT{setjmp()}, comme si l'appel à \TT{setjmp()} avait renvoyé une valeur non nulle
(qui avait été passée à \TT{longjmp()}).
\myindex{UNIX!fork}
Ceci nous rappel l'appel système \TT{fork()} sous UNIX.

Maintenant, regardons un exemple épuré:

\lstinputlisting[style=customc]{advanced/625_setjmp/1.c}

Si nous le lançons, nous voyons:

\begin{lstlisting}
f1() begin
f2() begin
Error 1234
\end{lstlisting}

La structure \TT{jmp\_buf} est généralement non-documentée, pour préserver la compatibilité
ascendante.

Regardons comment setjmp() est implémenté dans MSVC 2013 x64:

\lstinputlisting[style=customasmx86]{advanced/625_setjmp/setjmp_VC2013_x64.asm}

Cela remplit juste la structure jmp\_buf avec la valeur courante de presque tous les registres.
Aussi, la valeur courante de \ac{RA} est prise de la pile et sauvée dans jmp\_buf:
elle sera utilisée comme nouvelle valeur de \ac{PC} dans le futur.

Maintenant longjmp():

\lstinputlisting[style=customasmx86]{advanced/625_setjmp/longjmp_VC2013_x64.asm}

Cela restaure (presque) tous les registres, prend \ac{RA} dans la structure et y saute.
Ceci fonctionne en effet comme si setjmp() retournait à l'appelant.
Aussi, \TT{RAX} est mis pour être égal au second argument de longjmp().
Ceci fonctionne comme si setjmp() renvoyait une valeur non-zéro en première place. % clarify

Comme effet de bord de la restauration de \ac{SP}, toutes les valeurs dans la pile qui
ont été définies et utilisées entre les appels à setjmp() et longjmp() sont laissées
tomber.
Elles ne seront plus utilisées du tout.
Ainsi, longjmp() saute usuellement en arrière
\footnote{Toutefois, il y a des gens qui l'utilisent pour des choses bien plus compliquées,
imitation des coroutines, etc.: \url{https://www.embeddedrelated.com/showarticle/455.php},
\url{http://fanf.livejournal.com/105413.html}}.

Ceci implique que, contrairement au mécanisme throw/catch en \Cpp, aucune mémoire
ne sera libérée, aucun destructeur ne sera appelé, etc.
Ainsi, cette technique peut parfois être dangereuse.
Néanmoins, elle est assez populaire. C'est toujours utilisé dans \oracle.

Cela a aussi un effet de bord inattendu: si un buffer a été dépassé dans une des
fonctions (peut-être à cause d'une attaque distante), et qu'une fonction veut signaler
une erreur, et ça appelle longjmp(), la partie de la pile récrite ne sera pas utilisée.

À titre d'exercice, vous pouvez essayer de comprendre pourquoi tous les registres
ne sont pas sauvegardés.
Pourquoi XMM0-XMM5 et d'autres registres sont évités?
