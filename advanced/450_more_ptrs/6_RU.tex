\subsection{Указатель на функцию}

Имя ф-ции в \CCpp{} без скобок, как ``printf'' это указатель на ф-цию типа \emph{void (*)()}.
Попробуем прочитать содержимое ф-ции и пропатчить его:

\lstinputlisting[style=customc]{advanced/450_more_ptrs/6.c}

При запуске видно что первые 3 байта ф-ции это \TT{55 89 e5}.
Действительно, это опкоды инструкций \INS{PUSH EBP} и \INS{MOV EBP, ESP} (это опкоды x86).
Но потом процесс падает, потому что секция \emph{text} доступна только для чтения.

Мы можем перекомпилировать наш пример и сделать так, чтобы секция \emph{text} была доступна для записи
\footnote{\url{http://stackoverflow.com/questions/27581279/make-text-segment-writable-elf}}:

\begin{lstlisting}
gcc --static -g -Wl,--omagic -o example example.c
\end{lstlisting}

Это работает!

\begin{lstlisting}
we are in print_something()
first 3 bytes: 55 89 e5...
going to call patched print_something():
it must exit at this point
\end{lstlisting}

