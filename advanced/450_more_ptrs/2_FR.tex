\subsection{Passer des valeurs en tant que pointeurs; tagged unions}

Voici un exemple montrant comment passer des valeurs dans des pointeurs:

\begin{lstlisting}[label=unsigned_multiply_C,style=customc]
#include <stdio.h>
#include <stdint.h>

uint64_t multiply1 (uint64_t a, uint64_t b)
{
	return a*b;
};

uint64_t* multiply2 (uint64_t *a, uint64_t *b)
{
	return (uint64_t*)((uint64_t)a*(uint64_t)b);
};

int main()
{
	printf ("%d\n", multiply1(123, 456));
	printf ("%d\n", (uint64_t)multiply2((uint64_t*)123, (uint64_t*)456));
};
\end{lstlisting}

Il fonctionne sans problème et GCC 4.8.4 compile les fonctions multiply1() et multiply2()
de manière identique!

\begin{lstlisting}[label=unsigned_multiply_lst,style=customasmx86]
multiply1:
	mov	rax, rdi
	imul	rax, rsi
	ret

multiply2:
	mov	rax, rdi
	imul	rax, rsi
	ret
\end{lstlisting}

Tant que vous ne déréférencez pas le pointeur (autrement dit, que vous ne lisez aucune
donnée depuis l'adresse stockée dans le pointeur), tout se passera bien.
Un pointeur est une variable qui peut stocker n'importe quoi, comme une variable
usuelle.

\myindex{x86!\Instructions!MUL}
\myindex{x86!\Instructions!IMUL}
L'instruction de multiplication signée (\IMUL) est utilisée ici au lieu de la non-signée
(\MUL), lisez-en plus à ce sujet ici:
\ref{IMUL_over_MUL}.

\myindex{Tagged pointers}
À propos, il y a une astuce très connue pour abuser des pointeurs appelée \emph{tagged pointers}.
En gros, si tous vos pointeurs pointent sur des blocs de mémoire de taille, disons,
16 octets (ou qu'ils sont toujours alignés sur une limite de 16-octet), les 4 bits
les plus bas du pointeur sont toujours zéro et cet espace peut être utilisé d'une
certaine façon.
\myindex{LISP}
C'est très répandu dans les compilateurs et interpréteurs LISP.
Ils stockent le type de cell/objet dans ces bits inutilisés, ceci peut économiser
un peu de mémoire.
Encore mieux, vous pouvez évaluer le type de cell/objet en utilisant seulement le pointeur,
sans accès supplémentaire à la mémoire.
En lire plus à ce sujet: \InSqBrackets{\CNotes\ 1.3}.

% TODO Example of tagged ptrs here
