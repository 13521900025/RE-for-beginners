\subsection{Работа с адресами вместо указателей}

Указатель это просто адрес в памяти. Но почему мы пишем \TT{char* string} вместо чего-нибудь вроде \TT{address string}?
Переменная-указатель дополнена типом переменной, на которую указатель указывает.
Тогда у компилятора будет возможность находить потенциальные ошибки типизации во время компиляции.

Если быть педантом, типизация данных в языках программирования существует для предотвращения ошибок и самодокументации.
Вполне возможно использовать только два типа данных вроде \emph{int} (или \emph{int64\_t}) и байт --- это те единственные типы, которые
доступны для программистов на ассемблере.
Но написать что-то больше и практичное на ассемблере, при этом без ошибок, это трудная задача.
Любая мелкая опечатка может привести к труднонаходимой ошибке.

Информации о типах нет в скомпилированном коде (и это одна из основных проблем для декомпиляторов),
и я могу это продемонстрировать.

Вот как напишет обычный программист на \CCpp{}:

\begin{lstlisting}[style=customc]
#include <stdio.h>
#include <stdint.h>

void print_string (char *s)
{
	printf ("(address: 0x%llx)\n", s);
	printf ("%s\n", s);
};

int main()
{
	char *s="Hello, world!";

	print_string (s);
};
\end{lstlisting}

А вот что могу написать я:

\begin{lstlisting}[style=customc]
#include <stdio.h>
#include <stdint.h>

void print_string (uint64_t address)
{
	printf ("(address: 0x%llx)\n", address);
	puts ((char*)address);
};

int main()
{
	char *s="Hello, world!";

	print_string ((uint64_t)s);
};
\end{lstlisting}

Я использую \emph{uint64\_t} потому что я запускаю этот пример на Linux x64. \emph{int} сгодится для 32-битных \ac{OS}.
В начале, указатель на символ (самый первый в строке с приветствием) приводится к \emph{uint64\_t}, затем он передается далее.
Ф-ция \TT{print\_string()} приводит тип переданного значения из \emph{uint64\_t} в указатель на символ.

Но вот что интересно, это то что GCC 4.8.4 генерирует идентичный результат на ассемблере для обеих версий:

\begin{lstlisting}
gcc 1.c -S -masm=intel -O3 -fno-inline
\end{lstlisting}

\begin{lstlisting}[style=customasmx86]
.LC0:
	.string	"(address: 0x%llx)\n"
print_string:
	push	rbx
	mov	rdx, rdi
	mov	rbx, rdi
	mov	esi, OFFSET FLAT:.LC0
	mov	edi, 1
	xor	eax, eax
	call	__printf_chk
	mov	rdi, rbx
	pop	rbx
	jmp	puts
.LC1:
	.string	"Hello, world!"
main:
	sub	rsp, 8
	mov	edi, OFFSET FLAT:.LC1
	call	print_string
	add	rsp, 8
	ret
\end{lstlisting}

(Я убрал незначительные директивы GCC.)

Я также пробовал утилиту UNIX \emph{diff} и не нашел разницы вообще.

Продолжим и дальше издеваться над традициями программирования в \CCpp{}.
Кто-то может написать так:

\begin{lstlisting}[style=customc]
#include <stdio.h>
#include <stdint.h>

uint8_t load_byte_at_address (uint8_t* address)
{
	return *address;
	//this is also possible: return address[0]; 
};

void print_string (char *s)
{
	char* current_address=s;
	while (1)
	{
		char current_char=load_byte_at_address(current_address);
		if (current_char==0)
			break;
		printf ("%c", current_char);
		current_address++;
	};
};

int main()
{
	char *s="Hello, world!";

	print_string (s);
};
\end{lstlisting}

И это может быть переписано так:

\begin{lstlisting}[style=customc]
#include <stdio.h>
#include <stdint.h>

uint8_t load_byte_at_address (uint64_t address)
{
	return *(uint8_t*)address;
};

void print_string (uint64_t address)
{
	uint64_t current_address=address;
	while (1)
	{
		char current_char=load_byte_at_address(current_address);
		if (current_char==0)
			break;
		printf ("%c", current_char);
		current_address++;
	};
};

int main()
{
	char *s="Hello, world!";

	print_string ((uint64_t)s);
};
\end{lstlisting}

И тот и другой исходный код преобразуется в одинаковый результат на ассемблере:

\begin{lstlisting}
gcc 1.c -S -masm=intel -O3 -fno-inline
\end{lstlisting}

\begin{lstlisting}[style=customasmx86]
load_byte_at_address:
	movzx	eax, BYTE PTR [rdi]
	ret
print_string:
.LFB15:
	push	rbx
	mov	rbx, rdi
	jmp	.L4
.L7:
	movsx	edi, al
	add	rbx, 1
	call	putchar
.L4:
	mov	rdi, rbx
	call	load_byte_at_address
	test	al, al
	jne	.L7
	pop	rbx
	ret
.LC0:
	.string	"Hello, world!"
main:
	sub	rsp, 8
	mov	edi, OFFSET FLAT:.LC0
	call	print_string
	add	rsp, 8
	ret
\end{lstlisting}

(Здесь я также убрал незначительные директивы GCC.)

Разницы нет: указатели в \CCpp{}, в сущности, адреса, но несут в себе также информацию о типе, чтобы предотвратить ошибки
во время компиляции.
Типы не проверяются во время исполнения, иначе это был бы огромный (и ненужный) прирост времени исполнения.

