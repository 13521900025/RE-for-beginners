\subsection{Pointer as object identificator}

Both assembly language and C has no \ac{OOP} features, but it's possible to write a code in \ac{OOP} style
(just treat structure as an object).

It's interesting, that sometimes, pointer to an object (or its address) is called as ID
(in sense of data hiding/encapsulation).

\myindex{win32!LoadLibrary()}
\myindex{win32!GetProcAddress()}
For example, LoadLibrary(), according to \ac{MSDN}, returns ``handle to the module''
\footnote{\url{https://msdn.microsoft.com/ru-ru/library/windows/desktop/ms684175(v=vs.85).aspx}}.
Then you pass this ``handle'' to other functions like GetProcAddress().
But in fact, LoadLibrary() returns pointer to DLL file mapped into memory
\footnote{\url{https://blogs.msdn.microsoft.com/oldnewthing/20041025-00/?p=37483}}.
You can read two bytes from the address LoadLibrary() returns, and that would be ``MZ'' (first two bytes of any
.EXE/.DLL file in Windows).

\myindex{win32!HMODULE}
\myindex{win32!HINSTANCE}
Apparently, Microsoft ``hides'' that fact to provide better forward compatibility.
Also, HMODULE and HINSTANCE data types had another meaning in 16-bit Windows.

Probably, this is reason why \printf has ``\%p'' modifier, which is used for printing pointers (32-bit integers
on 32-bit architectures, 64-bit on 64-bit, etc) in hexadecimal form.
Address of a structure dumped into debug log may help in finding it in another place of log.

\myindex{SQLite}
Here is also from SQLite source code:

\begin{lstlisting}

...

struct Pager {
  sqlite3_vfs *pVfs;          /* OS functions to use for IO */
  u8 exclusiveMode;           /* Boolean. True if locking_mode==EXCLUSIVE */
  u8 journalMode;             /* One of the PAGER_JOURNALMODE_* values */
  u8 useJournal;              /* Use a rollback journal on this file */
  u8 noSync;                  /* Do not sync the journal if true */

....

static int pagerLockDb(Pager *pPager, int eLock){
  int rc = SQLITE_OK;

  assert( eLock==SHARED_LOCK || eLock==RESERVED_LOCK || eLock==EXCLUSIVE_LOCK );
  if( pPager->eLock<eLock || pPager->eLock==UNKNOWN_LOCK ){
    rc = sqlite3OsLock(pPager->fd, eLock);
    if( rc==SQLITE_OK && (pPager->eLock!=UNKNOWN_LOCK||eLock==EXCLUSIVE_LOCK) ){
      pPager->eLock = (u8)eLock;
      IOTRACE(("LOCK %p %d\n", pPager, eLock))
    }
  }
  return rc;
}

...

  PAGER_INCR(sqlite3_pager_readdb_count);
  PAGER_INCR(pPager->nRead);
  IOTRACE(("PGIN %p %d\n", pPager, pgno));
  PAGERTRACE(("FETCH %d page %d hash(%08x)\n",
               PAGERID(pPager), pgno, pager_pagehash(pPg)));

...

\end{lstlisting}

