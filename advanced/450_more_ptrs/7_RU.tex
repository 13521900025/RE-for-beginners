\subsection{Указатель как идентификатор объекта}

В ассемблере и Си нет возможностей \ac{OOP}, но там вполне можно писать код в стиле \ac{OOP}
(просто относитесь к структуре, как к объекту).

Интересно что, иногда, указатель на объект (или его адрес) называется идентификатором (в смысле сокрытия данных/инкапсуляции).

\myindex{win32!LoadLibrary()}
\myindex{win32!GetProcAddress()}
Например, LoadLibrary(), судя по \ac{MSDN}, возвращает ``handle'' модуля
\footnote{\url{https://msdn.microsoft.com/ru-ru/library/windows/desktop/ms684175(v=vs.85).aspx}}.
Затем вы передаете этот ``handle'' в другую ф-цию вроде GetProcAddress().
Но на самом деле, LoadLibrary() возвращает указатель на DLL-файл загруженный (\emph{mapped}) в памяти
\footnote{\url{https://blogs.msdn.microsoft.com/oldnewthing/20041025-00/?p=37483}}.
Вы можете прочитать два байта по адресу возвращенному LoadLibrary(), и это будет ``MZ'' (первые два байта любого файла
типа .EXE/.DLL в Windows).

\myindex{win32!HMODULE}
\myindex{win32!HINSTANCE}
Очевидно, Microsoft ``скрывает'' этот факт для обеспечения лучшей совместимости в будущем.
Также, типы данных HMODULE и HINSTANCE имели другой смысл в 16-битной Windows.

Возможно, это причина, почему \printf имеет модификатор ``\%p'', который используется для вывода указателей (32-битные
целочисленные на 32-битных архитектурах, 64-битные на 64-битных, итд) в шестнадцатеричной форме.
Адрес структуры сохраненный в отладочном протоколе может помочь в поисках такого же в том же протоколе.

\myindex{SQLite}
Вот например из исходного кода SQLite:

\begin{lstlisting}

...

struct Pager {
  sqlite3_vfs *pVfs;          /* OS functions to use for IO */
  u8 exclusiveMode;           /* Boolean. True if locking_mode==EXCLUSIVE */
  u8 journalMode;             /* One of the PAGER_JOURNALMODE_* values */
  u8 useJournal;              /* Use a rollback journal on this file */
  u8 noSync;                  /* Do not sync the journal if true */

....

static int pagerLockDb(Pager *pPager, int eLock){
  int rc = SQLITE_OK;

  assert( eLock==SHARED_LOCK || eLock==RESERVED_LOCK || eLock==EXCLUSIVE_LOCK );
  if( pPager->eLock<eLock || pPager->eLock==UNKNOWN_LOCK ){
    rc = sqlite3OsLock(pPager->fd, eLock);
    if( rc==SQLITE_OK && (pPager->eLock!=UNKNOWN_LOCK||eLock==EXCLUSIVE_LOCK) ){
      pPager->eLock = (u8)eLock;
      IOTRACE(("LOCK %p %d\n", pPager, eLock))
    }
  }
  return rc;
}

...

  PAGER_INCR(sqlite3_pager_readdb_count);
  PAGER_INCR(pPager->nRead);
  IOTRACE(("PGIN %p %d\n", pPager, pgno));
  PAGERTRACE(("FETCH %d page %d hash(%08x)\n",
               PAGERID(pPager), pgno, pager_pagehash(pPg)));

...

\end{lstlisting}

