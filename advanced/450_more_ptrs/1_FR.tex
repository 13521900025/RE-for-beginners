\subsection{Travailler avec des adresses au lieu de pointeurs}

Un pointeur est juste une adresse en mémoire. Mais pourquoi écrivons-nous \TT{char* string}
au lieu de quelque chose comme \TT{address string}?
La variable pointeur est fournie avec un type de la valeur sur laquelle le pointeur
pointe.
Donc le compilateur est capable de détecter des bugs de type de données lors de la
compilation.

Pour être pédant, les types de données des langages de programmation ne servent qu'a
prévenir des bugs et à l'auto-documentation.
Il est possible de n'utiliser que deux types de données, comme \emph{int} (ou \emph{int64\_t})
et l'octet---ce sont les seuls types disponible aux programmeurs en langage d'assemblage.
Mais c'est une tâche extrêmement difficile d'écrire des programmes en assembleur
pratique et sans bugs méchants.
La plus petite typo peut conduire à un bug difficile-à-trouver.

L'information sur le type de données est absente d'un code compilé (et c'est l'un
des problèmes majeurs pour les dé-compilateurs), et je peux le prouver.

Ceci est ce qu'un programmeur \CCpp peut écrire:

\begin{lstlisting}[style=customc]
#include <stdio.h>
#include <stdint.h>

void print_string (char *s)
{
	printf ("(address: 0x%llx)\n", s);
	printf ("%s\n", s);
};

int main()
{
	char *s="Hello, world!";

	print_string (s);
};
\end{lstlisting}

Ceci est ce que je peux écrire:

\begin{lstlisting}[style=customc]
#include <stdio.h>
#include <stdint.h>

void print_string (uint64_t address)
{
	printf ("(address: 0x%llx)\n", address);
	puts ((char*)address);
};

int main()
{
	char *s="Hello, world!";

	print_string ((uint64_t)s);
};
\end{lstlisting}

J'utilise \emph{uint64\_t} car j'ai effectué cet exemple sur Linux x64. \emph{int} fonctionnerait
pour des \ac{OS}-s 32-bit.
D'abord, un pointeur sur un caractère (la toute première chaîne de bienvenu) est
casté en \emph{uint64\_t}, puis passé plus loin.
La fonction \TT{print\_string()} re-caste la valeur \emph{uint64\_t} en un pointeur
sur un caractère.

Ce qui est intéressant, c'est que GCC 4.8.4 produit une sortie assembleur identique
pour les deux versions:

\begin{lstlisting}
gcc 1.c -S -masm=intel -O3 -fno-inline
\end{lstlisting}

\begin{lstlisting}[style=customasmx86]
.LC0:
	.string	"(address: 0x%llx)\n"
print_string:
	push	rbx
	mov	rdx, rdi
	mov	rbx, rdi
	mov	esi, OFFSET FLAT:.LC0
	mov	edi, 1
	xor	eax, eax
	call	__printf_chk
	mov	rdi, rbx
	pop	rbx
	jmp	puts
.LC1:
	.string	"Hello, world!"
main:
	sub	rsp, 8
	mov	edi, OFFSET FLAT:.LC1
	call	print_string
	add	rsp, 8
	ret
\end{lstlisting}

(j'ai supprimé toutes les directives non significatives de GCC.)

J'ai aussi essayé différents utilitaires UNIX de \emph{diff} et ils ne montrent aucune
différence.

Continuons à abuser massivement des traditions de programmation de \CCpp.
On pourrait écrire ceci:

\begin{lstlisting}[style=customc]
#include <stdio.h>
#include <stdint.h>

uint8_t load_byte_at_address (uint8_t* address)
{
	return *address;
	//this is also possible: return address[0]; 
};

void print_string (char *s)
{
	char* current_address=s;
	while (1)
	{
		char current_char=load_byte_at_address(current_address);
		if (current_char==0)
			break;
		printf ("%c", current_char);
		current_address++;
	};
};

int main()
{
	char *s="Hello, world!";

	print_string (s);
};
\end{lstlisting}

Ça pourrait être récrit comme ceci:

\begin{lstlisting}[style=customc]
#include <stdio.h>
#include <stdint.h>

uint8_t load_byte_at_address (uint64_t address)
{
	return *(uint8_t*)address;
};

void print_string (uint64_t address)
{
	uint64_t current_address=address;
	while (1)
	{
		char current_char=load_byte_at_address(current_address);
		if (current_char==0)
			break;
		printf ("%c", current_char);
		current_address++;
	};
};

int main()
{
	char *s="Hello, world!";

	print_string ((uint64_t)s);
};
\end{lstlisting}

Les deux codes source donnent la même sortie assembleur:

\begin{lstlisting}
gcc 1.c -S -masm=intel -O3 -fno-inline
\end{lstlisting}

\begin{lstlisting}[style=customasmx86]
load_byte_at_address:
	movzx	eax, BYTE PTR [rdi]
	ret
print_string:
.LFB15:
	push	rbx
	mov	rbx, rdi
	jmp	.L4
.L7:
	movsx	edi, al
	add	rbx, 1
	call	putchar
.L4:
	mov	rdi, rbx
	call	load_byte_at_address
	test	al, al
	jne	.L7
	pop	rbx
	ret
.LC0:
	.string	"Hello, world!"
main:
	sub	rsp, 8
	mov	edi, OFFSET FLAT:.LC0
	call	print_string
	add	rsp, 8
	ret
\end{lstlisting}

(j'ai supprimé toutes les directives non significatives de GCC.)

Aucune différence: les pointeurs \CCpp sont essentiellement des adresses, mais fournies
avec une information sur le type, afin de prévenir des erreurs possible lors de la
compilation.
