\subsection{Cas de la fonction \emph{vprintf()}}
\myindex{\CStandardLibrary!vprintf}
\myindex{\CStandardLibrary!va\_list}

De nombreux programmeurs définissent leur propre fonction de logging, qui prend une
chaîne de format du type de celle de printf+une liste variable d'arguments.

Un autre exemple répandu est la fonction die(), qui affiche un message et sort.

Nous avons besoin d'un moyen de transmettre un nombre d'arguments inconnu et de les
passer à la fonction \printf.
Mais comment?

À l'aide des fonctions avec un \q{v} dans le nom.

Une d'entre elles est \emph{vprintf()}: elle prend une chaîne de format et un pointeur
sur une variable du type \TT{va\_list}:

\lstinputlisting[style=customc]{\CURPATH/die.c}

En examinant plus précisément, nous voyons que \TT{va\_list} est un pointeur sur
un tableau.
Compilons:

\lstinputlisting[caption=MSVC 2010 \Optimizing,style=customasmx86]{\CURPATH/die_MSVC2010_Ox_FR.asm}

Nous voyons que tout ce que fait notre fonction est de prendre un pointeur sur les
arguments et le passe à la fonction \emph{vprintf()}, et que cette fonction le traite
comme un tableau infini d'arguments!

\lstinputlisting[caption=MSVC 2012 x64 \Optimizing,style=customasmx86]{\CURPATH/die_MSVC2012_x64_Ox_FR.asm}
