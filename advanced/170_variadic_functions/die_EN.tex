\subsection{\emph{vprintf()} function case}
\myindex{\CStandardLibrary!vprintf}
\myindex{\CStandardLibrary!va\_list}

Many programmers define their own logging functions which take a printf-like format string + 
a variable number of arguments.

Another popular example is the die() function, which prints some message and exits.

We need some way to pack input arguments of unknown number and pass them to the \printf function.
But how?

That's why there are functions with \q{v} in name.

One of them is \emph{vprintf()}: it takes a format-string and a pointer to a variable of type \TT{va\_list}:

\lstinputlisting[style=customc]{\CURPATH/die.c}

By closer examination, we can see that \TT{va\_list} is a pointer to an array.
Let's compile:

\lstinputlisting[caption=\Optimizing MSVC 2010,style=customasmx86]{\CURPATH/die_MSVC2010_Ox_EN.asm}

We see that all our function does is just taking a pointer to the arguments and
passing it to \emph{vprintf()}, and that function is treating it like an infinite array of arguments!

\lstinputlisting[caption=\Optimizing MSVC 2012 x64,style=customasmx86]{\CURPATH/die_MSVC2012_x64_Ox_EN.asm}
