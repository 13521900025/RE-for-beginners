\chapter{Libri/blog da leggere}

\mysection{Libri ed altro materiale}

\subsection{Reverse Engineering}

\begin{itemize}
\item Eldad Eilam, \emph{Reversing: Secrets of Reverse Engineering}, (2005)

\item Bruce Dang, Alexandre Gazet, Elias Bachaalany, Sebastien Josse, \emph{Practical Reverse Engineering: x86, x64, ARM, Windows Kernel, Reversing Tools, and Obfuscation}, (2014)

\item Michael Sikorski, Andrew Honig, \emph{Practical Malware Analysis: The Hands-On Guide to Dissecting Malicious Software}, (2012)

\item Chris Eagle, \emph{IDA Pro Book}, (2011)

\item Reginald Wong, \emph{Mastering Reverse Engineering: Re-engineer your ethical hacking skills}, (2018)

\end{itemize}


Inoltre, i libri di Kris Kaspersky.

\subsection{Windows}

\input{Win_reading}

\subsection{\CCpp}

\input{CCppBooks}

\subsection{x86 / x86-64}

\label{x86_manuals}
\begin{itemize}
\item Manuali Intel\footnote{\AlsoAvailableAs \url{http://www.intel.com/content/www/us/en/processors/architectures-software-developer-manuals.html}}

\item Manuali AMD\footnote{\AlsoAvailableAs \url{http://developer.amd.com/resources/developer-guides-manuals/}}

\item \AgnerFog{}\footnote{\AlsoAvailableAs \url{http://agner.org/optimize/microarchitecture.pdf}}

\item \AgnerFogCC{}\footnote{\AlsoAvailableAs \url{http://www.agner.org/optimize/calling_conventions.pdf}}

\item \IntelOptimization

\item \AMDOptimization
\end{itemize}

Un po' datati ma sempre interessanti:

\MAbrash\footnote{\AlsoAvailableAs \url{https://github.com/jagregory/abrash-black-book}}
(è conosciuto per i suoi lavori di ottimizzazione a basso livello su progetti come Windows NT 3.1 e id Quake).

\subsection{ARM}

\begin{itemize}
\item Manuali ARM\footnote{\AlsoAvailableAs \url{http://infocenter.arm.com/help/index.jsp?topic=/com.arm.doc.subset.architecture.reference/index.html}}

\item \ARMSevenRef

\item \ARMSixFourRefURL

\item \ARMCookBook\footnote{\AlsoAvailableAs \url{http://go.yurichev.com/17273}}
\end{itemize}

\subsection{Assembly}

Richard Blum --- Professional Assembly Language.

\subsection{Java}

\JavaBook.

\subsection{UNIX}

\TAOUP

\subsection{Programmazione in generale}

\begin{itemize}

\item \RobPikePractice

\item \HenryWarren.
Alcune persone sostengono che i trucchi e gli hack di questo libro non siano più attuali adesso perchè erano validi solo per le \ac{CPU} \ac{RISC},
dove le istruzioni di branching sono costose.
Ad ogni modo, possono aiutare enormemente a comprendere l'algebra booleana e tutta la matematica coinvolta.

\item (Per hard-core geek con background in computer science e matematica) \TAOCP.
Alcune persone si domandano se sia utile, per programmatori mediocri, il tentare di leggere questi testi abbastanza complessi sui fondamenti della programmazione.
Io direi che è utile sfogliarli, per comprendere di cosa tratta la \ac{CS}.

\end{itemize}

% subsection:
\subsection{\EN{Cryptography}\ES{Criptograf\'ia}\ITA{Crittografia}\RU{Криптография}\FR{Cryptographie}\DE{Kryptografie}\JPN{暗号学}}
\label{crypto_books}

\begin{itemize}
\item \Schneier{}

\item (Free) lvh, \emph{Crypto 101}\footnote{\AlsoAvailableAs \url{https://www.crypto101.io/}}

\item (Free) Dan Boneh, Victor Shoup, \emph{A Graduate Course in Applied Cryptography}\footnote{\AlsoAvailableAs \url{https://crypto.stanford.edu/~dabo/cryptobook/}}.
\end{itemize}



\iffalse
\subsection{Dedica}

Come scritto nella prima pagina di questo libro, ``Questo libro è dedicato a Robert Jourdain, John Socha, Ralf Brown e Peter Abel''.
Si tratta di autori famosi di libri sul linguaggio assembly e di riferimento negli anni 1980 e 1990:

\begin{itemize}
\item Robert Jourdain -- Programmer's problem solver for the IBM PC, XT, \& AT (1986)

\item Peter Norton e John Socha -- The Peter Norton Programmer's Guide to the IBM PC (1985), Peter Norton's Assembly Language Book for the IBM PC (1989).
Di fatto, John Socha è un vero autore di questi libri, si può dire, era un ghostwriter.
Inoltre è autore del Norton Commander.

\item Ralph Brown era conosciuto per la ``Ralf Brown's Interrupt List''\footnote{\url{http://www.ctyme.com/rbrown.htm}}.

\item Peter Abel -- IBM PC assembly language and programming (1991)
\end{itemize}

Chiaramente si tratta di libri antiquati.
Ma magari qualcuno si ricorderà di ``quei tempi''.
\fi
