\mysection{Thread Local Storage}
\label{TLS}
\myindex{TLS}

TLS est un espace de données propre à chaque thread, qui peut y conserver ce qu'il estime utile.
Un exemple d'utilisation bien connu en est la variable globale \emph{errno} du standard C.

Plusieurs threads peuvent invoquer simultanément différentes fonctions qui retournent toutes un code 
erreur dans la variable \emph{errno}. L'utilisation d'une variable globale dans le contexte d'un 
programme comportant plusieurs threads serait donc inadaptée dans ce cas. C'est pourquoi la variable 
\emph{errno} est conservée dans l'espace \ac{TLS}.\\
\\
\myindex{\Cpp!C++11}
La version 11 du standard C++ a ajouté un nouveau modificateur \emph{thread\_local} qui indique que 
chaque thread possède sa propre copie de la variable décorée par ce modificateur. La variable en 
question est alors conservée dans l'espace \ac{TLS}
\footnote{\myindex{C11} C11 also has thread support, optional though}:

\begin{lstlisting}[caption=C++11,style=customc]
#include <iostream>
#include <thread>

thread_local int tmp=3;

int main()
{
	std::cout << tmp << std::endl;
};
\end{lstlisting}

Compilé avec MinGW GCC 4.8.1, mais pas avec MSVC 2012.

Dans le contexte des fichiers au format PE, la variable \emph{tmp} sera allouée dans la section dédiée 
au \ac{TLS} du fichier exécutable résultant de la compilation.

\subsection{Amélioration du générateur linéaire congruent}
\label{LCG_TLS}

Le générateur de nombres pseudo-aléatoires \myref{LCG_simple} que nous avons étudié précédemment 
contient une faiblesse. Son comportement est buggé dans un environnement multi-thread. Il utilise 
en effet une variable d'état dont la valeur peut être lue et/ou modifiée par plusieurs threads 
simultanément.

% subsections
\input{OS/TLS/LCG_win32_FR}
\input{OS/TLS/LCG_linux_FR}

