\subsection{\CapitalPICcode}
\myindex{\PICcode}
\myindex{Linux}
\label{sec:PIC}

Lorsque l'on analyse des bibliothèques partagées sous Linux (.so), on rencontre souvent
ce genre de code:

\lstinputlisting[style=customasmx86,caption=libc-2.17.so x86]{OS/tmp1.asm}

Tous les pointeurs sur des chaînes sont corrigés avec une certaine constante et la valeur de \EBX,
qui est calculée au début de chaque fonction.

C'est ce que l'on appelle du code \ac{PIC}, qui peut être exécuté à n'importe quelle adresse mémoire, c'est pourquoi il ne doit contenir aucune adresse absolue.

\ac{PIC} était crucial dans les premiers ordinateurs, et l'est encore aujourd'hui dans les systèmes embarqués
sans support de la mémoire virtuelle (où tous les processus se trouvent dans un seul bloc continu de mémoire).

C'est encore utilisé sur les systèmes *NIX pour les bibliothèques partagées, car elles
sont utilisées par de nombreux processus mais ne sont chargées qu'une seule fois en mémoire.
Mais tous ces processus peuvent mapper la même bibliothèque partagée à des adresses différentes,
c'est pourquoi elles doivent fonctionner correctement sans aucune adresse absolue.


Faisons un petit essai:

\begin{lstlisting}[style=customc]
#include <stdio.h>

int global_variable=123;

int f1(int var)
{
    int rt=global_variable+var;
    printf ("returning %d\n", rt);
    return rt;
};
\end{lstlisting}

Compilons le avec GCC 4.7.3 et examinons le fichier .so généré dans \IDA:

\begin{lstlisting}
gcc -fPIC -shared -O3 -o 1.so 1.c
\end{lstlisting}

\lstinputlisting[style=customasmx86,caption=GCC 4.7.3]{OS/tmp2.asm}

\newcommand{\retstring}{\emph{<<returning \%d\textbackslash{}n>>}}
\newcommand{\globvar}{\emph{global\_variable}}

C'est ça: les pointeurs sur \retstring{} et \globvar{} sont corrigés à chaque exécution de la fonction.

\par La fonction \TT{\_\_x86\_get\_pc\_thunk\_bx()} renvoie dans \EBX l'adresse de l'instruction après son appel (\TT{0x57C} ici).

C'est un moyen simple d'obtenir la valeur du compteur de programme (\EIP) à un endroit quelconque.
La constante \TT{0x1A84} est relative à la différence entre le début de cette fonction et ce que l'on appelle
\emph{Global Offset Table Procedure Linkage Table} (GOT PLT), la section juste après la \emph{Global Offset Table} (GOT), où se trouve le pointeur sur \globvar{}.
\IDA montre ces offsets sous leur forme calculée pour rendre les choses plus facile à comprendre, mais en fait, le code est:

\begin{lstlisting}[style=customasmx86]
.text:00000577                 call    __x86_get_pc_thunk_bx
.text:0000057C                 add     ebx, 1A84h
.text:00000582                 mov     [esp+1Ch+var_4], esi
.text:00000586                 mov     eax, [ebx-0Ch]
.text:0000058C                 mov     esi, [eax]
.text:0000058E                 lea     eax, [ebx-1A30h]
\end{lstlisting}

Ici, \EBX pointe sur la section \TT{GOT PLT} et pour calculer le pointeur sur \globvar{} ( qui est stocké dans
la \TT{GOT}), il faut soustraire \TT{0xC}.

Pour calculer la valeur du pointeur sur la chaîne \retstring{}, il faut soustraire \TT{0x1A30}.

\myindex{x86-64}
\myindex{x86!\Registers!RIP}

A propos, c'est la raison pour laquelle le jeu d'instructions AMD64 ajoute le support d'adressage relatif de RIP\footnote{compteur de programme sur AMD64} --- pour simplifier le code PIC.

Compilons le même code C en utilisant la même version de GCC, mais pour x64.

\myindex{objdump}
\IDA simplifierait le code en supprimant les détails de l'adressage relatif à RIP,
donc utilisons \emph{objdump} à la place d'\IDA pour tout voir:

\lstinputlisting[style=customasmx86]{OS/tmp3.asm}

\TT{0x2008b9} est la différence entre l'adresse de l'instruction en \TT{0x720} et \globvar{}, et 
\TT{0x20} est la différence entre l'adresse de l'instruction en
\TT{0x72A} et la chaîne \retstring{}.

Comme on le voit, le fait d'avoir à recalculer fréquemment les adresses rend l'exécution plus lente
(cela dit, ça c'est amélioré en x64).

Donc, il est probablement mieux de lier statiquement si vous vous préoccupez des performances \InSqBrackets{voir: \AgnerFogCPP}.

\subsubsection{Windows}
\myindex{Windows!Win32}

Le mécanisme PIC n'est pas utilisé dans les DLLs de Windows. Si le chargeur de Windows doit charger une DLL
à une autre adresse, il \q{patche} la DLL en mémoire (aux places \emph{FIXUP}) afin de corriger toutes les
adresses.

Cela implique que plusieurs processus Windows ne peuvent pas partager une DLL déjà chargée à une adresse différente
dans des blocs mémoire de différents processus --- puisque chaque instance qui est chargée en mémoire est \emph{fixé} pour 
fonctionner uniquement à ces adresses...

