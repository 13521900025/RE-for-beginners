\subsection{Win32 PE}
\label{win32_pe}
\myindex{Windows!Win32}

\acs{PE} ist ein Dateiformat für ausführbare Dateien unter Windows.
Der Unterschied zwischen .exe, .dll und .sys ist, dass .exe und .sys in der Regel
nur Imports und keine Exports haben.

\myindex{OEP}

Eine \ac{DLL} hat wie jede andere PE-Datei einen Eintrittspunkt (\ac{OEP}) (die
Funktion DllMain() befindet sich hier), allerdings macht diese Funktion in der
Regel nichts.
.sys ist normalerweise ein Gerätetreiber.
Wie auch bei Treibern, erwartet Windows eine Prüfsumme in der PE-Datei, die korrekt
sein muss\footnote{Hiew(\myref{Hiew}) kann diese berechnen}.

\myindex{Windows!Windows Vista}
Ab Windows Vista, muss eine Treiberdatei auch mit einer digitalen Signatur versehen
sein. Anderseits wird das Laden des Treibers fehlschlagen.

\myindex{MS-DOS}
Jede PE-Datei beginnt mit einem kleinen DOS-Programm, welches eine Nachricht in der
Art wie folgt ausgibt:
\q{Dieses Programm kann nicht im MS-DOS-Modus gestartet werden.}---wenn versucht wird
das Programm unter DOS oder Windows 3.1 zu starten (\ac{OS}e die das PE-Format
nicht kennen).

\subsubsection{Terminologie}

\myindex{VA}
\myindex{Base address}
\myindex{RVA}
\myindex{Windows!IAT}
\myindex{Windows!INT}

\begin{itemize}
\item Modul: eine separate Datei, .exe oder .dll

\item Prozess: ein Programm das in den Speicher geladen wurde und gerade ausgeführt
wird. Besteht meist aus einer .exe-Datei und einer Reihe von .dll-Dateien.

\item Prozessspeicher: der Speicher, mit dem der Prozess arbeitet. Jeder Prozess
hat seinen eigenen Bereich. Hier befinden sich in der Regel Module, der Stack,
\gls{heap}(s) usw.

\item \ac{VA}: Eine Adresse welche zur Laufzeit in einem Programm verwendet wird.

\item Basisadresse (eines Moduls): die Adresse im Prozessspeicher an der das Modul
geladen wird. Der \ac{OS}-Lader kann diese ändern wenn die Basisadresse bereits
von einem vorher geladenen Modul verwendet wird.

\item \ac{RVA}: die \ac{VA}-Adresse minus der Basisadresse.

Viele Adressen in PE-Dateien-Tabellen nutzen \ac{RVA}-Adressen.

%\item
%Data directory\EMDASH{}...

\item \ac{IAT}: Ein Array von Adressen importierter Module\footnote{\PietrekPE}.
Manchmal zeigt das \TT{IMAGE\_DIRECTORY\_ENTRY\_IAT}-Daten-Verzeichnis auf \ac{IAT}.
\label{IDA_idata}
Erwähnenswert ist es dass \ac{IDA} (v6.1) eine Pseudosektion namens \TT{.idata}
für \ac{IAT} allozieren kann, selbst wenn \ac{IAT} Teil einer anderen Sektion ist.

\item \ac{INT}: Ein Array von Namen zu importierender Symbole\footnote{\PietrekPE}.
\end{itemize}

\subsubsection{Basisadresse}

Das Problem ist, dass mehrere Modulprogrammierer DLL-Dateien für die Nutzung für andere
vorbereiten können, es jedoch nicht möglich ist festzulegen welche Adressen für diese
Module verwendet werden.

Das ist der Grund, warum in dem Fall wenn zwei für einen Prozess notwendige DLLs
dieselbe Basisadresse haben, eine davon an die Basisadresse geladen wird und die
andere an eine andere freie Stelle im Prozessspeicher. Jede virtuelle Adresse der
zweiten DLL wird korrigiert.

\par Mit \ac{MSVC} generiert der Linker oft .exe-Dateien mit der Basisadresse
\TT{0x400000}\footnote{der Ursprung dieser Adress-Auswahl ist hier: \href{http://go.yurichev.com/17041}{MSDN}},
und mit der Code-Sektion die bei \TT{0x401000} beginnt.
Dies bedeutet, dass die \ac{RVA} des Beginns der Code-Sektion \TT{0x1000} ist.

DLLs werden vom MSVC-Linker oft mit der Basisadresse \TT{0x10000000}
erzeugt\footnote{Dies kann mit der Linker-Option /BASE geändert werden}.

\myindex{ASLR}

Es gibt einen weiteren Grund warum Module an verschiedenen Basisadressen, in diesem
Fall an zufälligen Adressen, geladen werden: \ac{ASLR}\footnote{\href{http://go.yurichev.com/17140}{Wikipedia}}.

\myindex{Shellcode}

Ein Shellcode der auf einem kompromittierten System ausgeführt werden soll, muss
Systemfunktionen aufrufen und dementsprechend deren Adressen kennen.

In älteren \ac{OS} (in der \gls{Windows NT}-Reihe: bevor Windows Vista) wurden
System-DLL (wie kernel32.dll, user32.dll) immer an bekannte Adressen geladen und
mit dem Wissen, dass deren Versionen selten wechseln, waren die Adressen der Funktionen
fest und konnten direkt aufgerufen werden.

Um dies zu verhindern läd \ac{ASLR} das Programm und alle benötigten Module jedes
Mal an zufällige Basisadressen.

\ac{ASLR}-Unterstützung ist in der PE-Datei mit dem Flag
\TT{IMAGE\_DLL\_CHARACTERISTICS\_DYNAMIC\_BASE} markiert \InSqBrackets{siehe \Russinovich}.

\subsubsection{Subsystem}

Es existiert auch ein \emph{Subsystem}-Feld, dass normalerweise wie folgt ist:

\myindex{Native API}

\begin{itemize}
\item native\footnote{bedeutet, dass das Modul eine eigene API statt Win32 nutzt} (.sys-Treiber),

\item console (Konsolenanwendung) oder

\item \ac{GUI} (Anwendung mit grafischer Oberfläche).
\end{itemize}

\subsubsection{Betriebssystem-Version}

Eine PE-Datei spezifiziert auch die minimale Windows-Version die sie benötigt um
ladbar zu sein.

Die Tabelle mit Versionsnummern in der PE-Datei und die entsprechenden Windows-Codenamen
ist hier\footnote{\href{http://go.yurichev.com/17044}{Wikipedia}}.

\myindex{Windows!Windows NT4}
\myindex{Windows!Windows 2000}
Beispielsweise kompiliert \ac{MSVC} 2005 .exe-Dateien für Windows NT4 (Version 4.00),
\ac{MSVC} 2008 jedoch nicht (die erzeugten Dateien haben die Version 5.00, es ist
mindestens Windows 2000 notwendig um sie ausführen zu können).

\myindex{Windows!Windows XP}

\ac{MSVC} 2012 erzeugt standardmäßig .exe-Dateien mit der Version 6.00, die mindestens
Windows Vista benötigen.
Mit Änderungen an den Compiler-Option\footnote{\href{http://go.yurichev.com/17045}{MSDN}}
ist es jedoch möglich eine Kompilierung für Windows XP zu erzwingen.

\subsubsection{Sektionen}

Abschnitte in Sektionen sind in allen Formaten für ausführbare Dateien vorhanden.

Dies wurde entwickelt um Code von Daten und Daten von Konstanten zu trennen.

\begin{itemize}
\item Entweder das \emph{IMAGE\_SCN\_CNT\_CODE}- oder \emph{IMAGE\_SCN\_MEM\_EXECUTE}-Flag
ist in der Code-Sektion gesetzt. Dies kennzeichnet ausführbaren Code.

\item In der Daten-Sektion sind die Flags \emph{IMAGE\_SCN\_CNT\_INITIALIZED\_DATA},
\emph{IMAGE\_SCN\_MEM\_READ} und \emph{IMAGE\_SCN\_MEM\_WRITE} gesetzt.

\item In einer leeren Sektion mit uninitialisierten Daten sind die Flags
\emph{IMAGE\_SCN\_CNT\_UNINITIALIZED\_DATA}, \emph{IMAGE\_SCN\_MEM\_READ} und
\emph{IMAGE\_SCN\_MEM\_WRITE} gesetzt.

\item In der Sektion mit konstanten Daten (die schreibgeschützt ist), sind die
Flags \emph{IMAGE\_SCN\_CNT\_INITIALIZED\_DATA} und \emph{IMAGE\_SCN\_MEM\_READ}
gesetzt, nicht jedoch \emph{IMAGE\_SCN\_MEM\_WRITE}.
Ein Prozess wird abstürzen, wenn er versucht hier schreibend zuzugreifen.

\end{itemize}

\myindex{TLS}
\myindex{BSS}
Jede Sektion in einer PE-Datei kann einen Namen haben, auch wenn dieser nicht
unbedingt wichtig ist.
Oft (aber nicht immer) ist die Code-Sektion mit \TT{.text} benannt, die Datensektion
mit \TT{.data}, und die Sektion mit konstanten Daten \TT{.rdata} für \emph{(readable data)}.

Andere verbreitete Sektionsnamen sind:

\myindex{MIPS}
\begin{itemize}
\item \TT{.idata}: Import-Sektion
\ac{IDA} kann eine Pseudosektion mit dem Namen \myref{IDA_idata} erzeugen.
\item \TT{.edata}: Export-Sektion (selten)
\item \TT{.pdata}: Sektion, die alle Informationen über Ausnahmen in Windows NT
für MIPS, \ac{IA64} und x64 enthält: \myref{SEH_win64}
\item \TT{.reloc}: Reloc-Sektion
\item \TT{.bss}: uninitialisierte Daten (\ac{BSS})
\item \TT{.tls}: Thread-lokaler Speicher (\ac{TLS})
\item \TT{.rsrc}: Ressourcen
\item \TT{.CRT}: kann in Binärdateien vorhanden sein die mit alten MSVC-Compilern
erzeugt wurden
\end{itemize}

Pack- und Verschlüsselungsprogramme für PE-Dateien verändern häufig die Sektionsnamen
oder ersetzen sie durch eigene Namen.

\ac{MSVC} ermöglicht es Daten in beliebigen benannte Sektion zu
deklarieren\footnote{\href{http://go.yurichev.com/17047}{MSDN}}.

Einige Compiler und Linker können eine Sektion mit Debug-Symbolen und anderen
Debug-Informationen hinzufügen (MinGW zum Beispiel).
\myindex{Windows!PDB}
Nichtsdestotrotz ist dies in der aktuellen Version von \ac{MSVC} möglich.
Hier gibt es separate \gls{PDB}-Dateien für diesen Zweck.

Nachfolgend der Aufbau der PE-Sektion in dieser Datei:

\begin{lstlisting}
typedef struct _IMAGE_SECTION_HEADER {
  BYTE  Name[IMAGE_SIZEOF_SHORT_NAME];
  union {
    DWORD PhysicalAddress;
    DWORD VirtualSize;
  } Misc;
  DWORD VirtualAddress;
  DWORD SizeOfRawData;
  DWORD PointerToRawData;
  DWORD PointerToRelocations;
  DWORD PointerToLinenumbers;
  WORD  NumberOfRelocations;
  WORD  NumberOfLinenumbers;
  DWORD Characteristics;
} IMAGE_SECTION_HEADER, *PIMAGE_SECTION_HEADER;
\end{lstlisting}
\footnote{\href{http://go.yurichev.com/17048}{MSDN}}

\myindex{Hiew}
Ein Wort zur Terminologie: \emph{PointerToRawData} wird in Hiew \q{Offset}
und \emph{VirtualAddress} \q{RVA} genannt.

\subsubsection{Datensektion}
Die Datensektion in der Datei kann kleiner sein als im Arbeitsspeicher. Zum
Beispiel können einige Variablen initialisiert sein und andere nicht.
Compiler und Linker sammeln diese alle in einer einzelnen Sektion, aber
der erste Teil davon ist initialisiert und in der Datei enthalten, während
ein anderer in der Datei fehlt um diese kleiner zu machen.
\emph{VirtualSize} wird gleich groß sein wie die Sektion im Speicher und
\emph{SizeOfRawData} wie die Sektion in der Datei.

IDA kann die Grenze der initialisierten und nicht initialisierten Teile
wie folgt anzeigen:

\begin{lstlisting}
...

.data:10017FFA                 db    0
.data:10017FFB                 db    0
.data:10017FFC                 db    0
.data:10017FFD                 db    0
.data:10017FFE                 db    0
.data:10017FFF                 db    0
.data:10018000                 db    ? ;
.data:10018001                 db    ? ;
.data:10018002                 db    ? ;
.data:10018003                 db    ? ;
.data:10018004                 db    ? ;
.data:10018005                 db    ? ;

...
\end{lstlisting}

\subsubsection{Relocations (relocs)}
\label{subsec:relocs}

\ac{AKA} FIXUPs (zumindest in Hiew).
Diese sind ebenfalls in den meisten Formaten für ausführbare Dateien
vorhanden\footnote{Sogar in .exe-Dateien für MS-DOS.}.
Ausnahmen sind gemeinsam genutzte (dynamische) Bibliotheken, die \ac{PIC} enthalten

Wofür dienen die relocs?

Offensichtlich können Module an verschiedene Basisadressen geladen werden. Wie
wird jedoch zum Beispiel mit globalen Variablen umgegangen?
Auf diese muss anhand der Adresse zugegriffen werden. Eine Möglichkeit dazu ist
\PICcode{} (\myref{sec:PIC}), was aber nicht immer komfortabel ist

Aus diesem Grund existieren Relocation-Tabellen. Hier sind die Adressen die korrigiert
werden müssen aufgelistet, falls an eine andere Basisadresse geladen wird.

% TODO тут бы пример с HIEW или objdump..
Beispielsweise ist eine globale Variable an Adresse \TT{0x410000}. Das folgende
Listing zeigt wie auf diese zugegriffen wird:

\begin{lstlisting}
A1 00 00 41 00         mov         eax,[000410000]
\end{lstlisting}

Die Basisadresse des Moduls ist \TT{0x400000}, die \ac{RVA} der globalen Variablen
ist \TT{0x10000}.

Falls das Modul an die Basisadresse \TT{0x500000} geladen wird, muss die reale
Adresse der globalen Variablen auf \TT{0x510000} geändert werden.

\myindex{x86!\Instructions!MOV}

Wie man sieht ist die Adresse der Variablen in der Anweisung \TT{MOV}, nach dem
\TT{0xA1} kodiert.

Aus diesem Grund wird die Adresse der vier Byte nach \TT{0xA1} in die
Relocation-Tabelle geschrieben.

Wenn das Modul an einer anderen Basisadresse geladen wird, listet der \ac{OS}-Lader
alle Adressen in der Tabelle auf, findet jedes 32-Bit-Word auf das die Adresse zeigt,
subtrahiert die Basisadresse davon (das ergibt hier die \ac{RVA}) und addiert die
neue Basisadresse hinzu.

Wird das Modul an der originalen Basisadresse geladen passiert nichts.

Alle globalen Variablen können auf diese Weise behandelt werden.

Relocs können verschiedene Typen haben. In Windows für x86-Prozessoren ist dieser
üblicherweise \emph{IMAGE\_REL\_BASED\_HIGHLOW}.

\myindex{Hiew}

Übrigens sind Relocs in Hiew abgedunkelt, wie hier zu sehen: \figref{fig:scanf_ex3_hiew_1}.

\myindex{\olly}
\olly unterstreicht die Orte im Speicher auf die Relocs angewendet wurden,
beispielsweise: \figref{fig:switch_lot_olly3}.

\subsubsection{Exports und Imports}

\label{PE_exports_imports}
Wie bereits bekannt ist, muss jedes ausführbare Programm in irgendeiner Weise die
Dienste des \ac{OS} oder anderer DLL-Bibliotheken nutzen.

Die Funktionen eines Moduls (in der Regel eine DLL) muss irgendwie mit den Aufrufen
in anderen Modulen (.exe-Dateien oder eine andere DLL) verbunden werden.

Aus diesem Grund hat jede DLL eine \q{Export}-Tabelle die aus Funktionen und deren
Adressen in einem Modul besteht.

Außerdem hat jede .exe-Datei oder DLL \q{Imports}, eine Tabelle von Funktionen und
Liste von DLL-Dateinamen, die sie für die Ausführung benötigt.

Nach dem laden der Haupt-exe-Datei verarbeitet der \ac{OS}-Lader die Import-Tabelle:
er lädt die zusätzlichen DLL-Dateien, findet die Funktionsnamen in den DLL-Exports
und schreibt deren Adressen in die \ac{IAT} der .exe-Datei.

\myindex{Windows!Win32!Ordinal}

Wie man sieht, muss der Lader während seiner Aufgabe viele Funktionsnamen vergleichen.
Vergleiche von Zeichenketten sind jedoch nicht sehr performant, so dass es Unterstützung
für \q{ordinals} oder \q{hints} gibt, welche aus Funktionsnamen in der Tabelle
bestehen statt deren Namen.

Auf diese Weise können sie beim Laden einer DLL schneller gefunden werden.

Die \q{ordinals} also Zahlworte, sind in der \qquad{Export}-Tabelle immer vorhanden.

\myindex{MFC}
Auf diese Weise laden Programme welche die \ac{MFC}-Bibliothek nutzen die mfc*.dll
und es existieren keine \ac{MFC}-Namen wie in \ac{INT}.

% TODO example!
Wenn solche Programme in \IDA geladen werden, wird nach dem Pfad zu den mfc*.dll-Dateien
gefragt um die Funktionsnamen herausfinden zu können.

Wenn der Pfad zu diesen DLLs in \IDA nicht angegeben wird, erscheint \emph{mfc80\_123}
statt der Funktionsnamen.

\myparagraph{Import-Sektion}

Häufig wird eine separate Sektion mit dem Namen \TT{.idata} für die Import-Tabelle
und alle dafür relevanten Dinge angelegt. Dies ist aber keine strikte Regel.

Imports sind manchmal etwas verwirrend wegen der uneinheitlichen Terminologie.
Versuchen wir alle Informationen an einer Stelle zu sammeln.

\begin{figure}[H]
\centering
\myincludegraphics{OS/PE/unnamed0.png}
\caption{
Ein Schema dass alle PE-Datei-Strukturen im Zusammenhang mit Imports vereint}
\end{figure}

Die Hauptstruktur ist das Array \emph{IMAGE\_IMPORT\_DESCRIPTOR}. Jedes Element für
jede DLL wird importiert.

Jedes Element hat die \ac{RVA}-Adresse der Zeichenkette (DLL-Name) (\emph{Name}).

\emph{OriginalFirstThunk} ist die \ac{RVA}-Adresse der \ac{INT}-Tabelle. Diese ist
ein Array von \ac{RVA}-Adressen, jede davon zeigt auf eine Zeichenkette mit einem
Funktionsnamen. Jede Zeichenkette wird eine 16-Bit-Integerzahl voran gestellt
(\q{hint})---\q{ordinal} der Funktion).

Während des Ladens und falls es möglich ist die Funktion anhand der Zahl zu finden,
wird der Vergleich der Zeichenketten nicht auftauchen. Das Array ist mit Null terminiert.

Es gibt auch einen Zeiger zur \ac{IAT}-Tabelle mit dem Namen \emph{FirstThunk}. Dies
entspricht der \ac{RVA}-Adresse der Stelle an der der Lader die Adressen der aufgelösten
Funktionen schreibt.

Die Punkte an denen der Lader die Adressen schreibt werden in \IDA mit \emph{\_\_imp\_CreateFileA}
und so weiter gekennzeichnet.

Es gibt zumindest zwei Arten die vom Lader geschriebenen Adressen zu nutzen.

\myindex{x86!\Instructions!CALL}
\begin{itemize}
\item Der Code enthält eine Anweisung wie \emph{call \_\_imp\_CreateFileA}. Da das Feld
mit der Adresse der importierten Funktion in gewisser Weise eine globale Variable
ist, wird die Adresse der \emph{call}-Anweisung (plus 1 oder 2) zur Relocation-Tabelle
hinzugefügt. Dies gilt für den Fall falls das Modul an eine andere Basisadresse
geladen wird.

Aber offensichtlich kann dies die Relocation-Tabelle erheblich vergrößern, da
möglicherweise in dem Modul viele Aufrufe von importierten Funktionen enthalten sind.

Des weiteren verlangsamen die großen Tabellen das Laden der Module.

\myindex{x86!\Instructions!JMP}
\myindex{thunk-functions}
\item Für jede importierte Funktion wird nur ein Sprung mittels der \JMP-Anweisung
und einem Reloc darauf alloziert.
Solche Punkte werden auch \q{thunk}s genannt.

Alle Aufrufe zu den importierten Funktionen sind lediglich \CALL-Anweisungen auf
die entsprechenden \q{thunks}.
In diesem Fall sind keine zusätzlichen Relocs notwendig, da diese \CALL{}s eine
relative Adresse haben und nicht korrigiert werden müssen.
\end{itemize}

Diese beiden Methoden können miteinander kombiniert werden.

Der Linker kann mehrere einzelne \q{thunk}s erstellen, falls zu viele Aufrufe der
Funktion vorliegen. Dies ist jedoch nicht das Standardverhalten.

Übrigens muss das Array der Funktionsadressen auf das FirstThunk zeigt nicht unbedingt
in der \ac{IAT}-Sektion sein. Beispielsweise hat der Autor dieser Zeilen einmal das
PE\_add\_import\footnote{\href{http://go.yurichev.com/17049}{yurichev.com}}-Tool
geschrieben um Imports zu einer existierenden .exe-Datei hinzufügen zu können.

In einer früheren Version des Tools hat dieses an die Stelle der Funktion an der
ein Aufruf zu einer anderen DLL geschrieben werden sollte, folgenden Code erzeugt:

\begin{lstlisting}
MOV EAX, [yourdll.dll!function]
JMP EAX
\end{lstlisting}

FirstThunk zeigt auf die erste Anweisung. Mit anderen Worten, wenn yourdll.dll
geladen wird, schreibt der Lader die Adresse der Funktion \emph{function} direkt
in den Code.

Erwähnenswert ist, dass die Code-Sektion normalerweise schreibgeschützt ist.
Daher fügt das Tool das \emph{IMAGE\_SCN\_MEM\_WRITE}-Flag hinzu, da andernfalls
das Programm beim Laden mit dem Fehlercode 5 (Zugriff verweigert.) abstürzen
würde.

Man mag sich fragen: was ist wenn ein Programm mit ein paar DLLs kommt, die
sich nicht ändern (inklusive der Adressen aller DLL-Funktionen). Ist es möglich
den Lade-Vorgang zu beschleunigen?

Dies ist in der Tat möglich, wenn die Adressen der Funktionen bereits im Voraus
in das FirstThunk-Array geschrieben werden. Das \emph{Timestamp}-Feld ist in der
\emph{IMAGE\_IMPORT\_DESCRIPTOR}-Struktur vorhanden.

Wenn der Wert dort verfügbar ist, vergleicht der Lader diesen Wert mit dem
Zeitstempel der DLL-Datei.

Wenn der Wert gleich ist, macht der Lader nichts und der Vorgang kann schneller
sein. Dies wird \q{old-style binding}\footnote{\href{http://go.yurichev.com/17050}{MSDN}. Dort auch \q{new-style binding}.}
genannt.

\myindex{BIND.EXE}
Das Tool BIND.EXE ist für diesen Vorgang gedacht.

Um das Laden eigener Programme zu beschleunigen empfiehlt Matt Pietrek in \PietrekPEURL
das Binden kurz nach der Installation des Programms auf dem Rechner des Endanwenders
durchzuführen.

Komprimierungs- und Verschlüsselungsprogramme für PE-Dateien komprimieren bzw.
Verschlüsseln auch die Import-Tabellen.

In diesem Fall wird der Windows-Lader natürlich nicht alle notwendigen DLL-Dateien lasen.
\myindex{Windows!Win32!LoadLibrary}
\myindex{Windows!Win32!GetProcAddress}

Der Komprimierer / Verschlüsseler macht dies selber mithilfe der Funktionen
\emph{LoadLibrary()} und \emph{GetProcAddress()}.

Aus diesem Grund sind die beiden Funktionen oft im \ac{IAT} von gepackten Dateien.

In den Standard-DLLs der Windows-Installation ist \ac{IAT} häufig zu Beginn einer
PE-Datei zu finden. Vermutlich geschieht dies aus Optimierungsgründen.

Während die .exe-Datei geladen wird, befindet sich diese nicht als Ganzes im Speicher
(man denke an riesige Installationsprogramme welche verdächtig schnell geladen werden),
sondern ist \q{gemapped} und wird in Teilen geladen, wenn auf diese zugegriffen wird.

Vielleicht waren die Microsoft-Entwickler der Meinung, dass dies schneller ist.

\subsubsection{Ressourcen}

\label{PEresources}

Ressourcen in einer PE-Datei sind lediglich Sammlungen von Icons, Bildern,
Zeichenketten und Dialog-Beschreibungen.

Möglicherweise wurden Sie vom Hauptcode getrennt um mehrere Sprachen unterstützen
zu können und es einfacher ist einen Text oder ein Bild in der Sprache auszuwählen,
die zur Zeit im \ac{OS} eingestellt ist.

Ein Seiteneffekt ist, dass diese einfach editiert und in der ausführbaren Datei
zurück gespeichert werden können. Mit speziellen Editoren wie beispielsweise
(\myref{ResHack}) ist dies auch ohne spezielles Wissen möglich.

\subsubsection{.NET}

\myindex{.NET}

.NET-Progamme werden nicht in Maschinencode übersetzt sondern in einen speziellen Bytecode.
\myindex{OEP}
Streng genommen befindet sich anstatt des gewöhnlichen x86-Code der Bytecode in
der .exe-Datei. Der Einsprungpunkt (\ac{OEP}) jedoch zeigt auf ein kleines Fragment
x86-Code:

\begin{lstlisting}
jmp         mscoree.dll!_CorExeMain
\end{lstlisting}

Der .NET-Lader befindet sich in mscoree.dll, welche die PE-Datei verarbeitet.
\myindex{Windows!Windows XP}

Dies war in allen Windows-Versionen vor Windows XP der Fall. Seit Windows XP
ist der \ac{OS}-Lader in der Lage .NET-Dateien zu erkennen und diese ohne eine
\JMP-Anweisung auszuführen\footnote{\href{http://go.yurichev.com/17051}{MSDN}}.

\myindex{TLS}
\subsubsection{TLS}

Diese Sektion beinhaltet initialisierte Daten für \ac{TLS}(\myref{TLS}) (falls notwendig).
Wird ein neuer Thread gestartet, werden dessen \ac{TLS}-Daten mit den Daten dieser
Sektion initialisiert.

\myindex{TLS!Callbacks}
Abgesehen davon beinhaltet die Spezifikation für PE-Dateien auch die Möglichkeit
der Initialisierung der \ac{TLS}-Sektion, sogenannte TLS-Callbacks.

Falls diese vorhanden sind, werden sie aufgerufen bevor die Ausführungskontrolle
an den Haupteinsprungpunkt (\ac{OEP}) übergeben wird.

Dies ist sehr verbreitet bei Packern und Verschlüsselungsprogrammen für PE-Dateien.

\subsubsection{Tools}

\myindex{objdump}
\myindex{Cygwin}
\myindex{Hiew}
\label{ResHack}

\begin{itemize}
\item objdump (in Cygwin enthalten) um alle PE-Dateistrukturen auszugeben.

\item Hiew(\myref{Hiew}) als Editor.

\item pefile: Python-Bibliothek für die Verarbeitung von PE-Dateien\footnote{\url{http://go.yurichev.com/17052}}.

\item ResHack \acs{AKA} Resource Hacker: Ressourcen-Editor\footnote{\url{http://go.yurichev.com/17052}}.

\item PE\_add\_import\footnote{\url{http://go.yurichev.com/17049}}: einfaches Tool
um Symbole zur PE Importtabelle hinzuzufügen.

\item PE\_patcher\footnote{\href{http://go.yurichev.com/17054}{yurichev.com}}:
einfaches Tool um ausführbare PE-Dateien zu patchen.

\item PE\_search\_str\_refs\footnote{\href{http://go.yurichev.com/17055}{yurichev.com}}:
einfaches Tool zum Suchen von Funktionen in ausführbaren PE-Dateien die bestimmte
Zeichenketten nutzen.
\end{itemize}

\subsubsection{weitere Informationen}

% FIXME: bibliography per chapter or section
\begin{itemize}
\item Daniel Pistelli: The .NET File Format \footnote{\url{http://go.yurichev.com/17056}}
\end{itemize}
