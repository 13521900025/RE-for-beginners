\subsection{Win32 PE}
\label{win32_pe}
\myindex{Windows!Win32}

\acs{PE} est un format de fichier exécutable utilisé sur Windows.
La différence entre .exe, .dll et .sys est que les .exe et .sys n'ont en général
pas d'exports, uniquement des imports.

\myindex{OEP}

Une \ac{DLL}, tout comme n'importe quel fichier PE, possède un point d'entrée (\ac{OEP})
(la fonction DllMain() se trouve là) mais cette fonction ne fait généralement rien.
.sys est généralement un pilote de périphérique.
Pour les pilotes, Windows exige que la somme de contrôle soit présente dans le fichier
PE et qu'elle soit correcte
\footnote{Par exemple, Hiew(\myref{Hiew}) peut la calculer}.

\myindex{Windows!Windows Vista}
A partir de Windows Vista, un fichier de pilote doit aussi être signé avec une signature
numérique. Le chargement échouera sinon.

\myindex{MS-DOS}
Chaque fichier PE commence avec un petit programme DOS qui affiche un message comme
\q{Ce programme ne peut pas être lancé en mode DOS.}---si vous lancez sous DOS ou
Windows 3.1 (\ac{OS}-es qui ne supportent pas le format PE), ce message sera affiché.

\subsubsection{Terminologie}

\myindex{VA}
\myindex{Base address}
\myindex{RVA}
\myindex{Windows!IAT}
\myindex{Windows!INT}

\begin{itemize}
\item Module---un fichier séparé, .exe ou .dll.

\item Process---un programme chargé dans la mémoire et fonctionnant actuellement.
Consiste en général d'un fichier .exe et d'un paquet de fichiers .dll.

\item Process memory---la mémoire utilisée par un processus. Chaque processus a la
sienne. Il y a en général des modules chargés, la mémoire pour la pile, \gls{heap}(s),
etc.

\item \ac{VA}---une adresse qui est utilisée pendant le déroulement du programme.

\item Base address (du module)---l'adresse dans la mémoire du processus à laquelle
le module est chargé. Le chargeur de l'\ac{OS} peut la changer, si l'adresse de base
est occupée par un module déjà chargé.

\item \ac{RVA}---l'adresse \ac{VA}-moins l'adresse de base.

De nombreuses adresses dans les fichiers PE utilisent l'adresse \ac{RVA}.

%\item
%Data directory --- ...

\item \ac{IAT}---un tableau d'adresses des symboles importés \footnote{\PietrekPE}. 
Parfois, le répertoire de données \TT{IMAGE\_DIRECTORY\_ENTRY\_IAT} pointe sur l'\ac{IAT}.
\label{IDA_idata}
Il est utile de noter qu'\ac{IDA} (à partir de 6.1) peut allouer une pseudo-section
nommée \TT{.idata} pour l'\ac{IAT}, même si l'\ac{IAT} fait partie d'une autre section!

\item \ac{INT}---une tableau de noms de symboles qui doivent être importés\footnote{\PietrekPE}.
\end{itemize}

\subsubsection{Adresse de base}

Le problème est que plusieurs auteurs de module pevent préparer des fichiers DLL
pour que d'autres les utilisent et qu'il n'est pas possible de s'accorder pour assigner
une adresse à ces modules.

C'est pourquoi si deux DLLs nécessaires pour un processus ont la même adresse de base,
une des deux sera chargée à cette adresse de base, et l'autre---à un autre espace
libre de la mémoire du processus et chaque adresse virtuelle de la seconde DLL sera
corrigée.

\par Avec \ac{MSVC} l'éditeur de lien génère souvent les fichiers .exe avec une adresse
de base en \TT{0x400000}\footnote{L'origine de ce choix d'adresse est décrit ici:
\href{http://go.yurichev.com/17041}{MSDN}}, et avec une section de code débutant
en \TT{0x401000}.
Cela signifie que la \ac{RVA} du début de la section de code est \TT{0x1000}.

Les DLLs sont souvent générées par l'éditeur de lien de MSVC avec une adresse de
base de \TT{0x10000000}\footnote{Cela peut être changé avec l'option /BASE de l'éditeur
de liens}.

\myindex{ASLR}

Il y a une autre raison de charger les modules a des adresses de base variées, aléatoires
dans ce cas. C'est l'\ac{ASLR}\footnote{\href{http://go.yurichev.com/17140}{Wikipédia}}.

\myindex{Shellcode}

Un shellcode essayant d'être exécuté sur une système compromis doit appeler des fonctions
systèmes, par conséquent, connaitre leurs adresses.

Dans les anciens \ac{OS} (dans la série \gls{Windows NT}: avant Windows Vista), les
DLLs système (comme kernel32.dll, user32.dll) étaient toujours chargées à une adresse
connue, et si nous nous rappelons que leur version changeait rarement, l'adresse
des fonctions était fixe et le shellcode pouvait les appeler directement.

Pour éviter ceci, la méthode \ac{ASLR} charge votre programme et tous les modules
dont il a besoin à des adresses de base aléatoires, différentes à chaque fois.

Le support de l'\ac{ASLR} est indiqué dans un fichier PE en mettant le flag
\par \TT{IMAGE\_DLL\_CHARACTERISTICS\_DYNAMIC\_BASE} \InSqBrackets{voir \Russinovich}.

\subsubsection{Sous-système}

Il a aussi un champ \emph{sous-système}, d'habitude c'est:

\myindex{Native API}

\begin{itemize}
\item natif\footnote{Signifiant que le module utilise l'API Native au lieu de Win32} (.sys-driver), 

\item console (application en console) ou

\item \ac{GUI} (non-console).
\end{itemize}

\subsubsection{Version d'OS}

Un fichier PE indique aussi la version de Windows minimale requise pour pouvoir être
chargé.

La table des numéros de version stockés dans un fichier PE et les codes de Windows
correspondants est ici\footnote{\href{http://go.yurichev.com/17044}{Wikipédia}}.

\myindex{Windows!Windows NT4}
\myindex{Windows!Windows 2000}
Par exemple, \ac{MSVC} 2005 compile les fichiers .exe pour tourner sur Windows NT4
(version 4.00), mais \ac{MSVC} 2008 ne le fait pas (les fichiers générés ont une
version de 5.00, il faut au moins Windows 2000 pour les utiliser).

\myindex{Windows!Windows XP}

\ac{MSVC} 2012 génère des fichiers .exe avec la version 6.00 par défaut, visant au
moins Windows Vista.
Toutefois, en changeant l'option du compilateur\footnote{\href{http://go.yurichev.com/17045}{MSDN}},
il est possible de le forcer à compiler pour Windows XP.

\subsubsection{Sections}

Il semble que la division en section soit présente dans tous les formats de fichier
exécutable.

C'est divisé afin de séparer le code des données, et les données---des données constantes.

\begin{itemize}
\item Si l'un des flags \emph{IMAGE\_SCN\_CNT\_CODE} ou \emph{IMAGE\_SCN\_MEM\_EXECUTE}
est mis dans la section de code---c'est de code exécutable.

\item Dans la section de données---les flags \emph{IMAGE\_SCN\_CNT\_INITIALIZED\_DATA},\\
\emph{IMAGE\_SCN\_MEM\_READ} et \emph{IMAGE\_SCN\_MEM\_WRITE}.

\item Dans une section vide avec des données non initialisées---\\
\emph{IMAGE\_SCN\_CNT\_UNINITIALIZED\_DATA}, \emph{IMAGE\_SCN\_MEM\_READ} \\
et \emph{IMAGE\_SCN\_MEM\_WRITE}.

\item Dans une section de données constantes (une qui est protégée contre l'écriture),
les flags \emph{IMAGE\_SCN\_CNT\_INITIALIZED\_DATA} et \emph{IMAGE\_SCN\_MEM\_READ} peuvent
être mis, mais pas \emph{IMAGE\_SCN\_MEM\_WRITE}.
Un process plantera s'il essaye d'écrire dans cette section.
\end{itemize}

\myindex{TLS}
\myindex{BSS}
Chaque section dans un fichier PE peut avoir un nom, toutefois, ce n'est pas très
important.
Souvent (mais pas toujours) la section de code est appelée \TT{.text}, la section
de données---\TT{.data}, la section de données constante --- \TT{.rdata} \emph{(readable data)}.
D'autres noms de section courants sont:

\myindex{MIPS}
\begin{itemize}
\item \TT{.idata}---section d'imports.
\ac{IDA} peut créer une pseudo-setion appelée ainsi: \myref{IDA_idata}.
\item \TT{.edata}---section d'exports section (rare)
\item \TT{.pdata}---section contenant toutes les informations sur les exceptions
dans Windows NT pout MIPS, \ac{IA64} et x64: \myref{SEH_win64}
\item \TT{.reloc}---section de relocalisation
\item \TT{.bss}---données non initialisées (\ac{BSS})
\item \TT{.tls}---stockage local d'un thread (\ac{TLS})
\item \TT{.rsrc}---ressources
\item \TT{.CRT}---peut être présente dans les fichiers binaire compilés avec des
anciennes versions de MSVC
\end{itemize}

Les packeurs/chiffreurs rendent souvent les noms de sections initelligibles ou les
remplacent avec des noms qui leurs sont propres.

\ac{MSVC} permet de déclarer des données dans une section de n'importe quel nom
\footnote{\href{http://go.yurichev.com/17047}{MSDN}}.

Certains compilateurs et éditeurs de liens peuvent ajouter une section avec des symboles
et d'autres informations de débogage (MinGW par exemple).
\myindex{Windows!PDB}
Toutefois ce n'est pas comme cela dans les dernières versions de \ac{MSVC} (des fichiers
\gls{PDB} séparés sont utilisés dans ce but).\\
\\
Voici comment une section PE est décrite dans le fichier:

\begin{lstlisting}
typedef struct _IMAGE_SECTION_HEADER {
  BYTE  Name[IMAGE_SIZEOF_SHORT_NAME];
  union {
    DWORD PhysicalAddress;
    DWORD VirtualSize;
  } Misc;
  DWORD VirtualAddress;
  DWORD SizeOfRawData;
  DWORD PointerToRawData;
  DWORD PointerToRelocations;
  DWORD PointerToLinenumbers;
  WORD  NumberOfRelocations;
  WORD  NumberOfLinenumbers;
  DWORD Characteristics;
} IMAGE_SECTION_HEADER, *PIMAGE_SECTION_HEADER;
\end{lstlisting}
\footnote{\href{http://go.yurichev.com/17048}{MSDN}}

\myindex{Hiew}
Un mot à propos de le terminilogie: \emph{PointerToRawData} est appelé \q{Offset} dans
Hiew et \emph{VirtualAddress} est appelé \q{RVA} ici.

\subsubsection{Relocs (relocalisation)}
\label{subsec:relocs}

\ac{AKA} FIXUP-s (au moins dans Hiew).

Elles sont aussi présentes dans presque tous les formats de fichiers exécutables
\footnote{Même dans les fichiers .exe pour MS-DOS}.
Les exceptions sont les bibliothèques dynamiques partagées compilées avec \ac{PIC},
ou tout autre code \ac{PIC}.

À quoi servent-elles?

Évidemment, les modules peuvent être chargés à différentes adresse de base, mais
comment traiter les variables globales, par exemple? Elles doivent être accèdées
par adresse. Une solution est le \PICcode{} (\myref{sec:PIC}).
Mais ce n'est pas toujours pratique.

C'est pourquoi une table des relocalisations est présente.
Elle contient les adresses des points qui doivent être corrigés en cas de chargement
à une adresse de base différente.

% TODO тут бы пример с HIEW или objdump..
Par exemple, il y a une variable globale à l'adresse \TT{0x410000} et voici comment
elle est accédée:

\begin{lstlisting}
A1 00 00 41 00         mov         eax,[000410000]
\end{lstlisting}

L'adresse de base du module est \TT{0x400000}, le \ac{RVA} de la variable globale est
\TT{0x10000}.

Si le module est chargé à l'adresse de base \TT{0x500000}, l'adresse réelle de la
variable globale doit être \TT{0x510000}.

\myindex{x86!\Instructions!MOV}

Comme nous pouvons le voir, l'adresse de la variable est encodée dans l'instruction
\TT{MOV}, après l'octet \TT{0xA1}.

C'est pourquoi l'adresse des 4 octets aprés \TT{0xA1} est écrite dans la table de
relocalisations.

Si le module est chargé à une adresse de base différente, le chargeur de l'\ac{OS}
parcourt toutes les adresses dans la table, trouve chaque mot de 32-bit vers lequel
elles pointent, soustrait l'adresse de base d'origine (nous obtenons le \ac{RVA} ici),
et ajoute la nouvelle adresse de base.

Si un module est chargé à son adresse de base originale, rien ne se passe.

Toutes les variables globales sont traitées ainsi.

Relocs peut avoir différent types, toutefois, dans Windows pour processeurs x86, le type est généralement \\
\emph{IMAGE\_REL\_BASED\_HIGHLOW}.

\myindex{Hiew}

Á propos, les relocs sont plus foncés dans Hiew, par exemple: \figref{fig:scanf_ex3_hiew_1}.

\myindex{\olly}
\olly souligne les endroits en mémoire où sont appliqués les relocs, par exemple:
\figref{fig:switch_lot_olly3}.

\subsubsection{Exports et imports}

\label{PE_exports_imports}
Comme nous le savons, tout programme exécutable doit utiliser les services de l'\ac{OS}
et autres bibliothèques d'une manière ou d'une autre.

On peut dire que les fonctions d'un module (en général DLL) doivent être reliées
aux points de leur appel dans les autres modules (fichier .exe ou eutre DLL).

Pour cela, chaque DLL a une table d'\q{exports}, qui consiste en les fonctions plus
leur adresse dans le module.

Et chaque fichier .exe ou DLL possède des \q{imports}, une table des fonctions requises
pour l'exécution incluant une liste des noms de DLL.

Après le chargement du fichier .exe principal, le chargeur de l'\ac{OS} traite la
table des imports: il charge les fichiers DLL additionnels, trouve les noms des fonctions
parmis les exports des DLL et écrit leur adresse dans le module \ac{IAT} de l'.exe
principal.

\myindex{Windows!Win32!Ordinal}

Comme on le voit, pendant le chargement, le chargeur doit comparer de nombreux noms
de fonctions, mais la comparaison des chaînes n'est pas une procédure très rapide,
donc il y a un support pour \q{ordinaux} ou \q{hints}, qui sont les numéros de fonctions
stockés dans une table au lieu de leurs noms.

C'est ainsi qu'elles peuvent être localisées plus rapidement lors du chargement d'une
DLL. Les ordinaux sont toujours présents dans la table d'\q{export}.

\myindex{MFC}
% TODO clarifier
Par exemple, un programme utilisant la bibliothèque \ac{MFC} charge en général mfc*.dll
par ordinaux, et dans un programme n'ayant aucun nom de fonction \ac{MFC} dans \ac{INT}.

% TODO example!
Lors du chargement d'un tel programme dans \IDA, il va demander le chemin vers les
fichiers mfc*.dll, afin de déterminer le nom des fonctions.

Si vous ne donnez pas le chemin vers ces DLLs à \IDA, il y aura \emph{mfc80\_123} à
la place des noms de fonctions.

\myparagraph{Section d'imports}

Souvent, une section séparée est allouée pour la table d'imports et tout ce qui y
est relatif (avec un nom comme \TT{.idata}), toutefois, ce n'est pas une règle stricte.

Les imports sont un sujet prêtant à confusion à cause de la terminologie confuse.
Essayons de regrouper toutes les informations au même endroit.

\begin{figure}[H]
\centering
\myincludegraphics{OS/PE/unnamed0.png}
\caption{
Un schéma qui regroupe toutes les structures concernant les imports d'un fichier PE}
\end{figure}

La structure principale est le tableau \emph{IMAGE\_IMPORT\_DESCRIPTOR}, qui comprend
un élément pour chaque DLL importée.

Chaque élément contient l'adresse \ac{RVA} de la chaîne de texte (nom de la DLL) (\emph{Name}).

\emph{OriginalFirstThunk} est l'adresse \ac{RVA} de la table \ac{INT}.
C'est un tableau d'adresses \ac{RVA}, qui pointe chacune sur une chaîne de texte
avec le nom d'une fonction.
Chaque chaîne est préfixée par un entier 16-bit (\q{hint})---\q{ordinal} de la fonction.

Pendant le chargement, s'il est possible de trouver une fonction par ordinal, la
comparaison de chaînes n'a pas lieu. Le tableau est terminé par zéro.

Il y a aussi un pointeur sur la table \ac{IAT} appelé \emph{FirstThunk}, qui est juste
l'adresse \ac{RVA} de l'endroit où le chargeur écrit l'adresse résolue de la fonction.

Les endroits où le chargeur écrit les adresses sont marqués par \IDA comme ceci:
\emph{\_\_imp\_CreateFileA}, etc.

Il y a au moins deux façons d'utiliser les adresses écritent par le chargeur.

\myindex{x86!\Instructions!CALL}
\begin{itemize}
\item Le code aura des instructions comme \emph{call \_\_imp\_CreateFileA}, et puisque
le champ avec l'adresse de la fonction importée est en un certain sens une variable
globale, l'adresse de l'instruction \emph{call} (plus 1 ou 2) doit être ajoutée à la
table de relocs, pour le cas où le module est chargé à une adresse de base différente.

Mais, évidemment, ceci augmente significativement la table de relocs.

Car il peut y avoir un grand nombre d'appels aux fonctions importées dans le module.

En outre, une grosse table de relocs ralenti le processus de chargement des modules.

\myindex{x86!\Instructions!JMP}
\myindex{thunk-functions}
\item Pour chaque fonction importée, il y a seulement un saut alloué, en utilisant
l'instruction \JMP ainsi qu'un reloc sur lui.
De tel point sont aussi appelés \q{thunks}.

Tous les appels aux fonction importées sont juste des instructions \CALL au \q{thunk}
correspondant. Dans ce cas, des relocs supplémentaires ne sont pas nécessaire car
cex \CALL{}-s ont des adresses relatives et ne doivent pas être corrigés.
\end{itemize}

Ces deux méthodes peuvent être combinées.

Il est possible que l'éditeur de liens créé des \q{thunk}s individuels si il n'y
a pas trop d'appels à la fonction, mais ce n'est pas fait par défaut \\
\\
Á propos, le tableau d'adresses de fonction sur lequel pointe FirstThunk ne doit
pas nécessairement se trouver dans la section \ac{IAT}. Par exemple, j'ai écrit une
fois l'utilitaire PE\_add\_import\footnote{\href{http://go.yurichev.com/17049}{yurichev.com}}
pour ajouter des imports à un fichier .exe existant.

Quelques temps plus tôt, dans une version précédente de cet utilitaire, à la place
de la fonction que vous vouliez substituer par un appel à une autre DLL, mon utilitaire
écrivait le code suivant:

\begin{lstlisting}
MOV EAX, [yourdll.dll!function]
JMP EAX
\end{lstlisting}

FirstThunk pointe sur la première instruction. En d'autres mots, en chargeant yourdll.dll,
le chargeur écrit l'adresse de la fonction \emph{function} juste après le code.

Il est à noter qu'une section de code est en général protégée contre l'écriture,
donc mon utilitaire ajoute le flag \\
\emph{IMAGE\_SCN\_MEM\_WRITE}
pour la section de code. Autrement, le programme planterait pendant le chargement
avec le code d'erreur 5 (access denied). \\
\\
On pourrait demander: pourrais-je fournir à un programme un ensemble de fichiers
DLL qui n'est pas supposé changer (incluant les adresses des fonctions DLL), est-il
possible d'accélèrer le processus de chargement?

Oui, il est possible d'écrire les adresses des fonctions qui doivent être importées
en avance dans le tableau FirstThunk. Le champ \emph{Timestamp} est présent dans la
structure \emph{IMAGE\_IMPORT\_DESCRIPTOR}.

Si une valeur est présente ici, alors le loader compare cette valeur avec la date
et l'heure du fichier DLL.

Si les valeurs sont égales, alors le loader ne fait rien, et le processus de chargement
peut être plus rapide.
Ceci est appelé \q{old-style binding}
\footnote{\href{http://go.yurichev.com/17050}{MSDN}. Il y a aussi le \q{new-style binding}.}.
\myindex{BIND.EXE}

L'utilitaire BIND.EXE dans le SDK est fait pour ça.
Pour accélèrer le chargement de votre programme, Matt Pietrek dans \PietrekPEURL,
suggère de faire le binding juste après l'installation de votre programme sur l'ordinateur
de l'utilisateur final. \\
\\
Les packeurs/chiffreurs de fichier PE peuvent également compresser/chiffrer la table
des imports.

Dans ce cas, le loader de Windows, bien sûr, ne va pas charger les DLLs nécessaires.
\myindex{Windows!Win32!LoadLibrary}
\myindex{Windows!Win32!GetProcAddress}

Par conséquent, le packeur/chiffreur fait cela lui même, avec l'aide des fonctions
\emph{LoadLibrary()} et \emph{GetProcAddress()}.

C'est pourquoi ces deux fonctions sont souvent présentes dans l'\ac{IAT} des fichiers
packés.\\
\\
Dans les DLLs standards de l'installation de Windows, \ac{IAT} es souvent situé juste
après le début du fichier PE.
Supposément, c'est fait par soucis d'optimisation.

Pendant le chargement, le fichier .exe n'est pas chargé dans la mémoire d'un seul
coup (rappelez-vous ces fichiers d'installation énormes qui sont démarrés de façon
douteusement rapide), ils sont \q{mappés}, et chargés en mémoire par parties lorsqu'elles
sont accèdées.

Les développeurs de Microsoft ont sûrement décidé que ça serait plus rapide.

\subsubsection{Ressources}

\label{PEresources}

Les ressources dans un fichier PE sont seulement un ensemble d'icônes, d'images,
de chaînes de texte, de descriptions de boîtes de dialogues.

Peut-être qu'elles ont été séparées du code principal afin de pouvoir être multilingue,
et il est plus simple de prendre du texte ou une image pour la langue qui est définie
dans l'\ac{OS}. \\
\\
Par effet de bord, elles peuvent être éditées facilement et sauvées dans le fichier
exécutable, même si on n'a pas de connaissance particulière, en utilisant l'éditeur
ResHack, par exemple (\myref{ResHack}).

\subsubsection{.NET}

\myindex{.NET}

Les programmes .NET ne sont pas compilés en code machine, mais dans un bytecode spécial.
\myindex{OEP}
Strictement parlant, il y a du bytecode à la place du code x86 usuel dans le fichier
.eexe, toutefois, le point d'entrée (\ac{OEP}) point sur un petit fragment de code
x86:

\begin{lstlisting}
jmp         mscoree.dll!_CorExeMain
\end{lstlisting}

Le loader de .NET se trouve dans mscoree.dll, qui traite le fichier PE.
\myindex{Windows!Windows XP}

Il en était ainsi dans tous les \ac{OS}es pre-Windows XP. Á partir de XP, le loader
de l'\ac{OS} est capable de détecter les fichiers .NET et le lance sans exécuter
cette instruction \JMP
\footnote{\href{http://go.yurichev.com/17051}{MSDN}}.

\myindex{TLS}
\subsubsection{TLS}

Cette section contient les données initialisées pour le \ac{TLS}(\myref{TLS}) (si
nécessaire). Lorsque qu'une nouvelle thread démarre, ses données \ac{TLS} sont initialisées
en utilisant les données de cette section. \\
\\
\myindex{TLS!Callbacks}

Á part ça, la spécification des fichiers PE fourni aussi l'initialisation de la section
\ac{TLS}, aussi appelée TLS callbacks.

Si elles sont présentes, elles doivent être appelées avant que le contrôle ne soit
passé au point d'entrée principal (\ac{OEP}).

Ceci est largement utilisé dans les packeurs/chiffreurs de fichiers PE.

\subsubsection{Outils}

\myindex{objdump}
\myindex{Cygwin}
\myindex{Hiew}
\label{ResHack}

\begin{itemize}
\item objdump (présent dans cygwin) pour afficher toutes les structures d'un fichier PE.

\item Hiew(\myref{Hiew}) comme éditeur.

\item pefile---bibliothèque-Python pour le traitement des fichiers PE \footnote{\url{http://go.yurichev.com/17052}}.

\item ResHack \acs{AKA} Resource Hacker---éditeur de ressources\footnote{\url{http://go.yurichev.com/17052}}.

\item PE\_add\_import\footnote{\url{http://go.yurichev.com/17049}}---outil simple
pour ajouter des symboles à la table d'imports d'un exécutable au format PE.

\item PE\_patcher\footnote{\href{http://go.yurichev.com/17054}{yurichev.com}}---outil
simple pour patcher les exécutables PE.

\item PE\_search\_str\_refs\footnote{\href{http://go.yurichev.com/17055}{yurichev.com}}---outil
simple pour chercher une fonction dans un exécutable PE qui utilise une certaine chaîne de texte.
\end{itemize}

\subsubsection{Autres lectures}

% FIXME: bibliography per chapter or section
\begin{itemize}
\item Daniel Pistelli---The .NET File Format \footnote{\url{http://go.yurichev.com/17056}}
\end{itemize}

