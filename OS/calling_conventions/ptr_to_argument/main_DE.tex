\subsection{Einen Zeiger auf ein Argument verarbeiten}
\label{pointer_to_argument}

\dots mehr als das ist es sogar möglich, einen Zeiger auf ein Funktionsargument
zu nehmen und an eine weitere Funktion zu übergeben:

\lstinputlisting[style=customc]{OS/calling_conventions/ptr_to_argument/10.c}

Es ist schwierig die Funktionsweise zu verstehen, aber der folgende Code bring Klarheit:

\lstinputlisting[caption=\Optimizing MSVC 2010,style=customasmx86]{OS/calling_conventions/ptr_to_argument/MSVC_2010_O3.asm}

Die Adresse der Stelle im Stack an der $a$ übergeben wird, wird lediglich an eine
weitere Funktion übergeben.
Diese verändert den Wert der mit dem Zeiger übergeben wird und \printf gibt anschließend
den veränderten Wert aus.

\par Der aufmerksame Leser mag sich fragen, was mit der Aufrufkonvention ist, in der
Funktionsargumente in Registern übergeben werden.

Das ist eine Situation, in der \emph{Shadow Space} genutzt wird.

Der Eingangswert wird vom Register in den \emph{Shadow Space} des lokalen Stacks
kopiert und dann diese Adresse an die andere Funktion übergeben:

\lstinputlisting[caption=\Optimizing MSVC 2012 x64,style=customasmx86]{OS/calling_conventions/ptr_to_argument/MSVC_2012_x64_O3.asm}

GCC sichert den Eingangswert ebenfalls auf dem lokalen Stack:

\lstinputlisting[caption=\Optimizing GCC 4.9.1 x64,style=customasmx86]{OS/calling_conventions/ptr_to_argument/GCC491_x64_O3.s}

GCC für ARM64 tut genau das gleiche, allerdings wird hier der Platz \emph{Register Save Area} genannt:

\lstinputlisting[caption=\Optimizing GCC 4.9.1 ARM64,style=customasmARM]{OS/calling_conventions/ptr_to_argument/GCC49_ARM64_O3.s}

Übrigens eine gleiche Nutzung des \emph{Shadow Space} wird auch hier beschrieben: \myref{variadic_arith_registers}.
