% TODO proof-reading
\subsection{Простая вычисляющая функция}

Продолжим с простой вычисляющей функцией.

\begin{lstlisting}[style=customjava]
public class calc
{
	public static int half(int a)
	{
		return a/2;
	}
}
\end{lstlisting}


Это тот случай, когда используется инструкция \INS{iconst\_2}:

\begin{lstlisting}
  public static int half(int);
    flags: ACC_PUBLIC, ACC_STATIC
    Code:
      stack=2, locals=1, args_size=1
         0: iload_0       
         1: iconst_2      
         2: idiv          
         3: ireturn       
\end{lstlisting}
         
\INS{iload\_0} 
Берет нулевой аргумент функции и заталкивает его в стек.
\INS{iconst\_2} заталкивает в стек 2.

Вот как выглядит стек после исполнения этих двух инструкций:

% FIXME: TikZ
\begin{lstlisting}
      +---+
TOS ->| 2 |
      +---+
      | a |
      +---+
\end{lstlisting}


\INS{idiv} просто берет два значения на вершине стека (\ac{TOS}), 
делит одно на другое и оставляет результат на вершине (\ac{TOS}):

% FIXME: TikZ
\begin{lstlisting}
      +--------+
TOS ->| result |
      +--------+
\end{lstlisting}

\INS{ireturn} берет его и возвращает.


Продолжим с числами с плавающей запятой, двойной точности:

\begin{lstlisting}[style=customjava]
public class calc
{
	public static double half_double(double a)
	{
		return a/2.0;
	}
}
\end{lstlisting}

\begin{lstlisting}[caption=Constant pool]
...
   #2 = Double             2.0d
...
\end{lstlisting}

\begin{lstlisting}
  public static double half_double(double);
    flags: ACC_PUBLIC, ACC_STATIC
    Code:
      stack=4, locals=2, args_size=1
         0: dload_0       
         1: ldc2_w        #2                  // double 2.0d
         4: ddiv          
         5: dreturn       
\end{lstlisting}


Почти то же самое, но инструкция \INS{ldc2\_w} используется для загрузки константы 
2.0 из пула констант.

Также, все три инструкции имеют префикс \emph{d}, что означает, что они работают с переменными
типа \emph{double}.


Теперь перейдем к функции с двумя аргументами:

\begin{lstlisting}[style=customjava]
public class calc
{
	public static int sum(int a, int b)
	{
		return a+b;
	}
}
\end{lstlisting}

\begin{lstlisting}
  public static int sum(int, int);
    flags: ACC_PUBLIC, ACC_STATIC
    Code:
      stack=2, locals=2, args_size=2
         0: iload_0       
         1: iload_1       
         2: iadd          
         3: ireturn       
\end{lstlisting}


\INS{iload\_0} загружает первый аргумент функции (a), \INS{iload\_1} --- второй (b).

Вот так выглядит стек после исполнения обоих инструкций:

\begin{lstlisting}
      +---+
TOS ->| b |
      +---+
      | a |
      +---+
\end{lstlisting}


\INS{iadd} складывает два значения и оставляет результат на \ac{TOS}:

\begin{lstlisting}
      +--------+
TOS ->| result |
      +--------+
\end{lstlisting}


Расширим этот пример до типа данных \emph{long}:

\begin{lstlisting}[style=customjava]
	public static long lsum(long a, long b)
	{
		return a+b;
	}
\end{lstlisting}

\dots получим:

\begin{lstlisting}
  public static long lsum(long, long);
    flags: ACC_PUBLIC, ACC_STATIC
    Code:
      stack=4, locals=4, args_size=2
         0: lload_0       
         1: lload_2       
         2: ladd          
         3: lreturn       
\end{lstlisting}


Вторая инструкция \TT{lload} берет второй аргумент из второго слота.

Это потому что 64-битное значение \emph{long} занимает ровно два 32-битных слота.


Немного более сложный пример:

\begin{lstlisting}[style=customjava]
public class calc
{
	public static int mult_add(int a, int b, int c)
	{
		return a*b+c;
	}
}
\end{lstlisting}

\begin{lstlisting}
  public static int mult_add(int, int, int);
    flags: ACC_PUBLIC, ACC_STATIC
    Code:
      stack=2, locals=3, args_size=3
         0: iload_0       
         1: iload_1       
         2: imul          
         3: iload_2       
         4: iadd          
         5: ireturn       
\end{lstlisting}


Первый шаг это умножение. Произведение остается на \ac{TOS}:

\begin{lstlisting}
      +---------+
TOS ->| product |
      +---------+
\end{lstlisting}

\TT{iload\_2} загружает третий аргумент (c) в стек:

\begin{lstlisting}
      +---------+
TOS ->|    c    |
      +---------+
      | product |
      +---------+
\end{lstlisting}


Теперь инструкция \TT{iadd} может сложить два значения.
