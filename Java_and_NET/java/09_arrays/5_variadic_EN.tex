% TODO proof-reading
\subsubsection{Variadic functions}

Variadic functions actually use arrays:


\begin{lstlisting}[style=customjava]
	public static void f(int... values)
	{
		for (int i=0; i<values.length; i++)
			System.out.println(values[i]);
	}

	public static void main(String[] args) 
	{
		f (1,2,3,4,5);
	}
\end{lstlisting}

\begin{lstlisting}
  public static void f(int...);
    flags: ACC_PUBLIC, ACC_STATIC, ACC_VARARGS
    Code:
      stack=3, locals=2, args_size=1
         0: iconst_0      
         1: istore_1      
         2: iload_1       
         3: aload_0       
         4: arraylength   
         5: if_icmpge     23
         8: getstatic     #2       // Field java/lang/System.out:Ljava/io/PrintStream;
        11: aload_0       
        12: iload_1       
        13: iaload        
        14: invokevirtual #3       // Method java/io/PrintStream.println:(I)V
        17: iinc          1, 1
        20: goto          2
        23: return        
\end{lstlisting}

\ttf just takes an array of integers using \TT{aload\_0} at offset 3.

Then it gets the array's size, etc.


\begin{lstlisting}
  public static void main(java.lang.String[]);
    flags: ACC_PUBLIC, ACC_STATIC
    Code:
      stack=4, locals=1, args_size=1
         0: iconst_5      
         1: newarray       int
         3: dup           
         4: iconst_0      
         5: iconst_1      
         6: iastore       
         7: dup           
         8: iconst_1      
         9: iconst_2      
        10: iastore       
        11: dup           
        12: iconst_2      
        13: iconst_3      
        14: iastore       
        15: dup           
        16: iconst_3      
        17: iconst_4      
        18: iastore       
        19: dup           
        20: iconst_4      
        21: iconst_5      
        22: iastore       
        23: invokestatic  #4       // Method f:([I)V
        26: return        
\end{lstlisting}

The array is constructed in \main using the \TT{newarray} instruction, 
then it's filled, and \ttf is called.


Oh, by the way, array object is not destroyed at the end of \main.

\myindex{Garbage collector}
There are no destructors in Java at all, because the JVM has a garbage collector which does this
automatically, when it feels it needs to.


What about the \TT{format()} method?

It takes two arguments at input: a string and an array of objects:


\begin{lstlisting}
	public PrintStream format(String format, Object... args)
\end{lstlisting}
( \url{http://docs.oracle.com/javase/tutorial/java/data/numberformat.html} )

Let's see:

\begin{lstlisting}[style=customjava]
	public static void main(String[] args)
	{
		int i=123;
		double d=123.456;
		System.out.format("int: %d double: %f.%n", i, d);
	}
\end{lstlisting}

\begin{lstlisting}
  public static void main(java.lang.String[]);
    flags: ACC_PUBLIC, ACC_STATIC
    Code:
      stack=7, locals=4, args_size=1
         0: bipush        123
         2: istore_1      
         3: ldc2_w        #2         // double 123.456d
         6: dstore_2      
         7: getstatic     #4         // Field java/lang/System.out:Ljava/io/PrintStream;
        10: ldc           #5         // String int: %d double: %f.%n
        12: iconst_2      
        13: anewarray     #6         // class java/lang/Object
        16: dup           
        17: iconst_0      
        18: iload_1       
        19: invokestatic  #7         // Method java/lang/Integer.valueOf:(I)Ljava/lang/Integer;
        22: aastore       
        23: dup           
        24: iconst_1      
        25: dload_2       
        26: invokestatic  #8         // Method java/lang/Double.valueOf:(D)Ljava/lang/Double;
        29: aastore       
        30: invokevirtual #9         // Method java/io/PrintStream.format:(Ljava/lang/String;[Ljava/lang/Object;)Ljava/io/PrintStream;
        33: pop           
        34: return        
\end{lstlisting}

So values of the \emph{int} and \emph{double} types are first promoted to \TT{Integer} and \TT{Double} 
objects using the \TT{valueOf} methods.

The \TT{format()} method needs objects of type \TT{Object} at input, and since the \TT{Integer} and 
\TT{Double} classes are derived from the root \TT{Object} class, they suitable for elements 
in the input array.

On the other hand, an array is always homogeneous, i.e., it can't hold elements of 
different types, which makes it impossible to push \emph{int} and \emph{double} values in it.


An array of \TT{Object} objects is created at offset 13, 
an \TT{Integer} object is added to the array at offset 22, 
and a \TT{Double} object is added to the array at offset 29.


The penultimate \TT{pop} instruction discards the element at \ac{TOS}, 
so when \TT{return} is executed, the stack becomes empty (or balanced).

