% TODO proof-reading
\subsubsection{The only argument of the \main function is an array too}

We'll be using the only argument of the \main function, which is an array of strings:


\begin{lstlisting}[style=customjava]
public class UseArgument
{
	public static void main(String[] args)
	{
		System.out.print("Hi, ");
		System.out.print(args[1]);
		System.out.println(". How are you?");
	}
}
\end{lstlisting}

The zeroth argument is the program's name (like in \CCpp, etc.), 
so the 1st argument supplied by the user is 1st.


\begin{lstlisting}
  public static void main(java.lang.String[]);
    flags: ACC_PUBLIC, ACC_STATIC
    Code:
      stack=3, locals=1, args_size=1
         0: getstatic     #2      // Field java/lang/System.out:Ljava/io/PrintStream;
         3: ldc           #3      // String Hi, 
         5: invokevirtual #4      // Method java/io/PrintStream.print:(Ljava/lang/String;)V
         8: getstatic     #2      // Field java/lang/System.out:Ljava/io/PrintStream;
        11: aload_0       
        12: iconst_1      
        13: aaload        
        14: invokevirtual #4      // Method java/io/PrintStream.print:(Ljava/lang/String;)V
        17: getstatic     #2      // Field java/lang/System.out:Ljava/io/PrintStream;
        20: ldc           #5      // String . How are you?
        22: invokevirtual #6      // Method java/io/PrintStream.println:(Ljava/lang/String;)V
        25: return        
\end{lstlisting}

\TT{aload\_0} at 11 loads a \emph{reference} of the zeroth \ac{LVA} slot 
(1st and only \main argument).

\TT{iconst\_1} and \TT{aaload} at 12 and 13 take a \emph{reference} to the first (counting at 0) 
element of array.

The \emph{reference} to the string object is at \ac{TOS} at offset 14, and it is 
taken from there by \TT{println} method.

