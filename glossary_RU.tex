\newglossaryentry{tail call}
{
  name=хвостовая рекурсия,
  description={Это когда компилятор или интерпретатор превращает рекурсию 
  (с которой возможно это проделать, т.е. \IT{хвостовую}) в итерацию для эффективности}
}

\newglossaryentry{endianness}
{
  name=endianness,
  description={Порядок байт}
}

\newglossaryentry{caller}
{
  name=caller,
  description={Функция вызывающая другую функцию}
}

\newglossaryentry{callee}
{
  name=callee,
  description={Вызываемая функция}
}

\newglossaryentry{debuggee}
{
  name=debuggee,
  description={Отлаживаемая программа}
}

\newglossaryentry{leaf function}
{
  name=leaf function,
  description={Функция не вызывающая больше никаких функций}
}

\newglossaryentry{link register}
{
  name=link register,
  description=(RISC) {Регистр в котором обычно записан адрес возврата.
  Это позволяет вызывать leaf-функции без использования стека, т.е. быстрее}
}

\newglossaryentry{anti-pattern}
{
  name=anti-pattern,
  description={Нечто широко известное как плохое решение}
}

\newglossaryentry{stack pointer}
{
  name=указатель стека,
  description={Регистр указывающий на место в стеке}
}

\newglossaryentry{decrement}
{
  name=декремент,
  description={Уменьшение на 1}
}

\newglossaryentry{increment}
{
  name=инкремент,
  description={Увеличение на 1}
}

\newglossaryentry{loop unwinding}
{
  name=loop unwinding,
  description={Это когда вместо организации цикла на $n$ итераций, компилятор генерирует $n$ копий тела
  цикла, для экономии на инструкциях, обеспечивающих сам цикл}
}

\newglossaryentry{register allocator}
{
  name=register allocator,
  description={Функция компилятора распределяющая локальные переменные по регистрам процессора}
}

\newglossaryentry{quotient}
{
  name=частное,
  description={Результат деления}
}

\newglossaryentry{product}
{
  name={произведение},
  description={Результат умножения}
}

\newglossaryentry{NOP}
{
  name=NOP,
  description={\q{no operation}, холостая инструкция}
}

\newglossaryentry{POKE}
{
  name=POKE,
  description={Инструкция языка BASIC записывающая байт по определенному адресу}
}

\newglossaryentry{keygenme}
{
  name=keygenme,
  description={Программа, имитирующая защиту вымышленной программы, для которой нужно сделать 
  генератор ключей/лицензий}
} % TODO clarify: A software which generate key/license value to bypass sotfware protection?

\newglossaryentry{dongle}
{
  name=dongle,
  description={Небольшое устройство подключаемое к LPT-порту для принтера (в прошлом) или к USB}
}

\newglossaryentry{thunk function}
{
  name=thunk function,
  description={Крохотная функция делающая только одно: вызывающая другую функцию}
}

\newglossaryentry{user mode}
{
  name=user mode,
  description={Режим CPU с ограниченными возможностями в котором он исполняет прикладное ПО. ср.}
}

\newglossaryentry{kernel mode}
{
  name=kernel mode,
  description={Режим CPU с неограниченными возможностями в котором он исполняет ядро OS и драйвера. ср.}
}

\newglossaryentry{Windows NT}
{
  name=Windows NT,
  description={Windows NT, 2000, XP, Vista, 7, 8, 10}
}

\newglossaryentry{atomic operation}
{
  name=atomic operation,
  description={
  \q{$\alpha{}\tau{}o\mu{}o\varsigma{}$}
  %\q{atomic}
  означает \q{неделимый} в греческом языке, так что атомарная операция ---
  это операция которая гарантированно не будет прервана другими тредами}
}

% to be proofreaded (begin)
\newglossaryentry{NaN}
{
  name=NaN,
  description={
  	не число: специальные случаи чисел с плавающей запятой, 
  	обычно сигнализирующие об ошибках
  }
}

\newglossaryentry{basic block}
{
  name=basic block,
  description={
  	группа инструкций, не имеющая инструкций переходов,
	а также не имеющая переходов в середину блока извне.
	В \IDA он выглядит как просто список инструкций без строк-разрывов
  }
}

\newglossaryentry{NEON}
{
  name=NEON,
  description={\ac{AKA} \q{Advanced SIMD} --- от ARM}
}

\newglossaryentry{reverse engineering}
{
  name=reverse engineering,
  description={процесс понимания как устроена некая вещь, иногда, с целью клонирования оной}
}

\newglossaryentry{compiler intrinsic}
{
  name=compiler intrinsic,
  description={Специфичная для компилятора функция не являющаяся обычной библиотечной функцией.
	Компилятор вместо её вызова генерирует определенный машинный код.
	Нередко, это псевдофункции для определенной инструкции \ac{CPU}. Читайте больше:}
}

\newglossaryentry{heap}
{
  name=heap,
  description={(куча) обычно, большой кусок памяти предоставляемый \ac{OS}, так что прикладное ПО может делить его
  как захочет. malloc()/free() работают с кучей}
}

\newglossaryentry{name mangling}
{
  name=name mangling,
  description={применяется как минимум в \Cpp, где компилятору нужно закодировать имя класса,
  метода и типы аргументов в одной
  строке, которая будет внутренним именем функции. читайте также здесь}: \myref{namemangling}}
}

\newglossaryentry{xoring}
{
  name=xoring,
  description={нередко применяемое в английском языке, означает применение операции 
  \ac{XOR}}
}

\newglossaryentry{security cookie}
{
  name=security cookie,
  description={Случайное значение, разное при каждом исполнении. Читайте больше об этом тут}
}

\newglossaryentry{tracer}
{
  name=tracer,
  description={Моя простейшая утилита для отладки. Читайте больше об этом тут}: \myref{tracer}}
}

\newglossaryentry{GiB}
{
  name=GiB,
  description={Гибибайт: $2^{30}$ или 1024 мебибайт или 1073741824 байт}
}

\newglossaryentry{CP/M}
{
  name=CP/M,
  description={Control Program for Microcomputers: очень простая дисковая \ac{OS} использовавшаяся перед MS-DOS}
}

\newglossaryentry{stack frame}
{
  name=stack frame,
  description={Часть стека, в которой хранится информация, связанная с текущей функцией: локальные переменные,
  аргументы функции, \ac{RA}, итд.}
}

\newglossaryentry{jump offset}
{
  name=jump offset,
  description={Часть опкода JMP или Jcc инструкции, просто прибавляется к адресу следующей инструкции,
  и так вычисляется новый \ac{PC}. Может быть отрицательным}
}

\newglossaryentry{integral type}
{
  name=интегральный тип данных,
  description={обычные числа, но не вещественные. могут использоваться для передачи булевых типов и перечислений (enumerations)}
}

\newglossaryentry{real number}
{
  name=вещественное число,
  description={числа, которые могут иметь точку. в \CCpp это \Tfloat и \Tdouble}
}

\newglossaryentry{PDB}
{
  name=PDB,
  description={(Win32) Файл с отладочной информацией, обычно просто имена функций, 
  но иногда имена аргументов функций и локальных переменных}
}

\newglossaryentry{NTAPI}
{
  name=NTAPI,
  description={\ac{API} доступное только в линии Windows NT. 
  Большей частью не документировано Microsoft-ом}
}

\newglossaryentry{stdout}
{
  name=stdout,
  description={standard output}
}

\newglossaryentry{word}
{
  name=word,
  description={(слово) тип данных помещающийся в \ac{GPR}. 
  В компьютерах старше персональных, память часто измерялась не в байтах, 
  а в словах}
}

\newglossaryentry{arithmetic mean}
{
  name=среднее арифметическое,
  description={сумма всех значений, разделенная на их количество}
}
\newglossaryentry{padding}
{
  name=padding,
  description=\IT{Padding} в английском языке означает набивание подушки чем-либо для придания ей желаемой (большей)
  формы. В компьютерных науках, \IT{padding} означает добавление к блоку дополнительных байт, чтобы он имел нужный
  размер, например, $2^n$ байт.
}

