\mysection{XOR (OU exclusif)}
\label{XOR_property}

\input{fundamentals/XOR_property_FR}

\subsection{Langage courant}

L'opération XOR est présente dans le langage courant.
Lorsque quelqu'un demande ``s'il te plaît, achète des pommes ou des bananes'', ceci
signifie généralement ``achète le premier item ou le second, mais pas les deux''---ceci
est exactement un OU exclusif, car le OU logique signifierait ``les deux objets sont bien aussi''.

Certaines personnes suggèrent que ``et/ou'' devraient être utilisés dans le langage
courant pour mettre l'accent sur le fait que le OU logique est utilisé à la place
du OU exclusif: \url{https://en.wikipedia.org/wiki/And/or}.

\subsection{Chiffrement}

XOR est beaucoup utilisé à la fois par le chiffrement amateur (\ref{simple_XOR_encryption})
et \emph{réel} (au moins dans le \emph{réseau de Feistel}).

XOR est trés pratique ici car:
$cipher\_text = plain\_text \oplus key$ et alors:
$(plain\_text \oplus key) \oplus key = plain\_text$.

\subsection{\ac{RAID}4}
\index{RAID4}

\ac{RAID}4 offre une méthode très simple pour protéger les disques dur.
Par exemple, il y a quelques disques ($D_1$, $D_2$, $D_3$, etc.) et un disque de
parité ($P$).
Chaque bit/octet écrit sur le disque de parité est calculé et écrit au vol:

\begin{equation} \label{eq:RAID4}
P = D_1 \oplus D_2 \oplus D_3
\end{equation}

Si n'importe lequel des disques est défaillant, par exemple, $D_2$, il est restauré
en utilisant la même méthode:

\begin{equation}
D_2 = D_1 \oplus P \oplus D_3
\end{equation}

Si le disque de parité est défaillant, il est restauré en utilisant la méthode \ref{eq:RAID4}.
Si deux disques sont défaillants, alors il n'est pas possible de les restaurer les
deux.

\ac{RAID}5 est plus avancé, mais cet propriété de XOR y est encore utilisé.

C'est pourquoi les contrôleurs \ac{RAID} ont des ``accélérateurs XOR'' matériel
pour aider les opérations XOR sur de larges morceaux de données écrites au vol.
Depuis que les ordinateurs deviennent de plus en plus rapide, cela peut maintenant
être effectué au niveau logiciel, en utilisant \ac{SIMD}.

\subsection{Algorithme d'échange XOR}

C'est difficile à croire, mais ce code échangent les valeurs dans \EAX et \EBX sans
l'aide d'aucun autre registre ni d'espace mémoire.

\begin{lstlisting}[style=customasmx86]
xor eax, ebx
xor ebx, eax
xor eax, ebx
\end{lstlisting}

Cherchons comment ça fonctionne.
D'abord, récrivons le afin de retirer le langage d'assemblage x86:

\begin{lstlisting}
X = X XOR Y
Y = Y XOR X
X = X XOR Y
\end{lstlisting}

Qu'est ce que X et Y valent à chaque étape?
Gardez à l'esprit cette règle simple: $(X \oplus Y) \oplus Y = X$ pour toutes valeurs
de X et Y.

Regardons,
après la 1ère étape $X$ vaut $X \oplus Y$;
après la 2ème étape $Y$ vaut $Y \oplus (X \oplus Y) = X$;
après la 3ème étape $X$ vaut $(X \oplus Y) \oplus X = Y$.

Difficile de dire si on doit utiliser cette astuce, mais elle est un bon exemple de
démonstration des propriétés de XOR.

L'article de Wikipédia (\url{https://en.wikipedia.org/wiki/XOR_swap_algorithm}) donne
d'autres explication: l'addition et la soustraction peuvent être utilisées à la place
de XOR:

\begin{lstlisting}
X = X + Y
Y = X - Y
X = X - Y
\end{lstlisting}

Regardons:
après la 1ère étape $X$ vaut $X+Y$;
après la 2ème étape $Y$ vaut $X+Y-Y=X$;
après la 3ème étape $X$ vaut $X+Y-X=Y$.

\subsection{liste chaînée XOR}
\index{Doubly linked list}

Une liste doublement chaînée est une liste dans laquelle chaque élément a un lien
sur l'élément précédent et sur le suivant.
Ainsi, il est très facile de traverser la liste dans un sens ou dans l'autre.
\TT{std::list}, qui implémente les listes doublement chaînées en C++, est également
examiné dans ce livre: \ref{std_list}.

Donc chaque élément possède deux pointeurs.
Est-il possible, peut-être dans un environnement avec peu de mémoire, de garder toutes
ces fonctionnalités, avec un seul pointeur au lieu de deux?
Oui, si la valeur de $prev \oplus next$ est stockée dans cette cellule mémoire, qui
est habituellement appelé ``lien''.

Peut-être que nous pouvons dire que l'adresse de l'élément précédent est ``chiffrée''
en utilisant l'adresse de l'élément suivant et réciproquement:
l'adresse de l'élément suivant est ``chiffrée'' en utilisant l'adresse de l'élément
précédent.

Lorsque nous traversons cette liste en avant, nous connaissons toujours l'adresse
de l'élément précédent, donc nous pouvons ``déchiffrer'' ce champ et obtenir l'adresse
de l'élément suivant.
De même, il est possible de traverser cette liste en arrière, ``déchiffrer'' ce champ
en utilisant l'adresse de l'élément suivant.

Mais il n'est pas possible de trouver l'adresse de l'élément précédent ou suivant
d'un élément spécifique sans connaître l'adresse du premier.

Deux éléments pour compléter cette solution: le premier élément aura toujours l'adresse
de l'élément suivant sans aucun XOR, le dernier élément aura l'adresse du premier
élément sans aucun XOR.

Maintenant, résumons. Ceci est un exemple d'une liste doublement chaînée de 5 éléments.
$A_x$ est l'adresse de l'élément.

\begin{center}
\begin{tabular}{ | l | l | }
	\hline
	\HeaderColor adresse & \HeaderColor contenu du champ \emph{link} \\
	\hline
	$A_0$ & $A_1$ \\
	\hline
	$A_1$ & $A_0 \oplus A_2$ \\
	\hline
	$A_2$ & $A_1 \oplus A_3$ \\
	\hline
	$A_3$ & $A_2 \oplus A_4$ \\
	\hline
	$A_4$ & $A_3$ \\
	\hline
\end{tabular}
\end{center}

À nouveau, il est difficile de dire si quelqu'un doit utiliser ce truc rusé, mais
c'est une bonne démonstration des propriétés de XOR. Avec l'algorithme d'échange
avec XOR, l'article de Wikipédia montre des méthodes pour utiliser l'addition et
la soustraction au lieu de XOR:
\url{https://en.wikipedia.org/wiki/XOR_linked_list}.

% subsection:
\input{fundamentals/zobrist_FR}

\subsection{À propos}

Le \emph{OR} usuel est parfois appelé \emph{OU inclusif} (ou même \emph{IOR}), par opposition
au \emph{OU exclusif}.
C'est ainsi dans la bibliothèque Python \emph{operator}: il y est appelé \emph{operator.ior}.

