\mysection{\SignedNumbersSectionName}
\label{sec:signednumbers}
\myindex{Signed numbers}

\newcommand{\URLS}{\href{http://go.yurichev.com/17117}{wikipedia}}

Es gibt verschiedene Arten vorzeichenbehaftete Zahlen\footnote{\URLS} darzustellen,
jedoch ist das \q{Zweierkomplement} die am weitesten verbreitete in Computern.

Hier ist eine Tabelle für einige Byte-Werte:

\begin{center}
\begin{tabular}{ | l | l | l | l | }
\hline
\HeaderColor binär & \HeaderColor hexadezimal & \HeaderColor vorzeichenlos & \HeaderColor vorzeichenbehaftet \\
\hline
01111111 & 0x7f & 127 & 127 \\
\hline
01111110 & 0x7e & 126 & 126 \\
\hline
\multicolumn{4}{ |c| }{...} \\
\hline
00000110 & 0x6 & 6 & 6 \\
\hline
00000101 & 0x5 & 5 & 5 \\
\hline
00000100 & 0x4 & 4 & 4 \\
\hline
00000011 & 0x3 & 3 & 3 \\
\hline
00000010 & 0x2 & 2 & 2 \\
\hline
00000001 & 0x1 & 1 & 1 \\
\hline
00000000 & 0x0 & 0 & 0 \\
\hline
11111111 & 0xff & 255 & -1 \\
\hline
11111110 & 0xfe & 254 & -2 \\
\hline
11111101 & 0xfd & 253 & -3 \\
\hline
11111100 & 0xfc & 252 & -4 \\
\hline
11111011 & 0xfb & 251 & -5 \\
\hline
11111010 & 0xfa & 250 & -6 \\
\hline
\multicolumn{4}{ |c| }{...} \\
\hline
10000010 & 0x82 & 130 & -126 \\
\hline
10000001 & 0x81 & 129 & -127 \\
\hline
10000000 & 0x80 & 128 & -128 \\
\hline
\end{tabular}
\end{center}

\myindex{x86!\Instructions!JA}
\myindex{x86!\Instructions!JB}
\myindex{x86!\Instructions!JL}
\myindex{x86!\Instructions!JG}
Der Unterschied zwischen vorzeichenbehafteten und vorzeichenlosen Zahlen ist, dass wenn \TT{0xFFFFFFFE}
und \TT{0x00000002} ohne Vorzeichen repräsentiert werden, die erste Zahl (4294967294) größer ist als
die zweite Zahl (2).
Wenn beide Zahlen als vorzeichenbehaftet repräsentiert werden, wird die erste Zahl $-2$ und ist kleiner
als die zweite Zahl (2).
Das ist der Grund, warum bedingte Sprünge ~(\myref{sec:Jcc}) sowohl für vorzeichenbehaftete (e.g. \JG, \JL)
als auch vorzeichenlose (\JA, \JB) Operationen vorhanden sind.

Aus Gründen der Einfachheit ist hier was man wissen muss:

\begin{itemize}
\item Zahlen können mit oder ohne Vorzeichen sein.

\item vorzeichenbehaftete Datentypen in \CCpp:

  \begin{itemize}
    \item \TT{int64\_t} (-9,223,372,036,854,775,808 .. 9,223,372,036,854,775,807)
	  (-~9.2..~9.2 Quintillionen) oder \\
                \TT{0x8000000000000000..0x7FFFFFFFFFFFFFFF}),
    \item \Tint (-2,147,483,648..2,147,483,647 (-~2.15..~2.15Gb) oder \\
	    \TT{0x80000000..0x7FFFFFFF}),
    \item \Tchar (-128..127 oder \TT{0x80..0x7F}),
    \item \TT{ssize\_t}.
   \end{itemize}

	Vorzeichenlos:
  \begin{itemize}
	  \item \TT{uint64\_t} (0..18,446,744,073,709,551,615 
		  (~18 Quintillionen) oder \TT{0..0xFFFFFFFFFFFFFFFF}),
   \item \TT{unsigned int} (0..4,294,967,295 (~4.3Gb) oder \TT{0..0xFFFFFFFF}),
   \item \TT{unsigned char} (0..255 oder \TT{0..0xFF}), 
   \item \TT{size\_t}.
  \end{itemize}

\item Vorzeichenbehaftete Typen haben das Vorzeichen am \ac{MSB}: 1 bedeutet \q{Minus}, 0 bedeutet \q{Plus}.

\item Erweitern auf größere Datentypen ist einfach:
\myref{subsec:sign_extending_32_to_64}.

\label{sec:signednumbers:negation}
\item Negieren ist einfach: es müssen lediglich alle Bits invertiert und anschließend 1 addiert werden.

%TODO Not sure how to translate this into an understandable in german.
%We can keep in mind that a number of inverse sign is located on the opposite side at the same proximity from zero.
%The addition of one is needed because zero is present in the middle.

\myindex{x86!\Instructions!IDIV}
\myindex{x86!\Instructions!DIV}
\myindex{x86!\Instructions!IMUL}
\myindex{x86!\Instructions!MUL}
\myindex{x86!\Instructions!CBW}
\myindex{x86!\Instructions!CWD}
\myindex{x86!\Instructions!CWDE}
\myindex{x86!\Instructions!CDQ}
\myindex{x86!\Instructions!CDQE}
\myindex{x86!\Instructions!MOVSX}
\myindex{x86!\Instructions!SAR}
\item 
	Die Addition und Subtraktion funktioniert sowohl für Zahlen mit als auch ohne Vorzeichen.
	Für Multiplikation und Division gibt es bei x86 unterschiedliche Anweisungen:
	\TT{IDIV}/\TT{IMUL} für vorzeichenbehaftete und \TT{DIV}/\TT{MUL} für vorzeichenlose Zahlen.
\item
	Hier sind einige weitere Anweisungen die mit vorzeichenbehafteten Zahlen arbeiten:\\
	\TT{CBW/CWD/CWDE/CDQ/CDQE} (\myref{ins:CBW_CWD_etc}), \TT{MOVSX} (\myref{MOVSX}), \TT{SAR} (\myref{ins:SAR}).
\end{itemize}

Eine Tabelle mit negativen und positiven Werten (\ref{signed_tbl}) sieht aus wie ein Thermometer mit Celsius-Skala.
Das ist der Grund warum Addition und Subtraktion für Zahlen mit und ohne Vorzeichen gleich funktioniert:
wenn der erste Summand eine Markierung auf dem Thermometer ist und ein weiterer Summand addiert werden soll,
der positiv ist, muss lediglich die Markierung auf dem Thermometer um den Wert des zweiten Summanden nach
oben verschoben werden.
Ist der zweite Summand negativ, wird die Markierung um den entsprechenden, absoluten Wert nach unten verschoben.

Die Addition zweier negativer Zahlen funktioniert wie folgt:
Wenn beispielsweise -2 und -3 unter Verwendung eines 16-Bit-Registers addiert werden sollen,
ist die Darstellung 0xfffe beziehungsweise 0xfffd.
Wenn diese Werte ohne Vorzeichen addiert werden, ist das Ergebnis 0xfffe+0xfffd=0x1fffb.
Allerdings sollen 16-Bit-Register verwendet werden, also wird beim Ergebis die erste 1 abgeschnitten.
Es bleibt 0xfffb was -5 entspricht.
Dies funktioniert, weil -2 (oder 0xfffe) in natürlicher Sprache wie folgt repräsentiert werden kann:
``2 fehlt bei diesem Werte bis zum maximalen Wert in einem 16-Bit-Register + 1''.
-3 kann repräsentiert werden als ``\dots 3 fehlt in diesem Wert bis zu \dots''.
Der maximale Wert eines 16-Bit-Registers + 1 ist 0x10000.
Bei der Addition der beiden Zahlen und \emph{Abschneiden}  durch $2^{16}$ modulo,
wird $2+3=5$ \emph{fehlen}.

% subsections:
\subsection{Nutzen von IMUL anstatt MUL}
\label{IMUL_over_MUL}

\myindex{x86!\Instructions!MUL}
\myindex{x86!\Instructions!IMUL}
Beispiele wie \lstref{unsigned_multiply_C} in denen zwei vorzeichenlose Werte miteinander multipliziert
werden, werden zu \lstref{unsigned_multiply_lst} kompiliert, so dass \IMUL statt \MUL genutzt wird.

Dies ist eine wichtige Eigenschaft der \MUL- und \IMUL-Anweisung.
Zunächst produzieren beide einen 64-Bit-Wert wenn zwei 32-Bit-Werte miteinander multipliziert werden,
oder einen 128-Bit-Wert wenn zwei 64-Bit-Werte miteinander multipliziert werden (größtes mögliches \gls{product}
in 32-Bit-Umgebungen ist \\
\GTT{0xffffffff*0xffffffff=0xfffffffe00000001}).
Der \CCpp-Standard kennt keine Möglichkeit auf die höherwertige Hälfte eines Ergebnisses zuzugreifen
und ein \gls{product} hat immer die gleiche Größe wie die Faktoren. % TODO \gls{}?
Beide Anweisungen \MUL und \IMUL arbeiten auf die gleiche Weise wenn die höherwertige Hälfte ignoriert wird.
Das heißt die niederwertigere Hälfte ist die gleiche.
Dies ist eine wichtige Eigenschaft der Repräsentation von vorzeichenbehafteten Zahlen im \q{Zweierkomplements}.

Somit kann der \CCpp-Compiler jede dieser Anweisungen nutzen.

Die \IMUL-Anweisung ist jedoch vielseitiger als \MUL weil sie jedes Register als Quelle akzeptiert,
während \MUL einen der Faktoren in den Registern \AX, \EAX oder \RAX erwartet.
Des weiteren sichert \MUL das Ergebnis in dem \GTT{EDX:EAX} Paar in einer 32-Bit-Umgebung oder
in \GTT{RDX:RAX} in einer 64-Bit-Umgebung. Die Anweisung berechnet also immer das gesamte Ergebnis.
Im Gegensatz dazu ist es möglich beim Nutzen von \IMUL statt eines Paares von Zielregistern ein
einzelnes Register anzugeben. Die \ac{CPU} wird dann lediglich den niederwertigen Teil berechnen,
was zu einer höheren Geschwindigkeit führt [siehe Torborn Granlund, \emph{Instruction latencies and throughput for AMD and Intel x86 processors}\footnote{\url{http://yurichev.com/mirrors/x86-timing.pdf}]}).

Aus diesen Gründen ist es möglich, dass ein \CCpp-Compilers öfter \IMUL-Anweisungen als \MUL nutzt.

\myindex{Compiler intrinsic}
Trotzdem ist es möglich mit intrinsischen Funktionen (Intrinsics) des Compilers vorzeichenlose Multiplikationen durchzuführen
und das \emph{volle} Ergebnis zu erhalten.
Dies wird manchmal \emph{erweiterte Multiplikation} genannt.
MSVC hat Intrinsics zu diesem Zweck die \emph{\_\_emul}\footnote{\url{https://msdn.microsoft.com/en-us/library/d2s81xt0(v=vs.80).aspx}}
und \emph{\_umul128}\footnote{\url{https://msdn.microsoft.com/library/3dayytw9%28v=vs.100%29.aspx}} genannt werden.
GCC stellt einen \emph{\_\_int128}-Datentyp zur Verfügung und 64-Bit-Faktoren werden zuerst auf 128-Bit erweitert,
Anschließend wird das \gls{product} in einem anderen \emph{\_\_int128} gesichert.
Das Ergebnis ist um 64-Bit nach rechts geshiftet um die höherwertigen Hälfte des Ergebnisses
zu erhalten\footnote{Example: \url{http://stackoverflow.com/a/13187798}}.

\subsubsection{MulDiv()-Funktion in Windows}
\myindex{Windows!Win32!MulDiv()}

Windows hat eine MulDiv()-Funktion
\footnote{\url{https://msdn.microsoft.com/en-us/library/windows/desktop/aa383718(v=vs.85).aspx}},
welche die Multiplikation und Division vereint und zwei 32-Bit-Integer in einen temporären 64-Bit-Wert
speichert. Anschließend findet eine Division durch eine dritte 32-Bit-Integerzahl statt.
Dies ist einfacher als zwei Compiler-Intrinsics zu nutzen, weswegen die Microsoft-Entwickler diese
spezielle Funktion dafür einführten.
Gemessen an der Häufigkeit der Nutzung ist dies eine populäre Funktion.

\subsection{Einige weitere Anmerkungen zum Zweierkomplement}

\epigraph{Exercise 2-1. Write a program to determine the ranges of \TT{char}, \TT{short}, \TT{int}, and \TT{long}
variables, both \TT{signed} and \TT{unsigned}, by printing appropriate values from standard headers
and by direct computation.}{\KRBook}

\subsubsection{Das Maximum eines \gls{word} finden}
Die maximale Zahl in vorzeichenloser Form ist lediglich eine Zahl in der alle
Bits gesetzt sind: \emph{0xFF....FF}
(das entspricht -1 wenn das \gls{word} als vorzeichenbehaftete Zahl behandelt wird).
Also nimmt man ein \gls{word}, setzt alle Bits und bekommt den Wert:

\begin{lstlisting}[style=customc]
#include <stdio.h>

int main()
{
	unsigned int val=~0; // Aendern zu "unsigned char" um den maximalen Wert fur vorzeichenlose 8-Bit-Zahlen
	// 0-1 funktionieren auch oder nur -1
	printf ("%u\n", val); // %u fuer vorzeichenlos
};
\end{lstlisting}

Dies entspricht 4294967295 für 32-Bit-Integer.

\subsubsection{Das Minimum eines vorzeichenbehafteten \gls{word} finden}

%Minimum signed number has form of \emph{0x80....00}, i.e., most significant bit is set, while others are cleared.
%Maximum signed number has the same form, but all bits are inverted: \emph{0x7F....FF}.

%Let's shift a lone bit left until it disappears:

\begin{lstlisting}[style=customc]
#include <stdio.h>

int main()
{
	signed int val=1; // Aendern zu signed char um den maximalen Wert fur vorzeichenbehaftete 8-Bit-Zahlen
	while (val!=0)
	{
		printf ("%d %d\n", val, ~val);
		val=val<<1;
	};
};
\end{lstlisting}

Die Ausgabe ist:

\begin{lstlisting}
...

536870912 -536870913
1073741824 -1073741825
-2147483648 2147483647
\end{lstlisting}

Die letzten beiden Zahlen sind minimaler beziehungsweise maximaler Wert eines 32-Bit \emph{int}.


