\newglossaryentry{tail call}
{
  name=tail call,
  description={コンパイラ(またはインタプリタ)が再帰(\emph{末尾再帰}が可能なとき)を効率のため、反復に変換するときです}
}

\newglossaryentry{endianness}
{
  name=endianness,
  description={バイトオーダー: \myref{sec:endianness}}
}

\newglossaryentry{caller}
{
  name=caller,
  description={呼び出し元の関数}
}

\newglossaryentry{callee}
{
  name=callee,
  description={呼び出された関数}
}

\newglossaryentry{debuggee}
{
  name=debuggee,
  description={デバッグされるプログラム}
}

\newglossaryentry{leaf function}
{
  name=leaf function,
  description={他の関数から呼び出されない関数}
}

\newglossaryentry{link register}
{
  name=link register,
  description=(RISC) リターンアドレスが保存されるレジスタ。これはleaf functionをスタックを使わずに呼び出すのを可能にする
}

\newglossaryentry{anti-pattern}
{
  name=anti-pattern,
  description=
  {一般に、よくないと考えられるやり方}
}

\newglossaryentry{stack pointer}
{
  name={スタックポインタ},
  description=
  {スタックの場所を示すレジスタ}
}

\newglossaryentry{decrement}
{
  name={デクリメント},
  description={1の減算}
}

\newglossaryentry{increment}
{
  name={インクリメント},
  description={1の加算}
}

\newglossaryentry{loop unwinding}
{
  name=loop unwinding,
  description={
  $n$回のイテレーションのループコードをコンパイラが生成する代わりに、ループボディを$n$回コピーするコードを生成する。
  ループに使用する命令を削除するため}
}

\newglossaryentry{register allocator}
{
  name=register allocator,
  description=
  {CPUレジスタをローカル変数に割り当てるコンパイラの機構}
}

\newglossaryentry{quotient}
{
  name=商,
  description={除算結果}
}

\newglossaryentry{product}
{
  name=積,
  description={乗算結果}
}

\newglossaryentry{NOP}
{
  name=NOP,
  description={何もしない命令}
}

\newglossaryentry{POKE}
{
  name=POKE,
  description={特定のアドレスにバイトを書き込むための基本的な言語命令}
}

\newglossaryentry{keygenme}
{
  name=keygenme,
  description={キー/ライセンスジェネレータのようなソフトウェア保護を模倣するプログラム}
} % TODO clarify: A software which generate key/license value to bypass sotfware protection?

\newglossaryentry{dongle}
{
  name=dongle,
  description={LPTプリンタポートやUSBにさす小さなハードウェア。セキュリティトークンと似て、メモリを持ち、時には秘密の(暗号学的)ハッシュアルゴリズムを持つ}
}

\newglossaryentry{thunk function}
{
  name=thunk function,
  description={単一の役割だけ持つ小さな関数:他の関数を呼び出す等}
}

\newglossaryentry{user mode}
{
  name=user mode,
  description={アプリケーションソフトウェアが実行される制限されたCPUモード \gls{kernel mode}}
}

\newglossaryentry{kernel mode}
{
  name=kernel mode,
  description={OSカーネルやドライバが実行される制限のないCPUモード \gls{user mode}}
}

\newglossaryentry{Windows NT}
{
  name=Windows NT,
  description={Windows NT, 2000, XP, Vista, 7, 8, 10}
}

\newglossaryentry{atomic operation}
{
  name=atomic operation,
  description={
  \q{$\alpha{}\tau{}o\mu{}o\varsigma{}$}
  %\q{atomic}
  ギリシャ語で分割不可能を意味し、処理は他のスレッドに割り込まれないことが保証される
  }
}

% to be proofreaded (begin)
\newglossaryentry{NaN}
{
  name=NaN,
  description=
    {非数:float型の数の特殊なケースで、エラーが通知される}
}

\newglossaryentry{basic block}
{
  name=basic block,
  description=
    {jump/branch命令を持たない命令群で、ブロックの外側から内側へのjumpも持たない}
}

\newglossaryentry{NEON}
{
  name=NEON,
  description={\ac{AKA} \q{Advanced SIMD} --- \ac{SIMD} 拡張、ARM における}
}

\newglossaryentry{reverse engineering}
{
  name=reverse engineering,
  description={時にはクローンするため、どうやって動いているのかを理解しようとする行為}
}

\newglossaryentry{compiler intrinsic}
{
  name=compiler intrinsic,
  description={
    通常のライブラリ関数ではない、コンパイラ特有の関数。コンパイラは特定の機械語を生成する。しばしば、特定のCPU命令のための疑似関数 (\myref{sec:compiler_intrinsic})
  }
}

\newglossaryentry{heap}
{
  name=ヒープ,
  description={\ac{OS}が提供する大きなメモリの塊のことで、アプリケーションが好きなように分割することができる。malloc()/free()を呼び出して使用する}
}

\newglossaryentry{name mangling}
{
  name=name mangling,
  description={コンパイラがクラス、メソッド、および引数型の名前を1つの文字列にエンコードする必要がある少なくとも \CCpp で使用され、
  関数の内部名になります。 以下で詳細を読むことができます: \myref{namemangling}}
}

\newglossaryentry{xoring}
{
  name=xoring,
  description={英語圏でしばしばみられ、\ac{XOR}操作を適用する意味になる}
}

\newglossaryentry{security cookie}
{
  name=security cookie,
  description={
  ランダムな値で、実行の度に異なった値になります。以下で詳細を読むことができます: \myref{subsec:BO_protection}}
}

\newglossaryentry{tracer}
{
  name=tracer,
  description={シンプルなデバッグツールです。以下で詳細を読むことができます: \myref{tracer}}
}

\newglossaryentry{GiB}
{
  name=GiB,
  description={ギガバイト:$2^{30}$ または1024メガバイトまたは1073741824バイト}
}

\newglossaryentry{CP/M}
{
  name=CP/M,
  description={Control Program for Microcomputers: 以前使用されていたとても基本的なディスク\ac{OS}です MS-DOS}
}

\newglossaryentry{stack frame}
{
  name=stack frame,
  description=
  {現在の関数に固有の情報(ローカル変数、関数の引数、\ac{RA}など)を含むスタックの一部}
}

\newglossaryentry{jump offset}
{
  name=jump offset,
  description=
  {JMP命令またはJcc命令のオペコードの一部を次の命令のアドレスに追加する必要があります。これが新しい\ac{PC}の計算方法です。 負となる場合もあります}
}

\newglossaryentry{integral type}
{
  name=整数型,
  description=
  {通常の数字ですが、実際の数字はありません。 ブールと列挙型の変数を渡すために使用できます}
}

\newglossaryentry{real number}
{
  name=実数,
  description={
  小数点以下を含む可能性のある数字。 これは \CCpp で \Tfloat と \Tdouble です
  }
}

\newglossaryentry{PDB}
{
  name=PDB,
  description={(Win32) 
  デバッグ情報ファイルで、通常は関数名だけではなく、関数の引数とローカル変数名を含む}
}

\newglossaryentry{NTAPI}
{
  name=NTAPI,
  description=
  {APIはWindows NT系列でのみ使用できます。 Microsoftは大部分で文書化していません}
}

\newglossaryentry{stdout}
{
  name=stdout,
  description={standard output}
}

\newglossaryentry{word}
{
  name=word,
  description=
  {PCよりも昔のコンピュータでは、メモリサイズはバイトではなくワードで測定されることがしばしばでした}
}

\newglossaryentry{arithmetic mean}
{
  name={算術平均},
  description=
  {すべての値の合計を数で割った値}
}
\newglossaryentry{padding}
{
  name=padding,
  description=
  {英語で\emph{パディング}とは、(より大きな)望ましい形状にするために、あるものを枕に詰めることを意味します。
  コンピュータサイエンスでは、パディングとはブロックにバイトを追加することで、$2^n$バイトのようなサイズにすることを意味します。}
}

