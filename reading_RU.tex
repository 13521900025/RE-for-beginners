% TODO sync with English version
\chapter{Что стоит почитать}

\mysection{Книги и прочие материалы}

\subsection{Reverse Engineering}

\begin{itemize}
\item Eldad Eilam, \emph{Reversing: Secrets of Reverse Engineering}, (2005)

\item Bruce Dang, Alexandre Gazet, Elias Bachaalany, Sebastien Josse, \emph{Practical Reverse Engineering: x86, x64, ARM, Windows Kernel, Reversing Tools, and Obfuscation}, (2014)

\item Michael Sikorski, Andrew Honig, \emph{Practical Malware Analysis: The Hands-On Guide to Dissecting Malicious Software}, (2012)

\item Chris Eagle, \emph{IDA Pro Book}, (2011)

\item Reginald Wong, \emph{Mastering Reverse Engineering: Re-engineer your ethical hacking skills}, (2018)

\end{itemize}


Дмитрий Скляров --- ``Искусство защиты и взлома информации''.

Также, книги Криса Касперски.

\subsection{Windows}

\input{Win_reading}

\subsection{\CCpp}

\input{CCppBooks}

\subsection{x86 / x86-64}

\label{x86_manuals}
\begin{itemize}
\item Документация от Intel\footnote{\AlsoAvailableAs \url{http://www.intel.com/content/www/us/en/processors/architectures-software-developer-manuals.html}}

\item Документация от AMD\footnote{\AlsoAvailableAs \url{http://developer.amd.com/resources/developer-guides-manuals/}}

\item \AgnerFog{}\footnote{\AlsoAvailableAs \url{http://agner.org/optimize/microarchitecture.pdf}}

\item \AgnerFogCC{}\footnote{\AlsoAvailableAs \url{http://www.agner.org/optimize/calling_conventions.pdf}}

\item \IntelOptimization

\item \AMDOptimization
\end{itemize}

Немного устарело, но всё равно интересно почитать:

\MAbrash\footnote{\AlsoAvailableAs \url{https://github.com/jagregory/abrash-black-book}}
(он известен своей работой над низкоуровневой оптимизацией в таких проектах как Windows NT 3.1 и id Quake).

\subsection{ARM}

\begin{itemize}
\item Документация от ARM\footnote{\AlsoAvailableAs \url{http://infocenter.arm.com/help/index.jsp?topic=/com.arm.doc.subset.architecture.reference/index.html}}

\item \ARMSevenRef

\item \ARMSixFourRefURL

\item \ARMCookBook\footnote{\AlsoAvailableAs \url{http://go.yurichev.com/17273}}
\end{itemize}

\subsection{Язык ассемблера}

Richard Blum --- Professional Assembly Language.

\subsection{Java}

\JavaBook.

\subsection{UNIX}

\TAOUP

\subsection{Программирование}

\begin{itemize}

\item \RobPikePractice

\item Александр Шень\footnote{\url{http://imperium.lenin.ru/~verbit/Shen.dir/shen-progra.html}}

\item \HenryWarren.
Некоторые люди говорят, что трюки и хаки из этой книги уже не нужны, потому что годились только для \ac{RISC}-процессоров,
где инструкции перехода слишком дорогие.
Тем не менее, всё это здорово помогает лучше понять булеву алгебру и всю математику рядом.

\item (Для хард-корных гиков от информатики и математики) \TAOCP.
Некоторые люди спорят, стоит ли среднему программисту читать этот довольно тяжелый фундаментальный труд.
Но я бы сказал, что стоит просто пролистать, чтобы понимать, из чего вообще состоит информатика.

\end{itemize}

% subsection:
\subsection{\EN{Cryptography}\ES{Criptograf\'ia}\ITA{Crittografia}\RU{Криптография}\FR{Cryptographie}\DE{Kryptografie}\JPN{暗号学}}
\label{crypto_books}

\begin{itemize}
\item \Schneier{}

\item (Free) lvh, \emph{Crypto 101}\footnote{\AlsoAvailableAs \url{https://www.crypto101.io/}}

\item (Free) Dan Boneh, Victor Shoup, \emph{A Graduate Course in Applied Cryptography}\footnote{\AlsoAvailableAs \url{https://crypto.stanford.edu/~dabo/cryptobook/}}.
\end{itemize}



\iffalse
\subsection{Посвящение}

Как написано на первой странице этой книги, ``Эта книга посвящается Роберту Джордейну, Джону Соухэ, Ральфу Брауну и Питеру Абелю''.
Это авторы хорошо известных книг и справочников по языку ассемблеру из 1980-х и 1990-х:

\begin{itemize}
\item Роберт Джордейн -- Справочник программиста персональных компьютеров типа IBM PC, XT и AT (1986, в бывшем СССР издался в 1992)

\item Питер Нортон и Джон Соухэ -- Персональный компьютер фирмы IBM и операционная система MS-DOS (в бывшем ССР издалась в 1991), Язык ассемблера для IBM PC (там же издалась в 1992)
На самом деле, Джон Соухэ настоящий автор этих книг, можно сказать, он был литературным негром.
Также он настоящий автор Norton Commander-а.

\item Ральф Браун был известен справочником по прерываниям MS-DOS и BIOS -- ``Ralf Brown's Interrupt List''\footnote{\url{http://www.ctyme.com/rbrown.htm}}.

\item Питер Абель -- Ассемблер и программирование для IBM PC (в бывшем СССР издалась в 1992)
\end{itemize}

Это всё устаревшие книги, конечно.
Но может быть кто-то вспомнит ``те времена''.
\fi

