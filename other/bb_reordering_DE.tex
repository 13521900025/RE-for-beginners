\mysection{Basic Block Reordering}

% TODO __builtin_expect in GCC?

\subsection{Profile-guided Optimization}
\label{PGO}

\myindex{\oracle}
\myindex{Intel C++}

Diese Optimierungsmethode kann einige \gls{basic block}s zu anderen Sektionen der
ausführbaren Datei verschieben.

Offensichtlich gibt es Teile einer Funktion die öfter ausgeführt werden als andere
(zum Beispiel Schleifen-Rümpfe) und welche, die weniger oft ausgeführt werden
(beispielsweise Fehlerberichte oder Ausnahmebehandlungen).

Der Compiler fügt Messcode in die ausführbare Datei ein. Anschließend führt der
Programmierer diesen mit vielen Tests aus um Statistiken zu erstellen.

Der Compiler präpariert die ausführbare Datei mithilfe der erstellten Statistiken
insofern, dass alle weniger häufige Codeteile in eine andere Sektion der Datei
verschoben werden.

Als Ergebnis ist der häufig ausgeführte Funktionscode zusammengefasst, was sehr
wichtig für die Ausführgeschwindigkeit und die Cachebenutzung ist.

Ein Beispiel vom \oracle-Code, der mit dem Intel C++-Compiler übersetzt wurde:

\lstinputlisting[caption=orageneric11.dll (win32),style=customasmx86]{other/orageneric.lst}

Der Abstand der Adressen zwischen diesen beiden Code-Fragmenten beträgt fast 9 MB.

Alle weniger oft ausgeführten Codeteile wurden an das Ende der Code-Sektion der
DLL-Datei verschoben.

Dieser Teil der Funktion wurde vom Intel C++-Compiler mit dem \TT{VInfreq}-Präfix
markiert.

Man kann hier sehen, dass der Teil der Funktion der in die Logdatei schreibt (zum
Beispiel im Falle eines Fehlers oder einer Warnung) vermutlich selten oder vielleicht
gar nicht ausgeführt wurde als der Entwickler von Oracle die Statistiken erstellt hat.

Das Schreiben in log basic block gibt die Ausführkontrolle letztendlich wieder zurück
an den \q{heißen} Teil der Funktion.

Ein weiterer \q{seltener} Teil ist der \gls{basic block}, welcher den Fehlercode
27050 zurück gibt.

In Linux ELF-Dateien wird der selten ausgeführte Code vom Intel C++-Compiler in die
separate \TT{text.unlikely}-Sektion verschoben und der \q{heiße} Code in die Sektion
\TT{text.hot}.

Aus Sicht eines Reverse-Engineers kann diese Information helfen um die Funktion in
den Hauptteil und den Fehlerbehandlungsteil zu unterteilen.
