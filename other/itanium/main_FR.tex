\mysection{Itanium}
\label{itanium}
\myindex{Itanium}

Bien qu'elle n'ai pas réussi à percer, l'architecture  Intel Itanium (\ac{IA64}) est très intéressante.

Là où les CPUs \ac{OOE} réarrangent les instructions afin de les exécuter en parallèle, l'architecture
\ac{EPIC} a constitué une tentative pour déléguer cette décision au compilateur.

Les compilateurs en question étaient évidemment particulièrement complexes.

Voici un exemple de code pour l'architecture \ac{IA64} qui implémente un algorithme de
chiffrage simple du noyau Linux:

\lstinputlisting[caption=Linux kernel 3.2.0.4,style=customc]{other/itanium/tea_from_linux.c}

Et voici maintenant le résultat de la compilation:

\lstinputlisting[caption=Linux Kernel 3.2.0.4 pour Itanium 2 (McKinley)]{other/itanium/ia64_linux_3.2.0.4_mckinley.lst}

Nous constatons tout d'abord que toutes les instructions \ac{IA64} sont regroupées par 3.

Chaque groupe représente 16 octets (128 bits) et se décompose en une catégorie de code sur 5 bits
puis 3 instructions de 41 bits chacune.

Dans \IDA les groupes apparaissent sous la forme 6+6+4 octets~---le motif est facilement
reconnaissable.

En règle générale les trois instructions d'un groupe s'exécutent en parallèle, sauf si l'une d'elles
est associée à un \q{stop bit}.

Il est probable que les ingénieurs d'Intel et de HP ont collecté des statistiques qui leur ont permis
d'identifier les motifs les plus fréquents. Ils auraient alors décidé d'introduire une notion de
type de groupe (\ac{AKA} \q{templates}). Le type du groupe définit la catégorie des instructions
qu'il contient. Ces catégories sont au nombre de 12.

Par exemple, un groupe de type 0 représente \TT{MII}. Ceci signifie que la première instruction est
une lecture ou écriture en mémoire (M), la seconde et la troisième sont des manipulations d'entiers
(I).

Un autre exemple est le groupe de type 0x1d: \TT{MFB}. La première instruction est la aussi de type
mémoire (M), la seconde manipule un nombre flottant (F instruction \ac{FPU}), et la dernière est un
branchement (B).

Lorsque le compilateur ne parvient pas à sélectionner une instruction à inclure dans le groupe en
cours de construction, il utilise une instruction de type \ac{NOP}. Il existe donc des instructions
\TT{nop.i} pour remplacer ce qui devrait être une manipulation d'entier. De même un \TT{nop.m} est
utilisé pour combler un trou là où une instruction de type mémoire devrait se trouver.

Lorsque le programme est directement rédigé en assembleur, les instructions \ac{NOP}s sont ajoutées
de manière automatique.

Et ce n'est pas tout. Les groupes font eux-mêmes l'objet de regroupements.

Chaque instruction peut être marquée avec un \q{stop bit}. Le processeur exécute en parallèle toutes
les instructions, jusqu'à ce qu'il rencontre un \q{stop bit}.

En pratique, les processeurs Itanium 2 peuvent exécuter jusqu'à deux groupes simultanément, soit un
total de  6 instructions en parallèle.

Il faut évidemment que les instructions exécutées en parallèle, n'aient pas d'effet de bord entre
elles. Dans le cas contraire, le résultat n'est pas défini.
Le compilateur doit respecter cette contrainte ainsi que le nombre maximum de groupes simultanés du
processeur cible en plaçant les \q{stop bit} au bon endroit.

En langage d'assemblage, les bits d'arrêt sont identifiés par deux point-virgule situés après
l'instruction.

Ainsi dans notre exemple les instructions [90-ac] peuvent être exécutées simultanément. Le prochain
groupe est [b0-cc].

Nous observons également un bit d'arrêt en 10c. L'instruction suivante comporte elle aussi un bit
d'arrêt.

Ces deux instructions doivent donc être exécutées isolément des autres, (comme dans une architecture
\ac{CISC}).

En effet, l'instruction en 110 utilise le résultat produit par l'instruction précédente (valeur du
registre r26). Les deux instructions ne peuvent s'exécuter simultanément.

Il semble que le compilateur n'ai pas trouvé de meilleure manière de paralléliser les instructions,
ou en d'autres termes, de plus charger la \ac{CPU}. Les bits d'arrêt sont donc en trop grand nombre.

La rédaction manuelle de code en assembleur est une tâche pénible. Le programmeur doit effectuer
lui-même les regroupements d'instructions.

Bien sûr, il peut ajouter un bit d'arrêt à chaque instruction mais cela dégrade les performances
telles qu'elles ont été pensée pour l'Itanium.

Les codes source du noyau Linux contiennent un exemple intéressant d'un code assembleur produit
manuellement pour \ac{IA64}:

\url{http://go.yurichev.com/17322}.

On trouvera une introduction à l'assembleur Itanium dans :
[Mike Burrell, \emph{Writing Efficient Itanium 2 Assembly Code} (2010)]\footnote{\AlsoAvailableAs \url{http://yurichev.com/mirrors/RE/itanium.pdf}},
[papasutra of haquebright, \emph{WRITING SHELLCODE FOR IA-64} (2001)]\footnote{\AlsoAvailableAs \url{http://phrack.org/issues/57/5.html}}.

Deux autres caractéristiques très intéressantes d'Itanium sont l'\emph{exécution spéculative}
et le bit NaT (\q{not a thing}) qui ressemble un peu aux nombres \gls{NaN} : \\
\href{http://go.yurichev.com/17323}{MSDN}.

