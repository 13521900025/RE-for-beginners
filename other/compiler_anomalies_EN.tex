\mysection{Compiler's anomalies}
\label{anomaly:Intel}
\myindex{\CompilerAnomaly}

\subsection{\oracle 11.2 and Intel C++ 10.1}

\myindex{Intel C++}
\myindex{\oracle}
\myindex{x86!\Instructions!JZ}

Intel C++ 10.1, which was used for \oracle 11.2 Linux86 compilation, may emit two \JZ in row,
and there are no references to the second \JZ. The second \JZ is thus meaningless.

\lstinputlisting[caption=kdli.o from libserver11.a,style=customasmx86]{other/kdli.lst}

\begin{lstlisting}[caption=from the same code,style=customasmx86]
.text:0811A2A5                   loc_811A2A5: ; CODE XREF: kdliSerLengths+11C
.text:0811A2A5                                ; kdliSerLengths+1C1
.text:0811A2A5 8B 7D 08              mov     edi, [ebp+arg_0]
.text:0811A2A8 8B 7F 10              mov     edi, [edi+10h]
.text:0811A2AB 0F B6 57 14           movzx   edx, byte ptr [edi+14h]
.text:0811A2AF F6 C2 01              test    dl, 1
.text:0811A2B2 75 3E                 jnz     short loc_811A2F2
.text:0811A2B4 83 E0 01              and     eax, 1
.text:0811A2B7 74 1F                 jz      short loc_811A2D8
.text:0811A2B9 74 37                 jz      short loc_811A2F2
.text:0811A2BB 6A 00                 push    0
.text:0811A2BD FF 71 08              push    dword ptr [ecx+8]
.text:0811A2C0 E8 5F FE FF FF        call    len2nbytes
\end{lstlisting}

It is supposedly a code generator bug that was not found by tests, because 
resulting code works correctly anyway.

\input{other/anomaly2_EN}

\subsection{Summary}

Other compiler anomalies here in this book: 
\myref{anomaly:LLVM}, \myref{loops_iterators_loop_anomaly}, \myref{Keil_anomaly},
\myref{MSVC2013_anomaly},
\myref{MSVC_double_JMP_anomaly},
\myref{MSVC2012_anomaly}.

Such cases are demonstrated here in this book, to show that such compilers errors are possible and sometimes
one should not to rack one's brain while thinking why did the compiler generate such strange code.

