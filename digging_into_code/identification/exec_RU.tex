\mysection{Идентификация исполняемых файлов}

\subsection{Microsoft Visual C++}
\label{MSVC_versions}

Версии MSVC и DLL которые могут быть импортированы:

%\small
\begin{center}
\begin{tabular}{ | l | l | l | l | l | }
\hline
\HeaderColor Маркетинговая вер. & 
\HeaderColor Внутренняя вер. & 
\HeaderColor Вер. CL.EXE &
\HeaderColor Импорт.DLL &
\HeaderColor Дата выхода \\
\hline
% 4.0, April 1995
% 97 & 5.0 & February 1997
6		&  6.0	& 12.00	& msvcrt.dll	& June 1998		\\
		&	&	& msvcp60.dll	&			\\
\hline
.NET (2002)	&  7.0	& 13.00	& msvcr70.dll	& February 13, 2002	\\
		&	&	& msvcp70.dll	&			\\
\hline
.NET 2003	&  7.1	& 13.10 & msvcr71.dll	& April 24, 2003	\\
		&	&	& msvcp71.dll	&			\\
\hline
2005		&  8.0	& 14.00 & msvcr80.dll	& November 7, 2005	\\
		&	&	& msvcp80.dll	&			\\
\hline
2008		&  9.0	& 15.00 & msvcr90.dll	& November 19, 2007	\\
		&	&	& msvcp90.dll	&			\\
\hline
2010		& 10.0	& 16.00 & msvcr100.dll	& April 12, 2010 	\\
		&	&	& msvcp100.dll	&			\\
\hline
2012		& 11.0	& 17.00 & msvcr110.dll	& September 12, 2012 	\\
		&	&	& msvcp110.dll	&			\\
\hline
2013		& 12.0	& 18.00 & msvcr120.dll	& October 17, 2013 	\\
		&	&	& msvcp120.dll	&			\\
\hline
\end{tabular}
\end{center}
%\normalsize

msvcp*.dll содержит функции связанные с \Cpp{}, так что если она импортируется, скорее всего, 
вы имеете дело с программой на \Cpp.

\subsubsection{Name mangling}

Имена обычно начинаются с символа \TT{?}.

О \gls{name mangling} в MSVC читайте также здесь: \myref{namemangling}.

\subsection{GCC}
\myindex{GCC}

Кроме компиляторов под *NIX, GCC имеется так же и для win32-окружения: в виде Cygwin и MinGW.

\subsubsection{Name mangling}

Имена обычно начинаются с символов \TT{\_Z}.

О \gls{name mangling} в GCC читайте также здесь: \myref{namemangling}.

\subsubsection{Cygwin}
\myindex{Cygwin}

cygwin1.dll часто импортируется.

\subsubsection{MinGW}
\myindex{MinGW}

msvcrt.dll может импортироваться.

\subsection{Intel Fortran}
\myindex{Фортран}

libifcoremd.dll, libifportmd.dll и libiomp5md.dll (поддержка OpenMP) могут импортироваться.

В libifcoremd.dll много функций с префиксом \TT{for\_}, что значит \IT{Fortran}.

\subsection{Watcom, OpenWatcom}
\myindex{Watcom}
\myindex{OpenWatcom}

\subsubsection{Name mangling}

Имена обычно начинаются с символа \TT{W}.

Например, так кодируется метод \q{method} класса \q{class} не имеющий аргументов и возвращающий \Tvoid{}:

\begin{lstlisting}
W?method$_class$n__v
\end{lstlisting}

\subsection{Borland}
\myindex{Borland Delphi}
\myindex{Borland C++Builder}

Вот пример \gls{name mangling} в Borland Delphi и C++Builder:

\lstinputlisting{digging_into_code/identification/borland_mangling.txt}

Имена всегда начинаются с символа \TT{@} 
затем следует имя класса, имя метода
и закодированные типы аргументов.

Эти имена могут присутствовать с импортах .exe, экспортах .dll, отладочной информации, итд.

Borland Visual Component Libraries (VCL) находятся в файлах .bpl вместо .dll, например, vcl50.dll, rtl60.dll.

Другие DLL которые могут импортироваться: BORLNDMM.DLL.

\subsubsection{Delphi}

Почти все исполняемые файлы имеют текстовую строку \q{Boolean} 
в самом начале сегмента кода, среди остальных имен типов.

Вот очень характерное для Delphi начало сегмента \TT{CODE}, 
этот блок следует сразу за заголовком win32 PE-файла:

\lstinputlisting{digging_into_code/identification/delphi.txt}

Первые 4 байта сегмента данных (\TT{DATA}) в исполняемых файлах могут быть \TT{00 00 00 00}, \TT{32 13 8B C0} или \TT{FF FF FF FF}.
Эта информация может помочь при работе с запакованными/зашифрованными программами на Delphi.

\subsection{Другие известные DLL}

\myindex{OpenMP}
\begin{itemize}
\item vcomp*.dll --- Реализация OpenMP от Microsoft.
\end{itemize}

