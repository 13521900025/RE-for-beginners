% FIXME comparison!
\subsection{Memory \q{snapshots} comparing}
\label{snapshots_comparing}

The technique of the straightforward comparison of two memory snapshots in order to see changes was often used to hack
8-bit computer games and for hacking \q{high score} files.

For example, if you had a loaded game on an 8-bit computer (there isn't much memory on these, but the game usually
consumes even less memory) and you know that you have now, let's say, 100 bullets, you can do a \q{snapshot}
of all memory and back it up to some place. Then shoot once, the bullet count goes to 99, do a second \q{snapshot}
and then compare both: it must be a byte somewhere which has been 100 at the beginning, and now it is 99.

Considering the fact that these 8-bit games were often written in assembly language and such variables were global,
it can be said for sure which address in memory has holding the bullet count. If you searched for all references to the
address in the disassembled game code, it was not very hard to find a piece of code \glslink{decrement}{decrementing} the bullet count,
then to write a \gls{NOP} instruction there, or a couple of \gls{NOP}-s, 
and then have a game with 100 bullets forever.
\myindex{BASIC!POKE}
Games on these 8-bit computers were commonly loaded at the constant
address, also, there were not much different versions of each game (commonly just one version was popular for a long span of time),
so enthusiastic gamers knew which bytes must be overwritten (using the BASIC's instruction \gls{POKE}) at which address in
order to hack it. This led to \q{cheat} lists that contained \gls{POKE} instructions, published in magazines related to
8-bit games. See also: \href{http://go.yurichev.com/17114}{wikipedia}.

\myindex{MS-DOS}

Likewise, it is easy to modify \q{high score} files, this does not work with just 8-bit games. Notice 
your score count and back up the file somewhere. When the \q{high score} count gets different, just compare the two files,
it can even be done with the DOS utility FC\footnote{MS-DOS utility for comparing binary files} (\q{high score} files
are often in binary form).

There will be a point where a couple of bytes are different and it is easy to see which ones are
holding the score number.
However, game developers are fully aware of such tricks and may defend the program against it.

Somewhat similar example in this book is: \myref{Millenium_DOS_game}.

% TODO: пример с какой-то простой игрушкой?

\subsubsection{A real story from 1999}

\myindex{ICQ}
There was a time of ICQ messenger's popularity, at least in ex-USSR countries.
The messenger had a peculiarity --- some users didn't want to share their online status with everyone.
And you had to ask an \emph{authorization} from that user.
That user could allow you seeing his/her status, or maybe not.

This is what the author of these lines did:

\begin{itemize}
\item Added a user.
\item A user appeared in a contact-list, in a ``wait for authorization'' section.
\item Closed ICQ.
\item Backed up the ICQ database.
\item Loaded ICQ again.
\item User \emph{authorized}.
\item Closed ICQ and compared two databases.
\end{itemize}

It turned out: two database differed by only one byte.
In the first version: \verb|RESU\x03|, in the second: \verb|RESU\x02|.
(``RESU'', presumably, means ``USER'', i.e., a header of a structure where all the information about user was stored.)
That means the information about authorization was stored not at the server, but at the client.
Presumably, 2/3 value reflected \emph{authorization} status.

\subsubsection{Windows registry}

It is also possible to compare the Windows registry before and after a program installation.

It is a very popular method of finding which registry elements are used by the program.
Perhaps, this is the reason why the \q{windows registry cleaner} shareware is so popular.

By the way, this is how to dump Windows registry to text files:

\begin{lstlisting}
reg export HKLM HKLM.reg
reg export HKCU HKCU.reg
reg export HKCR HKCR.reg
reg export HKU HKU.reg
reg export HKCC HKCC.reg
\end{lstlisting}

\myindex{UNIX!diff}
They can be compared using diff...

\subsubsection{Blink-comparator}

Comparison of files or memory snapshots remind us blink-comparator
\footnote{\url{http://go.yurichev.com/17348}}:
a device used by astronomers in past, intended to find moving celestial objects.

Blink-comparator allows to switch quickly between two photographies shot in different time,
so astronomer would spot the difference visually.

By the way, Pluto was discovered by blink-comparator in 1930.
