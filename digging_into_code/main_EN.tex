\chapter{Finding important/interesting stuff in the code}

Minimalism it is not a prominent feature of modern software.

\myindex{\Cpp!STL}

But not because the programmers are writing a lot, but because a lot of libraries are commonly linked statically
to executable files.
If all external libraries were shifted into an external DLL files, the world would be different.
(Another reason for C++ are the \ac{STL} and other template libraries.)

\newcommand{\FOOTNOTEBOOST}{\footnote{\url{http://go.yurichev.com/17036}}}
\newcommand{\FOOTNOTELIBPNG}{\footnote{\url{http://go.yurichev.com/17037}}}

Thus, it is very important to determine the origin of a function, if it is from standard library or 
well-known library (like Boost\FOOTNOTEBOOST, libpng\FOOTNOTELIBPNG),
or if it is related to what we are trying to find in the code.

It is just absurd to rewrite all code in \CCpp to find what we're looking for.

One of the primary tasks of a reverse engineer is to find quickly the code he/she needs.

\myindex{\GrepUsage}

The \IDA disassembler allow us to search among text strings, byte sequences and constants.
It is even possible to export the code to .lst or .asm text files and then use \TT{grep}, \TT{awk}, etc.

When you try to understand what some code is doing, this easily could be some open-source library like libpng.
So when you see some constants or text strings which look familiar, it is always worth to \emph{google} them.
And if you find the opensource project where they are used, 
then it's enough just to compare the functions.
It may solve some part of the problem.

For example, if a program uses XML files, the first step may be determining which
XML library is used for processing, since the standard (or well-known) libraries are usually used
instead of self-made one.

\myindex{SAP}
\myindex{Windows!PDB}

For example, the author of these lines once tried to understand how the compression/decompression of network packets works in SAP 6.0. 
It is a huge software, but a detailed .\gls{PDB} with debugging information is present, 
and that is convenient.
He finally came to the idea that one of the functions, that was called \emph{CsDecomprLZC}, was doing the decompression of network packets.
Immediately he tried to google its name and he quickly found the function was used in MaxDB
(it is an open-source SAP project) \footnote{More about it in relevant section~(\myref{sec:SAPGUI})}.

\url{http://www.google.com/search?q=CsDecomprLZC}

Astoundingly, MaxDB and SAP 6.0 software shared likewise code for the compression/decompression of network packets.

\mysection{Identification of executable files}

\subsection{Microsoft Visual C++}
\label{MSVC_versions}

MSVC versions and DLLs that can be imported:

%\small
\begin{center}
\begin{tabular}{ | l | l | l | l | l | }
\hline
\HeaderColor Marketing ver. & 
\HeaderColor Internal ver. & 
\HeaderColor CL.EXE ver. &
\HeaderColor DLLs imported &
\HeaderColor Release date \\
\hline
% 4.0, April 1995
% 97 & 5.0 & February 1997
6		&  6.0	& 12.00	& msvcrt.dll	& June 1998		\\
		&	&	& msvcp60.dll	&			\\
\hline
.NET (2002)	&  7.0	& 13.00	& msvcr70.dll	& February 13, 2002	\\
		&	&	& msvcp70.dll	&			\\
\hline
.NET 2003	&  7.1	& 13.10 & msvcr71.dll	& April 24, 2003	\\
		&	&	& msvcp71.dll	&			\\
\hline
2005		&  8.0	& 14.00 & msvcr80.dll	& November 7, 2005	\\
		&	&	& msvcp80.dll	&			\\
\hline
2008		&  9.0	& 15.00 & msvcr90.dll	& November 19, 2007	\\
		&	&	& msvcp90.dll	&			\\
\hline
2010		& 10.0	& 16.00 & msvcr100.dll	& April 12, 2010 	\\
		&	&	& msvcp100.dll	&			\\
\hline
2012		& 11.0	& 17.00 & msvcr110.dll	& September 12, 2012 	\\
		&	&	& msvcp110.dll	&			\\
\hline
2013		& 12.0	& 18.00 & msvcr120.dll	& October 17, 2013 	\\
		&	&	& msvcp120.dll	&			\\
\hline
\end{tabular}
\end{center}
%\normalsize

msvcp*.dll has \Cpp{}-related functions, so if it is imported, 
this is probably a \Cpp program.

\subsubsection{Name mangling}

The names usually start with the \TT{?} symbol.

You can read more about MSVC's \gls{name mangling} here: \myref{namemangling}.

\subsection{GCC}
\myindex{GCC}

Aside from *NIX targets, GCC is also present in the win32 environment, in the form of Cygwin and MinGW.

\subsubsection{Name mangling}

Names usually start with the \TT{\_Z} symbols.

You can read more about GCC's \gls{name mangling} here: \myref{namemangling}.

\subsubsection{Cygwin}
\myindex{Cygwin}

cygwin1.dll is often imported.

\subsubsection{MinGW}
\myindex{MinGW}

msvcrt.dll may be imported.

\subsection{Intel Fortran}
\myindex{Fortran}

libifcoremd.dll, libifportmd.dll and libiomp5md.dll (OpenMP support) may be imported.

libifcoremd.dll has a lot of functions prefixed with \TT{for\_}, which means \IT{Fortran}.

\subsection{Watcom, OpenWatcom}
\myindex{Watcom}
\myindex{OpenWatcom}

\subsubsection{Name mangling}

Names usually start with the \TT{W} symbol.

For example, that is how the method named \q{method} of the class \q{class} that does not have any arguments and returns
\Tvoid is encoded:

\begin{lstlisting}
W?method$_class$n__v
\end{lstlisting}

\subsection{Borland}
\myindex{Borland Delphi}
\myindex{Borland C++Builder}

Here is an example of Borland Delphi's and C++Builder's \gls{name mangling}:

\lstinputlisting{digging_into_code/identification/borland_mangling.txt}

The names always start with the \TT{@} 
symbol, then we have the class name came, method name, and encoded the types of the arguments of the method.

These names can be in the .exe imports, .dll exports, debug data, etc.

Borland Visual Component Libraries (VCL) 
are stored in .bpl files instead of .dll ones, for example, vcl50.dll, rtl60.dll.

Another DLL that might be imported: BORLNDMM.DLL.

\subsubsection{Delphi}

Almost all Delphi executables has the \q{Boolean} text string at the beginning of the code segment, along with other type names.

This is a very typical beginning of the \TT{CODE} 
segment of a Delphi program, this block came right after the win32 PE file header:

\lstinputlisting{digging_into_code/identification/delphi.txt}

The first 4 bytes of the data segment (\TT{DATA}) can be \TT{00 00 00 00}, \TT{32 13 8B C0} or \TT{FF FF FF FF}.%

This information can be useful when dealing with packed/encrypted Delphi executables.

\subsection{Other known DLLs}

\myindex{OpenMP}
\begin{itemize}
\item vcomp*.dll---Microsoft's implementation of OpenMP.
\end{itemize}



% binary files might be also here

\mysection{Communication with outer world (function level)}
It's often advisable to track function arguments and return values in debugger or \ac{DBI}.
For example, the author once tried to understand meaning of some obscure function, which happens to be incorrectly
implemented bubble sort\footnote{\url{https://yurichev.com/blog/weird_sort/}}.
(It worked correctly, but slower.)
Meanwhile, watching inputs and outputs of this function helps instantly to understand what it does.

Often, when you see division by multiplication (\myref{sec:divisionbymult}),
but forgot all details about its mechanics, you can just observe input
and output and quickly find divisor.

% sections:
\input{digging_into_code/communication_win32_EN}
\mysection{Strings}
\label{sec:digging_strings}

\subsection{Text strings}

\subsubsection{\CCpp}

\label{C_strings}
The normal C strings are zero-terminated (\ac{ASCIIZ}-strings).

The reason why the C string format is as it is (zero-terminated) is apparently historical.
In [Dennis M. Ritchie, \emph{The Evolution of the Unix Time-sharing System}, (1979)]
we read:

\begin{framed}
\begin{quotation}
A minor difference was that the unit of I/O was the word, not the byte, because the PDP-7 was a word-addressed
machine. In practice this meant merely that all programs dealing with character streams ignored null
characters, because null was used to pad a file to an even number of characters.
\end{quotation}
\end{framed}

\myindex{Hiew}

In Hiew or FAR Manager these strings look like this:

\begin{lstlisting}[style=customc]
int main()
{
	printf ("Hello, world!\n");
};
\end{lstlisting}

\begin{figure}[H]
\centering
\includegraphics[width=0.6\textwidth]{digging_into_code/strings/C-string.png}
\caption{Hiew}
\end{figure}

% FIXME видно \n в конце, потом пробел

\subsubsection{Borland Delphi}
\myindex{Pascal}
\myindex{Borland Delphi}

The string in Pascal and Borland Delphi is preceded by an 8-bit or 32-bit string length.

For example:

\begin{lstlisting}[caption=Delphi,style=customasmx86]
CODE:00518AC8                 dd 19h
CODE:00518ACC aLoading___Plea db 'Loading... , please wait.',0

...

CODE:00518AFC                 dd 10h
CODE:00518B00 aPreparingRun__ db 'Preparing run...',0
\end{lstlisting}

\subsubsection{Unicode}

\myindex{Unicode}

Often, what is called Unicode is a methods for encoding strings where each character occupies 2 bytes or 16 bits.
This is a common terminological mistake.
Unicode is a standard for assigning a number to each character in the many writing systems of the 
world, but does not describe the encoding method.

\myindex{UTF-8}
\myindex{UTF-16LE}
The most popular encoding methods are: UTF-8 (is widespread in Internet and *NIX systems) and UTF-16LE (is used in Windows).

\myparagraph{UTF-8}

\myindex{UTF-8}
UTF-8 is one of the most successful methods for
encoding characters.
All Latin symbols are encoded just like in ASCII,
and the symbols beyond the ASCII table are encoded using several bytes.
0 is encoded as
before, so all standard C string functions work with UTF-8 strings just like any other string.

Let's see how the symbols in various languages are encoded in UTF-8 and how it looks like in FAR, using the 437 codepage
\footnote{The example and translations was taken from here: 
\url{http://go.yurichev.com/17304}}:

\begin{figure}[H]
\centering
\includegraphics[width=0.6\textwidth]{digging_into_code/strings/multilang_sampler.png}
\end{figure}

% FIXME: cut it
\begin{figure}[H]
\centering
\myincludegraphics{digging_into_code/strings/multilang_sampler_UTF8.png}
\caption{FAR: UTF-8}
\end{figure}

As you can see, the English language string looks the same as it is in ASCII.

The Hungarian language uses some Latin symbols plus symbols with diacritic marks.

These symbols are encoded using several bytes, these are underscored with red.
It's the same story with the Icelandic and Polish languages.

There is also the \q{Euro} currency symbol at the start, which is encoded with 3 bytes.

The rest of the writing systems here have no connection with Latin.

At least in Russian, Arabic, Hebrew and Hindi we can see some recurring bytes, and that is not surprise:
all symbols from a writing system are usually located in the same Unicode table, so their code begins with
the same numbers.

At the beginning, before the \q{How much?} string we see 3 bytes, which are in fact the \ac{BOM}.
The \ac{BOM} defines the encoding system to be
used.

\myparagraph{UTF-16LE}

\myindex{UTF-16LE}
\myindex{Windows!Win32}
Many win32 functions in Windows have the suffixes \TT{-A} and \TT{-W}.
The first type of functions works
with normal strings, the other with UTF-16LE strings (\emph{wide}).

In the second case, each symbol is usually stored in a 16-bit value of type \emph{short}.

The Latin symbols in UTF-16 strings look in Hiew or FAR like they are interleaved with zero byte:

\begin{lstlisting}[style=customc]
int wmain()
{
	wprintf (L"Hello, world!\n");
};
\end{lstlisting}

\begin{figure}[H]
\centering
\includegraphics[width=0.6\textwidth]{digging_into_code/strings/UTF16-string.png}
\caption{Hiew}
\end{figure}

We can see this often in \gls{Windows NT} system files:

\begin{figure}[H]
\centering
\includegraphics[width=0.6\textwidth]{digging_into_code/strings/ntoskrnl_UTF16.png}
\caption{Hiew}
\end{figure}

\myindex{IDA}
Strings with characters that occupy exactly 2 bytes are called \q{Unicode} in \IDA:

\begin{lstlisting}[style=customasmx86]
.data:0040E000 aHelloWorld:
.data:0040E000                 unicode 0, <Hello, world!>
.data:0040E000                 dw 0Ah, 0
\end{lstlisting}

Here is how the Russian language string is encoded in UTF-16LE:

\begin{figure}[H]
\centering
\includegraphics[width=0.6\textwidth]{digging_into_code/strings/russian_UTF16.png}
\caption{Hiew: UTF-16LE}
\end{figure}

What we can easily spot is that the symbols are interleaved by the diamond character (which has the ASCII code of 4).
Indeed, the Cyrillic symbols are located in the fourth Unicode plane
\footnote{\href{http://go.yurichev.com/17003}{wikipedia}}.
Hence, all Cyrillic symbols in UTF-16LE are located in the \TT{0x400-0x4FF} range.

Let's go back to the example with the string written in multiple languages.
Here is how it looks like in UTF-16LE.

% FIXME: cut it
\begin{figure}[H]
\centering
\myincludegraphics{digging_into_code/strings/multilang_sampler_UTF16.png}
\caption{FAR: UTF-16LE}
\end{figure}

Here we can also see the \ac{BOM} at the beginning.
All Latin characters are interleaved with a zero byte.

Some characters with diacritic marks (Hungarian and Icelandic languages) are also underscored in red.

% subsection:
\subsubsection{Base64}
\myindex{Base64}

The base64 encoding is highly popular for the cases when you have to transfer binary data as a text string.

In essence, this algorithm encodes 3 binary bytes into 4 printable characters:
all 26 Latin letters (both lower and upper case), digits, plus sign (\q{+}) and slash sign (\q{/}),
64 characters in total.

One distinctive feature of base64 strings is that they often (but not always) end with 1 or 2 \gls{padding}
equality symbol(s) (\q{=}), for example:

\begin{lstlisting}
AVjbbVSVfcUMu1xvjaMgjNtueRwBbxnyJw8dpGnLW8ZW8aKG3v4Y0icuQT+qEJAp9lAOuWs=
\end{lstlisting}

\begin{lstlisting}
WVjbbVSVfcUMu1xvjaMgjNtueRwBbxnyJw8dpGnLW8ZW8aKG3v4Y0icuQT+qEJAp9lAOuQ==
\end{lstlisting}

The equality sign (\q{=}) is never encounter in the middle of base64-encoded strings.

Now example of manual encoding.
Let's encode 0x00, 0x11, 0x22, 0x33 hexadecimal bytes into base64 string:

\lstinputlisting{digging_into_code/strings/base64_ex.sh}

Let's put all 4 bytes in binary form, then regroup them into 6-bit groups:

\begin{lstlisting}
|  00  ||  11  ||  22  ||  33  ||      ||      |
00000000000100010010001000110011????????????????
| A  || B  || E  || i  || M  || w  || =  || =  |
\end{lstlisting}

Three first bytes (0x00, 0x11, 0x22) can be encoded into 4 base64 characters (``ABEi''),
but the last one (0x33) --- cannot be,
so it's encoded using two characters (``Mw'') and \gls{padding} symbol (``='')
is added twice to pad the last group to 4 characters.
Hence, length of all correct base64 strings are always divisible by 4.

\myindex{XML}
\myindex{PGP}
Base64 is often used when binary data needs to be stored in XML.
``Armored'' (i.e., in text form) PGP keys and signatures are encoded using base64.

Some people tries to use base64 to obfuscate strings:
\url{http://blog.sec-consult.com/2016/01/deliberately-hidden-backdoor-account-in.html}
\footnote{\url{http://archive.is/nDCas}}.

\myindex{base64scanner}
There are utilities for scanning an arbitrary binary files for base64 strings.
One such utility is base64scanner\footnote{\url{https://github.com/DennisYurichev/base64scanner}}.

\myindex{UseNet}
\myindex{FidoNet}
\myindex{Uuencoding}
\myindex{Phrack}
Another encoding system which was much more popular in UseNet and FidoNet is Uuencoding.
Binary files are still encoded in Uuencode format in Phrack magazine.
It offers mostly the same features, but is different from base64 in the sense that file name
is also stored in header.

\myindex{Tor}
\myindex{base32}
By the way: there is also close sibling to base64: base32, alphabet of which has ~10 digits and ~26 Latin characters.
One well-known usage of it is onion addresses
\footnote{\url{https://trac.torproject.org/projects/tor/wiki/doc/HiddenServiceNames}},
like: \\
\url{http://3g2upl4pq6kufc4m.onion/}.
\ac{URL} can't have mixed-case Latin characters, so apparently, this is why Tor developers used base32.





\subsection{Finding strings in binary}

\epigraph{Actually, the best form of Unix documentation is frequently running the
\textbf{strings} command over a program’s object code. Using \textbf{strings}, you can get
a complete list of the program’s hard-coded file name, environment variables,
undocumented options, obscure error messages, and so forth.}{The Unix-Haters Handbook}

\myindex{UNIX!strings}
The standard UNIX \emph{strings} utility is quick-n-dirty way to see strings in file.
For example, these are some strings from OpenSSH 7.2 sshd executable file:

\lstinputlisting{digging_into_code/sshd_strings.txt}

There are options, error messages, file paths, imported dynamic modules and functions, some other strange strings (keys?)
There is also unreadable noise---x86 code sometimes has chunks consisting of printable ASCII characters, up to ~8 characters.

Of course, OpenSSH is open-source program.
But looking at readable strings inside of some unknown binary is often a first step of analysis.
\myindex{UNIX!grep}

\emph{grep} can be applied as well.

\myindex{Hiew}
\myindex{Sysinternals}
Hiew has the same capability (Alt-F6), as well as Sysinternals ProcessMonitor.

\subsection{Error/debug messages}

Debugging messages are very helpful if present.
In some sense, the debugging messages are reporting
what's going on in the program right now. Often these are \printf-like functions,
which write to log-files, or sometimes do not writing anything but the calls are still present 
since the build is not a debug one but \emph{release} one.
\myindex{\oracle}

If local or global variables are dumped in debug messages, it might be helpful as well 
since it is possible to get at least the variable names.
For example, one of such function in \oracle is \TT{ksdwrt()}.

Meaningful text strings are often helpful.
The \IDA disassembler may show from which function and from which point this specific string is used.
Funny cases sometimes happen\footnote{\href{http://go.yurichev.com/17223}{blog.yurichev.com}}.

The error messages may help us as well.
In \oracle, errors are reported using a group of functions.\\
You can read more about them here: \href{http://go.yurichev.com/17224}{blog.yurichev.com}.

\myindex{Error messages}

It is possible to find quickly which functions report errors and in which conditions.

By the way, this is often the reason for copy-protection systems to inarticulate cryptic error messages 
or just error numbers. No one is happy when the software cracker quickly understand why the copy-protection
is triggered just by the error message.

One example of encrypted error messages is here: \myref{examples_SCO}.

\subsection{Suspicious magic strings}

Some magic strings which are usually used in backdoors look pretty suspicious.

For example, there was a backdoor in the TP-Link WR740 home router\footnote{\url{http://sekurak.pl/tp-link-httptftp-backdoor/}}.
The backdoor can activated using the following URL:\\
\url{http://192.168.0.1/userRpmNatDebugRpm26525557/start_art.html}.\\

Indeed, the \q{userRpmNatDebugRpm26525557} string is present in the firmware.

This string was not googleable until the wide disclosure of information about the backdoor.

You would not find this in any \ac{RFC}.

You would not find any computer science algorithm which uses such strange byte sequences.

And it doesn't look like an error or debugging message.

So it's a good idea to inspect the usage of such weird strings.\\
\\
\myindex{base64}

Sometimes, such strings are encoded using base64.

So it's a good idea to decode them all and to scan them visually, even a glance should be enough.\\
\\
\myindex{Security through obscurity}
More precise, this method of hiding backdoors is called \q{security through obscurity}.


\input{digging_into_code/assert_EN}
\mysection{Constants}

Humans, including programmers, often use round numbers like 10, 100, 1000, 
in real life as well as in the code.

The practicing reverse engineer usually know them well in hexadecimal representation:
10=0xA, 100=0x64, 1000=0x3E8, 10000=0x2710.

The constants \TT{0xAAAAAAAA} (0b10101010101010101010101010101010) and \\
\TT{0x55555555} (0b01010101010101010101010101010101)  are also popular---those
are composed of alternating bits.

That may help to distinguish some signal from a signal where all bits are turned on (0b1111 \dots) or off (0b0000 \dots).
For example, the \TT{0x55AA} constant
is used at least in the boot sector, \ac{MBR}, 
and in the \ac{ROM} of IBM-compatible extension cards.

Some algorithms, especially cryptographical ones use distinct constants, which are easy to find
in code using \IDA.

\myindex{MD5}
\newcommand{\URLMD}{http://go.yurichev.com/17111}

For example, the MD5\footnote{\href{\URLMD}{wikipedia}} algorithm initializes its own internal variables like this:

\begin{verbatim}
var int h0 := 0x67452301
var int h1 := 0xEFCDAB89
var int h2 := 0x98BADCFE
var int h3 := 0x10325476
\end{verbatim}

If you find these four constants used in the code in a row, it is highly probable that this function is related to MD5.

\par Another example are the CRC16/CRC32 algorithms, 
whose calculation algorithms often use precomputed tables like this one:

\begin{lstlisting}[caption=linux/lib/crc16.c,style=customc]
/** CRC table for the CRC-16. The poly is 0x8005 (x^16 + x^15 + x^2 + 1) */
u16 const crc16_table[256] = {
	0x0000, 0xC0C1, 0xC181, 0x0140, 0xC301, 0x03C0, 0x0280, 0xC241,
	0xC601, 0x06C0, 0x0780, 0xC741, 0x0500, 0xC5C1, 0xC481, 0x0440,
	0xCC01, 0x0CC0, 0x0D80, 0xCD41, 0x0F00, 0xCFC1, 0xCE81, 0x0E40,
	...
\end{lstlisting}

See also the precomputed table for CRC32: \myref{sec:CRC32}.

In tableless CRC algorithms well-known polynomials are used, for example, 0xEDB88320 for CRC32.

\subsection{Magic numbers}
\label{magic_numbers}

\newcommand{\FNURLMAGIC}{\footnote{\href{http://go.yurichev.com/17112}{wikipedia}}}

A lot of file formats define a standard file header where a \emph{magic number(s)}\FNURLMAGIC{} is used, single one or even several.

\myindex{MS-DOS}

For example, all Win32 and MS-DOS executables start with the two characters \q{MZ}\footnote{\href{http://go.yurichev.com/17113}{wikipedia}}.

\myindex{MIDI}

At the beginning of a MIDI file the \q{MThd} signature must be present. 
If we have a program which uses MIDI files for something,
it's very likely that it must check the file for validity by checking at least the first 4 bytes.

This could be done like this:
(\emph{buf} points to the beginning of the loaded file in memory)

\begin{lstlisting}[style=customasmx86]
cmp [buf], 0x6468544D ; "MThd"
jnz _error_not_a_MIDI_file
\end{lstlisting}

\myindex{\CStandardLibrary!memcmp()}
\myindex{x86!\Instructions!CMPSB}

\dots or by calling a function for comparing memory blocks like \TT{memcmp()} or any other equivalent code
up to a \TT{CMPSB} (\myref{REPE_CMPSx}) instruction.

When you find such point you already can say where the loading of the MIDI file starts,
also, we could see the location
of the buffer with the contents of the MIDI file, what is used from the buffer, and how.

\subsubsection{Dates}

\myindex{UFS2}
\myindex{FreeBSD}
\myindex{HASP}

Often, one may encounter number like \TT{0x19870116}, which is clearly looks like a date (year 1987, 1th month (January), 16th day).
This may be someone's birthday (a programmer, his/her relative, child), or some other important date.
The date may also be written in a reverse order, like \TT{0x16011987}.
American-style dates are also popular, like \TT{0x01161987}.

Well-known example is \TT{0x19540119} (magic number used in UFS2 superblock structure), which is a birthday of Marshall Kirk McKusick, prominent FreeBSD contributor.

\myindex{Stuxnet}
Stuxnet uses the number ``19790509'' (not as 32-bit number, but as string, though), and this led to speculation
that the malware is connected to Israel
\footnote{This is a date of execution of Habib Elghanian, persian jew.}

Also, numbers like those are very popular in amateur-grade cryptography, for example, excerpt from the \emph{secret function} internals from HASP3 dongle
\footnote{\url{https://web.archive.org/web/20160311231616/http://www.woodmann.com/fravia/bayu3.htm}}:

\begin{lstlisting}[style=customc]
void xor_pwd(void) 
{ 
	int i; 
	
	pwd^=0x09071966;
	for(i=0;i<8;i++) 
	{ 
		al_buf[i]= pwd & 7; pwd = pwd >> 3; 
	} 
};

void emulate_func2(unsigned short seed)
{ 
	int i, j; 
	for(i=0;i<8;i++) 
	{ 
		ch[i] = 0; 
		
		for(j=0;j<8;j++)
		{ 
			seed *= 0x1989; 
			seed += 5; 
			ch[i] |= (tab[(seed>>9)&0x3f]) << (7-j); 
		}
	} 
}
\end{lstlisting}

\subsubsection{DHCP}

This applies to network protocols as well.
For example, the DHCP protocol's network packets contains the so-called \emph{magic cookie}: \TT{0x63538263}.
Any code that generates DHCP packets somewhere must embed this constant into the packet.
If we find it in the code we may find where this happens and, not only that.
Any program which can receive DHCP packet must verify the \emph{magic cookie}, comparing it with the constant.

For example, let's take the dhcpcore.dll file from Windows 7 x64 and search for the constant.
And we can find it, twice:
it seems that the constant is used in two functions with descriptive names\\
\TT{DhcpExtractOptionsForValidation()} and \TT{DhcpExtractFullOptions()}:

\begin{lstlisting}[caption=dhcpcore.dll (Windows 7 x64),style=customasmx86]
.rdata:000007FF6483CBE8 dword_7FF6483CBE8 dd 63538263h          ; DATA XREF: DhcpExtractOptionsForValidation+79
.rdata:000007FF6483CBEC dword_7FF6483CBEC dd 63538263h          ; DATA XREF: DhcpExtractFullOptions+97
\end{lstlisting}

And here are the places where these constants are accessed:

\begin{lstlisting}[caption=dhcpcore.dll (Windows 7 x64),style=customasmx86]
.text:000007FF6480875F  mov     eax, [rsi]
.text:000007FF64808761  cmp     eax, cs:dword_7FF6483CBE8
.text:000007FF64808767  jnz     loc_7FF64817179
\end{lstlisting}

And:

\begin{lstlisting}[caption=dhcpcore.dll (Windows 7 x64),style=customasmx86]
.text:000007FF648082C7  mov     eax, [r12]
.text:000007FF648082CB  cmp     eax, cs:dword_7FF6483CBEC
.text:000007FF648082D1  jnz     loc_7FF648173AF
\end{lstlisting}

\subsection{Specific constants}

Sometimes, there is a specific constant for some type of code.
For example, the author once dug into a code, where number 12 was encountered suspiciously often.
Size of many arrays is 12, or multiple of 12 (24, etc).
As it turned out, that code takes 12-channel audio file at input and process it.

And vice versa: for example, if a program works with text field which has length of 120 bytes,
there has to be a constant 120 or 119 somewhere in the code.
If UTF-16 is used, then $2 \cdot 120$.
If a code works with network packets of fixed size, it's good idea to search for this constant in the code as well.

This is also true for amateur cryptography (license keys, etc).
If encrypted block has size of $n$ bytes, you may want to try to find occurences of this number throughout the code.
Also, if you see a piece of code which is been repeated $n$ times in loop during execution,
this may be encryption/decryption routine.

\subsection{Searching for constants}

It is easy in \IDA: Alt-B or Alt-I.
\myindex{binary grep}
And for searching for a constant in a big pile of files, or for searching in non-executable files,
there is a small utility called \emph{binary grep}\footnote{\BGREPURL}.


\input{digging_into_code/instructions_EN}
\mysection{Suspicious code patterns}

\subsection{XOR instructions}
\myindex{x86!\Instructions!XOR}

Instructions like \TT{XOR op, op} (for example, \TT{XOR EAX, EAX}) 
are usually used for setting the register value
to zero, but if the operands are different, the \q{exclusive or} operation
is executed.

This operation is rare in common programming, but widespread in cryptography,
including amateur one.
It's especially suspicious if the
second operand is a big number.

This may point to encrypting/decrypting, checksum computing, etc.\\
\\

One exception to this observation worth noting is the \q{canary} (\myref{subsec:BO_protection}). 
Its generation and checking are often done using the \XOR instruction. \\
\\
\myindex{AWK}

This AWK script can be used for processing \IDA{} listing (.lst) files:

\lstinputlisting{digging_into_code/awk.sh}

It is also worth noting that this kind of script can also match incorrectly disassembled code 
(\myref{sec:incorrectly_disasmed_code}).

\subsection{Hand-written assembly code}

\myindex{Function prologue}
\myindex{Function epilogue}
\myindex{x86!\Instructions!LOOP}
\myindex{x86!\Instructions!RCL}

Modern compilers do not emit the \TT{LOOP} and \TT{RCL} instructions.
On the other hand, these instructions are well-known to coders who like to code directly in assembly language.
If you spot these, it can be said that there is a high probability that this fragment of code was hand-written.
Such instructions are marked as (M) in the instructions list in this appendix: \myref{sec:x86_instructions}.

\par

Also the function prologue/epilogue are not commonly present in hand-written assembly.
\par

Commonly there is no fixed system for passing arguments to functions in the hand-written code.

\par
Example from the Windows 2003 kernel 
(ntoskrnl.exe file):

\lstinputlisting[style=customasmx86]{digging_into_code/ntoskrnl.lst}

Indeed, if we look in the 
\ac{WRK} v1.2 source code, this code
can be found easily in file \\
\emph{WRK-v1.2\textbackslash{}base\textbackslash{}ntos\textbackslash{}ke\textbackslash{}i386\textbackslash{}cpu.asm}.

\mysection{Using magic numbers while tracing}

Often, our main goal is to understand how the program uses a value that has been either read from file or received via network. 
The manual tracing of a value is often a very labor-intensive task. One of the simplest techniques for this (although not 100\% reliable) 
is to use your own \emph{magic number}.

This resembles X-ray computed tomography is some sense: a radiocontrast agent is injected into the patient's blood,
which is then used to improve the visibility of the patient's internal structure in to the X-rays.
It is well known how the blood of healthy humans
percolates in the kidneys and if the agent is in the blood, it can be easily seen on tomography, how blood is percolating,
and are there any stones or tumors.

We can take a 32-bit number like \TT{0x0badf00d}, or someone's birth date like \TT{0x11101979}
and write this 4-byte number to some point in a file used by the program we investigate.

\myindex{\GrepUsage}
\myindex{tracer}

Then, while tracing this program with \tracer in \emph{code coverage} mode, with the help of \emph{grep}
or just by searching in the text file (of tracing results), we can easily see where the value has been used and how.

Example 
of \emph{grepable} \tracer results in \emph{cc} mode:

\begin{lstlisting}[style=customasmx86]
0x150bf66 (_kziaia+0x14), e=       1 [MOV EBX, [EBP+8]] [EBP+8]=0xf59c934 
0x150bf69 (_kziaia+0x17), e=       1 [MOV EDX, [69AEB08h]] [69AEB08h]=0 
0x150bf6f (_kziaia+0x1d), e=       1 [FS: MOV EAX, [2Ch]] 
0x150bf75 (_kziaia+0x23), e=       1 [MOV ECX, [EAX+EDX*4]] [EAX+EDX*4]=0xf1ac360 
0x150bf78 (_kziaia+0x26), e=       1 [MOV [EBP-4], ECX] ECX=0xf1ac360 
\end{lstlisting}
% TODO: good example!

This can be used for network packets as well.
It is important for the \emph{magic number} to be unique and not to be present in the program's code.

\newcommand{\DOSBOXURL}{\href{http://go.yurichev.com/17222}{blog.yurichev.com}}

\myindex{DosBox}
\myindex{MS-DOS}
Aside of 
the \tracer, DosBox (MS-DOS emulator) in heavydebug mode
is able to write information about all registers' states for each executed instruction of the program to a plain text file\footnote{See also my 
blog post about this DosBox feature: \DOSBOXURL{}}, so this technique may be useful for DOS programs as well.


\input{digging_into_code/loops_EN}
\mysection{\MinesweeperWinXPExampleChapterName}
\label{minesweeper_winxp}
\myindex{Windows!Windows XP}

For those who are not very good at playing Minesweeper, we could try to reveal the hidden mines in the debugger.

\myindex{\CStandardLibrary!rand()}
\myindex{Windows!PDB}

As we know, Minesweeper places mines randomly, so there has to be some kind of random number generator or
a call to the standard \TT{rand()} C-function.

What is really cool about reversing Microsoft products is that there are \gls{PDB} 
file with symbols (function names, \etc{}).
When we load \TT{winmine.exe} into \IDA, it downloads the 
\gls{PDB} file exactly for this 
executable and shows all names.

So here it is, the only call to \TT{rand()} is this function:

\lstinputlisting[style=customasmx86]{examples/minesweeper/tmp1.lst}

\IDA named it so, and it was the name given to it by Minesweeper's developers.

The function is very simple:

\begin{lstlisting}[style=customc]
int Rnd(int limit)
{
    return rand() % limit;
};
\end{lstlisting}

(There is no \q{limit} name in the \gls{PDB} file; we manually named this argument like this.)

So it returns 
a random value from 0 to a specified limit.

\TT{Rnd()} is called only from one place, 
a function called \TT{StartGame()}, 
and as it seems, this is exactly 
the code which place the mines:

\begin{lstlisting}[style=customasmx86]
.text:010036C7                 push    _xBoxMac
.text:010036CD                 call    _Rnd@4          ; Rnd(x)
.text:010036D2                 push    _yBoxMac
.text:010036D8                 mov     esi, eax
.text:010036DA                 inc     esi
.text:010036DB                 call    _Rnd@4          ; Rnd(x)
.text:010036E0                 inc     eax
.text:010036E1                 mov     ecx, eax
.text:010036E3                 shl     ecx, 5          ; ECX=ECX*32
.text:010036E6                 test    _rgBlk[ecx+esi], 80h
.text:010036EE                 jnz     short loc_10036C7
.text:010036F0                 shl     eax, 5          ; EAX=EAX*32
.text:010036F3                 lea     eax, _rgBlk[eax+esi]
.text:010036FA                 or      byte ptr [eax], 80h
.text:010036FD                 dec     _cBombStart
.text:01003703                 jnz     short loc_10036C7
\end{lstlisting}

Minesweeper allows you to set the board size, so the X (xBoxMac) and Y (yBoxMac) of the board are global variables.
They are passed to \TT{Rnd()} and random 
coordinates are generated.
A mine is placed by the \TT{OR} instruction at \TT{0x010036FA}. 
And if it has been placed before 
(it's possible if the pair of \TT{Rnd()} 
generates a coordinates pair which has been already 
generated), 
then \TT{TEST} and \TT{JNZ} at \TT{0x010036E6} 
jumps to the generation routine again.

\TT{cBombStart} is the global variable containing total number of mines. So this is loop.

The width of the array is 32 
(we can conclude this by looking at the \TT{SHL} instruction, which multiplies one of the coordinates by 32).

The size of the \TT{rgBlk} 
global array can be easily determined by the difference 
between the \TT{rgBlk} 
label in the data segment and the next known one. 
It is 0x360 (864):

\begin{lstlisting}[style=customasmx86]
.data:01005340 _rgBlk          db 360h dup(?)          ; DATA XREF: MainWndProc(x,x,x,x)+574
.data:01005340                                         ; DisplayBlk(x,x)+23
.data:010056A0 _Preferences    dd ?                    ; DATA XREF: FixMenus()+2
...
\end{lstlisting}

$864/32=27$.

So the array size is $27*32$?
It is close to what we know: when we try to set board size to $100*100$ in Minesweeper settings, it fallbacks to a board of size $24*30$.
So this is the maximal board size here.
And the array has a fixed size for any board size.

So let's see all this in \olly.
We will ran Minesweeper, attaching \olly to it and now we can see the memory dump at the address of the \TT{rgBlk} array (\TT{0x01005340})
\footnote{All addresses here are for Minesweeper for Windows XP SP3 English. 
They may differ for other service packs.}.

So we got this memory dump of the array:

\lstinputlisting[style=customasmx86]{examples/minesweeper/1.lst}

\olly, like any other hexadecimal editor, shows 16 bytes per line.
So each 32-byte array row occupies exactly 2 lines here.

This is beginner level (9*9 board).

There is some square 
structure can be seen visually (0x10 bytes).

We will click \q{Run} in \olly to unfreeze the Minesweeper process, then we'll clicked randomly at the Minesweeper window 
and trapped into mine, but now all mines are visible:

\begin{figure}[H]
\centering
\myincludegraphicsSmall{examples/minesweeper/1.png}
\caption{Mines}
\label{fig:minesweeper1}
\end{figure}

By comparing the mine places and the dump, we can conclude that 0x10 stands for border, 0x0F---empty block, 0x8F---mine.
Perhaps, 0x10 is just a \emph{sentinel value}.

Now we'll add comments and also enclose all 0x8F bytes into square brackets:

\lstinputlisting[style=customasmx86]{examples/minesweeper/2.lst}

Now we'll remove all \emph{border bytes} (0x10) and what's beyond those:

\lstinputlisting[style=customasmx86]{examples/minesweeper/3.lst}

Yes, these are mines, now it can be clearly seen and compared with the screenshot.

\clearpage
What is interesting is that we can modify the array right in \olly.
We can remove all mines by changing all 0x8F bytes by 0x0F, and here is what we'll get in Minesweeper:

\begin{figure}[H]
\centering
\myincludegraphicsSmall{examples/minesweeper/3.png}
\caption{All mines are removed in debugger}
\label{fig:minesweeper3}
\end{figure}

We can also move all of them to the first line: 

\begin{figure}[H]
\centering
\myincludegraphicsSmall{examples/minesweeper/2.png}
\caption{Mines set in debugger}
\label{fig:minesweeper2}
\end{figure}

Well, the debugger is not very convenient for eavesdropping (which is our goal anyway), so we'll write a small utility
to dump the contents of the board:

\lstinputlisting[style=customc]{examples/minesweeper/minesweeper_cheater.c}

Just set the \ac{PID}
\footnote{PID it can be seen in Task Manager 
(enable it in \q{View $\rightarrow$ Select Columns})} 
and the address of the array (\TT{0x01005340} for Windows XP SP3 English) 
and it will dump it
\footnote{The compiled executable is here: 
\href{http://go.yurichev.com/17165}{beginners.re}}.

It attaches itself to a win32 process by \ac{PID} and just reads process memory at the address.

\subsection{Finding grid automatically}

This is kind of nuisance to set address each time when we run our utility.
Also, various Minesweeper versions may have the array on different address.
Knowing the fact that there is always a border (0x10 bytes), we can just find it in memory:

\lstinputlisting[style=customc]{examples/minesweeper/cheater2_fragment.c}

Full source code: \url{https://github.com/DennisYurichev/RE-for-beginners/blob/master/examples/minesweeper/minesweeper_cheater2.c}.

\subsection{\Exercises}

\begin{itemize}

\item 
Why do the \emph{border bytes} (or \emph{sentinel values}) (0x10) exist in the array?

What they are for if they are not visible in Minesweeper's interface?
How could it work without them?

\item 
As it turns out, there are more values possible (for open blocks, for flagged by user, \etc{}).
Try to find the meaning of each one.

\item 
Modify my utility so it can remove all mines or set them in a fixed pattern that you want in the Minesweeper
process currently running.

\end{itemize}

% FIXME comparison!
\subsection{Memory \q{snapshots} comparing}
\label{snapshots_comparing}

The technique of the straightforward comparison of two memory snapshots in order to see changes was often used to hack
8-bit computer games and for hacking \q{high score} files.

For example, if you had a loaded game on an 8-bit computer (there isn't much memory on these, but the game usually
consumes even less memory) and you know that you have now, let's say, 100 bullets, you can do a \q{snapshot}
of all memory and back it up to some place. Then shoot once, the bullet count goes to 99, do a second \q{snapshot}
and then compare both: it must be a byte somewhere which has been 100 at the beginning, and now it is 99.

Considering the fact that these 8-bit games were often written in assembly language and such variables were global,
it can be said for sure which address in memory has holding the bullet count. If you searched for all references to the
address in the disassembled game code, it was not very hard to find a piece of code \glslink{decrement}{decrementing} the bullet count,
then to write a \gls{NOP} instruction there, or a couple of \gls{NOP}-s, 
and then have a game with 100 bullets forever.
\myindex{BASIC!POKE}
Games on these 8-bit computers were commonly loaded at the constant
address, also, there were not much different versions of each game (commonly just one version was popular for a long span of time),
so enthusiastic gamers knew which bytes must be overwritten (using the BASIC's instruction \gls{POKE}) at which address in
order to hack it. This led to \q{cheat} lists that contained \gls{POKE} instructions, published in magazines related to
8-bit games. See also: \href{http://go.yurichev.com/17114}{wikipedia}.

\myindex{MS-DOS}

Likewise, it is easy to modify \q{high score} files, this does not work with just 8-bit games. Notice 
your score count and back up the file somewhere. When the \q{high score} count gets different, just compare the two files,
it can even be done with the DOS utility FC\footnote{MS-DOS utility for comparing binary files} (\q{high score} files
are often in binary form).

There will be a point where a couple of bytes are different and it is easy to see which ones are
holding the score number.
However, game developers are fully aware of such tricks and may defend the program against it.

Somewhat similar example in this book is: \myref{Millenium_DOS_game}.

% TODO: пример с какой-то простой игрушкой?

\subsubsection{A real story from 1999}

\myindex{ICQ}
There was a time of ICQ messenger's popularity, at least in ex-USSR countries.
The messenger had a peculiarity --- some users didn't want to share their online status with everyone.
And you had to ask an \emph{authorization} from that user.
That user could allow you seeing his/her status, or maybe not.

This is what the author of these lines did:

\begin{itemize}
\item Added a user.
\item A user appeared in a contact-list, in a ``wait for authorization'' section.
\item Closed ICQ.
\item Backed up the ICQ database.
\item Loaded ICQ again.
\item User \emph{authorized}.
\item Closed ICQ and compared two databases.
\end{itemize}

It turned out: two database differed by only one byte.
In the first version: \verb|RESU\x03|, in the second: \verb|RESU\x02|.
(``RESU'', presumably, means ``USER'', i.e., a header of a structure where all the information about user was stored.)
That means the information about authorization was stored not at the server, but at the client.
Presumably, 2/3 value reflected \emph{authorization} status.

\subsubsection{Windows registry}

It is also possible to compare the Windows registry before and after a program installation.

It is a very popular method of finding which registry elements are used by the program.
Perhaps, this is the reason why the \q{windows registry cleaner} shareware is so popular.

By the way, this is how to dump Windows registry to text files:

\begin{lstlisting}
reg export HKLM HKLM.reg
reg export HKCU HKCU.reg
reg export HKCR HKCR.reg
reg export HKU HKU.reg
reg export HKCC HKCC.reg
\end{lstlisting}

\myindex{UNIX!diff}
They can be compared using diff...

\subsubsection{Blink-comparator}

Comparison of files or memory snapshots remind us blink-comparator
\footnote{\url{http://go.yurichev.com/17348}}:
a device used by astronomers in past, intended to find moving celestial objects.

Blink-comparator allows to switch quickly between two photographies shot in different time,
so astronomer would spot the difference visually.

By the way, Pluto was discovered by blink-comparator in 1930.

\input{digging_into_code/ISA_detect_EN}

\mysection{Other things}

\subsection{General idea}

A reverse engineer should try to be in programmer's shoes as often as possible. 
To take his/her viewpoint and ask himself, how would one solve some task the specific case.

\subsection{Order of functions in binary code}

All functions located in a single .c or .cpp-file are compiled into corresponding object (.o) file.
Later, linker puts all object files it needs together, not changing order or functions in them.
As a consequence, if you see two or more consecutive functions, it means, that they were placed together
in a single source code file (unless you're on border of two object files, of course.)
This means these functions have something in common, that they are from the same \ac{API} level, from same library, etc.

\subsection{Tiny functions}

Tiny functions like empty functions (\myref{empty_func})
or function which returns just ``true'' (1) or ``false'' (0) (\myref{ret_val_func}) are very common,
and almost all decent compilers tend to put only one such function into resulting executable code even if there were several
similar functions in source code.
So, whenever you see a tiny function consisting just of \TT{mov eax, 1 / ret}
which is referenced (and can be called) from many places,
which are seems unconnected to each other, this may be a result of such optimization.%

\subsection{\Cpp}

\ac{RTTI}~(\myref{RTTI})-data may be also useful for \Cpp class identification.

