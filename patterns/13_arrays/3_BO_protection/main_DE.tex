\subsection{Schutz vor Buffer Overflows}
\label{subsec:BO_protection}
Es gibt verschiedene Möglichkeiten um sich vor solchen Problemen zu schützen, unabhängig von der Unachtsamkeit des \CCpp
Programmierers. MSVC kennt Optionen wie\footnote{Compilerseitiger Schutz vor Buffer Overflows:
\href{http://go.yurichev.com/17133}{wikipedia.org/wiki/Buffer\_overflow\_protection}}:

\begin{lstlisting}
 /RTCs Stack Frame runtime checking
 /GZ Enable stack checks (/RTCs)
\end{lstlisting}

\myindex{x86!\Instructions!RET}
\myindex{Function prologue}
\myindex{Security cookie}
Eine Methode ist eine Zufallszahl zwischen die lokalen Variablen auf dem Stack am Funktionsprolog zu schreiben und
diesen im Funktionsepilog vor dem Beenden der Funktion zu überprüfen.
Wenn der Wert nicht identisch ist, sollte der letzte \RET Befehl nicht ausgeführt werden, sondern das Programm
angehalten werden. Der Prozess wird anhalten, aber das ist deutlich besser als eine Fernattacke auf Ihren Rechner.
    
\newcommand{\CANARYURL}{\href{http://go.yurichev.com/17134}{wikipedia.org/wiki/Domestic\_canary\#Miner.27s\_canary}}

\myindex{Canary}
Die Zufallszahl wird auch \q{canary} (dt. Kanarienvogel) genannt. Der Begriff stammt von den Kanarienvögeln der
Minenarbeiter\footnote{\CANARYURL}, die früher benutzt wurde, um giftige Gase schnell zu erkennen

Kanarienvögel reagieren sehr sensibel auf Grubengase und werden bei Gefahr sehr nervös oder sterben sogar.

Wenn wir unser einfaches Arraybeispiel in \ac{MSVC} mit Optionen RTC1 und RTCs kompilieren~(\myref{arrays_simple})
finden wir einen Aufruf von \TT{@\_RTC\_CheckStackVars@8}, eine Funktion am Ende der Funktion, die prüft, ob der
\q{canary} korrekt ist.

Schauen wir uns an, wie GCC die Sache handhabt.
Betrachten wir ein Beispiel mit \TT{alloca()}~(\myref{alloca}):

\lstinputlisting[style=customc]{patterns/02_stack/04_alloca/2_1.c}
Ohne zusätzliche Optionen fügt GCC 4.7.3 standardmäßig dem Code einen \q{canary} zum Überprüfen hinzu:

\lstinputlisting[caption=GCC 4.7.3,style=customasmx86]{patterns/13_arrays/3_BO_protection/gcc_canary_DE.asm}

\myindex{x86!\Registers!GS}
Der Zufallswert befindet sich in \TT{gs:20}.
Er wird auf den Stack geschrieben und am Ende der Funktion wird der Wert auf dem Stack mit dem korrekten \q{canary} in
\TT{gs:20} verglichen.
Wenn die Werte ungleich sind, wird die Funktion \TT{\_\_stack\_chk\_fail} aufgerufen und wir erkennen in der Konsole in
etwa das Folgende (Ubuntu 13.04 x86):

\begin{lstlisting}
*** buffer overflow detected ***: ./2_1 terminated
======= Backtrace: =========
/lib/i386-linux-gnu/libc.so.6(__fortify_fail+0x63)[0xb7699bc3]
/lib/i386-linux-gnu/libc.so.6(+0x10593a)[0xb769893a]
/lib/i386-linux-gnu/libc.so.6(+0x105008)[0xb7698008]
/lib/i386-linux-gnu/libc.so.6(_IO_default_xsputn+0x8c)[0xb7606e5c]
/lib/i386-linux-gnu/libc.so.6(_IO_vfprintf+0x165)[0xb75d7a45]
/lib/i386-linux-gnu/libc.so.6(__vsprintf_chk+0xc9)[0xb76980d9]
/lib/i386-linux-gnu/libc.so.6(__sprintf_chk+0x2f)[0xb7697fef]
./2_1[0x8048404]
/lib/i386-linux-gnu/libc.so.6(__libc_start_main+0xf5)[0xb75ac935]
======= Memory map: ========
08048000-08049000 r-xp 00000000 08:01 2097586    /home/dennis/2_1
08049000-0804a000 r--p 00000000 08:01 2097586    /home/dennis/2_1
0804a000-0804b000 rw-p 00001000 08:01 2097586    /home/dennis/2_1
094d1000-094f2000 rw-p 00000000 00:00 0          [heap]
b7560000-b757b000 r-xp 00000000 08:01 1048602    /lib/i386-linux-gnu/libgcc_s.so.1
b757b000-b757c000 r--p 0001a000 08:01 1048602    /lib/i386-linux-gnu/libgcc_s.so.1
b757c000-b757d000 rw-p 0001b000 08:01 1048602    /lib/i386-linux-gnu/libgcc_s.so.1
b7592000-b7593000 rw-p 00000000 00:00 0
b7593000-b7740000 r-xp 00000000 08:01 1050781    /lib/i386-linux-gnu/libc-2.17.so
b7740000-b7742000 r--p 001ad000 08:01 1050781    /lib/i386-linux-gnu/libc-2.17.so
b7742000-b7743000 rw-p 001af000 08:01 1050781    /lib/i386-linux-gnu/libc-2.17.so
b7743000-b7746000 rw-p 00000000 00:00 0
b775a000-b775d000 rw-p 00000000 00:00 0
b775d000-b775e000 r-xp 00000000 00:00 0          [vdso]
b775e000-b777e000 r-xp 00000000 08:01 1050794    /lib/i386-linux-gnu/ld-2.17.so
b777e000-b777f000 r--p 0001f000 08:01 1050794    /lib/i386-linux-gnu/ld-2.17.so
b777f000-b7780000 rw-p 00020000 08:01 1050794    /lib/i386-linux-gnu/ld-2.17.so
bff35000-bff56000 rw-p 00000000 00:00 0          [stack]
Aborted (core dumped)
\end{lstlisting}

\myindex{MS-DOS}
gs ist das sogenannte Segmentregister. Diese Register wurden zu Zeiten von MS-DOS und DOS-Erweiterungen häufig
verwendet. Heute ist sein Zweck ein anderer:
\myindex{TLS}
\myindex{Windows!TIB}
Kurz gesagt, zeigt das \TT{gs} Register in Linux stets auf den \ac{TLS}~(\myref{TLS})--hier werden threadspezifische
Informationen gespeichert. In win32 spielt das \TT{fs} Register übrigens die gleiche Rolle und zeigt stets auf
\ac{TIB}\footnote{\href{http://go.yurichev.com/17104}{wikipedia.org/wiki/Win32\_Thread\_Information\_Block}}.

Mehr Informationen finden sich im Quellcode des Linux Kernels (zumindest in der Version 3.11), in\\
\emph{arch/x86/include/asm/stackprotector.h} wird diese Variable in den Kommentaren beschrieben.

\subsubsection{ARM + \OptimizingKeilVI (\ARMMode)}

\begin{lstlisting}[caption=\OptimizingKeilVI (\ARMMode),style=customasmARM]
02 0C C0 E3          BIC     R0, R0, #0x200
01 09 80 E3          ORR     R0, R0, #0x4000
1E FF 2F E1          BX      LR
\end{lstlisting}

\myindex{ARM!\Instructions!BIC}
Der Befehl \INS{BIC} (\emph{BItwise bit Clear}) dient zum Löschen spezifischer
Bits. Er arbeitet wie ein \AND Befehl mit invertierten Operanden. Er entspricht
also einem \NOT+\AND Befehlspaar.

\myindex{ARM!\Instructions!ORR}
\INS{ORR} bedeutet \q{logical or} und ist analog zu \OR in x86.

So weit, so gut.

\subsubsection{ARM + \OptimizingKeilVI (\ThumbMode)}

\begin{lstlisting}[caption=\OptimizingKeilVI (\ThumbMode),style=customasmARM]
01 21 89 03          MOVS    R1, 0x4000
08 43                ORRS    R0, R1
49 11                ASRS    R1, R1, #5   ; erzeuge 0x200 und speichere in R1
88 43                BICS    R0, R1
70 47                BX      LR
\end{lstlisting}
Es scheint, dass Keil enschieden hat, dass der Code im Thumn mode, der
\TT{0x200} statt \TT{0x4000} verwendet, kompakter ist, als einer, der \TT{0x200}
in ein beliebiges Register schreibt.

% TODO1 пример, как компилятор при помощи сдвигов оптизирует такое: a=0x1000; b=0x2000; c=0x4000, etc

\myindex{ARM!\Instructions!ASRS}
Das ist der Grund dafür, dass dieser Wert mithilfe von \INS{ASRS} (\ASRdesc) als
$\TT{0x4000} \gg 5$ berechnet wird.

\subsubsection{ARM + \OptimizingXcodeIV (\ARMMode)}
\label{anomaly:LLVM}
\myindex{\CompilerAnomaly}

\begin{lstlisting}[caption=\OptimizingXcodeIV (\ARMMode),label=ARM_leaf_example3,style=customasmARM]
42 0C C0 E3          BIC             R0, R0, #0x4200
01 09 80 E3          ORR             R0, R0, #0x4000
1E FF 2F E1          BX              LR
\end{lstlisting}

Der Code, der von LLVM erzeugt wurde, könnte als Quellcode etwa wie folgt
ausgesehen haben:

\begin{lstlisting}[style=customc]
    REMOVE_BIT (rt, 0x4200);
    SET_BIT (rt, 0x4000);
\end{lstlisting}
Er macht genau das, was wir erwarten. Aber woher das \TT{0x4200}? Möglicherweise
handelt es sich um ein Artefakt des Optimierers von LLVM.

\footnote{Hier wurde der LLVM Build 2410.2.00 mit Apple Xcode 4.6.3 gebündelt}.
Möglicherweise also ein Fehler des Optimierers im Compiler; aber der erzeugte
Code funktioniert trotzdem korrekt.

Mehr zu Compiler-Anomalien hier:~(\myref{anomaly:Intel}).

\OptimizingXcodeIV im Thumb mode erzeugt identischen Code.

\subsubsection{ARM: Mehr zum Befehl \INS{BIC}}
\myindex{ARM!\Instructions!BIC}

Überarbeiten wir unser Beispiel ein wenig:

\begin{lstlisting}[style=customc]
int f(int a)
{
    int rt=a;

    REMOVE_BIT (rt, 0x1234);

    return rt;
};
\end{lstlisting}

Jetzt erzeugt der optimierende Keil 5.03 im ARM mode folgenden Code:

\begin{lstlisting}[style=customasmARM]
f PROC
        BIC      r0,r0,#0x1000
        BIC      r0,r0,#0x234
        BX       lr
        ENDP
\end{lstlisting}
Es gibt zwei \INS{BIC} Befehle, d.h. die Bits \TT{0x1234} werden in zwei
Durchgängen gelöscht.

Das liegt daran, dass es nicht möglich ist, \TT{0x1234} in einem einzigen
\INS{BIC} Befehl zu kodieren; deshalb müssen \TT{0x1000} und \TT{0x234} getrennt
werden.

\subsubsection{ARM64: \Optimizing GCC (Linaro) 4.9}

\Optimizing GCCcompiling für ARM64 kann den Befehl \AND anstelle von
\INS{BIC} verwenden:

\begin{lstlisting}[caption=\Optimizing GCC (Linaro) 4.9,style=customasmARM]
f:
	and	w0, w0, -513	; 0xFFFFFFFFFFFFFDFF
	orr	w0, w0, 16384	; 0x4000
	ret
\end{lstlisting}

\subsubsection{ARM64: \NonOptimizing GCC (Linaro) 4.9}

\NonOptimizing GCC erzeugt mehr redundanten Code; dieser funktiniert aber wie
die optimierte Variante:

\begin{lstlisting}[caption=\NonOptimizing GCC (Linaro) 4.9,style=customasmARM]
f:
	sub	sp, sp, #32
	str	w0, [sp,12]
	ldr	w0, [sp,12]
	str	w0, [sp,28]
	ldr	w0, [sp,28]
	orr	w0, w0, 16384	; 0x4000
	str	w0, [sp,28]
	ldr	w0, [sp,28]
	and	w0, w0, -513	; 0xFFFFFFFFFFFFFDFF
	str	w0, [sp,28]
	ldr	w0, [sp,28]
	add	sp, sp, 32
	ret
\end{lstlisting}


