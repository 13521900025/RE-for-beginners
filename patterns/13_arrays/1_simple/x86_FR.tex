\subsubsection{x86}

\myparagraph{MSVC}

Compilons:

\lstinputlisting[caption=MSVC 2008,style=customasmx86]{patterns/13_arrays/1_simple/simple_msvc.asm}

\myindex{x86!\Instructions!SHL}

Rien de très particulier, juste deux boucles: la première est celle de remplissage
et la seconde celle d'affichage.
L'instruction \TT{shl ecx, 1} est utilisée pour la multiplication par 2 de la valeur
dans \ECX, voir à ce sujet ci-après~\myref{SHR}.

80 octets sont alloués sur la pile pour le tableau, 20 éléments de 4 octets.

\clearpage
Essayons cet exemple dans \olly.
\myindex{\olly}

Nous voyons comment le tableau est rempli:

chaque élément est un mot de 32-bit de type \Tint et sa valeur est l'index multiplié
par 2:

\begin{figure}[H]
\centering
\myincludegraphics{patterns/13_arrays/1_simple/olly.png}
\caption{\olly: après remplissage du tableau}
\label{fig:array_simple_olly}
\end{figure}

Puisque le tableau est situé sur la pile, nous pouvons voir ses 20 éléments ici.

\myparagraph{GCC}

Voici ce que GCC 4.4.1 génère:

\lstinputlisting[caption=GCC 4.4.1,style=customasmx86]{patterns/13_arrays/1_simple/simple_gcc.asm}

À propos, la variable $a$ est de type \IT{int*} (un pointeur sur un \Tint{})---vous
pouvez passer un pointeur sur un tableau à une autre fonction, mais c'est plus juste
de dire qu'un pointeur sur le premier élément du tableau est passé (les adresses
du reste des éléments sont calculées de manière évidente).

% TODO: clarifier
Si vous indexez ce pointeur en \IT{a[idx]}, il suffit d'ajouter \IT{idx} au pointeur
et l'élément placé ici (sur lequel pointe le pointeur calculé) est renvoyé.

Un exemple intéressant: une chaîne de caractères comme \IT{\q{string}} est un tableau
de caractères et a un type \IT{const char[]}.

Un index peut aussi être appliqué à ce pointeur.

Et c'est pourquoi il est possible d'écrire des choses comme \TT{\q{string}[i]}---c'est
une expression \CCpp correcte!

