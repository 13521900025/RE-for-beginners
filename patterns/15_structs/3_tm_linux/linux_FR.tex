\subsubsection{Linux}

Prenons pour exemple la structure \TT{tm} dans l'en-tête \TT{time.h} de Linux:

\lstinputlisting[style=customc]{patterns/15_structs/3_tm_linux/GCC_tm.c}

Compilons l'exemple avec GCC 4.4.1:

\lstinputlisting[caption=GCC 4.4.1,style=customasmx86]{patterns/15_structs/3_tm_linux/GCC_tm_FR.asm}

\IDA n'a pas utilisé le nom des variables locales pour identifier les éléments de la pile.
Mais comme nous sommes déjà des rétro ingénieurs expérimentés :-) nous pouvons nous en passer dans
cet exemple simple.

\myindex{x86!\Instructions!LEA}

Notez l'instruction \TT{lea edx, [eax+76Ch]}~---qui incrémente la valeur de \EAX de \TT{0x76C} (1900)
sans modifier aucun des drapeaux. Référez-vous également à la section au sujet de \LEA{}~(\myref{sec:LEA}).

\myparagraph{GDB}

Tentons de charger l'exemple dans GDB \footnote{Le résultat \emph{date} est légèrement modifié pour
les besoins de la démonstration, car il est bien entendu impossible d'exécuter GDB aussi rapidement.}:

\lstinputlisting[caption=GDB]{patterns/15_structs/3_tm_linux/GCC_tm_GDB.txt}

Nous retrouvons facilement notre structure dans la pile. Commençons par observer sa définition dans
\emph{time.h}:

\begin{lstlisting}[caption=time.h, label=struct_tm,style=customc]
struct tm
{
  int	tm_sec;
  int	tm_min;
  int	tm_hour;
  int	tm_mday;
  int	tm_mon;
  int	tm_year;
  int	tm_wday;
  int	tm_yday;
  int	tm_isdst;
};
\end{lstlisting}

Faites attention au fait qu'ici les champs sont des \Tint sur 32 bits et non des WORD comme dans
SYSTEMTIME.

Voici donc les champs de notre structure tels qu'ils sont présents dans la pile:

\begin{lstlisting}
0xbffff0dc:	0x080484c3	0x080485c0	0x000007de	0x00000000
0xbffff0ec:	0x08048301	0x538c93ed	0x00000025 sec	0x0000000a min
0xbffff0fc:	0x00000012 hour	0x00000002 mday	0x00000005 mon 	0x00000072 year
0xbffff10c:	0x00000001 wday	0x00000098 yday	0x00000001 isdst0x00002a30
0xbffff11c:	0x0804b090	0x08048530	0x00000000	0x00000000
\end{lstlisting}

Représentés sous forme tabulaire, cela donne:

\begin{center}
\begin{tabular}{ | l | l | l | }
\hline
\headercolor{} Hexadécimal & 
\headercolor{} Décimal & 
\headercolor{} nom \\
\hline
0x00000025 & 37 	& tm\_sec \\
\hline
0x0000000a & 10 	& tm\_min \\
\hline
0x00000012 & 18 	& tm\_hour \\	
\hline
0x00000002 & 2 		& tm\_mday \\	
\hline
0x00000005 & 5 		& tm\_mon \\	
\hline
0x00000072 & 114 	& tm\_year \\
\hline
0x00000001 & 1 		& tm\_wday \\	
\hline
0x00000098 & 152 	& tm\_yday \\	
\hline
0x00000001 & 1 		& tm\_isdst \\
\hline
\end{tabular}
\end{center}

C'est très similaire à SYSTEMTIME (\myref{sec:SYSTEMTIME}),
Là encore certains champs sont présents qui ne sont pas utilisés tels que tm\_wday, tm\_yday,
tm\_isdst.
