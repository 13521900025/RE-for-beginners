\subsubsection{\WorkingWithFloatAsWithStructSubSubSectionName}
\label{sec:floatasstruct}

FPUについてのセクション(\myref{sec:FPU})ですでに述べたように、
\Tfloat と \Tdouble の両方は、\emph{符号}、
\emph{仮数部}(または\emph{小数部})、\emph{指数部}で構成されます。 
しかし、これらのフィールドを直接編集することはでるでしょうか? \Tfloat で試してみましょう。

\bigskip
% a hack used here! http://tex.stackexchange.com/questions/73524/bytefield-package
\begin{center}
\begin{bytefield}{32}
	\bitheader[endianness=big]{0,22,23,30,31} \\
	\bitbox{1}{S} &
	\bitbox{8}{%
		\RU{экспонента}%
		\EN{exponent}%
		\ES{exponente}%
		\PTBRph{}%
		\DEph{}\PLph{}%
		\ITA{esponente}%
		\FR{exposant}%
		\JPN{指数}
	} &
	\bitbox{23}{%
		\RU{мантисса}%
		\EN{mantissa or fraction}%
		\ES{mantisa o fracci\'on}%
		\PTBRph{}%
		\DEph{}
		\PLph{}%
		\ITA{mantissa}%
		\FR{mantisse ou fraction}%
		\JPN{対数または比}
	}
\end{bytefield}
\end{center}

\begin{center}
( S --- %
	\RU{знак}%
	\EN{sign}%
	\ES{signo}%
	\PTBRph{}%
	\DEph{}\PLph{}%
	\ITA{segno}%
	\FR{signe}%
	\JPN{記号}
)
\end{center}


\lstinputlisting[style=customc]{patterns/15_structs/6_bitfields/float/float_JA.c}

\TT{float\_as\_struct}構造体は、 \Tfloat と同じ量のメモリ、すなわち4バイトまたは32ビットを占有します。

ここで、入力値に負の符号を設定し、指数に2を加えることによって、
整数を\TT{$2^2$}、すなわち4で乗算します。

最適化をオンにしないでMSVC 2008でコンパイルしましょう:

\lstinputlisting[caption=\NonOptimizing MSVC 2008,style=customasmx86]{patterns/15_structs/6_bitfields/float/float_msvc_JA.asm}

少し冗長です。
\Ox フラグを付けてコンパイルした場合、\TT{memcpy()}呼び出しはなく、
\TT{f}変数が直接使用されます。 
しかし、最適化されていないバージョンを見れば分かりやすくなります。

GCC 4.4.1を \Othree つきで実行したらどうなりますか?

\lstinputlisting[caption=\Optimizing GCC 4.4.1,style=customasmx86]{patterns/15_structs/6_bitfields/float/float_gcc_O3_JA.asm}

\ttf 関数はほぼ理解できます。 しかし、興味深いのは、GCCが構造体フィールドを持つこのような問題にかかわらず、
コンパイル時に\TT{f(1.234)}の結果を計算でき、
コンパイル時に事前計算されて \printf にこの引数を用意したことです。
