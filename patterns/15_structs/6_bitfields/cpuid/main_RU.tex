\subsubsection{Пример CPUID}

Язык \CCpp позволяет указывать, сколько именно бит отвести для каждого поля структуры. 
Это удобно если нужно экономить место в памяти. К примеру, для переменной типа \Tbool достаточно одного бита.
Но, это не очень удобно, если нужна скорость.

% FIXME!
% another use of this is to parse binary protocols/packets, for example
% the definition of struct iphdr in include/linux/ip.h

\newcommand{\FNCPUID}{\footnote{\href{http://go.yurichev.com/17069}{wikipedia}}}

\myindex{x86!\Instructions!CPUID}
\label{cpuid}
Рассмотрим пример с инструкцией \CPUID\FNCPUID. 
Эта инструкция возвращает информацию о том, какой процессор имеется в наличии и какие возможности он имеет.

Если перед исполнением инструкции в \EAX будет 1, 
то \CPUID вернет упакованную в \EAX такую информацию о процессоре:

\begin{center}
\begin{tabular}{ | l | l | }
\hline
3:0 (4 бита)& Stepping \\
7:4 (4 бита) & Model \\
11:8 (4 бита) & Family \\
13:12 (2 бита) & Processor Type \\
19:16 (4 бита) & Extended Model \\
27:20 (8 бит) & Extended Family \\
\hline
\end{tabular}
\end{center}

\newcommand{\FNGCCAS}{\footnote{\href{http://go.yurichev.com/17070}
{Подробнее о встроенном ассемблере GCC}}}

MSVC 2010 имеет макрос для \CPUID, а GCC 4.4.1 ~--- нет. 
Поэтому для GCC сделаем эту функцию сами, используя его встроенный ассемблер\FNGCCAS.

\lstinputlisting[style=customc]{patterns/15_structs/6_bitfields/cpuid/CPUID.c}

После того как \CPUID заполнит \EAX/\EBX/\ECX/\EDX, у нас они отразятся в массиве \TT{b[]}. 
Затем, мы имеем указатель на структуру \TT{CPUID\_1\_EAX}, и мы указываем его на значение 
\EAX из массива \TT{b[]}.

Иными словами, мы трактуем 32-битный \Tint как структуру.

Затем мы читаем отдельные биты из структуры.

\myparagraph{MSVC}

Компилируем в MSVC 2008 с опцией \Ox:

\lstinputlisting[caption=\Optimizing MSVC 2008,style=customasmx86]{patterns/15_structs/6_bitfields/cpuid/CPUID_msvc_Ox.asm}

\myindex{x86!\Instructions!SHR}
Инструкция \TT{SHR} сдвигает значение из \EAX на то количество бит, 
которое нужно \emph{пропустить}, то есть, мы игнорируем некоторые биты \emph{справа}.

\myindex{x86!\Instructions!AND}
А инструкция \AND очищает биты \emph{слева} которые нам не нужны, или же, говоря иначе, 
она оставляет по маске только те биты в \EAX, которые нам сейчас нужны.

\clearpage
\subsubsection{MSVC + \olly}
\myindex{\olly}

Попробуем этот же пример в \olly.
Загружаем, нажимаем F8 (\stepover) до тех пор, пока не окажемся в своем исполняемом файле,
а не в \TT{ntdll.dll}.
Прокручиваем вверх до тех пор, пока не найдем \main.
Щелкаем на первой инструкции (\TT{PUSH EBP}), нажимаем F2 (\emph{set a breakpoint}), 
затем F9 (\emph{Run}) и точка останова срабатывает на начале \main.

Трассируем до того места, где готовится адрес переменной $x$:

\begin{figure}[H]
\centering
\myincludegraphics{patterns/04_scanf/1_simple/ex1_olly_1.png}
\caption{\olly: вычисляется адрес локальной переменной}
\label{fig:scanf_ex1_olly_1}
\end{figure}

На \EAX в окне регистров можно нажать правой кнопкой и далее выбрать \q{Follow in stack}.
Этот адрес покажется в окне стека.

Смотрите, это переменная в локальном стеке. Там дорисована красная стрелка.
И там сейчас какой-то мусор (\TT{0x6E494714}).
Адрес этого элемента стека сейчас, при помощи \PUSH запишется в этот же стек рядом.
Трассируем при помощи F8 вплоть до конца исполнения \scanf.
А пока \scanf исполняется, в консольном окне, вводим, например, 123:

\lstinputlisting{patterns/04_scanf/1_simple/console.txt}

\clearpage
Вот тут \scanf отработал:

\begin{figure}[H]
\centering
\myincludegraphics{patterns/04_scanf/1_simple/ex1_olly_3.png}
\caption{\olly: \scanf исполнилась}
\label{fig:scanf_ex1_olly_3}
\end{figure}

\scanf вернул 1 в \EAX, что означает, что он успешно прочитал одно значение.
В наблюдаемом нами элементе стека теперь \TT{0x7B} (123).

\clearpage
Чуть позже это значение копируется из стека в регистр \ECX и передается в \printf{}:

\begin{figure}[H]
\centering
\myincludegraphics{patterns/04_scanf/1_simple/ex1_olly_4.png}
\caption{\olly: готовим значение для передачи в \printf}
\label{fig:scanf_ex1_olly_4}
\end{figure}


\myparagraph{GCC}

Попробуем GCC 4.4.1 с опцией \Othree.

\lstinputlisting[caption=\Optimizing GCC 4.4.1,style=customasmx86]{patterns/15_structs/6_bitfields/cpuid/CPUID_gcc_O3.asm}

Практически, то же самое. Единственное что стоит отметить это то, что GCC решил зачем-то объединить 
вычисление \\
\TT{extended\_model\_id} и \TT{extended\_family\_id} в один блок, 
вместо того чтобы вычислять их перед соответствующим вызовом \printf.

