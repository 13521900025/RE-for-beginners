\subsubsection{CPUIDの例}

\CCpp 言語では、各構造体フィールドの正確なビット数を定義できます。 
メモリ空間を節約する必要がある場合に非常に便利です。 
たとえば、 \Tbool 変数には1ビットで十分です。 
しかし、スピードが重要なら合理的ではありません。
% FIXME!
% another use of this is to parse binary protocols/packets, for example
% the definition of struct iphdr in include/linux/ip.h

\newcommand{\FNCPUID}{\footnote{\href{http://go.yurichev.com/17069}{wikipedia}}}

\myindex{x86!\Instructions!CPUID}
\label{cpuid}

\CPUID\FNCPUID 命令の例を考えてみましょう。
この命令は、現在のCPUとその機能に関する情報を返します。

命令が実行される前に \EAX が1に設定されている場合、
\CPUID は \EAX レジスタに情報がパックされて返ります。

\begin{center}
\begin{tabular}{ | l | l | }
\hline
3:0 (4 bits)& ステッピング \\
7:4 (4 bits) & モデル \\
11:8 (4 bits) & ファミリーy \\
13:12 (2 bits) & プロセッサタイプ \\
19:16 (4 bits) & 拡張モデル \\
27:20 (8 bits) & 拡張ファミリー \\
\hline
\end{tabular}
\end{center}

\newcommand{\FNGCCAS}{\footnote{\href{http://go.yurichev.com/17070}
{GCCアセンブラ内部の詳細}}}

MSVC 2010には \CPUID マクロがありますが、GCC 4.4.1にはありません。 
ですから組み込みアセンブラ \FNGCCAS の助けを借りてGCCのためにこの機能を自分自身で作ってみましょう。

\lstinputlisting[style=customc]{patterns/15_structs/6_bitfields/cpuid/CPUID.c}

\CPUID が \EAX/\EBX/\ECX/\EDX を満たすと、これらのレジスタは\TT{b[]}配列に書き込まれます。 
次に、\TT{CPUID\_1\_EAX}構造体へのポインタを持ち、それを\TT{b[]}配列から \EAX の値に向けます。

つまり、32ビットの \Tint 値を構造体として扱います。 
次に、構造体から特定のビットを読み込みます。

\myparagraph{MSVC}

\Ox オプションを付けてMSVC 2008でコンパイルしてみましょう。

\lstinputlisting[caption=\Optimizing MSVC 2008,style=customasmx86]{patterns/15_structs/6_bitfields/cpuid/CPUID_msvc_Ox.asm}

\myindex{x86!\Instructions!SHR}

\TT{SHR}命令は、 \EAX 内の値を、\emph{スキップ}しなければならないビット数だけシフトします。
例えば、\emph{右側の}ビットを無視します。

\myindex{x86!\Instructions!AND}

\AND 命令は、\emph{左側の}不要ビットをクリアします。言い換えれば、
必要な \EAX レジスタのビットだけを残します。

\clearpage
\myparagraph{\Optimizing MSVC + \olly}
\myindex{\olly}

\olly でこの(最適化された)例を試すことができます。ここに最初のイテレーションがあります:

\begin{figure}[H]
\centering
\myincludegraphics{patterns/10_strings/1_strlen/olly1.png}
\caption{\olly: 最初のイテレーションの開始}
\label{fig:strlen_olly_1}
\end{figure}

\olly がループを見つけ、便宜上、その指示を括弧で\emph{囲んだ}ことがわかります。 
\EAX の右ボタンをクリックして、\q{Follow in Dump}を選択すると、メモリウィンドウが正しい場所にスクロールします。
ここではメモリ内に \q{hello!}という文字列があります。
その後に少なくとも1つのゼロバイトがあり、次にランダムなごみがあります。

\olly が有効なアドレスを持つレジスタを見ると、それは文字列を指しており、
文字列として表示されます。

\clearpage
F8(\stepover)を数回押して、ループの本体の先頭に移動しましょう:

\begin{figure}[H]
\centering
\myincludegraphics{patterns/10_strings/1_strlen/olly2.png}
\caption{\olly: 2回目のイテレーションの開始}
\label{fig:strlen_olly_2}
\end{figure}

\EAX には文字列中の2番目の文字のアドレスが含まれていることがわかります。

\clearpage

我々はループから脱出するためにF8を十分な回数押す必要があります:

\begin{figure}[H]
\centering
\myincludegraphics{patterns/10_strings/1_strlen/olly3.png}
\caption{\olly: 計算すべきポインタの差}
\label{fig:strlen_olly_3}
\end{figure}

% FIXME:
\EAX には文字列の直後に0バイトのアドレスが含まれることがわかりました。一方、
\EDX は変更されていないので、まだ文字列の先頭を指しています。

これらの2つのアドレスの差がここで計算されています。

\clearpage
\SUB 命令が実行されました。

\begin{figure}[H]
\centering
\myincludegraphics{patterns/10_strings/1_strlen/olly4.png}
\caption{\olly: \EAX がデクリメントされる}
\label{fig:strlen_olly_4}
\end{figure}

ポインタの違いは \EAX レジスタにあります(値は7)。
確かに、 \q{hello!}文字列の長さは6ですが、7にはゼロバイトが含まれています。
しかし、\TT{strlen()}は文字列中のゼロ以外の文字数を返さなければなりません。
したがって、デクリメントが実行され、関数が戻ります。


\myparagraph{GCC}

\Othree オプション付きのGCC 4.4.1を試してみましょう。

\lstinputlisting[caption=\Optimizing GCC 4.4.1,style=customasmx86]{patterns/15_structs/6_bitfields/cpuid/CPUID_gcc_O3.asm}

ほとんど同じです。 
唯一注目すべきは、GCCは、 \printf の各呼び出しの前に個別に計算するのではなく、
\TT{extended\_model\_id}と\TT{extended\_family\_id}の計算を
どういうわけか1つのブロックに組み合わせることです。
