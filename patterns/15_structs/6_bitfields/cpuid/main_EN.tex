\subsubsection{CPUID example}

The \CCpp language allows to define the exact number of bits for each structure field.
It is very useful if one needs to save memory space. 
For example, one bit is enough for a \Tbool variable.
But of course, it is not rational if speed is important.
% FIXME!
% another use of this is to parse binary protocols/packets, for example
% the definition of struct iphdr in include/linux/ip.h

\newcommand{\FNCPUID}{\footnote{\href{http://go.yurichev.com/17069}{wikipedia}}}

\myindex{x86!\Instructions!CPUID}
\label{cpuid}

Let's consider the \CPUID\FNCPUID instruction example.
This instruction returns information about the current CPU and its features.

If the \EAX is set to 1 before the instruction's execution, 
\CPUID returning this information packed into the \EAX register:

\begin{center}
\begin{tabular}{ | l | l | }
\hline
3:0 (4 bits)& Stepping \\
7:4 (4 bits) & Model \\
11:8 (4 bits) & Family \\
13:12 (2 bits) & Processor Type \\
19:16 (4 bits) & Extended Model \\
27:20 (8 bits) & Extended Family \\
\hline
\end{tabular}
\end{center}

\newcommand{\FNGCCAS}{\footnote{\href{http://go.yurichev.com/17070}
{More about internal GCC assembler}}}

MSVC 2010 has \CPUID macro, but GCC 4.4.1 does not.
So let's make this function by ourselves for GCC with the help of its built-in assembler\FNGCCAS.

\lstinputlisting[style=customc]{patterns/15_structs/6_bitfields/cpuid/CPUID.c}

After \CPUID fills \EAX/\EBX/\ECX/\EDX, these registers are to be written in the \TT{b[]} array.
Then, we have a pointer to the \TT{CPUID\_1\_EAX} structure and we point it to the value in \EAX from the \TT{b[]} array.

In other words, we treat a 32-bit \Tint value as a structure.
Then we read specific bits from the structure.

\myparagraph{MSVC}

Let's compile it in MSVC 2008 with \Ox option:

\lstinputlisting[caption=\Optimizing MSVC 2008,style=customasmx86]{patterns/15_structs/6_bitfields/cpuid/CPUID_msvc_Ox.asm}

\myindex{x86!\Instructions!SHR}

The \TT{SHR} instruction shifting the value in \EAX by the number of bits that must be
\emph{skipped}, e.g., we ignore some bits \emph{at the right side}.

\myindex{x86!\Instructions!AND}

The \AND instruction clears the unneeded bits \emph{on the left}, or, in other words, 
leaves only those bits in the \EAX register we need.

\clearpage
\subsubsection{MSVC: x86 + \olly}
\myindex{\olly}

Things are even simpler here:

\begin{figure}[H]
\centering
\myincludegraphics{patterns/04_scanf/2_global/ex2_olly_1.png}
\caption{\olly: after \scanf execution}
\label{fig:scanf_ex2_olly_1}
\end{figure}

The variable is located in the data segment.
After the \PUSH instruction (pushing the address of $x$) gets executed, 
the address appears in the stack window. Right-click on that row and select \q{Follow in dump}.
The variable will appear in the memory window on the left.
After we have entered 123 in the console, 
\TT{0x7B} appears in the memory window (see the highlighted screenshot regions).

But why is the first byte \TT{7B}?
Thinking logically, \TT{00 00 00 7B} must be there.
The cause for this is referred as  \gls{endianness}, and x86 uses \emph{little-endian}.
This implies that the lowest byte is written first, and the highest written last.
Read more about it at: \myref{sec:endianness}.
Back to the example, the 32-bit value is loaded from this memory address into \EAX and passed to \printf.

The memory address of $x$ is \TT{0x00C53394}.

\clearpage
In \olly we can review the process memory map (Alt-M)
and we can see that this address is inside the \TT{.data} PE-segment of our program:

\label{olly_memory_map_example}
\begin{figure}[H]
\centering
\myincludegraphics{patterns/04_scanf/2_global/ex2_olly_2.png}
\caption{\olly: process memory map}
\label{fig:scanf_ex2_olly_2}
\end{figure}



\myparagraph{GCC}

Let's try GCC 4.4.1 with \Othree option.

\lstinputlisting[caption=\Optimizing GCC 4.4.1,style=customasmx86]{patterns/15_structs/6_bitfields/cpuid/CPUID_gcc_O3.asm}

Almost the same.
The only thing worth noting is that GCC somehow combines the calculation of\\
\TT{extended\_model\_id} and \TT{extended\_family\_id} into one block,
instead of calculating them separately before each \printf call.
