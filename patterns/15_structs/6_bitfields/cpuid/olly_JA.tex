\clearpage
\myparagraph{MSVC + \olly}
\myindex{\olly}

\olly にサンプルをロードして、
CPUIDの実行後にEAX/EBX/ECX/EDXにどのような値が設定されているかを見てみましょう。

\begin{figure}[H]
\centering
\myincludegraphics{patterns/15_structs/6_bitfields/cpuid/olly.png}
\caption{\olly: CPUID実行後}
\label{fig:cpuid_olly_1}
\end{figure}

EAXは\TT{0x000206A7}(\ac{CPU}はIntel Xeon E3-1220)です。
二進数では $0b0000 0000 0000 0010 0000 0110 1010 0111$ です。

ビットがフィールドにどのように分配されるかは次のとおりです。

\begin{center}
\begin{tabular}{ | l | l | l | }
\hline
\headercolor{} フィールド &
\headercolor{} 二進数 &
\headercolor{} 十進数 \\
\hline
reserved2		& 0000 & 0 \\
\hline
extended\_family\_id	& 00000000 & 0 \\
\hline
extended\_model\_id	& 0010 & 2 \\
\hline
reserved1		& 00 & 0 \\
\hline
processor\_id		& 00 & 0 \\
\hline
family\_id		& 0110 & 6 \\
\hline
model			& 1010 & 10 \\
\hline
stepping		& 0111 & 7 \\
\hline
\end{tabular}
\end{center}

\lstinputlisting[caption=Console output]{patterns/15_structs/6_bitfields/cpuid/console.txt}
