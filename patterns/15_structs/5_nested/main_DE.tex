\subsection{Verschachtelte structs}
Wir betrachten nun Situationen, in denen ein struct innerhalb eines anderen definiert ist.

\lstinputlisting[style=customc]{patterns/15_structs/5_nested/nested.c}
Beide Felder von \TT{inner\_struct} werden hier zwischen den Feldern a,b und d,e von \TT{outer\_struct} platziert.

Kompilieren wir mit (MSVC 2010):

% TBT:
\lstinputlisting[caption=\Optimizing MSVC 2010 /Ob0,style=customasmx86]{patterns/15_structs/5_nested/nested_msvc_EN.asm}

Eine Kuriosität ist hier, dass wir, wenn wir in den Assemblercode schauen, nicht einmal feststellen können, dass hier
ein struct innerhalb eines anderen verwendet wurde! Durch diese Beobachtung können wir sagen, dass verschachtelte
structs in lineare oder eindimensionale structs ausgefaltet werden.

Wenn wir aber die Deklaration von \TT{struct inner\_struct c;} durch \TT{struct inner\_struct *c;} ersetzen (also einen
Pointer verwenden), sieht die Sache ganz anders aus.
% FIXME1: нарисовать вложенную структуру и развернутую

\clearpage
\subsubsection{\olly}
\myindex{\olly}
Laden wir dieses Beispiel in \olly und schauen uns \TT{outer\_struct} im Speicher an:

\begin{figure}[H]
\centering
\myincludegraphics{patterns/15_structs/5_nested/olly.png}
\caption{\olly: vor der Ausführung von \printf}
\label{fig:nested_olly}
\end{figure}
Die Werte werden wie folgt im Speicher abgelegt:
\begin{itemize}
\item \emph{(outer\_struct.a)} (byte) 1 + 3 Byte mit zufälligen Werten;
\item \emph{(outer\_struct.b)} (32-bit word) 2;
\item \emph{(inner\_struct.a)} (32-bit word) 0x64 (100);
\item \emph{(inner\_struct.b)} (32-bit word) 0x65 (101);
\item \emph{(outer\_struct.d)} (byte) 3 + 3 Byte mit zufälligen Werten;
\item \emph{(outer\_struct.e)} (32-bit word) 4.
\end{itemize}

