\subsubsection{x86}

This compiles to:

\lstinputlisting[caption=MSVC 2012 /GS- /Ob0,label=src:struct_packing_4,numbers=left,style=customasmx86]{patterns/15_structs/4_packing/packing_EN.asm}

We pass the structure as a whole, but in fact, as we can see, the structure
is being copied to a temporary one (a place in stack is allocated in line 10 for it,
and then all 4 fields, one by one, are copied in lines 12 \ldots\ 19), 
and then its pointer (address) is to be passed.

The structure is copied because it's not known whether the \ttf{} 
function going to modify the structure or not.
If it gets changed, then the structure in \main has to remain as it has been.

We could use \CCpp pointers, and the resulting code will be almost the same, but without
the copying.

As we can see, each field's address is aligned on a 4-byte boundary.
That's why each \Tchar occupies 4 bytes here (like \Tint). Why?
Because it is easier for the CPU to access memory at aligned addresses and to cache data from it.

However, it is not very economical.

Let's try to compile it with option (\TT{/Zp1}) 
(\IT{/Zp[n] pack structures on n-byte boundary}).

\lstinputlisting[caption=MSVC 2012 /GS- /Zp1,label=src:struct_packing_1,numbers=left,style=customasmx86]{patterns/15_structs/4_packing/packing_msvc_Zp1_EN.asm}

Now the structure takes only 10 bytes and each \Tchar value takes 1 byte. What does it give to us?
Size economy. And as drawback~---the CPU accessing these fields slower than it could.

\label{short_struct_copying_using_MOV}

The structure is also copied in \main. Not field-by-field, but directly 10 bytes, using three pairs of \MOV.
Why not 4?

The compiler decided that it's better to copy 10 bytes using 3 \MOV pairs than to copy two 32-bit words
and two bytes using 4 \MOV pairs.

By the way, such copy implementation using \MOV instead of calling the \TT{memcpy()} function is widely
used, because it's faster than a call to \TT{memcpy()}---for short blocks, of course:
\myref{copying_short_blocks}.

As it can be easily guessed, if the structure is used in many source and object files,
all these must be compiled with the same convention about structures packing.

\newcommand{\FNURLMSDNZP}{\footnote{\href{http://go.yurichev.com/17067}
{MSDN: Working with Packing Structures}}}
\newcommand{\FNURLGCCPC}{\footnote{\href{http://go.yurichev.com/17068}
{Structure-Packing Pragmas}}}

Aside from MSVC \TT{/Zp} option which sets how to align each structure field, there is also
the \TT{\#pragma pack} compiler option, which can be defined right in the source code.
It is available in both MSVC\FNURLMSDNZP and GCC\FNURLGCCPC{}.

Let's get back to the \TT{SYSTEMTIME} structure that consists of 16-bit fields.
How does our compiler know to pack them on 1-byte alignment boundary?

\TT{WinNT.h} file has this:

\begin{lstlisting}[caption=WinNT.h,style=customc]
#include "pshpack1.h"
\end{lstlisting}

And this:

\begin{lstlisting}[caption=WinNT.h,style=customc]
#include "pshpack4.h"                   // 4 byte packing is the default
\end{lstlisting}

The file PshPack1.h looks like:

\lstinputlisting[caption=PshPack1.h,style=customc]{patterns/15_structs/4_packing/tmp1.c}

This tell the compiler how to pack the structures defined after \TT{\#pragma pack}.

\clearpage
\subsubsection{MSVC: x86 + \olly}
\myindex{\olly}

Things are even simpler here:

\begin{figure}[H]
\centering
\myincludegraphics{patterns/04_scanf/2_global/ex2_olly_1.png}
\caption{\olly: after \scanf execution}
\label{fig:scanf_ex2_olly_1}
\end{figure}

The variable is located in the data segment.
After the \PUSH instruction (pushing the address of $x$) gets executed, 
the address appears in the stack window. Right-click on that row and select \q{Follow in dump}.
The variable will appear in the memory window on the left.
After we have entered 123 in the console, 
\TT{0x7B} appears in the memory window (see the highlighted screenshot regions).

But why is the first byte \TT{7B}?
Thinking logically, \TT{00 00 00 7B} must be there.
The cause for this is referred as  \gls{endianness}, and x86 uses \emph{little-endian}.
This implies that the lowest byte is written first, and the highest written last.
Read more about it at: \myref{sec:endianness}.
Back to the example, the 32-bit value is loaded from this memory address into \EAX and passed to \printf.

The memory address of $x$ is \TT{0x00C53394}.

\clearpage
In \olly we can review the process memory map (Alt-M)
and we can see that this address is inside the \TT{.data} PE-segment of our program:

\label{olly_memory_map_example}
\begin{figure}[H]
\centering
\myincludegraphics{patterns/04_scanf/2_global/ex2_olly_2.png}
\caption{\olly: process memory map}
\label{fig:scanf_ex2_olly_2}
\end{figure}


