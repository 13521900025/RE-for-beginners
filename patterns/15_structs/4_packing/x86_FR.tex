\subsubsection{x86}

Le résultat de la compilation est:

\lstinputlisting[caption=MSVC 2012 /GS- /Ob0,label=src:struct_packing_4,numbers=left,style=customasmx86]{patterns/15_structs/4_packing/packing_FR.asm}

Nous passons la structure comme un tout, mais en réalité nous pouvons constater que la structure est copiée 
dans un espace temporaire. De l'espace est réservé pour cela ligne 10 et les 4 champs sont copiées par les 
lignes de 12 \ldots\ 19), puis le pointeur sur l'espace temporaire est passé à la fonction.

La structure est recopiée au cas où la fonction \ttf{} viendrait à en modifier le contenu. Si cela arrive, 
la copie de la structure qui existe dans \main restera inchangée.

Nous pourrions également utiliser des pointeurs \CCpp. Le résulta demeurerait le même, sans qu'il soit 
nécessaire de procéder à la copie.

Nous observons que l'adresse de chaque champ est alignée sur un multiple de 4  octets. C'est pourquoi chaque 
\Tchar occupe 4 octets (de même qu'un \Tint). Pourquoi en est-il ainsi? La réponse se situe au niveau de la 
CPU. Il est plus facile et performant pour elle d'accéder la mémoire et de gérer le cache de données en 
utilisant des adresses alignées.

En revanche ce n'est pas très économique en terme d'espace.

Tentons maintenant une compilation avec l'option (\TT{/Zp1}) (\IT{/Zp[n] indique qu'il faut compresser les 
structures en utilisant des frontières tous les n octets}).

\lstinputlisting[caption=MSVC 2012 /GS- /Zp1,label=src:struct_packing_1,numbers=left,style=customasmx86]{patterns/15_structs/4_packing/packing_msvc_Zp1_FR.asm}

La structure n'occupe plus que 10 octets et chaque valeur de type \Tchar n'occupe plus qu'un octet. Quelles 
sont les conséquences ? Nous économisons de la place au prix d'un accès à ces champs moins rapide que ne 
pourrait le faire la CPU.

\label{short_struct_copying_using_MOV}

La structure est également copiée dans \main. Cette opération ne s'effectue pas champ par champ mais par 
blocs en utilisant trois instructions \MOV. Et pourquoi pas 4 ?

Tout simplement parce que le compilateur a décidé qu'il était préférable d'effectuer la copie en utilisant 
3 paires d'instructions \MOV plutôt que de copier deux mots de 32 bits puis 2 fois un octet ce qui aurait 
nécessité 4 paires d'instructions \MOV.

Ce type d'implémentation de la copie qui repose sur les instructions \MOV plutôt que sur l'appel à la 
fonction \TT{memcpy()} est très répandu. La raison en est que pour de petits blocs, cette approche est 
plus rapide qu'un appel à \TT{memcpy()}: \myref{copying_short_blocks}.

Comme vous pouvez le deviner, si la structure est utilisée dans de nombreux fichiers sources et objets, ils 
doivent tous être compilés avec la même convention de compactage de la structure.

\newcommand{\FNURLMSDNZP}{\footnote{\href{http://go.yurichev.com/17067}
{MSDN: Working with Packing Structures}}}
\newcommand{\FNURLGCCPC}{\footnote{\href{http://go.yurichev.com/17068}
{Structure-Packing Pragmas}}}

Au delà de l'option MSVC \TT{/Zp} qui permet de définir l'alignement des champs des structures, il existe 
également l'option du compilateur \TT{\#pragma pack} qui peut être utilisée directement dans le code source.
Elle est supportée aussi bien par MSVC\FNURLMSDNZP que pars GCC\FNURLGCCPC{}.

Revenons à la structure \TT{SYSTEMTIME} qui contient des champs de 16 bits. Comment notre compilateur sait-il 
les aligner sur des frontières de 1 octet ?

Le fichier \TT{WinNT.h} contient ces instructions:

\begin{lstlisting}[caption=WinNT.h,style=customc]
#include "pshpack1.h"
\end{lstlisting}

et celles-ci:

\begin{lstlisting}[caption=WinNT.h,style=customc]
#include "pshpack4.h"                   // L'alignement sur 4 octets est la valeur par défaut
\end{lstlisting}

Le fichier PshPack1.h ressemble à ceci:

\lstinputlisting[caption=PshPack1.h,style=customc]{patterns/15_structs/4_packing/tmp1.c}

Ces instructions indiquent au compilateur comment compresser les structures définies après \TT{\#pragma pack}.

\clearpage
\myparagraph{MSVC + \olly}
\myindex{\olly}

2 paires de mots 32-bit sont marquées en rouge sur la pile.
Chaque paire est un double au format IEEE 754 et est passée depuis \main.

Nous voyons comment le premier \FLD charge une valeur ($1.2$) depuis la pile et la
stocke dans \ST{0}:

\begin{figure}[H]
\centering
\myincludegraphics{patterns/12_FPU/1_simple/olly1.png}
\caption{\olly: le premier \FLD a été exécuté}
\label{fig:FPU_simple_olly_1}
\end{figure}

À cause des inévitables erreurs de conversion depuis un nombre flottant 64-bit au
format IEEE 754 vers 80-bit (utilisé en interne par le FPU), ici nous voyons 1.199\ldots,
qui est proche de 1.2.

\EIP pointe maintenant sur l'instruction suivante (\FDIV), qui charge un double
(une constante) depuis la mémoire.
Par commodité, \olly affiche sa valeur: 3.14

\clearpage
Continuons l'exécution pas à pas.
\FDIV a été exécuté, maintenant \ST{0} contient 0.382\ldots
(\gls{quotient}):

\begin{figure}[H]
\centering
\myincludegraphics{patterns/12_FPU/1_simple/olly2.png}
\caption{\olly: \FDIV a été exécuté}
\label{fig:FPU_simple_olly_2}
\end{figure}

\clearpage
Troisième étape: le \FLD suivant a été exécuté, chargeant 3.4 dans \ST{0} (ici nous
voyons la valeur approximative 3.39999\ldots):

\begin{figure}[H]
\centering
\myincludegraphics{patterns/12_FPU/1_simple/olly3.png}
\caption{\olly: le second \FLD a été exécuté}
\label{fig:FPU_simple_olly_3}
\end{figure}

En même temps, le \gls{quotient} \emph{est poussé} dans \ST{1}.
Exactement maintenant, \EIP pointe sur la prochaine instruction: \FMUL.
Ceci charge la constante 4.1 depuis la mémoire, ce que montre \olly.

\clearpage
Suivante: \FMUL a été exécutée, donc maintenant le \glslink{product}{produit} est dans \ST{0}:

\begin{figure}[H]
\centering
\myincludegraphics{patterns/12_FPU/1_simple/olly4.png}
\caption{\olly: \FMUL a été exécuté}
\label{fig:FPU_simple_olly_4}
\end{figure}

\clearpage
Suivante: \FADDP a été exécutée, maintenant le résultat de l'addition est dans \ST{0},
et \ST{1} est vidé.

\begin{figure}[H]
\centering
\myincludegraphics{patterns/12_FPU/1_simple/olly5.png}
\caption{\olly: \FADDP a été exécuté}
\label{fig:FPU_simple_olly_5}
\end{figure}

Le résultat est laissé dans \ST{0}, car la fonction renvoie son résultat dans \ST{0}.

\main prend cette valeur depuis le registre plus loin.

Nous voyons quelque chose d'inhabituel: la valeur 13.93\ldots se trouve maintenant
dans \ST{7}.
Pourquoi?

\label{FPU_is_rather_circular_buffer}

Comme nous l'avons lu il y a quelque temps dans ce livre, les registres \ac{FPU} sont
une pile: \myref{FPU_is_stack}.
Mais ceci est une simplification.

Imaginez si cela était implémenté \emph{en hardware} comme cela est décrit, alors
tout le contenu des 7 registres devrait être déplacé (ou copié) dans les registres
adjacents lors d'un push ou d'un pop, et ceci nécessite beaucoup de travail.

En réalité, le \ac{FPU} a seulement 8 registres et un pointeur (appelé \GTT{TOP})
qui contient un numéro de registre, qui est le \q{haut de la pile} courant.

Lorsqu'une valeur est poussée sur la pile, \GTT{TOP} est déplacé sur le registre
disponible suivant, et une valeur est écrite dans ce registre.

La procédure est inversée si la valeur est lue, toutefois, le registre qui a été
libéré n'est pas vidé (il serait possible de le vider, mais ceci nécessite plus de
travail qui peut dégrader les performances).
Donc, c'est ce que nous voyons ici.

On peut dire que \FADDP sauve la somme sur la pile, et y supprime un élément.

Mais en fait, cette instruction sauve la somme et ensuite décale \GTT{TOP}.

Plus précisément, les registres du  \ac{FPU} sont un tampon circulaire.

