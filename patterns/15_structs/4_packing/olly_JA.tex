\clearpage
\myparagraph{\olly フィールドはデフォルトでパックされる}
\myindex{\olly}

\olly で我々の例を(フィールドがデフォルト(4バイト)で整列される)試してみましょう。

\begin{figure}[H]
\centering
\myincludegraphics{patterns/15_structs/4_packing/olly_packing_4.png}
\caption{\olly: \printf が実行される前}
\label{fig:packing_olly_4}
\end{figure}

データウィンドウに4つフィールドが見えます。

しかし、ランダム値(0x30, 0x37, 0x01)はどこから来だのでしょう、最初の(a)と3番目のフィールド(c)の次でしょうか。

私たちの \myref{src:struct_packing_4} のリストを見ると、最初のフィールドと3番目のフィールドが
\Tchar だと分かります。したがって、それぞれ1と3(6行目と8行目)が書かれています。

32ビットワードの残りの3バイトはメモリ内で変更されていません! 
したがって、ランダムなごみが残っています。
\myindex{x86!\Instructions!MOVSX}

このゴミは \printf の出力には何の影響も与えません。その値は、 \MOVSX 命令を使用して準備されるため、
ワードではなくバイトを使用します。
\lstref{src:struct_packing_4} (34行目と38行目)

ちなみに、 \Tchar はMSVCとGCCでデフォルトでは符号ありなので、
\MOVSX (符号拡張)命令がここで使用されます。 
\TT{符号なしのchar}データ型または\TT{uint8\_t}がここで使用された場合、
代わりに \MOVZX 命令が使用されます。

\clearpage
\myparagraph{\olly 1バイト境界でアラインメントされるフィールド}
\myindex{\olly}

ここでははっきりしています。4フィールドは10バイトを占め、値は並べて保存されます。

\begin{figure}[H]
\centering
\myincludegraphics{patterns/15_structs/4_packing/olly_packing_1.png}
\caption{\olly: \printf が実行される前}
\label{fig:packing_olly_1}
\end{figure}
