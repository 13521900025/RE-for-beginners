\subsubsection{x86}

Das Beispiel kompiliert zu folgendem Code:

\lstinputlisting[caption=MSVC 2012 /GS-
/Ob0,label=src:struct_packing_4,numbers=left,style=customasmx86]{patterns/15_structs/4_packing/packing_DE.asm}
Wir übergeben das struct als Ganzes, aber im Code können wir sehen, dass das struct in ein temporäres struct kopiert
wird (ein Platz hierfür wird in Zeile 10 auf dem Stack reserviert), und dass dann alle 4 Felder einzeln (in den Zeilen
12\ldots 19) kopiert werden und anschließend ihr Pointer (Adresse) übergeben wird.

Das struct wird kopiert, da nicht bekannt ist, ob die Funktion \ttf{} das struct verändern wird oder nicht.
Wenn es verändert wird, muss das struct in \main auf dem vorherigen Stand bleiben.

Wir könnten \CCpp Pointer verwenden und der erzeugte Code wäre fast der gleiche nur ohne das Kopieren.

Wie wir sehen können, wird die Adresse von jedem Feld auf einer 4 Byte Grenze angeordnet. Das ist der Grund dafür, dass
jeder \Tchar hier 4 Byte belegt (wie ein \Tint). Es ist für die CPU einfacher auf Speicher an entsprechend angeordneten
Adressen zuzugreifen und Daten von dort im Cache zwischenzuspeichern.

Trotzdem ist dieses Vorgehen nicht besonders ökonomisch.

Komplieren wir mit der Option (\TT{/Zp1})(\emph{/Zp[n] pack structures on n-byte boundary}).

\lstinputlisting[caption=MSVC 2012 /GS-
/Zp1,label=src:struct_packing_1,numbers=left,style=customasmx86]{patterns/15_structs/4_packing/packing_msvc_Zp1_DE.asm}
Das struct benötigt nun lediglich 10 Byte und jeder \Tchar Wert genau 1 Byte. Welchen Vorteil bringt uns das? 
In erster Linie spart man Platz. Ein Nachteil hieran~---die CPU greift hier auf die Felder langsamer zu als es möglich
wäre.

\label{short_struct_copying_using_MOV}
Das struct wird auch in \main kopiert. Nicht Feld für Feld, sondern alle 10 Byte direkt durch Verwendung von drei Paaren
von \MOV Befehlen. Warum aber nicht 4 Paare?

Der Compiler hat entschieden, dass es besser ist, die 10 Byte mit 3 \MOV Befehlspaaren zu kopieren als zwei 32-Bit-Worte
und dann zwei Byte mit insgesamt 4 \MOV Paaren.

Solch eine Implementierung, die zum Kopieren \MOV anstelle eines Aufrufs von \TT{memcpy()} verwendet, ist sehr
gebräuchlich, da es schneller ist als ein Funktionsaufruf von \TT{memcpy()}---zumindest für kurze Blöcke:
\myref{copying_short_blocks}.
Man kann sich leicht überlegen, dass, wenn ein struct in vielen Quellcode und Objektdateien verwendet wird, alle diese
mit der gleichen Konvention bezüglich des Packens in structs kompiliert werden müssen.

\newcommand{\FNURLMSDNZP}{\footnote{\href{http://go.yurichev.com/17067}
{MSDN: Working with Packing Structures}}}
\newcommand{\FNURLGCCPC}{\footnote{\href{http://go.yurichev.com/17068}
{Structure-Packing Pragmas}}}
Neben der Option \TT{Zp} in MSVC, die die Anordnung der Felder von structs festlegt, gibt es auch die
Compileroption \TT{\#pragma pack}, die direkt im Quellcode definiert werden kann.
Sie ist sowohl in MSVC\FNURLMSDNZP als auch in GCC\FNURLGCCPC{} verfügbar.

Gehen wir zurück zum \TT{SYSTEMTIME} struct, das aus 16-Bit-Feldern besteht.
Wie kann unser Compiler wissen, dass diese in 1-Byte-Anordnung gepackt werden müssen?

Die Datei \TT{WinNT.h} enthält dies:

\begin{lstlisting}[caption=WinNT.h,style=customc]
#include "pshpack1.h"
\end{lstlisting}

Und dies:

\begin{lstlisting}[caption=WinNT.h,style=customc]
#include "pshpack4.h"                   // 4 byte packing is the default
\end{lstlisting}

Die Datei PshPack1.h sieht wie folgt aus:

\lstinputlisting[caption=PshPack1.h,style=customc]{patterns/15_structs/4_packing/tmp1.c}

Dies sagt dem Compiler wie die structs, die hinter \TT{\#pragma pack} definiert werden, gepackt werden müssen.

\clearpage
\myparagraph{\Optimizing MSVC + \olly}
\myindex{\olly}

Wir untersuchen das (optimierte) Beispiel in \olly. Hier ist der erste
Durchlauf:

\begin{figure}[H]
\centering
\myincludegraphics{patterns/10_strings/1_strlen/olly1.png}
\caption{\olly: Beginn erster Durchlauf}
\label{fig:strlen_olly_1}
\end{figure}

Wir sehen, dass \olly eine Schleife gefunden hat, und zur Verbesserung der
Lesbarkeit, diese in eckige Klammern \emph{eingeschlossen} hat.
Nach Rechtsklick auf \EAX wählen wir \q{Follow in Dump} und das Speicherfenster
scrollt an die passende Stelle. 
Hier sehen wir den String \q{hello!} im Speicher. 
Dahinter befindet sich mindestens ein Nullbyte und im Anschluss Zufallsbits.

Wenn \olly ein Register mit einer gültigen Adresse, die auf einen String zeigt,
findet, wird dieser String angezeigt.

\clearpage
Wir drücken einige Male F8 (\stepover) um zum Anfang der Schleifenkörpers zu
gelangen:

\begin{figure}[H]
\centering
\myincludegraphics{patterns/10_strings/1_strlen/olly2.png}
\caption{\olly: Beginn zweiter Durchlauf}
\label{fig:strlen_olly_2}
\end{figure}

Wir sehen, dass \EAX nun die Adresse des zweiten Zeichens des Strings enthält.

\clearpage

Durch hinreichend häufiges Drücken von F8 verlassen wir schließlich die
Schleife:

\begin{figure}[H]
\centering
\myincludegraphics{patterns/10_strings/1_strlen/olly3.png}
\caption{\olly: Pointer Differenz wird berechnet}
\label{fig:strlen_olly_3}
\end{figure}

% FIXME:
Wir sehen, dass \EAX jetzt die Adresse des Nullbytes direkt hinter dem String
enthält. In der Zwischenzeit hat sich \EDX nicht verändert, es zeigt also immer
noch auf den Anfang des Strings. 

Die Differenz zwischen den beiden Adressen wird jetzt berechnet. 

\clearpage
Der \SUB Befehl wurde gerade ausgeführt:

\begin{figure}[H]
\centering
\myincludegraphics{patterns/10_strings/1_strlen/olly4.png}
\caption{\olly: \EAX muss dekrementiert werden}
\label{fig:strlen_olly_4}
\end{figure}

Die Differenz der Pointer im \EAX Register beträgt nun--7.
Tatsächlich beträgt die Länge des \q{hello!} Strings 6 Zeichen, aber mit dem
Nullbyte am Ende dazugezählt sind es 7.
Die Funktion \TT{strlen()} soll aber die Anzahl der Nicht-Null-Zeichen im String
zurückliefert, also wird einmal dekrementiert und der Funktionsaufruf
anschließend beendet.

