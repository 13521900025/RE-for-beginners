\subsection{scanf()}

Come gia' detto in precedenza, usare \scanf oggi e' un po' antiquato.
Se proprio dobbiamo, e' necessario almeno controllare se \scanf termina correttamente senza errori.

\lstinputlisting[style=customc]{patterns/04_scanf/3_checking_retval/ex3.c}

Per standard, la funzione \scanf\footnote{scanf, wscanf: \href{http://go.yurichev.com/17255}{MSDN}} restituisce il numero di campi che e' riuscita a leggere con successo.
Nel nostro caso, se tutto va bene e l'utente inserisce un numero, \scanf restituisce 1, oppure 0 (o \ac{EOF}) in caso di errore. 

Aggiungiamo un po' di codice C per controllare che \scanf restituisca un valore e stampi un messaggio in caso di errore.

Funziona come ci si aspetta:

\begin{lstlisting}
C:\...>ex3.exe
Enter X:
123
You entered 123...

C:\...>ex3.exe
Enter X:
ouch
What you entered? Huh?
\end{lstlisting}

% subsections
\EN{\subsubsection{MSVC: x86}

Here is what we get in the assembly output (MSVC 2010):

\lstinputlisting[style=customasmx86]{patterns/04_scanf/3_checking_retval/ex3_MSVC_x86.asm}

\myindex{x86!\Registers!EAX}
The \gls{caller} function (\main) needs the \gls{callee} function (\scanf) result, 
so the \gls{callee} returns it in the \EAX register.

\myindex{x86!\Instructions!CMP}
We check it with the help of the instruction \TT{CMP EAX, 1} (\emph{CoMPare}). In other words, we compare the value in the \EAX register with 1.

\myindex{x86!\Instructions!JNE}
A \JNE conditional jump follows the \CMP instruction. \JNE stands for \emph{Jump if Not Equal}.

So, if the value in the \EAX register is not equal to 1, the \ac{CPU} will pass the execution to the address mentioned in the \JNE operand, in our case \TT{\$LN2@main}.
Passing the control to this address results in the \ac{CPU} executing \printf with the argument \TT{What you entered? Huh?}.
But if everything is fine, the conditional jump is not be taken, and another \printf call is to be executed, with two arguments:\\
\TT{'You entered \%d...'} and the value of \TT{x}.

\myindex{x86!\Instructions!XOR}
\myindex{\CLanguageElements!return}
Since in this case the second \printf has not to be executed, there is a \JMP preceding it (unconditional jump). 
It passes the control to the point after the second \printf and just before the \TT{XOR EAX, EAX} instruction, which implements \TT{return 0}.

% FIXME internal \ref{} to x86 flags instead of wikipedia
\myindex{x86!\Registers!\Flags}
So, it could be said that comparing a value with another is \emph{usually} implemented by \CMP/\Jcc instruction pair, where \emph{cc} is \emph{condition code}.
\CMP compares two values and sets processor flags\footnote{x86 flags, see also: \href{http://go.yurichev.com/17120}{wikipedia}.}.
\Jcc checks those flags and decides to either pass the control to the specified address or not.

\myindex{x86!\Instructions!CMP}
\myindex{x86!\Instructions!SUB}
\myindex{x86!\Instructions!JNE}
\myindex{x86!\Registers!ZF}
\label{CMPandSUB}
This could sound paradoxical, but the \CMP instruction is in fact \SUB (subtract).
All arithmetic instructions set processor flags, not just \CMP.
If we compare 1 and 1, $1-1$ is 0 so the \ZF flag would be set (meaning that the last result is 0).
In no other circumstances \ZF can be set, except when the operands are equal.
\JNE checks only the \ZF flag and jumps only if it is not set.  \JNE is in fact a synonym for \JNZ (\emph{Jump if Not Zero}).
Assembler translates both \JNE and \JNZ instructions into the same opcode.
So, the \CMP instruction can be replaced with a \SUB instruction and almost everything will be fine, with the difference that \SUB alters the value of the first operand.
\CMP is \emph{SUB without saving the result, but affecting flags}.

\subsubsection{MSVC: x86: IDA}

\myindex{IDA}
It is time to run \IDA and try to do something in it.
By the way, for beginners it is good idea to use \TT{/MD} option in MSVC, which means that all these
standard functions are not be linked with the executable file, 
but are to be imported from the \TT{MSVCR*.DLL} file instead.
Thus it will be easier to see which standard function are used and where.

While analyzing code in \IDA, it is very helpful to leave notes for oneself (and others).
In instance, analyzing this example, 
we see that \TT{JNZ} is to be triggered in case of an error.
So it is possible to move the cursor to the label, press \q{n} and rename it to \q{error}.
Create another label---into \q{exit}.
Here is my result:

\lstinputlisting[style=customasmx86]{patterns/04_scanf/3_checking_retval/ex3.lst}

Now it is slightly easier to understand the code.
However, it is not a good idea to comment on every instruction.

% FIXME draw button?
You could also hide(collapse) parts of a function in \IDA.
To do that mark the block, then press \q{--} on the numerical pad and enter the text to be displayed instead.

Let's hide two blocks and give them names:

\lstinputlisting[style=customasmx86]{patterns/04_scanf/3_checking_retval/ex3_2.lst}

% FIXME draw button?
To expand previously collapsed parts of the code, use \q{+} on the numerical pad.

\clearpage
By pressing \q{space}, we can see how \IDA represents a function as a graph:

\begin{figure}[H]
\centering
\myincludegraphics{patterns/04_scanf/3_checking_retval/IDA.png}
\caption{Graph mode in IDA}
\label{fig:ex3_IDA_1}
\end{figure}

There are two arrows after each conditional jump: green and red.
The green arrow points to the block which executes if the jump is triggered, and red if otherwise.

\clearpage
It is possible to fold nodes in this mode and give them names as well (\q{group nodes}).
Let's do it for 3 blocks:

\begin{figure}[H]
\centering
\myincludegraphics{patterns/04_scanf/3_checking_retval/IDA2.png}
\caption{Graph mode in IDA with 3 nodes folded}
\label{fig:ex3_IDA_2}
\end{figure}

That is very useful.
It could be said that a very important part of the reverse engineers' job (and any other researcher as well) is to reduce the amount of information they deal with.

\clearpage
\subsubsection{MSVC: x86 + \olly}
\myindex{\olly}

Things are even simpler here:

\begin{figure}[H]
\centering
\myincludegraphics{patterns/04_scanf/2_global/ex2_olly_1.png}
\caption{\olly: after \scanf execution}
\label{fig:scanf_ex2_olly_1}
\end{figure}

The variable is located in the data segment.
After the \PUSH instruction (pushing the address of $x$) gets executed, 
the address appears in the stack window. Right-click on that row and select \q{Follow in dump}.
The variable will appear in the memory window on the left.
After we have entered 123 in the console, 
\TT{0x7B} appears in the memory window (see the highlighted screenshot regions).

But why is the first byte \TT{7B}?
Thinking logically, \TT{00 00 00 7B} must be there.
The cause for this is referred as  \gls{endianness}, and x86 uses \emph{little-endian}.
This implies that the lowest byte is written first, and the highest written last.
Read more about it at: \myref{sec:endianness}.
Back to the example, the 32-bit value is loaded from this memory address into \EAX and passed to \printf.

The memory address of $x$ is \TT{0x00C53394}.

\clearpage
In \olly we can review the process memory map (Alt-M)
and we can see that this address is inside the \TT{.data} PE-segment of our program:

\label{olly_memory_map_example}
\begin{figure}[H]
\centering
\myincludegraphics{patterns/04_scanf/2_global/ex2_olly_2.png}
\caption{\olly: process memory map}
\label{fig:scanf_ex2_olly_2}
\end{figure}



\clearpage
\subsubsection{MSVC: x86 + Hiew}
\myindex{Hiew}

This can also be used as a simple example of executable file patching.
We may try to patch the executable so the program would always print the input, no matter what we enter.

Assuming that the executable is compiled against external \TT{MSVCR*.DLL} (i.e., with \TT{/MD} option)
\footnote{that's what also called \q{dynamic linking}}, 
we see the \main function at the beginning of the \TT{.text} section.
Let's open the executable in Hiew and find the beginning of the \TT{.text} section (Enter, F8, F6, Enter, Enter).

We can see this:

\begin{figure}[H]
\centering
\myincludegraphics{patterns/04_scanf/3_checking_retval/hiew_1.png}
\caption{Hiew: \main function}
\label{fig:scanf_ex3_hiew_1}
\end{figure}

Hiew finds \ac{ASCIIZ} strings and displays them, as it does with the imported functions' names.

\clearpage
Move the cursor to address \TT{.00401027} (where the \TT{JNZ} instruction, we have to bypass, is located), press F3, and then type \q{9090} (meaning two \ac{NOP}s):

\begin{figure}[H]
\centering
\myincludegraphics{patterns/04_scanf/3_checking_retval/hiew_2.png}
\caption{Hiew: replacing \TT{JNZ} with two \ac{NOP}s}
\label{fig:scanf_ex3_hiew_2}
\end{figure}

Then press F9 (update). Now the executable is saved to the disk. It will behave as we wanted.

Two \ac{NOP}s are probably not the most \ae{}sthetic approach.
Another way to patch this instruction is to write just 0 to the second opcode byte (\gls{jump offset}), 
so that \TT{JNZ} will always jump to the next instruction.

We could also do the opposite: replace first byte with \TT{EB} while not touching the second byte (\gls{jump offset}).
We would get an unconditional jump that is always triggered.
In this case the error message would be printed every time, no matter the input.

}
\RU{\subsubsection{MSVC: x86}

Вот что выходит на ассемблере (MSVC 2010):

\lstinputlisting[style=customasmx86]{patterns/04_scanf/3_checking_retval/ex3_MSVC_x86.asm}

\myindex{x86!\Registers!EAX}
Для того чтобы вызывающая функция имела доступ к результату вызываемой функции, 
вызываемая функция (в нашем случае \scanf) оставляет это значение в регистре \EAX.

\myindex{x86!\Instructions!CMP}
Мы проверяем его инструкцией \TT{CMP EAX, 1} (\emph{CoMPare}), то есть сравниваем значение в \EAX с 1.

\myindex{x86!\Instructions!JNE}
Следующий за инструкцией \CMP: условный переход \JNE. Это означает \emph{Jump if Not Equal}, то есть условный переход \emph{если не равно}.

Итак, если \EAX не равен 1, то \JNE заставит \ac{CPU} перейти по адресу указанном в операнде \JNE, у нас это \TT{\$LN2@main}.
Передав управление по этому адресу, \ac{CPU} начнет исполнять вызов \printf с аргументом \TT{What you entered? Huh?}.
Но если всё нормально, перехода не случится и исполнится другой \printf с двумя аргументами:\\
\TT{'You entered \%d...'} и значением переменной \TT{x}.

\myindex{x86!\Instructions!XOR}
\myindex{\CLanguageElements!return}
Для того чтобы после этого вызова не исполнился сразу второй вызов \printf, 
после него есть инструкция \JMP, безусловный переход, который отправит процессор на место 
после второго \printf и перед инструкцией \TT{XOR EAX, EAX}, которая реализует \TT{return 0}.

% FIXME internal \ref{} to x86 flags instead of wikipedia
\myindex{x86!\Registers!\Flags}
Итак, можно сказать, что в подавляющих случаях сравнение какой-либо переменной с чем-то другим происходит при помощи пары инструкций \CMP и \Jcc, где \emph{cc} это \emph{condition code}.
\CMP сравнивает два значения и выставляет  флаги процессора\footnote{См. также о флагах x86-процессора: \href{http://go.yurichev.com/17120}{wikipedia}.}.
\Jcc проверяет нужные ему флаги и выполняет переход по указанному адресу (или не выполняет).

\myindex{x86!\Instructions!CMP}
\myindex{x86!\Instructions!SUB}
\myindex{x86!\Instructions!JNE}
\myindex{x86!\Registers!ZF}
\label{CMPandSUB}
Но на самом деле, как это не парадоксально поначалу звучит, \CMP это почти то же самое что и инструкция \SUB, которая отнимает числа одно от другого.
Все арифметические инструкции также выставляют флаги в соответствии с результатом, не только \CMP.
Если мы сравним 1 и 1, от единицы отнимется единица, получится 0, и выставится флаг \ZF (\emph{zero flag}), означающий, что последний полученный результат был 0.
Ни при каких других значениях \EAX, флаг \ZF не может быть выставлен, кроме тех, когда операнды равны друг другу.
Инструкция \JNE проверяет только флаг \ZF, и совершает переход только если флаг не поднят. Фактически, \JNE это синоним инструкции \JNZ (\emph{Jump if Not Zero}).
Ассемблер транслирует обе инструкции в один и тот же опкод.
Таким образом, можно \CMP заменить на \SUB и всё будет работать также, но разница в том, что \SUB всё-таки испортит значение в первом операнде.
\CMP это \emph{SUB без сохранения результата, но изменяющая флаги}.

\subsubsection{MSVC: x86: IDA}

\myindex{IDA}
Наверное, уже пора делать первые попытки анализа кода в \IDA.
Кстати, начинающим полезно компилировать в MSVC с ключом \TT{/MD}, что означает, что все эти стандартные
функции не будут скомпонованы с исполняемым файлом, а будут импортироваться из файла \TT{MSVCR*.DLL}.
Так будет легче увидеть, где какая стандартная функция используется.

Анализируя код в \IDA, очень полезно делать пометки для себя (и других).
Например, разбирая этот пример, мы сразу видим, что \TT{JNZ} срабатывает в случае ошибки.
Можно навести курсор на эту метку, нажать \q{n} и переименовать метку в \q{error}.
Ещё одну метку --- в \q{exit}.
Вот как у меня получилось в итоге:

\lstinputlisting[style=customasmx86]{patterns/04_scanf/3_checking_retval/ex3.lst}

Так понимать код становится чуть легче.
Впрочем, меру нужно знать во всем и комментировать каждую инструкцию не стоит.

% FIXME draw button?
В \IDA также можно скрывать части функций: нужно выделить скрываемую часть, нажать \q{--} на цифровой клавиатуре и ввести текст.

Скроем две части и придумаем им названия:

\lstinputlisting[style=customasmx86]{patterns/04_scanf/3_checking_retval/ex3_2.lst}

% FIXME draw button?
Раскрывать скрытые части функций можно при помощи \q{+} на цифровой клавиатуре.

\clearpage
Нажав \q{пробел}, мы увидим, как \IDA может представить функцию в виде графа:

\begin{figure}[H]
\centering
\myincludegraphics{patterns/04_scanf/3_checking_retval/IDA.png}
\caption{Отображение функции в IDA в виде графа}
\label{fig:ex3_IDA_1}
\end{figure}

После каждого условного перехода видны две стрелки: зеленая и красная.
Зеленая ведет к тому блоку, который исполнится если переход сработает, 
а красная~--- если не сработает.

\clearpage
В этом режиме также можно сворачивать узлы и давать им названия (\q{group nodes}).
Сделаем это для трех блоков:

\begin{figure}[H]
\centering
\myincludegraphics{patterns/04_scanf/3_checking_retval/IDA2.png}
\caption{Отображение в IDA в виде графа с тремя свернутыми блоками}
\label{fig:ex3_IDA_2}
\end{figure}

Всё это очень полезно делать.
Вообще, очень важная часть работы реверсера (да и любого исследователя) состоит в том, чтобы уменьшать количество имеющейся информации.

\clearpage
\subsubsection{MSVC + \olly}
\myindex{\olly}

Попробуем этот же пример в \olly.
Загружаем, нажимаем F8 (\stepover) до тех пор, пока не окажемся в своем исполняемом файле,
а не в \TT{ntdll.dll}.
Прокручиваем вверх до тех пор, пока не найдем \main.
Щелкаем на первой инструкции (\TT{PUSH EBP}), нажимаем F2 (\emph{set a breakpoint}), 
затем F9 (\emph{Run}) и точка останова срабатывает на начале \main.

Трассируем до того места, где готовится адрес переменной $x$:

\begin{figure}[H]
\centering
\myincludegraphics{patterns/04_scanf/1_simple/ex1_olly_1.png}
\caption{\olly: вычисляется адрес локальной переменной}
\label{fig:scanf_ex1_olly_1}
\end{figure}

На \EAX в окне регистров можно нажать правой кнопкой и далее выбрать \q{Follow in stack}.
Этот адрес покажется в окне стека.

Смотрите, это переменная в локальном стеке. Там дорисована красная стрелка.
И там сейчас какой-то мусор (\TT{0x6E494714}).
Адрес этого элемента стека сейчас, при помощи \PUSH запишется в этот же стек рядом.
Трассируем при помощи F8 вплоть до конца исполнения \scanf.
А пока \scanf исполняется, в консольном окне, вводим, например, 123:

\lstinputlisting{patterns/04_scanf/1_simple/console.txt}

\clearpage
Вот тут \scanf отработал:

\begin{figure}[H]
\centering
\myincludegraphics{patterns/04_scanf/1_simple/ex1_olly_3.png}
\caption{\olly: \scanf исполнилась}
\label{fig:scanf_ex1_olly_3}
\end{figure}

\scanf вернул 1 в \EAX, что означает, что он успешно прочитал одно значение.
В наблюдаемом нами элементе стека теперь \TT{0x7B} (123).

\clearpage
Чуть позже это значение копируется из стека в регистр \ECX и передается в \printf{}:

\begin{figure}[H]
\centering
\myincludegraphics{patterns/04_scanf/1_simple/ex1_olly_4.png}
\caption{\olly: готовим значение для передачи в \printf}
\label{fig:scanf_ex1_olly_4}
\end{figure}


\clearpage
\subsubsection{MSVC: x86 + Hiew}
\myindex{Hiew}

Это ещё может быть и простым примером исправления исполняемого файла.
Мы можем попробовать исправить его таким образом, что программа всегда будет выводить числа, вне зависимости от ввода.

Исполняемый файл скомпилирован с импортированием функций из
\TT{MSVCR*.DLL} (т.е. с опцией \TT{/MD})\footnote{то, что ещё называют \q{dynamic linking}}, 
поэтому мы можем отыскать функцию \main в самом начале секции \TT{.text}.
Откроем исполняемый файл в Hiew, найдем самое начало секции \TT{.text} (Enter, F8, F6, Enter, Enter).

Мы увидим следующее:

\begin{figure}[H]
\centering
\myincludegraphics{patterns/04_scanf/3_checking_retval/hiew_1.png}
\caption{Hiew: функция \main}
\label{fig:scanf_ex3_hiew_1}
\end{figure}

Hiew находит \ac{ASCIIZ}-строки и показывает их, также как и имена импортируемых функций.

\clearpage
Переведите курсор на адрес \TT{.00401027} (с инструкцией \TT{JNZ}, которую мы хотим заблокировать), нажмите F3, затем наберите \q{9090} (что означает два \ac{NOP}-а):

\begin{figure}[H]
\centering
\myincludegraphics{patterns/04_scanf/3_checking_retval/hiew_2.png}
\caption{Hiew: замена \TT{JNZ} на два \ac{NOP}-а}
\label{fig:scanf_ex3_hiew_2}
\end{figure}

Затем F9 (update). Теперь исполняемый файл записан на диск. Он будет вести себя так, как нам надо.

Два \ac{NOP}-а, возможно, не так эстетично, как могло бы быть.
Другой способ изменить инструкцию это записать 0 во второй байт опкода (смещение перехода),
так что \TT{JNZ} всегда будет переходить на следующую инструкцию.

Можно изменить и наоборот: первый байт заменить на \TT{EB}, второй байт (смещение перехода) не трогать.
Получится всегда срабатывающий безусловный переход.
Теперь сообщение об ошибке будет выдаваться всегда, даже если мы ввели число.

}
\PTBR{\subsubsection{MSVC: x86}

Aqui está o a saída em assembly (MSVC 2010):

\lstinputlisting[style=customasmx86]{patterns/04_scanf/3_checking_retval/ex3_MSVC_x86.asm}

\myindex{x86!\Registers!EAX}
A função que chamou (\main) precisa do resultado da função chamada (\scanf),
então a função chamada retorna esse valor no registrador \EAX.

\myindex{x86!\Instructions!CMP}
Nós verificamos com a ajuda da instrução \TT{CMP EAX, 1} (\emph{CoMParar}). Em outras palavras, comparamos o valor em \EAX com 1.

\myindex{x86!\Instructions!JNE}
O jump condicional \JNE está logo depois da instrução \CMP. \JNE significa \emph{Jump if Not Equal} ou seja, ela desvia se o valor não for igual ao comparado.

Então, se o valor em \EAX não é 1, a \ac{CPU} vai passar a execução para o endereço contido no operando de \JNE, no nosso caso \TT{\$LN2@main}.
Passando a execução para esse endereço resulta na \ac{CPU} executando \printf com o argumento \TT{What you entered? Huh?}.
Mas se tudo estiver correto, o jump condicional não será efetuado e outra chamada do \printf é executada, com dois argumentos: \TT{`You entered \%d...'} e o valor de \TT{x}.

\myindex{x86!\Instructions!XOR}
\myindex{\CLanguageElements!return}
Como nesse caso o segundo \printf() não tem que ser executado, tem um \JMP precedendo ele (jump incondicional).
Ele passa a execução para o ponto depois do segundo \printf e logo antes de \TT{XOR EAX, EAX}, que implementa \TT{return 0}.

% FIXME internal \ref{} to x86 flags instead of wikipedia
\myindex{x86!\Registers!\Flags}
Então, podemos dizer que comparar um valor com outro é geralmente realizado através do par de instruções \CMP/\Jcc, onde \emph{cc} é código condicional.
\CMP compara dois valores e altera os registros da \ac{CPU} (flags)\footnote{\ac{TBT}: x86 flags, see also: \href{http://go.yurichev.com/17120}{wikipedia}.}.
\Jcc checa esses registro e decide passar a execução para o endereço específico contido no operando ou não.

\myindex{x86!\Instructions!CMP}
\myindex{x86!\Instructions!SUB}
\myindex{x86!\Instructions!JNE}
\myindex{x86!\Registers!ZF}
\label{CMPandSUB}
Isso pode parecer meio paradoxal, mas a instrução \CMP é na verdade \SUB (subtrair).
Todo o conjunto de instruções aritiméticas alteram os registros da \ac{CPU}, não só \CMP.
Se compararmos 1 e 1, $1-1$ é 0 então \ZF (zero flag) será acionado (significando que o último resultado foi zero).
Em nenhuma outra circunstância \ZF pode ser acionado, exceto quando os operandos forem iguais.
\JNE verifica somente o ZF e desvia só não estiver acionado.
\JNE é na verdade um sinônimo para \JNZ (jump se não zero).
\JNE e \JNZ são traduzidos no mesmo código de operação.
Então, a instrução CMP pode ser substituida com a instrução \SUB e quase tudo estará certo, com a diferença de que \SUB altera o valor do primeiro operando.
\CMP é \SUB sem salvar o resultado, mas afetando os registros da \ac{CPU}.

\subsubsection{MSVC: x86: IDA}

\PTBRph{}

% TODO translate: \input{patterns/04_scanf/3_checking_retval/olly_PTBR.tex}

\clearpage
\subsubsection{MSVC: x86 + Hiew}
\myindex{Hiew}

Esse exemplo também pode ser usado como uma maneira simples de exemplificar o patch de arquivos executáveis.
Nós podemos tentar rearranjar o executável de forma que o programa sempre imprima a saída, não importando o que inserirmos.

Assumindo que o executavel está compilado com a opção \TT{/MD}\footnote{isso também é chamada ``linkagem dinâmica''}
(\TT{MSVCR*.DLL}), nós vemos a função main no começo da seção \TT{.text}.
Vamos abrir o executável no Hiew e procurar o começo da seção \TT{.text} (Enter, F8, F6, Enter, Enter).

Nós chegamos a isso:

\begin{figure}[H]
\centering
\myincludegraphics{patterns/04_scanf/3_checking_retval/hiew_1.png}
\caption{\PTBRph{}}
\label{fig:scanf_ex3_hiew_1}
\end{figure}

Hiew encontra strings em \ac{ASCIIZ} e as exibe, como faz com os nomes de funções importadas.

\clearpage
Mova o cursor para o endereço \TT{.00401027} (onde a instrução \TT{JNZ}, que temos de evitar, está localizada), aperte F3 e então digite \q{9090} (que significa dois \ac{NOP}s):

\begin{figure}[H]
\centering
\myincludegraphics{patterns/04_scanf/3_checking_retval/hiew_2.png}
\caption{PTBRph{}}
\label{fig:scanf_ex3_hiew_2}
\end{figure}

Então aperte F9 (atualizar). Agora o executável está salvo no disco. Ele executará da maneira que nós desejávamos.

Duas instruções \ac{NOP} não é a abordagem mais estética.
Outra maneira de rearranjar essa instrução é somente escrever um 0 no operando da instrução jump,
então \INS{JNZ} só avançará para a próxima instrução.

Nós poderíamos também ter feito o oposto: mudado o primeiro byte com \TT{EB} e deixa o segundo byte como está.
Nós teriamos um jump incondicional que é sempre deviado.
Nesse caso, a mensagem de erro seria mostrada todas as vezes, não importando a entrada.

}
\IT{\subsubsection{MSVC: x86}

Il seguente e' l'output assembly ottenuto con MSVC 2010:

\lstinputlisting[style=customasmx86]{patterns/04_scanf/3_checking_retval/ex3_MSVC_x86.asm}

\myindex{x86!\Registers!EAX}
La funzione chiamante (\gls{caller}) \main necessita di ottenere il risultato della funzione chiamata (\gls{callee}), 
e pertanto quest'ultima lo restituisce nel registro \EAX register.

\myindex{x86!\Instructions!CMP}
Il controllo viene eseguito con l'aiuto dell'istruzione \TT{CMP EAX, 1} (\emph{CoMPare}). In altre parole, confrontiamo il valore nel registro \EAX con 1.

\myindex{x86!\Instructions!JNE}
U jump condizionale \JNE segue l'istruzione \CMP. \JNE sta per \emph{Jump if Not Equal}.

Quindi, se il valore nel registro \EAX non e' uguale a 1, la \ac{CPU} passera' l'esecuzione all'indirizzo specificato nell'operando di \JNE, nel nostro caso \TT{\$LN2@main}.
Passare il controllo a questo indirizzo risulta nel fatto che la \ac{CPU} eseguira' la funzione \printf con l'argomento \TT{What you entered? Huh?}.
Ma se tutto va bene, il salto condizionale non viene effettuato, e viene eseguita un'altra chiamata a \printf con due argomenti: \TT{'You entered \%d...'} e il valore di \TT{x}.

\myindex{x86!\Instructions!XOR}
\myindex{\CLanguageElements!return}
Poiche' in questo caso la saconda \printf non deve essere eseguita, c'e' un jump non condizionale (unconditional jump) \JMP che la precede. 
Questo passa il controllo al punto dopo la seconda \printf e prima dell'istruzione \TT{XOR EAX, EAX}, che implementa \TT{return 0}.

% FIXME internal \ref{} to x86 flags instead of wikipedia
\myindex{x86!\Registers!\Flags}
Possiamo quindi dire che il confronto di valori e' \emph{solitamente} implementato con una coppia di istruzioni \CMP/\Jcc, dove \emph{cc} e' un \emph{condition code}.
\CMP confronta due valori e imposta i flag del processore \footnote{x86 flags, vedere anche: \href{http://go.yurichev.com/17120}{wikipedia}.}.
\Jcc controlla questi flag e decide se passare o meno il controllo all'indirizzo specificato.

\myindex{x86!\Instructions!CMP}
\myindex{x86!\Instructions!SUB}
\myindex{x86!\Instructions!JNE}
\myindex{x86!\Registers!ZF}
\label{CMPandSUB}
Puo' sembrare un paradosso, ma l'istruzione \CMP e' in effetti una \SUB (subtract).
Tutte le istruzioni aritmetiche settano i flag del processore, non solo \CMP.
Se confrontiamo 1 e 1, $1-1$ e' 0 e quindi il flag \ZF sarebbe impostato a 1 (significando che l'ultimo risultato era 0).
In nessun'altra circostanza il flag \ZF puo' essere impostato, eccetto il caso in cui gli operandi sono uguali.
\JNE controlla soltanto il flag \ZF e salta se e solo se il flag non e' settato.  \JNE e' infatti un sinonimo di \JNZ (\emph{Jump if Not Zero}).
L'assembler traduce entrambe le istruzioni \JNE e \JNZ nello stesso opcode.
Quindi l'istruzione \CMP puo' essere sostituita dall'istruzione \SUB e quasi tutto funzionera', con la differenza che \SUB altera il valore del primo operando.
\CMP e' uguale a \emph{SUB senza salvare il risultato, ma settando i flag}.

\subsubsection{MSVC: x86: IDA}

\myindex{IDA}
E' arrivato il momento di avviare \IDA. A proposito, per i principianti e' buona norma usare l'opzione \TT{/MD} in MSVC, che significa che tutte le funzioni
standard non saranno linkate dentro il file eseguibile, ma importate dal file \TT{MSVCR*.DLL}.
In questo modo sara' piu' facile vedere quali funzioni standard sono usate, e dove.

Quando si analizza il codice con \IDA, e' sempre molto utile lasciare note per se stessi (e per gli altri, nel caso in cui si lavori in gruppo).
Per esempio, analizzando questo esempio, notiamo che 
\TT{JNZ} sara' innescato in caso di errore.
E' possibile muovere il cursore fino alla label, premere \q{n} e rinominarla in \q{errore}.
Creare un'altra label ---in \q{exit}.
Ecco il mio risultato:

\lstinputlisting[style=customasmx86]{patterns/04_scanf/3_checking_retval/ex3.lst}

Adesso e' leggermente piu' facile capire il codice.
Non e' comunque una buona idea commentare ogni istruzione!

% FIXME draw button?
Si possono anche nascondere (collapse) parti di una funzione in \IDA.
Per farlo, selezionare il blocco e premere \q{--} sul tastierino numerico, inserendo il testo da visualizzare al posto del blocco di codice.

Nascondiamo due blocchi e diamogli un nome:

\lstinputlisting[style=customasmx86]{patterns/04_scanf/3_checking_retval/ex3_2.lst}

% FIXME draw button?
Per espandere dei blocchi nascosti, premere \q{+} sul tastierino numerico.

\clearpage
Premendo \q{spazio}, possiamo vedere come \IDA rappresenta una funzione in forma di grafo:

\begin{figure}[H]
\centering
\myincludegraphics{patterns/04_scanf/3_checking_retval/IDA.png}
\caption{Graph mode in IDA}
\label{fig:ex3_IDA_1}
\end{figure}

Ci sono due frecce dopo ogni jump condizionale: verde e rossa.
La freccia verde punta al blocco che viene eseguito se il jump e' innescato, la rossa nel caso opposto.

\clearpage
Anche in questa modalita' e' possibile "chiudere" i nodi e dargli un'etichetta (\q{group nodes}).
Facciamolo per 3 blocchi:

\begin{figure}[H]
\centering
\myincludegraphics{patterns/04_scanf/3_checking_retval/IDA2.png}
\caption{Graph mode in IDA with 3 nodes folded}
\label{fig:ex3_IDA_2}
\end{figure}

Come si puo' vedere questa funzione e' molto utile.
Si puo' dire che una buona parte del lavoro di un reverse engineer (cosi' come di altri tipi di ricercatori) e' rappresentata dalla riduzione della quantita' di informazioni da trattare.

\clearpage
\subsubsection{MSVC: x86 + \olly}

Proviamo ad hackerare il nostro programma in \olly, forzandolo a pensare che \scanf funzioni sempre senza errori.
Quando l'indirizzo di una variabile locale e' passato a \scanf, la variabile inizialmente contiene un valore random inutile, in questo caso \TT{0x6E494714}:

\begin{figure}[H]
\centering
\myincludegraphics{patterns/04_scanf/3_checking_retval/olly_1.png}
\caption{\olly: passing variable address into \scanf}
\label{fig:scanf_ex3_olly_1}
\end{figure}

\clearpage
Quando \scanf viene eseguita, immettiamo nella console qualcosa di diverso da un numero, come \q{asdasd}.
\scanf finisce con 0 in \EAX, indicante che un errore si e' verificato:

\begin{figure}[H]
\centering
\myincludegraphics{patterns/04_scanf/3_checking_retval/olly_2.png}
\caption{\olly: \scanf returning error}
\label{fig:scanf_ex3_olly_2}
\end{figure}

Possiamo anche controllare la variabile locale nello stack e notare che non e' stata modificata.
Infatti cosa avrebbe potuto scrivere \scanf in essa? Non ha fatto niente oltre che restituire zero. 


Proviamo ad \q{hackerare} il nostro programma.
Click destro su \EAX, 
Tra le opzioni vediamo \q{Set to 1}.
Esattamente cio' che ci serve.

Adesso abbiamo 1 in \EAX, il controllo successivo sta per essere eseguito come previsto,
e \printf stampera' il valore della variabile nello stack.

Quando avviamo il programma (F9) vediamo il seguente output nella finestra della console:

\lstinputlisting[caption=console window]{patterns/04_scanf/3_checking_retval/console.txt}

1850296084 e' infatti la rappresentazione decimale del numero nello stack (\TT{0x6E494714})!


\clearpage
\subsubsection{MSVC: x86 + Hiew}
\myindex{Hiew}

Quanto detto puo' essere anche usato come semplice esempio di patching di un eseguibile.
Possiamo provare a modificare l'eseguibile in modo che il programma stampi sempre l'input, a prescindere da cosa si inserisce.

Assumendo che l'eseguibile sia compilato rispetto \TT{MSVCR*.DLL} esterna (ovvero con l'opzione \TT{/MD})
\footnote{detta anche \q{dynamic linking}}, 
vediamo la funzione \main all'inizio della sezione \TT{.text}.
Apriamo l'eseguibile con Hiew e troviamo l'inizio della sezione \TT{.text} (Enter, F8, F6, Enter, Enter).

Vedremo questo:

\begin{figure}[H]
\centering
\myincludegraphics{patterns/04_scanf/3_checking_retval/hiew_1.png}
\caption{Hiew: \main function}
\label{fig:scanf_ex3_hiew_1}
\end{figure}

Hiew trova le stringhe \ac{ASCIIZ} e le visualizza, cosi' come i nomi delle funzioni importate.

\clearpage
Spostiamo il cursore all'indirizzo \TT{.00401027} (dove si trova l'istruzione \TT{JNZ} che vogliamo bypassare), premiamo F3, e scriviamo \q{9090} (cioe' due \ac{NOP}):

\begin{figure}[H]
\centering
\myincludegraphics{patterns/04_scanf/3_checking_retval/hiew_2.png}
\caption{Hiew: replacing \TT{JNZ} with two \ac{NOP}s}
\label{fig:scanf_ex3_hiew_2}
\end{figure}

Premiamo quindi F9 (update). L'eseguibile viene quindi salvato su disco, e si comportera' come vogliamo.

Utilizzare due \ac{NOP} non rappresenta l'approccio esteticamente migliore.
Un altro modo di patchare questa istruzione e' scrivere 0 al secondo byte dell'opcode (\gls{jump offset}), 
in modo che \TT{JNZ} salti sempre alla prossima istruzione.

Potremmo anche fare l'opposto: sostituire il primo byte con \TT{EB} senza toccare il secondo byte (\gls{jump offset}).
Otterremmo un jump non condizionale che e' sempre eseguito.
In questo caso il messaggio di errore sarebbe stampato sempre, a prescindere dall'input.

}
\FR{\subsubsection{MSVC: x86}

Voici ce que nous obtenons dans la sortie assembleur (MSVC 2010):

\lstinputlisting[style=customasmx86]{patterns/04_scanf/3_checking_retval/ex3_MSVC_x86.asm}

\myindex{x86!\Registers!EAX}
La fonction \glslink{caller}{appelante} (\main) à besoin du résultat de la fonction
\glslink{callee}{appelée}, donc la fonction \glslink{callee}{appelée} le renvoie
dans la registre \EAX.

\myindex{x86!\Instructions!CMP}
Nous le vérifions avec l'aide de l'instruction \TT{CMP EAX, 1} (\emph{CoMPare}).
En d'autres mots, nous comparons la valeur dans le registre \EAX avec 1.

\myindex{x86!\Instructions!JNE}
Une instruction de saut conditionnelle \JNE suit l'instruction \CMP. \JNE signifie
\emph{Jump if Not Equal} (saut si non égal).

Donc, si la valeur dans le registre \EAX n'est pas égale à 1, le \ac{CPU} va poursuivre
l'exécution à l'adresse mentionnée dans l'opérande \JNE, dans notre cas \TT{\$LN2@main}.
Passez le contrôle à cette adresse résulte en l'exécution par le \ac{CPU} de
\printf avec l'argument \TT{What you entered? Huh?}.
Mais si tout est bon, le saut conditionnel n'est pas pris, et un autre appel à \printf
est exécuté, avec deux arguments:\\
\TT{'You entered \%d...'} et la valeur de \TT{x}.

\myindex{x86!\Instructions!XOR}
\myindex{\CLanguageElements!return}
Puisque dans ce cas le second \printf n'a pas été exécuté, il y a un \JMP qui le précède (saut inconditionnel).
Il passe le contrôle au point après le second \printf et juste avant l'instruction \TT{XOR EAX, EAX}, qui implémente \TT{return 0}.

% FIXME internal \ref{} to x86 flags instead of wikipedia
\myindex{x86!\Registers!\Flags}
Donc, on peut dire que comparer une valeur avec une autre est \emph{usuellement} implémenté
par la paire d'instructions \CMP/\Jcc, où \emph{cc} est un \emph{code de condition}.
\CMP compare deux valeurs et met les flags\footnote{flags x86, voir aussi: \href{http://go.yurichev.com/17120}{Wikipédia}.}
du processeur.
\Jcc vérifie ces flags et décide de passer le contrôle à l'adresse spécifiée ou non.

\myindex{x86!\Instructions!CMP}
\myindex{x86!\Instructions!SUB}
\myindex{x86!\Instructions!JNE}
\myindex{x86!\Registers!ZF}
\label{CMPandSUB} 
Cela peut sembler paradoxal, mais l'instruction \CMP est en fait un \SUB (soustraction).
Toutes les instructions arithmétiques mettent les flags du processeur, pas seulement \CMP.
Si nous comparons 1 et 1, $1-1$ donne 0 donc le flag \ZF va être mis (signifiant
que le dernier résultat est 0).
Dans aucune autre circonstance \ZF ne sera mis, à l'exception que les opérandes
ne soient égaux.
\JNE vérifie seulement le flag \ZF et saute seulement si il n'est pas mis. \JNE
est un synonyme pour \JNZ (\emph{Jump if Not Zero} (saut si non zéro)).
L'assembleur génère le même opcode pour les instructions \JNE et \JNZ.
Donc, l'instruction \CMP peut être remplacée par une instruction \SUB et presque
tout ira bien, à la différence que \SUB altère la valeur du premier opérande.
\CMP est un \emph{SUB sans sauver le résultat, mais modifiant les flags}.

\subsubsection{MSVC: x86: IDA}

\myindex{IDA}
C'est le moment de lancer \IDA et d'essayer de faire quelque chose avec.
À propos, pour les débutants, c'est une bonne idée d'utiliser l'option \TT{/MD}
de MSVC, qui signifie que toutes les fonctions standards ne vont pas être liées
avec le fichier exécutable, mais vont à la place être importées depuis le fichier
\TT{MSVCR*.DLL}.
Ainsi il est plus facile de voir quelles fonctions standards sont utilisées et où.

En analysant du code dans \IDA, il est très utile de laisser des notes pour soi-même
(et les autres).
En la circonstance, analysons cet exemple, nous voyons que \TT{JNZ} sera déclenché
en cas d'erreur.
Donc il est possible de déplacer le curseur sur le label, de presser \q{n} et de
lui donner le nom \q{error}.
Créons un autre label---dans \q{exit}.
Voici mon résultat:

\lstinputlisting[style=customasmx86]{patterns/04_scanf/3_checking_retval/ex3.lst}

Maintenant, il est légèrement plus facile de comprendre le code.
Toutefois, ce n'est pas une bonne idée de commenter chaque instruction.

% FIXME draw button?
Vous pouvez aussi cacher (replier) des parties d'une fonction dans \IDA.
Pour faire cela, marquez le bloc, puis appuyez sur \q{--} sur le pavé numérique et
entrez le texte qui doit être affiché à la place.

Cachons deux blocs et donnons leurs un nom:

\lstinputlisting[style=customasmx86]{patterns/04_scanf/3_checking_retval/ex3_2.lst}

% FIXME draw button?
Pour étendre les parties de code précédemment cachées. utilisez \q{+} sur le
pavé numérique.

\clearpage
En appuyant sur \q{space}, nous voyons comment \IDA représente une fonction sous
forme de graphe:

\begin{figure}[H]
\centering
\myincludegraphics{patterns/04_scanf/3_checking_retval/IDA.png}
\caption{IDA en mode graphe}
\label{fig:ex3_IDA_1}
\end{figure}

Il y a deux flèches après chaque saut conditionnel: une verte et une rouge.
La flèche verte pointe vers le bloc qui sera exécuté si le saut est déclenché,
et la rouge sinon.

\clearpage
Il est possible de replier des nœuds dans ce mode et de leurs donner aussi un nom (\q{group nodes}).
Essayons avec 3 blocs:

\begin{figure}[H]
\centering
\myincludegraphics{patterns/04_scanf/3_checking_retval/IDA2.png}
\caption{IDA en mode graphe avec 3 nœuds repliés}
\label{fig:ex3_IDA_2}
\end{figure}

C'est très pratique.
On peut dire qu'une part importante du travail des rétro-ingénieurs (et de tout
autre chercheur également) est de réduire la quantité d'information avec laquelle
travailler.

\clearpage
\myparagraph{MSVC + \olly}
\myindex{\olly}

2 paires de mots 32-bit sont marquées en rouge sur la pile.
Chaque paire est un double au format IEEE 754 et est passée depuis \main.

Nous voyons comment le premier \FLD charge une valeur ($1.2$) depuis la pile et la
stocke dans \ST{0}:

\begin{figure}[H]
\centering
\myincludegraphics{patterns/12_FPU/1_simple/olly1.png}
\caption{\olly: le premier \FLD a été exécuté}
\label{fig:FPU_simple_olly_1}
\end{figure}

À cause des inévitables erreurs de conversion depuis un nombre flottant 64-bit au
format IEEE 754 vers 80-bit (utilisé en interne par le FPU), ici nous voyons 1.199\ldots,
qui est proche de 1.2.

\EIP pointe maintenant sur l'instruction suivante (\FDIV), qui charge un double
(une constante) depuis la mémoire.
Par commodité, \olly affiche sa valeur: 3.14

\clearpage
Continuons l'exécution pas à pas.
\FDIV a été exécuté, maintenant \ST{0} contient 0.382\ldots
(\gls{quotient}):

\begin{figure}[H]
\centering
\myincludegraphics{patterns/12_FPU/1_simple/olly2.png}
\caption{\olly: \FDIV a été exécuté}
\label{fig:FPU_simple_olly_2}
\end{figure}

\clearpage
Troisième étape: le \FLD suivant a été exécuté, chargeant 3.4 dans \ST{0} (ici nous
voyons la valeur approximative 3.39999\ldots):

\begin{figure}[H]
\centering
\myincludegraphics{patterns/12_FPU/1_simple/olly3.png}
\caption{\olly: le second \FLD a été exécuté}
\label{fig:FPU_simple_olly_3}
\end{figure}

En même temps, le \gls{quotient} \emph{est poussé} dans \ST{1}.
Exactement maintenant, \EIP pointe sur la prochaine instruction: \FMUL.
Ceci charge la constante 4.1 depuis la mémoire, ce que montre \olly.

\clearpage
Suivante: \FMUL a été exécutée, donc maintenant le \glslink{product}{produit} est dans \ST{0}:

\begin{figure}[H]
\centering
\myincludegraphics{patterns/12_FPU/1_simple/olly4.png}
\caption{\olly: \FMUL a été exécuté}
\label{fig:FPU_simple_olly_4}
\end{figure}

\clearpage
Suivante: \FADDP a été exécutée, maintenant le résultat de l'addition est dans \ST{0},
et \ST{1} est vidé.

\begin{figure}[H]
\centering
\myincludegraphics{patterns/12_FPU/1_simple/olly5.png}
\caption{\olly: \FADDP a été exécuté}
\label{fig:FPU_simple_olly_5}
\end{figure}

Le résultat est laissé dans \ST{0}, car la fonction renvoie son résultat dans \ST{0}.

\main prend cette valeur depuis le registre plus loin.

Nous voyons quelque chose d'inhabituel: la valeur 13.93\ldots se trouve maintenant
dans \ST{7}.
Pourquoi?

\label{FPU_is_rather_circular_buffer}

Comme nous l'avons lu il y a quelque temps dans ce livre, les registres \ac{FPU} sont
une pile: \myref{FPU_is_stack}.
Mais ceci est une simplification.

Imaginez si cela était implémenté \emph{en hardware} comme cela est décrit, alors
tout le contenu des 7 registres devrait être déplacé (ou copié) dans les registres
adjacents lors d'un push ou d'un pop, et ceci nécessite beaucoup de travail.

En réalité, le \ac{FPU} a seulement 8 registres et un pointeur (appelé \GTT{TOP})
qui contient un numéro de registre, qui est le \q{haut de la pile} courant.

Lorsqu'une valeur est poussée sur la pile, \GTT{TOP} est déplacé sur le registre
disponible suivant, et une valeur est écrite dans ce registre.

La procédure est inversée si la valeur est lue, toutefois, le registre qui a été
libéré n'est pas vidé (il serait possible de le vider, mais ceci nécessite plus de
travail qui peut dégrader les performances).
Donc, c'est ce que nous voyons ici.

On peut dire que \FADDP sauve la somme sur la pile, et y supprime un élément.

Mais en fait, cette instruction sauve la somme et ensuite décale \GTT{TOP}.

Plus précisément, les registres du  \ac{FPU} sont un tampon circulaire.


\clearpage
\subsubsection{MSVC: x86 + Hiew}
\myindex{Hiew}

Cela peut également être utilisé comme un exemple simple de modification de fichier
exécutable.
Nous pouvons essayer de modifier l'exécutable de telle sorte que le programme va
toujours afficher notre entrée, quelle quelle soit.

En supposant que l'exécutable est compilé avec la bibliothèque externe \TT{MSVCR*.DLL}
(i.e., avec l'option \TT{/MD}) \footnote{c'est aussi appelé \q{dynamic linking}},
nous voyons la fonction \main au début de la section \TT{.text}.
Ouvrons l'exécutable dans Hiew et cherchons le début de la section \TT{.text} (Enter,
F8, F6, Enter, Enter).

Nous pouvons voir cela:

\begin{figure}[H]
\centering
\myincludegraphics{patterns/04_scanf/3_checking_retval/hiew_1.png}
\caption{Hiew: fonction \main}
\label{fig:scanf_ex3_hiew_1}
\end{figure}

Hiew trouve les chaîne \ac{ASCIIZ} et les affiche, comme il le fait avec le nom
des fonctions importées.

\clearpage
Déplacez le curseur à l'adresse \TT{.00401027} (où se trouve l'instruction \TT{JNZ},
que l'on doit sauter), appuyez sur F3, et ensuite tapez \q{9090} (qui signifie deux
\ac{NOP}s):

\begin{figure}[H]
\centering
\myincludegraphics{patterns/04_scanf/3_checking_retval/hiew_2.png}
\caption{Hiew: remplacement de \TT{JNZ} par deux \ac{NOP}s}
\label{fig:scanf_ex3_hiew_2}
\end{figure}

Appuyez sur F9 (update). Maintenant, l'exécutable est sauvé sur le disque. Il va
se comporter comme nous le voulions.

Deux \ac{NOP}s ne constitue probablement pas l'approche la plus esthétique.
Une autre façon de modifier cette instruction est d'écrire simplement 0 dans le
second octet de l'opcode ((\gls{jump offset}), donc ce \TT{JNZ} va toujours sauter
à l'instruction suivante.

Nous pouvons également faire le contraire: remplacer le premier octet avec \TT{EB}
sans modifier le second octet (\gls{jump offset}).
Nous obtiendrions un saut inconditionnel qui est toujours déclenché.
Dans ce cas le message d'erreur sera affiché à chaque fois, peu importe l'entrée.

}
\DE{\subsubsection{MSVC: x86}
Wir erhalten den folgenden Assembleroutput (MSVC 2010):

\lstinputlisting[style=customasmx86]{patterns/04_scanf/3_checking_retval/ex3_MSVC_x86.asm}

\myindex{x86!\Registers!EAX}
Die aufrufende Funktion (\main) benötigt das Ergebnis der aufgerufenen Funktion (\scanf), weshalb diese es über das
Register \EAX zurückgibt.

\myindex{x86!\Instructions!CMP}
Wir prüfen mithilfe des Befehls \TT{CMP EAX, 1} (\emph{CoMPare}). Mit anderen Worten, wir vergleichen den Wert in \EAX mit
1.

\myindex{x86!\Instructions!JNE}
Ein bedingter \JNE Sprung folgt auf den \CMP Befehl. \JNE steht für \emph{Jump if Not Equal}.

Wenn also der Wert in \EAX ungleich 1 ist, wird die \ac{CPU} die Ausführung an der Stelle fortsetzen, die im Operanden
von \JNE steht, in unserem Fall \TT{\$LN2@main}.
Den Control Flow an diese Adresse zu übergeben hat zur Folge, dass die Funktion \printf mit dem Argument \TT{What you
entered? Huh?} aufgerufen wird.
Wenn aber alles funktioniert und der bedingte Sprung nicht ausgeführt wird, wird ein anderer Aufruf von \printf mit zwei
Argumenten ausgeführt:\\
\TT{'You entered \%d...'} und dem Wert von \TT{x}.


\myindex{x86!\Instructions!XOR}
\myindex{\CLanguageElements!return}
Da in diesem Fall das zweite \printf nicht ausgeführt werden darf, befindet sich davor ein \JMP (unbedingter Sprung).
Dieser gibt den Control Flow ab an den Punkt nach dem zweiten \printf, genau vor dem \TT{XOR EAX EAX} Befehl, welcher
die Rückgabe von 0 implementiert.

% FIXME internal \ref{} to x86 flags instead of wikipedia
\myindex{x86!\Registers!\Flags}
Man kann also festhalten, dass der Vergleich von zwei Werten gewöhnlich durch ein \CMP/\Jcc Befehlspaar implementiert
wird, wobei \emph{cc} für \emph{condition code}, also Sprungbedingung, steht. 
\CMP vergleicht zwei Werte und setzt die Flags des Prozessors\footnote{zu x86 Flags, siehe auch:
\href{http://go.yurichev.com/17120}{wikipedia}.}.
\Jcc prüft diese Flags und entscheidet entweder den Control Flow an die angegebene Adresse zu übergeben oder nicht.

\myindex{x86!\Instructions!CMP}
\myindex{x86!\Instructions!SUB}
\myindex{x86!\Instructions!JNE}
\myindex{x86!\Registers!ZF}
\label{CMPandSUB}
Es klingt möglicherweise paradox, aber der \CMP Befehl ist tatsächlich ein \SUB (subtract).
Alle arithmetischen Befehle setzen die Flags des Prozessors, nicht nur \CMP.
Wenn wir 1 und 1 vergleichen, ist $1-1=0$ und daher wird das \ZF Flag gesetzt (gleichbedeutend damit, dass das Ergebnis
der letzten Berechnung 0 ergeben hat).
\ZF kann nur durch diesen Umstand gesetzt werden, nämlich, dass zwei Operanden gleich sind.
\JNE prüft nur das \ZF Flag und springt nur, wenn dieses nicht gesetzt ist. \JNE ist daher ein Synonym für \JNZ
(\emph{Jump if Not Zero}).
Der Assembler übersetzt \JNE und \JNZ in den gleichen Opcode.
Der \CMP Befehl kann also durch ein \SUB ersetzt werden, aber mit dem Unterschied, dass \SUB den Wert des ersten
Operanden verändert. \CMP bedeutet also \emph{SUB ohne Speichern des Ergebnisses, aber mit Setzen der Flags}.

\subsubsection{MSVC: x86: IDA}

\myindex{IDA}
Es ist an der Zeit \IDA auszuprobieren und etwas damit zu machen.
Für Anfänger ist es übrigens eine gute Idee, die \TT{/MD} Option in MSVC zu verwenden, da diese bewirkt, dass alle
Standardfunktionen nicht mit der ausführbaren Datei verlinkt werden, sondern aus der Datei \TT{MSVCR*.DLL} importiert
werden. Dadurch ist es einfacher zu erkennen, welche Standardfunktionen verwendet werden und wo dies geschieht.

Bei der Codeanalyse in \IDA ist es hilfreich Notizen für sich selbst (und andere) zu hinterlassen.
Bei der Analyse dieses Beispiels sehen wir, dass \TT{JNZ} im Falle eines Fehlers ausgeführt wird.
Es ist nun möglich den Cursor auf das Label zu setzen, \q{n} zu drücken und es in \q{error} umzubenennen.
Wir erstellen noch ein Label---in \q{exit}.
Hier ist das Ergebnis:

\lstinputlisting[style=customasmx86]{patterns/04_scanf/3_checking_retval/ex3.lst}
So ist es etwas einfacher den Code zu verstehen. Natürlich ist es aber auch keine gute Idee, jeden Befehl zu
kommentieren.

% FIXME draw button?
Man kann Teile einer Funktion in \IDA auch einklappen. Um dies zu tun, markiert man den Block und drückt dann \q{--} auf
dem Zahlenblock der Tastatur und gibt den stattdessen anzuzeigenden Text ein.

Verstecken wir zwei Blöcke und geben ihnen Namen:

\lstinputlisting[style=customasmx86]{patterns/04_scanf/3_checking_retval/ex3_2.lst}

% FIXME draw button?
Um eingeklappte Teile des Code wieder auszuklappen, verwendet man \q{+} auf dem Zahlenblock der Tastatur.

\clearpage
Durch Drücken von der \q{Leertaste} sehen wir, wie \IDA die Funktion als Graph darstellt:

\begin{figure}[H]
\centering
\myincludegraphics{patterns/04_scanf/3_checking_retval/IDA.png}
\caption{Graph Modus in IDA}
\label{fig:ex3_IDA_1}
\end{figure}
Es gibt hinter jedem bedingten Sprung zwei Pfeile: einen grünen und einen roten.
Der grüne Pfeil zeigt auf den Codeblock der ausgeführt wird, wenn der Sprung ausgeführt wird und der rote den Codeblock,
der ausgeführt wird, falls nicht gesprungen wird.

\clearpage
In diesem Modus ist es möglich, Knoten einzuklappen und ihnen auch Namen zu geben (\q{group nodes}).
Wir probieren das mit 3 Blöcken aus:

\begin{figure}[H]
\centering
\myincludegraphics{patterns/04_scanf/3_checking_retval/IDA2.png}
\caption{Graph Modus in IDA mit 3 eingeklappten Knoten}
\label{fig:ex3_IDA_2}
\end{figure}

Das ist sehr nützlich.
Man kann sagen, dass ein großer Teil der Arbeit eines Reverse Engineers (und eines jeden anderen Forsches) darin
besteht, die Menge der zur Verfügung stehenden Informationen zu reduzieren.

\clearpage
\myparagraph{\Optimizing MSVC + \olly}
\myindex{\olly}

Wir untersuchen das (optimierte) Beispiel in \olly. Hier ist der erste
Durchlauf:

\begin{figure}[H]
\centering
\myincludegraphics{patterns/10_strings/1_strlen/olly1.png}
\caption{\olly: Beginn erster Durchlauf}
\label{fig:strlen_olly_1}
\end{figure}

Wir sehen, dass \olly eine Schleife gefunden hat, und zur Verbesserung der
Lesbarkeit, diese in eckige Klammern \emph{eingeschlossen} hat.
Nach Rechtsklick auf \EAX wählen wir \q{Follow in Dump} und das Speicherfenster
scrollt an die passende Stelle. 
Hier sehen wir den String \q{hello!} im Speicher. 
Dahinter befindet sich mindestens ein Nullbyte und im Anschluss Zufallsbits.

Wenn \olly ein Register mit einer gültigen Adresse, die auf einen String zeigt,
findet, wird dieser String angezeigt.

\clearpage
Wir drücken einige Male F8 (\stepover) um zum Anfang der Schleifenkörpers zu
gelangen:

\begin{figure}[H]
\centering
\myincludegraphics{patterns/10_strings/1_strlen/olly2.png}
\caption{\olly: Beginn zweiter Durchlauf}
\label{fig:strlen_olly_2}
\end{figure}

Wir sehen, dass \EAX nun die Adresse des zweiten Zeichens des Strings enthält.

\clearpage

Durch hinreichend häufiges Drücken von F8 verlassen wir schließlich die
Schleife:

\begin{figure}[H]
\centering
\myincludegraphics{patterns/10_strings/1_strlen/olly3.png}
\caption{\olly: Pointer Differenz wird berechnet}
\label{fig:strlen_olly_3}
\end{figure}

% FIXME:
Wir sehen, dass \EAX jetzt die Adresse des Nullbytes direkt hinter dem String
enthält. In der Zwischenzeit hat sich \EDX nicht verändert, es zeigt also immer
noch auf den Anfang des Strings. 

Die Differenz zwischen den beiden Adressen wird jetzt berechnet. 

\clearpage
Der \SUB Befehl wurde gerade ausgeführt:

\begin{figure}[H]
\centering
\myincludegraphics{patterns/10_strings/1_strlen/olly4.png}
\caption{\olly: \EAX muss dekrementiert werden}
\label{fig:strlen_olly_4}
\end{figure}

Die Differenz der Pointer im \EAX Register beträgt nun--7.
Tatsächlich beträgt die Länge des \q{hello!} Strings 6 Zeichen, aber mit dem
Nullbyte am Ende dazugezählt sind es 7.
Die Funktion \TT{strlen()} soll aber die Anzahl der Nicht-Null-Zeichen im String
zurückliefert, also wird einmal dekrementiert und der Funktionsaufruf
anschließend beendet.


\clearpage
\subsubsection{MSVC: x86 + Hiew}
\myindex{Hiew}
Unser Programm kann auch als einfaches Beispel für das Patchen einer Executable dienen.
Wir könnten versuchen, die Executable so zu patchen, dass das Programm unabhängig vom Input diesen stets auszugeben.

Angenommen, dass die Executable mit externer \TT{MSVCR*.DLL} (d.h. mit der Option \TT{MD}) kompiliert
wurde\footnote{dieser Vorgang wird auch \q{dynamisches Verlinken genannt}}, finden wir die Funktion \main am Anfang des
\TT{.text} Segments. 
Öffnen wir die Executable in Hiew und schauen uns den Anfang des \TT{.text} Segments an (Enter, F8, F6, Enter, Enter).

Wir sehen das Folgende:

\begin{figure}[H]
\centering
\myincludegraphics{patterns/04_scanf/3_checking_retval/hiew_1.png}
\caption{Hiew: \main Funktion}
\label{fig:scanf_ex3_hiew_1}
\end{figure}

Hiew erkennt \ac{ASCIIZ} Strings und die Namen importierter Funktionen und zeigt diese an.

\clearpage
Setzen wir den Cursor auf die Adresse \TT{.00401027}, an der sich der \TT{JNZ} Befehl, den wir umgehen müssen, befindet,
drücken F3 und fügen dann \q{9090} (zwei \ac{NOPS}s) ein. 

\begin{figure}[H]
\centering
\myincludegraphics{patterns/04_scanf/3_checking_retval/hiew_2.png}
\caption{Hiew: ersetzen von \TT{JNZ} durch zwei \ac{NOP}s}
\label{fig:scanf_ex3_hiew_2}
\end{figure}
Wir drücken F9 (update). Die Executable wird gespeichert und verhält sich wie gewünscht.

Zwei \ac{NOP}s sind wahrscheinlich nicht der ästhetischte Ansatz. Ein anderer Weg die Executable zu patchen besteht
darin, das zweite Byte des Opcodes (den \gls{jump offset}) auf 0 zu setzen, sodass \TT{JNZ} immer zum nächsten Befehl
springt.

Wir könnten auch das Gegenteil tun: das erste Byte durch \TT{EB} ersetzen und das zweite (\gls{jump offset})
unangetastet lassen. Wir würde einen unbedingten Sprung erhalten, der stets ausgeführt wird.
In diesem Fall würde unabhängig vom Input stets die Fehlermeldung ausgegeben.
}
\JPN{\subsubsection{MSVC: x86}

アセンブリ出力(MSVC 2010)の内容は次のとおりです。

\lstinputlisting[style=customasmx86]{patterns/04_scanf/3_checking_retval/ex3_MSVC_x86.asm}

\myindex{x86!\Registers!EAX}
\gls{caller} 関数( \main )は \gls{callee} 関数( \scanf )の結果を必要とするため、
呼び出し先は \EAX レジスタに返します。

\myindex{x86!\Instructions!CMP}
我々は、\TT{CMP EAX, 1} (\emph{CoMPare})の指示によりそれをチェックします。つまり、 \EAX レジスタの値と1を比較します。

\myindex{x86!\Instructions!JNE}
\JNE 条件ジャンプが \CMP 命令の後に続きます。  \JNE は\emph{Jump if Not Equal}の略です。

したがって、 \EAX レジスタの値が1に等しくない場合、\ac{CPU}は \JNE オペランドに記述されているアドレス(この場合は\TT{\$LN2@main})に実行を渡します。
このアドレスに制御を渡すと、\ac{CPU}は \printf を引数\TT{What you entered? Huh?} で実行します 。
しかし、すべてがうまくいけば、条件付きジャンプは取られず、別の \printf 呼び出しが\TT{'You entered \%d...'}と\TT{x}の値を引数にとって実行されます。

\myindex{x86!\Instructions!XOR}
\myindex{\CLanguageElements!return}
この場合、2番目の \printf は実行されないため、その前に \JMP があります(無条件ジャンプ)。
2番目の \printf の後、\TT{戻り値0}を実装する\TT{XOR EAX, EAX}命令の直前に制御を渡します。

% FIXME internal \ref{} to x86 flags instead of wikipedia
\myindex{x86!\Registers!\Flags}
したがって、ある値を別の値と比較することは、\TT{通常}、 \CMP/\Jcc 命令ペアによって実装されると言えます\emph{cc}は\emph{条件コード}です。 
\CMP は2つの値を比較し、プロセッサフラグ\footnote{x86フラグは以下を参照: \href{http://go.yurichev.com/17120}{wikipedia}}を設定します。 
\Jcc はこれらのフラグをチェックし、指定されたアドレスに制御を渡すかどうかを決定します。

\myindex{x86!\Instructions!CMP}
\myindex{x86!\Instructions!SUB}
\myindex{x86!\Instructions!JNE}
\myindex{x86!\Registers!ZF}
\label{CMPandSUB}
これは逆説的に聞こえるかもしれませんが、 \CMP 命令は実際には \SUB (減算)です。
すべての算術命令は、 \CMP だけでなくプロセッサフラグを設定します。 1と1を比較し、$1-1$が0であるため、ZFフラグが設定されます(最後の結果が0であることを意味します)。
オペランドが等しい場合を除いて、 \ZF は設定できません。  \JNE は \ZF フラグのみをチェックし、設定されていない場合にジャンプします。 
JNEは実際にはJNZ(\emph{Jump if Not Zero})の同義語です。アセンブラは、JNE命令とJNZ命令の両方を同じオペコードに変換します。
したがって、 \CMP 命令は \SUB 命令で置き換えることができ、 \SUB が最初のオペランドの値を変更するという違いを除けば、ほとんどすべてが問題ありません。 
\CMP \emph{は結果を保存しない \SUB ですが、フラグに影響}します。

\subsubsection{MSVC: x86: IDA}

\myindex{IDA}
IDAを実行してIDAを実行しようとします。 
ところで、初心者の方は、MSVCで\TT{/MD}オプションを使用することをお勧めします。つまり、
これらの標準関数はすべて実行可能ファイルにリンクされず、
代わりに\TT{MSVCR*.DLL}ファイルからインポートされます。 
したがって、どの標準関数が使用され、どこでどこが使用されているのかが分かりやすくなります。

\IDA のコードを分析する際には、自分自身(と他者)のためにノートを残すことが非常に役に立ちます。 
例えば、この例を分析すると、
エラーが発生した場合に \JNZ がトリガーされることがわかります。 
カーソルをラベルに移動して\q{n}を押し、\q{エラー}に名前を変更することができます。 
別のラベルを作成し、\q{終了}にします。 
以下が私の環境での結果です。

\lstinputlisting[style=customasmx86]{patterns/04_scanf/3_checking_retval/ex3.lst}

これで、コードを少し理解しやすくなりました。 
しかし、すべての命令についてコメントするのは良い考えではありません。

% FIXME draw button?
また、 \IDA の関数の一部を隠すこともできます。 
ブロックをマークするには、 \q{--} を数値パッドに入力し、代わりに表示するテキストを入力します。

2つのブロックを隠して名前を付けましょう。

\lstinputlisting[style=customasmx86]{patterns/04_scanf/3_checking_retval/ex3_2.lst}

% FIXME draw button?
以前に折りたたまれた部分を展開するには、数値パッドで\q{+}を使用します。

\clearpage
\q{スペース}を押すと、 \IDA が関数をグラフとして表示するのを見ることができます。

\begin{figure}[H]
\centering
\myincludegraphics{patterns/04_scanf/3_checking_retval/IDA.png}
\caption{Graph mode in IDA}
\label{fig:ex3_IDA_1}
\end{figure}

各条件ジャンプの後、緑と赤の2つの矢印があります。 
緑の矢印は、ジャンプがトリガされた場合に実行されるブロックを指し、そうでない場合は赤を指します。

\clearpage
このモードでノードを折りたたみ、名前を付けることもできます([q{グループノード})。
3つのブロックでやってみましょう。

\begin{figure}[H]
\centering
\myincludegraphics{patterns/04_scanf/3_checking_retval/IDA2.png}
\caption{Graph mode in IDA with 3 nodes folded}
\label{fig:ex3_IDA_2}
\end{figure}

それは非常に便利です。
リバースエンジニアの仕事(および他の研究者の仕事)の非常に重要な部分は、彼らが扱う情報の量を減らすことであると言えます。

\clearpage
\myparagraph{\Optimizing MSVC + \olly}
\myindex{\olly}

\olly でこの(最適化された)例を試すことができます。ここに最初のイテレーションがあります:

\begin{figure}[H]
\centering
\myincludegraphics{patterns/10_strings/1_strlen/olly1.png}
\caption{\olly: 最初のイテレーションの開始}
\label{fig:strlen_olly_1}
\end{figure}

\olly がループを見つけ、便宜上、その指示を括弧で\emph{囲んだ}ことがわかります。 
\EAX の右ボタンをクリックして、\q{Follow in Dump}を選択すると、メモリウィンドウが正しい場所にスクロールします。
ここではメモリ内に \q{hello!}という文字列があります。
その後に少なくとも1つのゼロバイトがあり、次にランダムなごみがあります。

\olly が有効なアドレスを持つレジスタを見ると、それは文字列を指しており、
文字列として表示されます。

\clearpage
F8(\stepover)を数回押して、ループの本体の先頭に移動しましょう:

\begin{figure}[H]
\centering
\myincludegraphics{patterns/10_strings/1_strlen/olly2.png}
\caption{\olly: 2回目のイテレーションの開始}
\label{fig:strlen_olly_2}
\end{figure}

\EAX には文字列中の2番目の文字のアドレスが含まれていることがわかります。

\clearpage

我々はループから脱出するためにF8を十分な回数押す必要があります:

\begin{figure}[H]
\centering
\myincludegraphics{patterns/10_strings/1_strlen/olly3.png}
\caption{\olly: 計算すべきポインタの差}
\label{fig:strlen_olly_3}
\end{figure}

% FIXME:
\EAX には文字列の直後に0バイトのアドレスが含まれることがわかりました。一方、
\EDX は変更されていないので、まだ文字列の先頭を指しています。

これらの2つのアドレスの差がここで計算されています。

\clearpage
\SUB 命令が実行されました。

\begin{figure}[H]
\centering
\myincludegraphics{patterns/10_strings/1_strlen/olly4.png}
\caption{\olly: \EAX がデクリメントされる}
\label{fig:strlen_olly_4}
\end{figure}

ポインタの違いは \EAX レジスタにあります(値は7)。
確かに、 \q{hello!}文字列の長さは6ですが、7にはゼロバイトが含まれています。
しかし、\TT{strlen()}は文字列中のゼロ以外の文字数を返さなければなりません。
したがって、デクリメントが実行され、関数が戻ります。


\clearpage
\subsubsection{MSVC: x86 + Hiew}
\myindex{Hiew}

これは、実行可能ファイルのパッチ適用の簡単な例としても使用できます。 
実行可能ファイルにパッチを適用して、入力内容にかかわらずプログラムが常に入力を出力するようにすることがあります。

実行可能ファイルが外部の\TT{MSVCR*.DLL}(つまり\TT{/MD}オプション付き)
\footnote{\q{ダイナミックリンク}とも呼ばれる}
に対してコンパイルされていると仮定すると、\TT{.text}セクションの先頭に \main 関数があります。 
Hiewで実行可能ファイルを開き、\TT{.text}セクションの先頭を見つけましょう(Enter、F8、F6、Enter、Enter)。 

以下のように見えます。

\begin{figure}[H]
\centering
\myincludegraphics{patterns/04_scanf/3_checking_retval/hiew_1.png}
\caption{Hiew: \main function}
\label{fig:scanf_ex3_hiew_1}
\end{figure}

Hiewは\ac{ASCIIZ}文字列を検索し、インポートされた関数の名前と同様に表示します。

\clearpage
カーソルを\TT{.00401027}番地(ここでバイパスする \JNZ 命令がある場所)に移動し、F3を押し、\q{9090}(2つの\ac{NOP}を意味する)と入力します。

\begin{figure}[H]
\centering
\myincludegraphics{patterns/04_scanf/3_checking_retval/hiew_2.png}
\caption{Hiew: replacing \TT{JNZ} with two \ac{NOP}s}
\label{fig:scanf_ex3_hiew_2}
\end{figure}

その後、F9(更新)を押します。 これで、実行可能ファイルがディスクに保存されます。 私たちが望むように動作します。

2つの\ac{NOP}はおそらく最も美しいアプローチではありません。 
この命令をパッチする別の方法は、第2オペコードバイトに0を書き込むことであり( \gls{jump offset} )、 
\JNZ は常に次の命令にジャンプします。

また、最初のバイトを\TT{EB}で置き換え、2番目のバイト( \gls{jump offset} )には触れないでください。 
私たちは常に無条件のジャンプを得るでしょう。 
この場合、エラーメッセージは入力に関係なく毎回表示されます。
}

\EN{\subsubsection{x64: 8 arguments}

\myindex{x86-64}
\label{example_printf8_x64}
To see how other arguments are passed via the stack, let's change our example again 
by increasing the number of arguments to 9 (\printf format string + 8 \Tint variables):

\lstinputlisting[style=customc]{patterns/03_printf/2.c}

\myparagraph{MSVC}

As it was mentioned earlier, the first 4 arguments has to be passed through the \RCX, \RDX, \Reg{8}, \Reg{9} registers in Win64, while all the rest---via the stack.
That is exactly what we see here.
However, the \MOV instruction, instead of \PUSH, is used for preparing the stack, so the values are stored to the stack in a straightforward manner.

\lstinputlisting[caption=MSVC 2012 x64,style=customasmx86]{patterns/03_printf/x86/2_MSVC_x64_EN.asm}

The observant reader may ask why are 8 bytes allocated for \Tint values, when 4 is enough?
Yes, one has to recall: 8 bytes are allocated for any data type shorter than 64 bits.
This is established for the convenience's sake: it makes it easy to calculate the address of arbitrary argument.
Besides, they are all located at aligned memory addresses.
It is the same in the 32-bit environments: 4 bytes are reserved for all data types.

% also for local variables?

\myparagraph{GCC}

The picture is similar for x86-64 *NIX OS-es, except that the first 6 arguments are passed through the \RDI, \RSI,
\RDX, \RCX, \Reg{8}, \Reg{9} registers.
All the rest---via the stack.
GCC generates the code storing the string pointer into \EDI instead of \RDI{}---we noted that previously: 
\myref{hw_EDI_instead_of_RDI}.

We also noted earlier that the \EAX register has been cleared before a \printf call: \myref{SysVABI_input_EAX}.

\lstinputlisting[caption=\Optimizing GCC 4.4.6 x64,style=customasmx86]{patterns/03_printf/x86/2_GCC_x64_EN.s}

\myparagraph{GCC + GDB}
\myindex{GDB}

Let's try this example in \ac{GDB}.

\begin{lstlisting}
$ gcc -g 2.c -o 2
\end{lstlisting}

\begin{lstlisting}
$ gdb 2
GNU gdb (GDB) 7.6.1-ubuntu
...
Reading symbols from /home/dennis/polygon/2...done.
\end{lstlisting}

\begin{lstlisting}[caption=let's set the breakpoint to \printf{,} and run]
(gdb) b printf
Breakpoint 1 at 0x400410
(gdb) run
Starting program: /home/dennis/polygon/2 

Breakpoint 1, __printf (format=0x400628 "a=%d; b=%d; c=%d; d=%d; e=%d; f=%d; g=%d; h=%d\n") at printf.c:29
29	printf.c: No such file or directory.
\end{lstlisting}

Registers \RSI/\RDX/\RCX/\Reg{8}/\Reg{9} have the expected values.
\RIP has the address of the very first instruction of the \printf function.

\begin{lstlisting}
(gdb) info registers
rax            0x0	0
rbx            0x0	0
rcx            0x3	3
rdx            0x2	2
rsi            0x1	1
rdi            0x400628	4195880
rbp            0x7fffffffdf60	0x7fffffffdf60
rsp            0x7fffffffdf38	0x7fffffffdf38
r8             0x4	4
r9             0x5	5
r10            0x7fffffffdce0	140737488346336
r11            0x7ffff7a65f60	140737348263776
r12            0x400440	4195392
r13            0x7fffffffe040	140737488347200
r14            0x0	0
r15            0x0	0
rip            0x7ffff7a65f60	0x7ffff7a65f60 <__printf>
...
\end{lstlisting}

\begin{lstlisting}[caption=let's inspect the format string]
(gdb) x/s $rdi
0x400628:	"a=%d; b=%d; c=%d; d=%d; e=%d; f=%d; g=%d; h=%d\n"
\end{lstlisting}

Let's dump the stack with the x/g command this time---\emph{g} stands for \emph{giant words}, i.e., 64-bit words.

\begin{lstlisting}
(gdb) x/10g $rsp
0x7fffffffdf38:	0x0000000000400576	0x0000000000000006
0x7fffffffdf48:	0x0000000000000007	0x00007fff00000008
0x7fffffffdf58:	0x0000000000000000	0x0000000000000000
0x7fffffffdf68:	0x00007ffff7a33de5	0x0000000000000000
0x7fffffffdf78:	0x00007fffffffe048	0x0000000100000000
\end{lstlisting}

The very first stack element, just like in the previous case, is the \ac{RA}.
3 values are also passed through the stack: 6, 7, 8.
We also see that 8 is passed with the high 32-bits not cleared: \GTT{0x00007fff00000008}.
That's OK, because the values are of \Tint type, which is 32-bit.
So, the high register or stack element part may contain \q{random garbage}.

If you take a look at where the control will return after the \printf execution,
\ac{GDB} will show the entire \main function:

\begin{lstlisting}[style=customasmx86]
(gdb) set disassembly-flavor intel
(gdb) disas 0x0000000000400576
Dump of assembler code for function main:
   0x000000000040052d <+0>:	push   rbp
   0x000000000040052e <+1>:	mov    rbp,rsp
   0x0000000000400531 <+4>:	sub    rsp,0x20
   0x0000000000400535 <+8>:	mov    DWORD PTR [rsp+0x10],0x8
   0x000000000040053d <+16>:	mov    DWORD PTR [rsp+0x8],0x7
   0x0000000000400545 <+24>:	mov    DWORD PTR [rsp],0x6
   0x000000000040054c <+31>:	mov    r9d,0x5
   0x0000000000400552 <+37>:	mov    r8d,0x4
   0x0000000000400558 <+43>:	mov    ecx,0x3
   0x000000000040055d <+48>:	mov    edx,0x2
   0x0000000000400562 <+53>:	mov    esi,0x1
   0x0000000000400567 <+58>:	mov    edi,0x400628
   0x000000000040056c <+63>:	mov    eax,0x0
   0x0000000000400571 <+68>:	call   0x400410 <printf@plt>
   0x0000000000400576 <+73>:	mov    eax,0x0
   0x000000000040057b <+78>:	leave  
   0x000000000040057c <+79>:	ret    
End of assembler dump.
\end{lstlisting}

Let's finish executing \printf, execute the instruction
zeroing \EAX, and note that the \EAX register has a value of exactly zero.
\RIP now points to the \INS{LEAVE} instruction, i.e., the penultimate one in the \main function.

\begin{lstlisting}
(gdb) finish
Run till exit from #0  __printf (format=0x400628 "a=%d; b=%d; c=%d; d=%d; e=%d; f=%d; g=%d; h=%d\n") at printf.c:29
a=1; b=2; c=3; d=4; e=5; f=6; g=7; h=8
main () at 2.c:6
6		return 0;
Value returned is $1 = 39
(gdb) next
7	};
(gdb) info registers
rax            0x0	0
rbx            0x0	0
rcx            0x26	38
rdx            0x7ffff7dd59f0	140737351866864
rsi            0x7fffffd9	2147483609
rdi            0x0	0
rbp            0x7fffffffdf60	0x7fffffffdf60
rsp            0x7fffffffdf40	0x7fffffffdf40
r8             0x7ffff7dd26a0	140737351853728
r9             0x7ffff7a60134	140737348239668
r10            0x7fffffffd5b0	140737488344496
r11            0x7ffff7a95900	140737348458752
r12            0x400440	4195392
r13            0x7fffffffe040	140737488347200
r14            0x0	0
r15            0x0	0
rip            0x40057b	0x40057b <main+78>
...
\end{lstlisting}
}
\RU{\subsubsection{x64}
\label{subsec:popcnt}

Немного изменим пример, расширив его до 64-х бит:

\lstinputlisting[label=popcnt_x64_example,style=customc]{patterns/14_bitfields/4_popcnt/shifts64.c}

\myparagraph{\NonOptimizing GCC 4.8.2}

Пока всё просто.

\lstinputlisting[caption=\NonOptimizing GCC 4.8.2,style=customasmx86]{patterns/14_bitfields/4_popcnt/shifts64_GCC_O0_RU.s}

\myparagraph{\Optimizing GCC 4.8.2}

\lstinputlisting[caption=\Optimizing GCC 4.8.2,numbers=left,label=shifts64_GCC_O3,style=customasmx86]{patterns/14_bitfields/4_popcnt/shifts64_GCC_O3_RU.s}

Код более лаконичный, но содержит одну необычную вещь.
Во всех примерах, что мы пока видели, инкремент значения переменной \q{rt} происходит после сравнения 
определенного бита с единицей, но здесь \q{rt} увеличивается на 1 до этого (строка 6), записывая новое значение
в регистр \EDX.

Затем, если последний бит был 1, инструкция \CMOVNE\footnote{Conditional MOVe if Not Equal (\MOV если не равно)}
(которая синонимична \CMOVNZ\footnote{Conditional MOVe if Not Zero (\MOV если не ноль)}) \emph{фиксирует} 
новое значение \q{rt}
копируя значение из \EDX (\q{предполагаемое значение rt}) 
в \EAX (\q{текущее rt} которое будет возвращено в конце функции).
Следовательно, инкремент происходит на каждом шаге цикла, т.е. 64 раза, вне всякой связи с входным
значением.

Преимущество этого кода в том, что он содержит только один условный переход (в конце цикла) вместо
двух (пропускающий инкремент \q{rt} и ещё одного в конце цикла).

И это может работать быстрее на современных CPU с предсказателем переходов: \myref{branch_predictors}.

\label{FATRET}
\myindex{x86!\Instructions!FATRET}
Последняя инструкция это \INS{REP RET} (опкод \TT{F3 C3}) 
которая также называется \INS{FATRET} в MSVC.
Это оптимизированная версия \RET, рекомендуемая AMD для вставки в конце функции, если \RET идет
сразу после условного перехода: 
\InSqBrackets{\AMDOptimization p.15}
\footnote{Больше об этом: \url{http://go.yurichev.com/17328}}.

\myparagraph{\Optimizing MSVC 2010}

\lstinputlisting[caption=\Optimizing MSVC 2010,style=customasmx86]{patterns/14_bitfields/4_popcnt/MSVC_2010_x64_Ox_RU.asm}

\myindex{x86!\Instructions!ROL}
Здесь используется инструкция \ROL вместо 
\SHL, которая на самом деле \q{rotate left} (прокручивать влево) 
вместо \q{shift left} (сдвиг влево),
но здесь, в этом примере, она работает так же как и  \TT{SHL}.

Об этих \q{прокручивающих} инструкциях больше читайте здесь: \myref{ROL_ROR}.

\Reg{8} здесь считает от 64 до 0. 
Это как бы инвертированная переменная $i$.

Вот таблица некоторых регистров в процессе исполнения:

\begin{center}
\begin{tabular}{ | l | l | }
\hline
\HeaderColor RDX & \HeaderColor R8 \\
\hline
0x0000000000000001 & 64 \\
\hline
0x0000000000000002 & 63 \\
\hline
0x0000000000000004 & 62 \\
\hline
0x0000000000000008 & 61 \\
\hline
... & ... \\
\hline
0x4000000000000000 & 2 \\
\hline
0x8000000000000000 & 1 \\
\hline
\end{tabular}
\end{center}

\myindex{x86!\Instructions!FATRET}
В конце видим инструкцию \INS{FATRET}, которая была описана здесь: \myref{FATRET}.

\myparagraph{\Optimizing MSVC 2012}

\lstinputlisting[caption=\Optimizing MSVC 2012,style=customasmx86]{patterns/14_bitfields/4_popcnt/MSVC_2012_x64_Ox_RU.asm}

\myindex{\CompilerAnomaly}
\label{MSVC2012_anomaly}
\Optimizing MSVC 2012 делает почти то же самое что и оптимизирующий MSVC 2010, но почему-то он генерирует 2 идентичных тела цикла и счетчик цикла теперь 32
вместо 64.
Честно говоря, нельзя сказать, почему. Какой-то трюк с оптимизацией? Может быть, телу цикла лучше быть
немного длиннее?

Так или иначе, такой код здесь уместен, чтобы показать, что результат компилятора
иногда может быть очень странный и нелогичный, но прекрасно работающий, конечно же.

}
\FR{\subsubsection{x64}
\label{subsec:popcnt}

Modifions légèrement l'exemple pour l'étendre à 64-bit:

\lstinputlisting[label=popcnt_x64_example,style=customc]{patterns/14_bitfields/4_popcnt/shifts64.c}

\myparagraph{GCC 4.8.2 \NonOptimizing}

Jusqu'ici, c'est facile.

\lstinputlisting[caption=GCC 4.8.2 \NonOptimizing,style=customasmx86]{patterns/14_bitfields/4_popcnt/shifts64_GCC_O0_FR.s}

\myparagraph{GCC 4.8.2 \Optimizing}

\lstinputlisting[caption=GCC 4.8.2 \Optimizing,numbers=left,label=shifts64_GCC_O3,style=customasmx86]{patterns/14_bitfields/4_popcnt/shifts64_GCC_O3_FR.s}

Ce code est plus concis, mais a une particularité.

% TODO: comment traduire commits ?
Dans tous les exemples que nous avons vu jusqu'ici, nous incrémentions la valeur
de \q{rt} après la comparaison d'un bit spécifique, mais le code ici incrémente \q{rt}
avant (ligne 6), écrivant la nouvelle valeur dans le registre \EDX.
Donc, si le dernier bit est à 1, l'instruction \CMOVNE\footnote{Conditional MOVe if Not Equal}
(qui est un synonyme pour \CMOVNZ\footnote{Conditional MOVe if Not Zero}) \emph{commits}
la nouvelle valeur de \q{rt} en déplaçant  \EDX (\q{valeur proposée de rt}) dans
\EAX (\q{rt courant} qui va être retourné à la fin).

C'est pourquoi l'incrémentation est effectuée à chaque étape de la boucle, i.e.,
64 fois, sans relation avec la valeur en entrée.

L'avantage de ce code est qu'il contient seulement un saut conditionnel (à la fin
de la boucle) au lieu de deux sauts (évitant l'incrément de la valeur de \q{rt} et
à la fin de la boucle).
Et cela doit s'exécuter plus vite sur les CPUs modernes avec des prédicteurs de branchement:
\myref{branch_predictors}.

\label{FATRET}
\myindex{x86!\Instructions!FATRET}
La dernière instruction est \INS{REP RET} (opcode \TT{F3 C3}) qui est aussi appelée
\INS{FATRET} par MSVC.
C'est en quelque sorte une version optimisée de \RET, qu'AMD recommande de mettre
en fin de fonction, si \RET se trouve juste après un saut conditionnel:
\InSqBrackets{\AMDOptimization p.15}
\footnote{Lire aussi à ce propos: \url{http://go.yurichev.com/17328}}.

\myparagraph{MSVC 2010 \Optimizing}

\lstinputlisting[caption=MSVC 2010 \Optimizing,style=customasmx86]{patterns/14_bitfields/4_popcnt/MSVC_2010_x64_Ox_FR.asm}

\myindex{x86!\Instructions!ROL}
Ici l'instruction \ROL est utilisée au lieu de \SHL, qui est en fait \q{rotate left /
pivoter à gauche} au lieu de \q{shift left / décaler à gauche}, mais dans cet exemple
elle fonctionne tout comme \TT{SHL}.

Vous pouvez en lire plus sur l'instruction de rotation ici: \myref{ROL_ROR}.

\Reg{8} ici est compté de 64 à 0.
C'est tout comme un $i$ inversé.

Voici une table de quelques registres pendant l'exécution:

\begin{center}
\begin{tabular}{ | l | l | }
\hline
\HeaderColor RDX & \HeaderColor R8 \\
\hline
0x0000000000000001 & 64 \\
\hline
0x0000000000000002 & 63 \\
\hline
0x0000000000000004 & 62 \\
\hline
0x0000000000000008 & 61 \\
\hline
... & ... \\
\hline
0x4000000000000000 & 2 \\
\hline
0x8000000000000000 & 1 \\
\hline
\end{tabular}
\end{center}

\myindex{x86!\Instructions!FATRET}
À la fin, nous voyons l'instruction \INS{FATRET}, qui a été expliquée ici: \myref{FATRET}.

\myparagraph{MSVC 2012 \Optimizing}

\lstinputlisting[caption=MSVC 2012 \Optimizing,style=customasmx86]{patterns/14_bitfields/4_popcnt/MSVC_2012_x64_Ox_FR.asm}

\myindex{\CompilerAnomaly}
\label{MSVC2012_anomaly}
MSVC 2012 \Optimizing fait presque le même job que MSVC 2010 \Optimizing, mais en
quelque sorte, il génère deux corps de boucles identiques et le nombre de boucles
est maintenant 32 au lieu de 64.

Pour être honnête, il n'est pas possible de dire pourquoi. Une ruse d'optimisation?
Peut-être est-il meilleur pour le corps de la boucle d'être légèrement plus long?

De toute façon, ce genre de code est pertinent ici pour montrer que parfois la sortie
du compilateur peut être vraiment bizarre et illogique, mais fonctionner parfaitement.

}
\PTBR{\input{patterns/04_scanf/3_checking_retval/x64_PTBR}}
\IT{\subsubsection{MSVC: x64}

\myindex{x86-64}

Poiche' qui lavoriamo con variabili di tipo \Tint{}, che sono sempre a 32-bit in x86-64, vediamo che viene usata la parte a 32-bit dei registri (con il prefisso \TT{E-}).
Lavorando invece con i puntatori, sono usate la parti a 64-bit dei registri (con il prefisso \TT{R-}).

% TODO translate
\lstinputlisting[caption=MSVC 2012 x64,style=customasmx86]{patterns/04_scanf/3_checking_retval/ex3_MSVC_x64_EN.asm}

}
\JPN{\subsection{x64}

\myindex{x86-64}

この話はx86-64では少し違っています。関数の引数(最初の4つまたは最初の6つ)
はレジスタに渡されます。つまり、\gls{callee}はレジスタからレジスタを読み込みます。

\subsubsection{MSVC}

\Optimizing MSVC:

\lstinputlisting[caption=\Optimizing MSVC 2012 x64,style=customasmx86]{patterns/05_passing_arguments/x64_MSVC_Ox_JPN.asm}

見てわかるように、コンパクトな関数 \ttf はすべての引数をレジスタから取ります。

ここでの \LEA 命令は加算に使用され、
明らかにコンパイラは \TT{ADD} よりも速いと考えました。
\myindex{x86!\Instructions!LEA}

\LEA は、第1および第3の \ttf 引数を準備するために \main 関数でも使用されます。
コンパイラは、 \MOV 命令を使用してレジスタに値をロードする通常の方法よりも速く動作すると判断する必要があります。

非最適化MSVCの出力を見てみましょう。

\lstinputlisting[caption=MSVC 2012 x64,style=customasmx86]{patterns/05_passing_arguments/x64_MSVC_IDA_JPN.asm}

レジスタからの3つの引数は何らかの理由でスタックに保存されるため、ややこしいことになっています。

\myindex{Shadow space}
\label{shadow_space}
これは "シャドウスペース"と呼ばれます。
\footnote{\href{http://go.yurichev.com/17256}{MSDN}}
すべてのWin64は、そこにある4つのレジスタ値をすべて保存することができます(必須ではありません)。
これは2つの理由で行われます。
1)入力引数にレジスタ全体(または4つのレジスタ)を割り当てるのはあまりにも贅沢なので、スタック経由でアクセスされます。 
2)デバッガはブレークで関数の引数をどこに見つけるか常に認識しています。
\footnote{\href{http://go.yurichev.com/17257}{MSDN}}

だから、大規模な関数の中には、実行中にそれらを使用したい場合、入力引数を \q{シャドウスペース}に保存することができますが、
私たちのような小さな関数ではそうでないかもしれません。

スタックに\q{シャドウスペース}を割り当てるのは\gls{caller}の責任です。

\subsubsection{GCC}

\Optimizing GCCはまあまあわかりやすいコードを生成します。

\lstinputlisting[caption=\Optimizing GCC 4.4.6 x64,style=customasmx86]{patterns/05_passing_arguments/x64_GCC_O3_JPN.s}

\NonOptimizing GCC:

\lstinputlisting[caption=GCC 4.4.6 x64,style=customasmx86]{patterns/05_passing_arguments/x64_GCC_JPN.s}

\myindex{Shadow space}

System V *NIX (\SysVABI)には\q{シャドースペース}の要件はありませんが、 \gls{callee}は
レジスタが不足している場合には引数をどこかに保存します。

\subsubsection{GCC: intの代わりのuint64\_t}

私たちの例は32ビットintで動作するため、32ビットのレジスタが使用されています(\TT{E-}が前に付いています)。 

64ビット値を使用するためには少し変更する必要があります。

\lstinputlisting[style=customc]{patterns/05_passing_arguments/ex64.c}

\lstinputlisting[caption=\Optimizing GCC 4.4.6 x64,style=customasmx86]{patterns/05_passing_arguments/ex64_GCC_O3_IDA_JPN.asm}

コードは同じですが、今回は\emph{フルサイズ}のレジスタ(\TT{R-}が前に付いています)が使用されています。
}


\EN{\subsection{ARM}

The ARM processor, just like in any other \q{pure} RISC processor lacks an instruction for division.
It also lacks a single instruction for multiplication by a 32-bit constant (recall that a 32-bit
constant cannot fit into a 32-bit opcode).

By taking advantage of this clever trick (or \emph{hack}), it is possible to do division using only three instructions: addition,
subtraction and bit shifts~(\myref{sec:bitfields}).

Here is an example that divides a 32-bit number by 10, from
\InSqBrackets{\ARMCookBook 3.3 Division by a Constant}.
The output consists of the quotient and the remainder.

\begin{lstlisting}[style=customasmARM]
; takes argument in a1
; returns quotient in a1, remainder in a2
; cycles could be saved if only divide or remainder is required
    SUB    a2, a1, #10             ; keep (x-10) for later
    SUB    a1, a1, a1, lsr #2
    ADD    a1, a1, a1, lsr #4
    ADD    a1, a1, a1, lsr #8
    ADD    a1, a1, a1, lsr #16
    MOV    a1, a1, lsr #3
    ADD    a3, a1, a1, asl #2
    SUBS   a2, a2, a3, asl #1      ; calc (x-10) - (x/10)*10
    ADDPL  a1, a1, #1              ; fix-up quotient
    ADDMI  a2, a2, #10             ; fix-up remainder
    MOV    pc, lr
\end{lstlisting}

\subsubsection{\OptimizingXcodeIV (\ARMMode)}

\begin{lstlisting}[style=customasmARM]
__text:00002C58 39 1E 08 E3 E3 18 43 E3  MOV    R1, 0x38E38E39
__text:00002C60 10 F1 50 E7              SMMUL  R0, R0, R1
__text:00002C64 C0 10 A0 E1              MOV    R1, R0,ASR#1
__text:00002C68 A0 0F 81 E0              ADD    R0, R1, R0,LSR#31
__text:00002C6C 1E FF 2F E1              BX     LR
\end{lstlisting}

This code is almost the same as the one generated by the optimizing MSVC and GCC.

Apparently, LLVM uses the same algorithm for generating constants.

\myindex{ARM!\Instructions!MOV}
\myindex{ARM!\Instructions!MOVT}

The observant reader may ask, how does \MOV writes a 32-bit value in a register, when this is not possible in ARM mode.

it is impossible indeed, but, as we see,
there are 8 bytes per instruction instead of the standard 4,
in fact, there are two instructions.

The first instruction loads \TT{0x8E39} into the low 16 bits of register and the second instruction is
\TT{MOVT}, it loads \TT{0x383E} into the high 16 bits of the register.
\IDA is fully aware of such sequences, and for the sake of compactness reduces them to one single \q{pseudo-instruction}.

\myindex{ARM!\Instructions!SMMUL}
The \TT{SMMUL} (\emph{Signed Most Significant Word Multiply}) 
instruction two multiplies numbers, treating them as signed numbers
and leaving the high 32-bit part of result in the \Reg{0} register,
dropping the low 32-bit part of the result.

\myindex{ARM!Optional operators!ASR}
The\TT{\q{MOV R1, R0,ASR\#1}} instruction is an arithmetic shift right by one bit.

\myindex{ARM!\Instructions!ADD}
\myindex{ARM!Data processing instructions}
\myindex{ARM!Optional operators!LSR}
\TT{\q{ADD R0, R1, R0,LSR\#31}} is $R0=R1 + R0>>31$

% FIXME какие именно инструкции? \myref{} ->
\label{shifts_in_ARM_mode}

There is no separate shifting instruction in ARM mode.
Instead, an instructions like 
(\MOV, \ADD, \SUB, \TT{RSB})\footnote{\DataProcessingInstructionsFootNote}
can have a suffix added, that says if the second operand must be shifted, and if yes, by what value and how.
\TT{ASR} stands for \emph{Arithmetic Shift Right}, \TT{LSR}---\emph{Logical Shift Right}.

\subsubsection{\OptimizingXcodeIV (\ThumbTwoMode)}

\begin{lstlisting}[style=customasmARM]
MOV             R1, 0x38E38E39
SMMUL.W         R0, R0, R1
ASRS            R1, R0, #1
ADD.W           R0, R1, R0,LSR#31
BX              LR
\end{lstlisting}

\myindex{ARM!\Instructions!ASRS}

There are separate instructions for shifting in Thumb mode, 
and one of them is used here---\TT{ASRS} (arithmetic shift right).

\subsubsection{\NonOptimizing Xcode 4.6.3 (LLVM) and Keil 6/2013}

\NonOptimizing LLVM
does not generate the code we saw before in this section, but instead inserts a call to the library function 
\emph{\_\_\_divsi3}.

What about Keil: it inserts a call to the library function \emph{\_\_aeabi\_idivmod} in all cases.
}
\RU{\subsubsection{ARM}

\myparagraph{ARM: \OptimizingKeilVI (\ThumbMode)}

\lstinputlisting[caption=\OptimizingKeilVI (\ThumbMode),style=customasmARM]{patterns/04_scanf/3_checking_retval/ex3_ARM_Keil_thumb_O3.asm}

\myindex{ARM!\Instructions!CMP}
\myindex{ARM!\Instructions!BEQ}
Здесь для нас есть новые инструкции: \CMP и \ac{BEQ}.

\CMP аналогична той что в x86: она отнимает один аргумент от второго и сохраняет флаги.

% TODO: в мануале ARM $op1 + NOT(op2) + 1$ вместо вычитания

\myindex{ARM!\Registers!Z}
\myindex{x86!\Instructions!JZ}
\ac{BEQ} совершает переход по другому адресу, 
если операнды при сравнении были равны, 
либо если результат последнего вычисления был 0, либо если флаг Z равен 1.
То же что и \JZ в x86.

Всё остальное просто: исполнение разветвляется на две ветки, затем они сходятся там, 
где в \Reg{0} записывается 0 как возвращаемое из функции значение и происходит выход из функции.

\myparagraph{ARM64}

\lstinputlisting[caption=\NonOptimizing GCC 4.9.1 ARM64,numbers=left,style=customasmARM]{patterns/04_scanf/3_checking_retval/ARM64_GCC491_O0_RU.s}

\myindex{ARM!\Instructions!CMP}
\myindex{ARM!\Instructions!Bcc}

Исполнение здесь разветвляется, используя пару инструкций \INS{CMP}/\INS{BNE} (Branch if Not Equal: переход если не равно).

}
\IT{\subsubsection{ARM}

\myparagraph{ARM: \OptimizingKeilVI (\ThumbMode)}

\lstinputlisting[caption=\OptimizingKeilVI (\ThumbMode),style=customasmARM]{patterns/04_scanf/3_checking_retval/ex3_ARM_Keil_thumb_O3.asm}

\myindex{ARM!\Instructions!CMP}
\myindex{ARM!\Instructions!BEQ}

Le due nuove istruzioni qui sono \CMP e \ac{BEQ}.

\CMP e' analoga all'istruzione omonima in x86, sottrae uno degli argomenti dall'altro e aggiorna il conditional flags (se necessario).
% TODO: в мануале ARM $op1 + NOT(op2) + 1$ вместо вычитания

\myindex{ARM!\Registers!Z}
\myindex{x86!\Instructions!JZ}
\ac{BEQ} salta ad un altro indirizzo se gli operandi sono uguali, o se il risultato dell'ultima operazione era 0, oppure ancora se il flag Z e' 1.
Si comporta come \JZ in x86.

Tutto il resto e' semplice: il flusso di esecuzione si divide in due rami, e successivamente i due rami convergono al punto in cui 0 viene scritto in 
\Reg{0} come valore di ritorno di una funzione, infine la funzione termina.

\myparagraph{ARM64}

\lstinputlisting[caption=\NonOptimizing GCC 4.9.1 ARM64,numbers=left,style=customasmARM]{patterns/04_scanf/3_checking_retval/ARM64_GCC491_O0_EN.s}

\myindex{ARM!\Instructions!CMP}
\myindex{ARM!\Instructions!Bcc}
Il flusso di codice in questo caso si divide con l'uso della coppia di istruzioni \INS{CMP}/\INS{BNE} (Branch if Not Equal).

}
\FR{\subsubsection{ARM}

\myparagraph{\OptimizingKeilVI (\ThumbMode)}

\lstinputlisting[caption=\OptimizingKeilVI (\ThumbMode),style=customasmARM]{patterns/15_structs/4_packing/packing_Keil_thumb.asm}

Rappelons-nous que c'est une structure qui est passée ici et non pas un pointeur vers une structure. Comme 
les 4 premiers arguments d'une fonction sont passés dans les registres sur les processeurs ARM, les champs 
de la structure sont passés dans les registres \TT{R0-R3}.

\myindex{ARM!\Instructions!LDRB}
\myindex{x86!\Instructions!MOVSX}
\TT{LDRB} charge un octet présent en mémoire et l'étend sur 32bits en prenant en compte son signe. Cette 
opération est similaire à celle effectuée par \MOVSX dans les architectures x86. Elle est utilisée ici pour 
charger les champs $a$ et $c$ de la structure.

\myindex{Epilogue de fonction}

Un autre détail que nous remarquons aisément est que la fonction ne s'achève pas sur un épilogue qui lui est 
propre. A la place, il y a un saut vers l'épilogue d'une autre fonction! Qui plus est celui d'une fonction 
très différente sans aucun lien avec la nôtre. Cependant elle possède exactement le même épilogue, 
probablement parce qu'elle accepte utilise elle aussi 5 variables locales ($5*4=0x14$).

De plus elle est située à une adresse proche.

En réalité, peut importe l'épilogue qui est utilisé du moment que le fonctionnement est celui attendu.

Il semble donc que le compilateur Keil décide de réutiliser à des fins d'économie un fragment d'une autre 
fonction. Notre épilogue aurait nécessité 4 octets. L'instruction de saut n'en utilise que 2.

\myparagraph{ARM + \OptimizingXcodeIV (\ThumbTwoMode)}

\lstinputlisting[caption=\OptimizingXcodeIV (\ThumbTwoMode),style=customasmARM]{patterns/15_structs/4_packing/packing_Xcode_thumb_EN.asm}

\myindex{ARM!\Instructions!SXTB}
\myindex{x86!\Instructions!MOVSX}
\TT{SXTB} (\IT{Signed Extend Byte}) est similaire à \MOVSX pour les architectures x86.
Pour le reste---c'est identique.
}
\JA{\subsubsection{ARM}

\myparagraph{ARM: \OptimizingKeilVI (\ThumbMode)}

\lstinputlisting[caption=\OptimizingKeilVI (\ThumbMode),style=customasmARM]{patterns/04_scanf/3_checking_retval/ex3_ARM_Keil_thumb_O3.asm}

\myindex{ARM!\Instructions!CMP}
\myindex{ARM!\Instructions!BEQ}

ここでの新しい命令は \CMP と\ac{BEQ}です。

\CMP は同じ名前のx86命令に似ていますが、他の引数から引数の1つを減算し、必要に応じて条件フラグを更新します。
% TODO: в мануале ARM $op1 + NOT(op2) + 1$ вместо вычитания

\myindex{ARM!\Registers!Z}
\myindex{x86!\Instructions!JZ}
オペランドが互いに等しい場合、
または最後の計算の結果が0の場合、またはZフラグが1の場合、
\ac{BEQ}は別のアドレスにジャンプします。
これはx86では \JZ として動作します。

それ以外はすべてシンプルです。実行フローが2つの分岐に分岐した後、
関数の戻り値として0が\Reg{0}に書き込まれた時点で分岐が収束し、関数が終了します。

\myparagraph{ARM64}

\lstinputlisting[caption=\NonOptimizing GCC 4.9.1 ARM64,numbers=left,style=customasmARM]{patterns/04_scanf/3_checking_retval/ARM64_GCC491_O0_JA.s}

\myindex{ARM!\Instructions!CMP}
\myindex{ARM!\Instructions!Bcc}
この場合のコードフローは、\INS{CMP}/\INS{BNE}(Branch if Not Equal)命令のペアを使用して分岐します。
}

\subsubsection{MIPS}

\lstinputlisting[caption=\Optimizing GCC 4.4.5 (IDA),style=customasmMIPS]{patterns/04_scanf/3_checking_retval/MIPS_O3_IDA.lst}

\myindex{MIPS!\Instructions!BEQ}

\EN{\subsubsection{MIPS}

\lstinputlisting[caption=\Optimizing GCC 4.4.5 (IDA),style=customasmMIPS]{patterns/10_strings/1_strlen/MIPS_O3_IDA_EN.lst}

\myindex{MIPS!\Instructions!NOR}
\myindex{MIPS!\Pseudoinstructions!NOT}

MIPS lacks a \NOT instruction, but has \NOR which is \TT{OR~+~NOT} operation.

This operation is widely used in digital electronics\footnote{NOR is called \q{universal gate}}.
\myindex{Apollo Guidance Computer}
For example, the Apollo Guidance Computer used in the Apollo program, 
was built by only using 5600 NOR gates:
[Jens Eickhoff, \emph{Onboard Computers, Onboard Software and Satellite Operations: An Introduction}, (2011)].
But NOR element isn't very popular in computer programming.

So, the NOT operation is implemented here as \TT{NOR~DST,~\$ZERO,~SRC}.

From fundamentals \myref{sec:signednumbers} we know that bitwise inverting a signed number is the same 
as changing its sign and subtracting 1 from the result.

So what \NOT does here is to take the value of $str$ and transform it into $-str-1$.
The addition operation that follows prepares result.

}
\RU{\subsubsection{MIPS}

\lstinputlisting[caption=\Optimizing GCC 4.4.5 (IDA),style=customasmMIPS]{patterns/08_switch/1_few/MIPS_O3_IDA_RU.lst}

\myindex{MIPS!\Instructions!JR}

Функция всегда заканчивается вызовом \puts, так что здесь мы видим переход на \puts (\INS{JR}: \q{Jump Register})
вместо перехода с сохранением \ac{RA} (\q{jump and link}).

Мы говорили об этом ранее: \myref{JMP_instead_of_RET}.

\myindex{MIPS!Load delay slot}
Мы также часто видим NOP-инструкции после \INS{LW}.
Это \q{load delay slot}: ещё один \emph{delay slot} в MIPS.
\myindex{MIPS!\Instructions!LW}
Инструкция после \INS{LW} может исполняться в тот момент, когда \INS{LW} загружает значение из памяти.

Впрочем, следующая инструкция не должна использовать результат \INS{LW}.

Современные MIPS-процессоры ждут, если следующая инструкция использует результат \INS{LW}, так что всё это уже
устарело, но GCC всё еще добавляет NOP-ы для более старых процессоров.

Вообще, это можно игнорировать.

}
\IT{\scanf restituisce il risultato del suo lavoro nel registro \$V0. Cio' viene controllato all'indirizzo 0x004006E4
confrontando il valore in \$V0 con quello in \$V1 (1 era stato memorizzato in \$V1 precedentemente, a 0x004006DC).
\INS{BEQ} sta per \q{Branch Equal}.
Se i due valori sono uguali (cioe' \scanf e' terminata con successo), l'esecuzione salta all'indirizzo 0x0040070C.

}
\JA{\scanf は、その作業の結果をレジスタ \$V0 に返します。 アドレス0x004006E4は、 
\$V0 の値と \$V1 (1は \$V1 以前の0x004006DCに格納されています)と比較することでチェックされます。 
\INS{BEQ}は\q{Branch Equal}の略です。 
2つの値が等しい場合(すなわち、成功した場合)、アドレス0x0040070Cにジャンプします。
}



\subsubsection{\Exercise}

\myindex{x86!\Instructions!Jcc}
\myindex{ARM!\Instructions!Bcc}
Come possiamo vedere, le istruzioni \INS{JNE}/\INS{JNZ} possono essere scambiate con \INS{JE}/\INS{JZ} e viceversa.
(lo stesso vale per \INS{BNE} e \INS{BEQ}).
Ma se cio' avviene i blocchi base devono anch'essi essere scambiati. Provate a farlo in qualche esempio.

