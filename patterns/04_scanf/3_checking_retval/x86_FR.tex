\subsubsection{MSVC: x86}

Voici ce que nous obtenons dans la sortie assembleur (MSVC 2010):

\lstinputlisting[style=customasmx86]{patterns/04_scanf/3_checking_retval/ex3_MSVC_x86.asm}

\myindex{x86!\Registers!EAX}
La fonction \glslink{caller}{appelante} (\main) à besoin du résultat de la fonction
\glslink{callee}{appelée}, donc la fonction \glslink{callee}{appelée} le renvoie
dans la registre \EAX.

\myindex{x86!\Instructions!CMP}
Nous le vérifions avec l'aide de l'instruction \TT{CMP EAX, 1} (\emph{CoMPare}).
En d'autres mots, nous comparons la valeur dans le registre \EAX avec 1.

\myindex{x86!\Instructions!JNE}
Une instruction de saut conditionnelle \JNE suit l'instruction \CMP. \JNE signifie
\emph{Jump if Not Equal} (saut si non égal).

Donc, si la valeur dans le registre \EAX n'est pas égale à 1, le \ac{CPU} va poursuivre
l'exécution à l'adresse mentionnée dans l'opérande \JNE, dans notre cas \TT{\$LN2@main}.
Passez le contrôle à cette adresse résulte en l'exécution par le \ac{CPU} de
\printf avec l'argument \TT{What you entered? Huh?}.
Mais si tout est bon, le saut conditionnel n'est pas pris, et un autre appel à \printf
est exécuté, avec deux arguments:\\
\TT{'You entered \%d...'} et la valeur de \TT{x}.

\myindex{x86!\Instructions!XOR}
\myindex{\CLanguageElements!return}
Puisque dans ce cas le second \printf n'a pas été exécuté, il y a un \JMP qui le précède (saut inconditionnel).
Il passe le contrôle au point après le second \printf et juste avant l'instruction \TT{XOR EAX, EAX}, qui implémente \TT{return 0}.

% FIXME internal \ref{} to x86 flags instead of wikipedia
\myindex{x86!\Registers!\Flags}
Donc, on peut dire que comparer une valeur avec une autre est \emph{usuellement} implémenté
par la paire d'instructions \CMP/\Jcc, où \emph{cc} est un \emph{code de condition}.
\CMP compare deux valeurs et met les flags\footnote{flags x86, voir aussi: \href{http://go.yurichev.com/17120}{Wikipédia}.}
du processeur.
\Jcc vérifie ces flags et décide de passer le contrôle à l'adresse spécifiée ou non.

\myindex{x86!\Instructions!CMP}
\myindex{x86!\Instructions!SUB}
\myindex{x86!\Instructions!JNE}
\myindex{x86!\Registers!ZF}
\label{CMPandSUB} 
Cela peut sembler paradoxal, mais l'instruction \CMP est en fait un \SUB (soustraction).
Toutes les instructions arithmétiques mettent les flags du processeur, pas seulement \CMP.
Si nous comparons 1 et 1, $1-1$ donne 0 donc le flag \ZF va être mis (signifiant
que le dernier résultat est 0).
Dans aucune autre circonstance \ZF ne sera mis, à l'exception que les opérandes
ne soient égaux.
\JNE vérifie seulement le flag \ZF et saute seulement si il n'est pas mis. \JNE
est un synonyme pour \JNZ (\emph{Jump if Not Zero} (saut si non zéro)).
L'assembleur génère le même opcode pour les instructions \JNE et \JNZ.
Donc, l'instruction \CMP peut être remplacée par une instruction \SUB et presque
tout ira bien, à la différence que \SUB altère la valeur du premier opérande.
\CMP est un \emph{SUB sans sauver le résultat, mais modifiant les flags}.

\subsubsection{MSVC: x86: IDA}

\myindex{IDA}
C'est le moment de lancer \IDA et d'essayer de faire quelque chose avec.
À propos, pour les débutants, c'est une bonne idée d'utiliser l'option \TT{/MD}
de MSVC, qui signifie que toutes les fonctions standards ne vont pas être liées
avec le fichier exécutable, mais vont à la place être importées depuis le fichier
\TT{MSVCR*.DLL}.
Ainsi il est plus facile de voir quelles fonctions standards sont utilisées et où.

En analysant du code dans \IDA, il est très utile de laisser des notes pour soi-même
(et les autres).
En la circonstance, analysons cet exemple, nous voyons que \TT{JNZ} sera déclenché
en cas d'erreur.
Donc il est possible de déplacer le curseur sur le label, de presser \q{n} et de
lui donner le nom \q{error}.
Créons un autre label---dans \q{exit}.
Voici mon résultat:

\lstinputlisting[style=customasmx86]{patterns/04_scanf/3_checking_retval/ex3.lst}

Maintenant, il est légèrement plus facile de comprendre le code.
Toutefois, ce n'est pas une bonne idée de commenter chaque instruction.

% FIXME draw button?
Vous pouvez aussi cacher (replier) des parties d'une fonction dans \IDA.
Pour faire cela, marquez le bloc, puis appuyez sur \q{--} sur le pavé numérique et
entrez le texte qui doit être affiché à la place.

Cachons deux blocs et donnons leurs un nom:

\lstinputlisting[style=customasmx86]{patterns/04_scanf/3_checking_retval/ex3_2.lst}

% FIXME draw button?
Pour étendre les parties de code précédemment cachées. utilisez \q{+} sur le
pavé numérique.

\clearpage
En appuyant sur \q{space}, nous voyons comment \IDA représente une fonction sous
forme de graphe:

\begin{figure}[H]
\centering
\myincludegraphics{patterns/04_scanf/3_checking_retval/IDA.png}
\caption{IDA en mode graphe}
\label{fig:ex3_IDA_1}
\end{figure}

Il y a deux flèches après chaque saut conditionnel: une verte et une rouge.
La flèche verte pointe vers le bloc qui sera exécuté si le saut est déclenché,
et la rouge sinon.

\clearpage
Il est possible de replier des nœuds dans ce mode et de leurs donner aussi un nom (\q{group nodes}).
Essayons avec 3 blocs:

\begin{figure}[H]
\centering
\myincludegraphics{patterns/04_scanf/3_checking_retval/IDA2.png}
\caption{IDA en mode graphe avec 3 nœuds repliés}
\label{fig:ex3_IDA_2}
\end{figure}

C'est très pratique.
On peut dire qu'une part importante du travail des rétro-ingénieurs (et de tout
autre chercheur également) est de réduire la quantité d'information avec laquelle
travailler.

\clearpage
\myparagraph{MSVC + \olly}
\myindex{\olly}

2 paires de mots 32-bit sont marquées en rouge sur la pile.
Chaque paire est un double au format IEEE 754 et est passée depuis \main.

Nous voyons comment le premier \FLD charge une valeur ($1.2$) depuis la pile et la
stocke dans \ST{0}:

\begin{figure}[H]
\centering
\myincludegraphics{patterns/12_FPU/1_simple/olly1.png}
\caption{\olly: le premier \FLD a été exécuté}
\label{fig:FPU_simple_olly_1}
\end{figure}

À cause des inévitables erreurs de conversion depuis un nombre flottant 64-bit au
format IEEE 754 vers 80-bit (utilisé en interne par le FPU), ici nous voyons 1.199\ldots,
qui est proche de 1.2.

\EIP pointe maintenant sur l'instruction suivante (\FDIV), qui charge un double
(une constante) depuis la mémoire.
Par commodité, \olly affiche sa valeur: 3.14

\clearpage
Continuons l'exécution pas à pas.
\FDIV a été exécuté, maintenant \ST{0} contient 0.382\ldots
(\gls{quotient}):

\begin{figure}[H]
\centering
\myincludegraphics{patterns/12_FPU/1_simple/olly2.png}
\caption{\olly: \FDIV a été exécuté}
\label{fig:FPU_simple_olly_2}
\end{figure}

\clearpage
Troisième étape: le \FLD suivant a été exécuté, chargeant 3.4 dans \ST{0} (ici nous
voyons la valeur approximative 3.39999\ldots):

\begin{figure}[H]
\centering
\myincludegraphics{patterns/12_FPU/1_simple/olly3.png}
\caption{\olly: le second \FLD a été exécuté}
\label{fig:FPU_simple_olly_3}
\end{figure}

En même temps, le \gls{quotient} \emph{est poussé} dans \ST{1}.
Exactement maintenant, \EIP pointe sur la prochaine instruction: \FMUL.
Ceci charge la constante 4.1 depuis la mémoire, ce que montre \olly.

\clearpage
Suivante: \FMUL a été exécutée, donc maintenant le \glslink{product}{produit} est dans \ST{0}:

\begin{figure}[H]
\centering
\myincludegraphics{patterns/12_FPU/1_simple/olly4.png}
\caption{\olly: \FMUL a été exécuté}
\label{fig:FPU_simple_olly_4}
\end{figure}

\clearpage
Suivante: \FADDP a été exécutée, maintenant le résultat de l'addition est dans \ST{0},
et \ST{1} est vidé.

\begin{figure}[H]
\centering
\myincludegraphics{patterns/12_FPU/1_simple/olly5.png}
\caption{\olly: \FADDP a été exécuté}
\label{fig:FPU_simple_olly_5}
\end{figure}

Le résultat est laissé dans \ST{0}, car la fonction renvoie son résultat dans \ST{0}.

\main prend cette valeur depuis le registre plus loin.

Nous voyons quelque chose d'inhabituel: la valeur 13.93\ldots se trouve maintenant
dans \ST{7}.
Pourquoi?

\label{FPU_is_rather_circular_buffer}

Comme nous l'avons lu il y a quelque temps dans ce livre, les registres \ac{FPU} sont
une pile: \myref{FPU_is_stack}.
Mais ceci est une simplification.

Imaginez si cela était implémenté \emph{en hardware} comme cela est décrit, alors
tout le contenu des 7 registres devrait être déplacé (ou copié) dans les registres
adjacents lors d'un push ou d'un pop, et ceci nécessite beaucoup de travail.

En réalité, le \ac{FPU} a seulement 8 registres et un pointeur (appelé \GTT{TOP})
qui contient un numéro de registre, qui est le \q{haut de la pile} courant.

Lorsqu'une valeur est poussée sur la pile, \GTT{TOP} est déplacé sur le registre
disponible suivant, et une valeur est écrite dans ce registre.

La procédure est inversée si la valeur est lue, toutefois, le registre qui a été
libéré n'est pas vidé (il serait possible de le vider, mais ceci nécessite plus de
travail qui peut dégrader les performances).
Donc, c'est ce que nous voyons ici.

On peut dire que \FADDP sauve la somme sur la pile, et y supprime un élément.

Mais en fait, cette instruction sauve la somme et ensuite décale \GTT{TOP}.

Plus précisément, les registres du  \ac{FPU} sont un tampon circulaire.


\clearpage
\subsubsection{MSVC: x86 + Hiew}
\myindex{Hiew}

Cela peut également être utilisé comme un exemple simple de modification de fichier
exécutable.
Nous pouvons essayer de modifier l'exécutable de telle sorte que le programme va
toujours afficher notre entrée, quelle quelle soit.

En supposant que l'exécutable est compilé avec la bibliothèque externe \TT{MSVCR*.DLL}
(i.e., avec l'option \TT{/MD}) \footnote{c'est aussi appelé \q{dynamic linking}},
nous voyons la fonction \main au début de la section \TT{.text}.
Ouvrons l'exécutable dans Hiew et cherchons le début de la section \TT{.text} (Enter,
F8, F6, Enter, Enter).

Nous pouvons voir cela:

\begin{figure}[H]
\centering
\myincludegraphics{patterns/04_scanf/3_checking_retval/hiew_1.png}
\caption{Hiew: fonction \main}
\label{fig:scanf_ex3_hiew_1}
\end{figure}

Hiew trouve les chaîne \ac{ASCIIZ} et les affiche, comme il le fait avec le nom
des fonctions importées.

\clearpage
Déplacez le curseur à l'adresse \TT{.00401027} (où se trouve l'instruction \TT{JNZ},
que l'on doit sauter), appuyez sur F3, et ensuite tapez \q{9090} (qui signifie deux
\ac{NOP}s):

\begin{figure}[H]
\centering
\myincludegraphics{patterns/04_scanf/3_checking_retval/hiew_2.png}
\caption{Hiew: remplacement de \TT{JNZ} par deux \ac{NOP}s}
\label{fig:scanf_ex3_hiew_2}
\end{figure}

Appuyez sur F9 (update). Maintenant, l'exécutable est sauvé sur le disque. Il va
se comporter comme nous le voulions.

Deux \ac{NOP}s ne constitue probablement pas l'approche la plus esthétique.
Une autre façon de modifier cette instruction est d'écrire simplement 0 dans le
second octet de l'opcode ((\gls{jump offset}), donc ce \TT{JNZ} va toujours sauter
à l'instruction suivante.

Nous pouvons également faire le contraire: remplacer le premier octet avec \TT{EB}
sans modifier le second octet (\gls{jump offset}).
Nous obtiendrions un saut inconditionnel qui est toujours déclenché.
Dans ce cas le message d'erreur sera affiché à chaque fois, peu importe l'entrée.

