\subsubsection{MSVC: x86}

\lstinputlisting[style=customasmx86]{patterns/04_scanf/2_global/ex2_MSVC.asm}

この場合、\emph{x}変数は\TT{\_DATA}セグメントに定義され、ローカルスタックにはメモリは割り当てられません。 スタックからではなく、直接アクセスされます。 
初期化されていないグローバル変数は、実行可能ファイルにスペースを入れません
(なぜ、最初に変数をゼロに設定する必要があるのでしょうか?)。
しかし、誰かが自分のアドレスにアクセスすると、\ac{OS}は0で初期化されたブロック\footnote{これが\ac{VM}の動作です}を割り当てます。

変数に明示的に値を割り当てましょう:

\lstinputlisting[style=customc]{patterns/04_scanf/2_global/default_value_EN.c}

以下を得ます。

\begin{lstlisting}[style=customasmx86]
_DATA	SEGMENT
_x	DD	0aH

...
\end{lstlisting}

ここでは、この変数のDWORDタイプの値\TT{0xA}(DDはDWORD = 32ビットを表します)が表示されます。

\IDA にコンパイルされた.exeを開くと、\TT{\_DATA}セグメントの先頭に\emph{x}変数が配置されていて、
その後にテキスト文字列が表示されます。

\emph{x}の値が設定されていない前の例のコンパイル済み.exeを \IDA で開くと、次のように表示されます。

\lstinputlisting[caption=\IDA,style=customasmx86]{patterns/04_scanf/2_global/IDA.lst}

\TT{\_x}に\TT{?}がマークされていると、残りの変数は初期化する必要はありません。 
これは、メモリに.exeをロードした後、これらすべての変数のための領域が割り当てられ、
0で満たされる\InSqBrackets{\CNineNineStd 6.7.8p10}ことを意味します。 
しかし、.exeファイルでは、これらの初期化されていない変数は何も占有しません。 
これは、例えば、大きな配列の場合に便利です。

\EN{\clearpage
\subsubsection{MSVC: x86 + \olly}
\myindex{\olly}

Things are even simpler here:

\begin{figure}[H]
\centering
\myincludegraphics{patterns/04_scanf/2_global/ex2_olly_1.png}
\caption{\olly: after \scanf execution}
\label{fig:scanf_ex2_olly_1}
\end{figure}

The variable is located in the data segment.
After the \PUSH instruction (pushing the address of $x$) gets executed, 
the address appears in the stack window. Right-click on that row and select \q{Follow in dump}.
The variable will appear in the memory window on the left.
After we have entered 123 in the console, 
\TT{0x7B} appears in the memory window (see the highlighted screenshot regions).

But why is the first byte \TT{7B}?
Thinking logically, \TT{00 00 00 7B} must be there.
The cause for this is referred as  \gls{endianness}, and x86 uses \emph{little-endian}.
This implies that the lowest byte is written first, and the highest written last.
Read more about it at: \myref{sec:endianness}.
Back to the example, the 32-bit value is loaded from this memory address into \EAX and passed to \printf.

The memory address of $x$ is \TT{0x00C53394}.

\clearpage
In \olly we can review the process memory map (Alt-M)
and we can see that this address is inside the \TT{.data} PE-segment of our program:

\label{olly_memory_map_example}
\begin{figure}[H]
\centering
\myincludegraphics{patterns/04_scanf/2_global/ex2_olly_2.png}
\caption{\olly: process memory map}
\label{fig:scanf_ex2_olly_2}
\end{figure}

}
\RU{\clearpage
\subsubsection{MSVC: x86 + \olly}
\myindex{\olly}

Тут даже проще:

\begin{figure}[H]
\centering
\myincludegraphics{patterns/04_scanf/2_global/ex2_olly_1.png}
\caption{\olly: после исполнения \scanf}
\label{fig:scanf_ex2_olly_1}
\end{figure}

Переменная хранится в сегменте данных.
Кстати, после исполнения инструкции \PUSH (заталкивающей адрес $x$) адрес появится в стеке, 
и на этом элементе можно нажать правой кнопкой, выбрать \q{Follow in dump}.
И в окне памяти слева появится эта переменная.

После того как в консоли введем 123, здесь появится \TT{0x7B}.

Почему самый первый байт это \TT{7B}?
По логике вещей, здесь должно было бы быть \TT{00 00 00 7B}.
Это называется \gls{endianness}, и в x86 принят формат \emph{little-endian}.
Это означает, что в начале записывается самый младший байт, а заканчивается самым старшим байтом.
Больше об этом: \myref{sec:endianness}.

Позже из этого места в памяти 32-битное значение загружается в \EAX и передается в \printf.

Адрес переменной $x$ в памяти \TT{0x00C53394}.

\clearpage
\label{olly_memory_map_example}

В \olly{} мы можем посмотреть карту памяти процесса (Alt-M) и увидим, что этот адрес
внутри PE-сегмента \TT{.data} нашей программы:

\begin{figure}[H]
\centering
\myincludegraphics{patterns/04_scanf/2_global/ex2_olly_2.png}
\caption{\olly: карта памяти процесса}
\label{fig:scanf_ex2_olly_2}
\end{figure}
}
\IT{\clearpage
\subsubsection{MSVC: x86 + \olly}
\myindex{\olly}

Il quadro qui è ancora più semplice:

\begin{figure}[H]
\centering
\myincludegraphics{patterns/04_scanf/2_global/ex2_olly_1.png}
\caption{\olly: after \scanf execution}
\label{fig:scanf_ex2_olly_1}
\end{figure}

La variabile è collocata nel data segment.
Dopo che l'istruzione \PUSH (che fa il push dell'indirizzo di $x$) viene eseguita, 
l'indirizzo appare nella finestra dello stack. Facciamo click destro su quella riga e selezioniamo \q{Follow in dump}.
La variabile apparirà nella finestra di memoria a sinistra.
Dopo aver inserito il valore 123 in console, 
\TT{0x7B} apparirà nella finestra della memoria (vedere regioni evidenziate nello screenshot).

Ma perchè il primo byte è \TT{7B}?
A rigor di logica, dovremmo trovare \TT{00 00 00 7B}.
La causa per cui troviamo invece \TT{7B} è detta \gls{endianness}, e x86 usa la convenzione \emph{little-endian}.
Ciò significa che il byte piu basso è scritto per primo, e quello più alto per ultimo.
Maggiori informazioni sono disponibili nella sezione: \myref{sec:endianness}.
Tornando all'esempio, il valore a 32-bit è caricato da questo indirizzo di memoria in \EAX e passato a \printf.

L'indirizzo in memoria di $x$ è \TT{0x00C53394}.

\clearpage
\label{olly_memory_map_example}

In \olly possiamo osservare la mappa di memoria di un processo  (process memory map, Alt-M)
e notare che questo indirizzo è dentro il segmento PE \TT{.data} del nostro programma:

\begin{figure}[H]
\centering
\myincludegraphics{patterns/04_scanf/2_global/ex2_olly_2.png}
\caption{\olly: process memory map}
\label{fig:scanf_ex2_olly_2}
\end{figure}

}
\DE{\clearpage
\subsubsection{MSVC: x86 + \olly}
\myindex{\olly}

Hier sehen die Dinge noch einfacher aus:

\begin{figure}[H]
\centering
\myincludegraphics{patterns/04_scanf/2_global/ex2_olly_1.png}
\caption{\olly: nach Ausführung von \scanf}
\label{fig:scanf_ex2_olly_1}
\end{figure}

Die Variable befindet sich im Datensegment.
Nachdem der \PUSH Befehl (der die Adresse von $x$ speichert) ausgeführt worden ist,
erscheint die Adresse im Stackfenster. Wir machen einen Rechtsklick auf die Zeile und wählen \q{Follow in dump}.
Die Variable erscheint nun im Speicherfenster auf der linken Seite. 
Nachdem wir in der Konsole 123 eingegeben haben, erscheint \TT{0x7B} im Speicherfenster (siehe markiertes Feld im
Screenshot).

Warum ist das erste Byte \TT{7B}?
Logisch gedacht müsste dort \TT{00 00 00 7B} sein. 
Der Grund dafür ist die sogenannte \glslink{endianness}{Endianess} und x86 verwendet \emph{litte Endian}. 
Dies bedeutet, dass das niederwertigste Byte zuerst und das höchstwertigste zuletzt geschrieben werden.
Für mehr Informationen dazu siehe: \myref{sec:endianness}.
Zurück zu Beispiel: der 32-Bit-Wert wird von dieser Speicheradresse nach \EAX geladen und an \printf übergeben. 

Die Speicheradresse von $x$ ist \TT{0x00C53394}.

\clearpage
In \olly können wir die Speicherzuordnung des Prozesses nachvollziehen (Alt-M) und wir erkennen, dass sich diese Adresse
innerhalb des \TT{.data} PE-Segments von unserem Programm befindet:

\label{olly_memory_map_example}
\begin{figure}[H]
\centering
\myincludegraphics{patterns/04_scanf/2_global/ex2_olly_2.png}
\caption{\olly: Speicherzuordnung}
\label{fig:scanf_ex2_olly_2}
\end{figure}

}
\FR{\clearpage
\subsubsection{MSVC: x86 + \olly}
\myindex{\olly}

Les choses sont encore plus simple ici:

\begin{figure}[H]
\centering
\myincludegraphics{patterns/04_scanf/2_global/ex2_olly_1.png}
\caption{\olly: après l'exécution de \scanf}
\label{fig:scanf_ex2_olly_1}
\end{figure}

La variable se trouve dans le segment de données.
Après que l'instruction \PUSH (pousser l'adresse de $x$) ait été exécutée,
l'adresse apparaît dans la fenêtre de la pile. Cliquer droit sur cette ligne
et choisir \q{Follow in dump}. % TODO olly French ?
La variable va apparaître dans la fenêtre de la mémoire sur la gauche.
Après que nous ayons entré 123 dans la console, \TT{0x7B} apparaît dans la fenêtre
de la mémoire (voir les régions surlignées dans la copie d'écran).

Mais pourquoi est-ce que le premier octet est \TT{7B}?
Logiquement, Il devrait y avoir \TT{00 00 00 7B} ici.
La cause de ceci est référé comme \gls{endianness}, et x86 utilise \emph{little-endian}.
Cela implique que l'octet le plus faible poids est écrit en premier, et le plus fort
en dernier.
Voir à ce propos: \myref{sec:endianness}.
Revenons à l'exemple, la valeur 32-bit est chargée depuis son adresse mémoire
dans \EAX et passée à \printf.

L'adresse mémoire de $x$ est \TT{0x00C53394}.

\clearpage
Dans \olly nous pouvons examiner l'espace mémoire du processus  (Alt-M) et nous
pouvons voir que cette adresse se trouve dans le segment PE \TT{.data} de notre
programme:

\label{olly_memory_map_example}
\begin{figure}[H]
\centering
\myincludegraphics{patterns/04_scanf/2_global/ex2_olly_2.png}
\caption{\olly: espace mémoire du processus}
\label{fig:scanf_ex2_olly_2}
\end{figure}

}
\JA{\clearpage
\subsubsection{MSVC: x86 + \olly}
\myindex{\olly}

ここではさらに単純です。

\begin{figure}[H]
\centering
\myincludegraphics{patterns/04_scanf/2_global/ex2_olly_1.png}
\caption{\olly: after \scanf execution}
\label{fig:scanf_ex2_olly_1}
\end{figure}

変数はデータセグメントにあります。 
\PUSH 命令($x$のアドレスを押す)が実行されると、
アドレスがスタックウィンドウに表示されます。 その行を右クリックし、\q{ダンプに従う}を選択します。 
変数は、左側のメモリウィンドウに表示されます。 
コンソールに123を入力すると、
メモリウィンドウに\TT{0x7B}が表示されます(ハイライトされたスクリーンショット領域を参照)。

しかし、最初のバイトはなぜ7Bでしょうか? 
論理的に考えると、\TT{00 00 00 7B}のはずです。 
この原因は\gls{endianness}と呼ばれるもので、x86はリトルエンディアンを使用します。 
これは、最下位バイトが最初に書き込まれ、最上位バイトが最後に書き込まれることを意味します。 
これについての詳細:\myref{sec:endianness}
この例では、32ビットの値がこのメモリアドレスから \EAX にロードされ、 \printf に渡されます。

$x$のメモリアドレスは\TT{0x00C53394}です。

\clearpage
\olly では、プロセスメモリマップ(Alt-M)を見ることができ、
このアドレスはプログラムの\TT{.data} PEセグメント内にあることがわかります。

\label{olly_memory_map_example}
\begin{figure}[H]
\centering
\myincludegraphics{patterns/04_scanf/2_global/ex2_olly_2.png}
\caption{\olly: process memory map}
\label{fig:scanf_ex2_olly_2}
\end{figure}

}


\subsubsection{GCC: x86}

\myindex{ELF}
Linuxの画像はほぼ同じですが、初期化されていない変数は\TT{\_bss}セグメントにあります。 
\ac{ELF}ファイルでは、このセグメントには次の属性があります。

\begin{lstlisting}
; Segment type: Uninitialized
; Segment permissions: Read/Write
\end{lstlisting}

ただし、変数をある値で初期化してください。 
10の場合、次の属性を持つ\TT{\_data}セグメントに配置されます。

\begin{lstlisting}
; Segment type: Pure data
; Segment permissions: Read/Write
\end{lstlisting}

\subsubsection{MSVC: x64}

\lstinputlisting[caption=MSVC 2012 x64,style=customasmx86]{patterns/04_scanf/2_global/ex2_MSVC_x64_EN.asm}

コードはx86とほとんど同じです。 
$x$変数のアドレスは、 \LEA 命令を使用して \scanf に渡され、
変数の値は \MOV 命令を使用して2番目の \printf に渡されることに注意してください。 
\TT{DWORD PTR}はアセンブリ言語の一部であり(マシンコードと無関係)、
可変データサイズが32ビットであり、 \MOV 命令がそれに応じてエンコードされなければならないことを示します。
