\subsubsection{ARM: \OptimizingKeilVI (\ThumbMode)}

\lstinputlisting[caption=IDA,style=customasmARM]{patterns/04_scanf/2_global/ARM.lst}

Итак, переменная \TT{x} теперь глобальная, и она расположена, почему-то, в другом сегменте, а именно сегменте данных (\emph{.data}).
Можно спросить, почему текстовые строки расположены в сегменте кода (\emph{.text}), а \TT{x} нельзя было разместить тут же?

Потому что эта переменная, и как следует из определения, она может меняться. И может быть, меняться часто.

Ну а текстовые строки имеют тип констант, они не будут меняться, поэтому они располагаются в сегменте \emph{.text}.

\myindex{\RAM}
\myindex{\ROM}
Сегмент кода иногда может быть расположен в ПЗУ микроконтроллера (не забывайте, 
мы сейчас имеем дело с встраиваемой (\emph{embedded}) микроэлектроникой,
где дефицит памяти~--- обычное дело),
а изменяемые переменные~--- в ОЗУ.

Хранить в ОЗУ неизменяемые данные, когда в наличии есть ПЗУ, не экономно.

К тому же, сегмент данных в ОЗУ с константами нужно инициализировать перед работой,
ведь, после включения ОЗУ, очевидно, она содержит в себе случайную информацию.

\myindex{Компоновщик}
Далее мы видим в сегменте кода хранится указатель на переменную \TT{x} (\TT{off\_2C}) и 
все операции с переменной происходят через этот указатель.

Это связано с тем, что переменная \TT{x} может быть расположена где-то довольно далеко от 
данного участка кода, так что её адрес нужно сохранить в непосредственной близости к этому коду.

\myindex{ARM!\Instructions!LDR}
Инструкция \INS{LDR} в Thumb-режиме может адресовать только переменные в пределах вплоть до 1020 байт от своего местоположения.

Эта же инструкция в ARM-режиме~--- переменные в пределах $\pm{}4095$ байт.

Таким образом,
адрес глобальной переменной \TT{x} нужно расположить в непосредственной близости, ведь нет никакой гарантии, 
что компоновщик\footnote{linker в англоязычной литературе} сможет разместить саму переменную где-то рядом, 
она может быть даже в другом чипе памяти!

\myindex{\CLanguageElements!const}
\myindex{\ROM}
Ещё одна вещь: если переменную объявить, как \emph{const}, то компилятор Keil разместит её в сегменте \TT{.constdata}.

Должно быть, впоследствии компоновщик и этот сегмент сможет разместить в ПЗУ вместе с сегментом кода.

\subsubsection{ARM64}

\lstinputlisting[caption=\NonOptimizing GCC 4.9.1 ARM64,numbers=left,style=customasmARM]{patterns/04_scanf/2_global/ARM64_GCC491_O0_RU.s}

\myindex{ARM!\Instructions!ADRP/ADD pair}
Теперь $x$ это глобальная переменная, и её адрес вычисляется при помощи пары инструкций \INS{ADRP}/\INS{ADD} (строки 21 и 25).

