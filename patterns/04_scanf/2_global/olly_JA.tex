\clearpage
\subsubsection{MSVC: x86 + \olly}
\myindex{\olly}

ここではさらに単純です。

\begin{figure}[H]
\centering
\myincludegraphics{patterns/04_scanf/2_global/ex2_olly_1.png}
\caption{\olly: after \scanf execution}
\label{fig:scanf_ex2_olly_1}
\end{figure}

変数はデータセグメントにあります。 
\PUSH 命令($x$のアドレスを押す)が実行されると、
アドレスがスタックウィンドウに表示されます。 その行を右クリックし、\q{ダンプに従う}を選択します。 
変数は、左側のメモリウィンドウに表示されます。 
コンソールに123を入力すると、
メモリウィンドウに\TT{0x7B}が表示されます(ハイライトされたスクリーンショット領域を参照)。

しかし、最初のバイトはなぜ7Bでしょうか? 
論理的に考えると、\TT{00 00 00 7B}のはずです。 
この原因は\gls{endianness}と呼ばれるもので、x86はリトルエンディアンを使用します。 
これは、最下位バイトが最初に書き込まれ、最上位バイトが最後に書き込まれることを意味します。 
これについての詳細:\myref{sec:endianness}
この例では、32ビットの値がこのメモリアドレスから \EAX にロードされ、 \printf に渡されます。

$x$のメモリアドレスは\TT{0x00C53394}です。

\clearpage
\olly では、プロセスメモリマップ(Alt-M)を見ることができ、
このアドレスはプログラムの\TT{.data} PEセグメント内にあることがわかります。

\label{olly_memory_map_example}
\begin{figure}[H]
\centering
\myincludegraphics{patterns/04_scanf/2_global/ex2_olly_2.png}
\caption{\olly: process memory map}
\label{fig:scanf_ex2_olly_2}
\end{figure}

