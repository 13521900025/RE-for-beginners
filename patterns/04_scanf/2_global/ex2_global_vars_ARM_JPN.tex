\subsubsection{ARM: \OptimizingKeilVI (\ThumbMode)}

\lstinputlisting[caption=IDA,style=customasmARM]{patterns/04_scanf/2_global/ARM.lst}

したがって、\TT{x}変数は現在グローバルであり、このために別のセグメント、つまりデータセグメント(\emph{.data})に配置されています。
テキスト文字列がコードセグメント(\emph{.text})にあり、\TT{x}がここにあるのはなぜでしょうか?
これは変数なので、定義上、その値は変更される可能性があります。さらに、頻繁に変更される可能性があります。
テキスト文字列は定数型ですが、変更されないため、\emph{.text}セグメントに配置されます。
\myindex{\RAM}
\myindex{\ROM}

コードセグメントは、時には\ac{ROM}チップに配置されることがあります。
(ここでは、組み込み電子機器を扱います。メモリ不足が普通です)
変更可能な変数は\ac{RAM}に配置されます。

ROMを持っているときは、定数変数をRAMに格納するのはそれほど経済的ではありません。

さらに、RAMの定数変数は初期化する必要があります。これは、電源投入後、明らかにRAMにランダム情報が含まれているためです。

\myindex{Linker}

次に、コードセグメント内の\TT{x}(\TT{off\_2C})変数へのポインターが表示され、
変数を使用するすべての操作はこのポインターを介して行われます。

これは、\TT{x}変数がこの特定のコードフラグメントから離れた場所に配置される可能性があるため、
そのアドレスをコードのすぐ近くに保存する必要があるためです。
\myindex{ARM!\Instructions!LDR}

Thumbモードの\INS{LDR}命令は、その位置から1020バイトの範囲内の変数と、
$\pm{}4095$バイトの範囲のARMモードの変数からのみアドレス可能です。

したがって、\TT{x}変数のアドレスは、
リンカがコードの近くのどこかに変数を格納できる保証がないため、近い場所に配置する必要があります。外部メモリチップでもうまくいくかもしれません!

\myindex{\CLanguageElements!const}
\myindex{\ROM}

もう1つ:変数が\emph{const}として宣言されている場合、Keilコンパイラは\TT{.constdata}セグメントにそれを割り当てます。

その後、リンカはこのセグメントをROMにコードセグメントとともに配置することができます。

\subsubsection{ARM64}

\lstinputlisting[caption=\NonOptimizing GCC 4.9.1 ARM64,numbers=left,style=customasmARM]{patterns/04_scanf/2_global/ARM64_GCC491_O0_EN.s}

\myindex{ARM!\Instructions!ADRP/ADD pair}

この場合、$x$変数はグローバルとして宣言され、そのアドレスは\INS{ADRP}/\INS{ADD}命令ペア
(21行目と25行目)を使用して計算されます。
