\subsection{Popularny błąd}

Bardzo popularnym błędem jest podanie jako argumentu zmiennej \emph{x} zamiast wskaźnika na zmienną \emph{x}:

\lstinputlisting[style=customc]{patterns/04_scanf/error.c}

Więc co tutaj się stanie?
\emph{x} jest niezainicjalizowany i zawiera losowe śmieci z lokalnego stosu.
Kiedy funkcja \scanf jest wywoływana, pobiera ciąg znaków od użytkownika, parsuje go do liczby i próbuje go zapisać w \emph{x}, traktując wartość \emph{x} jako adres w pamięci.
Jednak skoro tam są losowe śmieci ze stosu, \scanf będzie próbować uzyskać dostęp do losowego adresu.
Najczęściej wtedy proces zawiesza działanie.

\myindex{0xCCCCCCCC}
\myindex{0x0BADF00D}
Co ciekawe, niektóre biblioteki \ac{CRT}w wersji do debugowania, umieszczają w pamięci, która jest alokowana, widoczne wzory takie jak  0xCCCCCCCC albo 0x0BADF00D itp.
W tym przypadku, \emph{x} może zawierać 0xCCCCCCCC, więc \scanf będzie próbować dokonać zapisu pod adresem 0xCCCCCCCC.
Jeśli program wyświetli komunikat, że gdzieś w procesie wystąpi próba zapisu pod adresem 0xCCCCCCCC, to znaczy, że
została użyta niezainicjalizowana zmienna (lub wskaźnik).
Jest to lepsze rozwiązanie niż gdyby nowo alokowana pamięć była po prostu wyczyszczona.
