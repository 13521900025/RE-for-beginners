\subsubsection{x86}

\myparagraph{MSVC}
Den folgenden Code erhalten wie nach dem Kompilieren mit MSVC 2010:

\lstinputlisting[style=customasmx86]{patterns/04_scanf/1_simple/ex1_MSVC_DE.asm}

\TT{x} ist eine lokale Variable.

Gemäß dem \CCpp-Standard darf diese nur innerhalb dieser Funktion sichtbar sein und nicht aus einem anderen, äußeren Scope.
Traditionell werden lokale Variablen auf dem Stack gespeichert.
Es gibt möglicherweise andere Wege sie anzulegen, aber in x86 geschieht es auf diese Weise.


\myindex{x86!\Instructions!PUSH}
Das Ziel des Befehls direkt nach dem Funktionsprolog, \TT{PUSH ECX}), ist es nicht, den Status von \ECX zu sichern
(man beachte, dass Fehlen eines entsprechenden \TT{POP ECX} im Funktionsepilog).
Tatsächlich reserviert der Befehl 4 Byte auf dem Stack, um die Variable $x$ speichern zu können.

\label{stack_frame}
\myindex{\Stack!Stack frame}
\myindex{x86!\Registers!EBP}
Auf \TT{x} wird mithilfe des \TT{\_x\$} Makros (es entspricht -4) und des \EBP Registers, das auf den aktuellen Stack Frame zeigt, zugegriffen. 
Während der Dauer der Funktionsausführung zeigt \EBP auf den aktuellen \glslink{stack frame}{Stack Frame}, wodurch mittels \TT{EBP+offset} auf lokalen Variablen und Funktionsargumente zugegriffen werden kann.

\TT{x} is to be accessed with the assistance of the \TT{\_x\$} macro (it equals to -4) and the \EBP register pointing to the current frame.

\myindex{x86!\Registers!ESP}
Es ist auch möglich, das \ESP Register zu diesem Zweck zu verwenden, aber dies ist ungebräuchlich, da es sich häufig verändert.
Der Wert von \EBP kann als eingefrorener Wert des Wertes von \ESP zu Beginn der Funktionsausführung verstanden werden.

It is also possible to use \ESP for the same purpose, although that is not very convenient since it changes frequently.
The value of the \EBP could be perceived as a \emph{frozen state} of the value in \ESP at the start of the function's execution.

% FIXME1 это уже было в 02_stack?
Hier ist ein typisches Layour eines Stack Frames in einer 32-Bit-Umgebung:

\begin{center}
\begin{tabular}{ | l | l | }
\hline
\dots & \dots \\
\hline
EBP-8 & local variable \#2, \MarkedInIDAAs{} \TT{var\_8} \\
\hline
EBP-4 & local variable \#1, \MarkedInIDAAs{} \TT{var\_4} \\
\hline
EBP & saved value of \EBP \\
\hline
EBP+4 & return address \\
\hline
EBP+8 & \argument \#1, \MarkedInIDAAs{} \TT{arg\_0} \\
\hline
EBP+0xC & \argument \#2, \MarkedInIDAAs{} \TT{arg\_4} \\
\hline
EBP+0x10 & \argument \#3, \MarkedInIDAAs{} \TT{arg\_8} \\
\hline
\dots & \dots \\
\hline
\end{tabular}
\end{center}
Die Funktion \scanf in unserem Beispiel hat zwei Argumente.

Das erste ist ein Pointer auf den String \TT{\%d} und das zweite ist die Adresse der Variablen \TT{x}.

\myindex{x86!\Instructions!LEA}
Zunächst wird die Adresse der Variablen $x$ durch den Befehl \\
\TT{lea eax, DWORD PTR \_x\$[ebp]} in das \EAX Register geladen.

\LEA steht für \emph{load effective address} und wird häufig benutzt, um eine Adresse zu erstellen ~(\myref{sec:LEA}).
In diesem Fall speichert \LEA einfach die Summe des \EBP Registers und des \TT{\_\$} Makros im Register \EAX.
Dies entspricht dem Befehl \INS{lea eax, [ebp-4]}.

Es wird also 4 von Wert in \EBP abgezogen und das Ergebnis in das Register \EAX geladen.
Danach wird der Wert in \EAX auf dem Stack abgelegt und \scanf wird aufgerufen.

Anschließend wird \printf mit einem Argument aufgerufen--einen Pointer auf den String:
\TT{You entered \%d...\textbackslash{}n}.

Das zweite Argument wird mit \TT{mov ecx, [ebp-4]} vorbereitet.
Dieser Befehl speichert den Wert der Variablen $x$ (nicht seine Adresse) im Register \ECX.

Schließlich wird der Wert in \ECX auf dem Stack gespeichert und das letzte \printf wird aufgerufen.

\EN{\clearpage
\subsubsection{MSVC + \olly}
\myindex{\olly}

Let's try this example in \olly.
Let's load it and keep pressing F8 (\stepover) until we reach our executable file instead of \TT{ntdll.dll}.
Scroll up until \main appears.

Click on the first instruction (\TT{PUSH EBP}), press F2 (\emph{set a breakpoint}), then F9 (\emph{Run}).
The breakpoint will be triggered when \main begins.

Let's trace to the point where the address of the variable $x$ is calculated:

\begin{figure}[H]
\centering
\myincludegraphics{patterns/04_scanf/1_simple/ex1_olly_1.png}
\caption{\olly: The address of the local variable is calculated}
\label{fig:scanf_ex1_olly_1}
\end{figure}

Right-click the \EAX in the registers window and then select \q{Follow in stack}.

This address will appear in the stack window.
The red arrow has been added, pointing to the variable in the local stack.
At that moment this location contains some garbage (\TT{0x6E494714}).
Now with the help of \PUSH instruction the address of this stack element is going to be stored to the same stack on the next position.
Let's trace with F8 until the \scanf execution completes.
During the \scanf execution, we input, for example, 123, in the console window:

\lstinputlisting{patterns/04_scanf/1_simple/console.txt}

\clearpage
\scanf completed its execution already:

\begin{figure}[H]
\centering
\myincludegraphics{patterns/04_scanf/1_simple/ex1_olly_3.png}
\caption{\olly: \scanf executed}
\label{fig:scanf_ex1_olly_3}
\end{figure}

\scanf returns 1 in \EAX, which implies that it has read successfully one value.
If we look again at the stack element corresponding to the local variable it now contains \TT{0x7B} (123).

\clearpage

Later this value is copied from the stack to the \ECX register and passed to \printf:

\begin{figure}[H]
\centering
\myincludegraphics{patterns/04_scanf/1_simple/ex1_olly_4.png}
\caption{\olly: preparing the value for passing to \printf}
\label{fig:scanf_ex1_olly_4}
\end{figure}
}
\RU{\clearpage
\subsubsection{MSVC + \olly}
\myindex{\olly}

Попробуем этот же пример в \olly.
Загружаем, нажимаем F8 (\stepover) до тех пор, пока не окажемся в своем исполняемом файле,
а не в \TT{ntdll.dll}.
Прокручиваем вверх до тех пор, пока не найдем \main.
Щелкаем на первой инструкции (\TT{PUSH EBP}), нажимаем F2 (\emph{set a breakpoint}), 
затем F9 (\emph{Run}) и точка останова срабатывает на начале \main.

Трассируем до того места, где готовится адрес переменной $x$:

\begin{figure}[H]
\centering
\myincludegraphics{patterns/04_scanf/1_simple/ex1_olly_1.png}
\caption{\olly: вычисляется адрес локальной переменной}
\label{fig:scanf_ex1_olly_1}
\end{figure}

На \EAX в окне регистров можно нажать правой кнопкой и далее выбрать \q{Follow in stack}.
Этот адрес покажется в окне стека.

Смотрите, это переменная в локальном стеке. Там дорисована красная стрелка.
И там сейчас какой-то мусор (\TT{0x6E494714}).
Адрес этого элемента стека сейчас, при помощи \PUSH запишется в этот же стек рядом.
Трассируем при помощи F8 вплоть до конца исполнения \scanf.
А пока \scanf исполняется, в консольном окне, вводим, например, 123:

\lstinputlisting{patterns/04_scanf/1_simple/console.txt}

\clearpage
Вот тут \scanf отработал:

\begin{figure}[H]
\centering
\myincludegraphics{patterns/04_scanf/1_simple/ex1_olly_3.png}
\caption{\olly: \scanf исполнилась}
\label{fig:scanf_ex1_olly_3}
\end{figure}

\scanf вернул 1 в \EAX, что означает, что он успешно прочитал одно значение.
В наблюдаемом нами элементе стека теперь \TT{0x7B} (123).

\clearpage
Чуть позже это значение копируется из стека в регистр \ECX и передается в \printf{}:

\begin{figure}[H]
\centering
\myincludegraphics{patterns/04_scanf/1_simple/ex1_olly_4.png}
\caption{\olly: готовим значение для передачи в \printf}
\label{fig:scanf_ex1_olly_4}
\end{figure}
}
\IT{\clearpage
\subsubsection{MSVC + \olly}
\myindex{\olly}

Proviamo ad analizzare l'esempio con \olly.
Carichiamo l'eseguibile e premiamo F8 (\stepover) fino a raggiungere il nostro eseguibile invece che \TT{ntdll.dll}.
Scorriamo verso l'alto finche' appare \main .

Clicchiamo sulla prima istruzione (\TT{PUSH EBP}), premiamo F2 (\emph{set a breakpoint}), e quindi F9 (\emph{Run}).
Il breakpoint sara' scatenato all'inizio della funzione \main .

Tracciamo adesso fino al punto in cui viene calcolato l'indirizzo della variabile $x$:

\begin{figure}[H]
\centering
\myincludegraphics{patterns/04_scanf/1_simple/ex1_olly_1.png}
\caption{\olly: The address of the local variable is calculated}
\label{fig:scanf_ex1_olly_1}
\end{figure}


Click di destra su \EAX nella finestra dei registri e selezioniamo \q{Follow in stack}.

Questo indirizzo apparira' nella finestra dello stack.
La freccia rossa aggiunta punta alla variabile nello stack locale.
Al momento questa locazione contiene un po' di immondizia (garbage) (\TT{0x6E494714}).
Con l'aiuto dell'istruzione \PUSH l'indirizzo di questo elemento dello stack sara' memorizzato nello stesso stack alla posizione successiva.
Tracciamo con F8 finche' non viene completata l'esecuzione della funzione \scanf.
Durante l'esecuzione di \scanf, diamo in input un valore nella console. Ad esempio 123:

\lstinputlisting{patterns/04_scanf/1_simple/console.txt}

\clearpage
\scanf ha gia' completato la sua esecuzione:

\begin{figure}[H]
\centering
\myincludegraphics{patterns/04_scanf/1_simple/ex1_olly_3.png}
\caption{\olly: \scanf executed}
\label{fig:scanf_ex1_olly_3}
\end{figure}

\scanf restituisce 1 in \EAX, e cio' implica che ha letto con successo un valore.
Se guardiamo nuovamente l'elemento nello stack corrispondente alla variabile locale, adesso contiente \TT{0x7B} (123).

\clearpage

Successivamente questo valore viene copiato dallo stack al registro \ECX e passato a \printf:

\begin{figure}[H]
\centering
\myincludegraphics{patterns/04_scanf/1_simple/ex1_olly_4.png}
\caption{\olly: preparing the value for passing to \printf}
\label{fig:scanf_ex1_olly_4}
\end{figure}
}
\DE{\clearpage
\subsubsection{MSVC + \olly}
\myindex{\olly}
Schauen wir uns diese Beispiel in \olly an.
Wir laden es und drücken F8 (\stepover) bis wir unsere ausführbare Datei anstelle von \TT{ntdll.dll} erreicht haben. Wir scrollen nach oben bis \main erscheint.

Wir klicken auf den ersten Befehl (\TT{PUSH EBO}), drücken F2 (\emph{set a breakpoint}), dann F9 (\emph{Run}). Der Breakpoint wird ausgelöst, wenn die Funktion \main beginnt.

Verfolgen wir den Ablauf bis zu der Stelle, an der die Adresse der Variablen $x$ berechnet wird:

\begin{figure}[H]
\centering
\myincludegraphics{patterns/04_scanf/1_simple/ex1_olly_1.png}
\caption{\olly: Die Adresse der lokalen Variable wird berechnet.}
\label{fig:scanf_ex1_olly_1}
\end{figure}

Wir machen einen Rechtsklick auf \EAX in Registerfenster und wählen \q{Follow in stack}. 

Diese Adresse wird im Stackfenster erscheinen. Der rote Pfeil wurde nachträglich hinzugefügt; er zeigt auf die Variable im lokalen Stack. Im Moment enthält diese Speicherstelle Zufallswerte (\TT{0x6E494714}). Jetzt wird mithilfe des \PUSH Befehls die Adresse dieses Stackelements auf demselben Stack an der folgenden Position gespeichert. 
Verfolgen wir den Ablauf mit F8 bis die Ausführung von \scanf abgeschlossen ist. Während der Ausführung von \scanf geben wir beispielsweise 123 in der Konsole ein:

\lstinputlisting{patterns/04_scanf/1_simple/console.txt}

\clearpage
\scanf ist bereits beendet:

\begin{figure}[H]
\centering
\myincludegraphics{patterns/04_scanf/1_simple/ex1_olly_3.png}
\caption{\olly: \scanf wurde ausgeführt}
\label{fig:scanf_ex1_olly_3}
\end{figure}

\scanf liefert 1 im \EAX Register zurück, was aussagt, dass die Funktion einen Wert erfolgreich eingelesen hat. Wenn wir wiederum auf das zugehörige Stackelement für die lokale Variable schauen, enthält diese nun den Wert \TT{0x7B} (dez. 123).

\clearpage
Im weiteren Verlauf wird dieser Wert vom Stack in das \ECX Register kopiert und an \printf übergeben:

\begin{figure}[H]
\centering
\myincludegraphics{patterns/04_scanf/1_simple/ex1_olly_4.png}
\caption{\olly: Wert für Übergabe an \printf vorbereiten.}
\label{fig:scanf_ex1_olly_4}
\end{figure}

}
\FR{\clearpage
\subsubsection{MSVC + \olly}
\myindex{\olly}

% TODO look in French olly for text translation, if exists?
Essayons cet exemple dans \olly.
Chargeons-le et appuyons sur F8 (\stepover) jusqu'à ce que nous atteignons notre
exécutable au lieu de \TT{ntdll.dll}.
Défiler vers le haut jusqu'à ce que \main apparaisse.

Cliquer sur la première instruction  (\TT{PUSH EBP}), appuyer sur F2 (\emph{set a
breakpoint}), puis F9 (\emph{Run}).
Le point d'arrêt sera déclenché lorsque \main commencera.

Continuons jusqu'au point où la variable $x$ est calculée:

\begin{figure}[H]
\centering
\myincludegraphics{patterns/04_scanf/1_simple/ex1_olly_1.png}
\caption{\olly: L'adresse de la variable locale est calculée}
\label{fig:scanf_ex1_olly_1}
\end{figure}

Cliquer droit sur \EAX dans la fenêtre des registres et choisir \q{Follow in stack}.

Cette adresse va apparaître dans la fenêtre de la pile.
La flèche rouge a été ajoutée, pointant la variable dans la pile locale.
A ce point, cet espace contient des restes de données (\TT{0x6E494714}).
Maintenant. avec l'aide de l'instruction \PUSH, l'adresse de cet élément de pile
va être stockée sur la même pile à la position suivante.
Appuyons sur F8 jusqu'à la fin de l'exécution de \scanf.
Pendant l'exécution de \scanf, entrons, par exemple, 123, dans la fenêtre de la console:

\lstinputlisting{patterns/04_scanf/1_simple/console.txt}

\clearpage
\scanf a déjà fini de s'exécuter:

\begin{figure}[H]
\centering
\myincludegraphics{patterns/04_scanf/1_simple/ex1_olly_3.png}
\caption{\olly: \scanf s'est exécutée}
\label{fig:scanf_ex1_olly_3}
\end{figure}

\scanf renvoie 1 dans \EAX, ce qui indique qu'elle a lu avec succès une valeur.
Si nous regardons de nouveau l'élément de la pile correspondant à la variable
locale, il contient maintenant \TT{0x7B} (123).

\clearpage

Plus tard, cette valeur est copiée de la pile vers le registre \ECX et passée à \printf:

\begin{figure}[H]
\centering
\myincludegraphics{patterns/04_scanf/1_simple/ex1_olly_4.png}
\caption{\olly: préparation de la valeur pour la passer à \printf}
\label{fig:scanf_ex1_olly_4}
\end{figure}
}
\JA{\clearpage
\subsubsection{MSVC + \olly}
\myindex{\olly}

\olly でこの例を試してみましょう。 
私たちが\TT{ntdll.dll}の代わりに実行可能ファイルに達するまで、それをロードしてF8(\stepover)を押し続けましょう。
\main が表示されるまで上にスクロールします。

最初の命令(\TT{PUSH EBP})をクリックし、F2(\emph{ブレークポイントを設定})、次にF9(\emph{実行})を押します。 
\main が始まるとブレークポイントがトリガされます。

変数 $x$ のアドレスが計算されるポイントまでトレースしましょう:

\begin{figure}[H]
\centering
\myincludegraphics{patterns/04_scanf/1_simple/ex1_olly_1.png}
\caption{\olly: ローカル変数のアドレスが計算されます。}
\label{fig:scanf_ex1_olly_1}
\end{figure}

レジスタウィンドウで \EAX を右クリックして、\q{Follow in stack}を選択します。

このアドレスはスタックウィンドウに表示されます。 
赤い矢印が追加され、ローカルスタックの変数を指しています。 
その瞬間、この場所にはいくらかのゴミ(\TT{0x6E494714})が含まれています。 
今度は \PUSH 命令の助けを借りて、このスタック要素のアドレスが次の位置の同じスタックに格納されます。 
\scanf の実行が完了するまで、F8を使ってトレースしてみましょう。 
\scanf の実行中に、コンソールウィンドウに123などを入力します。

\lstinputlisting{patterns/04_scanf/1_simple/console.txt}

\clearpage
\scanf はすでに実行を完了しました:

\begin{figure}[H]
\centering
\myincludegraphics{patterns/04_scanf/1_simple/ex1_olly_3.png}
\caption{\olly: \scanf が実行された}
\label{fig:scanf_ex1_olly_3}
\end{figure}

\scanf は \EAX で1を返します。これは、1つの値を正常に読み取ったことを意味します。 
ローカル変数に対応するスタック要素をもう一度見ると、\TT{0x7B}(123)が含まれています。

\clearpage

その後、この値はスタックから \ECX レジスタにコピーされ、 \printf に渡されます:

\begin{figure}[H]
\centering
\myincludegraphics{patterns/04_scanf/1_simple/ex1_olly_4.png}
\caption{\olly: \printf に渡す値を準備する}
\label{fig:scanf_ex1_olly_4}
\end{figure}
}


\myparagraph{GCC}

Kompilieren wir diesen Code mit GCC 4.4.1 unter Linux:

\lstinputlisting[style=customasmx86]{patterns/04_scanf/1_simple/ex1_GCC.asm}

\myindex{puts() instead of printf()}
GCC ersetzt den Aufruf von \printf durch einen Aufruf von \puts. Der Grund hierfür wurde bereits in ~(\myref{puts}) erklärt.

% TODO: rewrite
%\RU{Почему \scanf переименовали в \TT{\_\_\_isoc99\_scanf}, я честно говоря, пока не знаю.}
%\EN{Why \scanf is renamed to \TT{\_\_\_isoc99\_scanf}, I do not know yet.}
% 
% Apparently it has to do with the ISO c99 standard compliance. By default GCC allows specifying a standard to adhere to.
% For example if you compile with -std=c89 the outputted assmebly file will contain scanf and not __isoc99__scanf. I guess current GCC version adhares to c99 by default.
% According to my understanding the two implementations differ in the set of suported modifyers (See printf man page)
Genau wie im MSVC Beispiel werden die Argumente mithilfe des Befehls \MOV auf dem Stack abgelegt.

\myparagraph{By the way}
Dieses einfache Beispiel ist übrigens eine Demonstration der Tatsache, dass der Compiler eine Liste von Ausdrücken in einem
\CCpp-Block in eine sequentielle Liste von Befehlen übersetzt.
Es gibt nichts zwischen zwei \CCpp-Anweisungen und genauso verhält es sich auch im Maschinencode.
Der Control Flow geht von einem Ausdruck direkt an den folgenden über.
