\clearpage
\subsubsection{MSVC + \olly}
\myindex{\olly}

\olly でこの例を試してみましょう。 
私たちが\TT{ntdll.dll}の代わりに実行可能ファイルに達するまで、それをロードしてF8(\stepover)を押し続けましょう。
\main が表示されるまで上にスクロールします。

最初の命令(\TT{PUSH EBP})をクリックし、F2(\emph{ブレークポイントを設定})、次にF9(\emph{実行})を押します。 
\main が始まるとブレークポイントがトリガされます。

変数 $x$ のアドレスが計算されるポイントまでトレースしましょう:

\begin{figure}[H]
\centering
\myincludegraphics{patterns/04_scanf/1_simple/ex1_olly_1.png}
\caption{\olly: ローカル変数のアドレスが計算されます。}
\label{fig:scanf_ex1_olly_1}
\end{figure}

レジスタウィンドウで \EAX を右クリックして、\q{Follow in stack}を選択します。

このアドレスはスタックウィンドウに表示されます。 
赤い矢印が追加され、ローカルスタックの変数を指しています。 
その瞬間、この場所にはいくらかのゴミ(\TT{0x6E494714})が含まれています。 
今度は \PUSH 命令の助けを借りて、このスタック要素のアドレスが次の位置の同じスタックに格納されます。 
\scanf の実行が完了するまで、F8を使ってトレースしてみましょう。 
\scanf の実行中に、コンソールウィンドウに123などを入力します。

\lstinputlisting{patterns/04_scanf/1_simple/console.txt}

\clearpage
\scanf はすでに実行を完了しました:

\begin{figure}[H]
\centering
\myincludegraphics{patterns/04_scanf/1_simple/ex1_olly_3.png}
\caption{\olly: \scanf が実行された}
\label{fig:scanf_ex1_olly_3}
\end{figure}

\scanf は \EAX で1を返します。これは、1つの値を正常に読み取ったことを意味します。 
ローカル変数に対応するスタック要素をもう一度見ると、\TT{0x7B}(123)が含まれています。

\clearpage

その後、この値はスタックから \ECX レジスタにコピーされ、 \printf に渡されます:

\begin{figure}[H]
\centering
\myincludegraphics{patterns/04_scanf/1_simple/ex1_olly_4.png}
\caption{\olly: \printf に渡す値を準備する}
\label{fig:scanf_ex1_olly_4}
\end{figure}
