\subsection{Prosty przykład}

\lstinputlisting[style=customc]{patterns/04_scanf/1_simple/ex1.c}

Teraz używanie \scanf do interakcji z użytkownikiem w programach nie jest zbyt popularne, jednak mimo wszystko funkcja ta jest dobrym przykładem użycia wskaźnika na zmienną typu  \Tint.

\subsubsection{O wskaźnikach}
\myindex{\CLanguageElements!\Pointers}

Wskaźniki są jedną z podstawowych koncepcji informatycznych. Często tworząc dużą tablicę, strukturę albo objekt jako argument jakiejś funkcji, przekazywanie wskaźnika do argumentu jest mniej pamięciożerne niż gdyby przekazać całą tablicę/strukturę, np. kiedy chcesz wypisać tekst w konsoli, najprościej będzie wskazać jego adres w pamięci.

W dodatku jeśli wywoływana funkcja potrzebuje zmodyfikować cokolwiek w dużej tablicy lub strukturze danych przekazanej jako parametr i zwrócić potem tą tablicę/strukturę, kopiowanie tylu danych byłoby prawie absurdalne. Dlatego najprościej będzie przekazać adres tej tablicy/struktury do wywoływanej funkcji i wtedy zmodyfikować to co wymagało modyfikacji.

Wskaźnik w \CCpp---jest adresem pewnego miejsca w pamięci.

\myindex{x86-64}
W x86, adresy są reprezentowane przy pomocy 32-bitowych liczb (czyli 4 bajtowych), a w x86-64 jako liczby 64-bitowe (czyli 8 bajtowe). Przy okazji jest to powód dlaczego niektórych ludzi oburza przeskok na x86-64- wszystkie wskaźniki w architekturze x64 wymagają dwa razy więcej miejsca, włączając pamięć cache, co jest bardzo "kosztownym" zużyciem pamięci.

% TODO ... а делать разные версии memcpy для разных типов - абсурд
\myindex{\CStandardLibrary!memcpy()}
Możliwa jest praca z tylko nietypowanymi wskaźnikami, wymagająca nieco wysiłku, np. funkcja z biblioteki standardowej C-\TT{memcpy()}, która kopiuje blok z jednej lokalizacji w pamięci do innej, jako argumenty przyjmuje 2 wskaźniki typu \TT{void*}, co umożliwia kopiowanie dowolnych typów danych. Typy danych nie są istotne, znaczenie mają tylko rozmiary bloków pamięci.

Wskaźniki są także często używane kiedy funkcja potrzebuje zwrócić więcej niż jedną wartość (wrócę do tego później
~(\myref{label_pointers})
).

Funkcja \emph{scanf()} jest takim przypadkiem.

Oprócz tego faktu, funkcja scanf() wymaga podania w argumencie ile wartości ma wczytać, żeby później móc je zwrócić.

W \CCpp typ wskaźnika jest potrzebny tylko do sprawdzania typów podczas kompilacji.

Wewnątrz skompilowanego kodu nie ma żadnej informacji jakiego typu są wskaźniki.
% TODO это сильно затрудняет декомпиляцию

\EN{\subsubsection{x86}

\myparagraph{MSVC}

Here is what we get after compiling with MSVC 2010:

\lstinputlisting[style=customasmx86]{patterns/04_scanf/1_simple/ex1_MSVC_EN.asm}

\TT{x} is a local variable.

According to the \CCpp standard it must be visible only in this function and not from any other external scope. 
Traditionally, local variables are stored on the stack. 
There are probably other ways to allocate them, but in x86 that is the way it is.

\myindex{x86!\Instructions!PUSH}
The goal of the instruction following the function prologue, \TT{PUSH ECX}, is not to save the \ECX state 
(notice the absence of corresponding \TT{POP ECX} at the function's end).

In fact it allocates 4 bytes on the stack for storing the \TT{x} variable.

\label{stack_frame}
\myindex{\Stack!Stack frame}
\myindex{x86!\Registers!EBP}
\TT{x} is to be accessed with the assistance of the \TT{\_x\$} macro (it equals to -4) and the \EBP register pointing to the current frame.

Over the span of the function's execution, \EBP is pointing to the current \gls{stack frame}
making it possible to access local variables and function arguments via \TT{EBP+offset}.

\myindex{x86!\Registers!ESP}
It is also possible to use \ESP for the same purpose, although that is not very convenient since it changes frequently.
The value of the \EBP could be perceived as a \emph{frozen state} of the value in \ESP at the start of the function's execution.

% FIXME1 это уже было в 02_stack?
Here is a typical \gls{stack frame} layout in 32-bit environment:

\begin{center}
\begin{tabular}{ | l | l | }
\hline
\dots & \dots \\
\hline
EBP-8 & local variable \#2, \MarkedInIDAAs{} \TT{var\_8} \\
\hline
EBP-4 & local variable \#1, \MarkedInIDAAs{} \TT{var\_4} \\
\hline
EBP & saved value of \EBP \\
\hline
EBP+4 & return address \\
\hline
EBP+8 & \argument \#1, \MarkedInIDAAs{} \TT{arg\_0} \\
\hline
EBP+0xC & \argument \#2, \MarkedInIDAAs{} \TT{arg\_4} \\
\hline
EBP+0x10 & \argument \#3, \MarkedInIDAAs{} \TT{arg\_8} \\
\hline
\dots & \dots \\
\hline
\end{tabular}
\end{center}

The \scanf function in our example has two arguments.

The first one is a pointer to the string containing \TT{\%d} and the second is the address of the \TT{x} variable.

\myindex{x86!\Instructions!LEA}
First, the \TT{x} variable's address is loaded into the \EAX register by the \\
\TT{lea eax, DWORD PTR \_x\$[ebp]} instruction.

\LEA stands for \emph{load effective address}, and is often used for forming an address ~(\myref{sec:LEA}).

We could say that in this case \LEA simply stores the sum of the \EBP register value and the \TT{\_x\$} macro in the \EAX register.

This is the same as \INS{lea eax, [ebp-4]}.

So, 4 is being subtracted from the \EBP register value and the result is loaded in the \EAX register.
Next the \EAX register value is pushed into the stack and \scanf is being called.

\printf is being called after that with its first argument --- a pointer to the string:
\TT{You entered \%d...\textbackslash{}n}.

The second argument is prepared with: \TT{mov ecx, [ebp-4]}.
The instruction stores the \TT{x} variable value and not its address, in the \ECX register.

Next the value in the \ECX is stored on the stack and the last \printf is being called.

\EN{\input{patterns/04_scanf/1_simple/olly_EN}}
\RU{\input{patterns/04_scanf/1_simple/olly_RU}}
\IT{\input{patterns/04_scanf/1_simple/olly_IT}}
\DE{\input{patterns/04_scanf/1_simple/olly_DE}}
\FR{\input{patterns/04_scanf/1_simple/olly_FR}}
\JA{\input{patterns/04_scanf/1_simple/olly_JA}}


\myparagraph{GCC}

Let's try to compile this code in GCC 4.4.1 under Linux:

\lstinputlisting[style=customasmx86]{patterns/04_scanf/1_simple/ex1_GCC.asm}

\myindex{puts() instead of printf()}
GCC replaced the \printf call with call to \puts. The reason for this was explained in ~(\myref{puts}).

% TODO: rewrite
%\RU{Почему \scanf переименовали в \TT{\_\_\_isoc99\_scanf}, я честно говоря, пока не знаю.}
%\EN{Why \scanf is renamed to \TT{\_\_\_isoc99\_scanf}, I do not know yet.}
% 
% Apparently it has to do with the ISO c99 standard compliance. By default GCC allows specifying a standard to adhere to.
% For example if you compile with -std=c89 the outputted assmebly file will contain scanf and not __isoc99__scanf. I guess current GCC version adhares to c99 by default.
% According to my understanding the two implementations differ in the set of suported modifyers (See printf man page)

As in the MSVC example---the arguments are placed on the stack using the \MOV instruction.

\myparagraph{By the way}

This simple example is a demonstration of the fact that compiler translates
list of expressions in \CCpp-block into sequential list of instructions.
There are nothing between expressions in \CCpp, and so in resulting machine code, 
there are nothing between, control flow slips from one expression to the next one.

}
\RU{\subsubsection{x86}

\myparagraph{MSVC}

Что получаем на ассемблере, компилируя в MSVC 2010:

\lstinputlisting[style=customasmx86]{patterns/04_scanf/1_simple/ex1_MSVC_RU.asm}

Переменная \TT{x} является локальной.

По стандарту \CCpp она доступна только из этой же функции и нигде более. 
Так получилось, что локальные переменные располагаются в стеке. 
Может быть, можно было бы использовать и другие варианты, но в x86 это традиционно так.

\myindex{x86!\Instructions!PUSH}
Следующая после пролога инструкция \TT{PUSH ECX} не ставит своей целью сохранить 
значение регистра \ECX. 
(Заметьте отсутствие соответствующей инструкции \TT{POP ECX} в конце функции).

Она на самом деле выделяет в стеке 4 байта для хранения \TT{x} в будущем.

\label{stack_frame}
\myindex{\Stack!Стековый фрейм}
\myindex{x86!\Registers!EBP}
Доступ к \TT{x} будет осуществляться при помощи объявленного макроса \TT{\_x\$} (он равен -4) и регистра \EBP указывающего на текущий фрейм.

Во всё время исполнения функции \EBP указывает на текущий \glslink{stack frame}{фрейм} и через \TT{EBP+смещение}
можно получить доступ как к локальным переменным функции, так и аргументам функции.

\myindex{x86!\Registers!ESP}
Можно было бы использовать \ESP, но он во время исполнения функции часто меняется, а это не удобно. 
Так что можно сказать, что \EBP это \emph{замороженное состояние} \ESP на момент начала исполнения функции.

% FIXME1 это уже было в 02_stack?
Разметка типичного стекового \glslink{stack frame}{фрейма} в 32-битной среде:

\begin{center}
\begin{tabular}{ | l | l | }
\hline
\dots & \dots \\
\hline
EBP-8 & локальная переменная \#2, \MarkedInIDAAs{} \TT{var\_8} \\
\hline
EBP-4 & локальная переменная \#1, \MarkedInIDAAs{} \TT{var\_4} \\
\hline
EBP & сохраненное значение \EBP \\
\hline
EBP+4 & адрес возврата \\
\hline
EBP+8 & \argument \#1, \MarkedInIDAAs{} \TT{arg\_0} \\
\hline
EBP+0xC & \argument \#2, \MarkedInIDAAs{} \TT{arg\_4} \\
\hline
EBP+0x10 & \argument \#3, \MarkedInIDAAs{} \TT{arg\_8} \\
\hline
\dots & \dots \\
\hline
\end{tabular}
\end{center}

У функции \scanf в нашем примере два аргумента.

Первый~--- указатель на строку, содержащую \TT{\%d} и второй~--- адрес переменной \TT{x}.

\myindex{x86!\Instructions!LEA}
Вначале адрес \TT{x} помещается в регистр \EAX при помощи инструкции \TT{lea eax, DWORD PTR \_x\$[ebp]}.

Инструкция \LEA означает \emph{load effective address}, и часто используется для формирования адреса чего-либо ~(\myref{sec:LEA}).

Можно сказать, что в данном случае \LEA просто помещает в \EAX результат суммы значения в регистре \EBP и макроса \TT{\_x\$}.

Это тоже что и \INS{lea eax, [ebp-4]}.

Итак, от значения \EBP отнимается 4 и помещается в \EAX.
Далее значение \EAX заталкивается в стек и вызывается \scanf.

После этого вызывается \printf. Первый аргумент вызова строка:
\TT{You entered \%d...\textbackslash{}n}.

Второй аргумент: \INS{mov ecx, [ebp-4]}.
Эта инструкция помещает в \ECX не адрес переменной \TT{x}, а её значение.

Далее значение \ECX заталкивается в стек и вызывается \printf.

\EN{\input{patterns/04_scanf/1_simple/olly_EN}}
\RU{\input{patterns/04_scanf/1_simple/olly_RU}}
\IT{\input{patterns/04_scanf/1_simple/olly_IT}}
\DE{\input{patterns/04_scanf/1_simple/olly_DE}}
\FR{\input{patterns/04_scanf/1_simple/olly_FR}}
\JA{\input{patterns/04_scanf/1_simple/olly_JA}}


\myparagraph{GCC}

Попробуем тоже самое скомпилировать в Linux при помощи GCC 4.4.1:

\lstinputlisting[style=customasmx86]{patterns/04_scanf/1_simple/ex1_GCC.asm}

\myindex{puts() вместо printf()}
GCC заменил первый вызов \printf на \puts. Почему это было сделано, 
уже было описано ранее~(\myref{puts}).

% TODO: rewrite
%\RU{Почему \scanf переименовали в \TT{\_\_\_isoc99\_scanf}, я честно говоря, пока не знаю.}
%\EN{Why \scanf is renamed to \TT{\_\_\_isoc99\_scanf}, I do not know yet.}
% 
% Apparently it has to do with the ISO c99 standard compliance. By default GCC allows specifying a standard to adhere to.
% For example if you compile with -std=c89 the outputted assmebly file will contain scanf and not __isoc99__scanf. I guess current GCC version adhares to c99 by default.
% According to my understanding the two implementations differ in the set of suported modifyers (See printf man page)


Далее всё как и прежде~--- параметры заталкиваются через стек при помощи \MOV.

\myparagraph{Кстати}

Этот простой пример иллюстрирует то обстоятельство, что компилятор преобразует
список выражений в \CCpp-блоке просто в последовательный набор инструкций.
Между выражениями в \CCpp ничего нет, и в итоговом машинном коде между ними тоже ничего нет, 
управление переходит от одной инструкции к следующей за ней.

}
\PTBR{\subsubsection{x86: 3 argumentos}

\myparagraph{MSVC}

Quando compilamos esse código com o MSVC 2010 Express temos:

\begin{lstlisting}[style=customasmx86]
$SG3830	DB	'a=%d; b=%d; c=%d', 00H

...

	push	3
	push	2
	push	1
	push	OFFSET $SG3830
	call	_printf
	add	esp, 16					; 00000010H
\end{lstlisting}

Quase a mesma coisa, mas agora nós podemos ver que os argumentos da função \printf são empurrados na pilha na ordem reversa. O primeiro argumento é empurrado por último.

A propósito, variáveis do tipo \Tint em ambientes 32-bits tem 32-bits de largura, isso é 4 bytes.

Então, nós temos quatro argumentos aqui, $4*4=16$ --- eles ocupam exatamente 16 bytes na pilha: um ponteiro de 32-bits para uma string e três números do tipo \Tint.

\myindex{x86!\Instructions!ADD}
\myindex{x86!\Registers!ESP}
\myindex{cdecl}
Quando o ponteiro da pilha (registrador \ESP) volta para seu valor anterior pela instrução \INS{ADD ESP, X} depois de uma chamada de função,
geralmente o número de argumentos pode ser obtido simplismente por se dividir X, o argumento da função \ADD, por 4.

Lógico que isso é por causa da convenção de chamada do \emph{cdecl} e somente para ambientes 32-bits.

% TODO: Olly, GCC, GDB

}
\IT{\subsubsection{x86}

\myparagraph{MSVC}

Questo e' cio' che si ottiene dopo la compilazione con MSVC 2010:

\lstinputlisting[style=customasmx86]{patterns/04_scanf/1_simple/ex1_MSVC_EN.asm}

\TT{x} e' una variabile locale.

In base allo standard \CCpp deve essere visibile soltanto in questa funzione e non in altri ambiti (esterni alla funzione).
Tradizionalmente, le variabili locali sono memorizzate sullo stack. 
Ci sono probabilmente altri modi per allocarle, ma in x86 e' cosi'.

\myindex{x86!\Instructions!PUSH}
Lo scopo dell'istruzione che segue il prologo della funzione, \TT{PUSH ECX}, non e' quello di salvare lo stato di \ECX  
(si noti infatti l'assenza della corrispondente istruzione \TT{POP ECX} alla fine della funzione).

Infatti alloca 4 byte sullo stack per memorizzare la variabile \TT{x}.

\label{stack_frame}
\myindex{\Stack!Stack frame}
\myindex{x86!\Registers!EBP}
\TT{x} sara' acceduta con l'aiuto della macro \TT{\_x\$} (che e' uguale a -4) ed il registro \EBP che punta al frame corrente.

Durante l'esecuzione delle funziona, \EBP punta allo \gls{stack frame} corrente 
rendendo possibile accedere alle variabili locali ed agli argomenti della funzione attraverso \TT{EBP+offset}.

\myindex{x86!\Registers!ESP}
E' anche possibile usare \ESP per lo stesso scopo, tuttavia non e' molto conveniente poiche' cambia di frequente.
Il valore di \EBP puo' essere pensato come uno \emph{stato congelato} del valore in \ESP all'inizio dell'esecuzione della funzione.

% FIXME1 это уже было в 02_stack?
Questo e' un tipico layout di uno \gls{stack frame} in un ambiente a 32-bit:

\begin{center}
\begin{tabular}{ | l | l | }
\hline
\dots & \dots \\
\hline
EBP-8 & local variable \#2, \MarkedInIDAAs{} \TT{var\_8} \\
\hline
EBP-4 & local variable \#1, \MarkedInIDAAs{} \TT{var\_4} \\
\hline
EBP & saved value of \EBP \\
\hline
EBP+4 & return address \\
\hline
EBP+8 & \argument \#1, \MarkedInIDAAs{} \TT{arg\_0} \\
\hline
EBP+0xC & \argument \#2, \MarkedInIDAAs{} \TT{arg\_4} \\
\hline
EBP+0x10 & \argument \#3, \MarkedInIDAAs{} \TT{arg\_8} \\
\hline
\dots & \dots \\
\hline
\end{tabular}
\end{center}

La funzione \scanf nel nostro esempio ha due argomenti.
Il primo e' un puntatore alla stringa contenente \TT{\%d} e il secondo e' l'indirizzo della variabile \TT{x}.

\myindex{x86!\Instructions!LEA}
Per prima cosa l'indirizzo della variabile \TT{x} e' caricato nel registro \EAX dall'istruzione \TT{lea eax, DWORD PTR \_x\$[ebp]}.

\LEA sta per \emph{load effective address}, ed e' spesso usata per formare un indirizzo ~(\myref{sec:LEA}).

Potremmo dire che in questo caso \LEA memorizza semplicemente la somma del valore nel registro \EBP e della macro \TT{\_x\$} nel registro \EAX.

E' l'equivalente di \INS{lea eax, [ebp-4]}.

Quindi, 4 viene sottratto dal valore del registro \EBP ed il risultato e' memorizzato nel registro \EAX.
Successivamente il registro \EAX e' messo sullo stack (push) e \scanf viene chiamata.

\printf viene chiamata subito dopo con il suo primo argomento --- un puntatore alla stringa:
\TT{You entered \%d...\textbackslash{}n}.

Il secondo argomento e' preparato con: \TT{mov ecx, [ebp-4]}.
L'istruzione memorizza il valore della variabile \TT{x},  non il suo indirizzo, nel registro \ECX.

Successivamente il valore in \ECX e' memorizzato sullo stack e l'ultima \printf viene chiamata.

\EN{\input{patterns/04_scanf/1_simple/olly_EN}}
\RU{\input{patterns/04_scanf/1_simple/olly_RU}}
\IT{\input{patterns/04_scanf/1_simple/olly_IT}}
\DE{\input{patterns/04_scanf/1_simple/olly_DE}}
\FR{\input{patterns/04_scanf/1_simple/olly_FR}}
\JA{\input{patterns/04_scanf/1_simple/olly_JA}}


\myparagraph{GCC}

Proviamo a compilare questo codice con GCC 4.4.1 su Linux:

\lstinputlisting[style=customasmx86]{patterns/04_scanf/1_simple/ex1_GCC.asm}

\myindex{puts() instead of printf()}
GCC ha sostituito la chiamata a \printf con \puts. La ragione per cui cio' avviene e' stata spiegata in ~(\myref{puts}).

% TODO: rewrite
%\RU{Почему \scanf переименовали в \TT{\_\_\_isoc99\_scanf}, я честно говоря, пока не знаю.}
%\EN{Why \scanf is renamed to \TT{\_\_\_isoc99\_scanf}, I do not know yet.}
% 
% Apparently it has to do with the ISO c99 standard compliance. By default GCC allows specifying a standard to adhere to.
% For example if you compile with -std=c89 the outputted assmebly file will contain scanf and not __isoc99__scanf. I guess current GCC version adhares to c99 by default.
% According to my understanding the two implementations differ in the set of suported modifyers (See printf man page)

Come nell'esempio compilato con MSVC ---gli argomenti sono messi sullo stack utilizzando l'istruzione \MOV.


}
\DE{\subsubsection{x86}

\myparagraph{MSVC}
Den folgenden Code erhalten wie nach dem Kompilieren mit MSVC 2010:

\lstinputlisting[style=customasmx86]{patterns/04_scanf/1_simple/ex1_MSVC_DE.asm}

\TT{x} ist eine lokale Variable.

Gemäß dem \CCpp-Standard darf diese nur innerhalb dieser Funktion sichtbar sein und nicht aus einem anderen, äußeren Scope.
Traditionell werden lokale Variablen auf dem Stack gespeichert.
Es gibt möglicherweise andere Wege sie anzulegen, aber in x86 geschieht es auf diese Weise.


\myindex{x86!\Instructions!PUSH}
Das Ziel des Befehls direkt nach dem Funktionsprolog, \TT{PUSH ECX}), ist es nicht, den Status von \ECX zu sichern
(man beachte, dass Fehlen eines entsprechenden \TT{POP ECX} im Funktionsepilog).
Tatsächlich reserviert der Befehl 4 Byte auf dem Stack, um die Variable $x$ speichern zu können.

\label{stack_frame}
\myindex{\Stack!Stack frame}
\myindex{x86!\Registers!EBP}
Auf \TT{x} wird mithilfe des \TT{\_x\$} Makros (es entspricht -4) und des \EBP Registers, das auf den aktuellen Stack Frame zeigt, zugegriffen. 
Während der Dauer der Funktionsausführung zeigt \EBP auf den aktuellen \glslink{stack frame}{Stack Frame}, wodurch mittels \TT{EBP+offset} auf lokalen Variablen und Funktionsargumente zugegriffen werden kann.

\TT{x} is to be accessed with the assistance of the \TT{\_x\$} macro (it equals to -4) and the \EBP register pointing to the current frame.

\myindex{x86!\Registers!ESP}
Es ist auch möglich, das \ESP Register zu diesem Zweck zu verwenden, aber dies ist ungebräuchlich, da es sich häufig verändert.
Der Wert von \EBP kann als eingefrorener Wert des Wertes von \ESP zu Beginn der Funktionsausführung verstanden werden.

It is also possible to use \ESP for the same purpose, although that is not very convenient since it changes frequently.
The value of the \EBP could be perceived as a \emph{frozen state} of the value in \ESP at the start of the function's execution.

% FIXME1 это уже было в 02_stack?
Hier ist ein typisches Layour eines Stack Frames in einer 32-Bit-Umgebung:

\begin{center}
\begin{tabular}{ | l | l | }
\hline
\dots & \dots \\
\hline
EBP-8 & local variable \#2, \MarkedInIDAAs{} \TT{var\_8} \\
\hline
EBP-4 & local variable \#1, \MarkedInIDAAs{} \TT{var\_4} \\
\hline
EBP & saved value of \EBP \\
\hline
EBP+4 & return address \\
\hline
EBP+8 & \argument \#1, \MarkedInIDAAs{} \TT{arg\_0} \\
\hline
EBP+0xC & \argument \#2, \MarkedInIDAAs{} \TT{arg\_4} \\
\hline
EBP+0x10 & \argument \#3, \MarkedInIDAAs{} \TT{arg\_8} \\
\hline
\dots & \dots \\
\hline
\end{tabular}
\end{center}
Die Funktion \scanf in unserem Beispiel hat zwei Argumente.

Das erste ist ein Pointer auf den String \TT{\%d} und das zweite ist die Adresse der Variablen \TT{x}.

\myindex{x86!\Instructions!LEA}
Zunächst wird die Adresse der Variablen $x$ durch den Befehl \\
\TT{lea eax, DWORD PTR \_x\$[ebp]} in das \EAX Register geladen.

\LEA steht für \emph{load effective address} und wird häufig benutzt, um eine Adresse zu erstellen ~(\myref{sec:LEA}).
In diesem Fall speichert \LEA einfach die Summe des \EBP Registers und des \TT{\_\$} Makros im Register \EAX.
Dies entspricht dem Befehl \INS{lea eax, [ebp-4]}.

Es wird also 4 von Wert in \EBP abgezogen und das Ergebnis in das Register \EAX geladen.
Danach wird der Wert in \EAX auf dem Stack abgelegt und \scanf wird aufgerufen.

Anschließend wird \printf mit einem Argument aufgerufen--einen Pointer auf den String:
\TT{You entered \%d...\textbackslash{}n}.

Das zweite Argument wird mit \TT{mov ecx, [ebp-4]} vorbereitet.
Dieser Befehl speichert den Wert der Variablen $x$ (nicht seine Adresse) im Register \ECX.

Schließlich wird der Wert in \ECX auf dem Stack gespeichert und das letzte \printf wird aufgerufen.

\EN{\input{patterns/04_scanf/1_simple/olly_EN}}
\RU{\input{patterns/04_scanf/1_simple/olly_RU}}
\IT{\input{patterns/04_scanf/1_simple/olly_IT}}
\DE{\input{patterns/04_scanf/1_simple/olly_DE}}
\FR{\input{patterns/04_scanf/1_simple/olly_FR}}
\JA{\input{patterns/04_scanf/1_simple/olly_JA}}


\myparagraph{GCC}

Kompilieren wir diesen Code mit GCC 4.4.1 unter Linux:

\lstinputlisting[style=customasmx86]{patterns/04_scanf/1_simple/ex1_GCC.asm}

\myindex{puts() instead of printf()}
GCC ersetzt den Aufruf von \printf durch einen Aufruf von \puts. Der Grund hierfür wurde bereits in ~(\myref{puts}) erklärt.

% TODO: rewrite
%\RU{Почему \scanf переименовали в \TT{\_\_\_isoc99\_scanf}, я честно говоря, пока не знаю.}
%\EN{Why \scanf is renamed to \TT{\_\_\_isoc99\_scanf}, I do not know yet.}
% 
% Apparently it has to do with the ISO c99 standard compliance. By default GCC allows specifying a standard to adhere to.
% For example if you compile with -std=c89 the outputted assmebly file will contain scanf and not __isoc99__scanf. I guess current GCC version adhares to c99 by default.
% According to my understanding the two implementations differ in the set of suported modifyers (See printf man page)
Genau wie im MSVC Beispiel werden die Argumente mithilfe des Befehls \MOV auf dem Stack abgelegt.

\myparagraph{By the way}
Dieses einfache Beispiel ist übrigens eine Demonstration der Tatsache, dass der Compiler eine Liste von Ausdrücken in einem
\CCpp-Block in eine sequentielle Liste von Befehlen übersetzt.
Es gibt nichts zwischen zwei \CCpp-Anweisungen und genauso verhält es sich auch im Maschinencode.
Der Control Flow geht von einem Ausdruck direkt an den folgenden über.
}
\FR{\subsubsection{x86}

\myparagraph{MSVC}

Voici ce que l'on obtient après avoir compilé avec MSVC 2010:

\lstinputlisting[style=customasmx86]{patterns/04_scanf/1_simple/ex1_MSVC_FR.asm}

\TT{x} est une variable locale.

D'après le standard \CCpp elle ne doit être visible que dans cette fonction et dans
aucune autre portée.
Traditionnellement, les variables locales sont stockées sur la pile.
Il y a probablement d'autres moyens de les allouer, mais en x86, c'est la façon de faire.

\myindex{x86!\Instructions!PUSH}
Le but de l'instruction suivant le prologue de la fonction, \TT{PUSH ECX}, n'est
pas de sauver l'état de \ECX (noter l'absence d'un \TT{POP ECX} à la fin de la
fonction).

En fait, cela alloue 4 octets sur la pile pour stocker la variable \TT{x}.

\label{stack_frame}
\myindex{\Stack!Stack frame}
\myindex{x86!\Registers!EBP}
\TT{x} est accédée à l'aide de la macro \TT{\_x\$} (qui vaut -4) et du registre \EBP
qui pointe sur la structure de pile courante.

Pendant la durée de l'exécution de la fonction, \EBP pointe sur la \glslink{stack frame}{structure locale de pile}
courante, rendant possible l'accès aux variables locales et aux arguments de la
fonction via \TT{EBP+offset}.

\myindex{x86!\Registers!ESP}
Il est aussi possible d'utiliser \ESP dans le même but, bien que ça ne soit pas
très commode, car il change fréquemment.
La valeur de \EBP peut être perçue comme un \emph{état figé} de la valeur de \ESP
au début de l'exécution de la fonction.

% FIXME1 это уже было в 02_stack?
Voici une \glslink{stack frame}{structure de pile} typique dans un environnement 32-bit:

\begin{center}
\begin{tabular}{ | l | l | }
\hline
\dots & \dots \\
\hline
EBP-8 & variable locale \#2, \MarkedInIDAAs{} \TT{var\_8} \\
\hline
EBP-4 & variable locale \#1, \MarkedInIDAAs{} \TT{var\_4} \\
\hline
EBP & valeur sauvée de \EBP \\
\hline
EBP+4 & adresse de retour \\
\hline
EBP+8 & \argument \#1, \MarkedInIDAAs{} \TT{arg\_0} \\
\hline
EBP+0xC & \argument \#2, \MarkedInIDAAs{} \TT{arg\_4} \\
\hline
EBP+0x10 & \argument \#3, \MarkedInIDAAs{} \TT{arg\_8} \\
\hline
\dots & \dots \\
\hline
\end{tabular}
\end{center}

La fonction \scanf de notre exemple a deux arguments.

Le premier est un pointeur sur la chaîne contenant \TT{\%d} et le second est l'adresse
de la variable \TT{x}.

\myindex{x86!\Instructions!LEA}
Tout d'abord, l'adresse de la variable \TT{x} est chargée dans le registre \EAX
par l'instruction \\ \TT{lea eax, DWORD PTR \_x\$[ebp]}.

\LEA signifie \emph{load effective address} (charger l'adresse effective) et est souvent
utilisée pour composer une adresse ~(\myref{sec:LEA}).

Nous pouvons dire que dans ce cas, \LEA stocke simplement la somme de la valeur du
registre \EBP et de la macro \TT{\_x\$} dans le registre \EAX.

C'est la même chose que \INS{lea eax, [ebp-4]}.

Donc, 4 est soustrait de la valeur du registre \EBP et le résultat est chargé dans
le registre \EAX.
Ensuite, la valeur du registre \EAX est poussée sur la pile et \scanf est appelée.

\printf est appelée ensuite avec son premier argument --- un pointeur sur la chaîne:
\TT{You entered \%d...\textbackslash{}n}.

Le second argument est préparé avec: \TT{mov ecx, [ebp-4]}.
L'instruction stocke la valeur de la variable \TT{x} et non son adresse, dans le
registre \ECX.

Puis, la valeur de \ECX est stockée sur la pile et le dernier appel à \printf
est effectué.

\EN{\input{patterns/04_scanf/1_simple/olly_EN}}
\RU{\input{patterns/04_scanf/1_simple/olly_RU}}
\IT{\input{patterns/04_scanf/1_simple/olly_IT}}
\DE{\input{patterns/04_scanf/1_simple/olly_DE}}
\FR{\input{patterns/04_scanf/1_simple/olly_FR}}
\JA{\input{patterns/04_scanf/1_simple/olly_JA}}


\myparagraph{GCC}

Compilons ce code avec GCC 4.4.1 sous Linux:

\lstinputlisting[style=customasmx86]{patterns/04_scanf/1_simple/ex1_GCC.asm}

\myindex{puts() instead of printf()}
GCC a remplacé l'appel à \printf avec un appel à \puts. La raison de cela a été
expliquée dans ~(\myref{puts}).

% TODO: rewrite
%\RU{Почему \scanf переименовали в \TT{\_\_\_isoc99\_scanf}, я честно говоря, пока не знаю.}
%\EN{Why \scanf is renamed to \TT{\_\_\_isoc99\_scanf}, I do not know yet.}
% 
% Apparently it has to do with the ISO c99 standard compliance. By default GCC allows specifying a standard to adhere to.
% For example if you compile with -std=c89 the outputted assmebly file will contain scanf and not __isoc99__scanf. I guess current GCC version adhares to c99 by default.
% According to my understanding the two implementations differ in the set of suported modifyers (See printf man page)

Comme dans l'exemple avec MSVC---les arguments sont placés dans la pile avec l'instruction
\MOV.

\myparagraph{À propos}

Ce simple exemple est la démonstration du fait que le compilateur traduit
une liste d'expression en bloc-\CCpp en une liste séquentielle d'instructions.
% TODO FIXME: better translation / clarify ?
Il n'y a rien entre les expressions en \CCpp, et le résultat en code machine,
il n'y a rien entre le déroulement du flux de contrôle d'une expression à la suivante.
}
\PL{\subsubsection{x86}

\myparagraph{MSVC}

Tutaj znajduje się wynik kompilacji programu w MSVC 2010:

\lstinputlisting[style=customasmx86]{patterns/04_scanf/1_simple/ex1_MSVC_EN.asm}

\TT{x} jest zmienną lokalną.

Według standardu \CCpp zmienna lokalna może być widoczna tylko w konkretnej funkcji. Tradycyjnie zmienne lokalne są przechowywane na stosie. Prawdopodnie są inne moliwości przechowywania tych zmiennych, ale tak akurat jest w x86.

\myindex{x86!\Instructions!PUSH}
Zadaniem instrukcji rozpoczynającej funkcję, \TT{PUSH ECX}, nie jest zapisanie stanu \ECX (można zauważyć brak odpowiadającej instrukcji POP ECX na końcu funkcji).

Tak naprawdę instrukcja ta alokuje 4 bajty na stosie do przechowania zmiennej x.

\label{stack_frame}
\myindex{\Stack!Stack frame}
\myindex{x86!\Registers!EBP}
Dostęp do \TT{x} odbywa się z asystującym makrem \TT{\_x\$} (równym -4) i rejestrem \EBP rwskazującym bieżącą ramkę.

We fragmencie wykonującym funkcje, \EBP wskazuje bieżącą \gls{stack frame}
umożliwiając dostęp do zmiennych lokalnych i argumentów funkcji poprzez \TT{EBP+offset}.

\myindex{x86!\Registers!ESP}
Możliwe jest także użycie \ESP w takim samym celu, le nie jest to zbyt wygodne, ponieważ wartość tego rejestru często się zmienia.
Wartość \EBP może być postrzegana jako \emph{frozen state} wartości w \ESP z początku wykonania funkcji.

% FIXME1 это уже было в 02_stack?
Tutaj znajduje się typowa ramka stosu w układzie środowiska 32-bitowego:

\begin{center}
\begin{tabular}{ | l | l | }
\hline
\dots & \dots \\
\hline
EBP-8 & zmienna lokalna \#2, \MarkedInIDAAs{} \TT{var\_8} \\
\hline
EBP-4 & zmienna lokalna \#1, \MarkedInIDAAs{} \TT{var\_4} \\
\hline
EBP & zapisana wartość \EBP \\
\hline
EBP+4 & adres powrotu \\
\hline
EBP+8 & \argument \#1, \MarkedInIDAAs{} \TT{arg\_0} \\
\hline
EBP+0xC & \argument \#2, \MarkedInIDAAs{} \TT{arg\_4} \\
\hline
EBP+0x10 & \argument \#3, \MarkedInIDAAs{} \TT{arg\_8} \\
\hline
\dots & \dots \\
\hline
\end{tabular}
\end{center}

Funkcja \scanf w naszym przykładzie ma dwa argumenty.

Pierwszy jest wskaźnikiem na string \TT{\%d} a drugi jest adresem zmiennej \TT{x}.

\myindex{x86!\Instructions!LEA}
Na początku adres zmiennej \TT{x} jest ładowany do rejestru \EAX przy pomocy instrukcji \\
\TT{lea eax, DWORD PTR \_x\$[ebp]}.

\LEA oznacza \emph{load effective address} i jest często używana do formowania adresów ~(\myref{sec:LEA}).

Można powiedzieć, że w tym przypadku \LEA po prostu umieszcza sumę rejestru \EBP i makra \TT{\_x\$} w rejestrze \EAX.

To jest to samo co \INS{lea eax, [ebp-4]}.

Więc od rejestru \EBP jest odejmowane 4 i wynik zostaje umieszczony w rejestrze \EAX.
Następnie wartość rejestru \EAX jest odkładana na stosie i funkcja \scanf zostaje wywołana.

\printf wywołuje się z pierwszym argumentem- wskaźnikiem na string:
\TT{You entered \%d...\textbackslash{}n}.

Drugi argument jest przygotowywany za pomocą: \TT{mov ecx, [ebp-4]}.
Instrukcja kopiuje zmienną \TT{x} (nie jej adres) do rejestru \ECX.

Następnie wartość z \ECX jest odkładana na stos, a na koniec zostaje wywołana funkcja  \printf.

\EN{\input{patterns/04_scanf/1_simple/olly_EN}}
\RU{\input{patterns/04_scanf/1_simple/olly_RU}}
\IT{\input{patterns/04_scanf/1_simple/olly_IT}}
\DE{\input{patterns/04_scanf/1_simple/olly_DE}}
\FR{\input{patterns/04_scanf/1_simple/olly_FR}}
\JA{\input{patterns/04_scanf/1_simple/olly_JA}}


\myparagraph{GCC}

Tak wygląda skompilowany kod w GCC 4.4.1 w systemie Linux:

\lstinputlisting[style=customasmx86]{patterns/04_scanf/1_simple/ex1_GCC.asm}

\myindex{puts() instead of printf()}
GCC zamienia wywołanie funkcji \printf na wywołanie funkcji \puts. Powód tego został wyjaśniony w ~(\myref{puts}).

% TODO: rewrite
%\RU{Почему \scanf переименовали в \TT{\_\_\_isoc99\_scanf}, я честно говоря, пока не знаю.}
%\EN{Why \scanf is renamed to \TT{\_\_\_isoc99\_scanf}, I do not know yet.}
% 
% Apparently it has to do with the ISO c99 standard compliance. By default GCC allows specifying a standard to adhere to.
% For example if you compile with -std=c89 the outputted assmebly file will contain scanf and not __isoc99__scanf. I guess current GCC version adhares to c99 by default.
% According to my understanding the two implementations differ in the set of suported modifyers (See printf man page)

Jak w przykładzie MSVC---argumenty funkcji są umieszczane na stosie przy użyciu instrukcji \MOV.

\myparagraph{By the way}

Ten prosty przykład pokazuje jak faktycznie kompilatory tłumaczą
listy wyrażeń w \CCpp-block na sekwencyjne listy instrukcji.
Nie ma nic pomiędzy wyrażeniami w \CCpp a wynikowym kodem maszynowym.
}
\JPN{\subsubsection{x86}

\myparagraph{MSVC}

MSVC 2010でコンパイルした後に得られるものは次のとおりです。

\lstinputlisting[style=customasmx86]{patterns/04_scanf/1_simple/ex1_MSVC_JPN.asm}

\TT{x}はローカル変数です。

\CCpp 標準によれば、この関数でのみ表示でき、他の外部スコープでは表示できません。
従来、ローカル変数はスタックに格納されていました。
それらを割り当てる方法はおそらく他にもありますが、それはx86の方法です。

\myindex{x86!\Instructions!PUSH}
関数プロローグ、\TT{PUSH ECX}に続く命令の目的は、 \ECX 状態を保存することではありません
(関数の最後に対応する\TT{POP ECX}が存在しないことに注意してください)。

実際、\TT{x}変数を格納するためにスタックに4バイトを割り当てます。

\label{stack_frame}
\myindex{\Stack!Stack frame}
\myindex{x86!\Registers!EBP}
\TT{x}は、\TT{\_x\$} マクロ (-4に等しい)と現在のフレームを指す \EBP レジスタの助けを借りてアクセスされます。

関数の実行の範囲にわたって、 \EBP は現在の\gls{stack frame}を指しており、 \TT{EBP+オフセット}
を介してローカル変数と関数引数にアクセスすることができます。

\myindex{x86!\Registers!ESP}
同じ目的で \ESP を使用することもできますが、 \ESP は頻繁に変更されるためあまり便利ではありません。 
\EBP の値は、関数の実行開始時に \ESP の値が固定された状態として認識される可能性があります。

% FIXME1 это уже было в 02_stack?
32ビット環境での典型的な\gls{stack frame}レイアウトを次に示します。

\begin{center}
\begin{tabular}{ | l | l | }
\hline
\dots & \dots \\
\hline
EBP-8 & local variable \#2, \MarkedInIDAAs{} \TT{var\_8} \\
\hline
EBP-4 & local variable \#1, \MarkedInIDAAs{} \TT{var\_4} \\
\hline
EBP & saved value of \EBP \\
\hline
EBP+4 & return address \\
\hline
EBP+8 & \argument \#1, \MarkedInIDAAs{} \TT{arg\_0} \\
\hline
EBP+0xC & \argument \#2, \MarkedInIDAAs{} \TT{arg\_4} \\
\hline
EBP+0x10 & \argument \#3, \MarkedInIDAAs{} \TT{arg\_8} \\
\hline
\dots & \dots \\
\hline
\end{tabular}
\end{center}

この例の \scanf 関数には2つの引数があります。

最初のものは\TT{\%d}を含む文字列へのポインタで、2番目のものは\TT{x}変数のアドレスです。

\myindex{x86!\Instructions!LEA}
最初に、\TT{x}変数のアドレスが\TT{lea eax, DWORD PTR \_x\$[ebp]}命令によって \EAX レジスタにロードされます。

\LEA は\emph{ロード実効アドレス}の略で、アドレスを形成するためによく使用されます(~(\myref{sec:LEA}))。

この場合、\LEA は単に \EBP レジスタ値と\TT{\_x\$}マクロの合計を \EAX レジスタに格納すると言うことができます。

これは\INS{lea eax, [ebp-4]}と同じです。

したがって、 \EBP レジスタ値から4が減算され、その結果が \EAX レジスタにロードされます。
次に、 \EAX レジスタの値がスタックにプッシュされ、 \scanf が呼び出されます。

\printf は最初の引数で呼び出されています。文字列へのポインタ:
\TT{You entered \%d...\textbackslash{}n}

2番目の引数は\TT{mov ecx, [ebp-4]}で準備されています。
命令は、 \ECX レジスタにそのアドレスではなく\TT{x}変数値を格納します。

次に、 \ECX の値がスタックに格納され、最後の \printf が呼び出されます。

\EN{\input{patterns/04_scanf/1_simple/olly_EN}}
\RU{\input{patterns/04_scanf/1_simple/olly_RU}}
\IT{\input{patterns/04_scanf/1_simple/olly_IT}}
\DE{\input{patterns/04_scanf/1_simple/olly_DE}}
\FR{\input{patterns/04_scanf/1_simple/olly_FR}}
\JA{\input{patterns/04_scanf/1_simple/olly_JA}}


\myparagraph{GCC}

Linux上のGCC 4.4.1でこのコードをコンパイルしようとしましょう。

\lstinputlisting[style=customasmx86]{patterns/04_scanf/1_simple/ex1_GCC.asm}

\myindex{puts() instead of printf()}
GCCは \printf 呼び出しを \puts の呼び出しで置き換えました。 この理由は、~(\myref{puts})で説明されました。

% TODO: rewrite
%\RU{Почему \scanf переименовали в \TT{\_\_\_isoc99\_scanf}, я честно говоря, пока не знаю.}
%\EN{Why \scanf is renamed to \TT{\_\_\_isoc99\_scanf}, I do not know yet.}
% 
% Apparently it has to do with the ISO c99 standard compliance. By default GCC allows specifying a standard to adhere to.
% For example if you compile with -std=c89 the outputted assmebly file will contain scanf and not __isoc99__scanf. I guess current GCC version adhares to c99 by default.
% According to my understanding the two implementations differ in the set of suported modifyers (See printf man page)

MSVCの例のように、引数は \MOV 命令を使用してスタックに配置されます。

\myparagraph{ところで}

ところで、この単純な例は、コンパイラが \CCpp ブロックの式のリストを命令の連続したリストに
変換するという事実のデモンストレーションです。
\CCpp の式の間には何もないので、結果のマシンコードには、
ある式から次の式への制御フローの間には何もありません。
}

\EN{\subsubsection{x64: 8 arguments}

\myindex{x86-64}
\label{example_printf8_x64}
To see how other arguments are passed via the stack, let's change our example again 
by increasing the number of arguments to 9 (\printf format string + 8 \Tint variables):

\lstinputlisting[style=customc]{patterns/03_printf/2.c}

\myparagraph{MSVC}

As it was mentioned earlier, the first 4 arguments has to be passed through the \RCX, \RDX, \Reg{8}, \Reg{9} registers in Win64, while all the rest---via the stack.
That is exactly what we see here.
However, the \MOV instruction, instead of \PUSH, is used for preparing the stack, so the values are stored to the stack in a straightforward manner.

\lstinputlisting[caption=MSVC 2012 x64,style=customasmx86]{patterns/03_printf/x86/2_MSVC_x64_EN.asm}

The observant reader may ask why are 8 bytes allocated for \Tint values, when 4 is enough?
Yes, one has to recall: 8 bytes are allocated for any data type shorter than 64 bits.
This is established for the convenience's sake: it makes it easy to calculate the address of arbitrary argument.
Besides, they are all located at aligned memory addresses.
It is the same in the 32-bit environments: 4 bytes are reserved for all data types.

% also for local variables?

\myparagraph{GCC}

The picture is similar for x86-64 *NIX OS-es, except that the first 6 arguments are passed through the \RDI, \RSI,
\RDX, \RCX, \Reg{8}, \Reg{9} registers.
All the rest---via the stack.
GCC generates the code storing the string pointer into \EDI instead of \RDI{}---we noted that previously: 
\myref{hw_EDI_instead_of_RDI}.

We also noted earlier that the \EAX register has been cleared before a \printf call: \myref{SysVABI_input_EAX}.

\lstinputlisting[caption=\Optimizing GCC 4.4.6 x64,style=customasmx86]{patterns/03_printf/x86/2_GCC_x64_EN.s}

\myparagraph{GCC + GDB}
\myindex{GDB}

Let's try this example in \ac{GDB}.

\begin{lstlisting}
$ gcc -g 2.c -o 2
\end{lstlisting}

\begin{lstlisting}
$ gdb 2
GNU gdb (GDB) 7.6.1-ubuntu
...
Reading symbols from /home/dennis/polygon/2...done.
\end{lstlisting}

\begin{lstlisting}[caption=let's set the breakpoint to \printf{,} and run]
(gdb) b printf
Breakpoint 1 at 0x400410
(gdb) run
Starting program: /home/dennis/polygon/2 

Breakpoint 1, __printf (format=0x400628 "a=%d; b=%d; c=%d; d=%d; e=%d; f=%d; g=%d; h=%d\n") at printf.c:29
29	printf.c: No such file or directory.
\end{lstlisting}

Registers \RSI/\RDX/\RCX/\Reg{8}/\Reg{9} have the expected values.
\RIP has the address of the very first instruction of the \printf function.

\begin{lstlisting}
(gdb) info registers
rax            0x0	0
rbx            0x0	0
rcx            0x3	3
rdx            0x2	2
rsi            0x1	1
rdi            0x400628	4195880
rbp            0x7fffffffdf60	0x7fffffffdf60
rsp            0x7fffffffdf38	0x7fffffffdf38
r8             0x4	4
r9             0x5	5
r10            0x7fffffffdce0	140737488346336
r11            0x7ffff7a65f60	140737348263776
r12            0x400440	4195392
r13            0x7fffffffe040	140737488347200
r14            0x0	0
r15            0x0	0
rip            0x7ffff7a65f60	0x7ffff7a65f60 <__printf>
...
\end{lstlisting}

\begin{lstlisting}[caption=let's inspect the format string]
(gdb) x/s $rdi
0x400628:	"a=%d; b=%d; c=%d; d=%d; e=%d; f=%d; g=%d; h=%d\n"
\end{lstlisting}

Let's dump the stack with the x/g command this time---\emph{g} stands for \emph{giant words}, i.e., 64-bit words.

\begin{lstlisting}
(gdb) x/10g $rsp
0x7fffffffdf38:	0x0000000000400576	0x0000000000000006
0x7fffffffdf48:	0x0000000000000007	0x00007fff00000008
0x7fffffffdf58:	0x0000000000000000	0x0000000000000000
0x7fffffffdf68:	0x00007ffff7a33de5	0x0000000000000000
0x7fffffffdf78:	0x00007fffffffe048	0x0000000100000000
\end{lstlisting}

The very first stack element, just like in the previous case, is the \ac{RA}.
3 values are also passed through the stack: 6, 7, 8.
We also see that 8 is passed with the high 32-bits not cleared: \GTT{0x00007fff00000008}.
That's OK, because the values are of \Tint type, which is 32-bit.
So, the high register or stack element part may contain \q{random garbage}.

If you take a look at where the control will return after the \printf execution,
\ac{GDB} will show the entire \main function:

\begin{lstlisting}[style=customasmx86]
(gdb) set disassembly-flavor intel
(gdb) disas 0x0000000000400576
Dump of assembler code for function main:
   0x000000000040052d <+0>:	push   rbp
   0x000000000040052e <+1>:	mov    rbp,rsp
   0x0000000000400531 <+4>:	sub    rsp,0x20
   0x0000000000400535 <+8>:	mov    DWORD PTR [rsp+0x10],0x8
   0x000000000040053d <+16>:	mov    DWORD PTR [rsp+0x8],0x7
   0x0000000000400545 <+24>:	mov    DWORD PTR [rsp],0x6
   0x000000000040054c <+31>:	mov    r9d,0x5
   0x0000000000400552 <+37>:	mov    r8d,0x4
   0x0000000000400558 <+43>:	mov    ecx,0x3
   0x000000000040055d <+48>:	mov    edx,0x2
   0x0000000000400562 <+53>:	mov    esi,0x1
   0x0000000000400567 <+58>:	mov    edi,0x400628
   0x000000000040056c <+63>:	mov    eax,0x0
   0x0000000000400571 <+68>:	call   0x400410 <printf@plt>
   0x0000000000400576 <+73>:	mov    eax,0x0
   0x000000000040057b <+78>:	leave  
   0x000000000040057c <+79>:	ret    
End of assembler dump.
\end{lstlisting}

Let's finish executing \printf, execute the instruction
zeroing \EAX, and note that the \EAX register has a value of exactly zero.
\RIP now points to the \INS{LEAVE} instruction, i.e., the penultimate one in the \main function.

\begin{lstlisting}
(gdb) finish
Run till exit from #0  __printf (format=0x400628 "a=%d; b=%d; c=%d; d=%d; e=%d; f=%d; g=%d; h=%d\n") at printf.c:29
a=1; b=2; c=3; d=4; e=5; f=6; g=7; h=8
main () at 2.c:6
6		return 0;
Value returned is $1 = 39
(gdb) next
7	};
(gdb) info registers
rax            0x0	0
rbx            0x0	0
rcx            0x26	38
rdx            0x7ffff7dd59f0	140737351866864
rsi            0x7fffffd9	2147483609
rdi            0x0	0
rbp            0x7fffffffdf60	0x7fffffffdf60
rsp            0x7fffffffdf40	0x7fffffffdf40
r8             0x7ffff7dd26a0	140737351853728
r9             0x7ffff7a60134	140737348239668
r10            0x7fffffffd5b0	140737488344496
r11            0x7ffff7a95900	140737348458752
r12            0x400440	4195392
r13            0x7fffffffe040	140737488347200
r14            0x0	0
r15            0x0	0
rip            0x40057b	0x40057b <main+78>
...
\end{lstlisting}
}
\RU{\subsubsection{x64}
\label{subsec:popcnt}

Немного изменим пример, расширив его до 64-х бит:

\lstinputlisting[label=popcnt_x64_example,style=customc]{patterns/14_bitfields/4_popcnt/shifts64.c}

\myparagraph{\NonOptimizing GCC 4.8.2}

Пока всё просто.

\lstinputlisting[caption=\NonOptimizing GCC 4.8.2,style=customasmx86]{patterns/14_bitfields/4_popcnt/shifts64_GCC_O0_RU.s}

\myparagraph{\Optimizing GCC 4.8.2}

\lstinputlisting[caption=\Optimizing GCC 4.8.2,numbers=left,label=shifts64_GCC_O3,style=customasmx86]{patterns/14_bitfields/4_popcnt/shifts64_GCC_O3_RU.s}

Код более лаконичный, но содержит одну необычную вещь.
Во всех примерах, что мы пока видели, инкремент значения переменной \q{rt} происходит после сравнения 
определенного бита с единицей, но здесь \q{rt} увеличивается на 1 до этого (строка 6), записывая новое значение
в регистр \EDX.

Затем, если последний бит был 1, инструкция \CMOVNE\footnote{Conditional MOVe if Not Equal (\MOV если не равно)}
(которая синонимична \CMOVNZ\footnote{Conditional MOVe if Not Zero (\MOV если не ноль)}) \emph{фиксирует} 
новое значение \q{rt}
копируя значение из \EDX (\q{предполагаемое значение rt}) 
в \EAX (\q{текущее rt} которое будет возвращено в конце функции).
Следовательно, инкремент происходит на каждом шаге цикла, т.е. 64 раза, вне всякой связи с входным
значением.

Преимущество этого кода в том, что он содержит только один условный переход (в конце цикла) вместо
двух (пропускающий инкремент \q{rt} и ещё одного в конце цикла).

И это может работать быстрее на современных CPU с предсказателем переходов: \myref{branch_predictors}.

\label{FATRET}
\myindex{x86!\Instructions!FATRET}
Последняя инструкция это \INS{REP RET} (опкод \TT{F3 C3}) 
которая также называется \INS{FATRET} в MSVC.
Это оптимизированная версия \RET, рекомендуемая AMD для вставки в конце функции, если \RET идет
сразу после условного перехода: 
\InSqBrackets{\AMDOptimization p.15}
\footnote{Больше об этом: \url{http://go.yurichev.com/17328}}.

\myparagraph{\Optimizing MSVC 2010}

\lstinputlisting[caption=\Optimizing MSVC 2010,style=customasmx86]{patterns/14_bitfields/4_popcnt/MSVC_2010_x64_Ox_RU.asm}

\myindex{x86!\Instructions!ROL}
Здесь используется инструкция \ROL вместо 
\SHL, которая на самом деле \q{rotate left} (прокручивать влево) 
вместо \q{shift left} (сдвиг влево),
но здесь, в этом примере, она работает так же как и  \TT{SHL}.

Об этих \q{прокручивающих} инструкциях больше читайте здесь: \myref{ROL_ROR}.

\Reg{8} здесь считает от 64 до 0. 
Это как бы инвертированная переменная $i$.

Вот таблица некоторых регистров в процессе исполнения:

\begin{center}
\begin{tabular}{ | l | l | }
\hline
\HeaderColor RDX & \HeaderColor R8 \\
\hline
0x0000000000000001 & 64 \\
\hline
0x0000000000000002 & 63 \\
\hline
0x0000000000000004 & 62 \\
\hline
0x0000000000000008 & 61 \\
\hline
... & ... \\
\hline
0x4000000000000000 & 2 \\
\hline
0x8000000000000000 & 1 \\
\hline
\end{tabular}
\end{center}

\myindex{x86!\Instructions!FATRET}
В конце видим инструкцию \INS{FATRET}, которая была описана здесь: \myref{FATRET}.

\myparagraph{\Optimizing MSVC 2012}

\lstinputlisting[caption=\Optimizing MSVC 2012,style=customasmx86]{patterns/14_bitfields/4_popcnt/MSVC_2012_x64_Ox_RU.asm}

\myindex{\CompilerAnomaly}
\label{MSVC2012_anomaly}
\Optimizing MSVC 2012 делает почти то же самое что и оптимизирующий MSVC 2010, но почему-то он генерирует 2 идентичных тела цикла и счетчик цикла теперь 32
вместо 64.
Честно говоря, нельзя сказать, почему. Какой-то трюк с оптимизацией? Может быть, телу цикла лучше быть
немного длиннее?

Так или иначе, такой код здесь уместен, чтобы показать, что результат компилятора
иногда может быть очень странный и нелогичный, но прекрасно работающий, конечно же.

}
\PTBR{\input{patterns/04_scanf/1_simple/x64_PTBR}}
\IT{\subsubsection{x64}

\myindex{x86-64}
La situazione e' simile, con l'unica differenza che, per il passaggio degli argomenti, i registri sono usati al posto dello stack.

\myparagraph{MSVC}

\lstinputlisting[caption=MSVC 2012 x64,style=customasmx86]{patterns/04_scanf/1_simple/ex1_MSVC_x64_EN.asm}

\myparagraph{GCC}

\lstinputlisting[caption=\Optimizing GCC 4.4.6 x64,style=customasmx86]{patterns/04_scanf/1_simple/ex1_GCC_x64_EN.s}

}
\DE{\subsection{x64}

\myindex{x86-64}

Die Geschichte bei x86-64 Funktions Argumenten ist ein wenig anders (zumindest für die ersten vier bis sechs)
sie werden über die Register übergeben z.b. der \gls{callee} liest direkt aus den Registern anstatt vom Stack 
zu lesen.

\subsubsection{MSVC}

\Optimizing MSVC:

\lstinputlisting[caption=\Optimizing MSVC 2012 x64,style=customasmx86]{patterns/05_passing_arguments/x64_MSVC_Ox_EN.asm}

Wie wir sehen können, die compact Funktion \ttf nimmt alle Argumente aus den Registern.

Die \LEA Instruktion wird hier für Addition benutzt,
scheinbar hat der Compiler die Instruktion für schneller befunden als
die \TT{ADD} Instruktion.

\myindex{x86!\Instructions!LEA}

\LEA wird auch benutzt in der \main Funktion um das erste und das dritte \ttf Argument vor zu bereiten.
Der Compiler muss entschieden haben das dies schneller abgearbeitet wird als die Werte in die Register 
zu laden mit der \MOV Instruktion.

Lasst uns einen Blick auf nicht optimierte MSVC Ausgabe werfen:

\lstinputlisting[caption=MSVC 2012 x64,style=customasmx86]{patterns/05_passing_arguments/x64_MSVC_IDA_EN.asm}

Es sieht ein bisschen wie ein Puzzle aus, weil alle drei Argumente aus den Registern auf dem Stack
gespeichert werden aus irgend einem Grund.

\myindex{Shadow space}
\label{shadow_space}
Dies bezeichnet man als \q{shadow space}

\footnote{\href{http://go.yurichev.com/17256}{MSDN}}: 
So wird sich wahrscheinlich jede Win64 EXE verhalten und alle vier Register Werte auf dem Stack speichern.

Das wird aus zwei Gründen so gemacht:

1) Es ist ziemlich übertrieben ein ganzes Register (oder gar vier Register) zu Reservieren für eine
Argument Übergabe, also werden die Argumente über den Stack zugänglich gemacht.
2) Der Debugger weiß immer wo die Funktions Argumente zu finden sind bei einem breakpoint\footnote{\href{http://go.yurichev.com/17257}{MSDN}}.


Also, so können größere Funktionen ihre Eingabe Argumente im \q{shadows space} speichern wenn die Funktion
auf die Argumente während der Laufzeit zugreifen will, kleinere Funktionen (wie unsere) zeigen dieses Verhalten 
nicht. 

Es liegt in der Verantwortung vom \gls{caller} den \q{shadow space} auf dem Stack zu allozieren.

\subsubsection{GCC}

Optimierter GCC generiert mehr oder minder verständlichen Code:

\lstinputlisting[caption=\Optimizing GCC 4.4.6 x64,style=customasmx86]{patterns/05_passing_arguments/x64_GCC_O3_EN.s}

\NonOptimizing GCC:

\lstinputlisting[caption=GCC 4.4.6 x64,style=customasmx86]{patterns/05_passing_arguments/x64_GCC_EN.s}

\myindex{Shadow space}

Bei System V *NIX Systemen (\SysVABI) ist kein \q{shadow space} nötig, aber der \gls{callee} will vielleicht
seine Argumente irgendwo speichern im Fall das keine oder zu wenig Register frei sind.

\subsubsection{GCC: uint64\_t statt int}

Unser Beispiel funktioniert mit 32-Bit \Tint, weshalb auch 32-Bit Register Bereiche benutzt werden (mit dem Präfix \TT{E-}).

Es lassen sich auch ohne Probleme 64-Bit Werte benutzen:

\lstinputlisting{patterns/05_passing_arguments/ex64.c}

\lstinputlisting[caption=\Optimizing GCC 4.4.6 x64,style=customasmx86]{patterns/05_passing_arguments/ex64_GCC_O3_IDA_EN.asm}

Der Code ist der gleiche, aber diesmal werden die \IT{full size} 64-Bit Register benutzt (mit dem \TT{R-} Präfix).

}
\FR{\subsubsection{x64}
\label{subsec:popcnt}

Modifions légèrement l'exemple pour l'étendre à 64-bit:

\lstinputlisting[label=popcnt_x64_example,style=customc]{patterns/14_bitfields/4_popcnt/shifts64.c}

\myparagraph{GCC 4.8.2 \NonOptimizing}

Jusqu'ici, c'est facile.

\lstinputlisting[caption=GCC 4.8.2 \NonOptimizing,style=customasmx86]{patterns/14_bitfields/4_popcnt/shifts64_GCC_O0_FR.s}

\myparagraph{GCC 4.8.2 \Optimizing}

\lstinputlisting[caption=GCC 4.8.2 \Optimizing,numbers=left,label=shifts64_GCC_O3,style=customasmx86]{patterns/14_bitfields/4_popcnt/shifts64_GCC_O3_FR.s}

Ce code est plus concis, mais a une particularité.

% TODO: comment traduire commits ?
Dans tous les exemples que nous avons vu jusqu'ici, nous incrémentions la valeur
de \q{rt} après la comparaison d'un bit spécifique, mais le code ici incrémente \q{rt}
avant (ligne 6), écrivant la nouvelle valeur dans le registre \EDX.
Donc, si le dernier bit est à 1, l'instruction \CMOVNE\footnote{Conditional MOVe if Not Equal}
(qui est un synonyme pour \CMOVNZ\footnote{Conditional MOVe if Not Zero}) \emph{commits}
la nouvelle valeur de \q{rt} en déplaçant  \EDX (\q{valeur proposée de rt}) dans
\EAX (\q{rt courant} qui va être retourné à la fin).

C'est pourquoi l'incrémentation est effectuée à chaque étape de la boucle, i.e.,
64 fois, sans relation avec la valeur en entrée.

L'avantage de ce code est qu'il contient seulement un saut conditionnel (à la fin
de la boucle) au lieu de deux sauts (évitant l'incrément de la valeur de \q{rt} et
à la fin de la boucle).
Et cela doit s'exécuter plus vite sur les CPUs modernes avec des prédicteurs de branchement:
\myref{branch_predictors}.

\label{FATRET}
\myindex{x86!\Instructions!FATRET}
La dernière instruction est \INS{REP RET} (opcode \TT{F3 C3}) qui est aussi appelée
\INS{FATRET} par MSVC.
C'est en quelque sorte une version optimisée de \RET, qu'AMD recommande de mettre
en fin de fonction, si \RET se trouve juste après un saut conditionnel:
\InSqBrackets{\AMDOptimization p.15}
\footnote{Lire aussi à ce propos: \url{http://go.yurichev.com/17328}}.

\myparagraph{MSVC 2010 \Optimizing}

\lstinputlisting[caption=MSVC 2010 \Optimizing,style=customasmx86]{patterns/14_bitfields/4_popcnt/MSVC_2010_x64_Ox_FR.asm}

\myindex{x86!\Instructions!ROL}
Ici l'instruction \ROL est utilisée au lieu de \SHL, qui est en fait \q{rotate left /
pivoter à gauche} au lieu de \q{shift left / décaler à gauche}, mais dans cet exemple
elle fonctionne tout comme \TT{SHL}.

Vous pouvez en lire plus sur l'instruction de rotation ici: \myref{ROL_ROR}.

\Reg{8} ici est compté de 64 à 0.
C'est tout comme un $i$ inversé.

Voici une table de quelques registres pendant l'exécution:

\begin{center}
\begin{tabular}{ | l | l | }
\hline
\HeaderColor RDX & \HeaderColor R8 \\
\hline
0x0000000000000001 & 64 \\
\hline
0x0000000000000002 & 63 \\
\hline
0x0000000000000004 & 62 \\
\hline
0x0000000000000008 & 61 \\
\hline
... & ... \\
\hline
0x4000000000000000 & 2 \\
\hline
0x8000000000000000 & 1 \\
\hline
\end{tabular}
\end{center}

\myindex{x86!\Instructions!FATRET}
À la fin, nous voyons l'instruction \INS{FATRET}, qui a été expliquée ici: \myref{FATRET}.

\myparagraph{MSVC 2012 \Optimizing}

\lstinputlisting[caption=MSVC 2012 \Optimizing,style=customasmx86]{patterns/14_bitfields/4_popcnt/MSVC_2012_x64_Ox_FR.asm}

\myindex{\CompilerAnomaly}
\label{MSVC2012_anomaly}
MSVC 2012 \Optimizing fait presque le même job que MSVC 2010 \Optimizing, mais en
quelque sorte, il génère deux corps de boucles identiques et le nombre de boucles
est maintenant 32 au lieu de 64.

Pour être honnête, il n'est pas possible de dire pourquoi. Une ruse d'optimisation?
Peut-être est-il meilleur pour le corps de la boucle d'être légèrement plus long?

De toute façon, ce genre de code est pertinent ici pour montrer que parfois la sortie
du compilateur peut être vraiment bizarre et illogique, mais fonctionner parfaitement.

}
\JA{\subsubsection{x64}

\myindex{x86-64}
ここの画像は、スタックではなくレジスタが引数の受け渡しに使用されるという違いと似ています。

\myparagraph{MSVC}

\lstinputlisting[caption=MSVC 2012 x64,style=customasmx86]{patterns/04_scanf/1_simple/ex1_MSVC_x64_EN.asm}

\myparagraph{GCC}

\lstinputlisting[caption=\Optimizing GCC 4.4.6 x64,style=customasmx86]{patterns/04_scanf/1_simple/ex1_GCC_x64_EN.s}

}

\EN{\subsubsection{ARM}

\myparagraph{\OptimizingKeilVI (\ThumbMode)}

\lstinputlisting[style=customasmARM]{patterns/04_scanf/1_simple/ARM_IDA.lst}

\myindex{\CLanguageElements!\Pointers}

In order for \scanf to be able to read item it needs a parameter---pointer to an \Tint.
\Tint is 32-bit, so we need 4 bytes to store it somewhere in memory, and it fits exactly in a 32-bit register.
\myindex{IDA!var\_?}
A place for the local variable \GTT{x} is allocated in the stack and \IDA
has named it \emph{var\_8}. It is not necessary, however, to allocate a such since \ac{SP} (\gls{stack pointer}) is already pointing to that space and it can be used directly.

So, \ac{SP}'s value is copied to the \Reg{1} register and, together with the format-string, passed to \scanf.

\INS{PUSH/POP} instructions behaves differently in ARM than in x86 (it's the other way around).
They are synonyms to \INS{STM/STMDB/LDM/LDMIA} instructions.
And \INS{PUSH} instruction first writes a value into the stack, \emph{and then} subtracts \ac{SP} by 4.
\INS{POP} first adds 4 to \ac{SP}, \emph{and then} reads a value from the stack.
Hence, after \INS{PUSH}, \ac{SP} points to an unused space in stack.
It is used by \scanf, and by \printf after.

\INS{LDMIA} means \emph{Load Multiple Registers Increment address After each transfer}.
\INS{STMDB} means \emph{Store Multiple Registers Decrement address Before each transfer}.

\myindex{ARM!\Instructions!LDR}
Later, with the help of the \INS{LDR} instruction, this value is moved from the stack to the \Reg{1} register in order to be passed to \printf.

\myparagraph{ARM64}

\lstinputlisting[caption=\NonOptimizing GCC 4.9.1 ARM64,numbers=left,style=customasmARM]{patterns/04_scanf/1_simple/ARM64_GCC491_O0_EN.s}

There is 32 bytes are allocated for stack frame, which is bigger than it needed. Perhaps some memory aligning issue?
The most interesting part is finding space for the $x$ variable in the stack frame (line 22).
Why 28? Somehow, compiler decided to place this variable at the end of stack frame instead of beginning.
The address is passed to \scanf, which just stores the user input value in the memory at that address.
This is 32-bit value of type \Tint.
The value is fetched at line 27 and then passed to \printf.

}
\RU{\subsubsection{ARM}

\myparagraph{\OptimizingKeilVI (\ThumbMode)}

\lstinputlisting[style=customasmARM]{patterns/04_scanf/1_simple/ARM_IDA.lst}

\myindex{\CLanguageElements!\Pointers}
Чтобы \scanf мог вернуть значение, ему нужно передать указатель на переменную типа \Tint.
\Tint~--- 32-битное значение, для его хранения нужно только 4 байта, и оно помещается в 32-битный регистр.

\myindex{IDA!var\_?}
Место для локальной переменной \GTT{x} выделяется в стеке, \IDA наименовала её \emph{var\_8}. 
Впрочем, место для неё выделять не обязательно, т.к. \glslink{stack pointer}{указатель стека} \ac{SP} уже указывает на место, 
свободное для использования.
Так что значение указателя \ac{SP} копируется в регистр \Reg{1}, и вместе с format-строкой, 
передается в \scanf.

Инструкции \INS{PUSH/POP} в ARM работают иначе, чем в x86 (тут всё наоборот).
Это синонимы инструкций \INS{STM/STMDB/LDM/LDMIA}.
И инструкция \INS{PUSH} в начале записывает в стек значение, \emph{затем} вычитает 4 из \ac{SP}.
\INS{POP} в начале прибавляет 4 к \ac{SP}, \emph{затем} читает значение из стека.
Так что после \INS{PUSH}, \ac{SP} указывает на неиспользуемое место в стеке.
Его и использует \scanf, а затем и \printf.

\INS{LDMIA} означает \emph{Load Multiple Registers Increment address After each transfer}.
\INS{STMDB} означает \emph{Store Multiple Registers Decrement address Before each transfer}.

\myindex{ARM!\Instructions!LDR}
Позже, при помощи инструкции \INS{LDR}, это значение перемещается из стека в регистр \Reg{1}, чтобы быть переданным в \printf.

\myparagraph{ARM64}

\lstinputlisting[caption=\NonOptimizing GCC 4.9.1 ARM64,numbers=left,style=customasmARM]{patterns/04_scanf/1_simple/ARM64_GCC491_O0_RU.s}

Под стековый фрейм выделяется 32 байта, что больше чем нужно. Может быть, это связано с выравниваем по границе памяти?
Самая интересная часть~--- это поиск места под переменную $x$ в стековом фрейме (строка 22).
Почему 28? Почему-то, компилятор решил расположить эту переменную в конце стекового фрейма, а не в начале.
Адрес потом передается в \scanf, которая просто сохраняет значение, введенное пользователем, в памяти по этому адресу.
Это 32-битное значение типа \Tint.
Значение загружается в строке 27 и затем передается в \printf.

}
\IT{\subsubsection{ARM}

\myparagraph{\OptimizingKeilVI (\ThumbMode)}

\lstinputlisting[style=customasmARM]{patterns/04_scanf/1_simple/ARM_IDA.lst}

\myindex{\CLanguageElements!\Pointers}

Affinche' \scanf possa leggere l'input, necessita di un parametro ---puntatore ad un \Tint.
\Tint e' 32-bit, quindi servono 4 byte per memorizzarlo da qualche parte in memoria, e entra perfettamente in un registro a 32-bit.
\myindex{IDA!var\_?}
Uno spazio per la variabile locale \GTT{x} e' allocato nello stack e \IDA
lo ha chiamato \emph{var\_8}. Non e' comunque necessario allocarlo in questo modo poiche' \ac{SP} (\gls{stack pointer}) punta gia' a quella posizione e puo' essere usato direttamente.

Successivamente il valore di \ac{SP} e' copiato nel registro \Reg{1} e sono passati, insieme alla format-string, a \scanf.

% TBT here
%\INS{PUSH/POP} instructions behaves differently in ARM than in x86 (it's the other way around).
%They are synonyms to \INS{STM/STMDB/LDM/LDMIA} instructions.
%And \INS{PUSH} instruction first writes a value into the stack, \emph{and then} subtracts \ac{SP} by 4.
%\INS{POP} first adds 4 to \ac{SP}, \emph{and then} reads a value from the stack.
%Hence, after \INS{PUSH}, \ac{SP} points to an unused space in stack.
%It is used by \scanf, and by \printf after.

%\INS{LDMIA} means \emph{Load Multiple Registers Increment address After each transfer}.
%\INS{STMDB} means \emph{Store Multiple Registers Decrement address Before each transfer}.

\myindex{ARM!\Instructions!LDR}
Questo valore, con l'aiuto dell'istruzione \INS{LDR} , viene poi spostato dallo stakc al registro \Reg{1} per essere passato a \printf.

\myparagraph{ARM64}

\lstinputlisting[caption=\NonOptimizing GCC 4.9.1 ARM64,numbers=left,style=customasmARM]{patterns/04_scanf/1_simple/ARM64_GCC491_O0_EN.s}

Ci sono 32 byte allocati per lo stack frame, che e' piu' grande del necessario. Forse a causa di meccanismi di allineamento della memoria?
La parte piu' interessante e' quella in cui trova spazio per la variabile $x$ nello stack frame (riga 22).
Perche' 28? Il compilatore ha in qualche modo deciso di piazzare questa variabile alla fine dello stack frame anziche' all'inizio.
L'indirizzo e' passato a \scanf, che memorizzera' il valore immesso dall'utente nella memoria a quell'indirizzo.
Si tratta di un valore a 32-bit di tipo \Tint.
Il valore e' recuperato successivamente a riga 27 e passato a \printf.

}
\DE{\subsubsection{ARM}

\myparagraph{\OptimizingKeilVI (\ThumbMode)}

\lstinputlisting[style=customasmARM]{patterns/04_scanf/1_simple/ARM_IDA.lst}

\myindex{\CLanguageElements!\Pointers}
Damit \scanf Elemente einlesen kann, benötigt die Funktion einen Paramter--einen Pointer vom Typ \Tint.
\Tint hat die Größe 32 Bit, wir benötigen also 4 Byte, um den Wert im Speicher abzulegen, und passt daher genau in ein 32-Bit-Register.
\myindex{IDA!var\_?}
Auf dem Stack wird Platz für die lokalen Variable \GTT{x} reserviert und IDA bezeichnet diese Variable mit \emph{var\_8}. 
Eigentlich ist aber an dieser Stelle gar nicht notwendig, Platz auf dem Stack zu reservieren, da \ac{SP} (\gls{stack pointer} 
bereits auf die Adresse zeigt und auch direkt verwendet werden kann.

Der Wert von \ac{SP} wird also in das \Reg{1} Register kopiert und zusammen mit dem Formatierungsstring an \scanf übergeben.

% TBT here
%\INS{PUSH/POP} instructions behaves differently in ARM than in x86 (it's the other way around).
%They are synonyms to \INS{STM/STMDB/LDM/LDMIA} instructions.
%And \INS{PUSH} instruction first writes a value into the stack, \emph{and then} subtracts \ac{SP} by 4.
%\INS{POP} first adds 4 to \ac{SP}, \emph{and then} reads a value from the stack.
%Hence, after \INS{PUSH}, \ac{SP} points to an unused space in stack.
%It is used by \scanf, and by \printf after.

%\INS{LDMIA} means \emph{Load Multiple Registers Increment address After each transfer}.
%\INS{STMDB} means \emph{Store Multiple Registers Decrement address Before each transfer}.

\myindex{ARM!\Instructions!LDR}
Später wird mithilfe des \INS{LDR} Befehls dieser Wert vom Stack in das \Reg{1} Register verschoben um an \printf übergeben werden zu können.

\myparagraph{ARM64}

\lstinputlisting[caption=\NonOptimizing GCC 4.9.1 ARM64,numbers=left,style=customasmARM]{patterns/04_scanf/1_simple/ARM64_GCC491_O0_DE.s}

Im Stack Frame werden 32 Byte reserviert, was deutlich mehr als benötigt ist. Vielleicht handelt es sich um eine Frage des Aligning (dt. Angleichens) von Speicheradressen.
Der interessanteste Teil ist, im Stack Frame einen Platz für die Variable $x$ zu finden (Zeile 22).
Warum 28? Irgendwie hat der Compiler entschieden die Variable am Ende des Stack Frames anstatt an dessen Beginn abzulegen.
Die Adresse wird an \scanf übergeben; diese Funktion speichert den Userinput an der genannten Adresse im Speicher.
Es handelt sich hier um einen 32-Bit-Wert vom Typ \Tint. 
Der Wert wird in Zeile 27 abgeholt und dann an \printf übergeben.


}
\FR{\subsubsection{ARM}

\myparagraph{\OptimizingKeilVI (\ThumbMode)}

\lstinputlisting[style=customasmARM]{patterns/04_scanf/1_simple/ARM_IDA.lst}

\myindex{\CLanguageElements!\Pointers}

Afin que \scanf puisse lire l'item, elle a besoin d'un paramètre---un pointeur sur un \Tint.
Le type \Tint est 32-bit, donc nous avons besoin de 4 octets pour le stocker quelque
part en mémoire, et il tient exactement dans un registre 32-bit.
\myindex{IDA!var\_?}
De l'espace pour la variable locale \GTT{x} est allouée sur la pile et \IDA l'a
nommée \emph{var\_8}. Il n'est toutefois pas nécessaire de définir cette macro, puisque
le \ac{SP} (\glslink{stack pointer}{pointeur de pile}) pointe déjà sur cet espace et
peut être utilisé directement.

Donc, la valeur de \ac{SP} est copiée dans la registre \Reg{1} et, avec la chaîne
de format, passée à \scanf.

Les instructions \INS{PUSH/POP} se comportent différemment en ARM et en x86 (c'est l'inverse)
Il y a des sysnonymes aux instructions \INS{STM/STMDB/LDM/LDMIA}.
Et l'instruction \INS{PUSH} écrit d'abord une valeur sur la pile, \emph{et ensuite}
soustrait 4 de \ac{SP}.
De ce fait, après \INS{PUSH}, \ac{SP} pointe sur de l'espace inutilisé sur la pile.
Il est utilisé par \scanf, et après par \printf.

\INS{LDMIA} signifie \emph{Load Multiple Registers Increment address After each transfer}
(charge plusieurs registres incrémente l'adresse après chaque transfert).
\INS{STMDB} signifie \emph{Store Multiple Registers Decrement address Before each transfer}
(socke plusieurs registres décrémente l'adresse avant chaque transfert).

\myindex{ARM!\Instructions!LDR}
Plus tard, avec l'aide de l'instruction \INS{LDR}, cette valeur est copiée depuis
la pile vers le registre \Reg{1} afin de la passer à \printf.

\myparagraph{ARM64}

\lstinputlisting[caption=GCC 4.9.1 ARM64 \NonOptimizing,numbers=left,style=customasmARM]{patterns/04_scanf/1_simple/ARM64_GCC491_O0_FR.s}

Il y a 32 octets alloués pour la structure de pile, ce qui est plus que nécessaire.
Peut-être dans un soucis d'alignement de mémoire?
La partie la plus intéressante est de trouver de l'espace pour la variable $x$ dans
la structure de pile (ligne 22).
Pourquoi 28? Pour une certaine raison, le compilateur a décidé de stocker cette
variable à la fin de la structure de pile locale au lieu du début.
L'adresse est passée à \scanf, qui stocke l'entrée de l'utilisateur en mémoire à
cette adresse.
Il s'agit d'une valeur sur 32-bit de type \Tint.
La valeur est prise à la ligne 27 puis passée à \printf.

}
\JPN{\subsubsection{ARM}

\myparagraph{\OptimizingKeilVI (\ThumbMode)}

\lstinputlisting[style=customasmARM]{patterns/04_scanf/1_simple/ARM_IDA.lst}

\myindex{\CLanguageElements!\Pointers}

\scanf がitemを読み込むためには、 \Tint へのparameter.pointerが必要です。 
\Tint は32ビットなので、メモリのどこかに格納するには4バイトが必要で、32ビットのレジスタに正確に収まります。 
\myindex{IDA!var\_?}
ローカル変数\GTT{x}の場所がスタックに割り当てられ、 
\IDA の名前は\emph{var\_8}です。 ただし、\ac{SP}(\gls{stack pointer})がすでにその領域を指しているため、その領域を直接割り当てることはできません。 

\INS{PUSH/POP}命令は、ARMとx86とでは動作が異なります(これは逆です)。 
これらは\INS{STM/STMDB/LDM/LDMIA}命令の同義語です。 
そして、\INS{PUSH}命令は最初に値をスタックに書き込み、\emph{次に} \ac{SP} を4で減算します。
\INS{POP}は最初に\ac{SP}に4を加算してから、スタックから値を読み取ります。 
したがって、\INS{PUSH}後、\ac{SP}はスタック内の未使用スペースを指します。 
それは \scanf によって、そして後に \printf によって使用されます。

\INS{LDMIA} は \emph{Load Multiple Registers Increment address After each transfer}の略です。
\INS{STMDB} は \emph{Store Multiple Registers Decrement address Before each transfer}の略です。

したがって、\ac{SP}の値は\Reg{1}レジスタにコピーされ、フォーマット文字列とともに \scanf に渡されます。 
その後、\INS{LDR}命令の助けを借りて、この値はスタックから\Reg{1}レジスタに移動され、 \printf に渡されます。

\myparagraph{ARM64}

\lstinputlisting[caption=\NonOptimizing GCC 4.9.1 ARM64,numbers=left,style=customasmARM]{patterns/04_scanf/1_simple/ARM64_GCC491_O0_JPN.s}

スタックフレームには32バイトが割り当てられており、必要なサイズよりも大きくなっています。 たぶんメモリのアラインメントの問題でしょうか? 
最も興味深いのはスタックフレーム内の$x$変数のためのスペースを見つけることです(22行目)。 
なぜ28なのでしょう? 何らかの理由で、コンパイラは、この変数をスタックフレームの最後に置きます。 
アドレスは \scanf に渡され、\scanf はユーザ入力値をそのアドレスのメモリに格納するだけです。 
これは \Tint 型の32ビット値です。 
値は27行目から取得され、 \printf に渡されます。
}

\EN{\mysection{\MinesweeperWinXPExampleChapterName}
\label{minesweeper_winxp}
\myindex{Windows!Windows XP}

For those who are not very good at playing Minesweeper, we could try to reveal the hidden mines in the debugger.

\myindex{\CStandardLibrary!rand()}
\myindex{Windows!PDB}

As we know, Minesweeper places mines randomly, so there has to be some kind of random number generator or
a call to the standard \TT{rand()} C-function.

What is really cool about reversing Microsoft products is that there are \gls{PDB} 
file with symbols (function names, \etc{}).
When we load \TT{winmine.exe} into \IDA, it downloads the 
\gls{PDB} file exactly for this 
executable and shows all names.

So here it is, the only call to \TT{rand()} is this function:

\lstinputlisting[style=customasmx86]{examples/minesweeper/tmp1.lst}

\IDA named it so, and it was the name given to it by Minesweeper's developers.

The function is very simple:

\begin{lstlisting}[style=customc]
int Rnd(int limit)
{
    return rand() % limit;
};
\end{lstlisting}

(There is no \q{limit} name in the \gls{PDB} file; we manually named this argument like this.)

So it returns 
a random value from 0 to a specified limit.

\TT{Rnd()} is called only from one place, 
a function called \TT{StartGame()}, 
and as it seems, this is exactly 
the code which place the mines:

\begin{lstlisting}[style=customasmx86]
.text:010036C7                 push    _xBoxMac
.text:010036CD                 call    _Rnd@4          ; Rnd(x)
.text:010036D2                 push    _yBoxMac
.text:010036D8                 mov     esi, eax
.text:010036DA                 inc     esi
.text:010036DB                 call    _Rnd@4          ; Rnd(x)
.text:010036E0                 inc     eax
.text:010036E1                 mov     ecx, eax
.text:010036E3                 shl     ecx, 5          ; ECX=ECX*32
.text:010036E6                 test    _rgBlk[ecx+esi], 80h
.text:010036EE                 jnz     short loc_10036C7
.text:010036F0                 shl     eax, 5          ; EAX=EAX*32
.text:010036F3                 lea     eax, _rgBlk[eax+esi]
.text:010036FA                 or      byte ptr [eax], 80h
.text:010036FD                 dec     _cBombStart
.text:01003703                 jnz     short loc_10036C7
\end{lstlisting}

Minesweeper allows you to set the board size, so the X (xBoxMac) and Y (yBoxMac) of the board are global variables.
They are passed to \TT{Rnd()} and random 
coordinates are generated.
A mine is placed by the \TT{OR} instruction at \TT{0x010036FA}. 
And if it has been placed before 
(it's possible if the pair of \TT{Rnd()} 
generates a coordinates pair which has been already 
generated), 
then \TT{TEST} and \TT{JNZ} at \TT{0x010036E6} 
jumps to the generation routine again.

\TT{cBombStart} is the global variable containing total number of mines. So this is loop.

The width of the array is 32 
(we can conclude this by looking at the \TT{SHL} instruction, which multiplies one of the coordinates by 32).

The size of the \TT{rgBlk} 
global array can be easily determined by the difference 
between the \TT{rgBlk} 
label in the data segment and the next known one. 
It is 0x360 (864):

\begin{lstlisting}[style=customasmx86]
.data:01005340 _rgBlk          db 360h dup(?)          ; DATA XREF: MainWndProc(x,x,x,x)+574
.data:01005340                                         ; DisplayBlk(x,x)+23
.data:010056A0 _Preferences    dd ?                    ; DATA XREF: FixMenus()+2
...
\end{lstlisting}

$864/32=27$.

So the array size is $27*32$?
It is close to what we know: when we try to set board size to $100*100$ in Minesweeper settings, it fallbacks to a board of size $24*30$.
So this is the maximal board size here.
And the array has a fixed size for any board size.

So let's see all this in \olly.
We will ran Minesweeper, attaching \olly to it and now we can see the memory dump at the address of the \TT{rgBlk} array (\TT{0x01005340})
\footnote{All addresses here are for Minesweeper for Windows XP SP3 English. 
They may differ for other service packs.}.

So we got this memory dump of the array:

\lstinputlisting[style=customasmx86]{examples/minesweeper/1.lst}

\olly, like any other hexadecimal editor, shows 16 bytes per line.
So each 32-byte array row occupies exactly 2 lines here.

This is beginner level (9*9 board).

There is some square 
structure can be seen visually (0x10 bytes).

We will click \q{Run} in \olly to unfreeze the Minesweeper process, then we'll clicked randomly at the Minesweeper window 
and trapped into mine, but now all mines are visible:

\begin{figure}[H]
\centering
\myincludegraphicsSmall{examples/minesweeper/1.png}
\caption{Mines}
\label{fig:minesweeper1}
\end{figure}

By comparing the mine places and the dump, we can conclude that 0x10 stands for border, 0x0F---empty block, 0x8F---mine.
Perhaps, 0x10 is just a \emph{sentinel value}.

Now we'll add comments and also enclose all 0x8F bytes into square brackets:

\lstinputlisting[style=customasmx86]{examples/minesweeper/2.lst}

Now we'll remove all \emph{border bytes} (0x10) and what's beyond those:

\lstinputlisting[style=customasmx86]{examples/minesweeper/3.lst}

Yes, these are mines, now it can be clearly seen and compared with the screenshot.

\clearpage
What is interesting is that we can modify the array right in \olly.
We can remove all mines by changing all 0x8F bytes by 0x0F, and here is what we'll get in Minesweeper:

\begin{figure}[H]
\centering
\myincludegraphicsSmall{examples/minesweeper/3.png}
\caption{All mines are removed in debugger}
\label{fig:minesweeper3}
\end{figure}

We can also move all of them to the first line: 

\begin{figure}[H]
\centering
\myincludegraphicsSmall{examples/minesweeper/2.png}
\caption{Mines set in debugger}
\label{fig:minesweeper2}
\end{figure}

Well, the debugger is not very convenient for eavesdropping (which is our goal anyway), so we'll write a small utility
to dump the contents of the board:

\lstinputlisting[style=customc]{examples/minesweeper/minesweeper_cheater.c}

Just set the \ac{PID}
\footnote{PID it can be seen in Task Manager 
(enable it in \q{View $\rightarrow$ Select Columns})} 
and the address of the array (\TT{0x01005340} for Windows XP SP3 English) 
and it will dump it
\footnote{The compiled executable is here: 
\href{http://go.yurichev.com/17165}{beginners.re}}.

It attaches itself to a win32 process by \ac{PID} and just reads process memory at the address.

\subsection{Finding grid automatically}

This is kind of nuisance to set address each time when we run our utility.
Also, various Minesweeper versions may have the array on different address.
Knowing the fact that there is always a border (0x10 bytes), we can just find it in memory:

\lstinputlisting[style=customc]{examples/minesweeper/cheater2_fragment.c}

Full source code: \url{https://github.com/DennisYurichev/RE-for-beginners/blob/master/examples/minesweeper/minesweeper_cheater2.c}.

\subsection{\Exercises}

\begin{itemize}

\item 
Why do the \emph{border bytes} (or \emph{sentinel values}) (0x10) exist in the array?

What they are for if they are not visible in Minesweeper's interface?
How could it work without them?

\item 
As it turns out, there are more values possible (for open blocks, for flagged by user, \etc{}).
Try to find the meaning of each one.

\item 
Modify my utility so it can remove all mines or set them in a fixed pattern that you want in the Minesweeper
process currently running.

\end{itemize}
}
\RU{\mysection{Разгон майнера биткоинов Cointerra}
\index{Bitcoin}
\index{BeagleBone}

Был такой майнер биткоинов Cointerra, выглядящий так:

\begin{figure}[H]
\centering
\myincludegraphics{examples/bitcoin_miner/board.jpg}
\caption{Board}
\end{figure}

И была также (возможно утекшая) утилита\footnote{Можно скачать здесь: \url{https://github.com/DennisYurichev/RE-for-beginners/raw/master/examples/bitcoin_miner/files/cointool-overclock}}
которая могла выставлять тактовую частоту платы.
Она запускается на дополнительной плате BeagleBone на ARM с Linux (маленькая плата внизу фотографии).

И у автора (этих строк) однажды спросили, можно ли хакнуть эту утилиту и посмотреть, какие частоты можно выставлять, и какие нет.
И можно ли твикнуть её?

Утилиту нужно запускать так: \TT{./cointool-overclock 0 0 900}, где 900 это частота в МГц.
Если частота слишком большая, утилита выведет ошибку \q{Error with arguments} и закончит работу.

Вот фрагмент кода вокруг ссылки на текстовую строку \q{Error with arguments}:

\begin{lstlisting}[style=customasmARM]

...

.text:0000ABC4         STR      R3, [R11,#var_28]
.text:0000ABC8         MOV      R3, #optind
.text:0000ABD0         LDR      R3, [R3]
.text:0000ABD4         ADD      R3, R3, #1
.text:0000ABD8         MOV      R3, R3,LSL#2
.text:0000ABDC         LDR      R2, [R11,#argv]
.text:0000ABE0         ADD      R3, R2, R3
.text:0000ABE4         LDR      R3, [R3]
.text:0000ABE8         MOV      R0, R3  ; nptr
.text:0000ABEC         MOV      R1, #0  ; endptr
.text:0000ABF0         MOV      R2, #0  ; base
.text:0000ABF4         BL       strtoll
.text:0000ABF8         MOV      R2, R0
.text:0000ABFC         MOV      R3, R1
.text:0000AC00         MOV      R3, R2
.text:0000AC04         STR      R3, [R11,#var_2C]
.text:0000AC08         MOV      R3, #optind
.text:0000AC10         LDR      R3, [R3]
.text:0000AC14         ADD      R3, R3, #2
.text:0000AC18         MOV      R3, R3,LSL#2
.text:0000AC1C         LDR      R2, [R11,#argv]
.text:0000AC20         ADD      R3, R2, R3
.text:0000AC24         LDR      R3, [R3]
.text:0000AC28         MOV      R0, R3  ; nptr
.text:0000AC2C         MOV      R1, #0  ; endptr
.text:0000AC30         MOV      R2, #0  ; base
.text:0000AC34         BL       strtoll
.text:0000AC38         MOV      R2, R0
.text:0000AC3C         MOV      R3, R1
.text:0000AC40         MOV      R3, R2
.text:0000AC44         STR      R3, [R11,#third_argument]
.text:0000AC48         LDR      R3, [R11,#var_28]
.text:0000AC4C         CMP      R3, #0
.text:0000AC50         BLT      errors_with_arguments
.text:0000AC54         LDR      R3, [R11,#var_28]
.text:0000AC58         CMP      R3, #1
.text:0000AC5C         BGT      errors_with_arguments
.text:0000AC60         LDR      R3, [R11,#var_2C]
.text:0000AC64         CMP      R3, #0
.text:0000AC68         BLT      errors_with_arguments
.text:0000AC6C         LDR      R3, [R11,#var_2C]
.text:0000AC70         CMP      R3, #3
.text:0000AC74         BGT      errors_with_arguments
.text:0000AC78         LDR      R3, [R11,#third_argument]
.text:0000AC7C         CMP      R3, #0x31
.text:0000AC80         BLE      errors_with_arguments
.text:0000AC84         LDR      R2, [R11,#third_argument]
.text:0000AC88         MOV      R3, #950
.text:0000AC8C         CMP      R2, R3
.text:0000AC90         BGT      errors_with_arguments
.text:0000AC94         LDR      R2, [R11,#third_argument]
.text:0000AC98         MOV      R3, #0x51EB851F
.text:0000ACA0         SMULL    R1, R3, R3, R2
.text:0000ACA4         MOV      R1, R3,ASR#4
.text:0000ACA8         MOV      R3, R2,ASR#31
.text:0000ACAC         RSB      R3, R3, R1
.text:0000ACB0         MOV      R1, #50
.text:0000ACB4         MUL      R3, R1, R3
.text:0000ACB8         RSB      R3, R3, R2
.text:0000ACBC         CMP      R3, #0
.text:0000ACC0         BEQ      loc_ACEC
.text:0000ACC4
.text:0000ACC4 errors_with_arguments
.text:0000ACC4                                         
.text:0000ACC4         LDR      R3, [R11,#argv]
.text:0000ACC8         LDR      R3, [R3]
.text:0000ACCC         MOV      R0, R3  ; path
.text:0000ACD0         BL       __xpg_basename
.text:0000ACD4         MOV      R3, R0
.text:0000ACD8         MOV      R0, #aSErrorWithArgu ; format
.text:0000ACE0         MOV      R1, R3
.text:0000ACE4         BL       printf
.text:0000ACE8         B        loc_ADD4
.text:0000ACEC ; ------------------------------------------------------------
.text:0000ACEC
.text:0000ACEC loc_ACEC                 ; CODE XREF: main+66C
.text:0000ACEC         LDR      R2, [R11,#third_argument]
.text:0000ACF0         MOV      R3, #499
.text:0000ACF4         CMP      R2, R3
.text:0000ACF8         BGT      loc_AD08
.text:0000ACFC         MOV      R3, #0x64
.text:0000AD00         STR      R3, [R11,#unk_constant]
.text:0000AD04         B        jump_to_write_power
.text:0000AD08 ; ------------------------------------------------------------
.text:0000AD08
.text:0000AD08 loc_AD08                 ; CODE XREF: main+6A4
.text:0000AD08         LDR      R2, [R11,#third_argument]
.text:0000AD0C         MOV      R3, #799
.text:0000AD10         CMP      R2, R3
.text:0000AD14         BGT      loc_AD24
.text:0000AD18         MOV      R3, #0x5F
.text:0000AD1C         STR      R3, [R11,#unk_constant]
.text:0000AD20         B        jump_to_write_power
.text:0000AD24 ; ------------------------------------------------------------
.text:0000AD24
.text:0000AD24 loc_AD24                 ; CODE XREF: main+6C0
.text:0000AD24         LDR      R2, [R11,#third_argument]
.text:0000AD28         MOV      R3, #899
.text:0000AD2C         CMP      R2, R3
.text:0000AD30         BGT      loc_AD40
.text:0000AD34         MOV      R3, #0x5A
.text:0000AD38         STR      R3, [R11,#unk_constant]
.text:0000AD3C         B        jump_to_write_power
.text:0000AD40 ; ------------------------------------------------------------
.text:0000AD40
.text:0000AD40 loc_AD40                 ; CODE XREF: main+6DC
.text:0000AD40         LDR      R2, [R11,#third_argument]
.text:0000AD44         MOV      R3, #999
.text:0000AD48         CMP      R2, R3
.text:0000AD4C         BGT      loc_AD5C
.text:0000AD50         MOV      R3, #0x55
.text:0000AD54         STR      R3, [R11,#unk_constant]
.text:0000AD58         B        jump_to_write_power
.text:0000AD5C ; ------------------------------------------------------------
.text:0000AD5C
.text:0000AD5C loc_AD5C                 ; CODE XREF: main+6F8
.text:0000AD5C         LDR      R2, [R11,#third_argument]
.text:0000AD60         MOV      R3, #1099
.text:0000AD64         CMP      R2, R3
.text:0000AD68         BGT      jump_to_write_power
.text:0000AD6C         MOV      R3, #0x50
.text:0000AD70         STR      R3, [R11,#unk_constant]
.text:0000AD74
.text:0000AD74 jump_to_write_power                     ; CODE XREF: main+6B0
.text:0000AD74                                         ; main+6CC ...
.text:0000AD74         LDR      R3, [R11,#var_28]
.text:0000AD78         UXTB     R1, R3
.text:0000AD7C         LDR      R3, [R11,#var_2C]
.text:0000AD80         UXTB     R2, R3
.text:0000AD84         LDR      R3, [R11,#unk_constant]
.text:0000AD88         UXTB     R3, R3
.text:0000AD8C         LDR      R0, [R11,#third_argument]
.text:0000AD90         UXTH     R0, R0
.text:0000AD94         STR      R0, [SP,#0x44+var_44]
.text:0000AD98         LDR      R0, [R11,#var_24]
.text:0000AD9C         BL       write_power
.text:0000ADA0         LDR      R0, [R11,#var_24]
.text:0000ADA4         MOV      R1, #0x5A
.text:0000ADA8         BL       read_loop
.text:0000ADAC         B        loc_ADD4

...

.rodata:0000B378 aSErrorWithArgu DCB "%s: Error with arguments",0xA,0 ; DATA XREF: main+684

...

\end{lstlisting}

Имена ф-ций присутствовали в отладочной информации в оригинальном исполняемом файле,
такие как \TT{write\_power}, \TT{read\_loop}.
Но имена меткам внутри ф-ции дал я.

\myindex{UNIX!getopt}
\myindex{strtoll()}
Имя \TT{optind} звучит знакомо. Это библиотека \IT{getopt} из *NIX предназначенная для парсинга командной строки ---
и это то, что внутри и происходит.
Затем, третий аргумент (где передается значение частоты) конвертируется из строку в число используя вызов ф-ции \IT{strtoll()}.

Значение затем сравнивается с разными константами.
На 0xACEC есть проверка, меньше ли оно или равно 499, и если это так, то 0x64 будет передано в ф-цию
\TT{write\_power()} (которая посылает команду через USB используя \TT{send\_msg()}).
Если значение больше 499, происходит переход на 0xAD08.

На 0xAD08 есть проверка, меньше ли оно или равно 799. Если это так, то 0x5F передается в ф-цию \TT{write\_power()}.

Есть еще проверки: на 899 на 0xAD24, на 0x999 на 0xAD40, и наконец, на 1099 на 0xAD5C.
Если входная частота меньше или равна 1099, 0x50 (на 0xAD6C) будет передано в ф-цию \TT{write\_power()}.
И тут что-то вроде баги.
Если значение все еще больше 1099, само значение будет передано в ф-цию \TT{write\_power()}.
Но с другой стороны это не бага, потому что мы не можем попасть сюда: значение в начале проверяется с 950 на 0xAC88,
и если оно больше, выводится сообщение об ошибке и утилита заканчивает работу.

Вот таблица между частотами в МГц и значениями передаваемыми в ф-цию \TT{write\_power()}:

\begin{center}
\begin{longtable}{ | l | l | l | }
\hline
\HeaderColor МГц & \HeaderColor шестнадцатеричное представление & \HeaderColor десятичное \\
\hline
499MHz & 0x64 & 100 \\
\hline
799MHz & 0x5f & 95 \\
\hline
899MHz & 0x5a & 90 \\
\hline
999MHz & 0x55 & 85 \\
\hline
1099MHz & 0x50 & 80 \\
\hline
\end{longtable}
\end{center}

Как видно, значение передаваемое в плату постепенно уменьшается с ростом частоты.

Видно что значение в 950МГц это жесткий предел, по крайней мере в этой утилите. Можно ли её обмануть?

Вернемся к этому фрагменту кода:

\begin{lstlisting}[style=customasmARM]
.text:0000AC84      LDR     R2, [R11,#third_argument]
.text:0000AC88      MOV     R3, #950
.text:0000AC8C      CMP     R2, R3
.text:0000AC90      BGT     errors_with_arguments ; Я пропатчил здесь на 00 00 00 00
\end{lstlisting}

Нам нужно как-то запретить инструкцию перехода \INS{BGT} на 0xAC90. И это ARM в режиме ARM, потому что, как мы видим,
все адреса увеличиваются на 4, т.е., длина каждой инструкции это 4 байта.
Инструкция \TT{NOP} (нет операции) в режиме ARM это просто 4 нулевых байта: \TT{00 00 00 00}.
Так что, записывая 4 нуля по адресу 0xAC90 (или по физическому смещению в файле: 0x2C90) мы можем выключить
эту проверку.

Теперь можно выставлять частоты вплоть до 1050МГц. И даже больше, но из-за ошибки, если входное значение больше 1099,
значение в МГц, \IT{как есть}, будет передано в плату, что неправильно.

Дальше я не разбирался, но если бы продолжил, я бы уменьшал значение передаваемое в ф-цию \TT{write\_power()}.

Теперь страшный фрагмент кода, который я в начале пропустил:

\lstinputlisting[style=customasmARM]{examples/bitcoin_miner/tmp1.lst}

Здесь используется деление через умножение, и константа 0x51EB851F.
Я написал для себя простой программистский калькулятор\footnote{\url{https://github.com/DennisYurichev/progcalc}}.
И там есть возможность вычислять обратное число по модулю.

\begin{lstlisting}
modinv32(0x51EB851F)
Warning, result is not integer: 3.125000
(unsigned) dec: 3 hex: 0x3 bin: 11
\end{lstlisting}

Это значит что инструкция \INS{SMULL} на 0xACA0 просто делит 3-й аргумент на 3.125.
На самом деле, все что делает ф-ция \TT{modinv32()} в моем калькуляторе, это:

\[
\frac{1}{\frac{input}{2^{32}}} = \frac{2^{32}}{input}
\]

Потом там есть дополнительные сдвиги и теперь мы видим что 3-й аргумент просто делится на 50.
И затем умножается снова на 50.
Зачем?
Это простейшая проверка, можно ли делить входное значение на 50 без остатка.
Если значение этого выражения ненулевое, $x$ не может быть разделено на 50 без остатка:

\[
x-((\frac{x}{50}) \cdot 50)
\]

На самом деле, это простой способ вычисления остатка от деления.

И затем, если остаток ненулевой, выводится сообщение об ошибке.
Так что эта утилита берет значения частотв вроде 850, 900, 950, 1000, итд, но не 855 или 911.

Вот и всё! Если вы делаете что-то такое, имейте ввиду, что это может испортить вашу плату, как и в случае разгона
чипов вроде \ac{CPU}, \ac{GPU}, итд.
Если у вас есть плата Cointerra, делайте всё это на свой собственный риск!

}
\IT{\subsubsection{MIPS}

Nello stack locale viene allocato spazio per la variabile $x$ , a cui viene fatto riferimento come $\$sp+24$.
\myindex{MIPS!\Instructions!LW}

Il suo indirizzo è passato a \scanf, il valore immesso dall'utente è caricato usand l'istruzione \INS{LW} (\q{Load Word}) ed è infine passato a \printf.

\lstinputlisting[caption=\Optimizing GCC 4.4.5 (\assemblyOutput),style=customasmMIPS]{patterns/04_scanf/1_simple/MIPS/ex1.O3_EN.s}

IDA mostra il layout dello stack nel modo seguente:

\lstinputlisting[caption=\Optimizing GCC 4.4.5 (IDA),style=customasmMIPS]{patterns/04_scanf/1_simple/MIPS/ex1.O3.IDA_EN.lst}

% TODO non-optimized version?
}
\DE{\mysection{\Stack}
\label{sec:stack}
\myindex{\Stack}

Der Stack ist eine der fundamentalen Datenstrukturen in der Informatik.
\footnote{\href{http://go.yurichev.com/17119}{wikipedia.org/wiki/Call\_Stack}}.
\ac{AKA} \ac{LIFO}.

Technisch betrachtet ist es ein Stapelspeicher innerhalb des Prozessspeichers der zusammen mit den \ESP (x86), \RSP (x64) oder dem \ac{SP} (ARM) Register als ein Zeiger in diesem Speicherblock fungiert.

\myindex{ARM!\Instructions!PUSH}
\myindex{ARM!\Instructions!POP}
\myindex{x86!\Instructions!PUSH}
\myindex{x86!\Instructions!POP}

Die häufigsten Stack-Zugriffsinstruktionen sind die \PUSH- und \POP-Instruktionen (in beidem x86 und ARM Thumb-Modus). \PUSH subtrahiert vom \ESP/\RSP/\ac{SP} 4 Byte im 32-Bit Modus (oder 8 im 64-Bit Modus) und schreibt dann den Inhalt des Zeigers an die Adresse auf die von \ESP/\RSP/\ac{SP} gezeigt wird.

\POP ist die umgekehrte Operation: Die Daten des Zeigers für die Speicherregion auf die von \ac{SP}
gezeigt wird werden ausgelesen und die Inhalte in den Instruktionsoperanden geschreiben (oft ist das ein Register). Dann werden 4 (beziehungsweise 8) Byte zum \gls{stack pointer} addiert.

Nach der Stackallokation, zeigt der \gls{stack pointer} auf den Boden des Stacks.
\PUSH verringert den \gls{stack pointer} und \POP erhöht ihn.
Der Boden des Stacks ist eigentlich der Anfang der Speicherregion die für den Stack reserviert wurde.
Das wirkt zunächst seltsam, aber so funktioniert es.

ARM unterstützt beides, aufsteigende und absteigende Stacks.

\myindex{ARM!\Instructions!STMFD}
\myindex{ARM!\Instructions!LDMFD}
\myindex{ARM!\Instructions!STMED}
\myindex{ARM!\Instructions!LDMED}
\myindex{ARM!\Instructions!STMFA}
\myindex{ARM!\Instructions!LDMFA}
\myindex{ARM!\Instructions!STMEA}
\myindex{ARM!\Instructions!LDMEA}

Zum Beispiel die \ac{STMFD}/\ac{LDMFD} und \ac{STMED}/\ac{LDMED} Instruktionen sind alle dafür gedacht mit einem absteigendem Stack zu arbeiten ( wächst nach unten, fängt mit hohen Adressen an und entwickelt sich zu niedrigeren Adressen). Die \ac{STMFA}/\ac{LDMFA} und \ac{STMEA}/\ac{LDMEA} Instruktionen sind dazu gedacht mit einem aufsteigendem Stack zu arbeiten (wächst nach oben und fängt mit niedrigeren Adressen an und wächst nach oben).

% It might be worth mentioning that STMED and STMEA write first,
% and then move the pointer, and that LDMED and LDMEA move the pointer first, and then read.
% In other words, ARM not only lets the stack grow in a non-standard direction,
% but also in a non-standard order.
% Maybe this can be in the glossary, which would explain why E stands for "empty".

\subsection{Warum wächst der Stack nach unten?}
\label{stack_grow_backwards}

Intuitiv, würden man annehmen das der Stack nach oben wächst z.B Richtung höherer Adressen, so wie bei jeder anderen Datenstruktur.

Der Grund das der Stack rückwärts wächst ist wohl historisch bedingt. Als Computer so groß waren das sie einen ganzen Raum beansprucht haben war es einfach Speicher in zwei Sektionen zu unterteilen, einen Teil für den \gls{heap} und einen Teil für den Stack. Sicher war zu dieser Zeit nicht bekannt wie groß der \gls{heap} und der Stack wachsen würden, während der Programm Laufzeit, also war die Lösung die einfachste mögliche.

\input{patterns/02_stack/stack_and_heap}

In \RitchieThompsonUNIX können wir folgendes lesen:

\begin{framed}
\begin{quotation}
Der user-core eines Programm Images wird in drei logische Segmente unterteilt. Das Programm-Text Segment beginnt bei 0 im virtuellen Adress Speicher. Während der Ausführung wird das Segment als schreibgeschützt markiert und eine einzelne Kopie des Segments wird unter allen Prozessen geteilt die das Programm ausführen. An der ersten 8K grenze über dem Programm Text Segment im Virtuellen Speicher, fängt der ``nonshared'' Bereich an, der nach Bedarf von Syscalls erweitert werden kann. Beginnend bei der höchsten Adresse im Virtuellen Speicher ist das Stack Segment, das Automatisch nach unten wächst während der Hardware Stackpointer sich ändert.
\end{quotation}
\end{framed}

Das erinnert daran wie manche Schüler Notizen zu  zwei Vorträgen in einem Notebook dokumentieren:
Notizen für den ersten Vortrag werden normal notiert, und Notizen zur zum zweiten Vortrag werden 
ans Ende des Notizbuches geschrieben, indem man das Notizbuch umdreht. Die Notizen treffen sich irgendwann
im Notizbuch aufgrund des fehlenden Freien Platzes.

% I think if we want to expand on this analogy,
% one might remember that the line number increases as as you go down a page.
% So when you decrease the address when pushing to the stack, visually,
% the stack does grow upwards.
% Of course, the problem is that in most human languages,
% just as with computers,
% we write downwards, so this direction is what makes buffer overflows so messy.

\subsection{Für was wird der Stack benutzt?}

% subsections
\EN{\input{patterns/02_stack/01_saving_ret_addr_EN}}
\RU{\input{patterns/02_stack/01_saving_ret_addr_RU}}
\DE{\input{patterns/02_stack/01_saving_ret_addr_DE}}
\FR{\input{patterns/02_stack/01_saving_ret_addr_FR}}
\PTBR{\input{patterns/02_stack/01_saving_ret_addr_PTBR}}
\IT{\input{patterns/02_stack/01_saving_ret_addr_IT}}
\PL{\input{patterns/02_stack/01_saving_ret_addr_PL}}
\JPN{\input{patterns/02_stack/01_saving_ret_addr_JPN}}

\EN{\input{patterns/02_stack/02_args_passing_EN}}
\RU{\input{patterns/02_stack/02_args_passing_RU}}
\PTBR{\input{patterns/02_stack/02_args_passing_PTBR}}
\DE{\input{patterns/02_stack/02_args_passing_DE}}
\IT{\input{patterns/02_stack/02_args_passing_IT}}
\FR{\input{patterns/02_stack/02_args_passing_FR}}
\JA{\input{patterns/02_stack/02_args_passing_JA}}
\PL{\input{patterns/02_stack/02_args_passing_PL}}


\EN{\input{patterns/02_stack/03_local_vars_EN}}
\RU{\input{patterns/02_stack/03_local_vars_RU}}
\DE{\input{patterns/02_stack/03_local_vars_DE}}
\PTBR{\input{patterns/02_stack/03_local_vars_PTBR}}
\EN{\input{patterns/02_stack/04_alloca/main_EN}}
\FR{\input{patterns/02_stack/04_alloca/main_FR}}
\RU{\input{patterns/02_stack/04_alloca/main_RU}}
\PTBR{\input{patterns/02_stack/04_alloca/main_PTBR}}
\IT{\input{patterns/02_stack/04_alloca/main_IT}}
\DE{\input{patterns/02_stack/04_alloca/main_DE}}
\PL{\input{patterns/02_stack/04_alloca/main_PL}}
\JPN{\input{patterns/02_stack/04_alloca/main_JPN}}

\input{patterns/02_stack/05_SEH}
\ifdefined\ENGLISH
\subsubsection{Buffer overflow protection}

More about it here~(\myref{subsec:bufferoverflow}).
\fi

\ifdefined\RUSSIAN
\subsubsection{Защита от переполнений буфера}

Здесь больше об этом~(\myref{subsec:bufferoverflow}).
\fi

\ifdefined\BRAZILIAN
\subsubsection{Proteção contra estouro de buffer}

Mais sobre aqui~(\myref{subsec:bufferoverflow}).
\fi

\ifdefined\ITALIAN
\subsubsection{Protezione contro buffer overflow}

Maggiori informazioni qui~(\myref{subsec:bufferoverflow}).
\fi

\ifdefined\FRENCH
\subsubsection{Protection contre les débordements de tampon}

Lire à ce propos~(\myref{subsec:bufferoverflow}).
\fi


\ifdefined\POLISH
\subsubsection{Metody zabiezpieczenia przed przepełnieniem stosu}

Więcej o tym tutaj~(\myref{subsec:bufferoverflow}).
\fi

\ifdefined\JAPANESE
\subsubsection{バッファオーバーフロー保護}

詳細はこちら~(\myref{subsec:bufferoverflow})
\fi


\subsubsection{Automatisches deallokieren der Daten auf dem Stack}

Vielleicht ist der Grund warum man lokale Variablen und SEH Einträge auf dem Stack speichert, weil sie beim 
verlassen der Funktion automatisch aufgeräumt werden. Man braucht dabei nur eine Instruktion um die Position
des Stackpointers zu korrigieren (oftmals ist es die \ADD Instruktion). Funktions Argumente, könnte man sagen 
werden auch am Ende der Funktion deallokiert. Im Kontrast dazu, alles was auf dem \emph{heap} gespeichert wird muss
explizit deallokiert werden. 

% sections
\EN{\input{patterns/02_stack/07_layout_EN}}
\RU{\input{patterns/02_stack/07_layout_RU}}
\DE{\input{patterns/02_stack/07_layout_DE}}
\PTBR{\input{patterns/02_stack/07_layout_PTBR}}
\EN{\input{patterns/02_stack/08_noise/main_EN}}
\FR{\input{patterns/02_stack/08_noise/main_FR}}
\RU{\input{patterns/02_stack/08_noise/main_RU}}
\IT{\input{patterns/02_stack/08_noise/main_IT}}
\DE{\input{patterns/02_stack/08_noise/main_DE}}
\PL{\input{patterns/02_stack/08_noise/main_PL}}
\JA{\input{patterns/02_stack/08_noise/main_JA}}

\input{patterns/02_stack/exercises}
}
\FR{\subsection{MIPS}

\subsubsection{3 arguments}

\myparagraph{GCC 4.4.5 \Optimizing}

La différence principale avec l'exemple \q{\HelloWorldSectionName} est que dans ce cas, \printf est appelée
à la place de \puts et 3 arguments de plus sont passés à travers les registres \$5\dots \$7 (ou \$A0\dots \$A2).
C'est pourquoi ces registres sont préfixés avec A-, ceci sous-entend qu'ils
sont utilisés pour le passage des arguments aux fonctions.

\lstinputlisting[caption=GCC 4.4.5 \Optimizing (\assemblyOutput),style=customasmMIPS]{patterns/03_printf/MIPS/printf3.O3_FR.s}

\lstinputlisting[caption=GCC 4.4.5 \Optimizing (IDA),style=customasmMIPS]{patterns/03_printf/MIPS/printf3.O3.IDA_FR.lst}

\IDA a agrégé la paire d'instructions \INS{LUI} et \INS{ADDIU} en une pseudo instruction \INS{LA}.
C'est pourquoi il n'y a pas d'instruction à l'adresse 0x1C: car \INS{LA} \emph{occupe} 8 octets.

\myparagraph{GCC 4.4.5 \NonOptimizing}

GCC \NonOptimizing est plus verbeux:

\lstinputlisting[caption=GCC 4.4.5 \NonOptimizing (\assemblyOutput),style=customasmMIPS]{patterns/03_printf/MIPS/printf3.O0_FR.s}

\lstinputlisting[caption=GCC 4.4.5 \NonOptimizing (IDA),style=customasmMIPS]{patterns/03_printf/MIPS/printf3.O0.IDA_FR.lst}

\subsubsection{8 arguments}

Utilisons encore l'exemple de la section précédente avec 9 arguments: \myref{example_printf8_x64}.

\lstinputlisting[style=customc]{patterns/03_printf/2.c}

\myparagraph{GCC 4.4.5 \Optimizing}

Seul les 4 premiers arguments sont passés dans les registres \$A0 \dots \$A3,
les autres sont passés par la pile.
\myindex{MIPS!O32}

C'est la convention d'appel O32 (qui est la plus commune dans le monde MIPS).
D'autres conventions d'appel (comme N32) peuvent utiliser les registres à d'autres fins.

\myindex{MIPS!\Instructions!SW}

\INS{SW} est l'abbréviation de \q{Store Word} (depuis un registre vers la mémoire).
En MIPS, il manque une instructions pour stocker une valeur dans la mémoire, donc
une paire d'instruction doit être utilisée à la place (\INS{LI}/\INS{SW}).

\lstinputlisting[caption=GCC 4.4.5 \Optimizing (\assemblyOutput),style=customasmMIPS]{patterns/03_printf/MIPS/printf8.O3_FR.s}

\lstinputlisting[caption=GCC 4.4.5 \Optimizing (IDA),style=customasmMIPS]{patterns/03_printf/MIPS/printf8.O3.IDA_FR.lst}

\myparagraph{GCC 4.4.5 \NonOptimizing}

GCC \NonOptimizing est plus verbeux:

\lstinputlisting[caption=\NonOptimizing GCC 4.4.5 (\assemblyOutput),style=customasmMIPS]{patterns/03_printf/MIPS/printf8.O0_FR.s}

\lstinputlisting[caption=\NonOptimizing GCC 4.4.5 (IDA),style=customasmMIPS]{patterns/03_printf/MIPS/printf8.O0.IDA_FR.lst}

}
\JPN{\subsection{擬似乱数生成器の例}
\label{FPU_PRNG}

0と1の間の浮動小数点の乱数が必要な場合、最も簡単なのはメルセンヌツイスターのような
\ac{PRNG}を使うことです。
ランダムな符号なし32ビット値を生成します(つまり、ランダム32ビットを生成します)。
この値をfloatに変換し、
\GTT{RAND\_MAX}(ここでは\GTT{0xFFFFFFFF})で割ります。我々は0..1の間で値を取得します。

しかし知ってのとおり、除算は遅いです。
また、できるだけ少ないFPU演算で実行したいと考えています。
私たちは除算を取り除くことができるでしょうか?

\myindex{IEEE 754}

浮動小数点数が符号ビット、仮数ビット、指数ビットからなるものを思い出してみましょう。
ランダムな浮動小数点数を得るには、すべての仮数ビットにランダムなビットを格納するだけです。

指数部はゼロではありません(浮動小数点はこの場合非正規化されています)ので、
指数部に0b01111111
を格納しています。指数部が1であることを意味します。
次に、仮数部をランダムビットで埋め、符号ビットを0に設定する(正の数)と出来上がり。
生成される数は1と2の間にあるので、1を減算する必要があります。

\newcommand{\URLXOR}{\url{http://go.yurichev.com/17308}}

私の例では、非常に単純な線形合同乱数ジェネレータが使用され、
\footnote{アイデアは以下から取りました: \URLXOR} これは32ビットの数値を生成します。 
\ac{PRNG}はUNIXのタイムスタンプ形式で現在の時刻で初期化されます。

ここでは \Tfloat 型を\emph{union}として表します。これは、メモリの種類を
異なる型として解釈できる \CCpp 構造です。
私たちの場合、\emph{union}型の変数を作成し、
それを \Tfloat  または \emph{uint32\_t} のようにアクセスすることができます。
それはまさに汚いハックだと言えるでしょう。

% WTF?

整数\ac{PRNG}コードは、すでに検討しているものと同じです:\myref{LCG_simple}
このコードはコンパイルされた形式では省略されています。

\lstinputlisting[style=customc]{patterns/17_unions/FPU_PRNG/FPU_PRNG_JPN.cpp}

\subsubsection{x86}

\lstinputlisting[caption=\Optimizing MSVC 2010,style=customasmx86]{patterns/17_unions/FPU_PRNG/MSVC2010_Ox_Ob0_JPN.asm}

この例はC++としてコンパイルされており、これはC++での名前の変換であるため、ここでは関数名が非常に奇妙です。
これについては後で説明します:\myref{namemangling}
これをMSVC 2012でコンパイルすると、FPU用のSIMD命令が使用されます。詳細については、こちらを参照してください:\myref{FPU_PRNG_SIMD}

\iffalse
A BUG HERE
\subsubsection{MIPS}

\lstinputlisting[caption=\Optimizing GCC 4.4.5,style=customasmMIPS]{patterns/17_unions/FPU_PRNG/MIPS_O3_IDA_JPN.lst}

いくつかの奇妙な理由のために追加された無駄な\INS{LUI}命令もあります。 
このアーティファクトを以前検討しました:\myref{MIPS_FPU_LUI}
\fi

\subsubsection{ARM (\ARMMode)}

\lstinputlisting[caption=\Optimizing GCC 4.6.3 (IDA),style=customasmARM]{patterns/17_unions/FPU_PRNG/raspberry_GCC_O3_IDA_JPN.lst}

\myindex{objdump}
\myindex{binutils}
\myindex{IDA}

また、objdumpにダンプを作成し、FPU命令の名前が \IDA とは異なることを確認します。 
見たところ、IDAとbinutilsの開発者は異なるマニュアルを使ったのでしょうか? 
おそらく、両方の命令の変種を知っておくとよいでしょう。

\lstinputlisting[caption=\Optimizing GCC 4.6.3 (objdump),style=customasmARM]{patterns/17_unions/FPU_PRNG/raspberry_GCC_O3_objdump.lst}

\TT{float\_rand()}の0x5cと \main の0x38の命令は、(疑似)乱数ノイズです。
}

