\subsubsection{ARM}

\myparagraph{\OptimizingKeilVI (\ThumbMode)}

\lstinputlisting[style=customasmARM]{patterns/04_scanf/1_simple/ARM_IDA.lst}

\myindex{\CLanguageElements!\Pointers}

\scanf がitemを読み込むためには、 \Tint へのparameter.pointerが必要です。 
\Tint は32ビットなので、メモリのどこかに格納するには4バイトが必要で、32ビットのレジスタに正確に収まります。 
\myindex{IDA!var\_?}
ローカル変数\GTT{x}の場所がスタックに割り当てられ、 
\IDA の名前は\emph{var\_8}です。 ただし、\ac{SP}(\gls{stack pointer})がすでにその領域を指しているため、その領域を直接割り当てることはできません。 

\INS{PUSH/POP}命令は、ARMとx86とでは動作が異なります(これは逆です)。 
これらは\INS{STM/STMDB/LDM/LDMIA}命令の同義語です。 
そして、\INS{PUSH}命令は最初に値をスタックに書き込み、\emph{次に} \ac{SP} を4で減算します。
\INS{POP}は最初に\ac{SP}に4を加算してから、スタックから値を読み取ります。 
したがって、\INS{PUSH}後、\ac{SP}はスタック内の未使用スペースを指します。 
それは \scanf によって、そして後に \printf によって使用されます。

\INS{LDMIA} は \emph{Load Multiple Registers Increment address After each transfer}の略です。
\INS{STMDB} は \emph{Store Multiple Registers Decrement address Before each transfer}の略です。

したがって、\ac{SP}の値は\Reg{1}レジスタにコピーされ、フォーマット文字列とともに \scanf に渡されます。 
その後、\INS{LDR}命令の助けを借りて、この値はスタックから\Reg{1}レジスタに移動され、 \printf に渡されます。

\myparagraph{ARM64}

\lstinputlisting[caption=\NonOptimizing GCC 4.9.1 ARM64,numbers=left,style=customasmARM]{patterns/04_scanf/1_simple/ARM64_GCC491_O0_JA.s}

スタックフレームには32バイトが割り当てられており、必要なサイズよりも大きくなっています。 たぶんメモリのアラインメントの問題でしょうか? 
最も興味深いのはスタックフレーム内の$x$変数のためのスペースを見つけることです(22行目)。 
なぜ28なのでしょう? 何らかの理由で、コンパイラは、この変数をスタックフレームの最後に置きます。 
アドレスは \scanf に渡され、\scanf はユーザ入力値をそのアドレスのメモリに格納するだけです。 
これは \Tint 型の32ビット値です。 
値は27行目から取得され、 \printf に渡されます。
