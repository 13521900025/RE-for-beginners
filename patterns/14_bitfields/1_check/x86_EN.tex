\subsubsection{x86}
\myindex{Windows!Win32}

Win32 API example:

\begin{lstlisting}[style=customc]
	HANDLE fh;

	fh=CreateFile ("file", GENERIC_WRITE | GENERIC_READ, FILE_SHARE_READ, NULL, OPEN_ALWAYS, FILE_ATTRIBUTE_NORMAL, NULL);
\end{lstlisting}

We get (MSVC 2010):

\begin{lstlisting}[caption=MSVC 2010,style=customasmx86]
	push	0
	push	128		; 00000080H
	push	4
	push	0
	push	1
	push	-1073741824	; c0000000H
	push	OFFSET $SG78813
	call	DWORD PTR __imp__CreateFileA@28
	mov	DWORD PTR _fh$[ebp], eax
\end{lstlisting}

Let's take a look in WinNT.h:

\begin{lstlisting}[caption=WinNT.h,style=customc]
#define GENERIC_READ                     (0x80000000L)
#define GENERIC_WRITE                    (0x40000000L)
#define GENERIC_EXECUTE                  (0x20000000L)
#define GENERIC_ALL                      (0x10000000L)
\end{lstlisting}

Everything is clear,
GENERIC\_READ | GENERIC\_WRITE = 0x80000000 | 0x40000000 = 0xC0000000,
and that value is used as the second argument for the \TT{CreateFile()}\footnote{\href{http://go.yurichev.com/17065}{msdn.microsoft.com/en-us/library/aa363858(VS.85).aspx}}function.

How would \TT{CreateFile()} check these flags?
\myindex{Windows!KERNEL32.DLL}

If we look in KERNEL32.DLL in Windows XP SP3 x86, we'll find this fragment of code in \TT{CreateFileW}:

\begin{lstlisting}[caption=KERNEL32.DLL (Windows XP SP3 x86),style=customasmx86]
.text:7C83D429     test    byte ptr [ebp+dwDesiredAccess+3], 40h
.text:7C83D42D     mov     [ebp+var_8], 1
.text:7C83D434     jz      short loc_7C83D417
.text:7C83D436     jmp     loc_7C810817
\end{lstlisting}

\myindex{x86!\Instructions!TEST}

Here we see the \TEST instruction, however it doesn't take the whole second argument,\\
but only the most significant byte (\TT{ebp+dwDesiredAccess+3}) and checks it for flag \TT{0x40}
(which implies the \TT{GENERIC\_WRITE} flag here).
\myindex{x86!\Instructions!AND}

\TEST is basically the same instruction as \AND, but without saving the result
(recall the fact \CMP is merely the same as \SUB, but without saving the result~(\myref{CMPandSUB})).

The logic of this code fragment is as follows:

\begin{lstlisting}[style=customc]
if ((dwDesiredAccess&0x40000000) == 0) goto loc_7C83D417
\end{lstlisting}

\myindex{x86!\Instructions!AND}
\myindex{x86!\Registers!ZF}

If \AND instruction leaves this bit, the \ZF flag is to be cleared and the 
\JZ conditional jump is not to be triggered.
The conditional jump is triggered only if the \TT{0x40000000} bit is absent in \TT{dwDesiredAccess} variable~---then the result of \AND is 0,
\ZF is to be set and the conditional jump is to be triggered.

Let's try GCC 4.4.1 and Linux:

\begin{lstlisting}[style=customc]
#include <stdio.h>
#include <fcntl.h>

void main()
{
	int handle;

	handle=open ("file", O_RDWR | O_CREAT);
};
\end{lstlisting}

We get:

\lstinputlisting[caption=GCC 4.4.1,style=customasmx86]{patterns/14_bitfields/1_check/check.asm}

\myindex{Linux!libc.so.6}
\myindex{syscall}

If we take a look in the \TT{open()} function in the \TT{libc.so.6} library, it is only a syscall:

\lstinputlisting[caption=open() (libc.so.6),style=customasmx86]{patterns/14_bitfields/1_check/tmp1.asm}

So, the bit fields for \TT{open()} are apparently checked somewhere in the Linux kernel.

Of course, it is easy to download both Glibc and the Linux kernel source code, 
but we are interested in understanding the matter without it.

So, as of Linux 2.6, when the \TT{sys\_open} syscall is called, control eventually passes to \TT{do\_sys\_open},
and from there---to the \TT{do\_filp\_open()} function (it's located in the kernel source tree in \TT{fs/namei.c}).

\newcommand{\URLREGPARM}{\href{http://go.yurichev.com/17040}{ohse.de/uwe/articles/gcc-attributes.html\#func-regparm}}

\myindex{fastcall}
\label{regparm}
N.B.  Aside from passing arguments via the stack,
there is also a method of passing some of them
via registers. This is also called fastcall~(\myref{fastcall}).
This works faster since CPU does not need to access the stack in memory to read argument values.
GCC has the option \emph{regparm}\footnote{\URLREGPARM},
through which it's possible to set the number of arguments that can be passed via registers.

\newcommand{\URLKERNELNEWB}{\href{http://go.yurichev.com/17066}{kernelnewbies.org/Linux\_2\_6\_20\#head-042c62f290834eb1fe0a1942bbf5bb9a4accbc8f}}
\newcommand{\CALLINGHFILE}{arch/x86/include/asm/calling.h}

The Linux 2.6 kernel is compiled with \TT{-mregparm=3} option~\footnote{\URLKERNELNEWB}
\footnote{See also \TT{\CALLINGHFILE} file in kernel tree}.

What this means to us is that the first 3 arguments are to be passed via registers \EAX, \EDX and \ECX, 
and the rest via the stack. 
Of course, if the number of arguments is less than 3, only part of registers set is to be used.

So, let's download Linux Kernel 2.6.31, compile it in Ubuntu: \TT{make vmlinux}, open it in \IDA, 
and find the \TT{do\_filp\_open()} function. At the beginning, we see (the comments are mine):

\lstinputlisting[caption=do\_filp\_open() (linux kernel 2.6.31),style=customasmx86]{patterns/14_bitfields/1_check/check2_EN.asm}

GCC saves the values of the first 3 arguments in the local stack. 
If that wasn't done, the compiler would not touch these registers, 
and that would be too tight environment for the compiler's \gls{register allocator}.

Let's find this fragment of code:

\lstinputlisting[caption=do\_filp\_open() (linux kernel 2.6.31),style=customasmx86]{patterns/14_bitfields/1_check/check3.asm}

\TT{0x40}---is what the \TT{O\_CREAT} macro equals to.
\TT{open\_flag} gets checked for the presence of the \TT{0x40} bit, and if this bit is 1, 
the next \JNZ instruction is triggered.
