\subsection{Compter les bits mis à 1}

Voici un exemple simple d'une fonction qui compte le nombre de bits mis à 1 dans
la valeur en entrée.

Cette opération est aussi appelée \q{population count}\footnote{les CPUs x86 modernes
(qui supportent SSE4) ont même une instruction POPCNT pour cela}.

\lstinputlisting[style=customc]{patterns/14_bitfields/4_popcnt/shifts.c}

Dans cette boucle, la variable d'itération $i$ prend les valeurs de 0 à 31, donc
la déclaration $1 \ll i$ prend les valeurs de 1 à \TT{0x80000000}.
Pour décrire cette opération en langage naturel, nous dirions \emph{décaler 1 par n bits à gauche}.
En d'autres mots, la déclaration $1 \ll i$ produit consécutivement toutes les positions
possible pour un bit dans un nombre de 32-bit.
Le bit libéré à droite est toujours à 0.

\label{2n_numbers_table}
Voici une table de tous les $1 \ll i$ possible
for $i=0 \ldots 31$:

%\small
\begin{center}
\begin{tabular}{ | l | l | l | l | }
\hline
\HeaderColor \CCpp expression & 
\HeaderColor Puissance de deux & 
\HeaderColor Forme décimale & 
\HeaderColor Forme hexadécimale \\
\hline
$1 \ll 0$ & $2^{0}$ & 1 & 1 \\
\hline
$1 \ll 1$ & $2^{1}$ & 2 & 2 \\
\hline
$1 \ll 2$ & $2^{2}$ & 4 & 4 \\
\hline
$1 \ll 3$ & $2^{3}$ & 8 & 8 \\
\hline
$1 \ll 4$ & $2^{4}$ & 16 & 0x10 \\
\hline
$1 \ll 5$ & $2^{5}$ & 32 & 0x20 \\
\hline
$1 \ll 6$ & $2^{6}$ & 64 & 0x40 \\
\hline
$1 \ll 7$ & $2^{7}$ & 128 & 0x80 \\
\hline
$1 \ll 8$ & $2^{8}$ & 256 & 0x100 \\
\hline
$1 \ll 9$ & $2^{9}$ & 512 & 0x200 \\
\hline
$1 \ll 10$ & $2^{10}$ & 1024 & 0x400 \\
\hline
$1 \ll 11$ & $2^{11}$ & 2048 & 0x800 \\
\hline
$1 \ll 12$ & $2^{12}$ & 4096 & 0x1000 \\
\hline
$1 \ll 13$ & $2^{13}$ & 8192 & 0x2000 \\
\hline
$1 \ll 14$ & $2^{14}$ & 16384 & 0x4000 \\
\hline
$1 \ll 15$ & $2^{15}$ & 32768 & 0x8000 \\
\hline
$1 \ll 16$ & $2^{16}$ & 65536 & 0x10000 \\
\hline
$1 \ll 17$ & $2^{17}$ & 131072 & 0x20000 \\
\hline
$1 \ll 18$ & $2^{18}$ & 262144 & 0x40000 \\
\hline
$1 \ll 19$ & $2^{19}$ & 524288 & 0x80000 \\
\hline
$1 \ll 20$ & $2^{20}$ & 1048576 & 0x100000 \\
\hline
$1 \ll 21$ & $2^{21}$ & 2097152 & 0x200000 \\
\hline
$1 \ll 22$ & $2^{22}$ & 4194304 & 0x400000 \\
\hline
$1 \ll 23$ & $2^{23}$ & 8388608 & 0x800000 \\
\hline
$1 \ll 24$ & $2^{24}$ & 16777216 & 0x1000000 \\
\hline
$1 \ll 25$ & $2^{25}$ & 33554432 & 0x2000000 \\
\hline
$1 \ll 26$ & $2^{26}$ & 67108864 & 0x4000000 \\
\hline
$1 \ll 27$ & $2^{27}$ & 134217728 & 0x8000000 \\
\hline
$1 \ll 28$ & $2^{28}$ & 268435456 & 0x10000000 \\
\hline
$1 \ll 29$ & $2^{29}$ & 536870912 & 0x20000000 \\
\hline
$1 \ll 30$ & $2^{30}$ & 1073741824 & 0x40000000 \\
\hline
$1 \ll 31$ & $2^{31}$ & 2147483648 & 0x80000000 \\
\hline
\end{tabular}
\end{center}
%\normalsize

Ces constantes (masques de bit) apparaissent très souvent le code et un rétro-ingénieur
pratiquant doit pouvoir les repérer rapidement.

Les nombres décimaux avant 65536 et les hexadécimaux sont faciles à mémoriser.
Tandis que les nombres décimaux après 65536 ne valent probablement pas la peine de
l'être.

Ces constantes sont utilisées très souvent pour mapper des flags sur des bits spécifiques.
Par exemple, voici un extrait de \TT{ssl\_private.h} du code source d'Apache 2.4.6:

\begin{lstlisting}[style=customc]
/**
 * Define the SSL options
 */
#define SSL_OPT_NONE           (0)
#define SSL_OPT_RELSET         (1<<0)
#define SSL_OPT_STDENVVARS     (1<<1)
#define SSL_OPT_EXPORTCERTDATA (1<<3)
#define SSL_OPT_FAKEBASICAUTH  (1<<4)
#define SSL_OPT_STRICTREQUIRE  (1<<5)
#define SSL_OPT_OPTRENEGOTIATE (1<<6)
#define SSL_OPT_LEGACYDNFORMAT (1<<7)
\end{lstlisting}

Revenons à notre exemple.

La macro \TT{IS\_SET} teste la présence d'un bit dans $a$.
\myindex{x86!\Instructions!AND}

La macro \TT{IS\_SET} est en fait l'opération logique AND (\emph{AND}) et elle renvoie
0 si le bit testé est absent (à 0), ou le masque de bit, si le bit est présent (à 1).
L'opérateur \emph{if()} en \CCpp exécute son code si l'expression n'est pas zéro,
cela peut même être 123456, c'est pourquoi il fonctionne toujours correctement.

% subsections
\subsubsection{x86}

Le résultat de la compilation est:

\lstinputlisting[caption=MSVC 2012 /GS- /Ob0,label=src:struct_packing_4,numbers=left,style=customasmx86]{patterns/15_structs/4_packing/packing_FR.asm}

Nous passons la structure comme un tout, mais en réalité nous pouvons constater que la structure est copiée 
dans un espace temporaire. De l'espace est réservé pour cela ligne 10 et les 4 champs sont copiées par les 
lignes de 12 \ldots\ 19), puis le pointeur sur l'espace temporaire est passé à la fonction.

La structure est recopiée au cas où la fonction \ttf{} viendrait à en modifier le contenu. Si cela arrive, 
la copie de la structure qui existe dans \main restera inchangée.

Nous pourrions également utiliser des pointeurs \CCpp. Le résulta demeurerait le même, sans qu'il soit 
nécessaire de procéder à la copie.

Nous observons que l'adresse de chaque champ est alignée sur un multiple de 4  octets. C'est pourquoi chaque 
\Tchar occupe 4 octets (de même qu'un \Tint). Pourquoi en est-il ainsi? La réponse se situe au niveau de la 
CPU. Il est plus facile et performant pour elle d'accéder la mémoire et de gérer le cache de données en 
utilisant des adresses alignées.

En revanche ce n'est pas très économique en terme d'espace.

Tentons maintenant une compilation avec l'option (\TT{/Zp1}) (\emph{/Zp[n] indique qu'il faut compresser les 
structures en utilisant des frontières tous les n octets}).

\lstinputlisting[caption=MSVC 2012 /GS- /Zp1,label=src:struct_packing_1,numbers=left,style=customasmx86]{patterns/15_structs/4_packing/packing_msvc_Zp1_FR.asm}

La structure n'occupe plus que 10 octets et chaque valeur de type \Tchar n'occupe plus qu'un octet. Quelles 
sont les conséquences ? Nous économisons de la place au prix d'un accès à ces champs moins rapide que ne 
pourrait le faire la CPU.

\label{short_struct_copying_using_MOV}

La structure est également copiée dans \main. Cette opération ne s'effectue pas champ par champ mais par 
blocs en utilisant trois instructions \MOV. Et pourquoi pas 4 ?

Tout simplement parce que le compilateur a décidé qu'il était préférable d'effectuer la copie en utilisant 
3 paires d'instructions \MOV plutôt que de copier deux mots de 32 bits puis 2 fois un octet ce qui aurait 
nécessité 4 paires d'instructions \MOV.

Ce type d'implémentation de la copie qui repose sur les instructions \MOV plutôt que sur l'appel à la 
fonction \TT{memcpy()} est très répandu. La raison en est que pour de petits blocs, cette approche est 
plus rapide qu'un appel à \TT{memcpy()}: \myref{copying_short_blocks}.

Comme vous pouvez le deviner, si la structure est utilisée dans de nombreux fichiers sources et objets, ils 
doivent tous être compilés avec la même convention de compactage de la structure.

\newcommand{\FNURLMSDNZP}{\footnote{\href{http://go.yurichev.com/17067}
{MSDN: Working with Packing Structures}}}
\newcommand{\FNURLGCCPC}{\footnote{\href{http://go.yurichev.com/17068}
{Structure-Packing Pragmas}}}

Au delà de l'option MSVC \TT{/Zp} qui permet de définir l'alignement des champs des structures, il existe 
également l'option du compilateur \TT{\#pragma pack} qui peut être utilisée directement dans le code source.
Elle est supportée aussi bien par MSVC\FNURLMSDNZP que pars GCC\FNURLGCCPC{}.

Revenons à la structure \TT{SYSTEMTIME} qui contient des champs de 16 bits. Comment notre compilateur sait-il 
les aligner sur des frontières de 1 octet ?

Le fichier \TT{WinNT.h} contient ces instructions:

\begin{lstlisting}[caption=WinNT.h,style=customc]
#include "pshpack1.h"
\end{lstlisting}

et celles-ci:

\begin{lstlisting}[caption=WinNT.h,style=customc]
#include "pshpack4.h"                   // L'alignement sur 4 octets est la valeur par défaut
\end{lstlisting}

Le fichier PshPack1.h ressemble à ceci:

\lstinputlisting[caption=PshPack1.h,style=customc]{patterns/15_structs/4_packing/tmp1.c}

Ces instructions indiquent au compilateur comment compresser les structures définies après \TT{\#pragma pack}.

\clearpage
\myparagraph{MSVC + \olly}
\myindex{\olly}

2 paires de mots 32-bit sont marquées en rouge sur la pile.
Chaque paire est un double au format IEEE 754 et est passée depuis \main.

Nous voyons comment le premier \FLD charge une valeur ($1.2$) depuis la pile et la
stocke dans \ST{0}:

\begin{figure}[H]
\centering
\myincludegraphics{patterns/12_FPU/1_simple/olly1.png}
\caption{\olly: le premier \FLD a été exécuté}
\label{fig:FPU_simple_olly_1}
\end{figure}

À cause des inévitables erreurs de conversion depuis un nombre flottant 64-bit au
format IEEE 754 vers 80-bit (utilisé en interne par le FPU), ici nous voyons 1.199\ldots,
qui est proche de 1.2.

\EIP pointe maintenant sur l'instruction suivante (\FDIV), qui charge un double
(une constante) depuis la mémoire.
Par commodité, \olly affiche sa valeur: 3.14

\clearpage
Continuons l'exécution pas à pas.
\FDIV a été exécuté, maintenant \ST{0} contient 0.382\ldots
(\gls{quotient}):

\begin{figure}[H]
\centering
\myincludegraphics{patterns/12_FPU/1_simple/olly2.png}
\caption{\olly: \FDIV a été exécuté}
\label{fig:FPU_simple_olly_2}
\end{figure}

\clearpage
Troisième étape: le \FLD suivant a été exécuté, chargeant 3.4 dans \ST{0} (ici nous
voyons la valeur approximative 3.39999\ldots):

\begin{figure}[H]
\centering
\myincludegraphics{patterns/12_FPU/1_simple/olly3.png}
\caption{\olly: le second \FLD a été exécuté}
\label{fig:FPU_simple_olly_3}
\end{figure}

En même temps, le \gls{quotient} \emph{est poussé} dans \ST{1}.
Exactement maintenant, \EIP pointe sur la prochaine instruction: \FMUL.
Ceci charge la constante 4.1 depuis la mémoire, ce que montre \olly.

\clearpage
Suivante: \FMUL a été exécutée, donc maintenant le \glslink{product}{produit} est dans \ST{0}:

\begin{figure}[H]
\centering
\myincludegraphics{patterns/12_FPU/1_simple/olly4.png}
\caption{\olly: \FMUL a été exécuté}
\label{fig:FPU_simple_olly_4}
\end{figure}

\clearpage
Suivante: \FADDP a été exécutée, maintenant le résultat de l'addition est dans \ST{0},
et \ST{1} est vidé.

\begin{figure}[H]
\centering
\myincludegraphics{patterns/12_FPU/1_simple/olly5.png}
\caption{\olly: \FADDP a été exécuté}
\label{fig:FPU_simple_olly_5}
\end{figure}

Le résultat est laissé dans \ST{0}, car la fonction renvoie son résultat dans \ST{0}.

\main prend cette valeur depuis le registre plus loin.

Nous voyons quelque chose d'inhabituel: la valeur 13.93\ldots se trouve maintenant
dans \ST{7}.
Pourquoi?

\label{FPU_is_rather_circular_buffer}

Comme nous l'avons lu il y a quelque temps dans ce livre, les registres \ac{FPU} sont
une pile: \myref{FPU_is_stack}.
Mais ceci est une simplification.

Imaginez si cela était implémenté \emph{en hardware} comme cela est décrit, alors
tout le contenu des 7 registres devrait être déplacé (ou copié) dans les registres
adjacents lors d'un push ou d'un pop, et ceci nécessite beaucoup de travail.

En réalité, le \ac{FPU} a seulement 8 registres et un pointeur (appelé \GTT{TOP})
qui contient un numéro de registre, qui est le \q{haut de la pile} courant.

Lorsqu'une valeur est poussée sur la pile, \GTT{TOP} est déplacé sur le registre
disponible suivant, et une valeur est écrite dans ce registre.

La procédure est inversée si la valeur est lue, toutefois, le registre qui a été
libéré n'est pas vidé (il serait possible de le vider, mais ceci nécessite plus de
travail qui peut dégrader les performances).
Donc, c'est ce que nous voyons ici.

On peut dire que \FADDP sauve la somme sur la pile, et y supprime un élément.

Mais en fait, cette instruction sauve la somme et ensuite décale \GTT{TOP}.

Plus précisément, les registres du  \ac{FPU} sont un tampon circulaire.


\subsubsection{x64}
\label{subsec:popcnt}

Modifions légèrement l'exemple pour l'étendre à 64-bit:

\lstinputlisting[label=popcnt_x64_example,style=customc]{patterns/14_bitfields/4_popcnt/shifts64.c}

\myparagraph{GCC 4.8.2 \NonOptimizing}

Jusqu'ici, c'est facile.

\lstinputlisting[caption=GCC 4.8.2 \NonOptimizing,style=customasmx86]{patterns/14_bitfields/4_popcnt/shifts64_GCC_O0_FR.s}

\myparagraph{GCC 4.8.2 \Optimizing}

\lstinputlisting[caption=GCC 4.8.2 \Optimizing,numbers=left,label=shifts64_GCC_O3,style=customasmx86]{patterns/14_bitfields/4_popcnt/shifts64_GCC_O3_FR.s}

Ce code est plus concis, mais a une particularité.

% TODO: comment traduire commits ?
Dans tous les exemples que nous avons vu jusqu'ici, nous incrémentions la valeur
de \q{rt} après la comparaison d'un bit spécifique, mais le code ici incrémente \q{rt}
avant (ligne 6), écrivant la nouvelle valeur dans le registre \EDX.
Donc, si le dernier bit est à 1, l'instruction \CMOVNE\footnote{Conditional MOVe if Not Equal}
(qui est un synonyme pour \CMOVNZ\footnote{Conditional MOVe if Not Zero}) \emph{commits}
la nouvelle valeur de \q{rt} en déplaçant  \EDX (\q{valeur proposée de rt}) dans
\EAX (\q{rt courant} qui va être retourné à la fin).

C'est pourquoi l'incrémentation est effectuée à chaque étape de la boucle, i.e.,
64 fois, sans relation avec la valeur en entrée.

L'avantage de ce code est qu'il contient seulement un saut conditionnel (à la fin
de la boucle) au lieu de deux sauts (évitant l'incrément de la valeur de \q{rt} et
à la fin de la boucle).
Et cela doit s'exécuter plus vite sur les CPUs modernes avec des prédicteurs de branchement:
\myref{branch_predictors}.

\label{FATRET}
\myindex{x86!\Instructions!FATRET}
La dernière instruction est \INS{REP RET} (opcode \TT{F3 C3}) qui est aussi appelée
\INS{FATRET} par MSVC.
C'est en quelque sorte une version optimisée de \RET, qu'AMD recommande de mettre
en fin de fonction, si \RET se trouve juste après un saut conditionnel:
\InSqBrackets{\AMDOptimization p.15}
\footnote{Lire aussi à ce propos: \url{http://go.yurichev.com/17328}}.

\myparagraph{MSVC 2010 \Optimizing}

\lstinputlisting[caption=MSVC 2010 \Optimizing,style=customasmx86]{patterns/14_bitfields/4_popcnt/MSVC_2010_x64_Ox_FR.asm}

\myindex{x86!\Instructions!ROL}
Ici l'instruction \ROL est utilisée au lieu de \SHL, qui est en fait \q{rotate left /
pivoter à gauche} au lieu de \q{shift left / décaler à gauche}, mais dans cet exemple
elle fonctionne tout comme \TT{SHL}.

Vous pouvez en lire plus sur l'instruction de rotation ici: \myref{ROL_ROR}.

\Reg{8} ici est compté de 64 à 0.
C'est tout comme un $i$ inversé.

Voici une table de quelques registres pendant l'exécution:

\begin{center}
\begin{tabular}{ | l | l | }
\hline
\HeaderColor RDX & \HeaderColor R8 \\
\hline
0x0000000000000001 & 64 \\
\hline
0x0000000000000002 & 63 \\
\hline
0x0000000000000004 & 62 \\
\hline
0x0000000000000008 & 61 \\
\hline
... & ... \\
\hline
0x4000000000000000 & 2 \\
\hline
0x8000000000000000 & 1 \\
\hline
\end{tabular}
\end{center}

\myindex{x86!\Instructions!FATRET}
À la fin, nous voyons l'instruction \INS{FATRET}, qui a été expliquée ici: \myref{FATRET}.

\myparagraph{MSVC 2012 \Optimizing}

\lstinputlisting[caption=MSVC 2012 \Optimizing,style=customasmx86]{patterns/14_bitfields/4_popcnt/MSVC_2012_x64_Ox_FR.asm}

\myindex{\CompilerAnomaly}
\label{MSVC2012_anomaly}
MSVC 2012 \Optimizing fait presque le même job que MSVC 2010 \Optimizing, mais en
quelque sorte, il génère deux corps de boucles identiques et le nombre de boucles
est maintenant 32 au lieu de 64.

Pour être honnête, il n'est pas possible de dire pourquoi. Une ruse d'optimisation?
Peut-être est-il meilleur pour le corps de la boucle d'être légèrement plus long?

De toute façon, ce genre de code est pertinent ici pour montrer que parfois la sortie
du compilateur peut être vraiment bizarre et illogique, mais fonctionner parfaitement.


\subsubsection{ARM + \OptimizingXcodeIV (\ARMMode)}

\lstinputlisting[caption=\OptimizingXcodeIV (\ARMMode),label=ARM_leaf_example4,style=customasmARM]{patterns/14_bitfields/4_popcnt/ARM_Xcode_O3_FR.lst}

\myindex{ARM!\Instructions!TST}
\TST est la même chose que \TEST en x86.

\myindex{ARM!Optional operators!LSL}
\myindex{ARM!Optional operators!LSR}
\myindex{ARM!Optional operators!ASR}
\myindex{ARM!Optional operators!ROR}
\myindex{ARM!Optional operators!RRX}
\myindex{ARM!\Instructions!MOV}
\myindex{ARM!\Instructions!TST}
\myindex{ARM!\Instructions!CMP}
\myindex{ARM!\Instructions!ADD}
\myindex{ARM!\Instructions!SUB}
\myindex{ARM!\Instructions!RSB}
Comme noté précédemment~(\myref{shifts_in_ARM_mode}), il n'y a pas d'instruction
de décalage séparée en mode ARM
Toutefois, il y a ces modificateurs
LSL (\emph{Logical Shift Left / décalage logique à gauche}), 
LSR (\emph{Logical Shift Right / décalage logique à droite}), 
ASR (\emph{Arithmetic Shift Right décalage arithmétique à droite}), 
ROR (\emph{Rotate Right / rotation à droite}) et
RRX (\emph{Rotate Right with Extend / rotation à droite avec extension}),
qui peuvent être ajoutés à des instructions comme \MOV, \TST,
\CMP, \ADD, \SUB, \RSB\footnote{\DataProcessingInstructionsFootNote}.

Ces modificateurs définissent comment décaler le second opérande et de combien de
bits.

\myindex{ARM!\Instructions!TST}
\myindex{ARM!Optional operators!LSL}
Ainsi l'instruction \TT{\q{TST R1, R2,LSL R3}} fonctionne ici comme $R1 \land (R2
\ll R3)$.

\subsubsection{ARM + \OptimizingXcodeIV (\ThumbTwoMode)}

\myindex{ARM!\Instructions!LSL.W}
\myindex{ARM!\Instructions!LSL}
Presque la même, mais ici il y a deux instructions utilisées, \INS{LSL.W}/\TST, au
lieu d'une seule \TST, car en mode Thumb il n'est pas possible de définir le modificateur
\LSL directement dans \TST.

\begin{lstlisting}[label=ARM_leaf_example5,style=customasmARM]
                MOV             R1, R0
                MOVS            R0, #0
                MOV.W           R9, #1
                MOVS            R3, #0
loc_2F7A
                LSL.W           R2, R9, R3
                TST             R2, R1
                ADD.W           R3, R3, #1
                IT NE
                ADDNE           R0, #1
                CMP             R3, #32
                BNE             loc_2F7A
                BX              LR
\end{lstlisting}

\subsubsection{ARM64 + GCC 4.9 \Optimizing}

Prenons un exemple en 64.bit qui a déjà été utilisé: \myref{popcnt_x64_example}.

\lstinputlisting[caption=GCC (Linaro) 4.8 \Optimizing,style=customasmARM]{patterns/14_bitfields/4_popcnt/ARM64_GCC_O3_FR.s}

Le résultat est très semblable à ce que GCC génère pour x64: \myref{shifts64_GCC_O3}.

\myindex{ARM!\Instructions!CSEL}
L'instruction \CSEL signifie \q{Conditional SELect / sélection conditionnelle}.
Elle choisi une des deux variables en fonction des flags mis par \TST et copie la
valeur dans \RegW{2}, qui contient la variable \q{rt}.

\subsubsection{ARM64 + GCC 4.9 \NonOptimizing}

De nouveau, nous travaillons sur un exemple 64-bit qui a déjà été utilisé: \myref{popcnt_x64_example}.
Le code est plus verbeux, comme d'habitude.

\lstinputlisting[caption=\NonOptimizing GCC (Linaro) 4.8,style=customasmARM]{patterns/14_bitfields/4_popcnt/ARM64_GCC_O0_FR.s}


\subsubsection{MIPS}
% FIXME better start at non-optimizing version?

La fonction utilise beaucoup de S- registres qui doivent être préservés, c'est pourquoi
leurs valeurs sont sauvegardées dans la prologue de la fonction et restaurées dans
l'épilogue.

\lstinputlisting[caption=GCC 4.4.5 \Optimizing (IDA),style=customasmMIPS]{patterns/13_arrays/1_simple/MIPS_O3_IDA_FR.lst}

Quelque chose d'intéressant: il y a deux boucles et la première n'a pas besoin de
$i$, elle a seulement besoin de $i*2$ (augmenté de 2 à chaque itération) et aussi
de l'adresse en mémoire (augmentée de 4 à chaque itération).

Donc ici nous voyons deux variables, une (dans \$V0) augmentée de 2 à chaque fois,
et une autre (dans \$V1) --- de 4.

La seconde boucle est celle où \printf est appelée et affiche la valeur de $i$ à
l'utilisateur, donc il y a une variable qui est incrémentée de 1 à chaque fois (dans
\$S0) et aussi l'adresse en mémoire (dans \$S1) incrémentée de 4 à chaque fois.

Cela nous rappelle l'optimisation de boucle que nous avons examiné avant: \myref{loop_iterators}.

Leur but est de se passer des multiplications.


