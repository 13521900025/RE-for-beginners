\subsubsection{MIPS}

\myparagraph{\NonOptimizing GCC}

\lstinputlisting[caption=\NonOptimizing GCC 4.4.5 (IDA),style=customasmMIPS]{patterns/14_bitfields/4_popcnt/MIPS_O0_IDA_JA.lst}

\myindex{MIPS!\Instructions!SLL}
\myindex{MIPS!\Instructions!SLLV}

これは冗長です。ローカル変数はすべてローカルスタックに配置され、必要なとき毎にリロードされます。

\SLLV 命令は\q{Shift Word Left Logical Variable}で、 \SLL との違いは \SLL 命令にエンコードされたシフトの量
だけです。(そして結果として固定されています)
しかし、 \SLLV はレジスタからシフト量を取ってきます。

\myparagraph{\Optimizing GCC}

これはより簡潔です。
1つではなく2つシフト命令がありますが、
なぜでしょうか?

最初の \SLLV 命令を 2つめの \SLLV にジャンプする無条件分岐命令に置き換えることは可能です。
しかし、これは関数内の別の分岐命令であり、常にそれらを取り除くのは好都合です:\myref{branch_predictors}

\lstinputlisting[caption=\Optimizing GCC 4.4.5 (IDA),style=customasmMIPS]{patterns/14_bitfields/4_popcnt/MIPS_O3_IDA_JA.lst}

