\subsubsection{x64}
\label{subsec:popcnt}

例を64ビットに拡張するように少し変更してみましょう。

\lstinputlisting[label=popcnt_x64_example,style=customc]{patterns/14_bitfields/4_popcnt/shifts64.c}

\myparagraph{\NonOptimizing GCC 4.8.2}

ここまでは簡単です。

\lstinputlisting[caption=\NonOptimizing GCC 4.8.2,style=customasmx86]{patterns/14_bitfields/4_popcnt/shifts64_GCC_O0_JA.s}

\myparagraph{\Optimizing GCC 4.8.2}

\lstinputlisting[caption=\Optimizing GCC 4.8.2,numbers=left,label=shifts64_GCC_O3,style=customasmx86]{patterns/14_bitfields/4_popcnt/shifts64_GCC_O3_JA.s}

コードは簡潔ですが、曖昧なところがあります。

これまでのすべての例では、特定のビットを比較した後に \q{rt}値をインクリメントしていましたが、
ここでコードは \q{rt}を先に増やして(6行目)、新しい値をレジスタ \EDX に書き込みます。
したがって、最後のビットが1である場合、 \CMOVNE\footnote{Conditional MOVe if Not Equal} 命令
(\CMOVNZ\footnote{Conditional MOVe if Not Zero}と同義)は、 \EDX (\q{提案されたrt値})
を \EAX (最後にリターンされる\q{現在のrt})に戻すことによって新しい値\q{rt}を\emph{コミット}します。

したがって、ループの各ステップで、言い換えると64回入力値に関係なく、インクリメントが実行されます。

このコードの利点は、2つのジャンプ( ループの最後で\q{rt}値のインクリメントをスキップする)ではなく、
条件ジャンプを1つだけ(ループの最後に)含むことです。
そして、それは分岐予測を持つ現代のCPUでより速く動作するでしょう:\myref{branch_predictors}

\label{FATRET}
\myindex{x86!\Instructions!FATRET}
最後の命令は、MSVCによって\INS{FATRET}とも呼ばれる
\INS{REP RET} (オペコード \TT{F3 C3})です。
これは、 \RET のいくらか最適化されたバージョンであり、
\RET が条件ジャンプの直後にある場合、AMDは関数の最後に置くことを推奨しています:
\InSqBrackets{\AMDOptimization p.15}
\footnote{詳細はこちら: \url{http://go.yurichev.com/17328}}.

\myparagraph{\Optimizing MSVC 2010}

\lstinputlisting[caption=\Optimizing MSVC 2010,style=customasmx86]{patterns/14_bitfields/4_popcnt/MSVC_2010_x64_Ox_JA.asm}

\myindex{x86!\Instructions!ROL}
ここでは、 \SHL の代わりに 
\ROL 命令が使用されています。
実際には、 \q{shift left}ではなく 
\q{rotate left}ですが、
この例では \SHL と同じように動作します。

ローテート命令の詳細については、こちらをご覧ください:\myref{ROL_ROR}

ここの\Reg{8}は64から0まで数えています。
$i$を逆にしたようなものです。

実行中のレジスタのテーブルを以下に示します。

\begin{center}
\begin{tabular}{ | l | l | }
\hline
\HeaderColor RDX & \HeaderColor R8 \\
\hline
0x0000000000000001 & 64 \\
\hline
0x0000000000000002 & 63 \\
\hline
0x0000000000000004 & 62 \\
\hline
0x0000000000000008 & 61 \\
\hline
... & ... \\
\hline
0x4000000000000000 & 2 \\
\hline
0x8000000000000000 & 1 \\
\hline
\end{tabular}
\end{center}

\myindex{x86!\Instructions!FATRET}
最後に、\INS{FATRET}命令がありますが、それは\myref{FATRET}で説明します。

\myparagraph{\Optimizing MSVC 2012}

\lstinputlisting[caption=\Optimizing MSVC 2012,style=customasmx86]{patterns/14_bitfields/4_popcnt/MSVC_2012_x64_Ox_JA.asm}

\myindex{\CompilerAnomaly}
\label{MSVC2012_anomaly}
\Optimizing MSVC 2012 は最適化されたMSVC 2010とほとんど同じことをしますが、
どういうわけか、2つの同じループボディを生成して、ループカウントが64ではなく32です。

正直なところ、なぜかはわかりません。何か最適化のトリックでしょうか。ループボディを
もう少し長くしたほうがよいのかもしれません。

とにかく、コンパイラの出力が本当に奇妙で非論理的なことがありますが、
このようなコードは完全に動作することを示すためにあげました。
