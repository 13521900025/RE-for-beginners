\subsection{Counting bits set to 1}

Here is a simple example of a function that calculates the number of bits set in the input value.

This operation is also called \q{population count}\footnote{modern x86 CPUs (supporting SSE4) even have a POPCNT instruction for it}.

\lstinputlisting[style=customc]{patterns/14_bitfields/4_popcnt/shifts.c}

In this loop, the iteration count value $i$ is counting from 0 to 31, 
so the $1 \ll i$ statement is counting from 1 to \TT{0x80000000}.
Describing this operation in natural language, we would say \IT{shift 1 by n bits left}.
In other words, $1 \ll i$ statement consequently produces all possible bit positions in a 32-bit number.
The freed bit at right is always cleared.

\label{2n_numbers_table}
Here is a table of all possible $1 \ll i$ 
for $i=0 \ldots 31$:

%\small
\begin{center}
\begin{tabular}{ | l | l | l | l | }
\hline
\HeaderColor \CCpp expression & 
\HeaderColor Power of two & 
\HeaderColor Decimal form & 
\HeaderColor Hexadecimal form \\
\hline
$1 \ll 0$ & $2^{0}$ & 1 & 1 \\
\hline
$1 \ll 1$ & $2^{1}$ & 2 & 2 \\
\hline
$1 \ll 2$ & $2^{2}$ & 4 & 4 \\
\hline
$1 \ll 3$ & $2^{3}$ & 8 & 8 \\
\hline
$1 \ll 4$ & $2^{4}$ & 16 & 0x10 \\
\hline
$1 \ll 5$ & $2^{5}$ & 32 & 0x20 \\
\hline
$1 \ll 6$ & $2^{6}$ & 64 & 0x40 \\
\hline
$1 \ll 7$ & $2^{7}$ & 128 & 0x80 \\
\hline
$1 \ll 8$ & $2^{8}$ & 256 & 0x100 \\
\hline
$1 \ll 9$ & $2^{9}$ & 512 & 0x200 \\
\hline
$1 \ll 10$ & $2^{10}$ & 1024 & 0x400 \\
\hline
$1 \ll 11$ & $2^{11}$ & 2048 & 0x800 \\
\hline
$1 \ll 12$ & $2^{12}$ & 4096 & 0x1000 \\
\hline
$1 \ll 13$ & $2^{13}$ & 8192 & 0x2000 \\
\hline
$1 \ll 14$ & $2^{14}$ & 16384 & 0x4000 \\
\hline
$1 \ll 15$ & $2^{15}$ & 32768 & 0x8000 \\
\hline
$1 \ll 16$ & $2^{16}$ & 65536 & 0x10000 \\
\hline
$1 \ll 17$ & $2^{17}$ & 131072 & 0x20000 \\
\hline
$1 \ll 18$ & $2^{18}$ & 262144 & 0x40000 \\
\hline
$1 \ll 19$ & $2^{19}$ & 524288 & 0x80000 \\
\hline
$1 \ll 20$ & $2^{20}$ & 1048576 & 0x100000 \\
\hline
$1 \ll 21$ & $2^{21}$ & 2097152 & 0x200000 \\
\hline
$1 \ll 22$ & $2^{22}$ & 4194304 & 0x400000 \\
\hline
$1 \ll 23$ & $2^{23}$ & 8388608 & 0x800000 \\
\hline
$1 \ll 24$ & $2^{24}$ & 16777216 & 0x1000000 \\
\hline
$1 \ll 25$ & $2^{25}$ & 33554432 & 0x2000000 \\
\hline
$1 \ll 26$ & $2^{26}$ & 67108864 & 0x4000000 \\
\hline
$1 \ll 27$ & $2^{27}$ & 134217728 & 0x8000000 \\
\hline
$1 \ll 28$ & $2^{28}$ & 268435456 & 0x10000000 \\
\hline
$1 \ll 29$ & $2^{29}$ & 536870912 & 0x20000000 \\
\hline
$1 \ll 30$ & $2^{30}$ & 1073741824 & 0x40000000 \\
\hline
$1 \ll 31$ & $2^{31}$ & 2147483648 & 0x80000000 \\
\hline
\end{tabular}
\end{center}
%\normalsize

These constant numbers (bit masks) very often appear in code and a practicing reverse engineer 
must be able to spot them quickly.

Decimal numbers below 65536 and hexadecimal ones are very easy to memorize.
While decimal numbers above 65536 are, probably, not worth memorizing.

These constants are very often used for mapping flags to specific bits.
For example, here is excerpt from \TT{ssl\_private.h} 
from Apache 2.4.6 source code:

\begin{lstlisting}[style=customc]
/**
 * Define the SSL options
 */
#define SSL_OPT_NONE           (0)
#define SSL_OPT_RELSET         (1<<0)
#define SSL_OPT_STDENVVARS     (1<<1)
#define SSL_OPT_EXPORTCERTDATA (1<<3)
#define SSL_OPT_FAKEBASICAUTH  (1<<4)
#define SSL_OPT_STRICTREQUIRE  (1<<5)
#define SSL_OPT_OPTRENEGOTIATE (1<<6)
#define SSL_OPT_LEGACYDNFORMAT (1<<7)
\end{lstlisting}

Let's get back to our example.

The \TT{IS\_SET} macro checks bit presence in $a$.
\myindex{x86!\Instructions!AND}

The \TT{IS\_SET} macro is in fact the logical AND operation (\IT{AND}) 
and it returns 0 if the specific bit is absent there,
or the bit mask, if the bit is present.
\IT{The if()} operator in \CCpp triggers if the expression in it is not zero, it might be even 123456, that is why
it always works correctly.

% subsections
\subsubsection{x86}

\myindex{x86!\Instructions!LOOP}

There is a special \LOOP instruction in x86 instruction set for checking the value in register \ECX and 
if it is not 0, to \gls{decrement} \ECX
and pass control flow to the label in the \LOOP operand. 
Probably this instruction is not very convenient, and there are no any modern compilers which emit it automatically.
So, if you see this instruction somewhere in code, it is most likely that this is a manually written piece 
of assembly code.

\par

In \CCpp loops are usually constructed using \TT{for()}, \TT{while()} or \TT{do/while()} statements.

Let's start with \TT{for()}.
\myindex{\CLanguageElements!for}

This statement defines loop initialization (set loop counter to initial value), 
loop condition (is the counter bigger than a limit?), what is performed at each iteration (\gls{increment}/\gls{decrement})
and of course loop body.

\lstinputlisting[style=customc]{patterns/09_loops/simple/loops_1_EN.c}

The generated code is consisting of four parts as well.

Let's start with a simple example:

\lstinputlisting[label=loops_src,style=customc]{patterns/09_loops/simple/loops_2.c}

The result (MSVC 2010):

\lstinputlisting[caption=MSVC 2010,style=customasmx86]{patterns/09_loops/simple/1_MSVC_EN.asm}

As we see, nothing special.

GCC 4.4.1 emits almost the same code, with one subtle difference:

\lstinputlisting[caption=GCC 4.4.1,style=customasmx86]{patterns/09_loops/simple/1_GCC_EN.asm}

Now let's see what we get with optimization turned on (\TT{\Ox}):

\lstinputlisting[caption=\Optimizing MSVC,style=customasmx86]{patterns/09_loops/simple/1_MSVC_Ox.asm}

What happens here is that space for the $i$ variable is not allocated in the local stack anymore,
but uses an individual register for it, \ESI.
This is possible in such small functions where there aren't many local variables.

One very important thing is that the \ttf function must not change the value in \ESI.
Our compiler is sure here. 
And if the compiler decides to use the \ESI register in \ttf too, its value would have to be saved 
at the function's prologue and restored at the function's epilogue,
almost like in our listing: please note \TT{PUSH ESI/POP ESI}
at the function start and end.

Let's try GCC 4.4.1 with maximal optimization turned on (\Othree option):

\lstinputlisting[caption=\Optimizing GCC 4.4.1,style=customasmx86]{patterns/09_loops/simple/1_GCC_O3.asm}

\myindex{Loop unwinding}

Huh, GCC just unwound our loop.

\Gls{loop unwinding} has an advantage in the cases when there aren't much iterations and 
we could cut some execution time by removing all loop support instructions. 
On the other side, the resulting code is obviously larger.

Big unrolled loops are not recommended in modern times, because bigger functions
may require bigger cache footprint%
%
\footnote{A very good article about it: \DrepperMemory.
Another recommendations about loop unrolling from Intel are here: 
\InSqBrackets{\IntelOptimization 3.4.1.7}.}.

OK, let's increase the maximum value of the $i$ variable to 100 and try again. GCC does:

\lstinputlisting[caption=GCC,style=customasmx86]{patterns/09_loops/simple/2_GCC_EN.asm}

It is quite similar to what MSVC 2010 with optimization (\Ox) produce, 
with the exception that the \EBX register is allocated for the $i$ variable.

GCC is sure this register will not be modified inside of the \ttf function, 
and if it will, it will be saved at the function prologue and restored at epilogue, 
just like here in the \main function.

\clearpage
\subsubsection{x86: \olly}
\myindex{\olly}

Let's compile our example in MSVC 2010 with \Ox and \Obzero 
options and load it into \olly.

It seems that \olly is able to detect simple loops and show them in square brackets, for convenience:

\begin{figure}[H]
\centering
\myincludegraphics{patterns/09_loops/simple/olly1.png}
\caption{\olly: \main begin}
\label{fig:loops_olly_1}
\end{figure}

By tracing (F8~--- \stepover) we see \ESI 
\glslink{increment}{incrementing}.
Here, for instance, $ESI=i=6$:

\begin{figure}[H]
\centering
\myincludegraphics{patterns/09_loops/simple/olly2.png}
\caption{\olly: loop body just executed with $i=6$}
\label{fig:loops_olly_2}
\end{figure}

9 is the last loop value.
That's why \JL is not triggering after the \gls{increment}, and the function will finish:

\begin{figure}[H]
\centering
\myincludegraphics{patterns/09_loops/simple/olly3.png}
\caption{\olly: $ESI=10$, loop end}
\label{fig:loops_olly_3}
\end{figure}

\subsubsection{x86: tracer}
\myindex{tracer}

As we might see, it is not very convenient to trace manually in the debugger.
That's a reason we will try \tracer.

We open compiled example in \IDA, find the address of the instruction \INS{PUSH ESI}
(passing the sole argument to \ttf), which is \TT{0x401026} for this case and we run the \tracer:

\begin{lstlisting}
tracer.exe -l:loops_2.exe bpx=loops_2.exe!0x00401026
\end{lstlisting}

\TT{BPX} just sets a breakpoint at the address and \tracer will then print the state of the registers.

In the \TT{tracer.log}, this is what we see:

\lstinputlisting{patterns/09_loops/simple/tracer.log}

We see how the value of \ESI register changes from 2 to 9.

Even more than that, the \tracer can collect register values for all addresses within the function.
This is called \IT{trace} there.
Every instruction gets traced, all interesting register values are recorded.

Then, an \IDA .idc-script is generated, that adds comments.
So, in the \IDA we've learned that the \main function address is \TT{0x00401020} and we run:

\begin{lstlisting}
tracer.exe -l:loops_2.exe bpf=loops_2.exe!0x00401020,trace:cc
\end{lstlisting}

\TT{BPF} stands for set breakpoint on function.

As a result, we get the \TT{loops\_2.exe.idc} and \TT{loops\_2.exe\_clear.idc} scripts.

\clearpage
We load \TT{loops\_2.exe.idc} into \IDA and see:

\begin{figure}[H]
\centering
\myincludegraphics{patterns/09_loops/simple/IDA_tracer_cc.png}
\caption{\IDA with .idc-script loaded}
\label{fig:loops_IDA_tracer}
\end{figure}

We see that \ESI can be from 2 to 9 at the start of the loop body,
but from 3 to 0xA (10) after the increment.
We can also see that \main is finishing with 0 in \EAX.

\tracer also generates \TT{loops\_2.exe.txt}, 
that contains information about how many times each instruction has been executed and
register values:

\lstinputlisting[caption=loops\_2.exe.txt]{patterns/09_loops/simple/loops_2.exe.txt}
\myindex{\GrepUsage}
We can use grep here.


\subsubsection{x64}
\label{subsec:popcnt}

Let's modify the example slightly to extend it to 64-bit:

\lstinputlisting[label=popcnt_x64_example,style=customc]{patterns/14_bitfields/4_popcnt/shifts64.c}

\myparagraph{\NonOptimizing GCC 4.8.2}

So far so easy.

\lstinputlisting[caption=\NonOptimizing GCC 4.8.2,style=customasmx86]{patterns/14_bitfields/4_popcnt/shifts64_GCC_O0_EN.s}

\myparagraph{\Optimizing GCC 4.8.2}

\lstinputlisting[caption=\Optimizing GCC 4.8.2,numbers=left,label=shifts64_GCC_O3,style=customasmx86]{patterns/14_bitfields/4_popcnt/shifts64_GCC_O3_EN.s}

This code is terser, but has a quirk.

In all examples that we see so far, we were incrementing the \q{rt} value after comparing a specific bit,
but the code here increments \q{rt} before (line 6), writing the new value into register \EDX .
Thus, if the last bit is 1, the \CMOVNE\footnote{Conditional MOVe if Not Equal} instruction
(which is a synonym for \CMOVNZ\footnote{Conditional MOVe if Not Zero}) \emph{commits} 
the new value of \q{rt}
by moving \EDX (\q{proposed rt value}) into \EAX (\q{current rt} to be returned at the end).

Hence, the incrementing is performed at each step of loop, i.e., 64 times, without any relation to the input value.

The advantage of this code is that it contain only one conditional jump (at the end of the loop) instead of 
two jumps (skipping the \q{rt} value increment and at the end of loop).
And that might work faster on the modern CPUs with branch predictors: \myref{branch_predictors}.

\label{FATRET}
\myindex{x86!\Instructions!FATRET}
The last instruction is \INS{REP RET} (opcode \TT{F3 C3}) 
which is also called \INS{FATRET} by MSVC.
This is somewhat optimized version of \RET, 
which is recommended by AMD to be placed at the end of function, if \RET goes right after conditional jump: 
\InSqBrackets{\AMDOptimization p.15}
\footnote{More information on it: \url{http://go.yurichev.com/17328}}.

\myparagraph{\Optimizing MSVC 2010}

\lstinputlisting[caption=\Optimizing MSVC 2010,style=customasmx86]{patterns/14_bitfields/4_popcnt/MSVC_2010_x64_Ox_EN.asm}

\myindex{x86!\Instructions!ROL}
Here the \ROL instruction is used instead of 
\SHL, which is in fact \q{rotate left} 
instead of \q{shift left},
but in this example it works just as \TT{SHL}.

You can read more about the rotate instruction here: \myref{ROL_ROR}.

\Reg{8} here is counting from 64 to 0.
It's just like an inverted $i$.

Here is a table of some registers during the execution:

\begin{center}
\begin{tabular}{ | l | l | }
\hline
\HeaderColor RDX & \HeaderColor R8 \\
\hline
0x0000000000000001 & 64 \\
\hline
0x0000000000000002 & 63 \\
\hline
0x0000000000000004 & 62 \\
\hline
0x0000000000000008 & 61 \\
\hline
... & ... \\
\hline
0x4000000000000000 & 2 \\
\hline
0x8000000000000000 & 1 \\
\hline
\end{tabular}
\end{center}

\myindex{x86!\Instructions!FATRET}
At the end we see the \INS{FATRET} instruction, which was explained here: \myref{FATRET}.

\myparagraph{\Optimizing MSVC 2012}

\lstinputlisting[caption=\Optimizing MSVC 2012,style=customasmx86]{patterns/14_bitfields/4_popcnt/MSVC_2012_x64_Ox_EN.asm}

\myindex{\CompilerAnomaly}
\label{MSVC2012_anomaly}
\Optimizing MSVC 2012 does almost the same job as 
optimizing MSVC 2010, but somehow, it generates two identical loop bodies and the loop count is now 32 instead of 64.

To be honest, it's not possible to say why. Some optimization trick? Maybe it's better for the loop body to be slightly 
longer?

Anyway, such code is relevant here to show that sometimes the compiler output may be really weird and 
illogical, but perfectly working.


\subsubsection{ARM + \OptimizingXcodeIV (\ARMMode)}

\lstinputlisting[caption=\OptimizingXcodeIV (\ARMMode),label=ARM_leaf_example4,style=customasmARM]{patterns/14_bitfields/4_popcnt/ARM_Xcode_O3_EN.lst}

\myindex{ARM!\Instructions!TST}
\TST is the same thing as \TEST in x86.

\myindex{ARM!Optional operators!LSL}
\myindex{ARM!Optional operators!LSR}
\myindex{ARM!Optional operators!ASR}
\myindex{ARM!Optional operators!ROR}
\myindex{ARM!Optional operators!RRX}
\myindex{ARM!\Instructions!MOV}
\myindex{ARM!\Instructions!TST}
\myindex{ARM!\Instructions!CMP}
\myindex{ARM!\Instructions!ADD}
\myindex{ARM!\Instructions!SUB}
\myindex{ARM!\Instructions!RSB}
As was noted before~(\myref{shifts_in_ARM_mode}),
there are no separate shifting instructions in ARM mode.
However, there are modifiers 
LSL (\emph{Logical Shift Left}), 
LSR (\emph{Logical Shift Right}), 
ASR (\emph{Arithmetic Shift Right}), 
ROR (\emph{Rotate Right}) and
RRX (\emph{Rotate Right with Extend}), which may be added to such instructions as \MOV, \TST,
\CMP, \ADD, \SUB, \RSB\footnote{\DataProcessingInstructionsFootNote}.

These modificators define how to shift the second operand and by how many bits.

\myindex{ARM!\Instructions!TST}
\myindex{ARM!Optional operators!LSL}
Thus the \TT{\q{TST R1, R2,LSL R3}} instruction works here as $R1 \land (R2 \ll R3)$.

\subsubsection{ARM + \OptimizingXcodeIV (\ThumbTwoMode)}

\myindex{ARM!\Instructions!LSL.W}
\myindex{ARM!\Instructions!LSL}
Almost the same, but here are two \INS{LSL.W}/\TST instructions are used instead of a single \TST, because in Thumb mode it is not
possible to define \LSL modifier directly in \TST.

\begin{lstlisting}[label=ARM_leaf_example5,style=customasmARM]
                MOV             R1, R0
                MOVS            R0, #0
                MOV.W           R9, #1
                MOVS            R3, #0
loc_2F7A
                LSL.W           R2, R9, R3
                TST             R2, R1
                ADD.W           R3, R3, #1
                IT NE
                ADDNE           R0, #1
                CMP             R3, #32
                BNE             loc_2F7A
                BX              LR
\end{lstlisting}

\subsubsection{ARM64 + \Optimizing GCC 4.9}

Let's take the 64-bit example which has been already used: \myref{popcnt_x64_example}.

\lstinputlisting[caption=\Optimizing GCC (Linaro) 4.8,style=customasmARM]{patterns/14_bitfields/4_popcnt/ARM64_GCC_O3_EN.s}

The result is very similar to what GCC generates for x64: \myref{shifts64_GCC_O3}.

\myindex{ARM!\Instructions!CSEL}
The \CSEL instruction is \q{Conditional SELect}. 
It just chooses one variable of two depending on the flags set by \TST and copies the value into \RegW{2}, which holds the \q{rt} variable.

\subsubsection{ARM64 + \NonOptimizing GCC 4.9}

And again, we'll work on the 64-bit example which was already used: \myref{popcnt_x64_example}.
The code is more verbose, as usual.

\lstinputlisting[caption=\NonOptimizing GCC (Linaro) 4.8,style=customasmARM]{patterns/14_bitfields/4_popcnt/ARM64_GCC_O0_EN.s}


\subsubsection{MIPS}

\lstinputlisting[caption=\Optimizing GCC 4.4.5 (IDA),style=customasmMIPS]{patterns/10_strings/1_strlen/MIPS_O3_IDA_EN.lst}

\myindex{MIPS!\Instructions!NOR}
\myindex{MIPS!\Pseudoinstructions!NOT}

MIPS lacks a \NOT instruction, but has \NOR which is \TT{OR~+~NOT} operation.

This operation is widely used in digital electronics\footnote{NOR is called \q{universal gate}}.
\myindex{Apollo Guidance Computer}
For example, the Apollo Guidance Computer used in the Apollo program, 
was built by only using 5600 NOR gates:
[Jens Eickhoff, \emph{Onboard Computers, Onboard Software and Satellite Operations: An Introduction}, (2011)].
But NOR element isn't very popular in computer programming.

So, the NOT operation is implemented here as \TT{NOR~DST,~\$ZERO,~SRC}.

From fundamentals \myref{sec:signednumbers} we know that bitwise inverting a signed number is the same 
as changing its sign and subtracting 1 from the result.

So what \NOT does here is to take the value of $str$ and transform it into $-str-1$.
The addition operation that follows prepares result.


