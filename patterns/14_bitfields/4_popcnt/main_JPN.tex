\subsection{Counting bits set to 1}

入力値のビットの数を計算する関数の単純な例です。

この操作は\q{集団カウント}とも呼ばれます。\footnote{(SSE4をサポートする)モダンx86 CPUはこのためにPOPCNT命令を持っています}

\lstinputlisting[style=customc]{patterns/14_bitfields/4_popcnt/shifts.c}

このループでは、ループカウント値 $i$ は0から31を数えます。
$1 \ll i$ 文は1から\TT{0x80000000}まで数えます。
自然言語でこの操作を説明すると、\emph{1をnビット左シフトする}といえます。
言い換えると、$1 \ll i$ 文は結果として32ビット数のすべての可能なビット位置を生成します。
右側の解放されたビットは常にクリアされます。

\label{2n_numbers_table}
$i=0 \ldots 31$ で取りうるすべての値の表です。

%\small
\begin{center}
\begin{tabular}{ | l | l | l | l | }
\hline
\HeaderColor \CCpp 表現 & 
\HeaderColor 2のべき乗 & 
\HeaderColor 10進数形式 & 
\HeaderColor 16進数形式 \\
\hline
$1 \ll 0$ & $2^{0}$ & 1 & 1 \\
\hline
$1 \ll 1$ & $2^{1}$ & 2 & 2 \\
\hline
$1 \ll 2$ & $2^{2}$ & 4 & 4 \\
\hline
$1 \ll 3$ & $2^{3}$ & 8 & 8 \\
\hline
$1 \ll 4$ & $2^{4}$ & 16 & 0x10 \\
\hline
$1 \ll 5$ & $2^{5}$ & 32 & 0x20 \\
\hline
$1 \ll 6$ & $2^{6}$ & 64 & 0x40 \\
\hline
$1 \ll 7$ & $2^{7}$ & 128 & 0x80 \\
\hline
$1 \ll 8$ & $2^{8}$ & 256 & 0x100 \\
\hline
$1 \ll 9$ & $2^{9}$ & 512 & 0x200 \\
\hline
$1 \ll 10$ & $2^{10}$ & 1024 & 0x400 \\
\hline
$1 \ll 11$ & $2^{11}$ & 2048 & 0x800 \\
\hline
$1 \ll 12$ & $2^{12}$ & 4096 & 0x1000 \\
\hline
$1 \ll 13$ & $2^{13}$ & 8192 & 0x2000 \\
\hline
$1 \ll 14$ & $2^{14}$ & 16384 & 0x4000 \\
\hline
$1 \ll 15$ & $2^{15}$ & 32768 & 0x8000 \\
\hline
$1 \ll 16$ & $2^{16}$ & 65536 & 0x10000 \\
\hline
$1 \ll 17$ & $2^{17}$ & 131072 & 0x20000 \\
\hline
$1 \ll 18$ & $2^{18}$ & 262144 & 0x40000 \\
\hline
$1 \ll 19$ & $2^{19}$ & 524288 & 0x80000 \\
\hline
$1 \ll 20$ & $2^{20}$ & 1048576 & 0x100000 \\
\hline
$1 \ll 21$ & $2^{21}$ & 2097152 & 0x200000 \\
\hline
$1 \ll 22$ & $2^{22}$ & 4194304 & 0x400000 \\
\hline
$1 \ll 23$ & $2^{23}$ & 8388608 & 0x800000 \\
\hline
$1 \ll 24$ & $2^{24}$ & 16777216 & 0x1000000 \\
\hline
$1 \ll 25$ & $2^{25}$ & 33554432 & 0x2000000 \\
\hline
$1 \ll 26$ & $2^{26}$ & 67108864 & 0x4000000 \\
\hline
$1 \ll 27$ & $2^{27}$ & 134217728 & 0x8000000 \\
\hline
$1 \ll 28$ & $2^{28}$ & 268435456 & 0x10000000 \\
\hline
$1 \ll 29$ & $2^{29}$ & 536870912 & 0x20000000 \\
\hline
$1 \ll 30$ & $2^{30}$ & 1073741824 & 0x40000000 \\
\hline
$1 \ll 31$ & $2^{31}$ & 2147483648 & 0x80000000 \\
\hline
\end{tabular}
\end{center}
%\normalsize

このような定数(ビットマスク)はコード上、非常によく現れます。現役のリバースエンジニアは
これらを素早く見つけなければなりません。

65536以下の10進数と16進数は簡単に記憶できます。
65536を超える10進数はおそらく記憶する価値はないでしょう。

これらの定数は、フラグを特定のビットにマッピングするために非常によく使用されます。
たとえば、Apache 2.4.6のソースコードから\TT{ssl\_private.h}
を抜粋した例を次に示します。

\begin{lstlisting}[style=customc]
/**
 * Define the SSL options
 */
#define SSL_OPT_NONE           (0)
#define SSL_OPT_RELSET         (1<<0)
#define SSL_OPT_STDENVVARS     (1<<1)
#define SSL_OPT_EXPORTCERTDATA (1<<3)
#define SSL_OPT_FAKEBASICAUTH  (1<<4)
#define SSL_OPT_STRICTREQUIRE  (1<<5)
#define SSL_OPT_OPTRENEGOTIATE (1<<6)
#define SSL_OPT_LEGACYDNFORMAT (1<<7)
\end{lstlisting}

私たちの例に戻りましょう。

\TT{IS\_SET}マクロはビットの数を $a$ でチェックします。
\myindex{x86!\Instructions!AND}

\TT{IS\_SET}マクロは実際、論理AND演算(\emph{AND})で、
特定のビットがそこになければ0を返すか、
ビットが存在すれば、ビットをマスクします。
\CCpp の\emph{if()}演算子は、その式がゼロでない場合に実行しますが、
123456であっても正しく動作します。

% subsections
\subsection{x86}

\subsubsection{MSVC}

コンパイルして得られるものを次に示します(MSVC 2010 Express)。

\lstinputlisting[label=src:passing_arguments_ex_MSVC_cdecl,caption=MSVC 2010 Express,style=customasmx86]{patterns/05_passing_arguments/msvc_JPN.asm}

\myindex{x86!\Registers!EBP}

\main 関数は3つの数値をスタックにプッシュし、\TT{f(int,int,int)}を呼び出すことがわかります。

\ttf 内の引数アクセスは、ローカル変数と同じ方法で
\TT{\_a\$ = 8}
のようなマクロの助けを借りて構成されますが、正のオフセット(\emph{プラス}で扱われます)を持ちます。 
したがって、\TT{\_a\$}マクロを \EBP レジスタの値に追加することによって\gls{stack frame}の\emph{外側}を処理しています。

\myindex{x86!\Instructions!IMUL}
\myindex{x86!\Instructions!ADD}

次に、 $a$ の値が \EAX に格納されます。 \IMUL 命令実行後、 
\EAX の値は \EAX の値と\TT{\_b}の内容の\gls{product}です。

その後、 \ADD は\TT{\_c}の値を \EAX に追加します。

\EAX の値は移動する必要はありません。すでに存在している必要があります。 
\gls{caller}に戻ると、 \EAX 値をとり、 \printf の引数として使用します。

\clearpage
\myparagraph{\Optimizing MSVC + \olly}
\myindex{\olly}

\olly でこの(最適化された)例を試すことができます。ここに最初のイテレーションがあります:

\begin{figure}[H]
\centering
\myincludegraphics{patterns/10_strings/1_strlen/olly1.png}
\caption{\olly: 最初のイテレーションの開始}
\label{fig:strlen_olly_1}
\end{figure}

\olly がループを見つけ、便宜上、その指示を括弧で\emph{囲んだ}ことがわかります。 
\EAX の右ボタンをクリックして、\q{Follow in Dump}を選択すると、メモリウィンドウが正しい場所にスクロールします。
ここではメモリ内に \q{hello!}という文字列があります。
その後に少なくとも1つのゼロバイトがあり、次にランダムなごみがあります。

\olly が有効なアドレスを持つレジスタを見ると、それは文字列を指しており、
文字列として表示されます。

\clearpage
F8(\stepover)を数回押して、ループの本体の先頭に移動しましょう:

\begin{figure}[H]
\centering
\myincludegraphics{patterns/10_strings/1_strlen/olly2.png}
\caption{\olly: 2回目のイテレーションの開始}
\label{fig:strlen_olly_2}
\end{figure}

\EAX には文字列中の2番目の文字のアドレスが含まれていることがわかります。

\clearpage

我々はループから脱出するためにF8を十分な回数押す必要があります:

\begin{figure}[H]
\centering
\myincludegraphics{patterns/10_strings/1_strlen/olly3.png}
\caption{\olly: 計算すべきポインタの差}
\label{fig:strlen_olly_3}
\end{figure}

% FIXME:
\EAX には文字列の直後に0バイトのアドレスが含まれることがわかりました。一方、
\EDX は変更されていないので、まだ文字列の先頭を指しています。

これらの2つのアドレスの差がここで計算されています。

\clearpage
\SUB 命令が実行されました。

\begin{figure}[H]
\centering
\myincludegraphics{patterns/10_strings/1_strlen/olly4.png}
\caption{\olly: \EAX がデクリメントされる}
\label{fig:strlen_olly_4}
\end{figure}

ポインタの違いは \EAX レジスタにあります(値は7)。
確かに、 \q{hello!}文字列の長さは6ですが、7にはゼロバイトが含まれています。
しかし、\TT{strlen()}は文字列中のゼロ以外の文字数を返さなければなりません。
したがって、デクリメントが実行され、関数が戻ります。


\subsubsection{GCC}

GCC 4.4.1で同じものをコンパイルし、 \IDA の結果を見てみましょう。

\lstinputlisting[caption=GCC 4.4.1,style=customasmx86]{patterns/05_passing_arguments/gcc_JPN.asm}

結果はほぼ同じで、以前に説明したいくつかの小さな違いがあります。

\gls{stack pointer}は2つの関数呼び出し(fとprintf)の後にセットバックされません。
最後から2番目の\TT{LEAVE}命令(\myref{x86_ins:LEAVE})
命令が最後にこれを処理するためです。

\subsubsection{x64}
\label{subsec:popcnt}

例を64ビットに拡張するように少し変更してみましょう。

\lstinputlisting[label=popcnt_x64_example,style=customc]{patterns/14_bitfields/4_popcnt/shifts64.c}

\myparagraph{\NonOptimizing GCC 4.8.2}

ここまでは簡単です。

\lstinputlisting[caption=\NonOptimizing GCC 4.8.2,style=customasmx86]{patterns/14_bitfields/4_popcnt/shifts64_GCC_O0_JPN.s}

\myparagraph{\Optimizing GCC 4.8.2}

\lstinputlisting[caption=\Optimizing GCC 4.8.2,numbers=left,label=shifts64_GCC_O3,style=customasmx86]{patterns/14_bitfields/4_popcnt/shifts64_GCC_O3_JPN.s}

コードは簡潔ですが、曖昧なところがあります。

これまでのすべての例では、特定のビットを比較した後に \q{rt}値をインクリメントしていましたが、
ここでコードは \q{rt}を先に増やして(6行目)、新しい値をレジスタ \EDX に書き込みます。
したがって、最後のビットが1である場合、 \CMOVNE\footnote{Conditional MOVe if Not Equal} 命令
(\CMOVNZ\footnote{Conditional MOVe if Not Zero}と同義)は、 \EDX (\q{提案されたrt値})
を \EAX (最後にリターンされる\q{現在のrt})に戻すことによって新しい値\q{rt}を\emph{コミット}します。

したがって、ループの各ステップで、言い換えると64回入力値に関係なく、インクリメントが実行されます。

このコードの利点は、2つのジャンプ( ループの最後で\q{rt}値のインクリメントをスキップする)ではなく、
条件ジャンプを1つだけ(ループの最後に)含むことです。
そして、それは分岐予測を持つ現代のCPUでより速く動作するでしょう:\myref{branch_predictors}

\label{FATRET}
\myindex{x86!\Instructions!FATRET}
最後の命令は、MSVCによって\INS{FATRET}とも呼ばれる
\INS{REP RET} (オペコード \TT{F3 C3})です。
これは、 \RET のいくらか最適化されたバージョンであり、
\RET が条件ジャンプの直後にある場合、AMDは関数の最後に置くことを推奨しています:
\InSqBrackets{\AMDOptimization p.15}
\footnote{詳細はこちら: \url{http://go.yurichev.com/17328}}.

\myparagraph{\Optimizing MSVC 2010}

\lstinputlisting[caption=\Optimizing MSVC 2010,style=customasmx86]{patterns/14_bitfields/4_popcnt/MSVC_2010_x64_Ox_JPN.asm}

\myindex{x86!\Instructions!ROL}
ここでは、 \SHL の代わりに 
\ROL 命令が使用されています。
実際には、 \q{shift left}ではなく 
\q{rotate left}ですが、
この例では \SHL と同じように動作します。

ローテート命令の詳細については、こちらをご覧ください:\myref{ROL_ROR}

ここの\Reg{8}は64から0まで数えています。
$i$を逆にしたようなものです。

実行中のレジスタのテーブルを以下に示します。

\begin{center}
\begin{tabular}{ | l | l | }
\hline
\HeaderColor RDX & \HeaderColor R8 \\
\hline
0x0000000000000001 & 64 \\
\hline
0x0000000000000002 & 63 \\
\hline
0x0000000000000004 & 62 \\
\hline
0x0000000000000008 & 61 \\
\hline
... & ... \\
\hline
0x4000000000000000 & 2 \\
\hline
0x8000000000000000 & 1 \\
\hline
\end{tabular}
\end{center}

\myindex{x86!\Instructions!FATRET}
最後に、\INS{FATRET}命令がありますが、それは\myref{FATRET}で説明します。

\myparagraph{\Optimizing MSVC 2012}

\lstinputlisting[caption=\Optimizing MSVC 2012,style=customasmx86]{patterns/14_bitfields/4_popcnt/MSVC_2012_x64_Ox_JPN.asm}

\myindex{\CompilerAnomaly}
\label{MSVC2012_anomaly}
\Optimizing MSVC 2012 は最適化されたMSVC 2010とほとんど同じことをしますが、
どういうわけか、2つの同じループボディを生成して、ループカウントが64ではなく32です。

正直なところ、なぜかはわかりません。何か最適化のトリックでしょうか。ループボディを
もう少し長くしたほうがよいのかもしれません。

とにかく、コンパイラの出力が本当に奇妙で非論理的なことがありますが、
このようなコードは完全に動作することを示すためにあげました。

\subsubsection{ARM + \OptimizingXcodeIV (\ARMMode)}

\lstinputlisting[caption=\OptimizingXcodeIV (\ARMMode),label=ARM_leaf_example4,style=customasmARM]{patterns/14_bitfields/4_popcnt/ARM_Xcode_O3_JPN.lst}

\myindex{ARM!\Instructions!TST}
\TST はx86では \TEST と同じです。

\myindex{ARM!Optional operators!LSL}
\myindex{ARM!Optional operators!LSR}
\myindex{ARM!Optional operators!ASR}
\myindex{ARM!Optional operators!ROR}
\myindex{ARM!Optional operators!RRX}
\myindex{ARM!\Instructions!MOV}
\myindex{ARM!\Instructions!TST}
\myindex{ARM!\Instructions!CMP}
\myindex{ARM!\Instructions!ADD}
\myindex{ARM!\Instructions!SUB}
\myindex{ARM!\Instructions!RSB}
前述のように(\myref{shifts_in_ARM_mode})、
ARMモードでは個別のシフト命令はありません。
ただし、 \MOV 、 \TST 、 \CMP 、 \ADD 、 \SUB 、 \RSB などの命令には、
LSL(\emph{Logical Shift Left})、
LSR(\emph{Logical Shift Right})、
ASR(\emph{Arithmetic Shift Right})、
ROR(\emph{Rotate Right})、
RRX(\emph{Rotate Right with Extend}) 
があります。

これらの変更子は、第2オペランドのシフト方法とビット数を定義します。

\myindex{ARM!\Instructions!TST}
\myindex{ARM!Optional operators!LSL}
したがって、 \TT{\q{TST R1, R2,LSL R3}}命令はここでは $R1 \land (R2 \ll R3)$ として機能します。

\subsubsection{ARM + \OptimizingXcodeIV (\ThumbTwoMode)}

\myindex{ARM!\Instructions!LSL.W}
\myindex{ARM!\Instructions!LSL}
ほぼ同じですが、Thumbモードでは \LSL 修飾子を直接 \TST に定義することはできないため、
1つの \TST の代わりに2つの \INS{LSL.W}/\TST 命令が使用されます。

\begin{lstlisting}[label=ARM_leaf_example5,style=customasmARM]
                MOV             R1, R0
                MOVS            R0, #0
                MOV.W           R9, #1
                MOVS            R3, #0
loc_2F7A
                LSL.W           R2, R9, R3
                TST             R2, R1
                ADD.W           R3, R3, #1
                IT NE
                ADDNE           R0, #1
                CMP             R3, #32
                BNE             loc_2F7A
                BX              LR
\end{lstlisting}

\subsubsection{ARM64 + \Optimizing GCC 4.9}

すでに使用した64ビットの例を見てみましょう:\myref{popcnt_x64_example}

\lstinputlisting[caption=\Optimizing GCC (Linaro) 4.8,style=customasmARM]{patterns/14_bitfields/4_popcnt/ARM64_GCC_O3_JPN.s}

結果は、GCCがx64に対して生成するものと非常によく似ています:\myref{shifts64_GCC_O3}

\myindex{ARM!\Instructions!CSEL}
\CSEL 命令は\q{Conditional SELect}です。 
\TST で設定されたフラグに応じて1つの変数が2つだけ選択され、値が \q{rt}変数を保持する\RegW{2}にコピーされます。

\subsubsection{ARM64 + \NonOptimizing GCC 4.9}

もう一度、すでに使用した64ビットの例について作業します:\myref{popcnt_x64_example}
例によって、コードはより冗長です。

\lstinputlisting[caption=\NonOptimizing GCC (Linaro) 4.8,style=customasmARM]{patterns/14_bitfields/4_popcnt/ARM64_GCC_O0_JPN.s}


\subsubsection{MIPS}

\lstinputlisting[caption=\Optimizing GCC 4.4.5 (IDA),style=customasmMIPS]{patterns/08_switch/1_few/MIPS_O3_IDA_JPN.lst}

\myindex{MIPS!\Instructions!JR}

関数は常に \puts を呼び出すことで終了するので、\puts(\INS{JR}:\q{Jump Register})へのジャンプは\q{jump and link}ではなく、ここにあります。
私たちは以前これについて話しました:\myref{JMP_instead_of_RET}

\myindex{MIPS!Load delay slot}
\INS{LW}命令の後に\INS{NOP}命令も表示されることがよくあります。
これは\q{load delay slot}:MIPSの別の\emph{delay slot}です。
\myindex{MIPS!\Instructions!LW}

\INS{LW}がメモリから値をロードする間に、\INS{LW}の次の命令が実行されることがあります。

ただし、次の命令は\INS{LW}の結果を使用してはなりません。

現代のMIPS CPUは、次の命令が\INS{LW}の結果を使用するのを待つ機能を持っているので、これは幾分時代遅れですが、
GCCは古いMIPS CPU用にNOPを追加します。
一般に、無視することができます。

