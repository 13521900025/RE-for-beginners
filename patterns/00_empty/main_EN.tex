\mysection{An Empty Function}
\label{empty_func}

The simplest possible function is arguably one that does nothing:

\lstinputlisting[caption=\EN{\CCpp Code},style=customc]{patterns/00_empty/1.c}

Let's compile it!

\subsection{x86}

Here's what both the GCC and MSVC compilers produce on the x86 platform:

\lstinputlisting[caption=\Optimizing GCC/MSVC (\assemblyOutput),style=customasmx86]{patterns/00_empty/1.s}

\myindex{x86!\Instructions!RET}
There is just one instruction: \RET, which returns execution to the \gls{caller}.

\subsection{ARM}

\lstinputlisting[caption=\OptimizingKeilVI (\ARMMode) \assemblyOutput,style=customasmARM]{patterns/00_empty/1_Keil_ARM_O3.s}

The return address is not saved on the local stack in the ARM \ac{ISA}, but rather in the link register, 
so the \INS{BX LR} instruction causes execution to jump to that address---effectively returning execution
to the \gls{caller}.

\subsection{MIPS}

There are two naming conventions used in the world of MIPS when naming registers:
by number (from \$0 to \$31) or by pseudo name (\$V0, \$A0, etc.).

The GCC assembly output below lists registers by number:

\lstinputlisting[caption=\Optimizing GCC 4.4.5 (\assemblyOutput),style=customasmMIPS]{patterns/00_empty/MIPS.s}

\dots while \IDA does it by pseudo name:

\lstinputlisting[caption=\Optimizing GCC 4.4.5 (IDA),style=customasmMIPS]{patterns/00_empty/MIPS_IDA.lst}

\myindex{MIPS!\Instructions!J}
The first instruction is the jump instruction (J or JR) which returns the execution flow to the \gls{caller},
jumping to the address in the \$31 (or \$RA) register.

This is the register analogous to \ac{LR} in ARM.

The second instruction is \ac{NOP}, which does nothing.
We can ignore it for now.

\subsubsection{A Note About MIPS Instructions and Register Names}

Register and instruction names in the world of MIPS are traditionally written in lowercase.
However, for the sake of consistency, this book will stick to using uppercase letters,
as it is the convention followed by all the other \ac{ISA}s featured in this book.

\subsection{Empty Functions in Practice}

Despite the fact empty functions seem useless, they are quite frequent in low-level code.

First of all, they are quite popular in debugging functions, like this one:

\lstinputlisting[caption=\CCpp code,style=customc]{patterns/00_empty/dbg_print_EN.c}

In a non-debug build (as in a ``release''), \TT{\_DEBUG} is not defined,
so the \TT{dbg\_print()} function, despite still being called during execution,
will be empty.

Similarly, a popular method of software protection is to make one build for legal customers, and another demo build.
A demo build can lack some important functions, as with this example:

\lstinputlisting[caption=\CCpp code,style=customc]{patterns/00_empty/demo_EN.c}

The \TT{save\_file()} function can be called when the user clicks \TT{File->Save} on the menu.
The demo version may be delivered with this menu item disabled, but even if a software cracker will enable it,
only an empty function with no useful code will be called.

IDA marks such functions with names like \TT{nullsub\_00}, \TT{nullsub\_01}, etc.

