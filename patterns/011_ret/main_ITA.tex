\mysection{Ritornare valori}
\label{ret_val_func}

Un'altra funzione semplice è quella che ritorna un valore costante:

\lstinputlisting[caption=\EN{\CCpp Code},style=customc]{patterns/011_ret/1.c}

Compiliamola.

\subsection{x86}

Questo è quello che i compilatori GCC e MSVC producono (con ottimizzazione) per x86:

\lstinputlisting[caption=\Optimizing GCC/MSVC (\assemblyOutput),style=customasmx86]{patterns/011_ret/1.s}

\myindex{x86!\Instructions!RET}
Ci sono solo due istruzioni: la prima inserisce il valore 123 nel registro \EAX,
che per convenzione viene utilizzato per memorizzare i valori di ritorno,
e la seconda è \RET, che ritorna l'esecuzione al \gls{caller}.

La funzione chiamante troverà quindi il valore di ritorno nel registro \EAX.

\subsection{ARM}

Ci sono alcune differenze nella piattaforma ARM:

\lstinputlisting[caption=\OptimizingKeilVI (\ARMMode) ASM Output,style=customasmARM]{patterns/011_ret/1_Keil_ARM_O3.s}

ARM utilizza il registro \Reg{0} per ritornare i risultati delle funzioni, quindi 123 viene copiato in \Reg{0}.

\myindex{ARM!\Instructions!MOV}
\myindex{x86!\Instructions!MOV}
Occorre notare che \MOV è un nome di funzione fuorviante sia nella \ac{ISA} x86 che ARM.

I dati non vengono infatti \emph{spostati}, ma \emph{copiati}.

\subsection{MIPS}

\label{MIPS_leaf_function_ex1}

L'output in assembly di GCC assembly qua sotto chiama i registri per numero:

\lstinputlisting[caption=\Optimizing GCC 4.4.5 (\assemblyOutput),style=customasmMIPS]{patterns/011_ret/MIPS.s}

\dots mentre \IDA utilizza gli pseudonimi:

\lstinputlisting[caption=\Optimizing GCC 4.4.5 (IDA),style=customasmMIPS]{patterns/011_ret/MIPS_IDA.lst}

Il registro \$2 (o \$V0) viene utilizzato per memorizzare il valore di ritorno della funzione.
\myindex{MIPS!\Pseudoinstructions!LI}
\INS{LI} sta per ``Load Immediate'' ed è l'equivalente MIPS di \MOV.

\myindex{MIPS!\Instructions!J}
L'altra istruzione è il salto (J or JR) che ritorna il flusso di esecuzione al \gls{caller}.

\myindex{MIPS!Branch delay slot}
Potresti domandarti perchè le posizioni delle istruzioni Load Immediate (LI) ed il jump (J or JR) siano invertite. Questo è dovuto ad una funzionalità di \ac{RISC} chiamata``branch delay slot''.

Il motivo per cui accade è dovuto ad un problema nell'architettura di alcune \ac{ISA} RISC e non è importante per i nostri
scopi---dobbiamo semplicemente tenere a mente che in MIPS, l'istruzione che segue un jump o un'istruzione condizionale
viene eseguita \emph{prima} del salto/ramificazione stessi.

Come conseguenza, le istruzioni di ramificazione vengono sempre scambiate di posto con l'istruzione immediatamente precedente.

In pratica, le funzioni che ritornano semplicemente 1 (\emph{true}) o 0 (\emph{false}) sono molto frequenti.

Le più piccole utility UNIX in assoluto, \emph{/bin/true} e \emph{/bin/false} ritornano 0 ed 1 rispettivamente, come codice di uscita.
(Zero come codice di uscita generalmente indica successo, valori diversi da zero indicano errori.)
