\subsubsection{x86}

\myparagraph{\NonOptimizing MSVC}

Kompilieren wir es:

\lstinputlisting[style=customasmx86]{patterns/10_strings/1_strlen/10_1_msvc_DE.asm}

\myindex{x86!\Instructions!MOVSX}
\myindex{x86!\Instructions!TEST}

Wir finden hier zwei neue Befehle: \MOVSX und \TEST.

\label{MOVSX}

Der erste --\MOVSX--nimmt ein Byte aus einer Speicheradresse und speichert den
Wert in einem 32-bit-Register.
\MOVSX steht für \emph{MOV with Sign-Extend}.
\MOVSX setzt die übrigen Bits vom 8. bis zum 31. auf 1, falls das Quellbyte
\emph{negativ} ist oder auf 0, falls es \emph{positiv} ist.

Und hier ist der Grund dafür.

Standardmäßig ist der \Tchar Datentyp in MSVC und GCC vorzeichenbehaftet
(signed). Wenn wir zwei Werte haben, einen \Tchar und einen \Tint, (\Tint ist
ebenfalls vorzeichenbehaftet) und der erste Wert enthält -2 (kodiert als
\TT{0xFE}) und wir kopieren dieses Byte in den \Tint Container, erhalten wir
\TT{0x000000FE} und dies entspricht als signed \Tint 254, aber nicht -2. Der
signed \Tint -2 wird als \TT{0xFFFFFFFE} dargestellt. Wenn wir also \TT{0xFE}
vom Datentyp \Tchar nach \Tint übertragen wollen, müssen wir das Vorzeichen
identifizieren und den Wert entsprechend erweitern. Genau dies tut der Befehl
\MOVSX.

Weitere Informationen dazu finden sich im Abschnitt
\q{\emph{\SignedNumbersSectionName}} ~(\myref{sec:signednumbers}).

Es ist schwer zu sagen, ob der Compiler tatsächlich eine \Tchar Variable in \EDX
speichern muss, er könnte auch einen 8-Bit-Registerteil (z.B. \DL) dafür
verwenden . Offenbar arbeitet der \gls{register allocator} des Compilers auf
diese Art.

\myindex{ARM!\Instructions!TEST}

Wir finden im Weiteren den Befehl \TT{TEST EDX, EDX}. 
Für mehr Informationen zum \TEST Befehl siehe auch den Abschnitt über
Bitfelder~(\myref{sec:bitfields}).
In unserem Fall überprüft der Befehl lediglich, ob der Wert im Register \EDX
gleich 0 ist.

\myparagraph{\NonOptimizing GCC}

Schauen wir uns GCC 4.4.1 an:

\lstinputlisting[style=customasmx86]{patterns/10_strings/1_strlen/10_3_gcc.asm}

\label{movzx}
\myindex{x86!\Instructions!MOVZX}

Das Ergebnis ist fast identisch mit dem von MSVC, aber hier finden wir \MOVZX
anstelle von \MOVSX. 
\MOVZX steht für \emph{MOV with Zero-Extend}. 
Dieser Befehl kopiert einen 8-Bit- oder 16-Bit-Wert in ein 32-Bit-Register und
setzt die übrigen Bits auf 0.
Tatsächlich findet dieser Befehl vor allem deshalb Anwendung, weil er es uns
erlaubt, folgendes Befehlspaar zu ersetzen:\\
\TT{xor eax, eax / mov al, [...]}.

Andererseits ist offensichtlich, dass der Compiler folgenden Code erzeugen kann:
\\
\TT{mov al, byte ptr [eax] / test al, al}--es ist fast das gleiche, aber die
oberen Bits des \EAX Registers enthalten hier Zufallswerte bzw.
sogenanntes Zufallsrauschen.
Aber bedenken wir den Nachteil des Compilers--er kann nicht leichter
verständlichen Code erzeugen. 
Genau genommen, ist der Compiler überhaupt nicht daran gebunden, (Menschen)
verständlichen Code zu erzeugen.

\myindex{x86!\Instructions!SETcc}

Der nächste neue Befehl für uns ist \SETNZ.
In diesem Fall setzt \TT{test al,al} das \ZF flag auf 0, falls \AL nicht 0
enthät, aber \SETNZ setzt \AL auf 1, falls \TT{ZF==0} (IT{NZ} steht für
\emph{non zero}).
In natürlicher Sprache, \emph{falls \AL ungleich 0, springe zu loc\_80483F0}. 
Der Compiler erzeugt leicht redundanten Code, aber bedenken wir, dass die
Optimierung hier deaktiviert ist.

\myparagraph{\Optimizing MSVC}
\label{strlen_MSVC_Ox}

Kompilieren wir nun alles in MSVC 2012 mit aktivierter Optimierung (\Ox):

\lstinputlisting[caption=\Optimizing MSVC 2012 /Ob0,style=customasmx86]{patterns/10_strings/1_strlen/10_2_DE.asm}

Jetzt ist alles einfacher.
Unnötig zu erwähnen, dass der Compiler Register mit solcher Effizienz nur in
kleinen Funktionen mit einigen wenigen lokalen Variablen verwenden kann.

\myindex{x86!\Instructions!INC}
\myindex{x86!\Instructions!DEC}
\INC/\DEC---sind \glslink{increment}{inkrement}/\glslink{decrement}{dekrement} Befehle; mit anderen Worten:
addiere oder subtrahiere 1 zu bzw. von einer Variable. 

\clearpage
\myparagraph{\Optimizing MSVC + \olly}
\myindex{\olly}

Wir untersuchen das (optimierte) Beispiel in \olly. Hier ist der erste
Durchlauf:

\begin{figure}[H]
\centering
\myincludegraphics{patterns/10_strings/1_strlen/olly1.png}
\caption{\olly: Beginn erster Durchlauf}
\label{fig:strlen_olly_1}
\end{figure}

Wir sehen, dass \olly eine Schleife gefunden hat, und zur Verbesserung der
Lesbarkeit, diese in eckige Klammern \emph{eingeschlossen} hat.
Nach Rechtsklick auf \EAX wählen wir \q{Follow in Dump} und das Speicherfenster
scrollt an die passende Stelle. 
Hier sehen wir den String \q{hello!} im Speicher. 
Dahinter befindet sich mindestens ein Nullbyte und im Anschluss Zufallsbits.

Wenn \olly ein Register mit einer gültigen Adresse, die auf einen String zeigt,
findet, wird dieser String angezeigt.

\clearpage
Wir drücken einige Male F8 (\stepover) um zum Anfang der Schleifenkörpers zu
gelangen:

\begin{figure}[H]
\centering
\myincludegraphics{patterns/10_strings/1_strlen/olly2.png}
\caption{\olly: Beginn zweiter Durchlauf}
\label{fig:strlen_olly_2}
\end{figure}

Wir sehen, dass \EAX nun die Adresse des zweiten Zeichens des Strings enthält.

\clearpage

Durch hinreichend häufiges Drücken von F8 verlassen wir schließlich die
Schleife:

\begin{figure}[H]
\centering
\myincludegraphics{patterns/10_strings/1_strlen/olly3.png}
\caption{\olly: Pointer Differenz wird berechnet}
\label{fig:strlen_olly_3}
\end{figure}

% FIXME:
Wir sehen, dass \EAX jetzt die Adresse des Nullbytes direkt hinter dem String
enthält. In der Zwischenzeit hat sich \EDX nicht verändert, es zeigt also immer
noch auf den Anfang des Strings. 

Die Differenz zwischen den beiden Adressen wird jetzt berechnet. 

\clearpage
Der \SUB Befehl wurde gerade ausgeführt:

\begin{figure}[H]
\centering
\myincludegraphics{patterns/10_strings/1_strlen/olly4.png}
\caption{\olly: \EAX muss dekrementiert werden}
\label{fig:strlen_olly_4}
\end{figure}

Die Differenz der Pointer im \EAX Register beträgt nun--7.
Tatsächlich beträgt die Länge des \q{hello!} Strings 6 Zeichen, aber mit dem
Nullbyte am Ende dazugezählt sind es 7.
Die Funktion \TT{strlen()} soll aber die Anzahl der Nicht-Null-Zeichen im String
zurückliefert, also wird einmal dekrementiert und der Funktionsaufruf
anschließend beendet.


\myparagraph{\Optimizing GCC}

Schauen wir uns GCC 4.4.1 mit aktiverter Optimierung (\Othree key) an:

\lstinputlisting[style=customasmx86]{patterns/10_strings/1_strlen/10_3_gcc_O3.asm}
 
Hier erzeugt GCC fast identischen Code zu MSVC, außer dass hier ein \MOVZX
auftritt. 
In der Tat könnte \MOVZX hier durch \TT{mov dl, byte ptr [eax]} ersetzt werden.
 
Möglicherweise ist es einfacher für den GCC Code Generator sich daran zu
\emph{erinnern}, dass das gesamte 32-bit-\EDX Register für eine \Tchar Variable
reserviert ist und so sicherzustellen, dass die oberen Bits zu keinem Zeitpunkt
Zufallsrauschen enthalten.

\label{strlen_NOT_ADD}
\myindex{x86!\Instructions!NOT}
\myindex{x86!\Instructions!XOR}

Danach finden wir also einen neuen Befehl--\NOT. Dieser Befehl kippt alle Bits
in seinem Operanden.\\
Man kann sagen, dass es sich um ein Synonym zum Befehl \TT{XOR ECX, 0ffffffffh}
handelt. 
\NOT und das darauf folgende \ADD berechnen die Differenz im Pointer und
subtrahieren 1, nur auf eine andere Art und Weise. 
Zu Beginn wird \ECX, in dem der Pointer auf \emph{str} gespeichert ist, invertiert
und vom Ergebnis wird 1 abgezogen.

Hierzu siehe auch: \q{\SignedNumbersSectionName}~(\myref{sec:signednumbers}).
 
Mit anderen Worten, am Ende der Funktion, direkt nach dem Schleifenkörper,
werden die folgenden Befehle ausgeführt:

\begin{lstlisting}[style=customc]
ecx=str;
eax=eos;
ecx=(-ecx)-1; 
eax=eax+ecx
return eax
\end{lstlisting}

\dots~und das ist äquivalent zu:

\begin{lstlisting}[style=customc]
ecx=str;
eax=eos;
eax=eax-ecx;
eax=eax-1;
return eax
\end{lstlisting}

Warum GCC entschieden hat, dass das eine besser ist als das andere? Schwer zu
sagen.
Möglicherweise sind aber beide Variante gleichermaßen effizient.
