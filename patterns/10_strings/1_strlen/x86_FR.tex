\subsubsection{x86}

\myparagraph{MSVC \NonOptimizing}

Compilons:

\lstinputlisting[style=customasmx86]{patterns/10_strings/1_strlen/10_1_msvc_FR.asm}

\myindex{x86!\Instructions!MOVSX}
\myindex{x86!\Instructions!TEST}

Nous avons ici deux nouvelles instructions: \MOVSX et \TEST.

\label{MOVSX}

La première---\MOVSX---prend un octet depuis une adresse en mémoire et stocke la
valeur dans un registre 32-bit. 
\MOVSX signifie \emph{MOV with Sign-Extend} (déplacement avec extension de signe).
\MOVSX met le reste des bits, du 8ème au 31ème, à 1 si l'octet source est \emph{négatif}
ou à 0 si il est \emph{positif}.

Et voici pourquoi.

Par défaut, le type \Tchar est signé dans MSVC et GCC. Si nous avons deux valeurs
dont l'une d'elle est un \Tchar et l'autre un \Tint, (\Tint est signé aussi), et
si la première valeur contient -2 (codé en \TT{0xFE}) et que nous copions simplement
cet octet dans le conteneur \Tint, cela fait \TT{0x000000FE}, et ceci, pour le type
\Tint représente 254, mais pas -2. Dans un entier signé, -2 est codé en \TT{0xFFFFFFFE}.
Donc, si nous devons transférer \TT{0xFE} depuis une variable de type \Tchar vers
une de type \Tint, nous devons identifier son signe et l'étendre. C'est ce qu'effectue
\MOVSX.

Lire à ce propos dans \q{\emph{\SignedNumbersSectionName}} section~(\myref{sec:signednumbers}).

Il est difficile de dire si le compilateur doit stocker une variable \Tchar dans
\EDX, il pourrait simplement utiliser une partie 8-bit du registre (par exemple \DL).
Apparemment, l'\glslink{register allocator}{allocateur de registre} fonctionne comme
ça.

\myindex{ARM!\Instructions!TEST}

Ensuite nous voyons \TT{TEST EDX, EDX}.
Vous pouvez en lire plus à propos de l'instruction \TEST dans la section concernant
les champs de bit~(\myref{sec:bitfields}).
Ici cette instruction teste simplement si la valeur dans \EDX est égale à 0.

\myparagraph{GCC \NonOptimizing}

Essayons GCC 4.4.1:

\lstinputlisting[style=customasmx86]{patterns/10_strings/1_strlen/10_3_gcc.asm}

\label{movzx}
\myindex{x86!\Instructions!MOVZX}

Le résultat est presque le même qu'avec MSVC, mais ici nous voyons \MOVZX au lieu
de \MOVSX.
\MOVZX signifie \emph{MOV with Zero-Extend} (déplacement avec extension à 0).
Cette instruction copie une valeur 8-bit ou 16-bit dans un registre 32-bit et met
les bits restant à 0.
En fait, cette instructions n'est pratique que pour nous permettre de remplacer cette
paire d'instructions:\\
\TT{xor eax, eax / mov al, [...]}.

D'un autre côté, il est évident que le compilateur pourrait produire ce code: \\
\TT{mov al, byte ptr [eax] / test al, al}---c'est presque le même, toutefois, les
bits les plus haut du registre \EAX vont contenir des valeurs aléatoires.
Mais, admettons que c'est un inconvénient du compilateur----il ne peut pas produire
du code plus compréhensible. À strictement parler, le compilateur n'est pas du tout
obligé de générer du code compréhensible par les humains.

\myindex{x86!\Instructions!SETcc}

La nouvelle instruction suivante est \SETNZ.
Ici, si \AL ne contient pas zéro, \TT{test al, al} met le flag \ZF à 0, mais \SETNZ,
si \TT{ZF==0} (\emph{NZ} signifie \emph{not zero}, non zéro) met \AL à 1. En langage
naturel, \emph{si \AL n'est pas zéro, sauter en loc\_80483F0}. Le compilateur génère
du code redondant, mais n'oublions pas qu'il n'est pas en mode optimisation.

\myparagraph{MSVC \Optimizing}
\label{strlen_MSVC_Ox}

Maintenant, compilons tout cela avec MSVC 2012, avec le flag d'optimisation (\Ox):

\lstinputlisting[caption=MSVC 2012 \Optimizing /Ob0,style=customasmx86]{patterns/10_strings/1_strlen/10_2_FR.asm}

C'est plus simple maintenant. Inutile de préciser que le compilateur ne peut utiliser
les registres aussi efficacement que dans une petite fonction, avec peu de variables
locales.

\myindex{x86!\Instructions!INC}
\myindex{x86!\Instructions!DEC}
\INC/\DEC---sont des instructions de \glslink{increment}{incrémentation}/\glslink{decrement}{décrémentation},
en d'autres mots: ajouter ou soustraire 1 d'une/à une variable.

\clearpage
\myparagraph{MSVC + \olly}
\myindex{\olly}

2 paires de mots 32-bit sont marquées en rouge sur la pile.
Chaque paire est un double au format IEEE 754 et est passée depuis \main.

Nous voyons comment le premier \FLD charge une valeur ($1.2$) depuis la pile et la
stocke dans \ST{0}:

\begin{figure}[H]
\centering
\myincludegraphics{patterns/12_FPU/1_simple/olly1.png}
\caption{\olly: le premier \FLD a été exécuté}
\label{fig:FPU_simple_olly_1}
\end{figure}

À cause des inévitables erreurs de conversion depuis un nombre flottant 64-bit au
format IEEE 754 vers 80-bit (utilisé en interne par le FPU), ici nous voyons 1.199\ldots,
qui est proche de 1.2.

\EIP pointe maintenant sur l'instruction suivante (\FDIV), qui charge un double
(une constante) depuis la mémoire.
Par commodité, \olly affiche sa valeur: 3.14

\clearpage
Continuons l'exécution pas à pas.
\FDIV a été exécuté, maintenant \ST{0} contient 0.382\ldots
(\gls{quotient}):

\begin{figure}[H]
\centering
\myincludegraphics{patterns/12_FPU/1_simple/olly2.png}
\caption{\olly: \FDIV a été exécuté}
\label{fig:FPU_simple_olly_2}
\end{figure}

\clearpage
Troisième étape: le \FLD suivant a été exécuté, chargeant 3.4 dans \ST{0} (ici nous
voyons la valeur approximative 3.39999\ldots):

\begin{figure}[H]
\centering
\myincludegraphics{patterns/12_FPU/1_simple/olly3.png}
\caption{\olly: le second \FLD a été exécuté}
\label{fig:FPU_simple_olly_3}
\end{figure}

En même temps, le \gls{quotient} \emph{est poussé} dans \ST{1}.
Exactement maintenant, \EIP pointe sur la prochaine instruction: \FMUL.
Ceci charge la constante 4.1 depuis la mémoire, ce que montre \olly.

\clearpage
Suivante: \FMUL a été exécutée, donc maintenant le \glslink{product}{produit} est dans \ST{0}:

\begin{figure}[H]
\centering
\myincludegraphics{patterns/12_FPU/1_simple/olly4.png}
\caption{\olly: \FMUL a été exécuté}
\label{fig:FPU_simple_olly_4}
\end{figure}

\clearpage
Suivante: \FADDP a été exécutée, maintenant le résultat de l'addition est dans \ST{0},
et \ST{1} est vidé.

\begin{figure}[H]
\centering
\myincludegraphics{patterns/12_FPU/1_simple/olly5.png}
\caption{\olly: \FADDP a été exécuté}
\label{fig:FPU_simple_olly_5}
\end{figure}

Le résultat est laissé dans \ST{0}, car la fonction renvoie son résultat dans \ST{0}.

\main prend cette valeur depuis le registre plus loin.

Nous voyons quelque chose d'inhabituel: la valeur 13.93\ldots se trouve maintenant
dans \ST{7}.
Pourquoi?

\label{FPU_is_rather_circular_buffer}

Comme nous l'avons lu il y a quelque temps dans ce livre, les registres \ac{FPU} sont
une pile: \myref{FPU_is_stack}.
Mais ceci est une simplification.

Imaginez si cela était implémenté \emph{en hardware} comme cela est décrit, alors
tout le contenu des 7 registres devrait être déplacé (ou copié) dans les registres
adjacents lors d'un push ou d'un pop, et ceci nécessite beaucoup de travail.

En réalité, le \ac{FPU} a seulement 8 registres et un pointeur (appelé \GTT{TOP})
qui contient un numéro de registre, qui est le \q{haut de la pile} courant.

Lorsqu'une valeur est poussée sur la pile, \GTT{TOP} est déplacé sur le registre
disponible suivant, et une valeur est écrite dans ce registre.

La procédure est inversée si la valeur est lue, toutefois, le registre qui a été
libéré n'est pas vidé (il serait possible de le vider, mais ceci nécessite plus de
travail qui peut dégrader les performances).
Donc, c'est ce que nous voyons ici.

On peut dire que \FADDP sauve la somme sur la pile, et y supprime un élément.

Mais en fait, cette instruction sauve la somme et ensuite décale \GTT{TOP}.

Plus précisément, les registres du  \ac{FPU} sont un tampon circulaire.


\myparagraph{GCC \Optimizing}

Regardons ce que génère GCC 4.4.1 avec l'option d'optimisation \Othree:

\lstinputlisting[style=customasmx86]{patterns/10_strings/1_strlen/10_3_gcc_O3.asm}

\TT{mov dl, byte ptr [eax]}.
Ici GCC génère presque le même code que MSVC, à l'exception de la présence de \MOVZX.
Toutefois, ici, \MOVZX pourrait être remplacé par\\
\TT{mov dl, byte ptr [eax]}.

Peut-être est-il plus simple pour le générateur de code de GCC se se \emph{rappeler}
que le registre 32-bit \EDX est alloué entièrement pour une variable \Tchar et il
est sûr que les bits en partie haute ne contiennent pas de bruit indéfini.

\label{strlen_NOT_ADD}
\myindex{x86!\Instructions!NOT}
\myindex{x86!\Instructions!XOR}

Après cela, nous voyons une nouvelle instruction---\NOT. Cette instruction inverse
tout les bits de l'opérande. \\
Elle peut être vu comme un synonyme de l'instruction \TT{XOR ECX, 0ffffffffh}.
\NOT et l'instruction suivante \ADD calcule la différence entre les pointeurs et
soustrait 1, d'une façon différente.
Au début, \ECX, où le pointeur sur \emph{str} est stocké, est inversé et 1 en est soustrait.

Voir aussi: \q{\SignedNumbersSectionName}~(\myref{sec:signednumbers}).

En d'autres mots, à la fin de la fonction juste après le corps de la boucle, ces opérations
sont exécutées:

\begin{lstlisting}[style=customc]
ecx=str;
eax=eos;
ecx=(-ecx)-1;
eax=eax+ecx
return eax
\end{lstlisting}

\dots~et ceci est effectivement équivalent à:

\begin{lstlisting}[style=customc]
ecx=str;
eax=eos;
eax=eax-ecx;
eax=eax-1;
return eax
\end{lstlisting}

Pourquoi est-ce que GCC décide que cela est mieux? Difficile à deviner.
Mais peut-être que les deux variantes sont également efficaces.
