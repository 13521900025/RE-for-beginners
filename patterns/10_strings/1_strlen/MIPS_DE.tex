\subsubsection{MIPS}

\lstinputlisting[caption=\Optimizing GCC 4.4.5 (IDA),style=customasmMIPS]{patterns/10_strings/1_strlen/MIPS_O3_IDA_DE.lst}

\myindex{MIPS!\Instructions!NOR}
\myindex{MIPS!\Pseudoinstructions!NOT}

MIPS besitzt keinen \NOT Befehl, dafür aber den Befehl \NOT, welcher der
Funktion \TT{OR~+~NOT} entspricht.

Diese Funktion wird häufig in der Digitaltechnik verwendet\footnote{NOR wird
 \q{universelles Gatter} genannt}.

\index{Apollo Guidance Computer}
Der Apollo Guidance Computer, der im Apollo Programm der NASA verwendet wurde,
bestand beispielsweise ausschließlich aus 5600 \NOR Gattern:
[Jens Eickhoff, \IT{Onboard Computers, Onboard Software and Satellite
Operations: An Introduction}, (2011)].
In der Programmierung ist die Funktion \NOT nicht besonders beliebt. 

Die \NOT Funktion ist hier also durch \TT{NOR~DST,~\$ZERO,~SRC} implementiert.

Aus dem Grundlagenteil \myref{sec:signednumbers} wissen wir, dass das bitweise
invertieren einer vorzeichenbehafteten Zahl gerade einem Wechsel des Vorzeichens
mit anschließender Subtraktion von 1 entspricht. 

Was \NOT hier also tut, ist, den Wert von $str$ in $-str-1$ umzuwandeln. 
Die folgende Addition bereitet das Ergebnis vor.
