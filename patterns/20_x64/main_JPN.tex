\mysection{64ビット}

\subsection{x86-64}
\myindex{x86-64}
\label{x86-64}

これはx86アーキテクチャの64ビット拡張です。

リバースエンジニアの観点からすると、最も重要な変更は次のとおりです。

\myindex{\CLanguageElements!\Pointers}
\begin{itemize}

\item

ほとんどすべてのレジスタ(FPUとSIMDを除く)は64ビットに拡張され、Rプレフィックスが付けられました。
8つの追加レジスタが追加されました。
\ac{GPR}は\RAX, \RBX, \RCX, \RDX, 
\RBP, \RSP, \RSI, \RDI, \Reg{8}, \Reg{9}, \Reg{10}, 
\Reg{11}, \Reg{12}, \Reg{13}, \Reg{14}, \Reg{15}です。

従来通りに\IT{古い}レジスタ部分にアクセスすることはまだ可能です。
例えば、 \EAX を使用して \RAX レジスタの下位32ビット部分にアクセスすることは可能です。

\RegTableOne{RAX}{EAX}{AX}{AH}{AL}

新しい\GTT{R8-R15}レジスタの\IT{下位部分}も、\GTT{R8D-R15D}(下位32ビット部分)、
\GTT{R8W-R15W}(下位16ビット部分)、\GTT{R8L-R15L}(下位8ビット部分)です。

\RegTableFour{R8}{R8D}{R8W}{R8L}

SIMDレジスタの数は8から16に倍増しました:\XMM{0}-\XMM{15}

\item

Win64では、関数呼び出し規約は多少異なり、ややfastcallに似ています
(\myref{fastcall})。
最初の4つの引数は \RCX 、 \RDX 、 \Reg{8}、 \Reg{9}レジスタに格納され、残りはスタックに格納されます。
また、\gls{caller}関数は32バイトを割り当てなければならないので、\gls{callee}はそこに4つの最初の引数を保存し、
それ自体の必要性のためにこれらのレジスタを使用することができます。
短い関数は単にレジスタからの引数を使用するかもしれませんが、大きいものはそれらの値をスタックに保存するかもしれません。

System V AMD64 ABI(Linux、*BSD、\MacOSX) \SysVABI はfastcallにも似ていますが、
最初の6つの引数に6つのレジスタ \RDI 、 \RSI 、 \RDX 、 \RCX 、 \Reg{8} 、 \Reg{9}を使用します。
残りはすべてスタックを介して渡されます。

呼び出し規約~(\myref{sec:callingconventions})に関するセクションも参照してください。

\item
互換性のため、 \CCpp の \Tint 型は32ビットのままです。

\item
ポインタはすべて64ビットです。

\end{itemize}

\myindex{Register allocation}

現在、レジスタの数が2倍になったため、コンパイラは\glslink{register allocator}{register allocation}
と呼ばれる操作のためのスペースをより多く持っています。
私たちにとってこれは、発行されたコードがより少ない数のローカル変数を含んでいることを意味します。

\myindex{DES}

たとえば、DES暗号化アルゴリズムの最初のSボックスを計算する関数は、
ビットスライスのDESメソッドを使用して(\GTT{DES\_type}型(uint32、uint64、SSE2、またはAVXに応じて))
32/64/128/256の値を一度に処理します。
(このテクニックの詳細はここにあります~(\myref{bitslicedes}))。

\lstinputlisting[style=customc]{patterns/20_x64/19_1.c}

ローカル変数がたくさんあります。
もちろん、すべてがローカルスタックに入るわけではありません。 
\GTT{/Ox} オプションを付けてMSVC 2008でコンパイルしてみましょう。

\lstinputlisting[caption=\Optimizing MSVC 2008,style=customasmx86]{patterns/20_x64/19_2_msvc_Ox.asm}

5つの変数がコンパイラによってローカルスタックに割り当てられました。

それでは、64ビット版のMSVC 2008でも同じことを試してみましょう。

\lstinputlisting[caption=\Optimizing MSVC 2008,style=customasmx86]{patterns/20_x64/19_3_msvc_x64.asm}

ローカルスタックにコンパイラによって割り当てられたものは何もありません。\GTT{x36}は\GTT{a5}と同義です。

\iffalse
% FIXME1 невнятно

By the way, we can see here that the function saved the \RCX and \RDX registers in space allocated by the \gls{caller},
but \Reg{8} and \Reg{9} were not saved but used from the beginning.
\fi

ところで、もっと多くの\ac{GPR}を持つCPUがあります。例えば、Itanium(128レジスタ)です。

\subsection{ARM}

64ビット命令はARMv8で登場しました。

\subsection{Float point numbers}

x86-64で浮動小数点数がどのように処理されるかは以下で説明されます:\myref{floating_SIMD}

\subsection{64-bit architecture criticism}

x64 \ac{CPU}は48ビットの外部\ac{RAM}しかアドレス指定できないという事実にもかかわらず、
キャッシュメモリを含め、ポインタを格納するために2倍のメモリが必要になることに、いら立つ人がいます。

\begin{framed}
\begin{quotation}
ここにある私の64ビットコンピュータでは、私が自分のマシンの能力を最大限使用する
ことを気にしているのであれば、ポインタを使わない方がいいと思います。
私は64ビットのレジスタを持つマシンを持っていますが、RAMは2ギガバイトしかありません。
そのため、ポインタは32を超える有効ビットを持ちません。しかし、
ポインタを使用するたびに64ビットのコストがかかり、データ構造のサイズが2倍になります。 
さらに悪いことに、ポインタはキャッシュに入り、私のキャッシュの半分がなくなり、
キャッシュをキャッシュするコストは高くつきます。

だから本当にエンベロープを押し込もうとするなら、私はポインタの代わりに
配列を使わなければなりません。 ポインタを使用しているように見えるように
複雑なマクロを作成しますが、実際は使ってはいません。
\end{quotation}
\end{framed}

( Donald Knuth in ``Coders at Work: Reflections on the Craft of Programming ''. )

自分自身のメモリアロケータを作る人もいます。 
\myindex{CryptoMiniSat}
CryptoMiniSat\footnote{\url{https://github.com/msoos/cryptominisat/}}のケースについて知っておくと面白いです。
このプログラムはめったに4GiBを超える\ac{RAM}を使用しませんが、ポインタを頻繁に使用します。
そのため、32ビットアーキテクチャでは64ビットアーキテクチャよりもメモリが少なくて済みます。
この問題を軽減するために、著者は独自のアロケータ(\IT{clauseallocator.(h|cpp)}ファイルに)を作成しました。
これにより、64ビットポインタの代わりに32ビット識別子を使用して割り当てられたメモリにアクセスできます。
