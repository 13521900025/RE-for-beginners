\subsection{除算}

\subsubsection{ビットシフトによる除算}
\label{division_by_shifting}

4で除算する例:

\begin{lstlisting}[style=customc]
unsigned int f(unsigned int a)
{
	return a/4;
};
\end{lstlisting}

MSVC 2010の結果

\begin{lstlisting}[caption=MSVC 2010,style=customasmx86]
_a$ = 8		; size = 4
_f	PROC
	mov	eax, DWORD PTR _a$[esp-4]
	shr	eax, 2
	ret	0
_f	ENDP
\end{lstlisting}

\label{SHR}
\myindex{x86!\Instructions!SHR}

この例の \SHR (\emph{SHift Right})命令は、右に2ビット分シフトしています。 
左の2つの解放されたビット(例えば、2つの最上位ビット)はゼロに設定されます。
2つの最下位ビットは破棄されます。 
実際には、これらの2つのドロップビットは除算演算の余りです。
SHR命令はSHLのように動作しますが、他の方向に動作します。

\myindex{x86!\Instructions!SHR}

\SHR 命令は \SHL のように動作しますが、別の方向に動作します。

\input{shift_right}

10進数の数字で23を想像すると理解しやすいでしょう。
最後の桁(3:除算した結果の余り)を落とすだけで、23を10で簡単に除算することができます。
2は、\gls{quotient}として動作の後に残されます。

残りの部分は削除されますが、それは問題ありません。整数値で作業しますが、
これらは\glslink{real number}{real numbers}ではありません!

ARMでの4の除算

\begin{lstlisting}[caption=\NonOptimizingKeilVI (\ARMMode),style=customasmARM]
f PROC
        LSR      r0,r0,#2
        BX       lr
        ENDP
\end{lstlisting}

MIPSでの4の除算

\begin{lstlisting}[caption=\Optimizing GCC 4.4.5 (IDA),style=customasmMIPS]
        jr      $ra
        srl     $v0, $a0, 2 ; branch delay slot
\end{lstlisting}

\myindex{MIPS!\Instructions!SRL}
SRL命令は\q{Shift Right Logical}の略です。
