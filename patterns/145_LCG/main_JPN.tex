\mysection[線形合同生成器]{擬似乱数生成器としての線形合同生成器}
\myindex{\CStandardLibrary!rand()}
\label{LCG_simple}

おそらく、線形合同ジェネレータは、乱数を生成するための最も簡単な方法です。

今日では\footnote{メルセンヌツイスターの方がいいです}選択されませんが、とても単純です
(1回の乗算、1回の加算とAND演算)。
これを例として使用できます。

\lstinputlisting[style=customc]{patterns/145_LCG/rand_JPN.c}

2つの関数があります:最初のものは内部状態を初期化するために使用され、2つ目は
擬似乱数を生成するために呼び出されます。

アルゴリズムでは2つの定数が使用されていることがわかります。 
それらは[William H. Press and Saul A. Teukolsky and William T. Vetterling and Brian P. Flannery, \emph{Numerical Recipes}, (2007)]
から取られています。

\TT{\#define} \CCpp 命令文を使ってそれらを定義しましょう。 これはマクロです。

\CCpp マクロと定数の違いは、すべてのマクロが \CCpp プリプロセッサでその値に置換され、
変数と異なりメモリを使用しないことです。

対照的に、定数は読み取り専用変数です。

定数変数のポインタ(またはアドレス)を取ることは可能ですが、マクロではできません。

C標準の\TT{my\_rand()}は0から32767の範囲の値を返さなければならないため、
最後のAND演算が必要です。

32ビットの擬似乱数値を取得する場合は、最後のAND演算を省略してください。

\subsection{x86}

\lstinputlisting[caption=\Optimizing MSVC 2013,style=customasmx86]{patterns/145_LCG/rand_MSVC_2013_x86_Ox.asm}

ここでは、両方の定数がコードに埋め込まれています。 
割り当てられたメモリはありません。

\TT{my\_srand()}関数は入力値を内部の
\TT{rand\_state}変数にコピーするだけです。

\TT{my\_rand()}はそれを受け取り、次の\TT{rand\_state}を計算し、それを切り取り、EAXレジスタに残します。

最適化されていないバージョンはより冗長です。

\lstinputlisting[caption=\NonOptimizing MSVC 2013,style=customasmx86]{patterns/145_LCG/rand_MSVC_2013_x86.asm}

\subsection{x64}

x64のバージョンはほとんど同じで、64ビットではなく32ビットのレジスタを使用しています。
(ここで \Tint 値を使用しているためです)

しかし、\TT{my\_srand()}は入力引数をスタックからではなく \ECX レジスタから取ります:

\lstinputlisting[caption=\Optimizing MSVC 2013 x64,style=customasmx86]{patterns/145_LCG/rand_MSVC_2013_x64_Ox_JPN.asm}

GCCコンパイラはほとんど同じコードを生成します。

\subsection{32ビットARM}

\lstinputlisting[caption=\OptimizingKeilVI (\ARMMode),style=customasmARM]{patterns/145_LCG/rand.s_Keil_ARM_O3_JPN.s}

32ビット定数をARM命令に埋め込むことはできないため、Keilはそれらを外部に配置して追加する必要があります。 
興味深いことに、0x7FFF定数も埋め込むことはできません。
Keilがやっているのは、\TT{rand\_state}を17ビット左にシフトし、右に17ビットシフトすることです。 
これは、 \CCpp の $(rand\_state \ll 17) \gg 17$ 命令文に似ています。
それは役に立たない操作だと思われますが、それは17ビットをクリアして15ビットをそのままにして、これが結局のところ私たちの目標です。\\
\\
Thumbモードの \Optimizing Keil はほとんど同じコードが生成します。

\subsubsection{MIPS}

\lstinputlisting[caption=\Optimizing GCC 4.4.5 (IDA),style=customasmMIPS]{patterns/08_switch/1_few/MIPS_O3_IDA_JPN.lst}

\myindex{MIPS!\Instructions!JR}

関数は常に \puts を呼び出すことで終了するので、\puts(\INS{JR}:\q{Jump Register})へのジャンプは\q{jump and link}ではなく、ここにあります。
私たちは以前これについて話しました:\myref{JMP_instead_of_RET}

\myindex{MIPS!Load delay slot}
\INS{LW}命令の後に\INS{NOP}命令も表示されることがよくあります。
これは\q{load delay slot}:MIPSの別の\emph{delay slot}です。
\myindex{MIPS!\Instructions!LW}

\INS{LW}がメモリから値をロードする間に、\INS{LW}の次の命令が実行されることがあります。

ただし、次の命令は\INS{LW}の結果を使用してはなりません。

現代のMIPS CPUは、次の命令が\INS{LW}の結果を使用するのを待つ機能を持っているので、これは幾分時代遅れですが、
GCCは古いMIPS CPU用にNOPを追加します。
一般に、無視することができます。


\subsection{スレッドセーフ版の例}

この例のスレッドセーフ版は、後で説明します:\myref{LCG_TLS}
