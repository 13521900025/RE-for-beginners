\subsection{MIPS}

\lstinputlisting[caption=\Optimizing GCC 4.4.5 (IDA),style=customasmMIPS]{patterns/145_LCG/MIPS_O3_IDA_JA.lst}

おっと、ここでは1つの定数(0x3C6EF35Fまたは1013904223)しか表示されません。 
もう1つはどこでしょうか(1664525)?

1664525による乗算は、シフトと加算だけを使用して実行されるようです! 
この仮定を確認してみましょう:

\lstinputlisting[style=customc]{patterns/145_LCG/test.c}

\lstinputlisting[caption=\Optimizing GCC 4.4.5 (IDA),style=customasmMIPS]{patterns/145_LCG/test_O3_MIPS.lst}

本当に!

\subsubsection{MIPSの再配置}

また、メモリやストアから実際にメモリにロードする操作がどのように機能するかにも焦点を当てます。

ここのリストはIDAによって作成され、IDAはいくつかの詳細を隠しています。

objdumpを2回実行します:逆アセンブルされたリストと再配置リストを取得します。

\lstinputlisting[caption=\Optimizing GCC 4.4.5 (objdump)]{patterns/145_LCG/MIPS_O3_objdump.txt}

\TT{my\_srand()}関数の2つの再配置を考えてみましょう。

最初のアドレス0は\TT{R\_MIPS\_HI16}のタイプを持ち、
アドレス8の2番目のアドレスは\TT{R\_MIPS\_LO16}のタイプです。

つまり、.bssセグメントの先頭のアドレスは、0(アドレスの上位部分)および
8(アドレスの下位部分)のアドレスに書き込まれることを意味します。

\TT{rand\_state}変数は、.bssセグメントの先頭にあります。

したがって、命令 \LUI と \SW のオペランドにはゼロがあります。何もまだ存在しないからです。
コンパイラは何をそこに書き込んだらいいかわかりません。

リンカがこれを修正し、アドレスの上位部分が \LUI のオペランドに書き込まれ、
アドレスの下位部分が \SW のオペランドに書き込まれます。

\SW はアドレスの下位部分とレジスタ \$V0 にあるものを合計します(上位部分はそこにあります)。

これは my\_rand() 関数の場合と同じです。 R\_MIPS\_HI16 再配置は、リンカに.bssセグメントアドレスの上位部分を 
\LUI 命令に書き込むように指示します。

したがって、rand\_state 変数アドレスの上位部分はレジスタ \$V1 に存在します。

アドレス0x10にある \LW 命令は、上位部分と下位部分を合計し、rand\_state
変数の値を \$V0 にロードします。

アドレス0x54にある \SW 命令は、加算を再度行い、新しい値をrand\_stateグローバル変数に
格納します。

IDAは、ロード中に再配置を処理するため、これらの詳細は隠していますが、それらを念頭に置いておく必要があります。

% TODO add example of compiled binary, GDB example, etc...
