\subsection{擬似乱数生成器の例}
\label{FPU_PRNG}

0と1の間の浮動小数点の乱数が必要な場合、最も簡単なのはメルセンヌツイスターのような
\ac{PRNG}を使うことです。
ランダムな符号なし32ビット値を生成します(つまり、ランダム32ビットを生成します)。
この値をfloatに変換し、
\GTT{RAND\_MAX}(ここでは\GTT{0xFFFFFFFF})で割ります。我々は0..1の間で値を取得します。

しかし知ってのとおり、除算は遅いです。
また、できるだけ少ないFPU演算で実行したいと考えています。
私たちは除算を取り除くことができるでしょうか?

\myindex{IEEE 754}

浮動小数点数が符号ビット、仮数ビット、指数ビットからなるものを思い出してみましょう。
ランダムな浮動小数点数を得るには、すべての仮数ビットにランダムなビットを格納するだけです。

指数部はゼロではありません(浮動小数点はこの場合非正規化されています)ので、
指数部に0b01111111
を格納しています。指数部が1であることを意味します。
次に、仮数部をランダムビットで埋め、符号ビットを0に設定する(正の数)と出来上がり。
生成される数は1と2の間にあるので、1を減算する必要があります。

\newcommand{\URLXOR}{\url{http://go.yurichev.com/17308}}

私の例では、非常に単純な線形合同乱数ジェネレータが使用され、
\footnote{アイデアは以下から取りました: \URLXOR} これは32ビットの数値を生成します。 
\ac{PRNG}はUNIXのタイムスタンプ形式で現在の時刻で初期化されます。

ここでは \Tfloat 型を\emph{union}として表します。これは、メモリの種類を
異なる型として解釈できる \CCpp 構造です。
私たちの場合、\emph{union}型の変数を作成し、
それを \Tfloat  または \emph{uint32\_t} のようにアクセスすることができます。
それはまさに汚いハックだと言えるでしょう。

% WTF?

整数\ac{PRNG}コードは、すでに検討しているものと同じです:\myref{LCG_simple}
このコードはコンパイルされた形式では省略されています。

\lstinputlisting[style=customc]{patterns/17_unions/FPU_PRNG/FPU_PRNG_JPN.cpp}

\subsubsection{x86}

\lstinputlisting[caption=\Optimizing MSVC 2010,style=customasmx86]{patterns/17_unions/FPU_PRNG/MSVC2010_Ox_Ob0_JPN.asm}

この例はC++としてコンパイルされており、これはC++での名前の変換であるため、ここでは関数名が非常に奇妙です。
これについては後で説明します:\myref{namemangling}
これをMSVC 2012でコンパイルすると、FPU用のSIMD命令が使用されます。詳細については、こちらを参照してください:\myref{FPU_PRNG_SIMD}

\iffalse
A BUG HERE
\subsubsection{MIPS}

\lstinputlisting[caption=\Optimizing GCC 4.4.5,style=customasmMIPS]{patterns/17_unions/FPU_PRNG/MIPS_O3_IDA_JPN.lst}

いくつかの奇妙な理由のために追加された無駄な\INS{LUI}命令もあります。 
このアーティファクトを以前検討しました:\myref{MIPS_FPU_LUI}
\fi

\subsubsection{ARM (\ARMMode)}

\lstinputlisting[caption=\Optimizing GCC 4.6.3 (IDA),style=customasmARM]{patterns/17_unions/FPU_PRNG/raspberry_GCC_O3_IDA_JPN.lst}

\myindex{objdump}
\myindex{binutils}
\myindex{IDA}

また、objdumpにダンプを作成し、FPU命令の名前が \IDA とは異なることを確認します。 
見たところ、IDAとbinutilsの開発者は異なるマニュアルを使ったのでしょうか? 
おそらく、両方の命令の変種を知っておくとよいでしょう。

\lstinputlisting[caption=\Optimizing GCC 4.6.3 (objdump),style=customasmARM]{patterns/17_unions/FPU_PRNG/raspberry_GCC_O3_objdump.lst}

\TT{float\_rand()}の0x5cと \main の0x38の命令は、(疑似)乱数ノイズです。
