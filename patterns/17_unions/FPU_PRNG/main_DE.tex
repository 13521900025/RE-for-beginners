\subsection{Pseudozufallszahlengenerator Beispiel}
\label{FPU_PRNG}
Wenn wir Zufallszahlen vom Typ \Tfloat zwischen 0 und 1 brauchen, ist die einfachste Möglichkeit einen \ac{PRNG} wie den
Mersenne-Twister zu verwenden.
Er produziert vorzeichenlose 32-Bit-Werte (mit anderen Worten: er erzeugt 32 zufällige Bits).
Diesen Wert können wir in einen \Tfloat umwandeln und dann durch \GTT{RAND\_MAX} (\GTT{0xFFFFFFFF} in unserem Fall)
teilen---wir erhalten einen Wert im Intervall 0..1.

Wie wir jedoch wissen, ist die Division langsam.
Auch würden wir gerne so wenig FPU Operation wie möglich verwenden.
Deshalb fragen wir uns, ob wir die Division loswerden können.

\myindex{IEEE 754}
Erinnern wir uns an den Aufbau einer Fließkommazahl: sie besteht aus einem Vorzeichenbit, Bits im Signifikanden und Bits
im Exponenten.
Wir müssen also lediglich Zufallsbits in allen Bits des Signifikanden speichern um einen zufällige Fließkommazahl zu
erhalten!

Der Exponent kann nicht null sein (in diesem Fall ist die Fließkommazahl denormalisiert); also speichern wir 0b01111111
im Exponenten---das bedeutet, dass der Exponent 1 ist.
Dann füllen wir den Signifikanden mit Zufallsbits, setzen das Vorzeichenbit auf 0 (entspricht einer positiven Zahl) und
voilà.
Die erzeugte Zahl liegt zwischen 1 und 2; wir müssen also am Ende noch 1 abziehen.

\newcommand{\URLXOR}{\url{http://go.yurichev.com/17308}}
Ein sehr einfacher linearer Kongruenzgenerator für Zufallszahlen wird in meinem Beispiel\footnote{die Idee stammt von:
\URLXOR} vorgestellt: er erzeugt 32-Bit-Zahlen.
Der \ac{PRNG} wird mit der aktuellen Zeit um UNIX-Timestamp-Format initialisiert.

Wir stellen hier den Typ \Tfloat als \emph{union} dar--das ist die \CCpp Konstruktion, die es uns ermöglicht, einen
Speicherblock als unterschiedliche Typen aufzufassen.
In unserem Fall sind wir in der Lage eine Variable vom Typ \emph{union} zu erzeugen und dann auf diese wie auf einen
\Tfloat oder \emph{uint32\_t} zuzugreifen.
Man kann sagen, dass es sich dabei um einen Hack, d.h. Trick handelt. Sogar um einen sehr schmutzigen.

% WTF?
Der \ac{PRNG} Code für Integer ist der bereits betrachtete: \myref{LCG_simple}.
Deshalb verzichten wir an dieser Stelle auf ein erneutes Listing des kompilierten Codes.

\lstinputlisting[style=customc]{patterns/17_unions/FPU_PRNG/FPU_PRNG_EN.cpp}

\subsubsection{x86}

\lstinputlisting[caption=\Optimizing MSVC 2010,style=customasmx86]{patterns/17_unions/FPU_PRNG/MSVC2010_Ox_Ob0_DE.asm}
Die Namen der Funktionen sind hier so seltsam, weil das Beispiel als C++ kompiliert wurde, und hier name mangling in C++
vorliegt. Dies werden wir später besprechen: \myref{namemangling}.
Wenn wir das Beispiel mit MSVC 2012 kompilieren, verwendet es SIMD Befehle für die FPU; mehr dazu unter:
\myref{FPU_PRNG_SIMD}.

\iffalse
A BUG HERE
\subsubsection{MIPS}

\lstinputlisting[caption=\Optimizing GCC 4.4.5,style=customasmMIPS]{patterns/17_unions/FPU_PRNG/MIPS_O3_IDA_DE.lst}
Hier wurde auch ein unnützer \INS{LUI} Befehl aus unerfindlichen Gründen hinzugefügt.
Wir haben solch ein Artefakt schon früher betrachtet: \myref{MIPS_FPU_LUI}.
\fi

\subsubsection{ARM (\ARMMode)}

\lstinputlisting[caption=\Optimizing GCC 4.6.3
(IDA),style=customasmARM]{patterns/17_unions/FPU_PRNG/raspberry_GCC_O3_IDA_DE.lst}

\myindex{objdump}
\myindex{binutils}
\myindex{IDA}
Wir ziehen auch einen Dump in objdump und shen, dass die FPU Befehle andere Namen als in \IDA haben.
Offenbar haben die Entwickler von \IDA und binutils unterschiedliche Handbücher verwendet.
Möglicherweise ist es hilfreich, beide Varianten der Befehlsnamen zu kennen.

\lstinputlisting[caption=\Optimizing GCC 4.6.3 (objdump),style=customasmARM]{patterns/17_unions/FPU_PRNG/raspberry_GCC_O3_objdump.lst}

Die Befehle an den Stellen 0x5c in \TT{float\_rand()} und 0x38 in \main sind (Pseudo-)Zufallsrauschen.

