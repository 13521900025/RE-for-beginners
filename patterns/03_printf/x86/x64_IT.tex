\subsubsection{x64: 8 argomenti}

\myindex{x86-64}
\label{example_printf8_x64}
Per vedere come altri argomenti sono passati tramite lo stack, cambiamo nuovamente l'esempio per aumentare il numero degli argomenti a 9 (format string di \printf + 8 variabili \Tint):

\lstinputlisting[style=customc]{patterns/03_printf/2.c}

\myparagraph{MSVC}

Come gia' detto in precedenza, i primi 4 argomenti devono essere passati tramite i registri \RCX, \RDX, \Reg{8}, \Reg{9} in Win64,
mentre tutto il resto ---tramite lo stack.
E' esattamente quello che vediamo qui.
Tuttavia, l'istruzione \MOV instruction e' usata al posto di \PUSH per preparare lo stack, in modo tale che i valori siano memorizzati
nello stack in maniera diretta.

% TODO translate to Italian
\lstinputlisting[caption=MSVC 2012 x64,style=customasmx86]{patterns/03_printf/x86/2_MSVC_x64_EN.asm}

Il lettore attento potrebbe chiedere perche' per i valori \Tint sono allocati 8 byte quando ne bastano 4?
Bisogna ricordare che per ogni tipo di dato piu' piccolo di 64 bit, sono allocati 8 byte.
Questo e' stabilito per convenienza: rende piu' facile calcolare l'indirizzo di argomenti arbitrari. E inoltre fa si che tutti
siano allocati ad indirizzi di memoria allineati.
Succede lo stesso in ambienti a 32-bit environments: sono riservati 4 byte per ogni tipo di dato.

% also for local variables?

\myparagraph{GCC}

L'immagine e' simile per i sistemi operativi x86-64 e *NIX, con l'eccezzione che i primi 6 argomenti sono passati attraverso
i registri \RDI, \RSI, \RDX, \RCX, \Reg{8}, \Reg{9} registers.
Tutto il resto ---tramite lo stack.
GCC genera il codice memorizzando il puntatore alla stringa in \EDI invece che \RDI{}---lo abbiamo visto in precedenza: 
\myref{hw_EDI_instead_of_RDI}.

Abbiamo anche notato prima che il registro \EAX e' stato azzerato prima di una chiamata a \printf: \myref{SysVABI_input_EAX}.

% TODO translate to Italian
\lstinputlisting[caption=\Optimizing GCC 4.4.6 x64,style=customasmx86]{patterns/03_printf/x86/2_GCC_x64_EN.s}

\myparagraph{GCC + GDB}
\myindex{GDB}

Proviamo l'esempio in \ac{GDB}.

\begin{lstlisting}
$ gcc -g 2.c -o 2
\end{lstlisting}

\begin{lstlisting}
$ gdb 2
GNU gdb (GDB) 7.6.1-ubuntu
...
Reading symbols from /home/dennis/polygon/2...done.
\end{lstlisting}

\begin{lstlisting}[caption=impostiamo il breakpoint su \printf{,} e avviamo]
(gdb) b printf
Breakpoint 1 at 0x400410
(gdb) run
Starting program: /home/dennis/polygon/2 

Breakpoint 1, __printf (format=0x400628 "a=%d; b=%d; c=%d; d=%d; e=%d; f=%d; g=%d; h=%d\n") at printf.c:29
29	printf.c: No such file or directory.
\end{lstlisting}

I registri \RSI/\RDX/\RCX/\Reg{8}/\Reg{9} hanno i valori previsti.
\RIP ha l'indirizzo della prima istruzione della funzione \printf.

\begin{lstlisting}
(gdb) info registers
rax            0x0	0
rbx            0x0	0
rcx            0x3	3
rdx            0x2	2
rsi            0x1	1
rdi            0x400628	4195880
rbp            0x7fffffffdf60	0x7fffffffdf60
rsp            0x7fffffffdf38	0x7fffffffdf38
r8             0x4	4
r9             0x5	5
r10            0x7fffffffdce0	140737488346336
r11            0x7ffff7a65f60	140737348263776
r12            0x400440	4195392
r13            0x7fffffffe040	140737488347200
r14            0x0	0
r15            0x0	0
rip            0x7ffff7a65f60	0x7ffff7a65f60 <__printf>
...
\end{lstlisting}

\begin{lstlisting}[caption=ispezioniamo la format string]
(gdb) x/s $rdi
0x400628:	"a=%d; b=%d; c=%d; d=%d; e=%d; f=%d; g=%d; h=%d\n"
\end{lstlisting}

Effettuiamo un dump dello stack, questa volta con il comando x/g ---\emph{g} sta per \emph{giant words}, ovvero 64-bit words.

\begin{lstlisting}
(gdb) x/10g $rsp
0x7fffffffdf38:	0x0000000000400576	0x0000000000000006
0x7fffffffdf48:	0x0000000000000007	0x00007fff00000008
0x7fffffffdf58:	0x0000000000000000	0x0000000000000000
0x7fffffffdf68:	0x00007ffff7a33de5	0x0000000000000000
0x7fffffffdf78:	0x00007fffffffe048	0x0000000100000000
\end{lstlisting}

Il primo elemento dello stack, proprio come nel caso precedente, e' il {RA}.
3 valori vengono passatri trvamite lo stack : 6, 7, 8.
Notiamo anche che 8 e' passato nei 32-bit alti non azzerati: \GTT{0x00007fff00000008}.
Cio' va bene, perche' i valori hanno tipo \Tint , che e' a 32-bit.
Quindi, i registri alti o gli elementi alti dello stack potrebbero contenere \q{random garbage}.

Se andiamo a vedere dove viene restituito il controllodopo l'esecuzione di \printf, 
\ac{GDB} mostrera' l'intera funzione \main:

\begin{lstlisting}[style=customasmx86]
(gdb) set disassembly-flavor intel
(gdb) disas 0x0000000000400576
Dump of assembler code for function main:
   0x000000000040052d <+0>:	push   rbp
   0x000000000040052e <+1>:	mov    rbp,rsp
   0x0000000000400531 <+4>:	sub    rsp,0x20
   0x0000000000400535 <+8>:	mov    DWORD PTR [rsp+0x10],0x8
   0x000000000040053d <+16>:	mov    DWORD PTR [rsp+0x8],0x7
   0x0000000000400545 <+24>:	mov    DWORD PTR [rsp],0x6
   0x000000000040054c <+31>:	mov    r9d,0x5
   0x0000000000400552 <+37>:	mov    r8d,0x4
   0x0000000000400558 <+43>:	mov    ecx,0x3
   0x000000000040055d <+48>:	mov    edx,0x2
   0x0000000000400562 <+53>:	mov    esi,0x1
   0x0000000000400567 <+58>:	mov    edi,0x400628
   0x000000000040056c <+63>:	mov    eax,0x0
   0x0000000000400571 <+68>:	call   0x400410 <printf@plt>
   0x0000000000400576 <+73>:	mov    eax,0x0
   0x000000000040057b <+78>:	leave  
   0x000000000040057c <+79>:	ret    
End of assembler dump.
\end{lstlisting}

Finiamo di eseguire \printf, eseguiamo l'istruzione che azzera \EAX, e notiamo che il registro \EAX register ha esattamente il valore zero.
\RIP adesso punta all'istruzione \INS{LEAVE} , ovvero la penultima nella funzione \main.

\begin{lstlisting}
(gdb) finish
Run till exit from #0  __printf (format=0x400628 "a=%d; b=%d; c=%d; d=%d; e=%d; f=%d; g=%d; h=%d\n") at printf.c:29
a=1; b=2; c=3; d=4; e=5; f=6; g=7; h=8
main () at 2.c:6
6		return 0;
Value returned is $1 = 39
(gdb) next
7	};
(gdb) info registers
rax            0x0	0
rbx            0x0	0
rcx            0x26	38
rdx            0x7ffff7dd59f0	140737351866864
rsi            0x7fffffd9	2147483609
rdi            0x0	0
rbp            0x7fffffffdf60	0x7fffffffdf60
rsp            0x7fffffffdf40	0x7fffffffdf40
r8             0x7ffff7dd26a0	140737351853728
r9             0x7ffff7a60134	140737348239668
r10            0x7fffffffd5b0	140737488344496
r11            0x7ffff7a95900	140737348458752
r12            0x400440	4195392
r13            0x7fffffffe040	140737488347200
r14            0x0	0
r15            0x0	0
rip            0x40057b	0x40057b <main+78>
...
\end{lstlisting}
