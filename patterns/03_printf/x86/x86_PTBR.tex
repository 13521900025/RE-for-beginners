\subsubsection{x86: 3 argumentos}

\myparagraph{MSVC}

Quando compilamos esse código com o MSVC 2010 Express temos:

\begin{lstlisting}[style=customasmx86]
$SG3830	DB	'a=%d; b=%d; c=%d', 00H

...

	push	3
	push	2
	push	1
	push	OFFSET $SG3830
	call	_printf
	add	esp, 16					; 00000010H
\end{lstlisting}

Quase a mesma coisa, mas agora nós podemos ver que os argumentos da função \printf são empurrados na pilha na ordem reversa. O primeiro argumento é empurrado por último.

A propósito, variáveis do tipo \Tint em ambientes 32-bits tem 32-bits de largura, isso é 4 bytes.

Então, nós temos quatro argumentos aqui, $4*4=16$ --- eles ocupam exatamente 16 bytes na pilha: um ponteiro de 32-bits para uma string e três números do tipo \Tint.

\myindex{x86!\Instructions!ADD}
\myindex{x86!\Registers!ESP}
\myindex{cdecl}
Quando o ponteiro da pilha (registrador \ESP) volta para seu valor anterior pela instrução \INS{ADD ESP, X} depois de uma chamada de função,
geralmente o número de argumentos pode ser obtido simplismente por se dividir X, o argumento da função \ADD, por 4.

Lógico que isso é por causa da convenção de chamada do \emph{cdecl} e somente para ambientes 32-bits.

% TODO: Olly, GCC, GDB

