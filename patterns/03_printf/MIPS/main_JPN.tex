\subsection{MIPS}

\subsubsection{3 arguments}

\myparagraph{\Optimizing GCC 4.4.5}

\q{\HelloWorldSectionName} の例との主な違いは、 \puts の代わりに \printf が呼び出され、
さらに3つの引数がレジスタ \$5\dots \$7 (または \$A0\dots \$A2 )に渡されるという点です。 
そのため、これらのレジスタの前にA-が付いています。これは、関数引数の受け渡しに使用されることを意味します。

\lstinputlisting[caption=\Optimizing GCC 4.4.5 (\assemblyOutput),style=customasmMIPS]{patterns/03_printf/MIPS/printf3.O3_JPN.s}

\lstinputlisting[caption=\Optimizing GCC 4.4.5 (IDA),style=customasmMIPS]{patterns/03_printf/MIPS/printf3.O3.IDA_JPN.lst}

\IDA は、\INS{LUI}および\INS{ADDIU}命令のペアを1つの\INS{LA}疑似命令に統合しました。 
だから、アドレス0x1Cに命令がないのは、\INS{LA}が8バイトを\emph{占有}しているからです。

\myparagraph{\NonOptimizing GCC 4.4.5}

\NonOptimizing GCC はもっと冗長です。

\lstinputlisting[caption=\NonOptimizing GCC 4.4.5 (\assemblyOutput),style=customasmMIPS]{patterns/03_printf/MIPS/printf3.O0_JPN.s}

\lstinputlisting[caption=\NonOptimizing GCC 4.4.5 (IDA),style=customasmMIPS]{patterns/03_printf/MIPS/printf3.O0.IDA_JPN.lst}

\subsubsection{8つの引数}

前のセクションの9つの引数を使用して、例を再度使用してみましょう:\myref{example_printf8_x64}

\lstinputlisting[style=customc]{patterns/03_printf/2.c}

\myparagraph{\Optimizing GCC 4.4.5}

最初の4つの引数だけが \$A0 \dots \$A3 レジスタに渡され、残りはスタックを介して渡されます。
\myindex{MIPS!O32}

これはO32呼び出し規約(MIPS世界で最も一般的なもの)です。 
他の呼び出し規則(N32のような)は、異なる目的のためにレジスタを使用するかもしれません。

\myindex{MIPS!\Instructions!SW}

\INS{SW}は\q{Store Word}(レジスタからメモリへ)の略語です。 
MIPSには値をメモリに格納する命令がないため、代わりに命令ペア(\INS{LI}/\INS{SW})を使用する必要があります。

\lstinputlisting[caption=\Optimizing GCC 4.4.5 (\assemblyOutput),style=customasmMIPS]{patterns/03_printf/MIPS/printf8.O3_JPN.s}

\lstinputlisting[caption=\Optimizing GCC 4.4.5 (IDA),style=customasmMIPS]{patterns/03_printf/MIPS/printf8.O3.IDA_JPN.lst}

\myparagraph{\NonOptimizing GCC 4.4.5}

\NonOptimizing GCC はもっと冗長です。

\lstinputlisting[caption=\NonOptimizing GCC 4.4.5 (\assemblyOutput),style=customasmMIPS]{patterns/03_printf/MIPS/printf8.O0_JPN.s}

\lstinputlisting[caption=\NonOptimizing GCC 4.4.5 (IDA),style=customasmMIPS]{patterns/03_printf/MIPS/printf8.O0.IDA_JPN.lst}

