\subsection{MIPS}

\subsubsection{3 аргумента}

\myparagraph{\Optimizing GCC 4.4.5}

Главное отличие от примера \q{\HelloWorldSectionName} в том, что здесь на самом деле
вызывается \printf вместо \puts и ещё три аргумента передаются в регистрах  \$5\dots \$7 (или \$A0\dots \$A2).
Вот почему эти регистры имеют префикс A-. Это значит, что они используются для передачи аргументов.

\lstinputlisting[caption=\Optimizing GCC 4.4.5 (\assemblyOutput),style=customasmMIPS]{patterns/03_printf/MIPS/printf3.O3_RU.s}

\lstinputlisting[caption=\Optimizing GCC 4.4.5 (IDA),style=customasmMIPS]{patterns/03_printf/MIPS/printf3.O3.IDA_RU.lst}

\IDA объединила пару инструкций \INS{LUI} и \INS{ADDIU} в одну псевдоинструкцию \INS{LA}.
Вот почему здесь нет инструкции по адресу 0x1C: потому что \INS{LA} \emph{занимает} 8 байт.

\myparagraph{\NonOptimizing GCC 4.4.5}

\NonOptimizing GCC более многословен:

\lstinputlisting[caption=\NonOptimizing GCC 4.4.5 (\assemblyOutput),style=customasmMIPS]{patterns/03_printf/MIPS/printf3.O0_RU.s}

\lstinputlisting[caption=\NonOptimizing GCC 4.4.5 (IDA),style=customasmMIPS]{patterns/03_printf/MIPS/printf3.O0.IDA_RU.lst}

\subsubsection{8 аргументов}

Снова воспользуемся примером с 9-ю аргументами из предыдущей секции: \myref{example_printf8_x64}.

\lstinputlisting[style=customc]{patterns/03_printf/2.c}

\myparagraph{\Optimizing GCC 4.4.5}

Только 4 первых аргумента передаются в регистрах \$A0 \dots \$A3, так что остальные передаются через стек.

\myindex{MIPS!O32}
Это соглашение о вызовах O32 (самое популярное в мире MIPS).
Другие соглашения о вызовах (например N32) могут наделять регистры другими функциями.

\myindex{MIPS!\Instructions!SW}
\INS{SW} означает \q{Store Word} (записать слово из регистра в память).
В MIPS нет инструкции для записи значения в память, так что для этого используется пара инструкций (\INS{LI}/\INS{SW}).

\lstinputlisting[caption=\Optimizing GCC 4.4.5 (\assemblyOutput),style=customasmMIPS]{patterns/03_printf/MIPS/printf8.O3_RU.s}

\lstinputlisting[caption=\Optimizing GCC 4.4.5 (IDA),style=customasmMIPS]{patterns/03_printf/MIPS/printf8.O3.IDA_RU.lst}

\myparagraph{\NonOptimizing GCC 4.4.5}

\NonOptimizing GCC более многословен:

\lstinputlisting[caption=\NonOptimizing GCC 4.4.5 (\assemblyOutput),style=customasmMIPS]{patterns/03_printf/MIPS/printf8.O0_RU.s}

\lstinputlisting[caption=\NonOptimizing GCC 4.4.5 (IDA),style=customasmMIPS]{patterns/03_printf/MIPS/printf8.O0.IDA_RU.lst}

