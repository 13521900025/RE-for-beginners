\subsubsection{ARM: 8 аргументов}

Снова воспользуемся примером с 9-ю аргументами из предыдущей секции: \myref{example_printf8_x64}.

\lstinputlisting[style=customc]{patterns/03_printf/2.c}

\myparagraph{\OptimizingKeilVI: \ARMMode}

\lstinputlisting[style=customasmARM]{patterns/03_printf/ARM/tmp1.asm}

Этот код можно условно разделить на несколько частей:

\myindex{Function prologue}
\begin{itemize}
\item Пролог функции:

\myindex{ARM!\Instructions!STR}
Самая первая инструкция \INS{STR LR, [SP,\#var\_4]!} 
сохраняет в стеке \ac{LR}, ведь нам придется использовать этот регистр для вызова \printf.
Восклицательный знак в конце означает \emph{pre-index}.
Это значит, что в начале \ac{SP} должно быть уменьшено на 4, затем по адресу в \ac{SP} должно быть записано значение \ac{LR}.

Это аналог знакомой в x86 инструкции \PUSH. Читайте больше об этом: \myref{ARM_postindex_vs_preindex}.

\myindex{ARM!\Instructions!SUB}
Вторая инструкция \INS{SUB SP, SP, \#0x14} уменьшает \glslink{stack pointer}{указатель стека} \ac{SP}, но, на самом деле, эта процедура нужна для выделения в локальном стеке места размером \GTT{0x14} (20) байт.
Действительно, нам нужно передать 5 32-битных значений через стек в \printf. Каждое значение занимает 4 байта, все вместе~--- $5*4=20$.
Остальные 4 32-битных значения будут переданы через регистры.

\item Передача 5, 6, 7 и 8 через стек:
они записываются в регистры \Reg{0}, \Reg{1}, \Reg{2} и \Reg{3} соответственно.\\
Затем инструкция \INS{ADD R12, SP, \#0x18+var\_14} 
записывает в регистр \Reg{12} адрес места в стеке, куда будут помещены эти 4 значения.
\myindex{IDA!var\_?}
\emph{var\_14}~--- это макрос ассемблера, равный -0x14.
Такие макросы создает \IDA, чтобы удобнее было показывать, как код обращается к стеку.

Макросы \emph{var\_?}, создаваемые \IDA, отражают локальные переменные в стеке. Так что в \Reg{12} будет записано \GTT{SP+4}.

\myindex{ARM!\Instructions!STMIA}
Следующая инструкция \INS{STMIA R12, {R0-R3}} записывает содержимое регистров \Reg{0}-\Reg{3} по адресу в памяти, на который указывает \Reg{12}.

Инструкция \INS{STMIA} означает \emph{Store Multiple Increment After}.\\
\emph{Increment After} означает, что \Reg{12} будет увеличиваться на 4 после записи каждого значения регистра.

\item Передача 4 через стек:
4 записывается в \Reg{0}, затем инструкция \INS{STR R0, [SP,\#0x18+var\_18]} записывает его в стек.
\emph{var\_18} равен -0x18, смещение будет 0, 
так что значение из регистра \Reg{0} (4) запишется туда, куда указывает \ac{SP}.

\item Передача 1, 2 и 3 через регистры:

Значения для первых трех чисел (a, b, c) (1, 2, 3 соответственно) передаются в регистрах 
\Reg{1}, \Reg{2} и \Reg{3} перед самим вызовом \printf, а остальные 5 значений передаются через стек, и вот как:

\item Вызов \printf.

\myindex{Function epilogue}
\item Эпилог функции:

Инструкция \INS{ADD SP, SP, \#0x14} возвращает \ac{SP} на прежнее место, 
аннулируя таким образом всё, что было записано в стек.
Конечно, то что было записано в стек, там пока и останется, но всё это будет многократно 
перезаписано во время исполнения последующих функций.

\myindex{ARM!\Instructions!LDR}
Инструкция \INS{LDR PC, [SP+4+var\_4],\#4} загружает в \ac{PC} 
сохраненное значение \ac{LR} из стека, обеспечивая таким образом выход из функции.

Здесь нет восклицательного знака~--- действительно, сначала \ac{PC} загружается из места, куда указывает \ac{SP}
($4+var\_4=4+(-4)=0$, так что эта инструкция аналогична \INS{LDR PC, [SP],\#4}), затем \ac{SP} увеличивается 
на 4.
Это называется \emph{post-index}\footnote{Читайте больше об этом: \myref{ARM_postindex_vs_preindex}.}.
Почему \IDA показывает инструкцию именно так?
Потому что она хочет показать разметку стека и тот факт, что \GTT{var\_4} выделена в локальном стеке именно для сохраненного
значения \ac{LR}.
Эта инструкция в каком-то смысле аналогична \INS{POP PC} в x86
\footnote{В x86 невозможно установить значение \GTT{IP/EIP/RIP} используя \POP, но будем надеяться, вы поняли аналогию.}.

\end{itemize}

\myparagraph{\OptimizingKeilVI: \ThumbMode}

\lstinputlisting[style=customasmARM]{patterns/03_printf/ARM/tmp2.asm}

Это почти то же самое что и в предыдущем примере, только код для Thumb и значения помещаются в 
стек немного иначе: сначала 8 за первый раз, затем 5, 6, 7 за второй раз и 4 за третий раз.

\myparagraph{\OptimizingXcodeIV: \ARMMode}

\begin{lstlisting}[style=customasmARM]
__text:0000290C             _printf_main2
__text:0000290C
__text:0000290C             var_1C = -0x1C
__text:0000290C             var_C  = -0xC
__text:0000290C
__text:0000290C 80 40 2D E9   STMFD  SP!, {R7,LR}
__text:00002910 0D 70 A0 E1   MOV    R7, SP
__text:00002914 14 D0 4D E2   SUB    SP, SP, #0x14
__text:00002918 70 05 01 E3   MOV    R0, #0x1570
__text:0000291C 07 C0 A0 E3   MOV    R12, #7
__text:00002920 00 00 40 E3   MOVT   R0, #0
__text:00002924 04 20 A0 E3   MOV    R2, #4
__text:00002928 00 00 8F E0   ADD    R0, PC, R0
__text:0000292C 06 30 A0 E3   MOV    R3, #6
__text:00002930 05 10 A0 E3   MOV    R1, #5
__text:00002934 00 20 8D E5   STR    R2, [SP,#0x1C+var_1C]
__text:00002938 0A 10 8D E9   STMFA  SP, {R1,R3,R12}
__text:0000293C 08 90 A0 E3   MOV    R9, #8
__text:00002940 01 10 A0 E3   MOV    R1, #1
__text:00002944 02 20 A0 E3   MOV    R2, #2
__text:00002948 03 30 A0 E3   MOV    R3, #3
__text:0000294C 10 90 8D E5   STR    R9, [SP,#0x1C+var_C]
__text:00002950 A4 05 00 EB   BL     _printf
__text:00002954 07 D0 A0 E1   MOV    SP, R7
__text:00002958 80 80 BD E8   LDMFD  SP!, {R7,PC}
\end{lstlisting}

\myindex{ARM!\Instructions!STMFA}
\myindex{ARM!\Instructions!STMIB}
Почти то же самое, что мы уже видели, за исключением того, что \INS{STMFA} (Store Multiple Full Ascending)~--- 
это синоним инструкции \INS{STMIB} (Store Multiple Increment Before). 
Эта инструкция увеличивает \ac{SP} и только затем записывает в память значение очередного регистра, но не наоборот.

Далее бросается в глаза то, что инструкции как будто бы расположены случайно.
Например, значение в регистре \Reg{0} подготавливается в трех местах, по адресам \GTT{0x2918}, \GTT{0x2920} и \GTT{0x2928}, 
когда это можно было бы сделать в одном месте.
Однако, у оптимизирующего компилятора могут быть свои доводы о том, как лучше составлять инструкции 
друг с другом для лучшей эффективности исполнения.
Процессор обычно пытается исполнять одновременно идущие друг за другом инструкции.
К примеру, инструкции \INS{MOVT R0, \#0} и \INS{ADD R0, PC, R0} не могут быть исполнены одновременно,
потому что обе инструкции модифицируют регистр \Reg{0}. 
А вот инструкции \INS{MOVT R0, \#0} и \INS{MOV R2, \#4} легко можно исполнить одновременно, 
потому что эффекты от их исполнения никак не конфликтуют друг с другом.
Вероятно, компилятор старается генерировать код именно таким образом там, где это возможно.
 
\myparagraph{\OptimizingXcodeIV: \ThumbTwoMode}

\begin{lstlisting}[style=customasmARM]
__text:00002BA0               _printf_main2
__text:00002BA0
__text:00002BA0               var_1C = -0x1C
__text:00002BA0               var_18 = -0x18
__text:00002BA0               var_C  = -0xC
__text:00002BA0
__text:00002BA0 80 B5          PUSH     {R7,LR}
__text:00002BA2 6F 46          MOV      R7, SP
__text:00002BA4 85 B0          SUB      SP, SP, #0x14
__text:00002BA6 41 F2 D8 20    MOVW     R0, #0x12D8
__text:00002BAA 4F F0 07 0C    MOV.W    R12, #7
__text:00002BAE C0 F2 00 00    MOVT.W   R0, #0
__text:00002BB2 04 22          MOVS     R2, #4
__text:00002BB4 78 44          ADD      R0, PC  ; char *
__text:00002BB6 06 23          MOVS     R3, #6
__text:00002BB8 05 21          MOVS     R1, #5
__text:00002BBA 0D F1 04 0E    ADD.W    LR, SP, #0x1C+var_18
__text:00002BBE 00 92          STR      R2, [SP,#0x1C+var_1C]
__text:00002BC0 4F F0 08 09    MOV.W    R9, #8
__text:00002BC4 8E E8 0A 10    STMIA.W  LR, {R1,R3,R12}
__text:00002BC8 01 21          MOVS     R1, #1
__text:00002BCA 02 22          MOVS     R2, #2
__text:00002BCC 03 23          MOVS     R3, #3
__text:00002BCE CD F8 10 90    STR.W    R9, [SP,#0x1C+var_C]
__text:00002BD2 01 F0 0A EA    BLX      _printf
__text:00002BD6 05 B0          ADD      SP, SP, #0x14
__text:00002BD8 80 BD          POP      {R7,PC}
\end{lstlisting}

Почти то же самое, что и в предыдущем примере,
лишь за тем исключением, что здесь используются Thumb-инструкции.

% FIXME: also STMIA is used instead of STMIB,
% which is why it uses LR, which is 4 bytes ahead of SP

\myparagraph{ARM64}

\mysubparagraph{\NonOptimizing GCC (Linaro) 4.9}

\lstinputlisting[caption=\NonOptimizing GCC (Linaro) 4.9,style=customasmARM]{patterns/03_printf/ARM/ARM8_O0_RU.lst}

Первые 8 аргументов передаются в X- или W-регистрах: \ARMPCS.
Указатель на строку требует 64-битного регистра, так что он передается в \RegX{0}.
Все остальные значения имеют 32-битный тип \Tint, так что они записываются в 32-битные части регистров (W-).
Девятый аргумент (8) передается через стек.
Действительно, невозможно передать большое количество аргументов в регистрах, потому что количество регистров ограничено.

\Optimizing GCC (Linaro) 4.9 генерирует почти такой же код.
