\clearpage
\myparagraph{x86 + MSVC + \olly}
\myindex{\olly}
\myindex{x86!\Registers!\Flags}

Possiamo vedere come vengono settati i flag facendo girare l'esempio in \olly.
Iniziamo con \TT{f\_unsigned()}, che funziona con numeri di tipo unsigned.

L'istruzione \CMP è eseguita tre volte e con gli stessi argomenti, pertanto i flag saranno ogni volta gli stessi.

Risultato del primo confronto:

\begin{figure}[H]
\centering
\myincludegraphics{patterns/07_jcc/simple/olly_unsigned1.png}
\caption{\olly: \TT{f\_unsigned()}: primo salto condizionale}
\label{fig:jcc_olly_unsigned_1}
\end{figure}

I flag sono: C=1, P=1, A=1, Z=0, S=1, T=0, D=0, O=0.
In \olly sono riportati per brevità con la sola iniziale.


\olly suggerisce che il jump (\JBE) sarà innescato.
Infatti, consultando i manuali Intel (\myref{x86_manuals}), vediamo che \JBE è innescato se CF=1 o ZF=1.
La condizione è vera, e quindi il salto viene effettuato.

\clearpage
Jump condizionale successivo:

\begin{figure}[H]
\centering
\myincludegraphics{patterns/07_jcc/simple/olly_unsigned2.png}
\caption{\olly: \TT{f\_unsigned()}: secondo conditional jump}
\label{fig:jcc_olly_unsigned_2}
\end{figure}

\olly dice che il \JNZ verrà seguito.
Infatti, \JNZ è innescato se ZF=0 (zero flag).

\clearpage
Il terzo conditional jump, \JNB:

\begin{figure}[H]
\centering
\myincludegraphics{patterns/07_jcc/simple/olly_unsigned3.png}
\caption{\olly: \TT{f\_unsigned()}: terzo conditional jump}
\label{fig:jcc_olly_unsigned_3}
\end{figure}

Nei manuali Intel (\myref{x86_manuals}) vediamo che \JNB è innescato se CF=0 (carry flag).
Nel nostro caso questa condizione non è vera, e quindi la terza \printf sarà eseguita.

\clearpage
Rivediamo ora la funzione \TT{f\_signed()}, che opera con valori signed, in \olly.
I flag sono settati allo stesso modo: C=1, P=1, A=1, Z=0, S=1, T=0, D=0, O=0.
Il primo salto condizionale \JLE sarà seguito:

\begin{figure}[H]
\centering
\myincludegraphics{patterns/07_jcc/simple/olly_signed1.png}
\caption{\olly: \TT{f\_signed()}: primo conditional jump}
\label{fig:jcc_olly_signed_1}
\end{figure}

Nei manuali Intel (\myref{x86_manuals}) vediamo che questa istruzione viene innescata se ZF=1 o SF$\neq$OF.
SF$\neq$OF nel nostro caso, quindi il salto viene effettuato.

\clearpage
Il secondo jump condizionale \JNZ viene innescato, se ZF=0 (zero flag):

\begin{figure}[H]
\centering
\myincludegraphics{patterns/07_jcc/simple/olly_signed2.png}
\caption{\olly: \TT{f\_signed()}: secondo conditional jump}
\label{fig:jcc_olly_signed_2}
\end{figure}

\clearpage
Il terzo jump \JGE non sarà innescato in quanto lo sarebbe solo se SF=OF, condizione non vera nel nostro caso:

\begin{figure}[H]
\centering
\myincludegraphics{patterns/07_jcc/simple/olly_signed3.png}
\caption{\olly: \TT{f\_signed()}: terzo conditional jump}
\label{fig:jcc_olly_signed_3}
\end{figure}
