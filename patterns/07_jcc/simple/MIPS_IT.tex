\subsubsection{MIPS}

Una caratteristica distintiva di MIPS è l'assenza dei flag.
Apparentemente è una scelta fatta per semplificare l'analisi della dipendenza dai dati.

\myindex{x86!\Instructions!SETcc}
\myindex{MIPS!\Instructions!SLT}
\myindex{MIPS!\Instructions!SLTU}

Esistono istruzioni simili a \INS{SETcc} in x86: \INS{SLT} (\q{Set on Less Than}: versione signed) e 
\INS{SLTU} (versione unsigned).
Queste istruzioni settano il valore del registro di destinazione a 1 se la condizione è vera, a 0 se è falsa.

\myindex{MIPS!\Instructions!BEQ}
\myindex{MIPS!\Instructions!BNE}

Il registro di destinazione viene quindi controllato usando \INS{BEQ} (\q{Branch on Equal}) oppure \INS{BNE} (\q{Branch on Not Equal}) 
ed in base al caso si può verificare un salto.
Questa coppia di istruzioni è usata in MIPS per eseguire confronti e conseguenti branch.
Iniziamo con la versione signed della nostra funzione:

\lstinputlisting[caption=\NonOptimizing GCC 4.4.5 (IDA),style=customasmMIPS]{patterns/07_jcc/simple/O0_MIPS_signed_IDA_IT.lst}

\INS{SLT REG0, REG0, REG1} è stata ridotta da Ida nella sua forma breve:\\
\INS{SLT REG0, REG1}.
\myindex{MIPS!\Pseudoinstructions!BEQZ}

Notiamo anche la pseudo istruzione \INS{BEQZ} (\q{Branch if Equal to Zero}),\\
che è in realtà \INS{BEQ REG, \$ZERO, LABEL}.

\myindex{MIPS!\Instructions!SLTU}

La versione unsigned è uguale, \INS{SLTU} (versione unsigned, da cui la \q{U} nel nome) è usata al posto di \INS{SLT}:

\lstinputlisting[caption=\NonOptimizing GCC 4.4.5 (IDA),style=customasmMIPS]{patterns/07_jcc/simple/O0_MIPS_unsigned_IDA.lst}

