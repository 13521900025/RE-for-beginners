\subsubsection{x86}

\myparagraph{x86 + MSVC}

Voici à quoi ressemble la fonction  \TT{f\_signed()}:

\lstinputlisting[caption=MSVC 2010 \NonOptimizing,style=customasmx86]{patterns/07_jcc/simple/signed_MSVC.asm}

\myindex{x86!\Instructions!JLE}

La première instruction, \JLE, représente \emph{Jump if Less or Equal} (saut si inférieur ou égal).
En d'autres mots, si le deuxième opérande est plus grand ou égal au premier,
le flux d'exécution est passé à l'adresse ou au label spécifié dans l'instruction.
Si la condition ne déclenche pas le saut, car le second opérande est plus petit que
le premier, le flux d'exécution ne sera pas altéré et le premier \printf sera
exécuté.
\myindex{x86!\Instructions!JNE}
Le second test est \JNE: \emph{Jump if Not Equal} (saut si non égal).
Le flux d'exécution ne changera pas si les opérandes sont égaux.

\myindex{x86!\Instructions!JGE}
Le troisième test est \JGE: \emph{Jump if Greater or Equal}---saute si le premier
opérande est supérieur ou égal au deuxième.
Donc, si les trois sauts conditionnels sont effectués, aucun des appels à \printf
ne sera exécuté.
Ceci est impossible sans intervention spéciale.
Regardons maintenant la fonction \TT{f\_unsigned()}.
La fonction \TT{f\_unsigned()} est la même que \TT{f\_signed()}, à la différence
que les instructions \JBE et \JAE sont utilisées à la place de \JLE et \JGE, comme
suit:

\lstinputlisting[caption=GCC,style=customasmx86]{patterns/07_jcc/simple/unsigned_MSVC.asm}

\myindex{x86!\Instructions!JBE}
\myindex{x86!\Instructions!JAE}

Comme déjà mentionné, les instructions de branchement sont différentes:
\JBE---\emph{Jump if Below or Equal} (saut si inférieur ou égal) et \JAE---\emph{Jump if Above or Equal}
(saut si supérieur ou égal).
Ces instructions (\JA/\JAE/\JB/\JBE) diffèrent de \JG/\JGE/\JL/\JLE par le fait qu'elles
travaillent avec des nombres non signés.

\myindex{x86!\Instructions!JA}
\myindex{x86!\Instructions!JB}
\myindex{x86!\Instructions!JG}
\myindex{x86!\Instructions!JL}
\myindex{Signed numbers}

Voir aussi la section sur la représentation des nombres signés~(\myref{sec:signednumbers}).
C'est pourquoi si nous voyons que \JG/\JL sont utilisés à la place de \JA/\JB ou
vice-versa, nous pouvons être presque sûr que les variables sont signées ou non
signées, respectivement.
Voici la fonction \main, où presque rien n'est nouveau pour nous:

\lstinputlisting[caption=\main,style=customasmx86]{patterns/07_jcc/simple/main_MSVC.asm}

\clearpage
\myparagraph{MSVC + \olly}
\myindex{\olly}

2 paires de mots 32-bit sont marquées en rouge sur la pile.
Chaque paire est un double au format IEEE 754 et est passée depuis \main.

Nous voyons comment le premier \FLD charge une valeur ($1.2$) depuis la pile et la
stocke dans \ST{0}:

\begin{figure}[H]
\centering
\myincludegraphics{patterns/12_FPU/1_simple/olly1.png}
\caption{\olly: le premier \FLD a été exécuté}
\label{fig:FPU_simple_olly_1}
\end{figure}

À cause des inévitables erreurs de conversion depuis un nombre flottant 64-bit au
format IEEE 754 vers 80-bit (utilisé en interne par le FPU), ici nous voyons 1.199\ldots,
qui est proche de 1.2.

\EIP pointe maintenant sur l'instruction suivante (\FDIV), qui charge un double
(une constante) depuis la mémoire.
Par commodité, \olly affiche sa valeur: 3.14

\clearpage
Continuons l'exécution pas à pas.
\FDIV a été exécuté, maintenant \ST{0} contient 0.382\ldots
(\gls{quotient}):

\begin{figure}[H]
\centering
\myincludegraphics{patterns/12_FPU/1_simple/olly2.png}
\caption{\olly: \FDIV a été exécuté}
\label{fig:FPU_simple_olly_2}
\end{figure}

\clearpage
Troisième étape: le \FLD suivant a été exécuté, chargeant 3.4 dans \ST{0} (ici nous
voyons la valeur approximative 3.39999\ldots):

\begin{figure}[H]
\centering
\myincludegraphics{patterns/12_FPU/1_simple/olly3.png}
\caption{\olly: le second \FLD a été exécuté}
\label{fig:FPU_simple_olly_3}
\end{figure}

En même temps, le \gls{quotient} \emph{est poussé} dans \ST{1}.
Exactement maintenant, \EIP pointe sur la prochaine instruction: \FMUL.
Ceci charge la constante 4.1 depuis la mémoire, ce que montre \olly.

\clearpage
Suivante: \FMUL a été exécutée, donc maintenant le \glslink{product}{produit} est dans \ST{0}:

\begin{figure}[H]
\centering
\myincludegraphics{patterns/12_FPU/1_simple/olly4.png}
\caption{\olly: \FMUL a été exécuté}
\label{fig:FPU_simple_olly_4}
\end{figure}

\clearpage
Suivante: \FADDP a été exécutée, maintenant le résultat de l'addition est dans \ST{0},
et \ST{1} est vidé.

\begin{figure}[H]
\centering
\myincludegraphics{patterns/12_FPU/1_simple/olly5.png}
\caption{\olly: \FADDP a été exécuté}
\label{fig:FPU_simple_olly_5}
\end{figure}

Le résultat est laissé dans \ST{0}, car la fonction renvoie son résultat dans \ST{0}.

\main prend cette valeur depuis le registre plus loin.

Nous voyons quelque chose d'inhabituel: la valeur 13.93\ldots se trouve maintenant
dans \ST{7}.
Pourquoi?

\label{FPU_is_rather_circular_buffer}

Comme nous l'avons lu il y a quelque temps dans ce livre, les registres \ac{FPU} sont
une pile: \myref{FPU_is_stack}.
Mais ceci est une simplification.

Imaginez si cela était implémenté \emph{en hardware} comme cela est décrit, alors
tout le contenu des 7 registres devrait être déplacé (ou copié) dans les registres
adjacents lors d'un push ou d'un pop, et ceci nécessite beaucoup de travail.

En réalité, le \ac{FPU} a seulement 8 registres et un pointeur (appelé \GTT{TOP})
qui contient un numéro de registre, qui est le \q{haut de la pile} courant.

Lorsqu'une valeur est poussée sur la pile, \GTT{TOP} est déplacé sur le registre
disponible suivant, et une valeur est écrite dans ce registre.

La procédure est inversée si la valeur est lue, toutefois, le registre qui a été
libéré n'est pas vidé (il serait possible de le vider, mais ceci nécessite plus de
travail qui peut dégrader les performances).
Donc, c'est ce que nous voyons ici.

On peut dire que \FADDP sauve la somme sur la pile, et y supprime un élément.

Mais en fait, cette instruction sauve la somme et ensuite décale \GTT{TOP}.

Plus précisément, les registres du  \ac{FPU} sont un tampon circulaire.


\clearpage
\myparagraph{x86 + MSVC + Hiew}
\myindex{Hiew}

Nous pouvons essayer de patcher l'exécutable afin que la fonction \TT{f\_unsigned()}
affiche toujours \q{a==b}, quelque soient les valeurs en entrée.
Voici à quoi ça ressemble dans Hiew:

\begin{figure}[H]
\centering
\myincludegraphics{patterns/07_jcc/simple/hiew_unsigned1.png}
\caption{Hiew: fonction \TT{f\_unsigned()}}
\label{fig:jcc_hiew_1}
\end{figure}

Essentiellement, nous devons accomplir ces trois choses:
\begin{itemize}
\item forcer le premier saut à toujours être effectué;
\item forcer le second saut à n'être jamais effectué;
\item forcer le troisième saut à toujours être effectué.
\end{itemize}

Nous devons donc diriger le déroulement du code pour toujours effectuer le second \printf,
et afficher \q{a==b}.

Trois instructions (ou octets) doivent être modifiées:

\begin{itemize}
\item Le premier saut devient un \JMP, mais l'\glslink{jump offset}{offset} reste
le même.

\item
Le second saut peut être parfois suivi, mais dans chaque cas il sautera à l'instruction
suivante, car nous avons mis l'\glslink{jump offset}{offset} à 0.

Dans cette instruction, l'\glslink{jump offset}{offset} est ajouté à l'adresse
de l'instruction suivante. Donc si l'offset est 0, le saut va transférer l'exécution
à l'instruction suivante.

\item
Le troisième saut est remplacé par \JMP comme nous l'avons fait pour le premier,
il sera donc toujours effectué.

\end{itemize}

\clearpage
Voici le code modifié:

\begin{figure}[H]
\centering
\myincludegraphics{patterns/07_jcc/simple/hiew_unsigned2.png}
\caption{Hiew: modifions la fonction \TT{f\_unsigned()}}
\label{fig:jcc_hiew_2}
\end{figure}

Si nous oublions de modifier une de ces sauts conditionnels, plusieurs appels à \printf
pourraient être faits, alors que nous voulons qu'un seul soit exécuté.

\myparagraph{GCC \NonOptimizing}

\myindex{puts() instead of printf()}
GCC 4.4.1 \NonOptimizing produit presque le même code, mais avec \puts~(\myref{puts})
à la place de \printf.

\myparagraph{GCC \Optimizing}

Le lecteur attentif pourrait demander pourquoi exécuter \CMP plusieurs fois, si
les flags ont les mêmes valeurs après l'exécution ?

Peut-être que l'optimiseur de de MSVC ne peut pas faire cela, mais celui de GCC
4.8.1 peut aller plus loin:

\lstinputlisting[caption=GCC 4.8.1 f\_signed(),style=customasmx86]{patterns/07_jcc/simple/GCC_O3_signed.asm}

% should be here instead of 'switch' section?
Nous voyons ici \TT{JMP puts} au lieu de \TT{CALL puts / RETN}.

Ce genre de truc sera expliqué plus loin: \myref{JMP_instead_of_RET}.

Ce genre de code x86 est plutôt rare.
Il semble que MSVC 2012 ne puisse pas générer un tel code.
D'un autre côté, les programmeurs en langage d'assemblage sont parfaitement conscients
du fait que les instructions \TT{Jcc} peuvent être empilées.

Donc si vous voyez ce genre d'empilement, il est très probable que le code a été
écrit à la main.

La fonction \TT{f\_unsigned()} n'est pas si esthétiquement courte:

\lstinputlisting[caption=GCC 4.8.1 f\_unsigned(),style=customasmx86]{patterns/07_jcc/simple/GCC_O3_unsigned_FR.asm}

Néanmoins, il y a deux instructions \TT{CMP} au lieu de trois.

Donc les algorithmes d'optimisation de GCC 4.8.1 ne sont probablement pas encore parfaits.

