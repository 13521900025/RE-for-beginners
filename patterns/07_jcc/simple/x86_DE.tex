\subsubsection{x86}

\myparagraph{x86 + MSVC}

Die Funktions \TT{f\_signed()} sieht folgendermaßen aus:

\lstinputlisting[caption=\NonOptimizing MSVC 2010,style=customasmx86]{patterns/07_jcc/simple/signed_MSVC.asm}

\myindex{x86!\Instructions!JLE}
Der erste Befehl, \JLE steht für \emph{Jump if Less or Equal}.
Mit anderen Worten, wenn der zweite Operand größer gleich dem ersten ist, wird der Control Flow an die angegebene
Adresse bzw. das angegebene Label übergeben.
Wenn die Bedingung falsch ist, weil der zweite Operand kleiner ist als der erste, wird der Control Flow nicht verändert
und das erste \printf wird ausgeführt.

Tmyindex{x86!\Instructions!JNE}
Der zweite Check ist \JNE: \emph{Jump if Not Equal}.
Der Control Flow wird nicht verändert, wenn die Operanden gleich sind.

\myindex{x86!\Instructions!JGE}
Der dritte Check ist \JGE: \emph{Jump if Greater or Equal}---springe, falls der erste Operand größer gleich dem zweiten
ist.
Wenn also alle drei bedingten Sprünge ausgeführt werden, wird also kein Aufruf von \printf ausgeführt.
Dies ist ohne manuellen Eingriff unmöglich.
Werfen wir nun einen Blick auf die Funktion \TT{f\_unsigned()}.
Diese Funktion macht prinzipiell das gleiche wie \TT{f\_signed()} mit der Ausnahme, dass die Befehle \JBE und \JAE
anstelle von \JLE und \JGE wie folgt verwendet werden:

\lstinputlisting[caption=GCC,style=customasmx86]{patterns/07_jcc/simple/unsigned_MSVC.asm}

\myindex{x86!\Instructions!JBE}
\myindex{x86!\Instructions!JAE}
Wie bereits erwähnt unterscheiden sich die Verzweigungsbefehle:
\JBE---\emph{Jump if Below or Equal} und \JAE---\emph{Jump if Above or Equal}.
Diese Befehle (\JA/\JAE/\JB/\JBE) unterscheiden sich von \JG/\JGE/\JL/\JLE dadurch, dass sie mit vorzeichenlosen Zahlen
arbeiten.

\myindex{x86!\Instructions!JA}
\myindex{x86!\Instructions!JB}
\myindex{x86!\Instructions!JG}
\myindex{x86!\Instructions!JL}
\myindex{Signed numbers}
Siehe hierzu auch den Abschnitt über Darstellung vorzeichenbehafteter Zahlen~(\myref{sec:signednumbers}).
Das ist der Grund warum wir, wenn wir \JG/\JL anstelle von \JA/\JB und umgekehrt finden, fast mit Gewissheit sagen
können, dass die Variablen vorzeichenbehaftet bzw. vorzeichenlos sind.
Hier befindet sich auch die Funktion \main, welche für uns nichts Neues bereithält:

\lstinputlisting[caption=\main,style=customasmx86]{patterns/07_jcc/simple/main_MSVC.asm}

\clearpage
\subsubsection{MSVC: x86 + \olly}
\myindex{\olly}

Things are even simpler here:

\begin{figure}[H]
\centering
\myincludegraphics{patterns/04_scanf/2_global/ex2_olly_1.png}
\caption{\olly: after \scanf execution}
\label{fig:scanf_ex2_olly_1}
\end{figure}

The variable is located in the data segment.
After the \PUSH instruction (pushing the address of $x$) gets executed, 
the address appears in the stack window. Right-click on that row and select \q{Follow in dump}.
The variable will appear in the memory window on the left.
After we have entered 123 in the console, 
\TT{0x7B} appears in the memory window (see the highlighted screenshot regions).

But why is the first byte \TT{7B}?
Thinking logically, \TT{00 00 00 7B} must be there.
The cause for this is referred as  \gls{endianness}, and x86 uses \emph{little-endian}.
This implies that the lowest byte is written first, and the highest written last.
Read more about it at: \myref{sec:endianness}.
Back to the example, the 32-bit value is loaded from this memory address into \EAX and passed to \printf.

The memory address of $x$ is \TT{0x00C53394}.

\clearpage
In \olly we can review the process memory map (Alt-M)
and we can see that this address is inside the \TT{.data} PE-segment of our program:

\label{olly_memory_map_example}
\begin{figure}[H]
\centering
\myincludegraphics{patterns/04_scanf/2_global/ex2_olly_2.png}
\caption{\olly: process memory map}
\label{fig:scanf_ex2_olly_2}
\end{figure}



\clearpage
\myparagraph{x86 + MSVC + Hiew}
\myindex{Hiew}
Wir können versuchen, die Executable so zu verändern, dass die Funktion \TT{f\_unsigned()} stets \q{a==b} ausgibt, egal
was wir eingben.
So sieht das ganze in Hiew aus:
\begin{figure}[H]
\centering
\myincludegraphics{patterns/07_jcc/simple/hiew_unsigned1.png}
\caption{Hiew: Funktion \TT{f\_unsigned()}}
\label{fig:jcc_hiew_1}
\end{figure}
Grundsätzlich haben wir drei Dinge zu erzwingen:
\begin{itemize}
  \item den ersten Sprung stets ausführen;
  \item den zweiten Sprung nie ausführen;
  \item den dritten Sprung stets ausführen.
\end{itemize}

Dadurch können wir den Code Flow so manupulieren, dass das zweite \printf immer ausgeführt wird und \q{a==b} ausgibt.
Drei Befehle (oder Bytes) müssen verändert werden:
\begin{itemize}
\item Der erste Sprung wird zu \JMP verändert, aber der \gls{jump offset} bleibt gleich.

 
\item Der zweite Sprung könnte manchmal ausgeführt werden, wird aber in jedem Fall zum nächsten Befehl springen, denn
wir setzen den \gls{jump offset} auf 0.
Bei diesen Befehlen wird der \gls{jump offset} zu der Adresse der nächsten Befehls addiert. Wenn der Offset 0 ist, wird
die Ausführung also beim nächsten Befehl fortgesetzt.
\item 
Den dritten Sprung ersetzen wie genau wie den ersten durch \JMP, damit er stets ausgeführt wird.

\end{itemize}

\clearpage
Hier ist der veränderte Code:

\begin{figure}[H]
\centering
\myincludegraphics{patterns/07_jcc/simple/hiew_unsigned2.png}
\caption{Hiew: Veränderte Funktion \TT{f\_unsigned()}}
\label{fig:jcc_hiew_2}
\end{figure}
Wenn wir es verpassen, einen dieser Sprünge zu verändern, könnten mehrere Aufrufe von \printf ausgeführt werden; wir
wollen aber nur genau einen Aufruf ausführen.

\myparagraph{\NonOptimizing GCC}

\myindex{puts() anstelle von printf()}
\NonOptimizing GCC 4.4.1 
erzeugt fast identischen Code, aber mit \puts~(\myref{puts}) anstelle von \printf.

\myparagraph{\Optimizing GCC}
Der aufmerksame Leser fragt sich vielleicht, warum \CMP mehrmals ausgeführt wird, wenn doch die Flags nach jeder
Ausführung dieselben Werte haben. 

Vielleicht kann der optimierende MSVC dies nicht leisten, aber der optimierende GCC 4.8.1 gräbt tiefer:

\lstinputlisting[caption=GCC 4.8.1 f\_signed(),style=customasmx86]{patterns/07_jcc/simple/GCC_O3_signed.asm}

% should be here instead of 'switch' section?
Wir finden auch hier \TT{JMP puts} anstelle von \TT{CALL puts / RETN}.

Dieser Trick wird später erklärt:\myref{JMP_instead_of_RET}.

Diese Sorte x86 Code ist trotzdem selten. MSVC 2012 kann wie es scheint solchen Code nicht erzeugen.
Andererseits sind Assemblerprogrammierer sich natürlich der Tatsache bewusst, dass \TT{Jcc} Befehle gestackt werden
können.
Wenn man solche gestackten Befehle findet, ist es sehr wahrscheinlich, dass der entsprechende Code von Hand geschrieben
wurde. 
Die Funktion \TT{f\_unsigned()} ist nicht so ästhetisch:


\lstinputlisting[caption=GCC 4.8.1 f\_unsigned(),style=customasmx86]{patterns/07_jcc/simple/GCC_O3_unsigned_DE.asm}
Trotzdem werden hier immerhin nur zwei statt drei \TT{CMP} Befehle verwendet.

Die Optimierungsalgorithmen von GCC 4.8.1 sind möglicherweise noch nicht so ausgereift.
