\mysection{Jump condizionali}
\label{sec:Jcc}
\myindex{\CLanguageElements!if}

% sections
\subsection{\RU{Простой пример}\EN{Simple example}\DE{einfaches Beispiel}
\FR{Exemple simple}\IT{Esempio semplice}\JA{シンプルな例}
}

\lstinputlisting[style=customc]{patterns/07_jcc/simple/ex.c}

% subsections
\EN{\subsubsection{x86}

\myparagraph{x86 + MSVC}

Here is how the \TT{f\_signed()} function looks like:

\lstinputlisting[caption=\NonOptimizing MSVC 2010,style=customasmx86]{patterns/07_jcc/simple/signed_MSVC.asm}

\myindex{x86!\Instructions!JLE}

The first instruction, \JLE, stands for \emph{Jump if Less or Equal}. 
In other words, if the second operand is 
larger or equal to the first one, the control flow will be passed to the address or label specified in the instruction.
If this condition does not trigger because the second operand is smaller than the first one, the control flow would not be altered and the first \printf would be executed.
\myindex{x86!\Instructions!JNE}
The second check is \JNE: \emph{Jump if Not Equal}.
The control flow will not change if the operands are equal.

\myindex{x86!\Instructions!JGE}
The third check is \JGE: \emph{Jump if Greater or Equal}---jump if the first operand is larger than 
the second or if they are equal.
So, if all three conditional jumps are triggered, none of the \printf calls would be executed whatsoever. 
This is impossible without special intervention.
Now let's take a look at the \TT{f\_unsigned()} function.
The \TT{f\_unsigned()} function is the same as \TT{f\_signed()}, with the exception that the \JBE and \JAE instructions
are used instead of \JLE and \JGE, as follows:

\lstinputlisting[caption=GCC,style=customasmx86]{patterns/07_jcc/simple/unsigned_MSVC.asm}

\myindex{x86!\Instructions!JBE}
\myindex{x86!\Instructions!JAE}

As already mentioned, the branch instructions are different:
\JBE---\emph{Jump if Below or Equal} and \JAE---\emph{Jump if Above or Equal}.
These instructions (\INS{JA}/\JAE/\JB/\JBE) differ from \JG/\JGE/\JL/\JLE in the fact that they work with unsigned numbers.

\myindex{x86!\Instructions!JA}
\myindex{x86!\Instructions!JB}
\myindex{x86!\Instructions!JG}
\myindex{x86!\Instructions!JL}
\myindex{Signed numbers}

See also the section about signed number representations~(\myref{sec:signednumbers}).
That is why if we see \JG/\JL in use instead of \INS{JA}/\JB or vice-versa, 
we can be almost sure that the variables are signed or unsigned, respectively.
Here is also the \main function, where there is nothing much new to us:

\lstinputlisting[caption=\main,style=customasmx86]{patterns/07_jcc/simple/main_MSVC.asm}

\input{patterns/07_jcc/simple/olly_EN.tex}

\clearpage
\myparagraph{x86 + MSVC + Hiew}
\myindex{Hiew}

We can try to patch the executable file in a way 
that the \TT{f\_unsigned()} function would always print \q{a==b}, 
no matter the input values.
Here is how it looks in Hiew:

\begin{figure}[H]
\centering
\myincludegraphics{patterns/07_jcc/simple/hiew_unsigned1.png}
\caption{Hiew: \TT{f\_unsigned()} function}
\label{fig:jcc_hiew_1}
\end{figure}

Essentially, we have to accomplish three tasks:
\begin{itemize}
\item force the first jump to always trigger;
\item force the second jump to never trigger;
\item force the third jump to always trigger.
\end{itemize}

Thus we can direct the code flow to always pass through the second \printf, and output \q{a==b}.

Three instructions (or bytes) has to be patched:

\begin{itemize}
\item The first jump becomes \JMP, but the \gls{jump offset} would remain the same.

\item 
The second jump might be triggered sometimes, but in any case it will jump to the next
instruction, because, we set the \gls{jump offset} to 0.

In these instructions the \gls{jump offset} is added to the address for the next instruction.
So if the offset is 0,
the jump will transfer the control to the next instruction.

\item 
The third jump we replace with \JMP just as we do with the first one, so it will always trigger.

\end{itemize}

\clearpage
Here is the modified code:

\begin{figure}[H]
\centering
\myincludegraphics{patterns/07_jcc/simple/hiew_unsigned2.png}
\caption{Hiew: let's modify the \TT{f\_unsigned()} function}
\label{fig:jcc_hiew_2}
\end{figure}

If we miss to change any of these jumps, then several \printf calls may execute, while we want to execute only one.

\myparagraph{\NonOptimizing GCC}

\myindex{puts() instead of printf()}
\NonOptimizing GCC 4.4.1 
produces almost the same code, but with \puts~(\myref{puts}) instead of \printf.

\myparagraph{\Optimizing GCC}

An observant reader may ask, why execute \CMP several times, 
if the flags has the same values after each execution?

Perhaps optimizing MSVC cannot do this, but optimizing GCC 4.8.1 can go deeper:

\lstinputlisting[caption=GCC 4.8.1 f\_signed(),style=customasmx86]{patterns/07_jcc/simple/GCC_O3_signed.asm}

% should be here instead of 'switch' section?
We also see \TT{JMP puts} here instead of \TT{CALL puts / RETN}.

This kind of trick will have explained later: \myref{JMP_instead_of_RET}.

This type of x86 code 
is somewhat rare.
MSVC 2012 as it seems, can't generate such code.
On the other hand, assembly language programmers are fully aware of the fact that \TT{Jcc} 
instructions can be stacked.

So if you see such stacking somewhere, it is highly probable that the code was hand-written.

The \TT{f\_unsigned()} function is not that 
\ae{}sthetically short:

\lstinputlisting[caption=GCC 4.8.1 f\_unsigned(),style=customasmx86]{patterns/07_jcc/simple/GCC_O3_unsigned_EN.asm}

Nevertheless, there are two \TT{CMP} instructions instead of three.

So optimization algorithms of GCC 4.8.1 are probably not perfect yet. 
}
\RU{\subsubsection{x86}

\myparagraph{x86 + MSVC}

Имеем в итоге функцию \TT{f\_signed()}:

\lstinputlisting[caption=\NonOptimizing MSVC 2010,style=customasmx86]{patterns/07_jcc/simple/signed_MSVC.asm}

\myindex{x86!\Instructions!JLE}
Первая инструкция \JLE значит \emph{Jump if Less or Equal}. 
Если второй операнд больше первого или равен ему, произойдет переход туда, где будет следующая проверка.

А если это условие не срабатывает (то есть второй операнд меньше первого), то перехода не будет, 
и сработает первый \printf.

\myindex{x86!\Instructions!JNE}
Вторая проверка это \JNE: \emph{Jump if Not Equal}.
Переход не произойдет, если операнды равны.

\myindex{x86!\Instructions!JGE}
Третья проверка \JGE: \emph{Jump if Greater or Equal} --- переход 
если первый операнд больше второго или равен ему.
Кстати, если все три условных перехода сработают, ни один \printf не вызовется. 
Но без внешнего вмешательства это невозможно.

Функция \TT{f\_unsigned()} точно такая же, за тем исключением, что используются инструкции 
\JBE и \JAE вместо \JLE и \JGE:

\lstinputlisting[caption=GCC,style=customasmx86]{patterns/07_jcc/simple/unsigned_MSVC.asm}

\myindex{x86!\Instructions!JBE}
\myindex{x86!\Instructions!JAE}
Здесь всё то же самое, только инструкции условных переходов немного другие:

\JBE --- \emph{Jump if Below or Equal} и \JAE --- \emph{Jump if Above or Equal}.
Эти инструкции (\JA/\JAE/\JB/\JBE) 
отличаются от \JG/\JGE/\JL/\JLE тем, что работают с беззнаковыми переменными.

\myindex{x86!\Instructions!JA}
\myindex{x86!\Instructions!JB}
\myindex{x86!\Instructions!JG}
\myindex{x86!\Instructions!JL}
\myindex{Signed numbers}
Отступление: смотрите также секцию о представлении знака в числах ~(\myref{sec:signednumbers}).
Таким образом, увидев где используется \JG/\JL вместо \JA/\JB и наоборот, 
можно сказать почти уверенно насчет того, 
является ли тип переменной знаковым (signed) или беззнаковым (unsigned).

Далее функция \main, где ничего нового для нас нет:

\lstinputlisting[caption=\main,style=customasmx86]{patterns/07_jcc/simple/main_MSVC.asm}

\input{patterns/07_jcc/simple/olly_RU.tex}

\clearpage
\myparagraph{x86 + MSVC + Hiew}
\myindex{Hiew}

Можем попробовать модифицировать исполняемый файл так, чтобы функция \TT{f\_unsigned()} всегда показывала \q{a==b},
при любых входящих значениях.
Вот как она выглядит в Hiew:

\begin{figure}[H]
\centering
\myincludegraphics{patterns/07_jcc/simple/hiew_unsigned1.png}
\caption{Hiew: функция \TT{f\_unsigned()}}
\label{fig:jcc_hiew_1}
\end{figure}

Собственно, задач три:
\begin{itemize}
\item заставить первый переход срабатывать всегда;
\item заставить второй переход не срабатывать никогда;
\item заставить третий переход срабатывать всегда.
\end{itemize}

Так мы направим путь исполнения кода (code flow) во второй \printf,
и он всегда будет срабатывать и выводить на консоль \q{a==b}.

Для этого нужно изменить три инструкции (или байта):

\begin{itemize}
\item Первый переход теперь будет \JMP, но смещение перехода 
(\gls{jump offset}) останется прежним.

\item Второй переход может быть и будет срабатывать иногда, но в любом случае он будет совершать переход
только на следующую инструкцию, потому что мы выставляем смещение перехода (\gls{jump offset}) в 0.

В этих инструкциях смещение перехода просто прибавляется к адресу следующей инструкции.

Когда смещение 0, переход будет на следующую инструкцию.

\item Третий переход конвертируем в \JMP точно так же, как и первый, он будет срабатывать всегда.

\end{itemize}

\clearpage
Что и делаем:

\begin{figure}[H]
\centering
\myincludegraphics{patterns/07_jcc/simple/hiew_unsigned2.png}
\caption{Hiew: модифицируем функцию \TT{f\_unsigned()}}
\label{fig:jcc_hiew_2}
\end{figure}

Если забыть про какой-то из переходов, то тогда будет срабатывать несколько вызовов \printf, 
а нам ведь нужно чтобы исполнялся только один.

\myparagraph{\NonOptimizing GCC}

\myindex{puts() вместо printf()}
\NonOptimizing GCC 4.4.1 производит почти такой же код, за исключением \puts~(\myref{puts}) вместо \printf.

\myparagraph{\Optimizing GCC}

Наблюдательный читатель может спросить, зачем исполнять \CMP так много раз,
если флаги всегда одни и те же?
По-видимому, оптимизирующий MSVC не может этого делать, но GCC 4.8.1 делает больше оптимизаций:

\lstinputlisting[caption=GCC 4.8.1 f\_signed(),style=customasmx86]{patterns/07_jcc/simple/GCC_O3_signed.asm}

% should be here instead of 'switch' section?
Мы здесь также видим \TT{JMP puts} вместо \TT{CALL puts / RETN}.
Этот прием описан немного позже: \myref{JMP_instead_of_RET}.

Нужно сказать, что x86-код такого типа редок.
MSVC 2012, как видно, не может генерировать подобное.
С другой стороны, программисты на ассемблере прекрасно осведомлены о том,
что инструкции \TT{Jcc} можно располагать последовательно.

Так что если вы видите это где-то, имеется немалая вероятность, что этот фрагмент кода был написан вручную.

Функция \TT{f\_unsigned()} получилась не настолько эстетически короткой:

\lstinputlisting[caption=GCC 4.8.1 f\_unsigned(),style=customasmx86]{patterns/07_jcc/simple/GCC_O3_unsigned_RU.asm}

Тем не менее, здесь 2 инструкции \TT{CMP} вместо трех.

Так что, алгоритмы оптимизации GCC 4.8.1, наверное, ещё пока не идеальны.
}
\DE{\subsubsection{x86}

\myparagraph{x86 + MSVC}

Die Funktions \TT{f\_signed()} sieht folgendermaßen aus:

\lstinputlisting[caption=\NonOptimizing MSVC 2010,style=customasmx86]{patterns/07_jcc/simple/signed_MSVC.asm}

\myindex{x86!\Instructions!JLE}
Der erste Befehl, \JLE steht für \emph{Jump if Less or Equal}.
Mit anderen Worten, wenn der zweite Operand größer gleich dem ersten ist, wird der Control Flow an die angegebene
Adresse bzw. das angegebene Label übergeben.
Wenn die Bedingung falsch ist, weil der zweite Operand kleiner ist als der erste, wird der Control Flow nicht verändert
und das erste \printf wird ausgeführt.

Tmyindex{x86!\Instructions!JNE}
Der zweite Check ist \JNE: \emph{Jump if Not Equal}.
Der Control Flow wird nicht verändert, wenn die Operanden gleich sind.

\myindex{x86!\Instructions!JGE}
Der dritte Check ist \JGE: \emph{Jump if Greater or Equal}---springe, falls der erste Operand größer gleich dem zweiten
ist.
Wenn also alle drei bedingten Sprünge ausgeführt werden, wird also kein Aufruf von \printf ausgeführt.
Dies ist ohne manuellen Eingriff unmöglich.
Werfen wir nun einen Blick auf die Funktion \TT{f\_unsigned()}.
Diese Funktion macht prinzipiell das gleiche wie \TT{f\_signed()} mit der Ausnahme, dass die Befehle \JBE und \JAE
anstelle von \JLE und \JGE wie folgt verwendet werden:

\lstinputlisting[caption=GCC,style=customasmx86]{patterns/07_jcc/simple/unsigned_MSVC.asm}

\myindex{x86!\Instructions!JBE}
\myindex{x86!\Instructions!JAE}
Wie bereits erwähnt unterscheiden sich die Verzweigungsbefehle:
\JBE---\emph{Jump if Below or Equal} und \JAE---\emph{Jump if Above or Equal}.
Diese Befehle (\JA/\JAE/\JB/\JBE) unterscheiden sich von \JG/\JGE/\JL/\JLE dadurch, dass sie mit vorzeichenlosen Zahlen
arbeiten.

\myindex{x86!\Instructions!JA}
\myindex{x86!\Instructions!JB}
\myindex{x86!\Instructions!JG}
\myindex{x86!\Instructions!JL}
\myindex{Signed numbers}
Siehe hierzu auch den Abschnitt über Darstellung vorzeichenbehafteter Zahlen~(\myref{sec:signednumbers}).
Das ist der Grund warum wir, wenn wir \JG/\JL anstelle von \JA/\JB und umgekehrt finden, fast mit Gewissheit sagen
können, dass die Variablen vorzeichenbehaftet bzw. vorzeichenlos sind.
Hier befindet sich auch die Funktion \main, welche für uns nichts Neues bereithält:

\lstinputlisting[caption=\main,style=customasmx86]{patterns/07_jcc/simple/main_MSVC.asm}

\input{patterns/07_jcc/simple/olly_EN.tex}

\clearpage
\myparagraph{x86 + MSVC + Hiew}
\myindex{Hiew}
Wir können versuchen, die Executable so zu verändern, dass die Funktion \TT{f\_unsigned()} stets \q{a==b} ausgibt, egal
was wir eingben.
So sieht das ganze in Hiew aus:
\begin{figure}[H]
\centering
\myincludegraphics{patterns/07_jcc/simple/hiew_unsigned1.png}
\caption{Hiew: Funktion \TT{f\_unsigned()}}
\label{fig:jcc_hiew_1}
\end{figure}
Grundsätzlich haben wir drei Dinge zu erzwingen:
\begin{itemize}
  \item den ersten Sprung stets ausführen;
  \item den zweiten Sprung nie ausführen;
  \item den dritten Sprung stets ausführen.
\end{itemize}

Dadurch können wir den Code Flow so manupulieren, dass das zweite \printf immer ausgeführt wird und \q{a==b} ausgibt.
Drei Befehle (oder Bytes) müssen verändert werden:
\begin{itemize}
\item Der erste Sprung wird zu \JMP verändert, aber der \gls{jump offset} bleibt gleich.

 
\item Der zweite Sprung könnte manchmal ausgeführt werden, wird aber in jedem Fall zum nächsten Befehl springen, denn
wir setzen den \gls{jump offset} auf 0.
Bei diesen Befehlen wird der \gls{jump offset} zu der Adresse der nächsten Befehls addiert. Wenn der Offset 0 ist, wird
die Ausführung also beim nächsten Befehl fortgesetzt.
\item 
Den dritten Sprung ersetzen wie genau wie den ersten durch \JMP, damit er stets ausgeführt wird.

\end{itemize}

\clearpage
Hier ist der veränderte Code:

\begin{figure}[H]
\centering
\myincludegraphics{patterns/07_jcc/simple/hiew_unsigned2.png}
\caption{Hiew: Veränderte Funktion \TT{f\_unsigned()}}
\label{fig:jcc_hiew_2}
\end{figure}
Wenn wir es verpassen, einen dieser Sprünge zu verändern, könnten mehrere Aufrufe von \printf ausgeführt werden; wir
wollen aber nur genau einen Aufruf ausführen.

\myparagraph{\NonOptimizing GCC}

\myindex{puts() anstelle von printf()}
\NonOptimizing GCC 4.4.1 
erzeugt fast identischen Code, aber mit \puts~(\myref{puts}) anstelle von \printf.

\myparagraph{\Optimizing GCC}
Der aufmerksame Leser fragt sich vielleicht, warum \CMP mehrmals ausgeführt wird, wenn doch die Flags nach jeder
Ausführung dieselben Werte haben. 

Vielleicht kann der optimierende MSVC dies nicht leisten, aber der optimierende GCC 4.8.1 gräbt tiefer:

\lstinputlisting[caption=GCC 4.8.1 f\_signed(),style=customasmx86]{patterns/07_jcc/simple/GCC_O3_signed.asm}

% should be here instead of 'switch' section?
Wir finden auch hier \TT{JMP puts} anstelle von \TT{CALL puts / RETN}.

Dieser Trick wird später erklärt:\myref{JMP_instead_of_RET}.

Diese Sorte x86 Code ist trotzdem selten. MSVC 2012 kann wie es scheint solchen Code nicht erzeugen.
Andererseits sind Assemblerprogrammierer sich natürlich der Tatsache bewusst, dass \TT{Jcc} Befehle gestackt werden
können.
Wenn man solche gestackten Befehle findet, ist es sehr wahrscheinlich, dass der entsprechende Code von Hand geschrieben
wurde. 
Die Funktion \TT{f\_unsigned()} ist nicht so ästhetisch:


\lstinputlisting[caption=GCC 4.8.1 f\_unsigned(),style=customasmx86]{patterns/07_jcc/simple/GCC_O3_unsigned_DE.asm}
Trotzdem werden hier immerhin nur zwei statt drei \TT{CMP} Befehle verwendet.

Die Optimierungsalgorithmen von GCC 4.8.1 sind möglicherweise noch nicht so ausgereift.
}
\FR{\subsubsection{x86}

\myparagraph{x86 + MSVC}

Voici à quoi ressemble la fonction  \TT{f\_signed()}:

\lstinputlisting[caption=MSVC 2010 \NonOptimizing,style=customasmx86]{patterns/07_jcc/simple/signed_MSVC.asm}

\myindex{x86!\Instructions!JLE}

La première instruction, \JLE, représente \emph{Jump if Less or Equal} (saut si inférieur ou égal).
En d'autres mots, si le deuxième opérande est plus grand ou égal au premier,
le flux d'exécution est passé à l'adresse ou au label spécifié dans l'instruction.
Si la condition ne déclenche pas le saut, car le second opérande est plus petit que
le premier, le flux d'exécution ne sera pas altéré et le premier \printf sera
exécuté.
\myindex{x86!\Instructions!JNE}
Le second test est \JNE: \emph{Jump if Not Equal} (saut si non égal).
Le flux d'exécution ne changera pas si les opérandes sont égaux.

\myindex{x86!\Instructions!JGE}
Le troisième test est \JGE: \emph{Jump if Greater or Equal}---saute si le premier
opérande est supérieur ou égal au deuxième.
Donc, si les trois sauts conditionnels sont effectués, aucun des appels à \printf
ne sera exécuté.
Ceci est impossible sans intervention spéciale.
Regardons maintenant la fonction \TT{f\_unsigned()}.
La fonction \TT{f\_unsigned()} est la même que \TT{f\_signed()}, à la différence
que les instructions \JBE et \JAE sont utilisées à la place de \JLE et \JGE, comme
suit:

\lstinputlisting[caption=GCC,style=customasmx86]{patterns/07_jcc/simple/unsigned_MSVC.asm}

\myindex{x86!\Instructions!JBE}
\myindex{x86!\Instructions!JAE}

Comme déjà mentionné, les instructions de branchement sont différentes:
\JBE---\emph{Jump if Below or Equal} (saut si inférieur ou égal) et \JAE---\emph{Jump if Above or Equal}
(saut si supérieur ou égal).
Ces instructions (\JA/\JAE/\JB/\JBE) diffèrent de \JG/\JGE/\JL/\JLE par le fait qu'elles
travaillent avec des nombres non signés.

\myindex{x86!\Instructions!JA}
\myindex{x86!\Instructions!JB}
\myindex{x86!\Instructions!JG}
\myindex{x86!\Instructions!JL}
\myindex{Signed numbers}

Voir aussi la section sur la représentation des nombres signés~(\myref{sec:signednumbers}).
C'est pourquoi si nous voyons que \JG/\JL sont utilisés à la place de \JA/\JB ou
vice-versa, nous pouvons être presque sûr que les variables sont signées ou non
signées, respectivement.
Voici la fonction \main, où presque rien n'est nouveau pour nous:

\lstinputlisting[caption=\main,style=customasmx86]{patterns/07_jcc/simple/main_MSVC.asm}

\input{patterns/07_jcc/simple/olly_FR.tex}

\clearpage
\myparagraph{x86 + MSVC + Hiew}
\myindex{Hiew}

Nous pouvons essayer de patcher l'exécutable afin que la fonction \TT{f\_unsigned()}
affiche toujours \q{a==b}, quelque soient les valeurs en entrée.
Voici à quoi ça ressemble dans Hiew:

\begin{figure}[H]
\centering
\myincludegraphics{patterns/07_jcc/simple/hiew_unsigned1.png}
\caption{Hiew: fonction \TT{f\_unsigned()}}
\label{fig:jcc_hiew_1}
\end{figure}

Essentiellement, nous devons accomplir ces trois choses:
\begin{itemize}
\item forcer le premier saut à toujours être effectué;
\item forcer le second saut à n'être jamais effectué;
\item forcer le troisième saut à toujours être effectué.
\end{itemize}

Nous devons donc diriger le déroulement du code pour toujours effectuer le second \printf,
et afficher \q{a==b}.

Trois instructions (ou octets) doivent être modifiées:

\begin{itemize}
\item Le premier saut devient un \JMP, mais l'\glslink{jump offset}{offset} reste
le même.

\item
Le second saut peut être parfois suivi, mais dans chaque cas il sautera à l'instruction
suivante, car nous avons mis l'\glslink{jump offset}{offset} à 0.

Dans cette instruction, l'\glslink{jump offset}{offset} est ajouté à l'adresse
de l'instruction suivante. Donc si l'offset est 0, le saut va transférer l'exécution
à l'instruction suivante.

\item
Le troisième saut est remplacé par \JMP comme nous l'avons fait pour le premier,
il sera donc toujours effectué.

\end{itemize}

\clearpage
Voici le code modifié:

\begin{figure}[H]
\centering
\myincludegraphics{patterns/07_jcc/simple/hiew_unsigned2.png}
\caption{Hiew: modifions la fonction \TT{f\_unsigned()}}
\label{fig:jcc_hiew_2}
\end{figure}

Si nous oublions de modifier une de ces sauts conditionnels, plusieurs appels à \printf
pourraient être faits, alors que nous voulons qu'un seul soit exécuté.

\myparagraph{GCC \NonOptimizing}

\myindex{puts() instead of printf()}
GCC 4.4.1 \NonOptimizing produit presque le même code, mais avec \puts~(\myref{puts})
à la place de \printf.

\myparagraph{GCC \Optimizing}

Le lecteur attentif pourrait demander pourquoi exécuter \CMP plusieurs fois, si
les flags ont les mêmes valeurs après l'exécution ?

Peut-être que l'optimiseur de de MSVC ne peut pas faire cela, mais celui de GCC
4.8.1 peut aller plus loin:

\lstinputlisting[caption=GCC 4.8.1 f\_signed(),style=customasmx86]{patterns/07_jcc/simple/GCC_O3_signed.asm}

% should be here instead of 'switch' section?
Nous voyons ici \TT{JMP puts} au lieu de \TT{CALL puts / RETN}.

Ce genre de truc sera expliqué plus loin: \myref{JMP_instead_of_RET}.

Ce genre de code x86 est plutôt rare.
Il semble que MSVC 2012 ne puisse pas générer un tel code.
D'un autre côté, les programmeurs en langage d'assemblage sont parfaitement conscients
du fait que les instructions \TT{Jcc} peuvent être empilées.

Donc si vous voyez ce genre d'empilement, il est très probable que le code a été
écrit à la main.

La fonction \TT{f\_unsigned()} n'est pas si esthétiquement courte:

\lstinputlisting[caption=GCC 4.8.1 f\_unsigned(),style=customasmx86]{patterns/07_jcc/simple/GCC_O3_unsigned_FR.asm}

Néanmoins, il y a deux instructions \TT{CMP} au lieu de trois.

Donc les algorithmes d'optimisation de GCC 4.8.1 ne sont probablement pas encore parfaits.

}
\IT{\subsubsection{x86}

\myparagraph{x86 + MSVC}

La funzione \TT{f\_signed()} appare così:

\lstinputlisting[caption=\NonOptimizing MSVC 2010,style=customasmx86]{patterns/07_jcc/simple/signed_MSVC.asm}

\myindex{x86!\Instructions!JLE}

La prima istruzione, \JLE, sta per \emph{Jump if Less or Equal} (\emph{salta se è minore o uguale}). 
In altre parole, se il secondo operando è
maggiore o uguale al primo, il flusso di controllo sarà pasato all'indirizzo o alla label specificata nell'istruzione. 
Se questa condizione non è soddisfatta, poiché il secondo operando è più piccolo del primo, il flusso non viene alterato e la prima \printf sarà eseguita.

\myindex{x86!\Instructions!JNE}
Il secondo controllo è \JNE: \emph{Jump if Not Equal}.
Il flusso non cambia se i due operandi sono uguali.

\myindex{x86!\Instructions!JGE}
Il terzo controllo è \JGE: \emph{Jump if Greater or Equal}---salta se il primo operando è maggiore del secondo, o se sono uguali.

Quindi, se tutti i tre salti condizionali vengono innescati, nessuna delle chiamate a \printf sarà eseguita.
Ciò è chiaramente impossibile, almeno senza un intervento speciale.

Diamo ora un'occhiata alla funzione \TT{f\_unsigned()}.
La funzione \TT{f\_unsigned()} è uguale a \TT{f\_signed()}, con l'eccezione che le istruzioni \JBE e \JAE
sono utilizzate al posto di \JLE e \JGE:

\lstinputlisting[caption=GCC,style=customasmx86]{patterns/07_jcc/simple/unsigned_MSVC.asm}

\myindex{x86!\Instructions!JBE}
\myindex{x86!\Instructions!JAE}

Come già detto, le istruzioni di salto (branch instructions) sono diverse:
\JBE---\emph{Jump if Below or Equal} e \JAE---\emph{Jump if Above or Equal}.
Queste istruzioni (\JA/\JAE/\JB/\JBE) differiscono da \JG/\JGE/\JL/\JLE in quanto operano con numeri senza segno (unsigned).

\myindex{x86!\Instructions!JA}
\myindex{x86!\Instructions!JB}
\myindex{x86!\Instructions!JG}
\myindex{x86!\Instructions!JL}
\myindex{Signed numbers}

Vedi anche la sezione sulle rappresentazioni di numeri con segno (signed) ~(\myref{sec:signednumbers}).

Questo è il motivo per cui se vediamo usare \JG/\JL al posto di \JA/\JB, o viceversa, 
possiamo essere quasi certi che le variabili sono rispettivamente di tipo signed o unsigned.

Di seguito è riportata anche la funzione \main, dove non c'è niente di nuovo:

\lstinputlisting[caption=\main,style=customasmx86]{patterns/07_jcc/simple/main_MSVC.asm}

\input{patterns/07_jcc/simple/olly_IT}

\clearpage
\myparagraph{x86 + MSVC + Hiew}
\myindex{Hiew}

Possiamo provare a patchare l'eseguibile (applicare una patch) in maniera tale che la funzione \TT{f\_unsigned()} stampi sempre \q{a==b}, 
a prescindere dai valori in input.
In Hiew appare così:

\begin{figure}[H]
\centering
\myincludegraphics{patterns/07_jcc/simple/hiew_unsigned1.png}
\caption{Hiew: \TT{f\_unsigned()} function}
\label{fig:jcc_hiew_1}
\end{figure}

Essenzialmente, per ottenere il risultato desiderato, dobbiamo:
\begin{itemize}
\item forzare il primo jump in modo che sia sempre seguito;
\item forzare il secondo jump a non essere mai seguito;
\item forzare il terzo jump ad essere sempre seguito.
\end{itemize}

Possiamo così diriggere il flusso di esecuzione in modo tale da farlo sempre passare attraverso la seconda \printf, dando in output \q{a==b}.

Devono essere corrette (patchate) tre istruzioni (o byte):

\begin{itemize}
\item Il primo jump diventa \JMP, ma il \gls{jump offset} resta invariato.

\item 
Il secondo jump potrebbe essere innescato in alcune occasioni, ma in ogni caso salterebbe alla prossima istruzione, poiché settiamo il \gls{jump offset} a 0.

In queste istruzioni il \gls{jump offset} viene sommato all'indirizzo della prossima istruzione.
Quindi se l'offset è 0, il jump trasferirà il controllo all'istruzione successiva,

\item 
Possiamo sostituire il terzo jump con \JMP allo stesso modo del primo, in modo che sia sempre innescato.

\end{itemize}

\clearpage
Ecco il codice modificato:

\begin{figure}[H]
\centering
\myincludegraphics{patterns/07_jcc/simple/hiew_unsigned2.png}
\caption{Hiew: funzione \TT{f\_unsigned()} modificata}
\label{fig:jcc_hiew_2}
\end{figure}

Se ci dimentichiamo di cambiare uno di questi jump, potrebbero essere eseguite diverse chiamate a \printf, ma noi vogliamo eseguirne solo una.

\myparagraph{\NonOptimizing GCC}

\myindex{puts() instead of printf()}
\NonOptimizing GCC 4.4.1 
procude pressoché lo stesso codice, ma usa \puts~(\myref{puts}) invece di \printf.

\myparagraph{\Optimizing GCC}

Un lettore attento potrebbe domandare: perchè eseguire \CMP più volte se i flag hanno gli stessi valori dopo ogni esecuzione?

Forse MSVC con ottimizzazioni non è in grado di applicare questa ottimizzazione, al contrario di GCC 4.8.1:

\lstinputlisting[caption=GCC 4.8.1 f\_signed(),style=customasmx86]{patterns/07_jcc/simple/GCC_O3_signed.asm}

% should be here instead of 'switch' section?
Notiamo anche l'uso di \TT{JMP puts} al posto di \TT{CALL puts / RETN}.
Questo trucco sarà spiegato più avanti: \myref{JMP_instead_of_RET}.

Questo tipo di codice x86 è piuttosto raro.
MSVC 2012 apparentemente non è in grado di generarne di simile.
Dall'altro lato, i programmatori assembly sanno perfettamente che le istruzioni \TT{Jcc} possono essere disposte in fila.

Se vedete codice con una disposizione simile, è molto probabile che sia stato scritto a mano.

La funzione \TT{f\_unsigned()} non è esteticamente corta allo stesso modo:

\lstinputlisting[caption=GCC 4.8.1 f\_unsigned(),style=customasmx86]{patterns/07_jcc/simple/GCC_O3_unsigned_EN.asm}

Ciò nonostante, ci sono due istruzioni \TT{CMP} invece di tre.
Gli algoritmi di ottimizzazione di GCC 4.8.1 probabilmente non sono ancora perfetti. 
}
\JPN{\subsubsection{x86}

\myparagraph{x86 + MSVC}

以下は、\TT{f\_signed()} 関数がどうなっているかを示しています。

\lstinputlisting[caption=\NonOptimizing MSVC 2010,style=customasmx86]{patterns/07_jcc/simple/signed_MSVC.asm}

\myindex{x86!\Instructions!JLE}

最初の命令 \JLE は、\emph{Jump if Less or Equal}の場合はJumpを表します。 
言い換えれば、第2オペランドが第1オペランドより大きいか等しい場合、
制御フローは命令で指定されたアドレスまたはラベルに移ります。 
第2オペランドが最初のオペランドより小さいためにこの条件がトリガされない場合、制御フローは変更されず、最初の \printf が実行されます。 
\myindex{x86!\Instructions!JNE}
2番目のチェックは、 \JNE :\emph{Jump if Not Equal}です。 
オペランドが等しい場合、制御フローは変更されません。

\myindex{x86!\Instructions!JGE}
3番目のチェックは、最初のオペランドが2番目のオペランドより大きい場合、または等しい場合は \JGE :\emph{Jump if Greater or Equal}です。 
したがって、3つの条件ジャンプがすべてトリガされた場合、\printf の呼び出しはまったく実行されません。 
これは特別な介入なしには不可能です。 
\TT{f\_unsigned()}関数を見てみましょう。 
\TT{f\_unsigned()}関数は、次のように、 \JLE および \JGE の代わりにJBEおよびJAE命令が使用される点を除いて、
\TT{f\_signed()}と同じです。

\lstinputlisting[caption=GCC,style=customasmx86]{patterns/07_jcc/simple/unsigned_MSVC.asm}

\myindex{x86!\Instructions!JBE}
\myindex{x86!\Instructions!JAE}

すでに説明したように、分岐命令は異なります。
\JBE---\emph{Jump if Below or Equal} and \JAE---\emph{Jump if Above or Equal}
これらの命令(\JA/\JAE/\JB/\JBE)は、 \JG/\JGE/\JL/\JLE とは、符号なしの数字で動作する点が異なります。

\myindex{x86!\Instructions!JA}
\myindex{x86!\Instructions!JB}
\myindex{x86!\Instructions!JG}
\myindex{x86!\Instructions!JL}
\myindex{Signed numbers}

また、符号付き数値表現についてのセクションも参照してください(\myref{sec:signednumbers})。 
\JA/\JB の代わりに \JG/\JL が使用されている場合や、その逆の場合は、
変数がそれぞれ符号付きか、または符号なしなのかがほぼはっきりします。
ここには、もう何も新しくない、 \main 関数もあります。

\lstinputlisting[caption=\main,style=customasmx86]{patterns/07_jcc/simple/main_MSVC.asm}

\input{patterns/07_jcc/simple/olly_JPN.tex}

\clearpage
\myparagraph{x86 + MSVC + Hiew}
\myindex{Hiew}

入力値にかかわらず、\TT{f\_unsigned()}関数が常に \q{a==b}を出力するように、
実行可能ファイルにパッチを当てることができます。 
ここで、Hiewでどのように見えるか見てみましょう。

\begin{figure}[H]
\centering
\myincludegraphics{patterns/07_jcc/simple/hiew_unsigned1.png}
\caption{Hiew: \TT{f\_unsigned()} 関数}
\label{fig:jcc_hiew_1}
\end{figure}

本質的には、次の3つのタスクを実行する必要があります。
\begin{itemize}
\item 最初のジャンプが常に起動しなければならない
\item 2番目のジャンプが決して起動してはならない
\item 3番目のジャンプが常にト起動しなければならない
\end{itemize}

したがって、コードフローは常に2番目の \printf を通過し、 \q{a==b}を出力するように指示できます。 

3つの命令(またはバイト)をパッチする必要があります。

\begin{itemize}
\item 最初のジャンプはJMPになりますが、\gls{jump offset}は同じままです。

\item 
2回目のジャンプがトリガされることもありますが、いずれにしても次の命令にジャンプします。

なぜなら、\gls{jump offset}を0に設定しているからです。これらの命令では、ジャンプオフセットが次の命令のアドレスに追加されます。 
オフセットが0の場合、
ジャンプは制御を次の命令に移します。

\item 
私たちが最初のものと同様に \JMP を置き換える3番目のジャンプは、常に起動します。

\end{itemize}

\clearpage
変更されたコードは次のとおりです。

\begin{figure}[H]
\centering
\myincludegraphics{patterns/07_jcc/simple/hiew_unsigned2.png}
\caption{Hiew: let's modify the \TT{f\_unsigned()} function}
\label{fig:jcc_hiew_2}
\end{figure}

これらのジャンプのいずれかを変更することができなければ、\printf 呼び出しを1回だけ実行したいのですが、何回か実行することになるでしょう。

\myparagraph{\NonOptimizing GCC}

\myindex{puts() instead of printf()}
\NonOptimizing GCC 4.4.1 
はほとんど同じコードを生成しますが、 \printf ではなく \puts~(\myref{puts}) が生成されます。

\myparagraph{\Optimizing GCC}

実行される度にフラグが同じ値を持つ場合、
鋭い読者はなぜ \CMP が何度も実行されるのかと尋ねるかもしれません。

おそらく、最適化されたMSVCではこうはできませんが、GCC 4.8.1の最適化はより深刻です。

\lstinputlisting[caption=GCC 4.8.1 f\_signed(),style=customasmx86]{patterns/07_jcc/simple/GCC_O3_signed.asm}

% should be here instead of 'switch' section?
また、\TT{CALL puts / RETN}の代わりにここに\TT{JMPを入れて}います。

この種のトリックは後で説明します:\myref{JMP_instead_of_RET}

この種のx86コードは、まれです。
MSVC 2012のように、そのようなコードを生成することはできません。 
一方、アセンブリ言語プログラマは、\TT{Jcc}命令を積み重ねることができるという
事実を十分に認識しています。

だから、どこかでそのような積み重ねを見ると、コードは手書きの可能性が高いです。

\TT{f\_unsigned()}関数は
巧妙に短いものではありません:

\lstinputlisting[caption=GCC 4.8.1 f\_unsigned(),style=customasmx86]{patterns/07_jcc/simple/GCC_O3_unsigned_JPN.asm}

それにもかかわらず、3つではなく2つの\TT{CMP}命令があります。

したがって、GCC 4.8.1の最適化アルゴリズムはまだ完璧ではないでしょう。
}

\subsubsection{ARM}

% subsubsections here
\EN{\myparagraph{32-bit ARM}
\label{subsec:jcc_ARM}

\mysubparagraph{\OptimizingKeilVI (\ARMMode)}

\lstinputlisting[caption=\OptimizingKeilVI (\ARMMode),style=customasmARM]{patterns/07_jcc/simple/ARM/ARM_O3_signed.asm}

\myindex{ARM!Condition codes}
% FIXME \ref -> which instructions?

Many instructions in ARM mode could be executed only when specific flags are set.
E.g. this is often used when comparing numbers.

\myindex{ARM!\Instructions!ADD}
\myindex{ARM!\Instructions!ADDAL}

For instance, the \ADD instruction is in fact named \TT{ADDAL} internally, where \TT{AL} stands for
\emph{Always}, i.e., execute always.
The predicates are encoded in 4 high bits of the 32-bit ARM instructions (\emph{condition field}).
\myindex{ARM!\Instructions!B}
The \TT{B} instruction for unconditional jumping is in fact conditional and encoded just like any other
conditional jump, but has \TT{AL} in the \emph{condition field}, and it implies \emph{execute ALways}, 
ignoring flags.

\myindex{ARM!\Instructions!ADR}
\myindex{ARM!\Instructions!ADRcc}
\myindex{ARM!\Instructions!CMP}

The \TT{ADRGT} instruction works just like \TT{ADR} but executes only in case the previous \CMP
instruction founds one of the numbers greater than the another, while comparing the two (\emph{Greater Than}).

\myindex{ARM!\Instructions!BL}
\myindex{ARM!\Instructions!BLcc}

The next \TT{BLGT} instruction behaves exactly as \TT{BL} 
and is triggered only if the result of the comparison has been (\emph{Greater Than}). 
\TT{ADRGT} writes a pointer to the string \TT{a>b\textbackslash{}n} into \Reg{0} and \TT{BLGT} calls \printf.
Therefore, instructions suffixed with \TT{-GT} are to execute only in case the value in \Reg{0} (which is $a$) is bigger than the value in \Reg{4} (which is $b$).

\myindex{ARM!\Instructions!ADRcc}
\myindex{ARM!\Instructions!BLcc}

Moving forward we see the \TT{ADREQ} and \TT{BLEQ} instructions.
They behave just like \TT{ADR} and \TT{BL}, but are to be executed only if operands were equal to each
other during the last comparison.
Another \CMP is located before them (because the \printf execution may have tampered the flags).

\myindex{ARM!\Instructions!LDMccFD}
\myindex{ARM!\Instructions!LDMFD}

Then we see \TT{LDMGEFD}, this instruction works just like \TT{LDMFD}\footnote{\ac{LDMFD}},
but is triggered only when one of the values is greater or equal than the other (\emph{Greater or Equal}).
The \TT{LDMGEFD SP!, \{R4-R6,PC\}} instruction acts like a function epilogue, but it will be triggered only if $a>=b$, and only then the function execution will finish.

\myindex{Function epilogue}

But if that condition is not satisfied, i.e., $a<b$, then the control flow will continue to the next \\
\TT{\q{LDMFD SP!, \{R4-R6,LR\}}} instruction, which is one more function epilogue. This instruction restores not only the \TT{R4-R6} registers state, but also \ac{LR} instead of \ac{PC}, thus, it does not return from the function.
The last two instructions call \printf with the string <<a<b\textbackslash{}n>> as a sole argument.
We already examined an unconditional jump to the \printf function instead of function return in <<\PrintfSeveralArgumentsSectionName>> section~(\myref{ARM_B_to_printf}).

\myindex{ARM!\Instructions!ADRcc}
\myindex{ARM!\Instructions!BLcc}
\myindex{ARM!\Instructions!LDMccFD}
\TT{f\_unsigned} is similar, only the \TT{ADRHI}, \TT{BLHI}, and \TT{LDMCSFD} instructions are used there, these predicates (\emph{HI = Unsigned higher, CS = Carry Set (greater than or equal)}) are analogous to those examined before, but for unsigned values.

There is not much new in the \main function for us:

\lstinputlisting[caption=\main,style=customasmARM]{patterns/07_jcc/simple/ARM/ARM_O3_main.asm}

That is how you can get rid of conditional jumps in ARM mode.

\myindex{RISC pipeline}
Why is this so good? Read here: \myref{branch_predictors}.

\myindex{x86!\Instructions!CMOVcc}

There is no such feature in x86, except the \TT{CMOVcc} instruction, it is the same as \MOV,
but triggered only when specific flags are set, usually set by \CMP.

\mysubparagraph{\OptimizingKeilVI (\ThumbMode)}

\lstinputlisting[caption=\OptimizingKeilVI (\ThumbMode),style=customasmARM]{patterns/07_jcc/simple/ARM/ARM_thumb_signed.asm}

\myindex{ARM!\Instructions!BLE}
\myindex{ARM!\Instructions!BNE}
\myindex{ARM!\Instructions!BGE}
\myindex{ARM!\Instructions!BLS}
\myindex{ARM!\Instructions!BCS}
\myindex{ARM!\Instructions!B}
\myindex{ARM!\ThumbMode}

Only \TT{B} instructions in Thumb mode may be supplemented by \emph{condition codes}, so the Thumb code 
looks more ordinary.

\TT{BLE} is a normal conditional jump \emph{Less than or Equal}, 
\TT{BNE}---\emph{Not Equal}, 
\TT{BGE}---\emph{Greater than or Equal}.

\TT{f\_unsigned} is similar, only other instructions are used while dealing 
with unsigned values: \TT{BLS} 
(\emph{Unsigned lower or same}) and \TT{BCS} (\emph{Carry Set (Greater than or equal)}).
}
\RU{\myparagraph{32-битный ARM}
\label{subsec:jcc_ARM}

\mysubparagraph{\OptimizingKeilVI (\ARMMode)}

\lstinputlisting[caption=\OptimizingKeilVI (\ARMMode),style=customasmARM]{patterns/07_jcc/simple/ARM/ARM_O3_signed.asm}

\myindex{ARM!Condition codes}
% FIXME \ref -> which instructions?
Многие инструкции в режиме ARM могут быть исполнены только при некоторых выставленных флагах.

Это нередко используется для сравнения чисел.

\myindex{ARM!\Instructions!ADD}
\myindex{ARM!\Instructions!ADDAL}
К примеру, инструкция \ADD на самом деле называется \TT{ADDAL} внутри, \TT{AL} означает \emph{Always}, то есть, исполнять всегда.
Предикаты кодируются в 4-х старших битах инструкции 32-битных ARM-инструкций (\emph{condition field}).
\myindex{ARM!\Instructions!B}
Инструкция безусловного перехода \TT{B} на самом деле условная и кодируется так же, 
как и прочие инструкции условных переходов, но имеет \TT{AL} в \emph{condition field}, 
то есть исполняется всегда (\emph{execute ALways}), игнорируя флаги.

\myindex{ARM!\Instructions!ADR}
\myindex{ARM!\Instructions!ADRcc}
\myindex{ARM!\Instructions!CMP}
Инструкция \TT{ADRGT} работает так же, как и \TT{ADR}, но исполняется только в случае,
если предыдущая инструкция \CMP,
сравнивая два числа, обнаруживает, что одно из них больше второго (\emph{Greater Than}).

\myindex{ARM!\Instructions!BL}
\myindex{ARM!\Instructions!BLcc}
Следующая инструкция \TT{BLGT} ведет себя так же, как и \TT{BL} и сработает, только если 
результат сравнения ``больше чем'' (\emph{Greater Than}).
\TT{ADRGT} записывает в \Reg{0} указатель на строку \TT{a>b\textbackslash{}n}, а \TT{BLGT} вызывает \printf.
Следовательно, эти инструкции с суффиксом \TT{-GT} исполнятся только в том случае, если значение
в \Reg{0} (там $a$) было больше, чем значение в \Reg{4} (там $b$).

\myindex{ARM!\Instructions!ADRcc}
\myindex{ARM!\Instructions!BLcc}
Далее мы увидим инструкции \TT{ADREQ} и \TT{BLEQ}.
Они работают так же, как и \TT{ADR} и \TT{BL}, но исполнятся только если значения при последнем сравнении были равны.
Перед ними расположен ещё один \CMP, потому что вызов \printf мог испортить состояние флагов.

\myindex{ARM!\Instructions!LDMccFD}
\myindex{ARM!\Instructions!LDMFD}
Далее мы увидим \TT{LDMGEFD}. Эта инструкция работает так же, как и \TT{LDMFD}\footnote{\ac{LDMFD}}, 
но сработает только если в результате сравнения одно из значений было больше или равно второму (\emph{Greater or Equal}).
Смысл инструкции \TT{LDMGEFD SP!, \{R4-R6,PC\}} 
в том, что это как бы эпилог функции, но он сработает только если $a>=b$, только тогда работа 
функции закончится.

\myindex{Function epilogue}
Но если это не так, то есть $a<b$, то исполнение дойдет до следующей инструкции 
\TT{LDMFD SP!, \{R4-R6,LR\}}. Это ещё один эпилог функции. Эта инструкция восстанавливает состояние регистров
\TT{R4-R6}, но и \ac{LR} вместо \ac{PC}, таким образом, пока что, не делая возврата из функции.

Последние две инструкции вызывают \printf 
со строкой <<a<b\textbackslash{}n>> в качестве единственного аргумента.
Безусловный переход на \printf вместо возврата из функции мы уже рассматривали в секции
 <<\PrintfSeveralArgumentsSectionName>>~(\myref{ARM_B_to_printf}).

\myindex{ARM!\Instructions!ADRcc}
\myindex{ARM!\Instructions!BLcc}
\myindex{ARM!\Instructions!LDMccFD}
Функция \TT{f\_unsigned} точно такая же, но там используются инструкции \TT{ADRHI}, \TT{BLHI}, и \TT{LDMCSFD}. Эти предикаты
(\emph{HI = Unsigned higher, CS = Carry Set (greater than or equal)})
аналогичны рассмотренным, но служат для работы с беззнаковыми значениями.

В функции \main ничего нового для нас нет:

\lstinputlisting[caption=\main,style=customasmARM]{patterns/07_jcc/simple/ARM/ARM_O3_main.asm}

Так, в режиме ARM можно обойтись без условных переходов.

\myindex{Конвейер RISC}
Почему это хорошо? Читайте здесь: \myref{branch_predictors}.

\myindex{x86!\Instructions!CMOVcc}
В x86 нет аналогичной возможности, если не считать инструкцию \TT{CMOVcc}, это то же что и \MOV, 
но она срабатывает только при определенных выставленных флагах, обычно выставленных при помощи \CMP во время сравнения.

\mysubparagraph{\OptimizingKeilVI (\ThumbMode)}

\lstinputlisting[caption=\OptimizingKeilVI (\ThumbMode),style=customasmARM]{patterns/07_jcc/simple/ARM/ARM_thumb_signed.asm}

\myindex{ARM!\Instructions!BLE}
\myindex{ARM!\Instructions!BNE}
\myindex{ARM!\Instructions!BGE}
\myindex{ARM!\Instructions!BLS}
\myindex{ARM!\Instructions!BCS}
\myindex{ARM!\Instructions!B}
\myindex{ARM!\ThumbMode}
В режиме Thumb только инструкции \TT{B} могут быть дополнены условием исполнения (\emph{condition code}), 
так что код для режима Thumb выглядит привычнее.

\TT{BLE} это обычный переход с условием \emph{Less than or Equal}, 
\TT{BNE} --- \emph{Not Equal}, 
\TT{BGE} --- \emph{Greater than or Equal}.

Функция \TT{f\_unsigned} точно такая же, но для работы с беззнаковыми величинами 
там используются инструкции \TT{BLS} 
(\emph{Unsigned lower or same}) и \TT{BCS} (\emph{Carry Set (Greater than or equal)}).
}
\DE{\myparagraph{32-bit ARM}
\label{subsec:jcc_ARM}

\mysubparagraph{\OptimizingKeilVI (\ARMMode)}

\lstinputlisting[caption=\OptimizingKeilVI (\ARMMode),style=customasmARM]{patterns/07_jcc/simple/ARM/ARM_O3_signed.asm}

\myindex{ARM!Condition codes}
% FIXME \ref -> which instructions?
Viele Befehle im ARM mode können nur ausgeführt werden, wenn spezielle Flags gesetzt sind.
Dies ist beispielsweise oft beim Vergleich von Zahlen der Fall.

\myindex{ARM!\Instructions!ADD}
\myindex{ARM!\Instructions!ADDAL}
Der \ADD Befehl zum Beispiel heißt hier intern \TT{ADDAL}, wobei \TT{AL} für \emph{Always} (dt. immer) steht, d.h. er wird
immer ausgeführt.
Die Prädikate werden in den 4 höchstwertigsten Bits des 32-Bit-ARM-Befehls kodiert, dem \emph{condition field}.

\myindex{ARM!\Instructions!B}
Der Befehl \TT{B} für einen unbedingten Sprung ist tatsächlich doch bedingt und genau wie jeder andere bedingte Sprung
kodiert, nut dass er \TT{AL} im \emph{condition field} hat und dadurch die Flags ignoriert und immer ausgeführt wird.

\myindex{ARM!\Instructions!ADR}
\myindex{ARM!\Instructions!ADRcc}
\myindex{ARM!\Instructions!CMP}
Der Befehl \TT{ADRGT} arbeitet wie \TT{ADR}, wird aber nur ausgeführt, wenn das vorangehende \CMP ergeben hat, dass eine
der beiden Eingabezahlen größer war als die andere. 

\myindex{ARM!\Instructions!BL}
\myindex{ARM!\Instructions!BLcc}
% ToBeUpdated
Der folgende \TT{BLGT} Befehl verhält sich genau wie \TT{BL} und wird nur dann ausgeführt, wenn das Ergebnis des
Vergleichs das gleiche war (d.h. größer als).
\TT{ADRGT} schreibt einen Pointer auf den String \TT{a>b\textbackslash{}n} nach \Reg{0} und \TT{BLGT} ruft \printf auf.
Das heißt, Befehl mit dem Suffix \TT{-GT} werden nur ausgeführt, wenn der Wert in \Reg{0} (das ist $a$) größer ist als
der Wert in \Reg{4} (das ist $b$).

\myindex{ARM!\Instructions!ADRcc}
\myindex{ARM!\Instructions!BLcc}
Im Folgenden finden wir die Befehle \TT{ADREQ} und \TT{BLEQ}.
Sie verhalten sich wie \TT{ADR} und \TT{BL}, werden aber nur ausgeführt, wenn die beiden Operanden des letzten
Vergleichs gleich waren.
Ein weiteres \CMP befindet sich davor (denn die Ausführung von \printf könnte die Flags verändert haben).

\myindex{ARM!\Instructions!LDMccFD}
\myindex{ARM!\Instructions!LDMFD}
Dann finden wir \TT{LDMGEFD}; dieser Befehl arbeitet genau wie \TT{LDMFD}\footnote{\ac{LDMFD}}, wird aber nur
ausgeführt, wenn einer der Werte größer gleich dem anderen ist. 
Der Befehl \TT{LDMGEFD SP!, \{R4-R6,PC\}} fungiert als Funktionsepilog, wird aber nur ausgeführt, ewnn $a>=b$ und nur
dann wird die Funktionsausführung beendet.
\myindex{Function epilogue}
Wenn aber diese Bedingung nicht erfüllt ist, d.h. $a<b$, wird der Control Flow zum nächsten \\
\TT{\q{LDMFD SP!, \{R4-R6,LR\}}} springen, der ebenfalls einen Funktionsepilog darstellt. Dieser Befehl stellt nicht nur
den Zustand der \TT{R4-R6} Register wieder her, sondern auch \ac{LR} anstatt \ac{PC}, dadurch gibt er nichts aus der
Funktion zurück.
Die letzten beiden Befehle rufen \printf mit dem String <<a<b\textbackslash{}n>> als einzigem Argument auf.
Wir haben bereits einen unbedingten Sprung zur Funktion \printf anstelle einer Funktionsrückgabe im Abschnitt
<<\PrintfSeveralArgumentsSectionName>>~(\myref{ARM_B_to_printf}) untersucht.

\myindex{ARM!\Instructions!ADRcc}
\myindex{ARM!\Instructions!BLcc}
\myindex{ARM!\Instructions!LDMccFD}
\TT{f\_unsigned} ist ähnlich, nur die Befehle \TT{ADRHI}, \TT{BLHI} und \TT{LDMCSFD} werden hier verwendet.
Deren Prädikaten (\emph{HI = Unsigned higher, CS = Carry Set (greater than or equal)}) sind analog zu den eben
betrachteten, nur eben für vorzeichenlose Werte. 

In der Funktion \main finden wir nicht viel Neues:

\lstinputlisting[caption=\main,style=customasmARM]{patterns/07_jcc/simple/ARM/ARM_O3_main.asm}
Auf diese Weise kann man bedingte Sprünge im ARM mode entfernen.


\myindex{RISC pipeline}
Für eine Begründung warum dies vorteilhaft ist, siehe: \myref{branch_predictors}.

\myindex{x86!\Instructions!CMOVcc}
In x86 gibt es kein solches Feature, außer dem \TT{CMOVcc} Befehl, der genau wie \MOV funktioniert, aber nur ausgeführt
wird, wenn spezielle Flags - normalerweise durch \CMP - gesetzt sind.


\mysubparagraph{\OptimizingKeilVI (\ThumbMode)}

\lstinputlisting[caption=\OptimizingKeilVI (\ThumbMode),style=customasmARM]{patterns/07_jcc/simple/ARM/ARM_thumb_signed.asm}

\myindex{ARM!\Instructions!BLE}
\myindex{ARM!\Instructions!BNE}
\myindex{ARM!\Instructions!BGE}
\myindex{ARM!\Instructions!BLS}
\myindex{ARM!\Instructions!BCS}
\myindex{ARM!\Instructions!B}
\myindex{ARM!\ThumbMode}
Nur der \TT{B} Befehl im Thumb mode kann mit condition codes versehen werden, sodass der Thumb Code gewöhnlicher
aussieht.


\TT{BLE} ist ein normaler bedingter Sprung \emph{Less than or Equal}, 
\TT{BNE}---\emph{Not Equal}, 
\TT{BGE}---\emph{Greater than or Equal}.

\TT{f\_unsigned} ist ähnlich, nur dass andere Befehle verwendet werden, wenn mit vorzeichenlosen Werten umgegangen wird:
\TT{BLS} (\emph{Unsigned lower or same}) und \TT{BCS} (\emph{Carry Set (Greater than or equal)}).
}
\FR{\myparagraph{ARM 32-bit}
\label{subsec:jcc_ARM}

\mysubparagraph{\OptimizingKeilVI (\ARMMode)}

\lstinputlisting[caption=\OptimizingKeilVI (\ARMMode),style=customasmARM]{patterns/07_jcc/simple/ARM/ARM_O3_signed.asm}

\myindex{ARM!Condition codes}
% FIXME \ref -> which instructions?

Beaucoup d'instructions en mode ARM ne peuvent être exécutées que lorsque certains
flags sont mis.
E.g, ceci est souvent utilisé lorsque l'on compare les nombres

\myindex{ARM!\Instructions!ADD}
\myindex{ARM!\Instructions!ADDAL}

Par exemple, l'instruction \ADD est en fait appelée \TT{ADDAL} en interne, où \TT{AL}
signifie \emph{Always}, i.e., toujours exécuter.
Les prédicats sont encodés dans les 4 bits du haut des instructions ARM 32-bit. (\emph{condition field}).
\myindex{ARM!\Instructions!B}
L'instruction de saut inconditionnel \TT{B} est en fait conditionnelle et encodée
comme toutes les autres instructions de saut conditionnel, mais a \TT{AL} dans le
\emph{champ de condition}, et \emph{s'exécute toujours} (ALways), ignorants les flags.

\myindex{ARM!\Instructions!ADR}
\myindex{ARM!\Instructions!ADRcc}
\myindex{ARM!\Instructions!CMP}

L'instruction \TT{ADRGT} fonctionne comme \TT{ADR} mais ne s'exécute que dans le
cas où l'instruction \CMP précédente a trouvé un des nombres plus grand que l'autre,
en comparant les deux (\emph{Greater Than}).

\myindex{ARM!\Instructions!BL}
\myindex{ARM!\Instructions!BLcc}

L'instruction \TT{BLGT} se comporte exactement comme \TT{BL} et n'est effectuée
que si le résultat de la comparaison était \emph{Greater Than} (plus grand).
\TT{ADRGT} écrit un pointeur sur la chaîne \TT{a>b\textbackslash{}n} dans \Reg{0}
et \TT{BLGT} appelle \printf.
Donc, les instructions suffixées par \TT{-GT} ne sont exécutées que si la valeur
dans \Reg{0} (qui est $a$) est plus grande que la valeur dans \Reg{4} (qui est $b$).

\myindex{ARM!\Instructions!ADRcc}
\myindex{ARM!\Instructions!BLcc}

En avançant, nous voyons les instructions \TT{ADREQ} et \TT{BLEQ}.
Elles se comportent comme \TT{ADR} et \TT{BL}, mais ne sont exécutées que si les
opérandes étaient égaux lors de la dernière comparaison.
Un autre \CMP se trouve avant elles (car l'exécution de \printf pourrait avoir
modifiée les flags).

\myindex{ARM!\Instructions!LDMccFD}
\myindex{ARM!\Instructions!LDMFD}

Ensuite nous voyons \TT{LDMGEFD}, cette instruction fonctionne comme \TT{LDMFD}\footnote{\ac{LDMFD}},
mais n'est exécutée que si l'une des valeurs est supérieure ou égale à l'autre
(\emph{Greater or Equal}).\\
L'instruction \TT{LDMGEFD SP!, \{R4-R6,PC\}} se comporte comme une fonction épilogue,
mais elle ne sera exécutée que si $a>=b$, et seulement lorsque l'exécution de la
fonction se terminera.

\myindex{Function epilogue}

Mais si cette condition n'est pas satisfaite, i.e., $a<b$, alors le flux d'exécution
continue à l'instruction suivante, \TT{\q{LDMFD SP!, \{R4-R6,LR\}}}, qui est aussi
un épilogue de la fonction. Cette instruction ne restaure pas seulement l'état des
registres \TT{R4-R6}, mais aussi \ac{LR} au lieu de \ac{PC}, donc il ne retourne
pas de la fonction.
Les deux dernières instructions appellent \printf avec la chaîne <<a<b\textbackslash{}n>>
comme unique argument.
Nous avons déjà examiné un saut inconditionnel à la fonction \printf au lieu
d'un appel avec retour dans <<\PrintfSeveralArgumentsSectionName>> section~(\myref{ARM_B_to_printf}).

\myindex{ARM!\Instructions!ADRcc}
\myindex{ARM!\Instructions!BLcc}
\myindex{ARM!\Instructions!LDMccFD}
\TT{f\_unsigned} est très similaire, à part les instructions \TT{ADRHI}, \TT{BLHI},
et \TT{LDMCSFD} utilisées ici, ces prédicats (\emph{HI = Unsigned higher, CS = Carry
Set (greater than or equal)}) sont analogues à ceux examinés avant, mais pour des
valeurs non signées.

Il n'y a pas grand chose de nouveau pour nous dans la fonction \main:

\lstinputlisting[caption=\main,style=customasmARM]{patterns/07_jcc/simple/ARM/ARM_O3_main.asm}

C'est ainsi que vous pouvez vous débarrasser des sauts conditionnels en mode ARM.

\myindex{RISC pipeline}
Pourquoi est-ce que c'est si utile? Lire ici: \myref{branch_predictors}.

\myindex{x86!\Instructions!CMOVcc}

Il n'y a pas de telle caractéristique en x86, exceptée l'instruction \TT{CMOVcc},
qui est comme un \MOV, mais effectuée seulement lorsque certains flags sont mis,
en général mis par \CMP.

\mysubparagraph{\OptimizingKeilVI (\ThumbMode)}

\lstinputlisting[caption=\OptimizingKeilVI (\ThumbMode),style=customasmARM]{patterns/07_jcc/simple/ARM/ARM_thumb_signed.asm}

\myindex{ARM!\Instructions!BLE}
\myindex{ARM!\Instructions!BNE}
\myindex{ARM!\Instructions!BGE}
\myindex{ARM!\Instructions!BLS}
\myindex{ARM!\Instructions!BCS}
\myindex{ARM!\Instructions!B}
\myindex{ARM!\ThumbMode}

En mode Thumb, seules les instructions \TT{B} peuvent être complétées par un
\emph{condition codes}, (code de condition) donc le code Thumb paraît plus ordinaire.

\TT{BLE} est un saut conditionnel normal \emph{Less than or Equal} (inférieur ou égal),
\TT{BNE}---\emph{Not Equal} (non égal),
\TT{BGE}---\emph{Greater than or Equal} (supérieur ou égal).

\TT{f\_unsigned} est similaire, seules d'autres instructions sont utilisées
pour travailler avec des valeurs non-signées: \TT{BLS}
(\emph{Unsigned lower or same} non signée, inférieur ou égal) et \TT{BCS} (\emph{Carry
Set (Greater than or equal)} supérieur ou égal).
}
\IT{\myparagraph{32-bit ARM}
\label{subsec:jcc_ARM}

\mysubparagraph{\OptimizingKeilVI (\ARMMode)}

\lstinputlisting[caption=\OptimizingKeilVI (\ARMMode),style=customasmARM]{patterns/07_jcc/simple/ARM/ARM_O3_signed.asm}

\myindex{ARM!Condition codes}
% FIXME \ref -> which instructions?

In modalità ARM molte istruzioni possono essere eseguite solo quando specifici flag sono settati.
Es. sono spesso usate quando si confrontano numeri.

\myindex{ARM!\Instructions!ADD}
\myindex{ARM!\Instructions!ADDAL}

Ad esempio, l'istruzione \ADD è infatti chiamata internamente is in fact named \TT{ADDAL}, il suffisso \TT{AL} sta per
\emph{Always}, ad indicare che viene eseguita sempre.
I predicati sono codificati nei 4 bit alti dell'istruzione ARM a 32-bit (\emph{condition field}).
\myindex{ARM!\Instructions!B}
L'istruzione \TT{B} per effettuare un salto non condizionale è in realtà condizionale ed è codificata proprio come ogni altro
jump condizionale, ma ha \TT{AL} (\emph{execute ALways}) nel \emph{condition field}, e ciò implica che venga sempre eseguito, ignorando i flag.

\myindex{ARM!\Instructions!ADR}
\myindex{ARM!\Instructions!ADRcc}
\myindex{ARM!\Instructions!CMP}

L'istruzione \TT{ADRGT} funziona come \TT{ADR}, ma viene eseguita soltanto nel caso in cui la precedente istruzione \CMP
trovi uno dei due numeri a confronto più grande dell'altro, (\emph{Greater Than}).

\myindex{ARM!\Instructions!BL}
\myindex{ARM!\Instructions!BLcc}

La successiva istruzione \TT{BLGT} si comporta esattamente come \TT{BL} 
ed il salto viene innescato solo se il risultato del confronto è (\emph{Greater Than}). 
\TT{ADRGT} scrive un putatore alla stringa \TT{a>b\textbackslash{}n} nel registro \Reg{0} e \TT{BLGT} chiama \printf.
Le istruzioni aventi il suffisso \TT{-GT} in questo caso sono quindi eseguite solo se il valore in \Reg{0} (ovvero $a$) è maggiore del valore 
in \Reg{4} (ovvero $b$).

\myindex{ARM!\Instructions!ADRcc}
\myindex{ARM!\Instructions!BLcc}

Andando avanti vediamo le istruzioni \TT{ADREQ} e \TT{BLEQ} instructions.
Si comporano come \TT{ADR} e \TT{BL}, ma vengono eseguite solo se gli operandi erano uguali al momento dell'ultimo confronto.
Un altra \CMP si trova subito prima di loro (poiché l'esecuzione di \printf potrebbe aver alterato i flag).

\myindex{ARM!\Instructions!LDMccFD}
\myindex{ARM!\Instructions!LDMFD}

Ancora più avanti vediamo \TT{LDMGEFD}, questa istruzione funziona come \TT{LDMFD}\footnote{\ac{LDMFD}},
ma viene eseguita solo quando uno dei valori e maggiore di o uguale all'altro (\emph{Greater or Equal}).
L'istruzione \TT{LDMGEFD SP!, \{R4-R6,PC\}} si comporta come un epilogo di funzione, ma viene eseguita solo se $a>=b$, e solo in tal caso avrà termine l'esecuzione della funzione.

\myindex{Function epilogue}

Nel caso in cui questa condizione non venga soddisfatta, ovvero se $a<b$, il flusso continuerà alla successiva istruzione \\
\TT{\q{LDMFD SP!, \{R4-R6,LR\}}} , un altro epilogo di funzione. Questa istruzione non ripristina soltanto lo stato dei registri \TT{R4-R6} , ma anche \ac{LR} invece di \ac{PC}, non ritornando così dalla funzione.
Le due ultime istruzioni chiamano \printf con la stringa <<a<b\textbackslash{}n>> come unico argomento.
Abbiamo già visto un salto diretto non condizionale alla funzione \printf senza altro codice di uscita/ritorno dalla funzione nella sezione <<\PrintfSeveralArgumentsSectionName>> section~(\myref{ARM_B_to_printf}).

\myindex{ARM!\Instructions!ADRcc}
\myindex{ARM!\Instructions!BLcc}
\myindex{ARM!\Instructions!LDMccFD}
\TT{f\_unsigned} è simile, e vengono utilizzate le funzioni \TT{ADRHI}, \TT{BLHI}, e \TT{LDMCSFD}. Questi predicati (\emph{HI = Unsigned higher, CS = Carry Set (maggiore di o uguale a)}) sono analoghi a quelli visti in precedenza, e operano su valori di tipo unsigned.

Nella funzione \main non c'è nulla di nuovo:

\lstinputlisting[caption=\main,style=customasmARM]{patterns/07_jcc/simple/ARM/ARM_O3_main.asm}

In questo modo ci si può sbarazzare dei salti condizionali in modalità ARM.
\myindex{RISC pipeline}
Perchè è bene? Leggi qui: \myref{branch_predictors}.

\myindex{x86!\Instructions!CMOVcc}

Non esiste una funzionalità simile in x86, eccetto per l'istruzione \TT{CMOVcc} , che è uguale a \MOV ma viene eseguita solo
se specifici flag sono settati, solitamente da \CMP.

\mysubparagraph{\OptimizingKeilVI (\ThumbMode)}

\lstinputlisting[caption=\OptimizingKeilVI (\ThumbMode),style=customasmARM]{patterns/07_jcc/simple/ARM/ARM_thumb_signed.asm}

\myindex{ARM!\Instructions!BLE}
\myindex{ARM!\Instructions!BNE}
\myindex{ARM!\Instructions!BGE}
\myindex{ARM!\Instructions!BLS}
\myindex{ARM!\Instructions!BCS}
\myindex{ARM!\Instructions!B}
\myindex{ARM!\ThumbMode}

Solo le istruzioni \TT{B} in modalità Thumb possono essere supplementate da \emph{condition codes}, pertanto il codice Thumb 
ha un aspetto più ordinario.

\TT{BLE} è un normale jump condizionale \emph{Less than or Equal}, 
\TT{BNE}---\emph{Not Equal}, 
\TT{BGE}---\emph{Greater than or Equal}.

\TT{f\_unsigned} è simile, con la differenza che vengono usate altre istruzioni per operare con valori
di tipo unsigned: \TT{BLS} 
(\emph{Unsigned lower or same}) e \TT{BCS} (\emph{Carry Set (Greater than or equal)}).
}
\JA{\myparagraph{32-bit ARM}
\label{subsec:jcc_ARM}

\mysubparagraph{\OptimizingKeilVI (\ARMMode)}

\lstinputlisting[caption=\OptimizingKeilVI (\ARMMode),style=customasmARM]{patterns/07_jcc/simple/ARM/ARM_O3_signed.asm}

\myindex{ARM!Condition codes}
% FIXME \ref -> which instructions?

ARMモードの多くの命令は、特定のフラグがセットされている場合にのみ実行できます。
例えば、これは数字を比較するときによく使用されます。

\myindex{ARM!\Instructions!ADD}
\myindex{ARM!\Instructions!ADDAL}

例えば、 \ADD 命令は実際には内部で\TT{ADDAL}と名付けられ、
ALは\emph{常に}、すなわち常に実行する。
述語は、32ビットARM命令の4つの上位ビット(\emph{条件フィールド})でエンコードされます。
\myindex{ARM!\Instructions!B}
無条件ジャンプの\TT{B}命令は、実際には条件付きで他の条件ジャンプと同様にエンコードされますが、
条件フィールドには\TT{AL}があり、フラグを無視して\emph{常に実行}することを意味します。

\myindex{ARM!\Instructions!ADR}
\myindex{ARM!\Instructions!ADRcc}
\myindex{ARM!\Instructions!CMP}

\TT{ADRGT}命令は\TT{ADR}と同じように動作しますが、前の \CMP 命令が2つ(\emph{大きい方})を比較しながら、
他の命令より大きな数値の1つを検出した場合にのみ実行されます。

\myindex{ARM!\Instructions!BL}
\myindex{ARM!\Instructions!BLcc}

次の\TT{BLGT}命令は\TT{BL}と同じように動作し、
比較の結果が(より大きい)場合にのみ実行されます。 
\TT{ADRGT}は文字列\TT{a>b\textbackslash{}n}へのポインタを\Reg{0}に書き込み、\TT{BLGT}は \printf を呼び出します。
したがって、\TT{-GT}の後に続く命令は、\Reg{0}($a$)の値が\Reg{4}($b$)の値より大きい場合にのみ実行されます。

\myindex{ARM!\Instructions!ADRcc}
\myindex{ARM!\Instructions!BLcc}

\TT{ADREQ}命令と\TT{BLEQ}命令が順方向に進みます。
それらは\TT{ADR}と\TT{BL}のように動作しますが、最後の比較時にオペランドが等しい場合にのみ実行されます。 
\printf の実行によってフラグが改ざんされた可能性があるため、別の \CMP がその前に配置されます。

\myindex{ARM!\Instructions!LDMccFD}
\myindex{ARM!\Instructions!LDMFD}

次に、\TT{LDMGEFD}を参照してください。この命令は\TT{LDMFD}\footnote{\ac{LDMFD}}のように機能しますが、
一方の値が他方の値より大きいか等しい場合にのみ実行されます。 
\TT{LDMGEFD SP!, \{R4-R6,PC\}}命令は関数エピローグのように動作しますが、$a>=b$の場合にのみトリガされ、その後に関数の実行が終了します。

\myindex{Function epilogue}

しかし、その条件が満たされない場合、すなわち$a<b$の場合、制御フローは次の
\TT{\q{LDMFD SP!, \{R4-R6,LR\}}}命令に続き、これはもう1つの関数エピローグです。この命令は、\TT{R4-R6}だけでなく\ac{PC}の代わりに\ac{LR}も登録されているため、関数からは戻りません。
最後の2つの命令は、文字列<<a<b\textbackslash{}n>>を唯一の引数として \printf を呼び出します。  
\printf セクション(\myref{ARM_B_to_printf})の関数の戻り値ではなく、 \printf 関数への無条件ジャンプを調べました。

\myindex{ARM!\Instructions!ADRcc}
\myindex{ARM!\Instructions!BLcc}
\myindex{ARM!\Instructions!LDMccFD}
\TT{f\_unsigned}は類似しており、\TT{ADRHI}、\TT{BLHI}、および\TT{LDMCSFD}命令のみが使用されています。これらの述部(\emph{HI = Unsigned higher, CS = Carry Set (greater than or equal)})は、前に説明したものと類似しています。

\main 関数にはそんなに新しい点はありません。

\lstinputlisting[caption=\main,style=customasmARM]{patterns/07_jcc/simple/ARM/ARM_O3_main.asm}

これは、ARMモードでの条件付きジャンプを取り除く方法です。

\myindex{RISC pipeline}
なぜこれがよいのでしょう? 以下を読んでください:\myref{branch_predictors}

\myindex{x86!\Instructions!CMOVcc}

x86では、\TT{CMOVcc}命令以外は \MOV と同じですが、
通常は \CMP によって設定された特定のフラグが設定されている場合にのみ実行されます。

\mysubparagraph{\OptimizingKeilVI (\ThumbMode)}

\lstinputlisting[caption=\OptimizingKeilVI (\ThumbMode),style=customasmARM]{patterns/07_jcc/simple/ARM/ARM_thumb_signed.asm}

\myindex{ARM!\Instructions!BLE}
\myindex{ARM!\Instructions!BNE}
\myindex{ARM!\Instructions!BGE}
\myindex{ARM!\Instructions!BLS}
\myindex{ARM!\Instructions!BCS}
\myindex{ARM!\Instructions!B}
\myindex{ARM!\ThumbMode}

Thumbモードの\TT{B}命令だけが\emph{条件コード}で補完されるため、
Thumbコードはより一般的に見えます。

\TT{BLE}は通常の条件ジャンプであり、\emph{Less than or Equal}の意味です。 
\TT{BNE}は\emph{Not Equal}の意味です。
\TT{BGE}は\emph{Greater than or Equal}の意味です。

\TT{f\_unsigned}は似ていますが、符号なしの値を扱う際には、
\TT{BLS}(\emph{Unsigned lower or same})および\TT{BCS}(\emph{Carry Set (Greater than or equal)})命令しか使用されません。
}

\EN{\subsubsection{ARM64}

\myparagraph{GCC}

Let's compile the example using GCC 4.8.1 in ARM64:

\lstinputlisting[numbers=left,label=hw_ARM64_GCC,caption=\NonOptimizing GCC 4.8.1 + objdump,style=customasmARM]{patterns/01_helloworld/ARM/hw.lst}

There are no Thumb and Thumb-2 modes in ARM64, only ARM, so there are 32-bit instructions only.
The Register count is doubled: \myref{ARM64_GPRs}.
64-bit registers have \TT{X-} prefixes, while its 32-bit parts---\TT{W-}.

\myindex{ARM!\Instructions!STP}
The \TT{STP} instruction (\emph{Store Pair}) 
saves two registers in the stack simultaneously: \RegX{29} and \RegX{30}.

Of course, this instruction is able to save this pair at an arbitrary place in memory, 
but the \ac{SP} register is specified here, so the pair is saved in the stack.

ARM64 registers are 64-bit ones, each has a size of 8 bytes, so one needs 16 bytes for saving two registers.

The exclamation mark (``!'') after the operand means that 16 is to be subtracted from \ac{SP} first, and only then
are values from register pair to be written into the stack.
This is also called \emph{pre-index}.
About the difference between \emph{post-index} and \emph{pre-index} 
read here: \myref{ARM_postindex_vs_preindex}.

Hence, in terms of the more familiar x86, the first instruction is just an analogue to a pair of
\TT{PUSH X29} and \TT{PUSH X30}.
\RegX{29} is used as \ac{FP} in ARM64, and \RegX{30} 
as \ac{LR}, so that's why they are saved in the function prologue and restored in the function epilogue.

The second instruction copies \ac{SP} in \RegX{29} (or \ac{FP}).
This is made so to set up the function stack frame.

\label{pointers_ADRP_and_ADD}
\myindex{ARM!\Instructions!ADRP/ADD pair}
\TT{ADRP} and \ADD instructions are used to fill the 
address of the string \q{Hello!} into the \RegX{0} register, 
because the first function argument is passed
in this register.
There are no instructions, whatsoever, in ARM that can store a large number into a register (because the instruction
length is limited to 4 bytes, read more about it here: \myref{ARM_big_constants_loading}).
So several instructions must be utilized. The first instruction (\TT{ADRP}) writes the address of the 4KiB page, where the string is
located, into \RegX{0}, 
and the second one (\ADD) just adds the remainder to the address.
More about that in: \myref{ARM64_relocs}.

\TT{0x400000 + 0x648 = 0x400648}, and we see our \q{Hello!} C-string in the \TT{.rodata} data segment at this address.

\myindex{ARM!\Instructions!BL}

\puts is called afterwards using the \TT{BL} instruction. This was already discussed: \myref{puts}.

\MOV writes 0 into \RegW{0}. 
\RegW{0} is the lower 32 bits of the 64-bit \RegX{0} register:

\input{ARM_X0_register}

The function result is returned via \RegX{0} and \main returns 0, so that's how the return result is prepared.
But why use the 32-bit part?

Because the \Tint data type in ARM64, just like in x86-64, is still 32-bit, for better compatibility.

So if a function returns a 32-bit \Tint, only the lower 32 bits of \RegX{0} register have to be filled.

In order to verify this, let's change this example slightly and recompile it.
Now \main returns a 64-bit value:

\begin{lstlisting}[caption=\main returning a value of \TT{uint64\_t} type,style=customc]
#include <stdio.h>
#include <stdint.h>

uint64_t main()
{
        printf ("Hello!\n");
        return 0;
}
\end{lstlisting}

The result is the same, but that's how \MOV at that line looks like now:

\begin{lstlisting}[caption=\NonOptimizing GCC 4.8.1 + objdump]
  4005a4:       d2800000        mov     x0, #0x0      // #0
\end{lstlisting}

\myindex{ARM!\Instructions!LDP}

\INS{LDP} (\emph{Load Pair}) then restores the \RegX{29} and \RegX{30} registers.

There is no exclamation mark after the instruction: this implies that the values are first loaded from the stack,
and only then is \ac{SP} increased by 16.
This is called \emph{post-index}.

\myindex{ARM!\Instructions!RET}
A new instruction appeared in ARM64: \RET. 
It works just as \TT{BX LR}, only a special \emph{hint} bit is added, informing the \ac{CPU}
that this is a return from a function, not just another jump instruction, so it can execute it more optimally.

Due to the simplicity of the function, optimizing GCC generates the very same code.
}
\RU{\subsubsection{ARM64}

\myparagraph{\Optimizing GCC (Linaro) 4.9}

\myindex{Fused multiply–add}
\myindex{ARM!\Instructions!MADD}
Тут всё просто.
\TT{MADD} это просто инструкция, производящая умножение и сложение одновременно (как \TT{MLA}, 
которую мы уже видели).
Все 3 аргумента передаются в 32-битных частях X-регистров.
Действительно, типы аргументов это 32-битные \emph{int}'ы.
Результат возвращается в \TT{W0}.

\lstinputlisting[caption=\Optimizing GCC (Linaro) 4.9,style=customasmARM]{patterns/05_passing_arguments/ARM/ARM64_O3_RU.s}

Также расширим все типы данных до 64-битных \TT{uint64\_t} и попробуем:

\lstinputlisting[style=customc]{patterns/05_passing_arguments/ex64.c}

\begin{lstlisting}[style=customasmARM]
f:
	madd	x0, x0, x1, x2
	ret
main:
	mov	x1, 13396
	adrp	x0, .LC8
	stp	x29, x30, [sp, -16]!
	movk	x1, 0x27d0, lsl 16
	add	x0, x0, :lo12:.LC8
	movk	x1, 0x122, lsl 32
	add	x29, sp, 0
	movk	x1, 0x58be, lsl 48
	bl	printf
	mov	w0, 0
	ldp	x29, x30, [sp], 16
	ret

.LC8:
	.string	"%lld\n"
\end{lstlisting}

Функция \ttf{} точно такая же, только теперь используются полные части 64-битных X-регистров.
Длинные 64-битные значения загружаются в регистры по частям, это описано здесь: \myref{ARM_big_constants_loading}.

\myparagraph{\NonOptimizing GCC (Linaro) 4.9}

Неоптимизирующий компилятор выдает немного лишнего кода:

\begin{lstlisting}[style=customasmARM]
f:
	sub	sp, sp, #16
	str	w0, [sp,12]
	str	w1, [sp,8]
	str	w2, [sp,4]
	ldr	w1, [sp,12]
	ldr	w0, [sp,8]
	mul	w1, w1, w0
	ldr	w0, [sp,4]
	add	w0, w1, w0
	add	sp, sp, 16
	ret
\end{lstlisting}

Код сохраняет входные аргументы в локальном стеке на случай если кому-то (или чему-то) в этой функции
понадобится использовать регистры \TT{W0...W2}, перезаписывая оригинальные аргументы функции, которые
могут понадобится в будущем.
Это называется \emph{Register Save Area.} (\ARMPCS)
Вызываемая функция не обязана сохранять их.
Это то же что и \q{Shadow Space}: \myref{shadow_space}.

Почему оптимизирующий GCC 4.9 убрал этот, сохраняющий аргументы, код?

Потому что он провел дополнительную работу по оптимизации и сделал вывод, 
что аргументы функции не понадобятся в будущем и регистры \TT{W0...W2} также не будут использоваться.

\myindex{ARM!\Instructions!MUL}
\myindex{ARM!\Instructions!ADD}
Также мы видим пару инструкций \TT{MUL}/\TT{ADD} вместо одной \TT{MADD}.

}
\DE{\subsubsection{ARM64}

\myparagraph{\Optimizing GCC (Linaro) 4.9}

\myindex{Fused multiply–add}
\myindex{ARM!\Instructions!MADD}

Hier ist alles ganz einfach. 
\TT{MADD} ist einfach eine Instruktion die Multiplikation/Addition verschmelzt ( ähnlich wie wir es bei \TT{MLA}
gesehen haben)

Alle drei Argumente werden über den 32-Bit Part des X-Registers übergeben.
In der tat, die Argumente sind 32-Bit \emph{int}'s.
Das Ergebnis wird in \TT{W0} gespeichert.

\lstinputlisting[caption=\Optimizing GCC (Linaro) 4.9]{patterns/05_passing_arguments/ARM/ARM64_O3_EN.s}

Lasst uns nun alle Datentypen nach 64-Bit \TT{uint64\_t} konvertieren und testen:

\lstinputlisting{patterns/05_passing_arguments/ex64.c}

\begin{lstlisting}[style=customasmARM]
f:
	madd	x0, x0, x1, x2
	ret
main:
	mov	x1, 13396
	adrp	x0, .LC8
	stp	x29, x30, [sp, -16]!
	movk	x1, 0x27d0, lsl 16
	add	x0, x0, :lo12:.LC8
	movk	x1, 0x122, lsl 32
	add	x29, sp, 0
	movk	x1, 0x58be, lsl 48
	bl	printf
	mov	w0, 0
	ldp	x29, x30, [sp], 16
	ret

.LC8:
	.string	"%lld\n"
\end{lstlisting}

Die \ttf{} Funktion ist die gleiche, nur das jetzt alle 64-Bit X-Register benutzt werden.
Lange 64-Bit Werte werden Stückweise in die Register geladen, genauer beschrieben hier: \myref{ARM_big_constants_loading}.

\myparagraph{\NonOptimizing GCC (Linaro) 4.9}

Der nicht optimierte Compiler lauf ist redundanter:

\begin{lstlisting}[style=customasmARM]
f:
	sub	sp, sp, #16
	str	w0, [sp,12]
	str	w1, [sp,8]
	str	w2, [sp,4]
	ldr	w1, [sp,12]
	ldr	w0, [sp,8]
	mul	w1, w1, w0
	ldr	w0, [sp,4]
	add	w0, w1, w0
	add	sp, sp, 16
	ret
\end{lstlisting}

Der Code speichert seine Eingabe Argumente auf dem lokalen Stack,
für den Fall das die Funktion die \TT{W0..W2} Register benutzen muss
das verhindert das überschreiben der Original Argumente, die vielleicht
noch in Zukunft gebraucht werden.

Das bezeichnet man auch als \emph{Register Save Area.} (\ARMPCS).
Der Callee, ist hier nicht in der Pflicht die Werte zu speichern.
So Ähnlich wie beim \q{Shadow Space}: \myref{shadow_space}.


Warum hat der optimierte GCC 4.9 Aufruf dieses Argument weg gelassen?
Weil der Compiler in dem Fall zusätzliche Optimierungen gemacht hat. Und
erkannt hat das die zusätzlichen Argumente in der weiteren Ausführung des 
Codes nicht mehr benötigt werden. Und auch das die Register \TT{W0..W2} auch 
nicht weiter benötigt werden.

\myindex{ARM!\Instructions!MUL}
\myindex{ARM!\Instructions!ADD}

Wir können auch ein \TT{MUL}/\TT{ADD} Instruktions paar sehen anstatt einem einzelnen
\TT{MADD}.

}
\FR{\subsubsection{ARM64}

\myparagraph{GCC (Linaro) 4.9 \Optimizing}

\lstinputlisting[style=customasmARM]{patterns/12_FPU/3_comparison/ARM/ARM64_GCC_O3_FR.lst}

L'ARM64 \ac{ISA} possède des instructions FPU qui mettent les flags CPU \ac{APSR}
au lieu de \ac{FPSCR}, par commodité.
Le \ac{FPU} n'est plus un device séparé (au moins, logiquement).
\myindex{ARM!\Instructions!FCMPE}
Ici, nous voyons \INS{FCMPE}. Ceci compare les deux valeurs passées dans \RegD{0}
et \RegD{1} (qui sont le premier et le second argument de la fonction) et met les
flags \ac{APSR} (N, Z, C, V).

\myindex{ARM!\Instructions!FCSEL}
\INS{FCSEL} (\emph{Floating Conditional Select} (sélection de flottant conditionnelle)
copie la valeur de \RegD{0} ou \RegD{1} dans \RegD{0} suivant le résultat de la comparaison
(\GTT{GT}---Greater Than), et de nouveau, il utilise les flags dans le registre \ac{APSR}
au lieu de \ac{FPSCR}.

Ceci est bien plus pratique, comparé au jeu d'instructions des anciens CPUs.

Si la condition est vraie (\GTT{GT}), alors la valeur de \RegD{0} est copiée dans
\RegD{0} (i.e., il ne se passe rien).
Si la condition n'est pas vraie, la valeur de \RegD{1} est copiée dans \RegD{0}.

\myparagraph{GCC (Linaro) 4.9 \NonOptimizing}

\lstinputlisting[style=customasmARM]{patterns/12_FPU/3_comparison/ARM/ARM64_GCC_FR.lst}

GCC sans optimisation est plus verbeux.

Tout d'abord, la fonction sauve la valeur de ses arguments en entrée dans la pile
locale (\emph{Register Save Area}, espace de sauvegarde des registres).
Ensuite, le code recharge ces valeurs dans les registres \RegX{0}/\RegX{1} et finalement
les copie dans \RegD{0}/\RegD{1} afin de les comparer en utilisant \INS{FCMPE}.
Beaucoup de code redondant, mais c'est ainsi que fonctionne les compilateurs sans
optimisation.
\INS{FCMPE} compare les valeurs et met les flags du registre \ac{APSR}.
À ce moment, le compilateur ne pense pas encore à l'instruction plus commode \INS{FCSEL},
donc il procède en utilisant de vieilles méthodes:
en utilisant l'instruction \INS{BLE} (\emph{Branch if Less than or Equal} branchement si
inférieur ou égal).
Dans le premier cas ($a>b$), la valeur de $a$ est chargée dans \RegX{0}.
Dans les autres cas ($a<=b$), la valeur de $b$ est chargée dans \RegX{0}.
Enfin, la valeur dans \RegX{0} est copiée dans \RegD{0}, car la valeur de retour
doit être dans ce registre.

\mysubparagraph{\Exercise}

À titre d'exercice, vous pouvez essayer d'optimiser ce morceau de code manuellement
en supprimant les instructions redondantes et sans en introduire de nouvelles (incluant
\INS{FCSEL}).

\myparagraph{GCC (Linaro) 4.9 \Optimizing---float}

Ré-écrivons cet exemple en utilisant des \Tfloat à la place de \Tdouble.

\begin{lstlisting}[style=customc]
float f_max (float a, float b)
{
	if (a>b)
		return a;

	return b;
};
\end{lstlisting}

\lstinputlisting[style=customasmARM]{patterns/12_FPU/3_comparison/ARM/ARM64_GCC_O3_float_FR.lst}

C'est le même code, mais des S-registres sont utilisés à la place de D-registres.
C'est parce que les nombres de type \Tfloat sont passés dans des S-registres de 32-bit
(qui sont en fait la partie basse des D-registres 64-bit).

}
\IT{\myparagraph{ARM64: \Optimizing GCC (Linaro) 4.9}

\lstinputlisting[caption=f\_signed(),style=customasmARM]{patterns/07_jcc/simple/ARM/ARM64_GCC_O3_signed_IT.lst}

\lstinputlisting[caption=f\_unsigned(),style=customasmARM]{patterns/07_jcc/simple/ARM/ARM64_GCC_O3_unsigned_IT.lst}

I commenti nel codice sono stati inseriti dall'autore di questo libro.
E' impressionante notare come il compilatore non si sia reso conto che alcuni condizioni sono del tutto impossibili, e per questo motivo
si trovano delle parti con codice "morto" (dead code), che non può mai essere eseguito.

\mysubparagraph{\Exercise}

Prova ad ottimizzare manualmente queste funzioni per ottenere una versione più compatta, rimuovendo istruzioni ridondanti e senza aggiungerne di nuove.

}
\JA{\myparagraph{ARM64: \Optimizing GCC (Linaro) 4.9}

\lstinputlisting[caption=f\_signed(),style=customasmARM]{patterns/07_jcc/simple/ARM/ARM64_GCC_O3_signed_JA.lst}

\lstinputlisting[caption=f\_unsigned(),style=customasmARM]{patterns/07_jcc/simple/ARM/ARM64_GCC_O3_unsigned_JA.lst}

コメントはこの本の著者によって追加されました。 
目立ったことは、コンパイラはいくつかの条件がまったく不可能であることを認識していないため、
決して実行できない場所ではデッドコードがあることです。

\mysubparagraph{\Exercise}

これらの機能をサイズが少なくなるように手動で最適化し、新しい命令を追加せずに冗長な命令を削除してください。
}

\EN{\subsubsection{MIPS}

\lstinputlisting[caption=\Optimizing GCC 4.4.5 (IDA),style=customasmMIPS]{patterns/10_strings/1_strlen/MIPS_O3_IDA_EN.lst}

\myindex{MIPS!\Instructions!NOR}
\myindex{MIPS!\Pseudoinstructions!NOT}

MIPS lacks a \NOT instruction, but has \NOR which is \TT{OR~+~NOT} operation.

This operation is widely used in digital electronics\footnote{NOR is called \q{universal gate}}.
\myindex{Apollo Guidance Computer}
For example, the Apollo Guidance Computer used in the Apollo program, 
was built by only using 5600 NOR gates:
[Jens Eickhoff, \emph{Onboard Computers, Onboard Software and Satellite Operations: An Introduction}, (2011)].
But NOR element isn't very popular in computer programming.

So, the NOT operation is implemented here as \TT{NOR~DST,~\$ZERO,~SRC}.

From fundamentals \myref{sec:signednumbers} we know that bitwise inverting a signed number is the same 
as changing its sign and subtracting 1 from the result.

So what \NOT does here is to take the value of $str$ and transform it into $-str-1$.
The addition operation that follows prepares result.

}
\RU{\subsubsection{MIPS}

\lstinputlisting[caption=\Optimizing GCC 4.4.5 (IDA),style=customasmMIPS]{patterns/08_switch/1_few/MIPS_O3_IDA_RU.lst}

\myindex{MIPS!\Instructions!JR}

Функция всегда заканчивается вызовом \puts, так что здесь мы видим переход на \puts (\INS{JR}: \q{Jump Register})
вместо перехода с сохранением \ac{RA} (\q{jump and link}).

Мы говорили об этом ранее: \myref{JMP_instead_of_RET}.

\myindex{MIPS!Load delay slot}
Мы также часто видим NOP-инструкции после \INS{LW}.
Это \q{load delay slot}: ещё один \emph{delay slot} в MIPS.
\myindex{MIPS!\Instructions!LW}
Инструкция после \INS{LW} может исполняться в тот момент, когда \INS{LW} загружает значение из памяти.

Впрочем, следующая инструкция не должна использовать результат \INS{LW}.

Современные MIPS-процессоры ждут, если следующая инструкция использует результат \INS{LW}, так что всё это уже
устарело, но GCC всё еще добавляет NOP-ы для более старых процессоров.

Вообще, это можно игнорировать.

}
\DE{\subsubsection{MIPS}

\lstinputlisting[caption=\Optimizing GCC 4.4.5 (IDA),style=customasmMIPS]{patterns/12_FPU/2_passing_floats/MIPS_O3_IDA_DE.lst}
Und wieder sehen wir hier, dass der Befehl \INS{LUI} einen 32-Bit-Teil einer
\Tdouble Zahl nach \$V0 lädt.
Und wiederum ist es schwer nachzuvollziehen warum dies geschieht.

\myindex{MIPS!\Instructions!MFC1}
Der für uns neue Befehl an dieser Stelle ist \INS{MFC1}(\q{Move From Coprocessor
1}). Die Nummer des FPU-Koprozessors ist 1, daher die \q{1} im Namen des
Befehls. 
Dieser Befehl überträgt Werte aus den Registern des Koprozessors in die Register
der CPU (\ac{GPR}).
Auf diese Weise wird das Ergebnis von \TT{pow()} schließlich in die Register
\$A3 und \$A2 verschoben und \printf übernimmt einen 64-Bit-Wert von doppelter
Genauigkeit aus diesem Registerpaar.
}
\FR{\subsubsection{MIPS}
% FIXME better start at non-optimizing version?

La fonction utilise beaucoup de S- registres qui doivent être préservés, c'est pourquoi
leurs valeurs sont sauvegardées dans la prologue de la fonction et restaurées dans
l'épilogue.

\lstinputlisting[caption=GCC 4.4.5 \Optimizing (IDA),style=customasmMIPS]{patterns/13_arrays/1_simple/MIPS_O3_IDA_FR.lst}

Quelque chose d'intéressant: il y a deux boucles et la première n'a pas besoin de
$i$, elle a seulement besoin de $i*2$ (augmenté de 2 à chaque itération) et aussi
de l'adresse en mémoire (augmentée de 4 à chaque itération).

Donc ici nous voyons deux variables, une (dans \$V0) augmentée de 2 à chaque fois,
et une autre (dans \$V1) --- de 4.

La seconde boucle est celle où \printf est appelée et affiche la valeur de $i$ à
l'utilisateur, donc il y a une variable qui est incrémentée de 1 à chaque fois (dans
\$S0) et aussi l'adresse en mémoire (dans \$S1) incrémentée de 4 à chaque fois.

Cela nous rappelle l'optimisation de boucle que nous avons examiné avant: \myref{loop_iterators}.

Leur but est de se passer des multiplications.

}
\IT{\subsubsection{MIPS}

Una caratteristica distintiva di MIPS è l'assenza dei flag.
Apparentemente è una scelta fatta per semplificare l'analisi della dipendenza dai dati.

\myindex{x86!\Instructions!SETcc}
\myindex{MIPS!\Instructions!SLT}
\myindex{MIPS!\Instructions!SLTU}

Esistono istruzioni simili a \INS{SETcc} in x86: \INS{SLT} (\q{Set on Less Than}: versione signed) e 
\INS{SLTU} (versione unsigned).
Queste istruzioni settano il valore del registro di destinazione a 1 se la condizione è vera, a 0 se è falsa.

\myindex{MIPS!\Instructions!BEQ}
\myindex{MIPS!\Instructions!BNE}

Il registro di destinazione viene quindi controllato usando \INS{BEQ} (\q{Branch on Equal}) oppure \INS{BNE} (\q{Branch on Not Equal}) 
ed in base al caso si può verificare un salto.
Questa coppia di istruzioni è usata in MIPS per eseguire confronti e conseguenti branch.
Iniziamo con la versione signed della nostra funzione:

\lstinputlisting[caption=\NonOptimizing GCC 4.4.5 (IDA),style=customasmMIPS]{patterns/07_jcc/simple/O0_MIPS_signed_IDA_IT.lst}

\INS{SLT REG0, REG0, REG1} è stata ridotta da Ida nella sua forma breve:\\
\INS{SLT REG0, REG1}.
\myindex{MIPS!\Pseudoinstructions!BEQZ}

Notiamo anche la pseudo istruzione \INS{BEQZ} (\q{Branch if Equal to Zero}),\\
che è in realtà \INS{BEQ REG, \$ZERO, LABEL}.

\myindex{MIPS!\Instructions!SLTU}

La versione unsigned è uguale, \INS{SLTU} (versione unsigned, da cui la \q{U} nel nome) è usata al posto di \INS{SLT}:

\lstinputlisting[caption=\NonOptimizing GCC 4.4.5 (IDA),style=customasmMIPS]{patterns/07_jcc/simple/O0_MIPS_unsigned_IDA.lst}

}
\JA{\subsubsection{MIPS}

1つの特徴的なMIPS機能は、フラグが存在しないことです。 
明らかに、データ依存性の分析を簡素化するために行われました。

\myindex{x86!\Instructions!SETcc}
\myindex{MIPS!\Instructions!SLT}
\myindex{MIPS!\Instructions!SLTU}

x86には\INS{SETcc}に似た命令があります。\INS{SLT}(\q{Set on Less Than}:符号付きバージョン)と\INS{SLTU}(符号なしバージョン)です。 
これらの命令は、条件が真であれば宛先レジスタの値を1に設定し、そうでない場合は0に設定します。

\myindex{MIPS!\Instructions!BEQ}
\myindex{MIPS!\Instructions!BNE}

宛先レジスタは、\INS{BEQ} (\q{Branch on Equal}) または \INS{BNE} (\q{Branch on Not Equal})を使用してチェックされ、
ジャンプが発生することがあります。 
したがって、この命令ペアは比較および分岐のためにMIPSで使用されなければなりません。
最初に関数の符号付きバージョンから始めましょう。

\lstinputlisting[caption=\NonOptimizing GCC 4.4.5 (IDA),style=customasmMIPS]{patterns/07_jcc/simple/O0_MIPS_signed_IDA_JA.lst}

\INS{SLT REG0, REG0, REG1}は、IDAによって短縮形式
\INS{SLT REG0, REG1}に縮小されます。
\myindex{MIPS!\Pseudoinstructions!BEQZ}

実際には\INS{BEQ REG, \$ZERO, LABEL}の
\INS{BEQZ}擬似命令もあります(\q{Branch if Equal to Zero})。

\myindex{MIPS!\Instructions!SLTU}

符号なしバージョンはまったく同じですが、\INS{SLT}の代わりに\INS{SLTU}(符号なしバージョン、したがって\q{U}という名前)が使用されます。

\lstinputlisting[caption=\NonOptimizing GCC 4.4.5 (IDA),style=customasmMIPS]{patterns/07_jcc/simple/O0_MIPS_unsigned_IDA.lst}
}


\EN{\mysection{\MinesweeperWinXPExampleChapterName}
\label{minesweeper_winxp}
\myindex{Windows!Windows XP}

For those who are not very good at playing Minesweeper, we could try to reveal the hidden mines in the debugger.

\myindex{\CStandardLibrary!rand()}
\myindex{Windows!PDB}

As we know, Minesweeper places mines randomly, so there has to be some kind of random number generator or
a call to the standard \TT{rand()} C-function.

What is really cool about reversing Microsoft products is that there are \gls{PDB} 
file with symbols (function names, \etc{}).
When we load \TT{winmine.exe} into \IDA, it downloads the 
\gls{PDB} file exactly for this 
executable and shows all names.

So here it is, the only call to \TT{rand()} is this function:

\lstinputlisting[style=customasmx86]{examples/minesweeper/tmp1.lst}

\IDA named it so, and it was the name given to it by Minesweeper's developers.

The function is very simple:

\begin{lstlisting}[style=customc]
int Rnd(int limit)
{
    return rand() % limit;
};
\end{lstlisting}

(There is no \q{limit} name in the \gls{PDB} file; we manually named this argument like this.)

So it returns 
a random value from 0 to a specified limit.

\TT{Rnd()} is called only from one place, 
a function called \TT{StartGame()}, 
and as it seems, this is exactly 
the code which place the mines:

\begin{lstlisting}[style=customasmx86]
.text:010036C7                 push    _xBoxMac
.text:010036CD                 call    _Rnd@4          ; Rnd(x)
.text:010036D2                 push    _yBoxMac
.text:010036D8                 mov     esi, eax
.text:010036DA                 inc     esi
.text:010036DB                 call    _Rnd@4          ; Rnd(x)
.text:010036E0                 inc     eax
.text:010036E1                 mov     ecx, eax
.text:010036E3                 shl     ecx, 5          ; ECX=ECX*32
.text:010036E6                 test    _rgBlk[ecx+esi], 80h
.text:010036EE                 jnz     short loc_10036C7
.text:010036F0                 shl     eax, 5          ; EAX=EAX*32
.text:010036F3                 lea     eax, _rgBlk[eax+esi]
.text:010036FA                 or      byte ptr [eax], 80h
.text:010036FD                 dec     _cBombStart
.text:01003703                 jnz     short loc_10036C7
\end{lstlisting}

Minesweeper allows you to set the board size, so the X (xBoxMac) and Y (yBoxMac) of the board are global variables.
They are passed to \TT{Rnd()} and random 
coordinates are generated.
A mine is placed by the \TT{OR} instruction at \TT{0x010036FA}. 
And if it has been placed before 
(it's possible if the pair of \TT{Rnd()} 
generates a coordinates pair which has been already 
generated), 
then \TT{TEST} and \TT{JNZ} at \TT{0x010036E6} 
jumps to the generation routine again.

\TT{cBombStart} is the global variable containing total number of mines. So this is loop.

The width of the array is 32 
(we can conclude this by looking at the \TT{SHL} instruction, which multiplies one of the coordinates by 32).

The size of the \TT{rgBlk} 
global array can be easily determined by the difference 
between the \TT{rgBlk} 
label in the data segment and the next known one. 
It is 0x360 (864):

\begin{lstlisting}[style=customasmx86]
.data:01005340 _rgBlk          db 360h dup(?)          ; DATA XREF: MainWndProc(x,x,x,x)+574
.data:01005340                                         ; DisplayBlk(x,x)+23
.data:010056A0 _Preferences    dd ?                    ; DATA XREF: FixMenus()+2
...
\end{lstlisting}

$864/32=27$.

So the array size is $27*32$?
It is close to what we know: when we try to set board size to $100*100$ in Minesweeper settings, it fallbacks to a board of size $24*30$.
So this is the maximal board size here.
And the array has a fixed size for any board size.

So let's see all this in \olly.
We will ran Minesweeper, attaching \olly to it and now we can see the memory dump at the address of the \TT{rgBlk} array (\TT{0x01005340})
\footnote{All addresses here are for Minesweeper for Windows XP SP3 English. 
They may differ for other service packs.}.

So we got this memory dump of the array:

\lstinputlisting[style=customasmx86]{examples/minesweeper/1.lst}

\olly, like any other hexadecimal editor, shows 16 bytes per line.
So each 32-byte array row occupies exactly 2 lines here.

This is beginner level (9*9 board).

There is some square 
structure can be seen visually (0x10 bytes).

We will click \q{Run} in \olly to unfreeze the Minesweeper process, then we'll clicked randomly at the Minesweeper window 
and trapped into mine, but now all mines are visible:

\begin{figure}[H]
\centering
\myincludegraphicsSmall{examples/minesweeper/1.png}
\caption{Mines}
\label{fig:minesweeper1}
\end{figure}

By comparing the mine places and the dump, we can conclude that 0x10 stands for border, 0x0F---empty block, 0x8F---mine.
Perhaps, 0x10 is just a \emph{sentinel value}.

Now we'll add comments and also enclose all 0x8F bytes into square brackets:

\lstinputlisting[style=customasmx86]{examples/minesweeper/2.lst}

Now we'll remove all \emph{border bytes} (0x10) and what's beyond those:

\lstinputlisting[style=customasmx86]{examples/minesweeper/3.lst}

Yes, these are mines, now it can be clearly seen and compared with the screenshot.

\clearpage
What is interesting is that we can modify the array right in \olly.
We can remove all mines by changing all 0x8F bytes by 0x0F, and here is what we'll get in Minesweeper:

\begin{figure}[H]
\centering
\myincludegraphicsSmall{examples/minesweeper/3.png}
\caption{All mines are removed in debugger}
\label{fig:minesweeper3}
\end{figure}

We can also move all of them to the first line: 

\begin{figure}[H]
\centering
\myincludegraphicsSmall{examples/minesweeper/2.png}
\caption{Mines set in debugger}
\label{fig:minesweeper2}
\end{figure}

Well, the debugger is not very convenient for eavesdropping (which is our goal anyway), so we'll write a small utility
to dump the contents of the board:

\lstinputlisting[style=customc]{examples/minesweeper/minesweeper_cheater.c}

Just set the \ac{PID}
\footnote{PID it can be seen in Task Manager 
(enable it in \q{View $\rightarrow$ Select Columns})} 
and the address of the array (\TT{0x01005340} for Windows XP SP3 English) 
and it will dump it
\footnote{The compiled executable is here: 
\href{http://go.yurichev.com/17165}{beginners.re}}.

It attaches itself to a win32 process by \ac{PID} and just reads process memory at the address.

\subsection{Finding grid automatically}

This is kind of nuisance to set address each time when we run our utility.
Also, various Minesweeper versions may have the array on different address.
Knowing the fact that there is always a border (0x10 bytes), we can just find it in memory:

\lstinputlisting[style=customc]{examples/minesweeper/cheater2_fragment.c}

Full source code: \url{https://github.com/DennisYurichev/RE-for-beginners/blob/master/examples/minesweeper/minesweeper_cheater2.c}.

\subsection{\Exercises}

\begin{itemize}

\item 
Why do the \emph{border bytes} (or \emph{sentinel values}) (0x10) exist in the array?

What they are for if they are not visible in Minesweeper's interface?
How could it work without them?

\item 
As it turns out, there are more values possible (for open blocks, for flagged by user, \etc{}).
Try to find the meaning of each one.

\item 
Modify my utility so it can remove all mines or set them in a fixed pattern that you want in the Minesweeper
process currently running.

\end{itemize}
}
\RU{\mysection{Разгон майнера биткоинов Cointerra}
\index{Bitcoin}
\index{BeagleBone}

Был такой майнер биткоинов Cointerra, выглядящий так:

\begin{figure}[H]
\centering
\myincludegraphics{examples/bitcoin_miner/board.jpg}
\caption{Board}
\end{figure}

И была также (возможно утекшая) утилита\footnote{Можно скачать здесь: \url{https://github.com/DennisYurichev/RE-for-beginners/raw/master/examples/bitcoin_miner/files/cointool-overclock}}
которая могла выставлять тактовую частоту платы.
Она запускается на дополнительной плате BeagleBone на ARM с Linux (маленькая плата внизу фотографии).

И у автора (этих строк) однажды спросили, можно ли хакнуть эту утилиту и посмотреть, какие частоты можно выставлять, и какие нет.
И можно ли твикнуть её?

Утилиту нужно запускать так: \TT{./cointool-overclock 0 0 900}, где 900 это частота в МГц.
Если частота слишком большая, утилита выведет ошибку \q{Error with arguments} и закончит работу.

Вот фрагмент кода вокруг ссылки на текстовую строку \q{Error with arguments}:

\begin{lstlisting}[style=customasmARM]

...

.text:0000ABC4         STR      R3, [R11,#var_28]
.text:0000ABC8         MOV      R3, #optind
.text:0000ABD0         LDR      R3, [R3]
.text:0000ABD4         ADD      R3, R3, #1
.text:0000ABD8         MOV      R3, R3,LSL#2
.text:0000ABDC         LDR      R2, [R11,#argv]
.text:0000ABE0         ADD      R3, R2, R3
.text:0000ABE4         LDR      R3, [R3]
.text:0000ABE8         MOV      R0, R3  ; nptr
.text:0000ABEC         MOV      R1, #0  ; endptr
.text:0000ABF0         MOV      R2, #0  ; base
.text:0000ABF4         BL       strtoll
.text:0000ABF8         MOV      R2, R0
.text:0000ABFC         MOV      R3, R1
.text:0000AC00         MOV      R3, R2
.text:0000AC04         STR      R3, [R11,#var_2C]
.text:0000AC08         MOV      R3, #optind
.text:0000AC10         LDR      R3, [R3]
.text:0000AC14         ADD      R3, R3, #2
.text:0000AC18         MOV      R3, R3,LSL#2
.text:0000AC1C         LDR      R2, [R11,#argv]
.text:0000AC20         ADD      R3, R2, R3
.text:0000AC24         LDR      R3, [R3]
.text:0000AC28         MOV      R0, R3  ; nptr
.text:0000AC2C         MOV      R1, #0  ; endptr
.text:0000AC30         MOV      R2, #0  ; base
.text:0000AC34         BL       strtoll
.text:0000AC38         MOV      R2, R0
.text:0000AC3C         MOV      R3, R1
.text:0000AC40         MOV      R3, R2
.text:0000AC44         STR      R3, [R11,#third_argument]
.text:0000AC48         LDR      R3, [R11,#var_28]
.text:0000AC4C         CMP      R3, #0
.text:0000AC50         BLT      errors_with_arguments
.text:0000AC54         LDR      R3, [R11,#var_28]
.text:0000AC58         CMP      R3, #1
.text:0000AC5C         BGT      errors_with_arguments
.text:0000AC60         LDR      R3, [R11,#var_2C]
.text:0000AC64         CMP      R3, #0
.text:0000AC68         BLT      errors_with_arguments
.text:0000AC6C         LDR      R3, [R11,#var_2C]
.text:0000AC70         CMP      R3, #3
.text:0000AC74         BGT      errors_with_arguments
.text:0000AC78         LDR      R3, [R11,#third_argument]
.text:0000AC7C         CMP      R3, #0x31
.text:0000AC80         BLE      errors_with_arguments
.text:0000AC84         LDR      R2, [R11,#third_argument]
.text:0000AC88         MOV      R3, #950
.text:0000AC8C         CMP      R2, R3
.text:0000AC90         BGT      errors_with_arguments
.text:0000AC94         LDR      R2, [R11,#third_argument]
.text:0000AC98         MOV      R3, #0x51EB851F
.text:0000ACA0         SMULL    R1, R3, R3, R2
.text:0000ACA4         MOV      R1, R3,ASR#4
.text:0000ACA8         MOV      R3, R2,ASR#31
.text:0000ACAC         RSB      R3, R3, R1
.text:0000ACB0         MOV      R1, #50
.text:0000ACB4         MUL      R3, R1, R3
.text:0000ACB8         RSB      R3, R3, R2
.text:0000ACBC         CMP      R3, #0
.text:0000ACC0         BEQ      loc_ACEC
.text:0000ACC4
.text:0000ACC4 errors_with_arguments
.text:0000ACC4                                         
.text:0000ACC4         LDR      R3, [R11,#argv]
.text:0000ACC8         LDR      R3, [R3]
.text:0000ACCC         MOV      R0, R3  ; path
.text:0000ACD0         BL       __xpg_basename
.text:0000ACD4         MOV      R3, R0
.text:0000ACD8         MOV      R0, #aSErrorWithArgu ; format
.text:0000ACE0         MOV      R1, R3
.text:0000ACE4         BL       printf
.text:0000ACE8         B        loc_ADD4
.text:0000ACEC ; ------------------------------------------------------------
.text:0000ACEC
.text:0000ACEC loc_ACEC                 ; CODE XREF: main+66C
.text:0000ACEC         LDR      R2, [R11,#third_argument]
.text:0000ACF0         MOV      R3, #499
.text:0000ACF4         CMP      R2, R3
.text:0000ACF8         BGT      loc_AD08
.text:0000ACFC         MOV      R3, #0x64
.text:0000AD00         STR      R3, [R11,#unk_constant]
.text:0000AD04         B        jump_to_write_power
.text:0000AD08 ; ------------------------------------------------------------
.text:0000AD08
.text:0000AD08 loc_AD08                 ; CODE XREF: main+6A4
.text:0000AD08         LDR      R2, [R11,#third_argument]
.text:0000AD0C         MOV      R3, #799
.text:0000AD10         CMP      R2, R3
.text:0000AD14         BGT      loc_AD24
.text:0000AD18         MOV      R3, #0x5F
.text:0000AD1C         STR      R3, [R11,#unk_constant]
.text:0000AD20         B        jump_to_write_power
.text:0000AD24 ; ------------------------------------------------------------
.text:0000AD24
.text:0000AD24 loc_AD24                 ; CODE XREF: main+6C0
.text:0000AD24         LDR      R2, [R11,#third_argument]
.text:0000AD28         MOV      R3, #899
.text:0000AD2C         CMP      R2, R3
.text:0000AD30         BGT      loc_AD40
.text:0000AD34         MOV      R3, #0x5A
.text:0000AD38         STR      R3, [R11,#unk_constant]
.text:0000AD3C         B        jump_to_write_power
.text:0000AD40 ; ------------------------------------------------------------
.text:0000AD40
.text:0000AD40 loc_AD40                 ; CODE XREF: main+6DC
.text:0000AD40         LDR      R2, [R11,#third_argument]
.text:0000AD44         MOV      R3, #999
.text:0000AD48         CMP      R2, R3
.text:0000AD4C         BGT      loc_AD5C
.text:0000AD50         MOV      R3, #0x55
.text:0000AD54         STR      R3, [R11,#unk_constant]
.text:0000AD58         B        jump_to_write_power
.text:0000AD5C ; ------------------------------------------------------------
.text:0000AD5C
.text:0000AD5C loc_AD5C                 ; CODE XREF: main+6F8
.text:0000AD5C         LDR      R2, [R11,#third_argument]
.text:0000AD60         MOV      R3, #1099
.text:0000AD64         CMP      R2, R3
.text:0000AD68         BGT      jump_to_write_power
.text:0000AD6C         MOV      R3, #0x50
.text:0000AD70         STR      R3, [R11,#unk_constant]
.text:0000AD74
.text:0000AD74 jump_to_write_power                     ; CODE XREF: main+6B0
.text:0000AD74                                         ; main+6CC ...
.text:0000AD74         LDR      R3, [R11,#var_28]
.text:0000AD78         UXTB     R1, R3
.text:0000AD7C         LDR      R3, [R11,#var_2C]
.text:0000AD80         UXTB     R2, R3
.text:0000AD84         LDR      R3, [R11,#unk_constant]
.text:0000AD88         UXTB     R3, R3
.text:0000AD8C         LDR      R0, [R11,#third_argument]
.text:0000AD90         UXTH     R0, R0
.text:0000AD94         STR      R0, [SP,#0x44+var_44]
.text:0000AD98         LDR      R0, [R11,#var_24]
.text:0000AD9C         BL       write_power
.text:0000ADA0         LDR      R0, [R11,#var_24]
.text:0000ADA4         MOV      R1, #0x5A
.text:0000ADA8         BL       read_loop
.text:0000ADAC         B        loc_ADD4

...

.rodata:0000B378 aSErrorWithArgu DCB "%s: Error with arguments",0xA,0 ; DATA XREF: main+684

...

\end{lstlisting}

Имена ф-ций присутствовали в отладочной информации в оригинальном исполняемом файле,
такие как \TT{write\_power}, \TT{read\_loop}.
Но имена меткам внутри ф-ции дал я.

\myindex{UNIX!getopt}
\myindex{strtoll()}
Имя \TT{optind} звучит знакомо. Это библиотека \IT{getopt} из *NIX предназначенная для парсинга командной строки ---
и это то, что внутри и происходит.
Затем, третий аргумент (где передается значение частоты) конвертируется из строку в число используя вызов ф-ции \IT{strtoll()}.

Значение затем сравнивается с разными константами.
На 0xACEC есть проверка, меньше ли оно или равно 499, и если это так, то 0x64 будет передано в ф-цию
\TT{write\_power()} (которая посылает команду через USB используя \TT{send\_msg()}).
Если значение больше 499, происходит переход на 0xAD08.

На 0xAD08 есть проверка, меньше ли оно или равно 799. Если это так, то 0x5F передается в ф-цию \TT{write\_power()}.

Есть еще проверки: на 899 на 0xAD24, на 0x999 на 0xAD40, и наконец, на 1099 на 0xAD5C.
Если входная частота меньше или равна 1099, 0x50 (на 0xAD6C) будет передано в ф-цию \TT{write\_power()}.
И тут что-то вроде баги.
Если значение все еще больше 1099, само значение будет передано в ф-цию \TT{write\_power()}.
Но с другой стороны это не бага, потому что мы не можем попасть сюда: значение в начале проверяется с 950 на 0xAC88,
и если оно больше, выводится сообщение об ошибке и утилита заканчивает работу.

Вот таблица между частотами в МГц и значениями передаваемыми в ф-цию \TT{write\_power()}:

\begin{center}
\begin{longtable}{ | l | l | l | }
\hline
\HeaderColor МГц & \HeaderColor шестнадцатеричное представление & \HeaderColor десятичное \\
\hline
499MHz & 0x64 & 100 \\
\hline
799MHz & 0x5f & 95 \\
\hline
899MHz & 0x5a & 90 \\
\hline
999MHz & 0x55 & 85 \\
\hline
1099MHz & 0x50 & 80 \\
\hline
\end{longtable}
\end{center}

Как видно, значение передаваемое в плату постепенно уменьшается с ростом частоты.

Видно что значение в 950МГц это жесткий предел, по крайней мере в этой утилите. Можно ли её обмануть?

Вернемся к этому фрагменту кода:

\begin{lstlisting}[style=customasmARM]
.text:0000AC84      LDR     R2, [R11,#third_argument]
.text:0000AC88      MOV     R3, #950
.text:0000AC8C      CMP     R2, R3
.text:0000AC90      BGT     errors_with_arguments ; Я пропатчил здесь на 00 00 00 00
\end{lstlisting}

Нам нужно как-то запретить инструкцию перехода \INS{BGT} на 0xAC90. И это ARM в режиме ARM, потому что, как мы видим,
все адреса увеличиваются на 4, т.е., длина каждой инструкции это 4 байта.
Инструкция \TT{NOP} (нет операции) в режиме ARM это просто 4 нулевых байта: \TT{00 00 00 00}.
Так что, записывая 4 нуля по адресу 0xAC90 (или по физическому смещению в файле: 0x2C90) мы можем выключить
эту проверку.

Теперь можно выставлять частоты вплоть до 1050МГц. И даже больше, но из-за ошибки, если входное значение больше 1099,
значение в МГц, \IT{как есть}, будет передано в плату, что неправильно.

Дальше я не разбирался, но если бы продолжил, я бы уменьшал значение передаваемое в ф-цию \TT{write\_power()}.

Теперь страшный фрагмент кода, который я в начале пропустил:

\lstinputlisting[style=customasmARM]{examples/bitcoin_miner/tmp1.lst}

Здесь используется деление через умножение, и константа 0x51EB851F.
Я написал для себя простой программистский калькулятор\footnote{\url{https://github.com/DennisYurichev/progcalc}}.
И там есть возможность вычислять обратное число по модулю.

\begin{lstlisting}
modinv32(0x51EB851F)
Warning, result is not integer: 3.125000
(unsigned) dec: 3 hex: 0x3 bin: 11
\end{lstlisting}

Это значит что инструкция \INS{SMULL} на 0xACA0 просто делит 3-й аргумент на 3.125.
На самом деле, все что делает ф-ция \TT{modinv32()} в моем калькуляторе, это:

\[
\frac{1}{\frac{input}{2^{32}}} = \frac{2^{32}}{input}
\]

Потом там есть дополнительные сдвиги и теперь мы видим что 3-й аргумент просто делится на 50.
И затем умножается снова на 50.
Зачем?
Это простейшая проверка, можно ли делить входное значение на 50 без остатка.
Если значение этого выражения ненулевое, $x$ не может быть разделено на 50 без остатка:

\[
x-((\frac{x}{50}) \cdot 50)
\]

На самом деле, это простой способ вычисления остатка от деления.

И затем, если остаток ненулевой, выводится сообщение об ошибке.
Так что эта утилита берет значения частотв вроде 850, 900, 950, 1000, итд, но не 855 или 911.

Вот и всё! Если вы делаете что-то такое, имейте ввиду, что это может испортить вашу плату, как и в случае разгона
чипов вроде \ac{CPU}, \ac{GPU}, итд.
Если у вас есть плата Cointerra, делайте всё это на свой собственный риск!

}
\DE{\subsection{Betrag berechnen}
\label{sec:abs}

Eine einfache Funktion:

\lstinputlisting[style=customc]{abs.c}

\subsubsection{\Optimizing MSVC}

Normalerweise wird folgender Code erzeugt:

\lstinputlisting[caption=\Optimizing MSVC 2012 x64,style=customasmx86]{patterns/07_jcc/abs/abs_MSVC2012_Ox_x64_DE.asm}

GCC 4.9 macht ungefähr das gleiche.

\subsubsection{\OptimizingKeilVI: \ThumbMode}

\lstinputlisting[caption=\OptimizingKeilVI: \ThumbMode,style=customasmARM]{patterns/07_jcc/abs/abs_Keil_thumb_O3_DE.s}

\myindex{ARM!\Instructions!RSB}
ARM fehlt ein Befehl zur Negation, sodass der Keil Compiler den \q{Reverse
Subtract} Befehl verwendet, der mit umgekehrten Operanden subtrahiert.

\subsubsection{\OptimizingKeilVI: \ARMMode}
Es ist im ARM mode möglich, einigen Befehlen condition codes hinzuzufügen und genau das tut der Keil Compiler:

\lstinputlisting[caption=\OptimizingKeilVI: \ARMMode,style=customasmARM]{patterns/07_jcc/abs/abs_Keil_ARM_O3_DE.s}
Jetzt sind keine bedingten Sprünge mehr übrig und das ist vorteilhaft: \myref{branch_predictors}.

\subsubsection{\NonOptimizing GCC 4.9 (ARM64)}

\myindex{ARM!\Instructions!XOR}

ARM64 kennt den Befehl \INS{NEG} zum Negieren:

\lstinputlisting[caption=\Optimizing GCC 4.9 (ARM64),style=customasmARM]{patterns/07_jcc/abs/abs_GCC49_ARM64_O0_DE.s}

\subsubsection{MIPS}

\lstinputlisting[caption=\Optimizing GCC 4.4.5 (IDA),style=customasmMIPS]{patterns/07_jcc/abs/MIPS_O3_IDA_DE.lst}

\myindex{MIPS!\Instructions!BLTZ}
Hier finden wir einen neuen Befehl: \INS{BLTZ} (\q{Branch if Less Than Zero}).
\myindex{MIPS!\Instructions!SUBU}
\myindex{MIPS!\Pseudoinstructions!NEGU}
Es gibt zusätzlich noch den \INS{NEGU} Pseudo-Befehl, der eine Subtraktion von Null durchführt. Der Suffix \q{U} bei
\INS{SUBU} und \INS{NEGU} zeigt an, dass keine Exception für den Fall eines Integer Overflows geworfen wird.


\subsubsection{Verzweigungslose Version?}
Man kann auch eine verzweigungslose Version dieses Codes erzeugen. Dies werden wir später betrachten:
\myref{chap:branchless_abs}. 
}
\FR{\subsection{MIPS}

\subsubsection{3 arguments}

\myparagraph{GCC 4.4.5 \Optimizing}

La différence principale avec l'exemple \q{\HelloWorldSectionName} est que dans ce cas, \printf est appelée
à la place de \puts et 3 arguments de plus sont passés à travers les registres \$5\dots \$7 (ou \$A0\dots \$A2).
C'est pourquoi ces registres sont préfixés avec A-, ceci sous-entend qu'ils
sont utilisés pour le passage des arguments aux fonctions.

\lstinputlisting[caption=GCC 4.4.5 \Optimizing (\assemblyOutput),style=customasmMIPS]{patterns/03_printf/MIPS/printf3.O3_FR.s}

\lstinputlisting[caption=GCC 4.4.5 \Optimizing (IDA),style=customasmMIPS]{patterns/03_printf/MIPS/printf3.O3.IDA_FR.lst}

\IDA a agrégé la paire d'instructions \INS{LUI} et \INS{ADDIU} en une pseudo instruction \INS{LA}.
C'est pourquoi il n'y a pas d'instruction à l'adresse 0x1C: car \INS{LA} \emph{occupe} 8 octets.

\myparagraph{GCC 4.4.5 \NonOptimizing}

GCC \NonOptimizing est plus verbeux:

\lstinputlisting[caption=GCC 4.4.5 \NonOptimizing (\assemblyOutput),style=customasmMIPS]{patterns/03_printf/MIPS/printf3.O0_FR.s}

\lstinputlisting[caption=GCC 4.4.5 \NonOptimizing (IDA),style=customasmMIPS]{patterns/03_printf/MIPS/printf3.O0.IDA_FR.lst}

\subsubsection{8 arguments}

Utilisons encore l'exemple de la section précédente avec 9 arguments: \myref{example_printf8_x64}.

\lstinputlisting[style=customc]{patterns/03_printf/2.c}

\myparagraph{GCC 4.4.5 \Optimizing}

Seul les 4 premiers arguments sont passés dans les registres \$A0 \dots \$A3,
les autres sont passés par la pile.
\myindex{MIPS!O32}

C'est la convention d'appel O32 (qui est la plus commune dans le monde MIPS).
D'autres conventions d'appel (comme N32) peuvent utiliser les registres à d'autres fins.

\myindex{MIPS!\Instructions!SW}

\INS{SW} est l'abbréviation de \q{Store Word} (depuis un registre vers la mémoire).
En MIPS, il manque une instructions pour stocker une valeur dans la mémoire, donc
une paire d'instruction doit être utilisée à la place (\INS{LI}/\INS{SW}).

\lstinputlisting[caption=GCC 4.4.5 \Optimizing (\assemblyOutput),style=customasmMIPS]{patterns/03_printf/MIPS/printf8.O3_FR.s}

\lstinputlisting[caption=GCC 4.4.5 \Optimizing (IDA),style=customasmMIPS]{patterns/03_printf/MIPS/printf8.O3.IDA_FR.lst}

\myparagraph{GCC 4.4.5 \NonOptimizing}

GCC \NonOptimizing est plus verbeux:

\lstinputlisting[caption=\NonOptimizing GCC 4.4.5 (\assemblyOutput),style=customasmMIPS]{patterns/03_printf/MIPS/printf8.O0_FR.s}

\lstinputlisting[caption=\NonOptimizing GCC 4.4.5 (IDA),style=customasmMIPS]{patterns/03_printf/MIPS/printf8.O0.IDA_FR.lst}

}
\IT{\subsection{Calcolo del valore assoluto}
\label{sec:abs}

Una funzione semplice:

\lstinputlisting[style=customc]{abs.c}

\subsubsection{\Optimizing MSVC}

Il codice solitamente generato è questo:

\lstinputlisting[caption=\Optimizing MSVC 2012 x64,style=customasmx86]{patterns/07_jcc/abs/abs_MSVC2012_Ox_x64_EN.asm}

GCC 4.9 fa più o meno lo stesso.

\subsubsection{\OptimizingKeilVI: \ThumbMode}

\lstinputlisting[caption=\OptimizingKeilVI: \ThumbMode,style=customasmARM]{patterns/07_jcc/abs/abs_Keil_thumb_O3_EN.s}

\myindex{ARM!\Instructions!RSB}

In ARM manca l'istruzione di negazione, quindi il compilatore Keil usa l'istruzione \q{Reverse Subtract}, che semplicemente sottrae gli operandi in modo inverso.

\subsubsection{\OptimizingKeilVI: \ARMMode}

In modalità ARM è possibile agguingere condition codes ad alcune istruzioni, e questo è ciò che fa il compilatore keil:

\lstinputlisting[caption=\OptimizingKeilVI: \ARMMode,style=customasmARM]{patterns/07_jcc/abs/abs_Keil_ARM_O3_EN.s}

Adesso non ci sono più jump condizionali, e ciò è bene: \myref{branch_predictors}.

\subsubsection{\NonOptimizing GCC 4.9 (ARM64)}

\myindex{ARM!\Instructions!XOR}

ARM64 ha un'istruzione \INS{NEG} per la negazione:

\lstinputlisting[caption=\Optimizing GCC 4.9 (ARM64),style=customasmARM]{patterns/07_jcc/abs/abs_GCC49_ARM64_O0_EN.s}

\subsubsection{MIPS}

\lstinputlisting[caption=\Optimizing GCC 4.4.5 (IDA),style=customasmMIPS]{patterns/07_jcc/abs/MIPS_O3_IDA_EN.lst}

\myindex{MIPS!\Instructions!BLTZ}
Qui vediamo una nuova istruzione: \INS{BLTZ} (\q{Branch if Less Than Zero}).
\myindex{MIPS!\Instructions!SUBU}
\myindex{MIPS!\Pseudoinstructions!NEGU}

C'è anche la pseudoistruzione \INS{NEGU} , che semplicemente fa la sottrazione da zero.
Il suffisso \q{U} suffix in entrambe \INS{SUBU} e \INS{NEGU} implica che non verrà sollevata nessuna eccezione in caso di integer overflow.

\subsubsection{Branchless version?}

You could have also a branchless version of this code. This we will review later: \myref{chap:branchless_abs}.
}
\JA{\subsection{絶対値の計算}
\label{sec:abs}

簡単な関数の例。

\lstinputlisting[style=customc]{abs.c}

\subsubsection{\Optimizing MSVC}

これは普通、どのようにコードが生成されるのかを示したものです。

\lstinputlisting[caption=\Optimizing MSVC 2012 x64,style=customasmx86]{patterns/07_jcc/abs/abs_MSVC2012_Ox_x64_JA.asm}

GCC 4.9はほとんど同じです。

\subsubsection{\OptimizingKeilVI: \ThumbMode}

\lstinputlisting[caption=\OptimizingKeilVI: \ThumbMode,style=customasmARM]{patterns/07_jcc/abs/abs_Keil_thumb_O3_JA.s}

\myindex{ARM!\Instructions!RSB}

ARMにはネゲート命令がないため、Keilコンパイラは\q{逆引き命令}を使用します。これは逆のオペランドで減算するだけです。

\subsubsection{\OptimizingKeilVI: \ARMMode}

ARMモードでは、いくつかの命令に条件コードを追加することができます。そのため、Keilコンパイラは次のように処理します。

\lstinputlisting[caption=\OptimizingKeilVI: \ARMMode,style=customasmARM]{patterns/07_jcc/abs/abs_Keil_ARM_O3_JA.s}

今度は条件付きジャンプはありません。これは良いですね。:\myref{branch_predictors}

\subsubsection{\NonOptimizing GCC 4.9 (ARM64)}

\myindex{ARM!\Instructions!XOR}

ARM64には、否定するための命令\INS{NEG}があります。

\lstinputlisting[caption=\Optimizing GCC 4.9 (ARM64),style=customasmARM]{patterns/07_jcc/abs/abs_GCC49_ARM64_O0_JA.s}

\subsubsection{MIPS}

\lstinputlisting[caption=\Optimizing GCC 4.4.5 (IDA),style=customasmMIPS]{patterns/07_jcc/abs/MIPS_O3_IDA_JA.lst}

\myindex{MIPS!\Instructions!BLTZ}
ここでは\INS{BLTZ}(\q{Branch if Less Than Zero})という新しい命令があります。
\myindex{MIPS!\Instructions!SUBU}
\myindex{MIPS!\Pseudoinstructions!NEGU}

\INS{NEGU}擬似命令もあります。これはゼロからの減算だけです。 
\INS{SUBU}と\INS{NEGU}の両方の \q{U}接尾辞は、整数オーバーフローの場合に発生する例外がないことを意味します。

\subsubsection{Branchless version?}

このコードを分岐がないバージョンにすることもできます。 これについては、後述の\myref{chap:branchless_abs}を参照してください。
}

\EN{\mysection{\MinesweeperWinXPExampleChapterName}
\label{minesweeper_winxp}
\myindex{Windows!Windows XP}

For those who are not very good at playing Minesweeper, we could try to reveal the hidden mines in the debugger.

\myindex{\CStandardLibrary!rand()}
\myindex{Windows!PDB}

As we know, Minesweeper places mines randomly, so there has to be some kind of random number generator or
a call to the standard \TT{rand()} C-function.

What is really cool about reversing Microsoft products is that there are \gls{PDB} 
file with symbols (function names, \etc{}).
When we load \TT{winmine.exe} into \IDA, it downloads the 
\gls{PDB} file exactly for this 
executable and shows all names.

So here it is, the only call to \TT{rand()} is this function:

\lstinputlisting[style=customasmx86]{examples/minesweeper/tmp1.lst}

\IDA named it so, and it was the name given to it by Minesweeper's developers.

The function is very simple:

\begin{lstlisting}[style=customc]
int Rnd(int limit)
{
    return rand() % limit;
};
\end{lstlisting}

(There is no \q{limit} name in the \gls{PDB} file; we manually named this argument like this.)

So it returns 
a random value from 0 to a specified limit.

\TT{Rnd()} is called only from one place, 
a function called \TT{StartGame()}, 
and as it seems, this is exactly 
the code which place the mines:

\begin{lstlisting}[style=customasmx86]
.text:010036C7                 push    _xBoxMac
.text:010036CD                 call    _Rnd@4          ; Rnd(x)
.text:010036D2                 push    _yBoxMac
.text:010036D8                 mov     esi, eax
.text:010036DA                 inc     esi
.text:010036DB                 call    _Rnd@4          ; Rnd(x)
.text:010036E0                 inc     eax
.text:010036E1                 mov     ecx, eax
.text:010036E3                 shl     ecx, 5          ; ECX=ECX*32
.text:010036E6                 test    _rgBlk[ecx+esi], 80h
.text:010036EE                 jnz     short loc_10036C7
.text:010036F0                 shl     eax, 5          ; EAX=EAX*32
.text:010036F3                 lea     eax, _rgBlk[eax+esi]
.text:010036FA                 or      byte ptr [eax], 80h
.text:010036FD                 dec     _cBombStart
.text:01003703                 jnz     short loc_10036C7
\end{lstlisting}

Minesweeper allows you to set the board size, so the X (xBoxMac) and Y (yBoxMac) of the board are global variables.
They are passed to \TT{Rnd()} and random 
coordinates are generated.
A mine is placed by the \TT{OR} instruction at \TT{0x010036FA}. 
And if it has been placed before 
(it's possible if the pair of \TT{Rnd()} 
generates a coordinates pair which has been already 
generated), 
then \TT{TEST} and \TT{JNZ} at \TT{0x010036E6} 
jumps to the generation routine again.

\TT{cBombStart} is the global variable containing total number of mines. So this is loop.

The width of the array is 32 
(we can conclude this by looking at the \TT{SHL} instruction, which multiplies one of the coordinates by 32).

The size of the \TT{rgBlk} 
global array can be easily determined by the difference 
between the \TT{rgBlk} 
label in the data segment and the next known one. 
It is 0x360 (864):

\begin{lstlisting}[style=customasmx86]
.data:01005340 _rgBlk          db 360h dup(?)          ; DATA XREF: MainWndProc(x,x,x,x)+574
.data:01005340                                         ; DisplayBlk(x,x)+23
.data:010056A0 _Preferences    dd ?                    ; DATA XREF: FixMenus()+2
...
\end{lstlisting}

$864/32=27$.

So the array size is $27*32$?
It is close to what we know: when we try to set board size to $100*100$ in Minesweeper settings, it fallbacks to a board of size $24*30$.
So this is the maximal board size here.
And the array has a fixed size for any board size.

So let's see all this in \olly.
We will ran Minesweeper, attaching \olly to it and now we can see the memory dump at the address of the \TT{rgBlk} array (\TT{0x01005340})
\footnote{All addresses here are for Minesweeper for Windows XP SP3 English. 
They may differ for other service packs.}.

So we got this memory dump of the array:

\lstinputlisting[style=customasmx86]{examples/minesweeper/1.lst}

\olly, like any other hexadecimal editor, shows 16 bytes per line.
So each 32-byte array row occupies exactly 2 lines here.

This is beginner level (9*9 board).

There is some square 
structure can be seen visually (0x10 bytes).

We will click \q{Run} in \olly to unfreeze the Minesweeper process, then we'll clicked randomly at the Minesweeper window 
and trapped into mine, but now all mines are visible:

\begin{figure}[H]
\centering
\myincludegraphicsSmall{examples/minesweeper/1.png}
\caption{Mines}
\label{fig:minesweeper1}
\end{figure}

By comparing the mine places and the dump, we can conclude that 0x10 stands for border, 0x0F---empty block, 0x8F---mine.
Perhaps, 0x10 is just a \emph{sentinel value}.

Now we'll add comments and also enclose all 0x8F bytes into square brackets:

\lstinputlisting[style=customasmx86]{examples/minesweeper/2.lst}

Now we'll remove all \emph{border bytes} (0x10) and what's beyond those:

\lstinputlisting[style=customasmx86]{examples/minesweeper/3.lst}

Yes, these are mines, now it can be clearly seen and compared with the screenshot.

\clearpage
What is interesting is that we can modify the array right in \olly.
We can remove all mines by changing all 0x8F bytes by 0x0F, and here is what we'll get in Minesweeper:

\begin{figure}[H]
\centering
\myincludegraphicsSmall{examples/minesweeper/3.png}
\caption{All mines are removed in debugger}
\label{fig:minesweeper3}
\end{figure}

We can also move all of them to the first line: 

\begin{figure}[H]
\centering
\myincludegraphicsSmall{examples/minesweeper/2.png}
\caption{Mines set in debugger}
\label{fig:minesweeper2}
\end{figure}

Well, the debugger is not very convenient for eavesdropping (which is our goal anyway), so we'll write a small utility
to dump the contents of the board:

\lstinputlisting[style=customc]{examples/minesweeper/minesweeper_cheater.c}

Just set the \ac{PID}
\footnote{PID it can be seen in Task Manager 
(enable it in \q{View $\rightarrow$ Select Columns})} 
and the address of the array (\TT{0x01005340} for Windows XP SP3 English) 
and it will dump it
\footnote{The compiled executable is here: 
\href{http://go.yurichev.com/17165}{beginners.re}}.

It attaches itself to a win32 process by \ac{PID} and just reads process memory at the address.

\subsection{Finding grid automatically}

This is kind of nuisance to set address each time when we run our utility.
Also, various Minesweeper versions may have the array on different address.
Knowing the fact that there is always a border (0x10 bytes), we can just find it in memory:

\lstinputlisting[style=customc]{examples/minesweeper/cheater2_fragment.c}

Full source code: \url{https://github.com/DennisYurichev/RE-for-beginners/blob/master/examples/minesweeper/minesweeper_cheater2.c}.

\subsection{\Exercises}

\begin{itemize}

\item 
Why do the \emph{border bytes} (or \emph{sentinel values}) (0x10) exist in the array?

What they are for if they are not visible in Minesweeper's interface?
How could it work without them?

\item 
As it turns out, there are more values possible (for open blocks, for flagged by user, \etc{}).
Try to find the meaning of each one.

\item 
Modify my utility so it can remove all mines or set them in a fixed pattern that you want in the Minesweeper
process currently running.

\end{itemize}
}
\RU{\mysection{Разгон майнера биткоинов Cointerra}
\index{Bitcoin}
\index{BeagleBone}

Был такой майнер биткоинов Cointerra, выглядящий так:

\begin{figure}[H]
\centering
\myincludegraphics{examples/bitcoin_miner/board.jpg}
\caption{Board}
\end{figure}

И была также (возможно утекшая) утилита\footnote{Можно скачать здесь: \url{https://github.com/DennisYurichev/RE-for-beginners/raw/master/examples/bitcoin_miner/files/cointool-overclock}}
которая могла выставлять тактовую частоту платы.
Она запускается на дополнительной плате BeagleBone на ARM с Linux (маленькая плата внизу фотографии).

И у автора (этих строк) однажды спросили, можно ли хакнуть эту утилиту и посмотреть, какие частоты можно выставлять, и какие нет.
И можно ли твикнуть её?

Утилиту нужно запускать так: \TT{./cointool-overclock 0 0 900}, где 900 это частота в МГц.
Если частота слишком большая, утилита выведет ошибку \q{Error with arguments} и закончит работу.

Вот фрагмент кода вокруг ссылки на текстовую строку \q{Error with arguments}:

\begin{lstlisting}[style=customasmARM]

...

.text:0000ABC4         STR      R3, [R11,#var_28]
.text:0000ABC8         MOV      R3, #optind
.text:0000ABD0         LDR      R3, [R3]
.text:0000ABD4         ADD      R3, R3, #1
.text:0000ABD8         MOV      R3, R3,LSL#2
.text:0000ABDC         LDR      R2, [R11,#argv]
.text:0000ABE0         ADD      R3, R2, R3
.text:0000ABE4         LDR      R3, [R3]
.text:0000ABE8         MOV      R0, R3  ; nptr
.text:0000ABEC         MOV      R1, #0  ; endptr
.text:0000ABF0         MOV      R2, #0  ; base
.text:0000ABF4         BL       strtoll
.text:0000ABF8         MOV      R2, R0
.text:0000ABFC         MOV      R3, R1
.text:0000AC00         MOV      R3, R2
.text:0000AC04         STR      R3, [R11,#var_2C]
.text:0000AC08         MOV      R3, #optind
.text:0000AC10         LDR      R3, [R3]
.text:0000AC14         ADD      R3, R3, #2
.text:0000AC18         MOV      R3, R3,LSL#2
.text:0000AC1C         LDR      R2, [R11,#argv]
.text:0000AC20         ADD      R3, R2, R3
.text:0000AC24         LDR      R3, [R3]
.text:0000AC28         MOV      R0, R3  ; nptr
.text:0000AC2C         MOV      R1, #0  ; endptr
.text:0000AC30         MOV      R2, #0  ; base
.text:0000AC34         BL       strtoll
.text:0000AC38         MOV      R2, R0
.text:0000AC3C         MOV      R3, R1
.text:0000AC40         MOV      R3, R2
.text:0000AC44         STR      R3, [R11,#third_argument]
.text:0000AC48         LDR      R3, [R11,#var_28]
.text:0000AC4C         CMP      R3, #0
.text:0000AC50         BLT      errors_with_arguments
.text:0000AC54         LDR      R3, [R11,#var_28]
.text:0000AC58         CMP      R3, #1
.text:0000AC5C         BGT      errors_with_arguments
.text:0000AC60         LDR      R3, [R11,#var_2C]
.text:0000AC64         CMP      R3, #0
.text:0000AC68         BLT      errors_with_arguments
.text:0000AC6C         LDR      R3, [R11,#var_2C]
.text:0000AC70         CMP      R3, #3
.text:0000AC74         BGT      errors_with_arguments
.text:0000AC78         LDR      R3, [R11,#third_argument]
.text:0000AC7C         CMP      R3, #0x31
.text:0000AC80         BLE      errors_with_arguments
.text:0000AC84         LDR      R2, [R11,#third_argument]
.text:0000AC88         MOV      R3, #950
.text:0000AC8C         CMP      R2, R3
.text:0000AC90         BGT      errors_with_arguments
.text:0000AC94         LDR      R2, [R11,#third_argument]
.text:0000AC98         MOV      R3, #0x51EB851F
.text:0000ACA0         SMULL    R1, R3, R3, R2
.text:0000ACA4         MOV      R1, R3,ASR#4
.text:0000ACA8         MOV      R3, R2,ASR#31
.text:0000ACAC         RSB      R3, R3, R1
.text:0000ACB0         MOV      R1, #50
.text:0000ACB4         MUL      R3, R1, R3
.text:0000ACB8         RSB      R3, R3, R2
.text:0000ACBC         CMP      R3, #0
.text:0000ACC0         BEQ      loc_ACEC
.text:0000ACC4
.text:0000ACC4 errors_with_arguments
.text:0000ACC4                                         
.text:0000ACC4         LDR      R3, [R11,#argv]
.text:0000ACC8         LDR      R3, [R3]
.text:0000ACCC         MOV      R0, R3  ; path
.text:0000ACD0         BL       __xpg_basename
.text:0000ACD4         MOV      R3, R0
.text:0000ACD8         MOV      R0, #aSErrorWithArgu ; format
.text:0000ACE0         MOV      R1, R3
.text:0000ACE4         BL       printf
.text:0000ACE8         B        loc_ADD4
.text:0000ACEC ; ------------------------------------------------------------
.text:0000ACEC
.text:0000ACEC loc_ACEC                 ; CODE XREF: main+66C
.text:0000ACEC         LDR      R2, [R11,#third_argument]
.text:0000ACF0         MOV      R3, #499
.text:0000ACF4         CMP      R2, R3
.text:0000ACF8         BGT      loc_AD08
.text:0000ACFC         MOV      R3, #0x64
.text:0000AD00         STR      R3, [R11,#unk_constant]
.text:0000AD04         B        jump_to_write_power
.text:0000AD08 ; ------------------------------------------------------------
.text:0000AD08
.text:0000AD08 loc_AD08                 ; CODE XREF: main+6A4
.text:0000AD08         LDR      R2, [R11,#third_argument]
.text:0000AD0C         MOV      R3, #799
.text:0000AD10         CMP      R2, R3
.text:0000AD14         BGT      loc_AD24
.text:0000AD18         MOV      R3, #0x5F
.text:0000AD1C         STR      R3, [R11,#unk_constant]
.text:0000AD20         B        jump_to_write_power
.text:0000AD24 ; ------------------------------------------------------------
.text:0000AD24
.text:0000AD24 loc_AD24                 ; CODE XREF: main+6C0
.text:0000AD24         LDR      R2, [R11,#third_argument]
.text:0000AD28         MOV      R3, #899
.text:0000AD2C         CMP      R2, R3
.text:0000AD30         BGT      loc_AD40
.text:0000AD34         MOV      R3, #0x5A
.text:0000AD38         STR      R3, [R11,#unk_constant]
.text:0000AD3C         B        jump_to_write_power
.text:0000AD40 ; ------------------------------------------------------------
.text:0000AD40
.text:0000AD40 loc_AD40                 ; CODE XREF: main+6DC
.text:0000AD40         LDR      R2, [R11,#third_argument]
.text:0000AD44         MOV      R3, #999
.text:0000AD48         CMP      R2, R3
.text:0000AD4C         BGT      loc_AD5C
.text:0000AD50         MOV      R3, #0x55
.text:0000AD54         STR      R3, [R11,#unk_constant]
.text:0000AD58         B        jump_to_write_power
.text:0000AD5C ; ------------------------------------------------------------
.text:0000AD5C
.text:0000AD5C loc_AD5C                 ; CODE XREF: main+6F8
.text:0000AD5C         LDR      R2, [R11,#third_argument]
.text:0000AD60         MOV      R3, #1099
.text:0000AD64         CMP      R2, R3
.text:0000AD68         BGT      jump_to_write_power
.text:0000AD6C         MOV      R3, #0x50
.text:0000AD70         STR      R3, [R11,#unk_constant]
.text:0000AD74
.text:0000AD74 jump_to_write_power                     ; CODE XREF: main+6B0
.text:0000AD74                                         ; main+6CC ...
.text:0000AD74         LDR      R3, [R11,#var_28]
.text:0000AD78         UXTB     R1, R3
.text:0000AD7C         LDR      R3, [R11,#var_2C]
.text:0000AD80         UXTB     R2, R3
.text:0000AD84         LDR      R3, [R11,#unk_constant]
.text:0000AD88         UXTB     R3, R3
.text:0000AD8C         LDR      R0, [R11,#third_argument]
.text:0000AD90         UXTH     R0, R0
.text:0000AD94         STR      R0, [SP,#0x44+var_44]
.text:0000AD98         LDR      R0, [R11,#var_24]
.text:0000AD9C         BL       write_power
.text:0000ADA0         LDR      R0, [R11,#var_24]
.text:0000ADA4         MOV      R1, #0x5A
.text:0000ADA8         BL       read_loop
.text:0000ADAC         B        loc_ADD4

...

.rodata:0000B378 aSErrorWithArgu DCB "%s: Error with arguments",0xA,0 ; DATA XREF: main+684

...

\end{lstlisting}

Имена ф-ций присутствовали в отладочной информации в оригинальном исполняемом файле,
такие как \TT{write\_power}, \TT{read\_loop}.
Но имена меткам внутри ф-ции дал я.

\myindex{UNIX!getopt}
\myindex{strtoll()}
Имя \TT{optind} звучит знакомо. Это библиотека \IT{getopt} из *NIX предназначенная для парсинга командной строки ---
и это то, что внутри и происходит.
Затем, третий аргумент (где передается значение частоты) конвертируется из строку в число используя вызов ф-ции \IT{strtoll()}.

Значение затем сравнивается с разными константами.
На 0xACEC есть проверка, меньше ли оно или равно 499, и если это так, то 0x64 будет передано в ф-цию
\TT{write\_power()} (которая посылает команду через USB используя \TT{send\_msg()}).
Если значение больше 499, происходит переход на 0xAD08.

На 0xAD08 есть проверка, меньше ли оно или равно 799. Если это так, то 0x5F передается в ф-цию \TT{write\_power()}.

Есть еще проверки: на 899 на 0xAD24, на 0x999 на 0xAD40, и наконец, на 1099 на 0xAD5C.
Если входная частота меньше или равна 1099, 0x50 (на 0xAD6C) будет передано в ф-цию \TT{write\_power()}.
И тут что-то вроде баги.
Если значение все еще больше 1099, само значение будет передано в ф-цию \TT{write\_power()}.
Но с другой стороны это не бага, потому что мы не можем попасть сюда: значение в начале проверяется с 950 на 0xAC88,
и если оно больше, выводится сообщение об ошибке и утилита заканчивает работу.

Вот таблица между частотами в МГц и значениями передаваемыми в ф-цию \TT{write\_power()}:

\begin{center}
\begin{longtable}{ | l | l | l | }
\hline
\HeaderColor МГц & \HeaderColor шестнадцатеричное представление & \HeaderColor десятичное \\
\hline
499MHz & 0x64 & 100 \\
\hline
799MHz & 0x5f & 95 \\
\hline
899MHz & 0x5a & 90 \\
\hline
999MHz & 0x55 & 85 \\
\hline
1099MHz & 0x50 & 80 \\
\hline
\end{longtable}
\end{center}

Как видно, значение передаваемое в плату постепенно уменьшается с ростом частоты.

Видно что значение в 950МГц это жесткий предел, по крайней мере в этой утилите. Можно ли её обмануть?

Вернемся к этому фрагменту кода:

\begin{lstlisting}[style=customasmARM]
.text:0000AC84      LDR     R2, [R11,#third_argument]
.text:0000AC88      MOV     R3, #950
.text:0000AC8C      CMP     R2, R3
.text:0000AC90      BGT     errors_with_arguments ; Я пропатчил здесь на 00 00 00 00
\end{lstlisting}

Нам нужно как-то запретить инструкцию перехода \INS{BGT} на 0xAC90. И это ARM в режиме ARM, потому что, как мы видим,
все адреса увеличиваются на 4, т.е., длина каждой инструкции это 4 байта.
Инструкция \TT{NOP} (нет операции) в режиме ARM это просто 4 нулевых байта: \TT{00 00 00 00}.
Так что, записывая 4 нуля по адресу 0xAC90 (или по физическому смещению в файле: 0x2C90) мы можем выключить
эту проверку.

Теперь можно выставлять частоты вплоть до 1050МГц. И даже больше, но из-за ошибки, если входное значение больше 1099,
значение в МГц, \IT{как есть}, будет передано в плату, что неправильно.

Дальше я не разбирался, но если бы продолжил, я бы уменьшал значение передаваемое в ф-цию \TT{write\_power()}.

Теперь страшный фрагмент кода, который я в начале пропустил:

\lstinputlisting[style=customasmARM]{examples/bitcoin_miner/tmp1.lst}

Здесь используется деление через умножение, и константа 0x51EB851F.
Я написал для себя простой программистский калькулятор\footnote{\url{https://github.com/DennisYurichev/progcalc}}.
И там есть возможность вычислять обратное число по модулю.

\begin{lstlisting}
modinv32(0x51EB851F)
Warning, result is not integer: 3.125000
(unsigned) dec: 3 hex: 0x3 bin: 11
\end{lstlisting}

Это значит что инструкция \INS{SMULL} на 0xACA0 просто делит 3-й аргумент на 3.125.
На самом деле, все что делает ф-ция \TT{modinv32()} в моем калькуляторе, это:

\[
\frac{1}{\frac{input}{2^{32}}} = \frac{2^{32}}{input}
\]

Потом там есть дополнительные сдвиги и теперь мы видим что 3-й аргумент просто делится на 50.
И затем умножается снова на 50.
Зачем?
Это простейшая проверка, можно ли делить входное значение на 50 без остатка.
Если значение этого выражения ненулевое, $x$ не может быть разделено на 50 без остатка:

\[
x-((\frac{x}{50}) \cdot 50)
\]

На самом деле, это простой способ вычисления остатка от деления.

И затем, если остаток ненулевой, выводится сообщение об ошибке.
Так что эта утилита берет значения частотв вроде 850, 900, 950, 1000, итд, но не 855 или 911.

Вот и всё! Если вы делаете что-то такое, имейте ввиду, что это может испортить вашу плату, как и в случае разгона
чипов вроде \ac{CPU}, \ac{GPU}, итд.
Если у вас есть плата Cointerra, делайте всё это на свой собственный риск!

}
\DE{\mysection{\Stack}
\label{sec:stack}
\myindex{\Stack}

Der Stack ist eine der fundamentalen Datenstrukturen in der Informatik.
\footnote{\href{http://go.yurichev.com/17119}{wikipedia.org/wiki/Call\_Stack}}.
\ac{AKA} \ac{LIFO}.

Technisch betrachtet ist es ein Stapelspeicher innerhalb des Prozessspeichers der zusammen mit den \ESP (x86), \RSP (x64) oder dem \ac{SP} (ARM) Register als ein Zeiger in diesem Speicherblock fungiert.

\myindex{ARM!\Instructions!PUSH}
\myindex{ARM!\Instructions!POP}
\myindex{x86!\Instructions!PUSH}
\myindex{x86!\Instructions!POP}

Die häufigsten Stack-Zugriffsinstruktionen sind die \PUSH- und \POP-Instruktionen (in beidem x86 und ARM Thumb-Modus). \PUSH subtrahiert vom \ESP/\RSP/\ac{SP} 4 Byte im 32-Bit Modus (oder 8 im 64-Bit Modus) und schreibt dann den Inhalt des Zeigers an die Adresse auf die von \ESP/\RSP/\ac{SP} gezeigt wird.

\POP ist die umgekehrte Operation: Die Daten des Zeigers für die Speicherregion auf die von \ac{SP}
gezeigt wird werden ausgelesen und die Inhalte in den Instruktionsoperanden geschreiben (oft ist das ein Register). Dann werden 4 (beziehungsweise 8) Byte zum \gls{stack pointer} addiert.

Nach der Stackallokation, zeigt der \gls{stack pointer} auf den Boden des Stacks.
\PUSH verringert den \gls{stack pointer} und \POP erhöht ihn.
Der Boden des Stacks ist eigentlich der Anfang der Speicherregion die für den Stack reserviert wurde.
Das wirkt zunächst seltsam, aber so funktioniert es.

ARM unterstützt beides, aufsteigende und absteigende Stacks.

\myindex{ARM!\Instructions!STMFD}
\myindex{ARM!\Instructions!LDMFD}
\myindex{ARM!\Instructions!STMED}
\myindex{ARM!\Instructions!LDMED}
\myindex{ARM!\Instructions!STMFA}
\myindex{ARM!\Instructions!LDMFA}
\myindex{ARM!\Instructions!STMEA}
\myindex{ARM!\Instructions!LDMEA}

Zum Beispiel die \ac{STMFD}/\ac{LDMFD} und \ac{STMED}/\ac{LDMED} Instruktionen sind alle dafür gedacht mit einem absteigendem Stack zu arbeiten ( wächst nach unten, fängt mit hohen Adressen an und entwickelt sich zu niedrigeren Adressen). Die \ac{STMFA}/\ac{LDMFA} und \ac{STMEA}/\ac{LDMEA} Instruktionen sind dazu gedacht mit einem aufsteigendem Stack zu arbeiten (wächst nach oben und fängt mit niedrigeren Adressen an und wächst nach oben).

% It might be worth mentioning that STMED and STMEA write first,
% and then move the pointer, and that LDMED and LDMEA move the pointer first, and then read.
% In other words, ARM not only lets the stack grow in a non-standard direction,
% but also in a non-standard order.
% Maybe this can be in the glossary, which would explain why E stands for "empty".

\subsection{Warum wächst der Stack nach unten?}
\label{stack_grow_backwards}

Intuitiv, würden man annehmen das der Stack nach oben wächst z.B Richtung höherer Adressen, so wie bei jeder anderen Datenstruktur.

Der Grund das der Stack rückwärts wächst ist wohl historisch bedingt. Als Computer so groß waren das sie einen ganzen Raum beansprucht haben war es einfach Speicher in zwei Sektionen zu unterteilen, einen Teil für den \gls{heap} und einen Teil für den Stack. Sicher war zu dieser Zeit nicht bekannt wie groß der \gls{heap} und der Stack wachsen würden, während der Programm Laufzeit, also war die Lösung die einfachste mögliche.

\input{patterns/02_stack/stack_and_heap}

In \RitchieThompsonUNIX können wir folgendes lesen:

\begin{framed}
\begin{quotation}
Der user-core eines Programm Images wird in drei logische Segmente unterteilt. Das Programm-Text Segment beginnt bei 0 im virtuellen Adress Speicher. Während der Ausführung wird das Segment als schreibgeschützt markiert und eine einzelne Kopie des Segments wird unter allen Prozessen geteilt die das Programm ausführen. An der ersten 8K grenze über dem Programm Text Segment im Virtuellen Speicher, fängt der ``nonshared'' Bereich an, der nach Bedarf von Syscalls erweitert werden kann. Beginnend bei der höchsten Adresse im Virtuellen Speicher ist das Stack Segment, das Automatisch nach unten wächst während der Hardware Stackpointer sich ändert.
\end{quotation}
\end{framed}

Das erinnert daran wie manche Schüler Notizen zu  zwei Vorträgen in einem Notebook dokumentieren:
Notizen für den ersten Vortrag werden normal notiert, und Notizen zur zum zweiten Vortrag werden 
ans Ende des Notizbuches geschrieben, indem man das Notizbuch umdreht. Die Notizen treffen sich irgendwann
im Notizbuch aufgrund des fehlenden Freien Platzes.

% I think if we want to expand on this analogy,
% one might remember that the line number increases as as you go down a page.
% So when you decrease the address when pushing to the stack, visually,
% the stack does grow upwards.
% Of course, the problem is that in most human languages,
% just as with computers,
% we write downwards, so this direction is what makes buffer overflows so messy.

\subsection{Für was wird der Stack benutzt?}

% subsections
\EN{\input{patterns/02_stack/01_saving_ret_addr_EN}}
\RU{\input{patterns/02_stack/01_saving_ret_addr_RU}}
\DE{\input{patterns/02_stack/01_saving_ret_addr_DE}}
\FR{\input{patterns/02_stack/01_saving_ret_addr_FR}}
\PTBR{\input{patterns/02_stack/01_saving_ret_addr_PTBR}}
\IT{\input{patterns/02_stack/01_saving_ret_addr_IT}}
\PL{\input{patterns/02_stack/01_saving_ret_addr_PL}}
\JPN{\input{patterns/02_stack/01_saving_ret_addr_JPN}}

\EN{\input{patterns/02_stack/02_args_passing_EN}}
\RU{\input{patterns/02_stack/02_args_passing_RU}}
\PTBR{\input{patterns/02_stack/02_args_passing_PTBR}}
\DE{\input{patterns/02_stack/02_args_passing_DE}}
\IT{\input{patterns/02_stack/02_args_passing_IT}}
\FR{\input{patterns/02_stack/02_args_passing_FR}}
\JA{\input{patterns/02_stack/02_args_passing_JA}}
\PL{\input{patterns/02_stack/02_args_passing_PL}}


\EN{\input{patterns/02_stack/03_local_vars_EN}}
\RU{\input{patterns/02_stack/03_local_vars_RU}}
\DE{\input{patterns/02_stack/03_local_vars_DE}}
\PTBR{\input{patterns/02_stack/03_local_vars_PTBR}}
\EN{\input{patterns/02_stack/04_alloca/main_EN}}
\FR{\input{patterns/02_stack/04_alloca/main_FR}}
\RU{\input{patterns/02_stack/04_alloca/main_RU}}
\PTBR{\input{patterns/02_stack/04_alloca/main_PTBR}}
\IT{\input{patterns/02_stack/04_alloca/main_IT}}
\DE{\input{patterns/02_stack/04_alloca/main_DE}}
\PL{\input{patterns/02_stack/04_alloca/main_PL}}
\JPN{\input{patterns/02_stack/04_alloca/main_JPN}}

\input{patterns/02_stack/05_SEH}
\ifdefined\ENGLISH
\subsubsection{Buffer overflow protection}

More about it here~(\myref{subsec:bufferoverflow}).
\fi

\ifdefined\RUSSIAN
\subsubsection{Защита от переполнений буфера}

Здесь больше об этом~(\myref{subsec:bufferoverflow}).
\fi

\ifdefined\BRAZILIAN
\subsubsection{Proteção contra estouro de buffer}

Mais sobre aqui~(\myref{subsec:bufferoverflow}).
\fi

\ifdefined\ITALIAN
\subsubsection{Protezione contro buffer overflow}

Maggiori informazioni qui~(\myref{subsec:bufferoverflow}).
\fi

\ifdefined\FRENCH
\subsubsection{Protection contre les débordements de tampon}

Lire à ce propos~(\myref{subsec:bufferoverflow}).
\fi


\ifdefined\POLISH
\subsubsection{Metody zabiezpieczenia przed przepełnieniem stosu}

Więcej o tym tutaj~(\myref{subsec:bufferoverflow}).
\fi

\ifdefined\JAPANESE
\subsubsection{バッファオーバーフロー保護}

詳細はこちら~(\myref{subsec:bufferoverflow})
\fi


\subsubsection{Automatisches deallokieren der Daten auf dem Stack}

Vielleicht ist der Grund warum man lokale Variablen und SEH Einträge auf dem Stack speichert, weil sie beim 
verlassen der Funktion automatisch aufgeräumt werden. Man braucht dabei nur eine Instruktion um die Position
des Stackpointers zu korrigieren (oftmals ist es die \ADD Instruktion). Funktions Argumente, könnte man sagen 
werden auch am Ende der Funktion deallokiert. Im Kontrast dazu, alles was auf dem \emph{heap} gespeichert wird muss
explizit deallokiert werden. 

% sections
\EN{\input{patterns/02_stack/07_layout_EN}}
\RU{\input{patterns/02_stack/07_layout_RU}}
\DE{\input{patterns/02_stack/07_layout_DE}}
\PTBR{\input{patterns/02_stack/07_layout_PTBR}}
\EN{\input{patterns/02_stack/08_noise/main_EN}}
\FR{\input{patterns/02_stack/08_noise/main_FR}}
\RU{\input{patterns/02_stack/08_noise/main_RU}}
\IT{\input{patterns/02_stack/08_noise/main_IT}}
\DE{\input{patterns/02_stack/08_noise/main_DE}}
\PL{\input{patterns/02_stack/08_noise/main_PL}}
\JA{\input{patterns/02_stack/08_noise/main_JA}}

\input{patterns/02_stack/exercises}
}
\FR{\subsection{MIPS}

\subsubsection{3 arguments}

\myparagraph{GCC 4.4.5 \Optimizing}

La différence principale avec l'exemple \q{\HelloWorldSectionName} est que dans ce cas, \printf est appelée
à la place de \puts et 3 arguments de plus sont passés à travers les registres \$5\dots \$7 (ou \$A0\dots \$A2).
C'est pourquoi ces registres sont préfixés avec A-, ceci sous-entend qu'ils
sont utilisés pour le passage des arguments aux fonctions.

\lstinputlisting[caption=GCC 4.4.5 \Optimizing (\assemblyOutput),style=customasmMIPS]{patterns/03_printf/MIPS/printf3.O3_FR.s}

\lstinputlisting[caption=GCC 4.4.5 \Optimizing (IDA),style=customasmMIPS]{patterns/03_printf/MIPS/printf3.O3.IDA_FR.lst}

\IDA a agrégé la paire d'instructions \INS{LUI} et \INS{ADDIU} en une pseudo instruction \INS{LA}.
C'est pourquoi il n'y a pas d'instruction à l'adresse 0x1C: car \INS{LA} \emph{occupe} 8 octets.

\myparagraph{GCC 4.4.5 \NonOptimizing}

GCC \NonOptimizing est plus verbeux:

\lstinputlisting[caption=GCC 4.4.5 \NonOptimizing (\assemblyOutput),style=customasmMIPS]{patterns/03_printf/MIPS/printf3.O0_FR.s}

\lstinputlisting[caption=GCC 4.4.5 \NonOptimizing (IDA),style=customasmMIPS]{patterns/03_printf/MIPS/printf3.O0.IDA_FR.lst}

\subsubsection{8 arguments}

Utilisons encore l'exemple de la section précédente avec 9 arguments: \myref{example_printf8_x64}.

\lstinputlisting[style=customc]{patterns/03_printf/2.c}

\myparagraph{GCC 4.4.5 \Optimizing}

Seul les 4 premiers arguments sont passés dans les registres \$A0 \dots \$A3,
les autres sont passés par la pile.
\myindex{MIPS!O32}

C'est la convention d'appel O32 (qui est la plus commune dans le monde MIPS).
D'autres conventions d'appel (comme N32) peuvent utiliser les registres à d'autres fins.

\myindex{MIPS!\Instructions!SW}

\INS{SW} est l'abbréviation de \q{Store Word} (depuis un registre vers la mémoire).
En MIPS, il manque une instructions pour stocker une valeur dans la mémoire, donc
une paire d'instruction doit être utilisée à la place (\INS{LI}/\INS{SW}).

\lstinputlisting[caption=GCC 4.4.5 \Optimizing (\assemblyOutput),style=customasmMIPS]{patterns/03_printf/MIPS/printf8.O3_FR.s}

\lstinputlisting[caption=GCC 4.4.5 \Optimizing (IDA),style=customasmMIPS]{patterns/03_printf/MIPS/printf8.O3.IDA_FR.lst}

\myparagraph{GCC 4.4.5 \NonOptimizing}

GCC \NonOptimizing est plus verbeux:

\lstinputlisting[caption=\NonOptimizing GCC 4.4.5 (\assemblyOutput),style=customasmMIPS]{patterns/03_printf/MIPS/printf8.O0_FR.s}

\lstinputlisting[caption=\NonOptimizing GCC 4.4.5 (IDA),style=customasmMIPS]{patterns/03_printf/MIPS/printf8.O0.IDA_FR.lst}

}
\IT{\subsection{Operatore ternario}
\label{chap:cond}

L'operatore ternario in \CCpp ha la seguente sintassi:

\begin{lstlisting}
expression ? expression : expression
\end{lstlisting}

Ecco un semplice esempio:

\lstinputlisting[style=customc]{patterns/07_jcc/cond_operator/cond.c}

\subsubsection{x86}

I vecchi compilatori e quelli non ottimizzanti generano codice assembly come se fosse stata usata una coppia \TT{if/else}:

\lstinputlisting[caption=\NonOptimizing MSVC 2008,style=customasmx86]{patterns/07_jcc/cond_operator/MSVC2008_EN.asm}

\lstinputlisting[caption=\Optimizing MSVC 2008,style=customasmx86]{patterns/07_jcc/cond_operator/MSVC2008_Ox_EN.asm}

I nuovi compilatori sono più concisi:

\lstinputlisting[caption=\Optimizing MSVC 2012 x64,style=customasmx86]{patterns/07_jcc/cond_operator/MSVC2012_Ox_x64_EN.asm}

\myindex{x86!\Instructions!CMOVcc}
\Optimizing GCC 4.8 per x86 usa anche l'istruzione \TT{CMOVcc}, mentre la versione non-optimizing usa jump condizionali.

\subsubsection{ARM}

\myindex{x86!\Instructions!ADRcc}
Anche Optimizing Keil per ARM mode usa le istruzioni condizionali \TT{ADRcc}:

\lstinputlisting[label=cond_Keil_ARM_O3,caption=\OptimizingKeilVI (\ARMMode),style=customasmARM]{patterns/07_jcc/cond_operator/Keil_ARM_O3_EN.s}

Senza alcun intervento manuale, le due istruzioni \TT{ADREQ} e \TT{ADRNE} non possono essere eseguite nella stessa istanza.

\Optimizing Keil per Thumb mode deve usare i jump condizionali, in quanto non esistono istruzioni di caricamento che supportano i flag condizionali:

\lstinputlisting[caption=\OptimizingKeilVI (\ThumbMode),style=customasmARM]{patterns/07_jcc/cond_operator/Keil_thumb_O3_EN.s}

\subsubsection{ARM64}

\Optimizing GCC (Linaro) 4.9 per ARM64 usa anch'esso i jump condizionali:

\lstinputlisting[label=cond_ARM64,caption=\Optimizing GCC (Linaro) 4.9,style=customasmARM]{patterns/07_jcc/cond_operator/ARM64_GCC_O3_EN.s}

Ciò avviene perchè ARM64 non ha una semplice istruzione di caricament con flag condizionali, come \TT{ADRcc} in ARM mode a 32-bit o \INS{CMOVcc} in x86.

\myindex{ARM!\Instructions!CSEL}
Tuttavia ha l'istruzione \q{Conditional SELect} (\TT{CSEL})\InSqBrackets{\ARMSixFourRef p390, C5.5},
ma GCC 4.9 non sembra essere abbastanza intelligente da usarla in un simile pezzo di codice.

\subsubsection{MIPS}

Sfortunatamente, anche GCC 4.4.5 per MIPS non è molto intelligente:

\lstinputlisting[caption=\Optimizing GCC 4.4.5 (\assemblyOutput),style=customasmMIPS]{patterns/07_jcc/cond_operator/MIPS_O3_EN.s}

\subsubsection{Riscriviamolo come un \TT{if/else}}

\lstinputlisting[style=customc]{patterns/07_jcc/cond_operator/cond2.c}

\myindex{x86!\Instructions!CMOVcc}

E' interessante notare che GCC 4.8 ottimizzante per x86 è stato in grado di usare \TT{CMOVcc} in questo caso:

\lstinputlisting[caption=\Optimizing GCC 4.8,style=customasmx86]{patterns/07_jcc/cond_operator/cond2_GCC_O3_EN.s}

Keil ottimizzante in ARM mode genera codice identico a \lstref{cond_Keil_ARM_O3}.

Ma MSVC 2012 ottimizzante non è (ancora) così in gamba.

% Do not translate, this is macro:
\subsubsection{\Conclusion{}}

Perchè i compilatori ottimizzanti cercano di sbarazzarsi dei jump condizionali? Leggi qui: \myref{branch_predictors}.
}
\JA{\subsection{三項条件演算子}
\label{chap:cond}

\CCpp の三項条件演算子の構文は次のとおりです。

\begin{lstlisting}
expression ? expression : expression
\end{lstlisting}

次に例を示します。

\lstinputlisting[style=customc]{patterns/07_jcc/cond_operator/cond.c}

\subsubsection{x86}

古いコンパイラと最適化していないコンパイラは、\TT{if/else}文が使用されたかのようにアセンブリコードを生成します。

\lstinputlisting[caption=\NonOptimizing MSVC 2008,style=customasmx86]{patterns/07_jcc/cond_operator/MSVC2008_JA.asm}

\lstinputlisting[caption=\Optimizing MSVC 2008,style=customasmx86]{patterns/07_jcc/cond_operator/MSVC2008_Ox_JA.asm}

新しいコンパイラはより簡潔です。

\lstinputlisting[caption=\Optimizing MSVC 2012 x64,style=customasmx86]{patterns/07_jcc/cond_operator/MSVC2012_Ox_x64_JA.asm}

\myindex{x86!\Instructions!CMOVcc}
x86用の\Optimizing GCC 4.8も\TT{CMOVcc}命令を使用し、非最適化GCC 4.8は条件付きジャンプを使用します。

\subsubsection{ARM}

\myindex{x86!\Instructions!ADRcc}
ARMモード用の \Optimizing Keilでは、条件付き命令\TT{ADRcc}を使います。

\lstinputlisting[label=cond_Keil_ARM_O3,caption=\OptimizingKeilVI (\ARMMode),style=customasmARM]{patterns/07_jcc/cond_operator/Keil_ARM_O3_JA.s}

手動で介入しなければ、2つの命令\TT{ADREQ} と \TT{ADRNE}を同じときにで実行することはできません。

Thumbモードでは、 \Optimizing Keilは、条件付きフラグをサポートするロード命令がないため、
条件付きジャンプ命令を使用する必要があります。

\lstinputlisting[caption=\OptimizingKeilVI (\ThumbMode),style=customasmARM]{patterns/07_jcc/cond_operator/Keil_thumb_O3_JA.s}

\subsubsection{ARM64}

ARM64の \Optimizing GCC(Linaro)4.9でも、条件付きジャンプが使用されます。

\lstinputlisting[label=cond_ARM64,caption=\Optimizing GCC (Linaro) 4.9,style=customasmARM]{patterns/07_jcc/cond_operator/ARM64_GCC_O3_JA.s}

これは、ARM64には32ビットARMモードの\TT{ADRcc}やx86の\INS{CMOVcc}などの
条件フラグを伴った単純なロード命令がないためです。

\myindex{ARM!\Instructions!CSEL}
しかし、\q{Conditional SELect}命令(\TT{CSEL})\InSqBrackets{\ARMSixFourRef p390, C5.5}を使用していますが、
GCC 4.9ではこのようなコードの中で使用するには十分スマートではないようです。

\subsubsection{MIPS}

残念ながら、MIPS用のGCC 4.4.5はそれほどスマートではありません。

\lstinputlisting[caption=\Optimizing GCC 4.4.5 (\assemblyOutput),style=customasmMIPS]{patterns/07_jcc/cond_operator/MIPS_O3_JA.s}

\subsubsection{\TT{if/else}の方法で書き直しましょう}

\lstinputlisting[style=customc]{patterns/07_jcc/cond_operator/cond2.c}

\myindex{x86!\Instructions!CMOVcc}

興味深いことに、x86用のGCC 4.8の最適化は、この場合に\TT{CMOVcc}を使用することもできました。

\lstinputlisting[caption=\Optimizing GCC 4.8,style=customasmx86]{patterns/07_jcc/cond_operator/cond2_GCC_O3_JA.s}

ARMモードの \Optimizing Keilでは、\lstref{cond_Keil_ARM_O3}と同じコードが生成されます。

しかし、MSVC 2012の最適化は(まだ)あまり良くありません。

\subsubsection{\Conclusion{}}

コンパイラを最適化するとどうして条件付きジャンプを取り除こうとするのでしょうか?以下を読んでください:\myref{branch_predictors}
}

\EN{\subsection{Getting minimal and maximal values}

\subsubsection{32-bit}

\lstinputlisting[style=customc]{patterns/07_jcc/minmax/minmax.c}

\lstinputlisting[caption=\NonOptimizing MSVC 2013,style=customasmx86]{patterns/07_jcc/minmax/minmax_MSVC_2013_EN.asm}

\myindex{x86!\Instructions!Jcc}

These two functions differ only in the conditional jump instruction: 
\INS{JGE} (\q{Jump if Greater or Equal}) is used in the first one
and \INS{JLE} (\q{Jump if Less or Equal}) in the second.

\myindex{\CompilerAnomaly}
\label{MSVC_double_JMP_anomaly}

There is one unneeded \JMP instruction in each function, which MSVC presumably left by mistake.

\myparagraph{Branchless}

ARM for Thumb mode reminds us of x86 code:

\lstinputlisting[caption=\OptimizingKeilVI (\ThumbMode),style=customasmARM]{patterns/07_jcc/minmax/minmax_Keil_Thumb_O3_EN.s}

\myindex{ARM!\Instructions!Bcc}

The functions differ in the branching instruction: \INS{BGT} and \INS{BLT}.
It's possible to use conditional suffixes in ARM mode, so the code is shorter.

\myindex{ARM!\Instructions!MOVcc}
\INS{MOVcc} is to be executed only if the condition is met:

\lstinputlisting[caption=\OptimizingKeilVI (\ARMMode),style=customasmARM]{patterns/07_jcc/minmax/minmax_Keil_ARM_O3_EN.s}

\myindex{x86!\Instructions!CMOVcc}
\Optimizing GCC 4.8.1 and optimizing MSVC 2013 can use \INS{CMOVcc} instruction, which is analogous to \INS{MOVcc} in ARM:

\lstinputlisting[caption=\Optimizing MSVC 2013,style=customasmx86]{patterns/07_jcc/minmax/minmax_GCC481_O3_EN.s}

\subsubsection{64-bit}

\lstinputlisting[style=customc]{patterns/07_jcc/minmax/minmax64.c}

There is some unneeded value shuffling, but the code is comprehensible:

\lstinputlisting[caption=\NonOptimizing GCC 4.9.1 ARM64,style=customasmARM]{patterns/07_jcc/minmax/minmax64_GCC_49_ARM64_O0.s}

\myparagraph{Branchless}

No need to load function arguments from the stack, as they are already in the registers:

\lstinputlisting[caption=\Optimizing GCC 4.9.1 x64,style=customasmx86]{patterns/07_jcc/minmax/minmax64_GCC_49_x64_O3_EN.s}

MSVC 2013 does almost the same.

\myindex{ARM!\Instructions!CSEL}

ARM64 has the \INS{CSEL} instruction, which works just as \INS{MOVcc} in ARM or \INS{CMOVcc} in x86, just the name is different:
\q{Conditional SELect}.

\lstinputlisting[caption=\Optimizing GCC 4.9.1 ARM64,style=customasmARM]{patterns/07_jcc/minmax/minmax64_GCC_49_ARM64_O3_EN.s}

\subsubsection{MIPS}

Unfortunately, GCC 4.4.5 for MIPS is not that good:

\lstinputlisting[caption=\Optimizing GCC 4.4.5 (IDA),style=customasmMIPS]{patterns/07_jcc/minmax/minmax_MIPS_O3_IDA_EN.lst}

Do not forget about the \emph{branch delay slots}: the first \INS{MOVE} is executed \emph{before} \INS{BEQZ}, 
the second \INS{MOVE} is executed only if the branch hasn't been taken.

}
\RU{\subsection{Поиск минимального и максимального значения}

\subsubsection{32-bit}

\lstinputlisting[style=customc]{patterns/07_jcc/minmax/minmax.c}

\lstinputlisting[caption=\NonOptimizing MSVC 2013,style=customasmx86]{patterns/07_jcc/minmax/minmax_MSVC_2013_RU.asm}

\myindex{x86!\Instructions!Jcc}
Эти две функции отличаются друг от друга только инструкцией условного перехода:
\INS{JGE} (\q{Jump if Greater or Equal}~--- переход если больше или равно) используется в первой
и \INS{JLE} (\q{Jump if Less or Equal}~--- переход если меньше или равно) во второй.

\myindex{\CompilerAnomaly}
\label{MSVC_double_JMP_anomaly}
Здесь есть ненужная инструкция \JMP в каждой функции, которую MSVC, наверное, оставил по ошибке.

\myparagraph{Без переходов}

ARM в режиме Thumb напоминает нам x86-код:

\lstinputlisting[caption=\OptimizingKeilVI (\ThumbMode),style=customasmARM]{patterns/07_jcc/minmax/minmax_Keil_Thumb_O3_RU.s}

\myindex{ARM!\Instructions!Bcc}
Функции отличаются только инструкцией перехода: \INS{BGT} и \INS{BLT}.
А в режиме ARM можно использовать условные суффиксы, так что код более плотный.
\INS{MOVcc} будет исполнена только если условие верно:

\myindex{ARM!\Instructions!MOVcc}

\lstinputlisting[caption=\OptimizingKeilVI (\ARMMode),style=customasmARM]{patterns/07_jcc/minmax/minmax_Keil_ARM_O3_RU.s}

\myindex{x86!\Instructions!CMOVcc}
\Optimizing GCC 4.8.1 и оптимизирующий MSVC 2013 
могут использовать инструкцию \INS{CMOVcc}, которая аналогична \INS{MOVcc} в ARM:

\lstinputlisting[caption=\Optimizing MSVC 2013,style=customasmx86]{patterns/07_jcc/minmax/minmax_GCC481_O3_RU.s}

\subsubsection{64-bit}

\lstinputlisting[style=customc]{patterns/07_jcc/minmax/minmax64.c}

Тут есть ненужные перетасовки значений, но код в целом понятен:

\lstinputlisting[caption=\NonOptimizing GCC 4.9.1 ARM64,style=customasmARM]{patterns/07_jcc/minmax/minmax64_GCC_49_ARM64_O0.s}

\myparagraph{Без переходов}

Нет нужды загружать аргументы функции из стека, они уже в регистрах:

\lstinputlisting[caption=\Optimizing GCC 4.9.1 x64,style=customasmx86]{patterns/07_jcc/minmax/minmax64_GCC_49_x64_O3_RU.s}

MSVC 2013 делает то же самое.

\myindex{ARM!\Instructions!CSEL}
В ARM64 есть инструкция \INS{CSEL}, которая работает точно также, как и \INS{MOVcc} в ARM и \INS{CMOVcc} в x86,
но название другое: \q{Conditional SELect}.

\lstinputlisting[caption=\Optimizing GCC 4.9.1 ARM64,style=customasmARM]{patterns/07_jcc/minmax/minmax64_GCC_49_ARM64_O3_RU.s}

\subsubsection{MIPS}

А GCC 4.4.5 для MIPS не так хорош, к сожалению:

\lstinputlisting[caption=\Optimizing GCC 4.4.5 (IDA),style=customasmMIPS]{patterns/07_jcc/minmax/minmax_MIPS_O3_IDA_RU.lst}

Не забывайте о \emph{branch delay slots}: первая \INS{MOVE} исполняется \emph{перед} \INS{BEQZ},
вторая \INS{MOVE} исполняется только если переход не произошел.

}
\DE{\subsection{Minimale und maximale Werte berechnen}

\subsubsection{32-bit}

\lstinputlisting[style=customc]{patterns/07_jcc/minmax/minmax.c}

\lstinputlisting[caption=\NonOptimizing MSVC 2013,style=customasmx86]{patterns/07_jcc/minmax/minmax_MSVC_2013_DE.asm}

\myindex{x86!\Instructions!Jcc}
Diese beiden Funktionen unterscheiden sich nur hinsichtliche der bedingten Sprungbefehle:
\INS{JGE} (\q{Jump if Greater or Equal}) wird in der ersten verwendet
und \INS{JLE} (\q{Jump if Less or Equal}) in der zweiten.

\myindex{\CompilerAnomaly}
\label{MSVC_double_JMP_anomaly}
Hier gibt es jeweils einen unnötigen \JMP Befehl pro Funtion, den MSVC wahrscheinlich fehlerhafterweise dort belassen
hat.

\myparagraph{Verzweigungslos}

ARM im Thumb mode erinnert uns an den x86 Code:

\lstinputlisting[caption=\OptimizingKeilVI
(\ThumbMode),style=customasmARM]{patterns/07_jcc/minmax/minmax_Keil_Thumb_O3_DE.s}

\myindex{ARM!\Instructions!Bcc}
Die Funktionen unterscheiden sich in den Verzweigebefehlen: \INS{BGT} und \INS{BLT}.
Es ist möglich im ARM mode conditional codes zu verwenden, sodass der Code kürzer ist.

\myindex{ARM!\Instructions!MOVcc}
\INS{MOVcc} wird nur ausgeführt, wenn die Bedingung erfüllt (d.h. wahr) ist:

\lstinputlisting[caption=\OptimizingKeilVI
(\ARMMode),style=customasmARM]{patterns/07_jcc/minmax/minmax_Keil_ARM_O3_DE.s}

\myindex{x86!\Instructions!CMOVcc}
\Optimizing GCC 4.8.1 und der optimierende MSVC 2013 können den \INS{CMOVcc} Befehl verwenden, der analog zu
\INS{MOVcc} in ARM funktioniert:

\lstinputlisting[caption=\Optimizing MSVC 2013,style=customasmx86]{patterns/07_jcc/minmax/minmax_GCC481_O3_DE.s}

\subsubsection{64-bit}

\lstinputlisting[style=customc]{patterns/07_jcc/minmax/minmax64.c}
Hier findet ein unnötiges Verschieben von Variablen statt, aber der Code ist verständlich:

\lstinputlisting[caption=\NonOptimizing GCC 4.9.1 ARM64,style=customasmARM]{patterns/07_jcc/minmax/minmax64_GCC_49_ARM64_O0.s}

\myparagraph{Verzweigungslos}
Die Funtionsargumente müssen nicht vom Stack geladen werden, da sie sich bereits in den Registern befinden:

\lstinputlisting[caption=\Optimizing GCC 4.9.1
x64,style=customasmx86]{patterns/07_jcc/minmax/minmax64_GCC_49_x64_O3_DE.s}

MSVC 2013 tut beinahe das gleiche:

\myindex{ARM!\Instructions!CSEL}

ARM64 verfügt über den \INS{CSEL} Befehl, der genau wie \INS{MOVcc} in ARM oder \INS{CMOVcc} in x86 arbeitet; er hat
lediglich einen anderen Namen:
\q{Conditional SELect}.

\lstinputlisting[caption=\Optimizing GCC 4.9.1
ARM64,style=customasmARM]{patterns/07_jcc/minmax/minmax64_GCC_49_ARM64_O3_DE.s}

\subsubsection{MIPS}

Leider ist GCC 4.4.5 für MIPS nicht so gut:

\lstinputlisting[caption=\Optimizing GCC 4.4.5
(IDA),style=customasmMIPS]{patterns/07_jcc/minmax/minmax_MIPS_O3_IDA_DE.lst} 
Vergessen Sie nicht die \emph{branch delay slots}: der erste \INS{MOVE} wird \emph{vor} \INS{BEQZ} ausgeführt, der zweite
\INS{MOVE} wird nur dann ausgeführt, wenn die Verzweigung nicht genommen wird.


}
\FR{\subsection{Trouver les valeurs minimale et maximale}

\subsubsection{32-bit}

\lstinputlisting[style=customc]{patterns/07_jcc/minmax/minmax.c}

\lstinputlisting[caption=MSVC 2013 \NonOptimizing,style=customasmx86]{patterns/07_jcc/minmax/minmax_MSVC_2013_FR.asm}

\myindex{x86!\Instructions!Jcc}

Ces deux fonctions ne diffèrent que de l'instruction de saut conditionnel:
\INS{JGE} (\q{Jump if Greater or Equal} saut si supérieur ou égal) est utilisée
dans la première et \INS{JLE} (\q{Jump if Less or Equal} saut si inférieur ou égal)
dans la seconde.

\myindex{\CompilerAnomaly}
\label{MSVC_double_JMP_anomaly}

Il y a une instruction \JMP en trop dans chaque fonction, que MSVC a probablement
mise par erreur.

\myparagraph{Sans branchement}

Le mode Thumb d'ARM nous rappelle le code x86:

\lstinputlisting[caption=\OptimizingKeilVI (\ThumbMode),style=customasmARM]{patterns/07_jcc/minmax/minmax_Keil_Thumb_O3_FR.s}

\myindex{ARM!\Instructions!Bcc}

Les fonctions diffèrent au niveau de l'instruction de branchement: \INS{BGT} et \INS{BLT}.
Il est possible d'utiliser le suffixe conditionnel en mode ARM, donc le code est plus
court.

\myindex{ARM!\Instructions!MOVcc}
\INS{MOVcc} n'est exécutée que si la condition est remplie:

\lstinputlisting[caption=\OptimizingKeilVI (\ARMMode),style=customasmARM]{patterns/07_jcc/minmax/minmax_Keil_ARM_O3_FR.s}

\myindex{x86!\Instructions!CMOVcc}
GCC 4.8.1 \Optimizing et MSVC 2013 \Optimizing peuvent utiliser l'instruction \INS{CMOVcc},
qui est analogue à \INS{MOVcc} en ARM:

\lstinputlisting[caption=MSVC 2013 \Optimizing,style=customasmx86]{patterns/07_jcc/minmax/minmax_GCC481_O3_FR.s}

\subsubsection{64-bit}

\lstinputlisting[style=customc]{patterns/07_jcc/minmax/minmax64.c}

Il y a beaucoup de code inutile qui embrouille, mais il est compréhensible:

\lstinputlisting[caption=GCC 4.9.1 ARM64 \NonOptimizing,style=customasmARM]{patterns/07_jcc/minmax/minmax64_GCC_49_ARM64_O0.s}

\myparagraph{Sans branchement}

Il n'y a pas besoin de lire les arguments dans la pile, puisqu'ils sont déjà dans
les registres:

\lstinputlisting[caption=GCC 4.9.1 x64 \Optimizing,style=customasmx86]{patterns/07_jcc/minmax/minmax64_GCC_49_x64_O3_FR.s}

MSVC 2013 fait presque la même chose.

\myindex{ARM!\Instructions!CSEL}

ARM64 possède l'instruction \INS{CSEL}, qui fonctionne comme \INS{MOVcc} en ARM ou
\INS{CMOVcc} en x86, seul le nom diffère:
\q{Conditional SELect}.

\lstinputlisting[caption=GCC 4.9.1 ARM64 \Optimizing,style=customasmARM]{patterns/07_jcc/minmax/minmax64_GCC_49_ARM64_O3_FR.s}

\subsubsection{MIPS}

Malheureusement, GCC 4.4.5 pour MIPS n'est pas si performant:

\lstinputlisting[caption=GCC 4.4.5 \Optimizing (IDA),style=customasmMIPS]{patterns/07_jcc/minmax/minmax_MIPS_O3_IDA_FR.lst}

N'oubliez pas le slot de délai de branchement (\emph{branch delay slots}): le premier
\INS{MOVE} est exécuté \emph{avant} \INS{BEQZ}, le second \INS{MOVE} n'est exécuté
que si la branche n'a pas été prise.

}
\JA{\subsection{最小値と最大値の取得}

\subsubsection{32-bit}

\lstinputlisting[style=customc]{patterns/07_jcc/minmax/minmax.c}

\lstinputlisting[caption=\NonOptimizing MSVC 2013,style=customasmx86]{patterns/07_jcc/minmax/minmax_MSVC_2013_JA.asm}

\myindex{x86!\Instructions!Jcc}

これらの2つの機能は条件ジャンプ命令でのみ異なります。
最初の命令では\INS{JGE} (\q{Jump if Greater or Equal})が使用され、
2番目の場合は\INS{JLE} (\q{Jump if Less or Equal})が使用されます。

\myindex{\CompilerAnomaly}
\label{MSVC_double_JMP_anomaly}

各関数には不必要な \JMP 命令が1つありますが、おそらく誤って残っています。

\myparagraph{分岐}

ThumbモードのARMは、x86コードを思い起こします。

\lstinputlisting[caption=\OptimizingKeilVI (\ThumbMode),style=customasmARM]{patterns/07_jcc/minmax/minmax_Keil_Thumb_O3_JA.s}

\myindex{ARM!\Instructions!Bcc}

関数は分岐命令が異なります。\INS{BGT} と \INS{BLT}です。
ARMモードでは条件付きの接尾辞を使用することができるため、コードは短くなります。

\myindex{ARM!\Instructions!MOVcc}
\INS{MOVcc}は、条件が満たされた場合にのみ実行されます。

\lstinputlisting[caption=\OptimizingKeilVI (\ARMMode),style=customasmARM]{patterns/07_jcc/minmax/minmax_Keil_ARM_O3_JA.s}

\myindex{x86!\Instructions!CMOVcc}
\Optimizing GCC 4.8.1とMSVC 2013の最適化では、ARMの\INS{CMOVcc}に似た\INS{MOVcc}命令を使用できます。

\lstinputlisting[caption=\Optimizing MSVC 2013,style=customasmx86]{patterns/07_jcc/minmax/minmax_GCC481_O3_JA.s}

\subsubsection{64-bit}

\lstinputlisting[style=customc]{patterns/07_jcc/minmax/minmax64.c}

いくつかの不要な値のシャッフルがありますが、コードは理解できます。

\lstinputlisting[caption=\NonOptimizing GCC 4.9.1 ARM64,style=customasmARM]{patterns/07_jcc/minmax/minmax64_GCC_49_ARM64_O0.s}

\myparagraph{分岐なし}

スタックから関数の引数をロードする必要はありません。レジスタにすでに入っています。

\lstinputlisting[caption=\Optimizing GCC 4.9.1 x64,style=customasmx86]{patterns/07_jcc/minmax/minmax64_GCC_49_x64_O3_JA.s}

MSVC 2013はほぼ同じです。

\myindex{ARM!\Instructions!CSEL}

ARM64にはARMの\INS{MOVcc}またはx86の\INS{CMOVcc}と同じように機能する\INS{CSEL}命令がありますが、
その名前は\q{Conditional SELect}とは異なります。

\lstinputlisting[caption=\Optimizing GCC 4.9.1 ARM64,style=customasmARM]{patterns/07_jcc/minmax/minmax64_GCC_49_ARM64_O3_JA.s}

\subsubsection{MIPS}

残念ながら、MIPS用のGCC 4.4.5はあまり良くありません。

\lstinputlisting[caption=\Optimizing GCC 4.4.5 (IDA),style=customasmMIPS]{patterns/07_jcc/minmax/minmax_MIPS_O3_IDA_JA.lst}

\emph{分岐遅延スロット}を忘れないでください。最初の\INS{MOVE}は\INS{BEQZ}の\emph{前に}実行され、
2番目の\INS{MOVE}は分岐が実行されなかった場合にのみ実行されます。
}


% Do not translate, this is macro:
\subsection{\Conclusion{}}

\subsubsection{x86}

La forma grezza di un jump condizionale e' la seguente:

\begin{lstlisting}[caption=x86,style=customasmx86]
CMP register, register/value
Jcc true ; cc=condition code
false:
... codice da eseguire se il risultato del confronto e' false ...
JMP exit 
true:
... codice da eseguire se il risultato del confronto e' true ...
exit:
\end{lstlisting}

\subsubsection{ARM}

\begin{lstlisting}[caption=ARM,style=customasmARM]
CMP register, register/value
Bcc true ; cc=condition code
false:
... codice da eseguire se il risultato del confronto e' false ...
JMP exit 
true:
... codice da eseguire se il risultato del confronto e' true ...
exit:
\end{lstlisting}

\subsubsection{MIPS}

\begin{lstlisting}[caption=Check for zero,style=customasmMIPS]
BEQZ REG, label
...
\end{lstlisting}

\begin{lstlisting}[caption=Check for less than zero (using pseudoinstruction),style=customasmMIPS]
BLTZ REG, label
...
\end{lstlisting}

\begin{lstlisting}[caption=Check for equal values,style=customasmMIPS]
BEQ REG1, REG2, label
...
\end{lstlisting}

\begin{lstlisting}[caption=Check for non-equal values,style=customasmMIPS]
BNE REG1, REG2, label
...
\end{lstlisting}

\begin{lstlisting}[caption=Check for less than (signed),style=customasmMIPS]
SLT REG1, REG2, REG3
BEQ REG1, label
...
\end{lstlisting}

\begin{lstlisting}[caption=Check for less than (unsigned),style=customasmMIPS]
SLTU REG1, REG2, REG3
BEQ REG1, label
...
\end{lstlisting}

\subsubsection{Branchless}

\myindex{ARM!\Instructions!MOVcc}
\myindex{x86!\Instructions!CMOVcc}
\myindex{ARM!\Instructions!CSEL}
Se il corpo di uno statement condizionale e' molto piccolo, puo' essere utilizzata l'istruzione "move" condizionale: 
\INS{MOVcc} in ARM (in ARM mode), \INS{CSEL} in ARM64, \INS{CMOVcc} in x86.

\myparagraph{ARM}

In ARM e' possibile usare suffissi condizionali per alcune istruzioni:

\begin{lstlisting}[caption=ARM (\ARMMode),style=customasmARM]
CMP register, register/value
instr1_cc ; istruzione che sara' eseguita se il condition code e' true
instr2_cc ; altra istruzione che sara' eseguita se il condition code e' true
... etc...
\end{lstlisting}

Ovviamente non c'e' limite al numero di istruzioni con il suffisso condizionale, a patto che le flag CPU non siano modificate da nessuna istruzione. 
% FIXME: list of such instructions or \myref{} to it

\myindex{ARM!\Instructions!IT}

La modalita' Thumb ha l'istruzione \INS{IT}, che permette di aggiungere suffissi condizionali alle prossime quattro istruzioni.
Maggiori informazioni qui: \myref{ARM_Thumb_IT}.

\begin{lstlisting}[caption=ARM (\ThumbMode),style=customasmARM]
CMP register, register/value
ITEEE EQ ; set these suffixes: if-then-else-else-else
instr1   ; istruzione da eseguire se la condizione e' true
instr2   ; istruzione da eseguire se la condizione e' false
instr3   ; istruzione da eseguire se la condizione e' false
instr4   ; istruzione da eseguire se la condizione e' false
\end{lstlisting}

% Do not translate, this is macro:
\subsection{\Exercise}

(ARM64) Prova a riscrivere il codice in \lstref{cond_ARM64} rimuovendo tutti i jump condizionali e usando al loro posto l'istruzione \TT{CSEL} instruction.
