\subsubsection{GCC}

% The text states that GCC uses Intel syntax, but the footnote sounds like in needs to be activated
% Maybe edit the text to: GCC can produce Intel syntax (like MSVC), and the footnote to: Use the \TT{-S -masm=intel}.} to activate this
LinuxのGCC 4.4.1コンパイラと同じ \CCpp コードをコンパイルしてみましょう:\TT{gcc 1.c -o 1}
次に、 \IDA 逆アセンブラの助けを借りて、どのように \main 関数が作成されるのかを見ていきましょう。
\IDA は、MSVCと同様に、Intel-syntaxを使用します。\footnote{\TT{-S -masm=intel}オプションを適用することで、Intel構文でGCCのアセンブリリストを生成させることもできます}

\lstinputlisting[caption=code in \IDA,style=customasmx86]{patterns/01_helloworld/IDA_x86.asm}

\myindex{Function prologue}
\myindex{x86!\Instructions!AND}
結果はほぼ同じです。 helloのワールド文字列(データセグメントに格納されている)のアドレスは、最初にEAXレジスタにロードされ、スタックに保存されます。

さらに、関数プロローグには \INS{AND ESP, 0FFFFFFF0h} があります。この命令は、 \ESP レジスタ値を16バイトの境界に揃えます。この結果、スタック内のすべての値が同じ方法で整列されます(処理中の値が4バイト境界または16バイト境界に整列したアドレスにある場合、CPUのパフォーマンスは向上します)。

\myindex{x86!\Instructions!SUB}
\INS{SUB ESP, 10h} はスタックに16バイトを割り当てます。しかし、私たちはこれから見るように、ここでは4つだけが必要です。

これは、割り当てられたスタックのサイズも16バイトの境界に揃えられているためです。

% TODO1: rewrite.
\myindex{x86!\Instructions!PUSH}
文字列アドレス(または文字列へのポインタ)は、 \PUSH 命令を使用せずにスタックに直接格納されます。 \emph{var\_10} はローカル変数であり、 \printf{} の引数でもあります。それについて下記を読んでください。

\printf 関数が呼び出されます。

MSVCと異なり、GCCは最適化をオンにしないでコンパイルすると、より短いオペコードの代わりに\TT{MOV EAX, 0}を発行します。

\myindex{x86!\Instructions!LEAVE}
最後の命令、 \LEAVE は、\TT{MOV ESP, EBP}、および\TT{POP EBP}命令ペアと同等です。言い換えれば、この命令は\gls{stack pointer} (\ESP)を戻し、 \EBP レジスタを初期状態に戻す。これは、これらのレジスタ値(\ESP と \EBP )を関数の始めに(\INS{MOV EBP, ESP} / \INS{AND ESP, \ldots}を実行することによって)変更したので必要です。

\subsubsection{GCC: \ATTSyntax}
\label{ATT_syntax}

これをアセンブリ言語のAT\&T構文でどのように表現できるかを見てみましょう。 この構文は
UNIXの世界でずっと人気があります。

\begin{lstlisting}[caption=let's compile in GCC 4.7.3]
gcc -S 1_1.c
\end{lstlisting}

下記を得ます。

\lstinputlisting[caption=GCC 4.7.3,style=customasmx86]{patterns/01_helloworld/GCC.s}

リストには、多くのマクロ(ドットで始まる部分)が含まれています。 現時点では興味深いものではありません。

今のところ、簡単にするため、無視してもかまいません(C言語の文字列のように終端文字列をエンコードする\emph{.string}マクロを除く)。 
それからこれを見てみましょう。
\footnote{このGCCオプションは\q{不要な}マクロを削除するために使用できます:\emph{-fno-asynchronous-unwind-tables}}

\lstinputlisting[caption=GCC 4.7.3,style=customasmx86]{patterns/01_helloworld/GCC_refined.s}

\myindex{\ATTSyntax}
\myindex{\IntelSyntax}
インテルとAT\&T構文の主な違いのいくつかは次のとおりです。

\begin{itemize}

\item
ソースオペランドとデスティネーションオペランドは逆の順序で記述されます。

インテル構文では:<命令> <デスティネーション・オペランド> <ソース・オペランド>

AT\&T構文の場合:<命令> <ソースオペランド> <デスティネーションオペランド>

\myindex{\CStandardLibrary!memcpy()}
\myindex{\CStandardLibrary!strcpy()}
違いを覚えるのは簡単な方法です:インテル構文を扱うとき、オペランド間に等号($=$)があり、AT&T-syntaxを扱うときに右矢印($\rightarrow$)があると想像してください。
\footnote{ところで、いくつかのC標準関数(例えば、memcpy()、strcpy())では、インテル構文と同じ方法で引数がリストされています。まず、デスティネーションメモリブロックへのポインタ、次にソースメモリブロックへのポインタです}

\item
AT\&T:レジスタ名の前にはパーセント記号(\%)を、数字の前にはドル記号(\$)を書く必要があります。 角カッコのかわりに丸カッコが使用されています。

\item
AT\&T:オペランドサイズを定義する命令に接尾辞が追加されます

\begin{itemize}
\item q --- quad (64 bits)
\item l --- long (32 bits)
\item w --- word (16 bits)
\item b --- byte (8 bits)
\end{itemize}

% TODO1 simple example may be? \RU{Например mov\textbf{l}, movb, movw представляют различые версии инсструкция mov} \EN {For example: movl, movb, movw are variations of the mov instruction}

\end{itemize}

コンパイル結果に戻るには、 \IDA が表示した結果とほぼ同じです。 微妙な違いが1つあります。\TT{0FFFFFFF0h}は\TT{\$-16}として表示されます。 これは同じことです:10進数の\TT{16}は16進数で\TT{0x10}です。 \TT{-0x10}は\TT{0xFFFFFFF0}に等しくなります(32ビットデータ型の場合)。

\myindex{x86!\Instructions!MOV}
もう1つ:戻り値は、 \XOR ではなく通常の \MOV を使用して0に設定されます。 \MOV はレジスタに値をロードするだけです。 
誤ってつけられた名前です(データは移動せず、コピーされるため)。 他のアーキテクチャでは、この命令の名前は\q{LOAD}または\q{STORE}などと同様です。

