\subsubsection{GCC}

Teraz chcemy skompilować to samo za pomocą GCC 4.4.1 w Linux: \TT{gcc 1.c -o 1}.
Dalej za pomocą \IDA podejrzmy jak się skompilowała f-cja \main.
\IDA, jak i MSVC, pokazuje kod w syntaksie Intel\footnote{Również możemy zmusić GCC do generowania listingów w tym formacie za pomocą kluczy \TT{-S -masm=intel}.}.

\lstinputlisting[caption=kod w \IDA,style=customasmx86]{patterns/01_helloworld/IDA_x86.asm}

\myindex{Function prologue}
\myindex{x86!\Instructions!AND}
Prawie to samo. 
Adres linii \TT{hello, world}, leżącej w segmencie danych, najpierw jest zapisywany do \EAX, a później odkładany na stos.
Jeszcze w prologu f-cji widzimy \TT{AND ESP, 0FFFFFFF0h}~--- 
ta instrukcja wyrównuje \ESP do granicy 16-bajtowej, wyrównując wszystkie pozycje 
na stosie według tego schematu (co zapewnia szybszy dostęp przez mikroprocesor)\footnote{\URLWPDA}.

\myindex{x86!\Instructions!SUB}
\INS{SUB ESP, 10h} przydziela na stosie 16 bajtów. Choć tutaj wystarczą 4 bajty, co będzie widoczne dalej.

Tak się dzieje dlatego, że ilość przydzielanego miejsca na stosie lokalnym jest również wyrównana do granicy 16-bajtowej.

% TODO1: rewrite.
\myindex{x86!\Instructions!PUSH}
Adres linii (lub wskaźnik na nią) jest zatem odkładany prosto na stos bez wykorzystywania instrukcji \PUSH.
\emph{var\_10} jednocześnie i lokalna zmienna i argument dla \printf{}. Dokładniej to będzie omawiane w dalszym ciągu.

Dalej jest wywoływane \printf.

W odróżnieniu od MSVC, GCC przy kompilacji z wyłączoną optymalizacją generuje \TT{MOV EAX, 0} zamiast krótszego opcode.

\myindex{x86!\Instructions!LEAVE}
Ostatnia instrukcja \LEAVE~--- jest analogiczna do poleceń \TT{MOV ESP, EBP} i \TT{POP EBP}~--- jest to powrót \glslink{stack pointer}{wskaźnika stosu} i rejestru \EBP do stanu początkowego.
Jest to niezbędne, ponieważ na początku f-cji modyfikowaliśmy rejestry \ESP i \EBP{}\\
(za pomocą \INS{MOV EBP, ESP} / \INS{AND ESP, \ldots}).

\subsubsection{GCC: \ATTSyntax}
\label{ATT_syntax}

Spróbujmy podejrzeć, jak wygląda to samo w syntaksie AT\&T.
Ten zapis jest bardziej rozpowszechniony w świecie UNIX.

\begin{lstlisting}[caption=kompilujemy w GCC 4.7.3]
gcc -S 1_1.c
\end{lstlisting}

Otrzymamy taki plik:

\lstinputlisting[caption=GCC 4.7.3,style=customasmx86]{patterns/01_helloworld/GCC.s}

Tu jest dużo makro (rozpoczynających się z kropki). Na razie one nas nie interesują.

Na razie możemy je ignorować (za wyjątkiem \emph{.string}, za pomocą którego kodujemy sekwencję znaków zakończonych znakiem null ~--- tak jak w C). Wtedy otrzymamy następujące:
\footnote{Dla zmniejszenia liczby generowanych \q{zbędnych} makro, można skorzystać z następującego klucza GCC: \emph{-fno-asynchronous-unwind-tables}}:

\lstinputlisting[caption=GCC 4.7.3,style=customasmx86]{patterns/01_helloworld/GCC_refined.s}

\myindex{\ATTSyntax}
\myindex{\IntelSyntax}
Główne różnice między syntaksem Intel i AT\&T :

\begin{itemize}

\item
Operandy są zapisywane na odwrót

W Intel: \\
<instrukcja> <dest byte> <source byte>.

W AT\&T: \\
<instrukcja> <source byte> <dest byte>.

\myindex{\CStandardLibrary!memcpy()}
\myindex{\CStandardLibrary!strcpy()}
Żeby łatwiej było zrozumieć różnicę można skorzystać z takiej reguły:
kiedy pracujecie z syntaksem Intel~--- możecie w głowie postawić znak równości ($=$) między operandami,
a z AT\&T~--- strałkę w prawo ($\rightarrow$)
\footnote{W niektórych standardowych funkcjach biblioteki C (np, memcpy(), strcpy()) również korzysta się z usytuowania argumentów jak w syntaksie Intel: najpierw wskaźnik w pamięci na docelowy bajt, 
następnie wskaźnik na bajt źródłowy.}.

\item
AT\&T: Przed nazwami rejestrów stawia się symbol (\%), a przed liczbami (\$).
Zamiast nawiasów kwadratowych korzysta się z okrągłych.

\item
AT\&T: Do każdej instrukcji dostawia się specjalny symbol określający typ danych:

\begin{itemize}
\item q --- quad (64 bity)
\item l --- long (32 bity)
\item w --- word (16 bitów)
\item b --- byte (8 bitów)
\end{itemize}


\end{itemize}

Wracając do wyniku kompilacji: jest on identyczny do tego, który widzieliśmy w \IDA.
Jedna drobnostka: \TT{0FFFFFFF0h} jest zapisywane jako \TT{\$-16}.
To jest dokładnie to samo: \TT{16} w sytemie dziesiętnym co \TT{0x10} w szesnastkowym.
\TT{-0x10} to będzie akurat \TT{0xFFFFFFF0} (w ramach 32-bitowych liczb).

\myindex{x86!\Instructions!MOV}
Zwracany wynik jest ustawiany na 0 za pomocą zwykłej instrukcji \MOV, a nie \XOR.
\MOV ładuje wartość do rejestru.
Nazwa tej instrukcji nie jest do końca poprawna (wartości nie są przemieszczane, tylko kopiowane). W innych architekturach instrukcja ta nosi nazwę \q{LOAD}, \q{STORE} lub coś podobnego.


