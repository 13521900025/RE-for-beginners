\subsubsection{MSVC: x86-64}

\myindex{x86-64}
Proviamo anche con MSVC a 64-bit:

\lstinputlisting[caption=MSVC 2012 x64,style=customasmx86]{patterns/01_helloworld/MSVC_x64.asm}

\myindex{fastcall}

In x86-64, tutti i registri sono stati estesi a 64-bit ed il loro nome ha il prefisso \TT{R-}.
Per usare lo stack meno spesso (in altre parole, per accedere meno spesso alla memoria esterna/cache), esiste un metodo molto diffuso per passare gli argomenti delle funzioni tramite i registri (\IT{fastcall})
\myref{fastcall}.
Ovvero, una parte degli argomenti viene passata attraverso i registri, il resto ---attraverso lo stack.
In Win64, 4 argomenti di funzione sono passati nei registri \RCX, \RDX, \Reg{8}, \Reg{9}.
Questo è ciò che vediamo qui: un puntatore alla stringa per \printf è adesso passato nel registro \RCX anziché tramite lo stack.
I puntatori adesso sono a 64-bit, quindi vengono passati nei registri a 64-bit (aventi il prefisso \TT{R-}).
E' comunque possibile, per retrocompatibilità, accedere alle parti a 32-bit, usando il prefisso \TT{E-}.
I registri \RAX/\EAX/\AX/\AL in x86-64 appaiono così:

\RegTableOne{RAX}{EAX}{AX}{AH}{AL}

La funzione \main restituisce un valore di tipo \Tint{}, che in \CCpp, per migliore retrocompatibilità e portabilità, resta ancora a 32-bit, motivo per cui il registro \EAX (quindi la parte a 32 bit del registro) viene svuotato invece di \RAX{} alla fine della funzione.
Ci sono anche 40 byte allocati nello stack locale.
Questo spazio è detto \q{shadow space}, e ne parleremo più avanti: \myref{shadow_space}.
