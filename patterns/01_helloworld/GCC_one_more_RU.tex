\subsection{GCC --- ещё кое-что}
\label{use_parts_of_C_strings}

Тот факт, что \emph{анонимная} Си-строка имеет тип \emph{const} (\myref{string_is_const_char}), 
и тот факт, что выделенные в сегменте констант Си-строки гарантировано неизменяемые (immutable), 
ведет к интересному следствию: компилятор может использовать определенную часть строки.

Вот простой пример:

\begin{lstlisting}[style=customc]

#include <stdio.h>

int f1()
{
	printf ("world\n");
}

int f2()
{
	printf ("hello world\n");
}

int main()
{
	f1();
	f2();
}
\end{lstlisting}

Среднестатистический компилятор с \CCpp (включая MSVC) выделит место для двух строк, но вот что делает GCC 4.8.1:

\lstinputlisting[caption=GCC 4.8.1 + листинг в IDA,style=customasmx86]{patterns/01_helloworld/two_strings.asm}

Действительно, когда мы выводим строку \q{hello world}, 
эти два слова расположены в памяти впритык друг к другу и \puts, вызываясь из функции \GTT{f2()}, вообще не знает,
что эти строки разделены. Они и не разделены на самом деле, они разделены
только \q{виртуально}, в нашем листинге.

Когда \puts вызывается из \GTT{f1()}, он использует строку \q{world} плюс нулевой байт. \puts не знает, что там ещё есть какая-то строка перед этой!

Этот трюк часто используется (по крайней мере в GCC) и может сэкономить немного памяти.
Это близко к \emph{string interning}.

Еще один связанный с этим пример находится здесь: \myref{strstr_example}.

