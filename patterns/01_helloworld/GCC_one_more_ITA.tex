\subsection{GCC---un'altra cosa}
\label{use_parts_of_C_strings}

Il fatto che una stringa C \IT{anonima} abbia tipo \IT{const} (\myref{string_is_const_char}),
e che le stringhe C allocate in in segmenti costanti siano garantite come immutabili, ha una conseguenza interessante:
il compilatore può utilizzare una parte specifica della stringa.

Proviamo questo esempio:

\begin{lstlisting}[style=customc]
#include <stdio.h>

int f1()
{
	printf ("world\n");
}

int f2()
{
	printf ("hello world\n");
}

int main()
{
	f1();
	f2();
}
\end{lstlisting}

Compilatori \CCpp{} comuni (incluso MSVC) allocano due stringhe, ma vediamo cosa fa GCC 4.8.1:

\lstinputlisting[caption=GCC 4.8.1 + IDA,style=customasmx86]{patterns/01_helloworld/two_strings.asm}

Infatti: quando stampiamo la stringa \q{hello world}
queste due parole sono memorizzate in posizioni adiacenti in memoria e \puts chiamata dalla funzione \GTT{f2()}
non sa che questa stringa è divisa.
In effett, non è veramente divisa; è separata solo \q{virtualmente}, in questo listato.

Quando \puts viene chiamato da \GTT{f1()}, usa la stringa \q{world} più un byte zero. \puts che è presente qualcosa prima di questa stringa!

Questo trucchetto furbo viene utilizzato perlomeno da GCC e può salvare un po' di memoria.
Questo è simile allo \IT{string interning}.

Un ulteriore esempio correlato è questo: \myref{strstr_example}.
