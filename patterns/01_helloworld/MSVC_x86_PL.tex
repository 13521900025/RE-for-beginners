\subsubsection{MSVC}

Skompilujmy kod w MSVC 2010:

\begin{lstlisting}
cl 1.cpp /Fa1.asm
\end{lstlisting}

(Klucz \TT{/Fa} oznacza wygenerowanie listingu w asemblerze)

\begin{lstlisting}[caption=MSVC 2010,style=customasmx86]
CONST	SEGMENT
$SG3830	DB	'hello, world', 0AH, 00H
CONST	ENDS
PUBLIC	_main
EXTRN	_printf:PROC
; Function compile flags: /Odtp
_TEXT	SEGMENT
_main	PROC
	push	ebp
	mov	ebp, esp
	push	OFFSET $SG3830
	call	_printf
	add	esp, 4
	xor	eax, eax
	pop	ebp
	ret	0
_main	ENDP
_TEXT	ENDS
\end{lstlisting}

MSVC generuje listingi w syntaksie Intel.
Różnica między syntaksą Intel a AT\&T będzie omówiona trochę później:

Kompilator wygenerował plik \TT{1.obj}, który następnie będzie połączony konsolidatorem ( ang. linker) w \TT{1.exe}.
W naszym przypadku ten plik składa się z dwóch segmentów: \TT{CONST} (dla danych-stałych) i \TT{\_TEXT} (dla kodu).

\myindex{\CLanguageElements!const}
\label{string_is_const_char}
Linia \TT{hello, world} w \CCpp ma typ \TT{const char[]}\InSqBrackets{\TCPPPL p176, 7.3.2}, jednak nie posiada nazwy.
Ale kompilator potrzebuje jakiejś nazwy, żeby z tą linią pracować, dlatego nadaje jej własną nazwę \TT{\$SG3830}.

Dlatego ten przykład można by było zapisać w ten sposób:

\lstinputlisting[style=customc]{patterns/01_helloworld/hw_2.c}

Wróćmy do przykładu w asemblerze. Jak widać, linia się kończy bajtem zerowym~--- są to wymagania standardu \CCpp do linii.
Więcej o liniach w \CCpp: \myref{C_strings}.

W segmencie kodu \TT{\_TEXT} znajduje się na razie tylko jedna f-cja: \main{}.
Funkcja \main, jak i prawie wszystkie funkcje, zaczyna się od prologu i kończy się epilogiem

\myindex{x86!\Instructions!CALL}
Dalej następuje wywołanie funkcji \printf{}: \INS{CALL \_printf}.
\myindex{x86!\Instructions!PUSH}
Przed tym wywołaniem adres linii (lub wskaźnik na nią) z naszym powiadomieniem (``Hello, world!'') za pomocą instrukcji \PUSH odkłada się na stos.

Po tym jak funkcja \printf zwaraca zarządzanie do funkcji \main, adres linii (lub wskaźnik na nią) dalej jest na stosie.
Z racji tego, że nie jest on już więcej potrzebny \glslink{stack pointer}{wskaźnik stosu} (rejestr \ESP) zostaje skorygowany.

\myindex{x86!\Instructions!ADD}
\INS{ADD ESP, 4} oznacza dodawanie wartości 4 do zawartości rejestru \ESP.

Dlaczego 4? Z racji tego, że jest to kod 32-bitowy, do przekazania adresu potrzebujemy 4 bajtów. W x64 są to 8 bajtów.\\
\INS{ADD ESP, 4} ekwiwalentnie \TT{POP rejestr}, ale bez wykorzystania jakiegokolwiek rejestru.

\myindex{Intel C++}
\myindex{\oracle}
\myindex{x86!\Instructions!POP}

Niektóre kompilatory, naprzykład, Intel C++ Compiler, w tej samej sytuacji mogą zamiast 
\ADD wygenerować \TT{POP ECX} (podobne rzeczy można również spotkać, np, w kodzie \oracle{}, skompilowanym w nim samym),
co jest w zasadzie tym samym, o tyle że wartość w rejestrze \ECX się psuje.
Możliwe, że kompilator stosuje tu \TT{POP ECX}, dlatego że ta instrukcja jest krótsza (1 bajt w \TT{POP} kontra 3 w \TT{ADD}).

Oto przykład wykorzystania \POP zamiast \ADD z \oracle{}:

\begin{lstlisting}[caption=\oracle 10.2 Linux (plik app.o),style=customasmx86]
.text:0800029A                 push    ebx
.text:0800029B                 call    qksfroChild
.text:080002A0                 pop     ecx
\end{lstlisting}

%Więcej informacji o stosie można znaleźć w odpowiednim rozdziale
% ~(\myref{sec:stack}).
\myindex{\CLanguageElements!return}
Po wywołaniu \printf w kodzie w \CCpp wskazane \TT{return 0}~--- zwrócić 0 jako wynik f-cji \main.

\myindex{x86!\Instructions!XOR}
W kodzie wygenerowanym robi to instrukcja \\
\INS{XOR EAX, EAX}.

\myindex{x86!\Instructions!MOV}

\XOR, jak można się domyśleć~--- \q{alternatywa wykluczająca}\footnote{\href{http://go.yurichev.com/17118}{wikipedia}}, ale kompilatory często korzystają z niego zamiast prostego
\INS{MOV EAX, 0} --- znowu dlatego, że opcode jest krótszy (2 bajty w \XOR kontra 5 w \MOV).

\myindex{x86!\Instructions!SUB}
Niektóre kompilatory genrują \INS{SUB EAX, EAX}, co oznacza \emph{odjąć wartość w} \EAX \emph{od wartości w }\EAX, co w każdym przypadku da w wyniku końcowym 0.

\myindex{x86!\Instructions!RET}
Ostatnia instrukcja \RET zwraca zarządzanie do funkcji wywołującej. Zwykle jest to kod \CCpp \ac{CRT}, który zwróci zarządzanie do \ac{OS}.


