\subsubsection{MSVC: x86-64}

\myindex{x86-64}
64ビットのMSVCも試してみましょう。

\lstinputlisting[caption=MSVC 2012 x64,style=customasmx86]{patterns/01_helloworld/MSVC_x64.asm}

\myindex{fastcall}

x86-64では、すべてのレジスタが64ビットに拡張されましたが、その名前には\TT{R-}プレフィックスが付いています。 
スタックをあまり頻繁に使用しないように(言い換えると、外部メモリ/キャッシュにアクセスする頻度を減らす)、
レジスタ(\emph{fastcall}) \myref{fastcall}を介して関数引数を渡す一般的な方法があります。 
すなわち、関数の引数の一部はレジスタに渡され、残りはスタックに渡されます。 
Win64では、4つの関数引数が \RCX 、 \RDX 、 \Reg{8} 、および \Reg{9} レジスタに渡されます。 
ここで見るものは: \printf の文字列へのポインタはスタックにではなく、 \RCX レジスタに渡されます。 
ポインタは現在64ビットであるため、64ビットレジスタ(\TT{R-}プレフィックスを持つ)に渡されます。 
ただし、下位互換性を保つために、\TT{E-}接頭辞を使用して32ビットのパーツにアクセスすることは可能です。 
これは、 \RAX/\EAX/\AX/\AL レジスタがx86-64のように見える方法です。

\RegTableOne{RAX}{EAX}{AX}{AH}{AL}

\main 関数は\Tint{}型の値を返します。これは \CCpp では32ビットのままで、下位互換性と移植性を向上させるため、
関数終了時に \RAX レジスタの代わりに \EAX レジスタがクリアされる理由です。(すなわちレジスタの32ビットの部分) 
ローカルスタックには40バイトも割り当てられています。 
これは\q{シャドースペース}と呼ばれます。これについては後で説明します:\myref{shadow_space}
