\subsubsection{MSVC}

Compiliamolo in MSVC 2010:

\begin{lstlisting}
cl 1.cpp /Fa1.asm
\end{lstlisting}

(l'opzione \TT{/Fa} indica al compilatore di generare un file con il listato assembly)

\begin{lstlisting}[caption=MSVC 2010,style=customasmx86]
CONST	SEGMENT
$SG3830	DB	'hello, world', 0AH, 00H
CONST	ENDS
PUBLIC	_main
EXTRN	_printf:PROC
; Function compile flags: /Odtp
_TEXT	SEGMENT
_main	PROC
	push	ebp
	mov	ebp, esp
	push	OFFSET $SG3830
	call	_printf
	add	esp, 4
	xor	eax, eax
	pop	ebp
	ret	0
_main	ENDP
_TEXT	ENDS
\end{lstlisting}

MSVC produce codice assembly in sintassi Intel.
La differenza tra le sintassi Intel e AT\&T-syntax sarà discussa al \myref{ATT_syntax}.

Il compilatore ha generato il file, \TT{1.obj}, che deve essere linkato in \TT{1.exe}.
Nel nostro caso, il file contiene due segmenti: \TT{CONST} (per i dati constanti) e \TT{\_TEXT} (per il codice).

\myindex{\CLanguageElements!const}
\label{string_is_const_char}
La stringa \TT{hello, world} in \CCpp ha tipo \TT{const char[]}\InSqBrackets{\TCPPPL p176, 7.3.2}, ma non ha un nome assegnato.
Il compilatore deve in qualche modo aver a che fare con la stringa, e la definisce quindi con il nome interno \TT{\$SG3830}.

Questo è il motivo per cui l'esempio potrebbe essere riscritto nel modo seguente:

\lstinputlisting[style=customc]{patterns/01_helloworld/hw_2.c}

Torniamo al listato assembly. Come possiamo vedere, la stringa è terminata con un byte zero, che è lo standard per la terminazione delle stringhe \CCpp.
Più informazioni sulle stringhe in \CCpp: \myref{C_strings}.

Nel code segment, \TT{\_TEXT}, esiste fino ad ora solo una funzione: \main{}.
La funzione \main inizia con il codice di prologo (prologue code) e termina con il codice di epilogo (epilogue code) (come quasi qualunque funzione)
\footnote{Maggiori informazioni si trovano nella sezione su prologo ed epilogo delle funzioni ~(\myref{sec:prologepilog}).}.

\myindex{x86!\Instructions!CALL}
Dopo il prologo della funzione, notiamo la chiamata alla funzione \printf{} : \INS{CALL \_printf}.
\myindex{x86!\Instructions!PUSH}
Prima della chiamata, l'indirizzo della stringa (o un puntatore ad essa) contenente il saluto viene messo sullo stack con l'aiuto dell'istruzione \PUSH.

Quando la funzione \printf restituisce il controllo alla funzione \main , l'indirizzo della stringa (o il puntatore) si trova ancora sullo stack.
Poichè non ne abbiamo più bisogno, lo \gls{stack pointer} (il registro \ESP ) deve essere corretto.

\myindex{x86!\Instructions!ADD}
\INS{ADD ESP, 4} significa aggiungi 4 al valore del registro \ESP.

Perchè 4? Essendo questo un programma a 32-bit, abbiamo esattamente bisogno di 4 bytes per passare un indirizzo attraverso lo stack. Se fosse stato codice x64 ne sarebbero serviti 8.
\INS{ADD ESP, 4} è a tutti gli effetti equivalente a \TT{POP register} ma senza usare alcun registro\footnote{i flag CPU vengono comunque modificati}.

\myindex{Intel C++}
\myindex{\oracle}
\myindex{x86!\Instructions!POP}

Per lo stesso scopo, alcuni compilatori (come l'Intel C++ Compiler) potrebbero emettere l'istruzione \TT{POP ECX}
invece di \ADD (ad esempio, queto tipo di pattern può essere osservato nel codice di \oracle{} che è compilato con il compilatore Intel C++).
Questa istruzione ha pressoché lo stesso effetto ma il contenuto del registro \ECX sarà sovrascritto.
Il compilatore Intel C++ usa probabilmente \TT{POP ECX} poichè l'opcode di questa istruzione è più corto di \TT{ADD ESP, x} (1 byte per \TT{POP} contro 3 per \TT{ADD}).

Ecco un esempio dell'uso di \POP al posto di \ADD da \oracle{}:

\begin{lstlisting}[caption=\oracle 10.2 Linux (\ITAph{}),style=customasmx86]
.text:0800029A                 push    ebx
.text:0800029B                 call    qksfroChild
.text:080002A0                 pop     ecx
\end{lstlisting}

%Read more about the stack in section
% ~(\myref{sec:stack}).
\myindex{\CLanguageElements!return}
Dopo la chiamata a \printf, il codice \CCpp originale contiene la direttiva \TT{return 0}~---restituisci 0 come risultato dalla funzione \main.

\myindex{x86!\Instructions!XOR}
Nel codice generato, questa è implementata dall'istruzione \INS{XOR EAX, EAX}.

\myindex{x86!\Instructions!MOV}

\XOR è infatti semplicemente \q{eXclusive OR, ovvero OR esclusivo}\footnote{\href{http://go.yurichev.com/17118}{wikipedia}} ma i compilatori lo usano spesso al posto di
\INS{MOV EAX, 0}---ancora una volta poichè è un opcode leggermente più corto (2 byte per \XOR contro 5 per \MOV).

\myindex{x86!\Instructions!SUB}
Alcuni compilatori emettono l'istruzione \INS{SUB EAX, EAX}, che significa \IT{sottrai (SUBtract) il valore nel registro} \EAX \IT{dal valore nel registro} \EAX, che, in ogni caso, risulta uguale a zero.

\myindex{x86!\Instructions!RET}
L'ultima istruzione \RET restituisce il controllo al chiamante (\gls{caller}). Solitamente, questo è codice \CCpp \ac{CRT} , che, a sua volta, restituisce il controllo all' \ac{OS}.
