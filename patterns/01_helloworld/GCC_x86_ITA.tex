\subsubsection{GCC}

% The text states that GCC uses Intel syntax, but the footnote sounds like in needs to be activated
% Maybe edit the text to: GCC can produce Intel syntax (like MSVC), and the footnote to: Use the \TT{-S -masm=intel}.} to activate this
Proviamo adesso a compilare lo stesso codice \CCpp con il compilatore GCC 4.4.1 su Linux: \TT{gcc 1.c -o 1}.
Successivamente, con l'aiuto del disassembler \IDA, vediamo come è stata creata la funzione \main .
\IDA, come MSVC, utilizza la sintassi Intel\footnote{Possiamo anche fare in modo che GCC produca un listato assembly con la sintassi Intel tramite l'opzione \TT{-S -masm=intel}.}.

\lstinputlisting[caption=codice in \IDA,style=customasmx86]{patterns/01_helloworld/IDA_x86.asm}

\myindex{Function prologue}
\myindex{x86!\Instructions!AND}
Il risultato è pressoché lo stesso.
L'indirizzo della stringa \TT{hello, world} (memorizzato nel data segment) è caricato prima nel registro \EAX e successivamente salvato sullo stack.
Inoltre, il prologo della funzione contiene \TT{AND ESP, 0FFFFFFF0h}~---questa
istruzione allinea il valore del registro \ESP a 16-byte.
Ciò fa sì che tutti i valori sullo stack siano allineati allo stesso modo (la CPU è più efficiente se i valori che tratta sono collocati in memoria ad indirizzi allineati a multipli di 4 o 16 byte)\footnote{\URLWPDA}.

\myindex{x86!\Instructions!SUB}
\INS{SUB ESP, 10h} alloca 16 byte sullo stack. Tuttavia, come vedremo a breve, solo 4 sono necessari in questo caso.

Ciò è dovuto al fatto che la dimensione dello stack allocato è anch'essa allineata a 16 byte.

% TODO1: rewrite.
\myindex{x86!\Instructions!PUSH}
L'indirizzo della stringa (o un puntatore alla stringa) è quindi memorizzato direttamente sullo stack senza utilizzare l'istruzione \PUSH .
\emph{var\_10}~--- è una variabile locale ed è anche un argomento di \printf{}.
Maggiori dettagli in seguito.

Infine viene chiamata la funzione \printf.

Diversamente da MSVC, quando GCC compila senza ottimizzazione emette \TT{MOV EAX, 0} invece di un opcode più breve.

\myindex{x86!\Instructions!LEAVE}
L'ultima istruzione, \LEAVE~---è l'equivalente della coppia di istruzioni \TT{MOV ESP, EBP} e \TT{POP EBP} ~---in altre parole, questa istruzione riporta indietro lo \gls{stack pointer} (\ESP) e ripristina il registro \EBP al suo stato iniziale.
Ciò è necessario poiché abbiamo modificato i valori di questi registri (\ESP and \EBP) all'inizio della funzione ( eseguendo \INS{MOV EBP, ESP} / \INS{AND ESP, \ldots}).

\subsubsection{GCC: \ATTSyntax}
\label{ATT_syntax}

Vediamo come tutto questo può essere rappresentato nella sintassi assembly AT\&T.
Questa sintassi è molto più popolare nel mondo UNIX.

\begin{lstlisting}[caption=compiliamo in GCC 4.7.3]
gcc -S 1_1.c
\end{lstlisting}

Otteniamo questo:

\lstinputlisting[caption=GCC 4.7.3,style=customasmx86]{patterns/01_helloworld/GCC.s}

Il listato contiene molte macro (iniziano con il punto). Per il momento non ci interessano.

Per il momento, e solo per una questione di semplificazione, possiamo ignorarle (fatta eccezione per la macro \emph{.string} che codifica una sequenza di caratteri che termina con il null-byte (zero) proprio come una stringa C). Consideriamo soltanto questo
\footnote{Questa opzione di GCC può essere usata per eliminare le macro \q{superflue}: \emph{-fno-asynchronous-unwind-tables}}:

\lstinputlisting[caption=GCC 4.7.3,style=customasmx86]{patterns/01_helloworld/GCC_refined.s}

\myindex{\ATTSyntax}
\myindex{\IntelSyntax}
Alcune delle differenze maggiori tra la sintassi Intel e quella AT\&T sono:

\begin{itemize}

\item
Gli operandi sorgente e destinazione sono scritti in ordine opposto.

Sintassi Intel: <istruzione> <operando di destinazione> <operando di origine>.

Sintassi AT\&T: <istruzione> <operando di origine> <operando di destinazione>.

\myindex{\CStandardLibrary!memcpy()}
\myindex{\CStandardLibrary!strcpy()}
Ecco un modo facile per memorizzare la differenza:
quando si tratta di sintassi Intel immagina che ci sia un segno di uguaglianza ($=$) tra i due operandi, quando si tratta di sintassi AT\&T immagina una freccia da sinistra a destra ($\rightarrow$)
\footnote{A proposito, in alcune funzioni standard C(es., memcpy(), strcpy()) gli argomenti sono elencati nello stesso modo della sintassi Intel: prima il puntatore al blocco di memoria di destinazione, e poi il puntatore al blocco di memoria di origine.}.

\item
AT\&T: Il simbolo di percentuale (\%) deve essere scritto prima del nome di un registro, e il dollaro (\$) prima dei numeri.
Vengono utilizzate le parentesi tonde invece di quelle quadre.

\item
AT\&T: All'istruzione si aggiunge un suffisso che definisce le dimensioni dell'operando:

\begin{itemize}
\item q --- quad (64 bit)
\item l --- long (32 bit)
\item w --- word (16 bit)
\item b --- byte (8 bit)
\end{itemize}

% TODO1 simple example may be? \RU{Например mov\textbf{l}, movb, movw представляют различые версии инсструкция mov} \EN {For example: movl, movb, movw are variations of the mov instruction}

\end{itemize}

Torniamo al risultato compilato: è identico a quello che abbiamo visto in \IDA.
Con una piccola differenza: \TT{0FFFFFFF0h} è presentato come \TT{\$-16}.
E' la stessa cosa: \TT{16} nel sistema decimale è \TT{0x10} in esadecimale.
\TT{-0x10} è uguale a \TT{0xFFFFFFF0} (per un tipo di dato a 32-bit).

\myindex{x86!\Instructions!MOV}
Ancora una cosa: il valore di ritorno viene settato a 0 usando \MOV, non \XOR.
\MOV semplicemente carica un valore in un registro.
Il suo nome è fuorviante (il dato non viene spostato, bensì copiato). In altre architetture questa istruzione è chiamata \q{LOAD} o \q{STORE} o qualcosa di simile.
