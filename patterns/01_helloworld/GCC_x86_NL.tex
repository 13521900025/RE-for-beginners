\subsubsection{GCC}

Nu zullen we dezelfde \CCpp code compileren in de GCC 4.4.1 compiler in Linux: \TT{gcc 1.c -o 1}.
Vervolgens, met de assistentie van de \IDA disassembler, zullen we kijken hoe de \main functie gemaakt is.
\IDA, maakt net als MSVC gebruik van de Intel-syntax\footnote{We hadden GCC ook assembly listings kunnen laten gereren in Intel-syntax door gebruik te maken van de opties \TT{-S -masm=intel}.}.

\lstinputlisting[caption=code in \IDA,style=customasmx86]{patterns/01_helloworld/IDA_x86.asm}

\myindex{Function prologue}
\myindex{x86!\Instructions!AND}
Het resultaat is bijna hetzelfde.
Het adres van de \TT{hello, world} string (opgeslagen in het data segment) wordt eerst ingeladen in het \EAX register en wordt daarna opgeslagen op de stack.
Daarbovenop vind je in de functie proloog hetvolgende terug: \TT{AND ESP, 0FFFFFFF0h}~---
deze instructie lijnt de \ESP registerwaarde uit op een 16-byte begrenzing.
Dit resulteert in het feit dat alle waarden op de stack op dezelfde manier uitgelijnd worden.
De CPU presteert beter als de waarden die hij moet behandelen gelokaliseerd zijn in het geheugen op adressen die gealigneerd zijn op een 4-byte of 16-byte begrenzing.\footnote{URLWPDA}.

\myindex{x86!\Instructions!SUB}
\INS{SUB ESP, 10h} reserveert 16 bytes op de stack. Zoals we hierna echter kunnen zijn, zijn er in dit geval slechts 4 nodig.

Dit komt doordat de grootte van de gereserveerde stack ook uitgelijnd is op een 16-byte begrenzing.

% TODO1: rewrite.
\myindex{x86!\Instructions!PUSH}
Het string adres (of een pointer naar de string) wordt dan rechtstreeks op de stack geplaatst zonder gebruik te maken van de \PUSH instructie.
\IT{var\_10}~---is een lokale variabele en is ook een argument voor \printf{}.
Lees er hieronder meer over.

\NLph{}

In tegenstelling tot MSVC, wanneer GCC compileert zonder optimizatie, maakt het gebruik van \TT{MOV EAX, 0} in plaats van kortere opcodes.

\myindex{x86!\Instructions!LEAVE}
De laatste instructie, \LEAVE~---is het equivalent van het \TT{MOV ESP, EBP} en \TT{POP EBP} instructiepaar.
Met andere woorden, deze instructie zet de \gls{stack pointer} (\ESP) terug, en herstelt het \EBP register
terug tot zijn oorspronkelijke staat.
Dit is nodig aangezien we deze registerwaarden hebben gewijzigd (\ESP en \EBP) in het begin van de functie (door het uitvoeren van \INS{MOV EBP, ESP} / \INS{AND ESP, \ldots}).

\subsubsection{GCC: \ATTSyntax}
\label{ATT_syntax}

Laat ons eens kijken hoe dit kan weergegeven worden in assembly in de AT\&T syntax.
Deze syntax is veel populairder in de UNIX-wereld.

\begin{lstlisting}[caption=\NLph{} GCC 4.7.3]
gcc -S 1_1.c
\end{lstlisting}

We krijgen dit resultaat:

\lstinputlisting[caption=GCC 4.7.3,style=customasmx86]{patterns/01_helloworld/GCC.s}

De lijst bevat vele macros (die beginnen met een punt). Maar deze zijn niet interessant voor ons momenteel.

Voorlopig, om het simpel te houden, kunnen we deze negeren (buiten de \IT{.string} macro, dewelke
een null-terminated karakter reeks encodeert net als een C-string). Daarna zien we dit
\footnote{Deze GCC optie kan gebruikt worden om alle \q{onnodige} macros te elimineren: \IT{-fno-asynchronous-unwind-tables}}:

\lstinputlisting[caption=GCC 4.7.3,style=customasmx86]{patterns/01_helloworld/GCC_refined.s}

\myindex{\ATTSyntax}
\myindex{\IntelSyntax}
Sommige grote verschillen tussen de Intel en AT\&T syntax zijn:

\begin{itemize}

\item
\NLph{}

In Intel-syntax: <instructie> <doel> <bron>.

In AT\&T syntax: <instructie> <bron> <doel>.

\myindex{\CStandardLibrary!memcpy()}
\myindex{\CStandardLibrary!strcpy()}
Een gemakkelijke manier om dit verschil te onthouden is: 
Wanneer je met Intel-syntax te doen krijgt, kan je je inbeelden dat er een gelijkheidsteken ($=$) staat tussen de operands
en met AT\&T-syntax beeld je je in dat er een pijl naar rechts staat ($\rightarrow$)
\footnote{Trouwens, in sommige C standaard functies (bv. memcpy(), strcpy()) worden
de argumenten opgelijst op dezelfde manier als in Intel-syntax: eerst een pointer naar het bestemmings geheugen block, 
gevolgd door een pointer naar de bron.}.

\item
AT\&T: Voor registernamen moet een percentteken geschreven worden (\%) en voor cijfers een dollarteken (\$).
Ronde haakjes worden gebruikt in plaats van haakjes.

\item
AT\&T: Een suffix wordt toegevoegd aan de instructies om de operand grootte te bepalen:

\begin{itemize}
\item q --- quad (64 bits)
\item l --- long (32 bits)
\item w --- word (16 bits)
\item b --- byte (8 bits)
\end{itemize}

\end{itemize}

Laten we even terugblikken op het gecompileerde resultaat: dit is identiek als wat we gezien hebben in \IDA.
Met een klein verschil: \TT{0FFFFFFF0h} wordt weergegeven als \TT{\$-16}.
Dit is hetzelfde: \TT{16} in het decimaalsysteem is \TT{0x10} in hexadecimal.
\TT{-0x10} is gelijk aan \TT{0xFFFFFFF0} (voor een 32-bit data type).

\myindex{x86!\Instructions!MOV}
Nog een ding: de return value wordt best op 0 gezet door gebruik te maken van \MOV, niet van \XOR.
\MOV laadt gewoon een waarde in het register.
De naam is een foute noemer (data wordt niet verplaatst, maar eerder gekopieerd). In andere architecturen wordt deze instructie \q{LOAD} of \q{STORE} of iets soortgelijks genoemd.

