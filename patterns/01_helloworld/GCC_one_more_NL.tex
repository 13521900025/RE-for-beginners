% TODO to be resynced with English version
\subsection{GCC---Nog een ding}
\label{use_parts_of_C_strings}

Het feit dat een \IT{anonieme} C-string het type \IT{const} heeft (\myref{string_is_const_char}), 
en dat C-string gealloceerd in het segment met de constanten gegarandeerd immuteerbaar zijn, heeft een interessant gevolg:
De compiler kan een specifiek gedeelte van de string gebruiken.

Laten we dit voorbeeld proberen:

\begin{lstlisting}[style=customc]
#include <stdio.h>

int f1()
{
	printf ("world\n");
}

int f2()
{
	printf ("hello world\n");
}

int main()
{
	f1();
	f2();
}
\end{lstlisting}

Veelgebruikte \CCpp{}-compilers (inclusief MSVC) alloceren twee strings, maar laat ons eens kijken wat GCC 4.8.1 doet:

\lstinputlisting[caption=GCC 4.8.1 + IDA,style=customasmx86]{patterns/01_helloworld/two_strings.asm}

Inderdaad: wanneer we de \q{hello world} string printen, 
Worden deze twee woorden aangrenzend in het geheugen geplaatsd, en \puts, die geroepen wordt van de \GTT{f2()}
functie, is er zicht niet van bewust dat deze string opgedeeld is. 
In feite is hij ook niet opgedeeld, hij is slechts \q{virtueel} opgedeeld in deze listing.

Wanneer \puts aangeroepen wordt vanuit \GTT{f1()}, gebruikt het de \q{world} string plus een nulbyte. \puts is er zich niet van bewust dat er nog iets voor deze string staat!

Dit slimme truukje wordt vaak gebruikt door op zijn minst GCC, en kan geheugen besparen.

