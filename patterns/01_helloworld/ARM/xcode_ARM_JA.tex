\subsubsection{\OptimizingXcodeIV (\ARMMode)}

最適化を有効にしない場合のXcode 4.6.3では、冗長なコードが多数生成されるため、命令カウントができるだけ小さい最適化された出力を検討し、コンパイラスイッチ \Othree を設定します。

\begin{lstlisting}[caption=\OptimizingXcodeIV (\ARMMode),style=customasmARM]
__text:000028C4             _hello_world
__text:000028C4 80 40 2D E9   STMFD           SP!, {R7,LR}
__text:000028C8 86 06 01 E3   MOV             R0, #0x1686
__text:000028CC 0D 70 A0 E1   MOV             R7, SP
__text:000028D0 00 00 40 E3   MOVT            R0, #0
__text:000028D4 00 00 8F E0   ADD             R0, PC, R0
__text:000028D8 C3 05 00 EB   BL              _puts
__text:000028DC 00 00 A0 E3   MOV             R0, #0
__text:000028E0 80 80 BD E8   LDMFD           SP!, {R7,PC}

__cstring:00003F62 48 65 6C 6C+aHelloWorld_0  DCB "Hello world!",0
\end{lstlisting}

命令\TT{STMFD} と \TT{LDMFD}はもうよく知っていますね。

\myindex{ARM!\Instructions!MOV}

\MOV 命令は、\Reg{0}レジスタに数値\TT{0x1686}を書き込むだけです。 
これは「Hello world!」文字列を指すオフセットです。

\TT{R7}レジスタ( \IOSABI で標準化されている)はフレームポインタです。 以下でもっとみてみましょう。

MOVT R0、#0(MOVe Top)命令は、レジスタの上位16ビットに0を書き込みます。 ここでの問題点は、ARMモードの汎用MOV命令がレジスタの下位16ビットだけを書き込むことができることです。

ARMモードの命令オペコードはすべて32ビットに制限されています。 もちろん、この制限はレジスタ間でのデータの移動には関係しません。 
そのため、上位ビット(16から31まで)に書き込むための命令\TT{MOVT}が追加されています。 
ただし、ここでの使用方法は冗長です。これは、\TT{MOV R0, \#0x1686}命令がレジスタの上位部分をクリアしたためです。 
これはおそらくコンパイラの欠点です。
% TODO:
% I think, more specifically, the string is not put in the text section,
% ie. the compiler is actually not using position-independent code,
% as mentioned in the next paragraph.
% MOVT is used because the assembly code is generated before the relocation,
% so the location of the string is not yet known,
% and the high bits may still be needed.

\myindex{ARM!\Instructions!ADD}
\TT{ADD R0, PC, R0}命令は、\ac{PC}の値を\Reg{0}の値に加算し、\q{Hello world!}文字列の絶対アドレスを計算します。 
すでにわかっているように、それは\q{位置独立コード}なので、この修正はここでは必須です。

\INS{BL}命令は \printf の代わりに \puts 関数を呼び出します。

\label{puts}
\myindex{\CStandardLibrary!puts()}
\myindex{puts() instead of printf()}

GCCは最初の \printf 呼び出しを \puts に置き換えました。 
確かに、唯一の引数を持つ \printf は、 \puts とほぼ同じです。

\emph{ほとんどの場合}、文字列に\emph{\%}で始まるprintf形式識別子が含まれていない場合にのみ、
2つの関数が同じ結果を生成するためです。 その場合、これらの2つの機能の効果は異なります
\footnote{ \puts は文字列の最後に改行記号`\textbackslash{}n'を必要としないので、ここでは見られません}

なぜコンパイラは \printf を \puts に置き換えたのでしょうか?おそらく \puts が高速であるためです。
\footnote{\href{http://go.yurichev.com/17063}{ciselant.de/projects/gcc\_printf/gcc\_printf.html}}. 

これは、文字を\emph{\%}と一緒に比較することなく、文字を\gls{stdout}に渡すだけです。

次に、\Reg{0}レジスタを0に設定するための使い慣れた\TT{MOV R0, \#0}命令があります。
