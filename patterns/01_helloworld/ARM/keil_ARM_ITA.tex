\subsubsection{\NonOptimizingKeilVI (\ARMMode)}

Iniziamo a compilare il nostro esempio in Keil:

\begin{lstlisting}
armcc.exe --arm --c90 -O0 1.c
\end{lstlisting}

\myindex{\IntelSyntax}
Il compilatore \emph{armcc} produce un listato assembly con sintassi Intel,
e utilizza macro di alto livello legate al processore ARM
\footnote{ad esempio, la modalità ARM è priva delle istruzioni \PUSH/\POP},
tuttavia è più importante per noi vedere le istruzioni \q{così come sono}, quindi guardiamo in \IDA il risultato compilato.

\begin{lstlisting}[caption=\NonOptimizingKeilVI (\ARMMode) \IDA,style=customasmARM]
.text:00000000             main
.text:00000000 10 40 2D E9    STMFD   SP!, {R4,LR}
.text:00000004 1E 0E 8F E2    ADR     R0, aHelloWorld ; "hello, world"
.text:00000008 15 19 00 EB    BL      __2printf
.text:0000000C 00 00 A0 E3    MOV     R0, #0
.text:00000010 10 80 BD E8    LDMFD   SP!, {R4,PC}

.text:000001EC 68 65 6C 6C+aHelloWorld  DCB "hello, world",0    ; DATA XREF: main+4
\end{lstlisting}

Nell'esempio possiamo facilmente vedere che ogni istruzione ha lunghezza pari a 4 byte.
Difatti abbiamo compilato il codice per la modalità ARM e non Thumb.

\myindex{ARM!\Instructions!STMFD}
\myindex{ARM!\Instructions!POP}
La prima istruzione, \TT{STMFD SP!, \{R4,LR\}}\footnote{\ac{STMFD}},
funziona come l'istruzione \PUSH in x86, scrivendo i valori di due registri (\Reg{4} e \ac{LR}) nello stack.

Infatti il listato di output prodotto dal compilatore \emph{armcc}, per semplificazione, mostra l'istruzione \INS{PUSH \{r4,lr\}}.
Ma questo non è del tutto esatto. L'istruzione \PUSH esiste solo in modalità Thumb.
Utilizziamo quindi \IDA per non fare confusione.

Questa istruzione dapprima \glslink{decrement}{decrementa} il valore di \ac{SP} così da farlo puntare alla porzione dello stack che è libera di ospitare nuovi dati,
quindi salva il valore dei registri \Reg{4} e \ac{LR} all'indirizzo memorizzato
nel registro \ac{SP} appena modificato.

Questa istruzione (esattamente come \PUSH in Thumb mode) è in grado di salvare il valore di più registri contemporaneamente, cosa che può risultare molto utile.
A proposito, non esiste un equivalente in x86.
Si può notare anche che l'istruzione \TT{STMFD} è una generalizzazione
dell'istruzione \PUSH (estendendone le sue funzionalità), poichè può funzionare con qualunque registro, e non solo \ac{SP}.
In altre parole, \TT{STMFD} può essere usata per memorizzare un insieme di registri all'indirizzo di memoria specificato.

\myindex{\PICcode}
\myindex{ARM!\Instructions!ADR}
L'istruzione \INS{ADR R0, aHelloWorld}
aggiunge o sottrae il valore del registro \ac{PC} all'offset in cui è memorizzata la stringa \TT{hello, world}.
Ci si potrebbe chiedere, come è utilizzato in questo caso il registro \TT{PC}?
Questo viene detto \q{\PICcode}\footnote{Maggiori informazioni sono fornite nella relativa sezione~(\myref{sec:PIC})}.

Questo tipo di codice può essere eseguito ad un indirizzo non fisso in memoria.
In altre parole, si tratta di un indirizzamento relativo a \ac{PC} (\ac{PC}-relative addressing).
L'istruzione \TT{ADR} tiene conto della differenza tra l'indirizzo di questa istruzione e l'indirizzo dove si trova la stringa.
Questa differenza (offset) dovrà sempre essere la stessa, a prescindere dall'indirizzo in cui nostro codice sarà caricato dall'\ac{OS}.
Ciò spiega perchè bisogna soltanto aggiungere l'indirizzo dell'istruzione corrente (tramite \ac{PC}) per ottenere l'indirizzo assoluto in memoria della nostra stringa C.

\myindex{ARM!\Registers!Link Register}
\myindex{ARM!\Instructions!BL}
L'istruzione \INS{BL \_\_2printf}\footnote{Branch with Link} chiama la funzione \printf.
Questa istruzione funziona così:

\begin{itemize}
\item memorizza l'indirizzo successivo all'istruzione \INS{BL} (\TT{0xC}) nel registro \ac{LR};
\item quindi passa il controllo a \printf scrivendo il suo indirizzo nel registro \ac{PC}.
\end{itemize}

Quando la funzione \printf termina la sua esecuzione, deve sapere a chi restituire il controllo (dove ritornare).
Per questo motivo ogni funzione passa il controllo all'indirizzo memorizzato nel registro \ac{LR}.

Questa è una differenza tra processori \ac{RISC} \q{puri} come ARM e processori \ac{CISC} come x86,
nei quali il return address viene solitamente memorizzato nello stack
Maggiori informazioni si trovano nella prossima sezione~(\myref{sec:stack}).

A proposito, un indirizzo assoluto o un offset a 32-bit non può essere codificato nell'istruzione a 32-bit \TT{BL}
poichè ha solo spazio per 24 bit.
Come potremmo ricordare, tutte le istruzioni in ARM-mode hanno dimensione fissa di 4 byte (32 bit).
Dunque possono essere collocate solo su indirizzi allineati su 4-byte.
Ciò implica che gli ultimi 2 bit dell'indirizzo dell'istruzione (che sono sempre zero) possono essere omessi.
Abbiamo in definitiva 26 bit per la codifica dell'offset (offset encoding). E cio è sufficiente per codificare $current\_PC \pm{} \approx{}32M$.

\myindex{ARM!\Instructions!MOV}
L'istruzione successiva, \INS{MOV R0, \#0}\footnote{abbreviazione di MOVe} scrive semplicemente 0 nel registro \Reg{0}.
Questo perchè la nostra funzione C restituisce 0, ed il valore di ritorno deve essere memorizzato nel registro \Reg{0}.

\myindex{ARM!\Registers!Link Register}
\myindex{ARM!\Instructions!LDMFD}
\myindex{ARM!\Instructions!POP}
L'ultima istruzione \INS{LDMFD SP!, {R4,PC}}\footnote{\ac{LDMFD} è l'istruzione inversa rispetto a \ac{STMFD}}.
Carica valori dallo stack (o qualunque altra zona di memoria) per salvarli nei registri \Reg{4} e \ac{PC}, e \glslink{increment}{incrementa} lo \gls{stack pointer} \ac{SP}.
In questo caso funziona come \POP.\\
N.B. La prima istruzione \TT{STMFD} aveva salvato la coppia di registri \Reg{4} e \ac{LR} sullo stack, ma \Reg{4} e \ac{PC} vengono \emph{ripristinati} durante l'esecuzione di \TT{LDMFD}.

Come già sappiamo, l'indirizzo del posto a cui ogni funzione devere restituire il controllo è solitamente memorizzato nel registro \ac{LR}.
La prima istruzione salva il suo valore nello stack perchè lo stesso registro sarà utilizzato dalla nostra funzione
\main per la chiamata a \printf.
Al termine della funzione, questo valore può essere scritto direttamente nel registro \ac{PC}, passando di fatto il controllo al punto in cui la nostra funzione era stata chiamata.

Dal momento che \main è solitamente la funzione principale in \CCpp,
il controllo verrà restituito al loader dell' \ac{OS} oppure ad un punto in una \ac{CRT},
o qualcosa del genere.

Tutto questo consente di omettere l'istruzione \TT{BX LR} alla fine della funzione.

\myindex{ARM!DCB}
\TT{DCB} è una direttiva assembly che definisce un array di byte o una stringa ASCII, analoga alla direttiva DB
nel linguaggio assembly x86.
