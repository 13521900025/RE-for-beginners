\subsubsection{ARM64}

\myparagraph{GCC}

ARM64環境で、GCC 4.8.1を使用してサンプルをコンパイルしましょう。

\lstinputlisting[numbers=left,label=hw_ARM64_GCC,caption=\NonOptimizing GCC 4.8.1 + objdump,style=customasmARM]{patterns/01_helloworld/ARM/hw.lst}

ARM64にはThumbモードとThumb-2モードはなく、ARMのみであるため、32ビット命令のみがあります。
レジスタ数は2倍になります:\myref{ARM64_GPRs}
64ビットレジスタは\TT{X-}プレフィックスを持ち、32ビット部分は\TT{W-}です。

\myindex{ARM!\Instructions!STP}
\TT{STP}命令(\emph{ストアペア})は、スタック内の2つのレジスタ\RegX{29}と\RegX{30}を同時に保存します。

もちろん、この命令はメモリ内の任意の場所にこのペアを保存できますが、
ここで\ac{SP}レジスタが指定されているため、ペアはスタックに保存されます。

ARM64レジスタは64ビットのレジスタで、それぞれ8バイトのサイズを持つため、2つのレジスタを保存するために16バイト必要です。

オペランドの後の感嘆符(``!'')は、最初に16が\ac{SP}から減算され、
次にスタックに書き込まれるレジスタ・ペアの値であることを意味します。
これは\emph{事前インデックス}とも呼ばれます。\emph{事後インデックス}と\emph{事前インデックス}の違いについては、
\myref{ARM_postindex_vs_preindex} を読んでください。

したがって、より使い慣れたx86では、最初の命令は\TT{PUSH X29} と \TT{PUSH X30}のペアのアナログに過ぎません。
\RegX{29}はARM64では\ac{FP}として、\ac{LR}では\RegX{30}として使用されているため、関数プロローグに保存され、関数エピローグで復元されます。

2番目の命令は\RegX{29}(または\ac{FP})の\ac{SP}をコピーします。
これは、関数スタックフレームを設定するために行われます。

\label{pointers_ADRP_and_ADD}
\myindex{ARM!\Instructions!ADRP/ADD pair}
\TT{ADRP}命令と \ADD 命令は、最初の関数引数がこのレジスタに渡されるため、
文字列\q{Hello!}のアドレスを\RegX{0}レジスタに入力するために使用されます。
命令長は4バイトに制限されているため、レジスタに多数の命令を格納できる命令はありません。
詳細は\myref{ARM_big_constants_loading}参照してください。
したがって、いくつかの命令を利用する必要があります。最初の命令(\TT{ADRP})は、文字列が配置されている4KiBページのアドレスを\RegX{0}に書き込み、
2番目の命令(\ADD)は残りのアドレスをアドレスに追加するだけです。
詳細については、\myref{ARM64_relocs}を参照してください。

\TT{0x400000 + 0x648 = 0x400648}であり、このアドレスの\TT{.rodata}データセグメントにある\q{Hello!} C文字列を参照してください。

\myindex{ARM!\Instructions!BL}

\TT{BL}命令を使用して \puts を呼び出します。 これについては既に説明しました:\myref{puts}

\MOV は\RegW{0}に0を書き込みます。 
\RegW{0}は64ビット\RegX{0}レジスタの下位32ビットです。

\begin{center}
\begin{tabular}{ | l | l | }
\hline
\RU{Старшие 32 бита}\EN{High 32-bit part}\ES{Parte alta de 32 bits}\PTBRph{}\PLph{}\ITph{}\DE{Oberer 32-Bit-Teil}\THAph{}\NLph{}\FR{Partie 32 bits haute} & \RU{младшие 32 бита}\EN{low 32-bit part}\ES{parte baja de 32 bits}\PTBRph{}\PL{Starsze 32 bity}\ITph{}\DE{Unterer 32-Bit-Teil}\THAph{}\NLph{}\FR{Partie 32 bits basse} \\
\hline
\multicolumn{2}{ | c | }{X0} \\
\hline
\multicolumn{1}{ | c | }{} & \multicolumn{1}{ c | }{W0} \\
\hline
\end{tabular}
\end{center}


関数の結果は\RegX{0}を介して返され、 \main は0を返します。これで、リターンされる結果がどのように準備されるのかがわかります。
しかし、なぜ32ビットの部分を使用するのでしょうか?

ARM64の \Tint データ型はx86-64の場合と同じように、互換性を高めるため、32ビットとなっています。

関数が32ビット \Tint を返す場合は、\RegX{0}レジスタの下位32ビットのみを埋めなければなりません。

これを確認するために、この例を少し変更して再コンパイルしましょう。 \main は64ビット値を返します:

\begin{lstlisting}[caption=\main returning a value of \TT{uint64\_t} type,style=customc]
#include <stdio.h>
#include <stdint.h>

uint64_t main()
{
        printf ("Hello!\n");
        return 0;
}
\end{lstlisting}

結果は同じですが、その行の \MOV は次のようになります:

\begin{lstlisting}[caption=\NonOptimizing GCC 4.8.1 + objdump]
  4005a4:       d2800000        mov     x0, #0x0      // #0
\end{lstlisting}

\myindex{ARM!\Instructions!LDP}

\INS{LDP} (\emph{Load Pair}) は\RegX{29}と\RegX{30}レジスタを復元します。

命令の後には感嘆符はありません。これは、値が最初にスタックからロードされ、
次に\ac{SP}が16だけ増加したことを意味します。
これは\emph{事後インデックス}と呼ばれます。

\myindex{ARM!\Instructions!RET}
ARM64: \RET という新しい命令が登場しました。 
これは\TT{BX LR}と同様に機能し、特別な\emph{ヒント}ビットのみが追加され、
これが別のジャンプ命令ではなく関数からの戻りであることを\ac{CPU}に通知するので、より最適に実行できます。

関数の単純さのために、GCCの最適化はまさに同じコードを生成します。
