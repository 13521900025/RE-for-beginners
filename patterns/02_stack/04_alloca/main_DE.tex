\subsubsection{x86: alloca() Funktion}
\label{alloca}
\myindex{\CStandardLibrary!alloca()}

\newcommand{\AllocaSrcPath}{C:\textbackslash{}Program Files (x86)\textbackslash{}Microsoft Visual Studio 10.0\textbackslash{}VC\textbackslash{}crt\textbackslash{}src\textbackslash{}intel}

Es macht Sinn einen Blick auf die \TT{alloca()} Funktion zu werfen
\footnote{In MSVC, kann die Funktions Implementierung in \TT{alloca16.asm} und \TT{chkstk.asm} in \\
\TT{\AllocaSrcPath{}}} gefunden werden.
Diese Funktion arbeitet wie \TT{malloc()}, nur das sie Speicher direkt auf dem Stack bereit stellt.

Der allozierte Speicher Chunk muss nicht wieder mit \TT{free()} freigegeben werden, weil
der Funktions Epilog (\myref{sec:prologepilog}) das \ESP Register wieder in seinen ursprünglichen 
Zustand versetzt und der allozierte Speicher wird einfach \emph{verworfen}. 
Es macht Sinn sich anzuschauen wie \TT{alloca()} implementiert ist.
Mit einfachen Begriffen erklärt, diese Funktion verschiebt \ESP in Richtung des Stack ende mit der 
Anzahl der Bytes die alloziert werden müssen und setzt \ESP als einen Zeiger auf den \emph{allozierten} block.

Beispiel:

\lstinputlisting[style=customc]{patterns/02_stack/04_alloca/2_1.c}


Die \TT{\_snprintf()} Funktion arbeitetet genau wie \printf, nur statt die Ergebnisse nach \gls{stdout} aus zu geben ( bsp. auf dem Terminal oder Konsole), schreibt sie in den \TT{buf} buffer. Die Funktion \puts kopiert den Inhalt aus \TT{buf} nach \gls{stdout}. Sicher könnte man die beiden Funktions Aufrufe könnten durch einen \printf Aufruf ersetzt werden, aber wir sollten einen genaueren Blick auf die Benutzung kleiner Buffer anschauen.

\myparagraph{MSVC}

Compilierung mit MSVC 2010: 

\lstinputlisting[caption=MSVC 2010,style=customasmx86]{patterns/02_stack/04_alloca/2_2_msvc.asm}

\myindex{Compiler intrinsisch}
Das einzige \\TT{alloca()} Argument wird über \EAX übergeben (anstatt es erst auf den Stack zu pushen)
\footnote{Das liegt daran, das alloca() Verhalten Compiler intrinsisch bestimmt (\myref{sec:compiler_intrinsic}) im Gegensatz zu einer normalen Funktion. Einer der Gründe dafür das man braucht eine separate Funktion braucht, statt ein paar Code Instruktionen im Code,  ist weil die \ac{MSCV} alloca() Implementierung ebenfalls Code hat welcher aus dem gerade allozierten Speicher gelesen wird. Damit in Folge das \ac{Betriebssystem} physikalischen Speicher in dieser \ac{VM} Region zu allozieren. Nach dem \TT{alloca()} Aufruf, zeigt \ESP auf den Block von 600 Bytes der nun als Speicher für das \TT{buf} Array dienen kann.}.

\myparagraph{GCC + \IntelSyntax}

GCC 4.4.1 macht das selbe, aber ohne externe Funktions aufrufe.

\lstinputlisting[caption=GCC 4.7.3,style=customasmx86]{patterns/02_stack/04_alloca/2_1_gcc_intel_O3_EN.asm}

\myparagraph{GCC + \ATTSyntax}

Nun der gleiche Code, aber in AT\&T Syntax:

\lstinputlisting[caption=GCC 4.7.3,style=customasmx86]{patterns/02_stack/04_alloca/2_1_gcc_ATT_O3.s}

\myindex{\ATTSyntax}
Der Code ist der gleiche wie im vorherigen listig.

Übrigens, \INS{movl \$3, 20(\%esp)} in AT\&T Syntax wird zu \
\INS{mov DWORD PTR [esp+20], 3} in Intel-syntax.
In der AT\&T Syntax, sehen Register+Offset Formatierungen einer Adresse so aus:
\TT{offset(\%{register})}.
