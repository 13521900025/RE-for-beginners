\mysection{\Stack}
\label{sec:stack}
\myindex{\Stack}

Стек в компьютерных науках~--- это одна из наиболее фундаментальных структур данных
\footnote{\href{http://go.yurichev.com/17119}{wikipedia.org/wiki/Call\_stack}}.
\ac{AKA} \ac{LIFO}.

Технически это просто блок памяти в памяти процесса + регистр \ESP в x86 или \RSP в x64, либо \ac{SP} в ARM, который указывает где-то в пределах этого блока.

\myindex{ARM!\Instructions!PUSH}
\myindex{ARM!\Instructions!POP}
\myindex{x86!\Instructions!PUSH}
\myindex{x86!\Instructions!POP}
Часто используемые инструкции для работы со стеком~--- это \PUSH и \POP (в x86 и Thumb-режиме ARM). 
\PUSH уменьшает \ESP/\RSP/\ac{SP} на 4 в 32-битном режиме (или на 8 в 64-битном),
затем записывает по адресу, на который указывает \ESP/\RSP/\ac{SP}, содержимое своего единственного операнда.

\POP это обратная операция~--- сначала достает из \glslink{stack pointer}{указателя стека} значение и помещает его в операнд 
(который очень часто является регистром) и затем увеличивает указатель стека на 4 (или 8).

В самом начале \glslink{stack pointer}{регистр-указатель} указывает на конец стека.
Конец стека находится в начале блока памяти, выделенного под стек. Это странно, но это так.
\PUSH уменьшает \glslink{stack pointer}{регистр-указатель}, а \POP~--- увеличивает.

В процессоре ARM, тем не менее, есть поддержка стеков, растущих как в сторону уменьшения, так и в сторону увеличения.

\myindex{ARM!\Instructions!STMFD}
\myindex{ARM!\Instructions!LDMFD}
\myindex{ARM!\Instructions!STMED}
\myindex{ARM!\Instructions!LDMED}
\myindex{ARM!\Instructions!STMFA}
\myindex{ARM!\Instructions!LDMFA}
\myindex{ARM!\Instructions!STMEA}
\myindex{ARM!\Instructions!LDMEA}

Например, инструкции \ac{STMFD}/\ac{LDMFD}, \ac{STMED}/\ac{LDMED} предназначены для descending-стека (растет назад, начиная с высоких адресов в сторону низких).\\
Инструкции \ac{STMFA}/\ac{LDMFA}, \ac{STMEA}/\ac{LDMEA} предназначены для ascending-стека (растет вперед, начиная с низких адресов в сторону высоких).

% It might be worth mentioning that STMED and STMEA write first,
% and then move the pointer,
% and that LDMED and LDMEA move the pointer first, and then read.
% In other words, ARM not only lets the stack grow in a non-standard direction,
% but also in a non-standard order.
% Maybe this can be in the glossary, which would explain why E stands for "empty".

\subsection{Почему стек растет в обратную сторону?}
\label{stack_grow_backwards}

Интуитивно мы можем подумать, что, как и любая другая структура данных, стек мог бы расти вперед, т.е. в сторону увеличения адресов.

Причина, почему стек растет назад, видимо, историческая.
Когда компьютеры были большие и занимали целую комнату, было очень легко разделить сегмент на две части: для \glslink{heap}{кучи} и для стека.
Заранее было неизвестно, насколько большой может быть \glslink{heap}{куча} или стек, так что это решение было самым простым.

\input{patterns/02_stack/stack_and_heap}

В \RitchieThompsonUNIX можно прочитать:

\begin{framed}
\begin{quotation}
The user-core part of an image is divided into three logical segments. The program text segment begins at location 0 in the virtual address space. During execution, this segment is write-protected and a single copy of it is shared among all processes executing the same program. At the first 8K byte boundary above the program text segment in the virtual address space begins a nonshared, writable data segment, the size of which may be extended by a system call. Starting at the highest address in the virtual address space is a stack segment, which automatically grows downward as the hardware's stack pointer fluctuates.
\end{quotation}
\end{framed}

Это немного напоминает как некоторые студенты
пишут два конспекта в одной тетрадке:
первый конспект начинается обычным образом, второй пишется с конца, перевернув тетрадку.
Конспекты могут встретиться где-то посредине, в случае недостатка свободного места.

% I think if we want to expand on this analogy,
% one might remember that the line number increases as as you go down a page.
% So when you decrease the address when pushing to the stack, visually,
% the stack does grow upwards.
% Of course, the problem is that in most human languages,
% just as with computers,
% we write downwards, so this direction is what makes buffer overflows so messy.

\subsection{Для чего используется стек?}

% subsections
\EN{\subsubsection{Save the function's return address}

\myparagraph{x86}

\myindex{x86!\Instructions!CALL}
When calling another function with a \CALL instruction, the address of the point exactly after the \CALL instruction is saved 
to the stack and then an unconditional jump to the address in the \CALL operand is executed.

\myindex{x86!\Instructions!PUSH}
\myindex{x86!\Instructions!JMP}
The \CALL instruction is equivalent to a\\
\INS{PUSH address\_after\_call / JMP operand} instruction pair.

\myindex{x86!\Instructions!RET}
\myindex{x86!\Instructions!POP}
\RET fetches a value from the stack and jumps to it~---that is equivalent to a \TT{POP tmp / JMP tmp} instruction pair.

\myindex{\Stack!\MLStackOverflow}
\myindex{\Recursion}
Overflowing the stack is straightforward. Just run eternal recursion:

\begin{lstlisting}[style=customc]
void f()
{
	f();
};
\end{lstlisting}

MSVC 2008 reports the problem:

\begin{lstlisting}
c:\tmp6>cl ss.cpp /Fass.asm
Microsoft (R) 32-bit C/C++ Optimizing Compiler Version 15.00.21022.08 for 80x86
Copyright (C) Microsoft Corporation.  All rights reserved.

ss.cpp
c:\tmp6\ss.cpp(4) : warning C4717: 'f' : recursive on all control paths, function will cause runtime stack overflow
\end{lstlisting}

\dots but generates the right code anyway:

\lstinputlisting[style=customasmx86]{patterns/02_stack/1.asm}

\dots Also if we turn on the compiler optimization (\TT{\Ox} option) the optimized code will not overflow the stack 
and will work \emph{correctly}\footnote{irony here} instead:

\lstinputlisting[style=customasmx86]{patterns/02_stack/2.asm}

GCC 4.4.1 generates similar code in both cases without, however,  issuing any warning about the problem.

\myparagraph{ARM}

\myindex{ARM!\Registers!Link Register}
ARM programs also use the stack for saving return addresses, but differently.
As mentioned in \q{\HelloWorldSectionName}~(\myref{sec:hw_ARM}),
the \ac{RA} is saved to the \ac{LR} (\gls{link register}).
If one needs, however, to call another function and use the \ac{LR} register
one more time, its value has to be saved.
\myindex{Function prologue}
Usually it is saved in the function prologue.

\myindex{ARM!\Instructions!PUSH}
\myindex{ARM!\Instructions!POP}
Often, we see instructions like \INS{PUSH {R4-R7,LR}} along with this instruction in epilogue
\INS{POP {R4-R7,PC}}---thus register values to be used in the function are saved in the stack, including \ac{LR}.

\myindex{ARM!Leaf function}
Nevertheless, if a function never calls any other function, in \ac{RISC} terminology it is called a
\emph{\gls{leaf function}}\footnote{\href{http://go.yurichev.com/17064}{infocenter.arm.com/help/index.jsp?topic=/com.arm.doc.faqs/ka13785.html}}. 
As a consequence, leaf functions do not save the \ac{LR} register (because they don't modify it).
If such function is small and uses a small number of registers, it may not use the stack at all.
Thus, it is possible to call leaf functions without using the stack,
which can be faster than on older x86 machines because external RAM is not used for the stack
\footnote{Some time ago, on PDP-11 and VAX, the CALL instruction (calling other functions) was expensive; up to 50\%
of execution time might be spent on it, so it was considered that having a big number of small functions is an \gls{anti-pattern} \InSqBrackets{\TAOUP Chapter 4, Part II}.}.
This can be also useful for situations when memory for the stack is not yet allocated or not available.

Some examples of leaf functions:
\myref{ARM_leaf_example1}, \myref{ARM_leaf_example2}, 
\myref{ARM_leaf_example3}, \myref{ARM_leaf_example4}, \myref{ARM_leaf_example5},
\myref{ARM_leaf_example6}, \myref{ARM_leaf_example7}, \myref{ARM_leaf_example10}.

}
\RU{\subsubsection{Сохранение адреса возврата управления}

\myparagraph{x86}

\myindex{x86!\Instructions!CALL}
При вызове другой функции через \CALL сначала в стек записывается адрес, указывающий на место после 
инструкции \CALL, затем делается безусловный переход (почти как \TT{JMP}) на адрес, указанный в операнде.

\myindex{x86!\Instructions!PUSH}
\myindex{x86!\Instructions!JMP}
\CALL~--- это аналог пары инструкций \INS{PUSH address\_after\_call / JMP}.

\myindex{x86!\Instructions!RET}
\myindex{x86!\Instructions!POP}
\RET вытаскивает из стека значение и передает управление по этому адресу~--- 
это аналог пары инструкций \TT{POP tmp / JMP tmp}.

\myindex{\Stack!\MLStackOverflow}
\myindex{\Recursion}
Крайне легко устроить переполнение стека, запустив бесконечную рекурсию:

\begin{lstlisting}[style=customc]
void f()
{
	f();
};
\end{lstlisting}

MSVC 2008 предупреждает о проблеме:

\begin{lstlisting}
c:\tmp6>cl ss.cpp /Fass.asm
Microsoft (R) 32-bit C/C++ Optimizing Compiler Version 15.00.21022.08 for 80x86
Copyright (C) Microsoft Corporation.  All rights reserved.

ss.cpp
c:\tmp6\ss.cpp(4) : warning C4717: 'f' : recursive on all control paths, function will cause runtime stack overflow
\end{lstlisting}

\dots но, тем не менее, создает нужный код:

\lstinputlisting[style=customasmx86]{patterns/02_stack/1.asm}

\dots причем, если включить оптимизацию (\TT{\Ox}), то будет даже интереснее, без переполнения стека, 
но работать будет \emph{корректно}\footnote{здесь ирония}:

\lstinputlisting[style=customasmx86]{patterns/02_stack/2.asm}

GCC 4.4.1 генерирует точно такой же код в обоих случаях, хотя и не предупреждает о проблеме.

\myparagraph{ARM}

\myindex{ARM!\Registers!Link Register}
Программы для ARM также используют стек для сохранения \ac{RA}, куда нужно вернуться, но несколько иначе.
Как уже упоминалось в секции \q{\HelloWorldSectionName}~(\myref{sec:hw_ARM}),
\ac{RA} записывается в регистр \ac{LR} (\gls{link register}).
Но если есть необходимость вызывать какую-то другую функцию и использовать регистр \ac{LR} ещё
раз, его значение желательно сохранить.
\myindex{Function prologue}
\myindex{ARM!\Instructions!PUSH}
\myindex{ARM!\Instructions!POP}

Обычно это происходит в прологе функции, часто мы видим там инструкцию вроде \INS{PUSH \{R4-R7,LR\}}, а в эпилоге
\INS{POP \{R4-R7,PC\}}~--- так сохраняются регистры, которые будут использоваться в текущей функции, в том числе \ac{LR}.

\myindex{ARM!Leaf function}
Тем не менее, если некая функция не вызывает никаких более функций, в терминологии \ac{RISC} она называется
\emph{\gls{leaf function}}\footnote{\href{http://go.yurichev.com/17064}{infocenter.arm.com/help/index.jsp?topic=/com.arm.doc.faqs/ka13785.html}}. 
Как следствие, \q{leaf}-функция не сохраняет регистр \ac{LR} (потому что не изменяет его).
А если эта функция небольшая, использует мало регистров, она может не использовать стек вообще.
Таким образом, в ARM возможен вызов небольших leaf-функций не используя стек.
Это может быть быстрее чем в старых x86, ведь внешняя память для стека не используется
\footnote{Когда-то, очень давно, на PDP-11 и VAX на инструкцию CALL (вызов других функций) могло тратиться
вплоть до 50\% времени (возможно из-за работы с памятью),
поэтому считалось, что много небольших функций это \glslink{anti-pattern}{анти-паттерн}
\InSqBrackets{\TAOUP Chapter 4, Part II}.}.
Либо это может быть полезным для тех ситуаций, когда память для стека ещё не выделена, либо недоступна,

Некоторые примеры таких функций:
\myref{ARM_leaf_example1}, \myref{ARM_leaf_example2}, 
\myref{ARM_leaf_example3}, \myref{ARM_leaf_example4}, \myref{ARM_leaf_example5},
\myref{ARM_leaf_example6}, \myref{ARM_leaf_example7}, \myref{ARM_leaf_example10}.

}
\DE{\subsection{Rückgabe Adresse der Funktion speichern}

\myparagraph{x86}

\myindex{x86!\Instructions!CALL}
Wenn man eine Funktion mit der \CALL Instruktion aufruft, wird die Adresse direkt nach der
\CALL Instruktion auf dem Stack gespeichert und der unbedingte jump wird ausgeführt.

\myindex{x86!\Instructions!PUSH}
\myindex{x86!\Instructions!JMP}
Die \CALL Instruktion ist äquivalent zu dem \INS{PUSH address\_after\_call / JMP operand} Instruktions paar.

\myindex{x86!\Instructions!RET}
\myindex{x86!\Instructions!POP}
\RET ruft die Rückkehr Adresse vom Stack ab und springt zu dieser~---was äquivalent zu einem \TT{POP tmp / JMP tmp} Instruktions
paar ist.

\myindex{\Stack!\MLStackOverflow}
\myindex{\Recursion}

Den Stack zum überlaufen zu bringen ist recht einfach, einfach eine 
endlos rekursive Funktion Aufrufen:


\begin{lstlisting}[style=customc]
void f()
{
	f();
};
\end{lstlisting}


MSVC 2008 hat eine Erkennung für das Problem:


\begin{lstlisting}
c:\tmp6>cl ss.cpp /Fass.asm
Microsoft (R) 32-bit C/C++ Optimizing Compiler Version 15.00.21022.08 for 80x86
Copyright (C) Microsoft Corporation.  All rights reserved.

ss.cpp
c:\tmp6\ss.cpp(4) : warning C4717: 'f' : recursive on all control paths, function will cause runtime stack overflow
\end{lstlisting}

\dots aber der Compiler erzeugt den Code trotzdem:

\lstinputlisting[style=customasmx86]{patterns/02_stack/1.asm}

\dots Auch wenn wir die Compiler Optimierungen einschalten (\TT{/0x} Option) wird der optimierte Code nicht
den Stack zum überlaufen bringen. Stattdessen wird der Code \emph{korrekt}\footnote{Ironie hier} ausgeführt: 

\lstinputlisting[style=customasmx86]{patterns/02_stack/2.asm}

GCC 4.4.1 generiert vergleichbaren Code in beiden Fällen, jedoch ohne über das Overflow Problem zu warnen.

\myparagraph{ARM}

\myindex{ARM!\Registers!Link Register}

ARM Programme benutzen den Stack um Rücksprung Adressen zu speichern, aber anders.
Wie bereits erwähnt in \q{\HelloWorldSectionName}~(\myref{sec:hw_ARM}),
wird der \ac{RA} Wert im \ac{LR} (\gls{link register}) gespeichert.
Wenn nun eine andere Funktion aufgerufen werden muss und auf das \ac{LR} Register 
zu greift, muss der aktuelle Wert im Register irgendwo gespeichert werden.

\myindex{Funktion Prologe}
Normal wird der Wert im Funktion Prolog gespeichert.

\myindex{ARM!\Instructions!PUSH}
\myindex{ARM!\Instructions!POP}

Oft sieht man Instruktionen wie z.B \INS{PUSH {R4-R7,LR}} zusammen mit dieser Instruktion im 
Epilog \INS{POP {R4-R7,PC}}---Somit werden Werte die in den Funktionen benötigt werden auf dem 
Stack gespeichert, inklusive \ac{LR}.

\myindex{ARM!Leaf Funktion}
Wenn eine Funktion nie eine andere Funktion aufruft, nennt man das in der \ac{RISC} Terminologie eine
\emph{\glslink{leaf function}{leaf Funktion}}\footnote{\href{http://go.yurichev.com/17064}{infocenter.arm.com/help/index.jsp?topic=/com.arm.doc.faqs/ka13785.html}}.  % <-- attention could be a compilier bug
Als Konsequenz ergibt sich, das leaf Funktionen nicht das \ac{LR} Register speichern (da sie es nicht modifizieren).
Wenn solche Funktionen klein sind und nur eine geringe Anzahl an Registern benutzt, ist es möglich das der Stack
gar nicht benutzt wird. Es ist also möglich leaf Funktionen zu benutzen ohne den Stack zurück zu greifen, die Ausführung
ist hier schneller als auf älteren x86 Maschinen weil kein externer RAM für den Stack benutzt wird 
\footnote{Bis vor einer weile war es sehr teuer auf PDP-11 und VAX Maschinen die CALL Instruktion zu benutzen; bis zu 50\%
der Rechenzeit wurde allein für diese Instruktion verschwendet, man hat dabei festgestellt das eine große Anzahl an kleinen
Funktionen zu haben ein \gls{anti-pattern} \InSqBrackets{\TAOUP Chapter 4, Part II}.} ist.
Diese Eigenschaft kann nützlich sein wenn der Speicher für den Stack noch nicht alloziert oder verfügbar ist.

Ein paar Beispiele für leaf Funktionen:

\myref{ARM_leaf_example1}, \myref{ARM_leaf_example2}, 
\myref{ARM_leaf_example3}, \myref{ARM_leaf_example4}, \myref{ARM_leaf_example5},
\myref{ARM_leaf_example6}, \myref{ARM_leaf_example7}, \myref{ARM_leaf_example10}.

}
\FR{\subsubsection{Sauvegarder l'adresse de retour de la fonction}

\myparagraph{x86}

\myindex{x86!\Instructions!CALL}
Lorsque l'on appelle une fonction avec une instruction \CALL, l'adresse du point
exactement après cette dernière est sauvegardée sur la pile et un saut inconditionnel
à l'adresse de l'opérande \CALL est exécuté.

\myindex{x86!\Instructions!PUSH}
\myindex{x86!\Instructions!JMP}
L'instruction \CALL est équivalente à la paire d'instructions\\
\INS{PUSH address\_after\_call / JMP operand}.

\myindex{x86!\Instructions!RET}
\myindex{x86!\Instructions!POP}
\RET va chercher une valeur sur la pile et y saute~---ce qui est équivalent à
la paire d'instructions \TT{POP tmp / JMP tmp}.

\myindex{\Stack!\MLStackOverflow}
\myindex{\Recursion}
Déborder de la pile est très facile. Il suffit de lancer une récursion éternelle:

\begin{lstlisting}[style=customc]
void f()
{
	f();
};
\end{lstlisting}

MSVC 2008 signale le problème:

\begin{lstlisting}
c:\tmp6>cl ss.cpp /Fass.asm
Microsoft (R) 32-bit C/C++ Optimizing Compiler Version 15.00.21022.08 for 80x86
Copyright (C) Microsoft Corporation.  All rights reserved.

ss.cpp
c:\tmp6\ss.cpp(4) : warning C4717: 'f' : recursive on all control paths, function will cause runtime stack overflow
\end{lstlisting}

\dots mais génère tout de même le code correspondant:

\lstinputlisting[style=customasmx86]{patterns/02_stack/1.asm}

\dots Si nous utilisons l'option d'optimisation du compilateur (option \TT{\Ox})
le code optimisé ne va pas déborder de la pile et au lieu de cela va fonctionner
\IT{correctemment}\footnote{ironique ici}:

\lstinputlisting[style=customasmx86]{patterns/02_stack/2.asm}

GCC 4.4.1 génère un code similaire dans les deux cas, sans, toutefois émettre
d'avertissement à propos de ce problème.

\myparagraph{ARM}

\myindex{ARM!\Registers!Link Register}
Les programmes ARM utilisent également la pile pour sauver les adresses de retour,
mais différemment.
Comme mentionné dans \q{\HelloWorldSectionName}~(\myref{sec:hw_ARM}),
\ac{RA} est sauvegardé dans \ac{LR} (\gls{link register}).
Si l'on a toutefois besoin d'appeler une autre fonction et d'utiliser le registre
\ac{LR} une fois de plus, sa valeur doit être sauvegardée.
\myindex{Function prologue}
Usuellement, cela se fait dans le prologue de la fonction.

\myindex{ARM!\Instructions!PUSH}
\myindex{ARM!\Instructions!POP}
Souvent, nous voyons des instructions comme \INS{PUSH {R4-R7,LR}} en même temps
que cette instruction dans l'épilogue \INS{POP {R4-R7,PC}}---ces registres qui
sont utilisés dans la fonction sont sauvegardés sur la pile, \ac{LR} inclus.

\myindex{ARM!Fonction leaf} % FIXME traduire avec feuille ?
Néanmoins, si une fonction n'appelle jamais d'autre fonction, dans la terminologie
\ac{RISC} elle est appelée \IT{\glslink{leaf function}{fonction leaf}}\footnote{\href{http://go.yurichev.com/17064}{infocenter.arm.com/help/index.jsp?topic=/com.arm.doc.faqs/ka13785.html}}.
Ceci a comme conséquence que les fonctions leaf ne sauvegardent pas le registre
\ac{LR} (car elles ne le modifient pas).
Si une telle fonction est petite et utilise un petit nombre de registres, elle
peut ne pas utiliser du tout la pile.
Ainsi, il est possible d'appeler des fonctions leaf sans utiliser la pile.
Ce qui peut être plus rapide sur des vieilles machines x86 car la mémoire externe
n'est pas utilisée pour la pile
\footnote{Il y a quelque temps, sur PDP-11 et VAX, l'instruction CALL (appel d'autres fonctions) était coûteuse; jusqu'à 50\%
du temps d'exécution pouvait être passé à ça, il était donc considéré qu'avoir un grand nombre de petites fonctions était un \gls{anti-pattern} \InSqBrackets{\TAOUP Chapter 4, Part II}.}.
Cela peut être utile pour des situations où la mémoire pour la pile n'est pas
encore allouée ou disponible.

Quelques exemples de fonctions leaf:
\myref{ARM_leaf_example1}, \myref{ARM_leaf_example2},
\myref{ARM_leaf_example3}, \myref{ARM_leaf_example4}, \myref{ARM_leaf_example5},
\myref{ARM_leaf_example6}, \myref{ARM_leaf_example7}, \myref{ARM_leaf_example10}.

}
\PTBR{\subsubsection{Salvar o endereço de retorno de uma função}

\myparagraph{x86}

\myindex{x86!\Instructions!CALL}
Quando você chama outra função utilizando a instrução CALL, o endereço do ponto exato onde a 
instrução \CALL se encontra é salvo na pilha e então um jump incondicional para o endereço no operando de \CALL é executado.

\myindex{x86!\Instructions!PUSH}
\myindex{x86!\Instructions!JMP}
A instrução \CALL é equivalente a usar o par de instruções \TT{PUSH endereço\_depois\_chamada / JMP}.

\myindex{x86!\Instructions!RET}
\myindex{x86!\Instructions!POP}
\RET pega um valor da pilha e usa um jump para ele --- isso é equivalente a usar \INS{POP tmp / JMP tmp}.

\myindex{\Stack!\MLStackOverflow}
\myindex{\Recursion}
Estourar uma stack é fácil. Só execute alguma recursão externa:

\begin{lstlisting}[style=customc]
void f()
{
	f();
};
\end{lstlisting}

O compilador MSVC 2008 informa o problema:

\begin{lstlisting}
c:\tmp6>cl ss.cpp /Fass.asm
Microsoft (R) 32-bit C/C++ Optimizing Compiler Version 15.00.21022.08 for 80x86
Copyright (C) Microsoft Corporation.  All rights reserved.

ss.cpp
c:\tmp6\ss.cpp(4) : warning C4717: 'f' : recursive on all control paths, function will cause runtime stack overflow
\end{lstlisting}

\dots mas gera o código de qualquer maneira:

\lstinputlisting[style=customasmx86]{patterns/02_stack/1.asm}

\dots também, se ativarmos a otimização do compilador (opção \TT{/Ox}) 
o código otimizado não vai estourar a pilha e funcionará \emph{corretamente} \footnote{ironia aqui}:

\lstinputlisting[style=customasmx86]{patterns/02_stack/2.asm}

\PTBRph{}

}
\IT{\subsubsection{Salvare l'indirizzo di ritorno della funzione}

\myparagraph{x86}

\myindex{x86!\Instructions!CALL}
Quando si chiama una funzione con l'istruzione \CALL, l'indirizzo del punto esattamente dopo la \CALL viene salvato nello stack, e successivamente
viene eseguito un jump non condizionale all'indirizzo dell'operando di \CALL.

\myindex{x86!\Instructions!PUSH}
\myindex{x86!\Instructions!JMP}
L'istruzione \CALL è equivalente alla coppia di istruzioni \INS{PUSH indirizzo\_dopo\_call / JMP operando}.

\myindex{x86!\Instructions!RET}
\myindex{x86!\Instructions!POP}
\RET preleva un valore dallo stack ed effettua un jump ad esso~--- ciò equivale alla coppia di istruzioni \TT{POP tmp / JMP tmp}.

\myindex{\Stack!\MLStackOverflow}
\myindex{\Recursion}

Riempire lo stack fino allo straripamento è semplicissimo. Basta ricorrere alla ricorsione eterna:

\begin{lstlisting}[style=customc]
void f()
{
	f();
};
\end{lstlisting}

MSVC 2008 riporta il problema:

\begin{lstlisting}
c:\tmp6>cl ss.cpp /Fass.asm
Microsoft (R) 32-bit C/C++ Optimizing Compiler Version 15.00.21022.08 for 80x86
Copyright (C) Microsoft Corporation.  All rights reserved.

ss.cpp
c:\tmp6\ss.cpp(4) : warning C4717: 'f' : recursive on all control paths, function will cause runtime stack overflow
\end{lstlisting}

\dots ma genera in ogni caso il codice correttamente:

\lstinputlisting[style=customasmx86]{patterns/02_stack/1.asm}

\dots Se attiviamo le ottimizzazioni del compilatore (\TT{\Ox} option) il codice ottimizzato non causerà overflow dello stack
e funzionerà invece \emph{correttamente}\footnote{sarcasmo, si fa per dire}:

\lstinputlisting[style=customasmx86]{patterns/02_stack/2.asm}

GCC 4.4.1 genera codice simile in antrambi i casi, senza avvertire del problema.

\myparagraph{ARM}

\myindex{ARM!\Registers!Link Register}
Anche i programmi ARM usano lo stack per salvare gli indirizzi di ritorno, ma lo fanno in maniera diversa.
Come detto in \q{\HelloWorldSectionName}~(\myref{sec:hw_ARM}),
il \ac{RA} viene salvato nel \ac{LR} (\gls{link register}).
Se si presenta comunque la necessità di chiamare un'altra funzione ed usare il registro \ac{LR} ancora una volta,
il suo valore deve essere salvato.
\myindex{Function prologue}
Solitamente questo valore viene salvato nel preambolo della funzione.

\myindex{ARM!\Instructions!PUSH}
\myindex{ARM!\Instructions!POP}
Spesso vediamo istruzioni come \INS{PUSH {R4-R7,LR}} insieme ad istruzioni nell'epilogo come
\INS{POP {R4-R7,PC}}---perciò i valori dei registri che saranno usati nella funzione vengono salvati nello stack, incluso \ac{LR}.

\myindex{ARM!Leaf function}
Ciononostante, se una funzione non chiama al suo interno nessun'altra funzione, in terminologia \ac{RISC} è detta
\emph{\gls{leaf function}}, o funzione foglia.\footnote{\href{http://go.yurichev.com/17064}{infocenter.arm.com/help/index.jsp?topic=/com.arm.doc.faqs/ka13785.html}}.
Di conseguenza, le leaf functions non salvano il registro \ac{LR} register (perchè difatti non lo modificano).
Se una simile funzione è molto breve e usa un piccolo numero di registri, potrebbe non usare del tutto lo stack.
E' quindi possible chiamare le leaf functions senza usare lo stack, cosa che può essere più veloce rispetto alle vecchie macchine x86 perchè la RAM esterna non viene usata per lo stack
\footnote{Tempo fa, su PDP-11 e VAX, l'istruzione CALL (usata per chiamare altre funzioni) era costosa; poteva richiedere fino al 50\%
del tempo di esecuzione, ed era quindi consuetudine pensare che avere un grande numero di piccole funzioni fosse un \gls{anti-pattern} \InSqBrackets{\TAOUP Chapter 4, Part II}.}.
Lo stesso principio può tornare utile quando la memoria per lo stack non è stata ancora allocata o non è disponibile.

Alcuni esempi di funzioni foglia:
\myref{ARM_leaf_example1}, \myref{ARM_leaf_example2},
\myref{ARM_leaf_example3}, \myref{ARM_leaf_example4}, \myref{ARM_leaf_example5},
\myref{ARM_leaf_example6}, \myref{ARM_leaf_example7}, \myref{ARM_leaf_example10}.
}
\PL{\subsubsection{Zapisywanie adresu powrotu}

\myparagraph{x86}

\myindex{x86!\Instructions!CALL}
Przy wywołaniu funkcji przez \CALL najpierw na stos jest odkładany adres, wskazujący na miejsce po 
instrukcji \CALL, następnie robi się przejście bezwzględne (prawie jak \TT{JMP}) pod adres, zapisany w operandzie.

\myindex{x86!\Instructions!PUSH}
\myindex{x86!\Instructions!JMP}
\CALL~--- jest analogiczny do pary instrukcji \INS{PUSH address\_after\_call / JMP}.

\myindex{x86!\Instructions!RET}
\myindex{x86!\Instructions!POP}
\RET zdejmuje ze stosu wartość i przekazuje zarządzanie pod ten adres~--- 
jest to analogiczne do działania pary instrukcji \TT{POP tmp / JMP tmp}.

\myindex{\Stack!\MLStackOverflow}
\myindex{\Recursion}
Bardzo łatwo przepełnić stos, poprzez rekurencję:

\begin{lstlisting}[style=customc]
void f()
{
	f();
};
\end{lstlisting}

MSVC 2008 uprzedza:

\begin{lstlisting}
c:\tmp6>cl ss.cpp /Fass.asm
Microsoft (R) 32-bit C/C++ Optimizing Compiler Version 15.00.21022.08 for 80x86
Copyright (C) Microsoft Corporation.  All rights reserved.

ss.cpp
c:\tmp6\ss.cpp(4) : warning C4717: 'f' : recursive on all control paths, function will cause runtime stack overflow
\end{lstlisting}

\dots ale, niemniej jednak, tworzy potrzebny kod:

\lstinputlisting[style=customasmx86]{patterns/02_stack/1.asm}

\dots do tego, jeśli optymalizacja jest wyłączona (\TT{\Ox}), to będzie ciekawiej, bez przepełnienia stosu, 
ale będzie działało \IT{poprawnie}\footnote{ironia}:

\lstinputlisting[style=customasmx86]{patterns/02_stack/22.asm}

GCC 4.4.1 generuje taki sam kod w obu przypadkach, chociaż i nie wydaje stosownego komunikatu.

\myparagraph{ARM}

\myindex{ARM!\Registers!Link Register}
Programy dla ARM również korzystają ze stosu do zapisywania \ac{RA}, gdzie trzeba wrócić, ale trochę w inny sposób.
Jak już było wspomniane w sekcji \q{\HelloWorldSectionName}~(\myref{sec:hw_ARM}),
\ac{RA} jest zapisywany do rejestru \ac{LR} (\gls{link register}).
Ale jeśli wynika potrzeba wywołania jeszcze jakiejś funkcji i trzeba skorzystać z rejestru \ac{LR} jeszcze
raz, to jego zawartość najlepiej gdzieś zapisać.
\myindex{Function prologue}
\myindex{ARM!\Instructions!PUSH}
\myindex{ARM!\Instructions!POP}

Zwykle to się odbywa w prologu funkcji, często widzimy tam instrukcje w stylu \INS{PUSH \{R4-R7,LR\}}, a w epilogu
\INS{POP \{R4-R7,PC\}}~--- w ten sposób są zapisywane rejestry, z których będzie korzystała bieżąca funkcja, w tym rejestr \ac{LR}.

\myindex{ARM!Leaf function}
Niemniej jednak, jeśli jakaś funkcja nie wywołuje żadnych innych funkcji w trakcie swojej roboty, według terminologii \ac{RISC} jest ona nazywana
\IT{\gls{leaf function}}\footnote{\href{http://go.yurichev.com/17064}{infocenter.arm.com/help/index.jsp?topic=/com.arm.doc.faqs/ka13785.html}}. 
Wskutek tego, \q{leaf}-funkcja nie zapisuje rejestru \ac{LR} (dlatego że ona go nie zmienia).
A jeśli funkcja jest niewielkich rozmiarów, korzysta z małej ilości rejestrów, to może nie korzystać ze stosu w ogóle.
W ten sposób, w ARM możliwe jest wywoływanie małych leaf-funkcji nie korzystając ze stosu.
jest to szybsze niż w starych x86, dlatego że nie korzysta się z pamięci zewnętrznej do stosu
\footnote{Kiedyś, bardzo dawno temu, na PDP-11 i VAX na wykonanie instrukcjii CALL (wywołanie innych funkcji) mogło być zatracone
nawet 50\% czasu (przwdopodobnie przez pracę z pamięcią zewnętrzną),
dlatego było uważane, że dużo małych funkcji to \glslink{anti-pattern}
\InSqBrackets{\TAOUP Chapter 4, Part II}.}.
Również to może być też korzystne, kiedy pamięć pod stos jeszcze nie jest zaalokowana, lub jest niedostępna,

kilka przykładów takich funkcji:
\myref{ARM_leaf_example1}, \myref{ARM_leaf_example2}, 
\myref{ARM_leaf_example3}, \myref{ARM_leaf_example4}, \myref{ARM_leaf_example5},
\myref{ARM_leaf_example6}, \myref{ARM_leaf_example7}, \myref{ARM_leaf_example10}.


}
\JPN{\subsubsection{関数のリターンアドレスを保存する}

\myparagraph{x86}

\myindex{x86!\Instructions!CALL}
\CALL 命令で別の関数を呼び出すと、 \CALL 命令の直後のポイントのアドレスがスタックに保存され、 
\CALL オペランドのアドレスへの無条件ジャンプが実行されます。

\myindex{x86!\Instructions!PUSH}
\myindex{x86!\Instructions!JMP}
\CALL 命令は、PUSHの\INS{PUSH address\_after\_call / JMP operand}命令対に相当する。

\RET はスタックから値を取り出し、ジャンプします。これは\TT{POP tmp / JMP tmp}命令の対に相当します。

\myindex{\Stack!\MLStackOverflow}
\myindex{\Recursion}
スタックのオーバーフローは簡単です。 永遠の再帰を実行するだけです:


\begin{lstlisting}[style=customc]
void f()
{
	f();
};
\end{lstlisting}

MSVC 2008が問題をレポートします:

\begin{lstlisting}
c:\tmp6>cl ss.cpp /Fass.asm
Microsoft (R) 32-bit C/C++ Optimizing Compiler Version 15.00.21022.08 for 80x86
Copyright (C) Microsoft Corporation.  All rights reserved.

ss.cpp
c:\tmp6\ss.cpp(4) : warning C4717: 'f' : recursive on all control paths, function will cause runtime stack overflow
\end{lstlisting}

\dots しかし、正しいコードを生成します。

\lstinputlisting[style=customasmx86]{patterns/02_stack/1.asm}

\dots また、コンパイラ最適化(\TT{\Ox}オプション)を有効にすると、最適化されたコードはスタックをオーバーフローせず、
代わりに\emph{正しく}\footnote{ここの皮肉}動作します。

\lstinputlisting[style=customasmx86]{patterns/02_stack/2.asm}

GCC 4.4.1はどちらの場合も問題の警告を出さずに同様のコードを生成します。

\myparagraph{ARM}

\myindex{ARM!\Registers!Link Register}
また、ARMプログラムはスタックを使用してリターンアドレスを保存しますが、別の方法でスタックを使用します。 
\q{\HelloWorldSectionName}~(\myref{sec:hw_ARM})で述べたように、\ac{RA}は\ac{LR}(\gls{link register})に保存されます。 
ただし、別の関数を呼び出してもう一度\ac{LR}レジスタを使用する必要がある場合は、その値を保存する必要があります。 
\myindex{Function prologue}
通常、関数プロローグに保存されます。

\myindex{ARM!\Instructions!PUSH}
\myindex{ARM!\Instructions!POP}
多くの場合、\INS{PUSH {R4-R7,LR}}のような命令が、エピローグで\INS{POP {R4-R7,PC}}とともに見られます。
したがって、関数で使用されるレジスタ値は、\ac{LR}を含めてスタックに保存されます。

\myindex{ARM!Leaf function}
それにもかかわらず、ある関数が他の関数を呼び出すことがなければ、\ac{RISC}の用語ではそれを
\emph{\gls{leaf function}}\footnote{\href{http://go.yurichev.com/17064}{infocenter.arm.com/help/index.jsp?topic=/com.arm.doc.faqs/ka13785.html}}と呼びます。
その結果、リーフ関数は\ac{LR}レジスタを保存しません(\ac{LR}レジスタを変更しないため)。 
このような関数が小さく、少数のレジスタを使用する場合は、スタックをまったく使用しないことがあります。 
したがって、スタックを使用せずにリーフ関数を呼び出すことができます。
\footnote{いくつかの時間前、PDP-11とVAXでは、CALL命令(他の関数を呼び出す)は高価でした。 
実行時間の50%までが費やされる可能性があるため、小さな機能を多数持つことは\gls{anti-pattern} \InSqBrackets{\TAOUP Chapter 4, Part II}}
これは、外部RAMがスタックに使用されないため、古いx86マシンよりも高速になる可能性があります。これは、スタックのメモリがまだ割り当てられていない状況 または利用できません。

リーフ関数のいくつかの例:
\myref{ARM_leaf_example1}, \myref{ARM_leaf_example2}, 
\myref{ARM_leaf_example3}, \myref{ARM_leaf_example4}, \myref{ARM_leaf_example5},
\myref{ARM_leaf_example6}, \myref{ARM_leaf_example7}, \myref{ARM_leaf_example10}.
}

\EN{\subsubsection{Passing function arguments}

The most popular way to pass parameters in x86 is called \q{cdecl}:

\begin{lstlisting}[style=customasmx86]
push arg3
push arg2
push arg1
call f
add esp, 12 ; 4*3=12
\end{lstlisting}

\Gls{callee} functions get their arguments via the stack pointer.

Therefore, this is how the argument values are located in the stack before the execution of the \ttf{} function's very first instruction:

\begin{center}
\begin{tabular}{ | l | l | }
\hline
ESP & return address \\
\hline
ESP+4 & \argument \#1, \MarkedInIDAAs{} \TT{arg\_0} \\
\hline
ESP+8 & \argument \#2, \MarkedInIDAAs{} \TT{arg\_4} \\
\hline
ESP+0xC & \argument \#3, \MarkedInIDAAs{} \TT{arg\_8} \\
\hline
\dots & \dots \\
\hline
\end{tabular}
\end{center}

For more information on other calling conventions see also section~(\myref{sec:callingconventions}).

\par
By the way, the \gls{callee} function does not have any information about how many arguments were passed.
C functions with a variable number of arguments (like \printf) determine their number using format string specifiers (which begin with the \% symbol).

If we write something like:

\begin{lstlisting}
printf("%d %d %d", 1234);
\end{lstlisting}

\printf will print 1234, and then two random numbers\footnote{Not random in strict sense, but rather unpredictable: \myref{noise_in_stack}}, which were lying next to it in the stack.

\par
That's why it is not very important how we declare the \main function: as \main, \\
\TT{main(int argc, char *argv[])} or \TT{main(int argc, char *argv[], char *envp[])}.

In fact, the \ac{CRT}-code is calling \main roughly as:
	
\begin{lstlisting}[style=customasmx86]
push envp
push argv
push argc
call main
...
\end{lstlisting}

If you declare \main as \main without arguments, they are, nevertheless, still present in the stack, but are not used.
If you declare \main as  \TT{main(int argc, char *argv[])},
you will be able to use first two arguments, and the third will remain \q{invisible} for your function.
Even more, it is possible to declare \TT{main(int argc)}, and it will work.

\myparagraph{Alternative ways of passing arguments}

It is worth noting that nothing obliges programmers to pass arguments through the stack. It is not a requirement.
One could implement any other method without using the stack at all.

A somewhat popular way among assembly language newbies is to pass arguments via global variables, like:

\lstinputlisting[caption=Assembly code,style=customasmx86]{patterns/02_stack/global_args.asm}

But this method has obvious drawback: \emph{do\_something()} function cannot call itself recursively (or via another function),
because it has to zap its own arguments.
The same story with local variables: if you hold them in global variables, the function couldn't call itself.
And this is also not thread-safe
\footnote{Correctly implemented, each thread would have its own stack with its own arguments/variables.}.
A method to store such information in stack makes this easier---it can hold as many function arguments and/or values,
as much space it has.

\InSqBrackets{\TAOCPvolI{}, 189} mentions even weirder schemes particularly convenient on IBM System/360.

\myindex{MS-DOS}
\myindex{x86!\Instructions!INT}

MS-DOS had a way of passing all function arguments via registers, for example, this is piece of
code for ancient 16-bit MS-DOS prints ``Hello, world!'':

\begin{lstlisting}[style=customasmx86]
mov  dx, msg      ; address of message
mov  ah, 9        ; 9 means "print string" function
int  21h          ; DOS "syscall"

mov  ah, 4ch      ; "terminate program" function
int  21h          ; DOS "syscall"

msg  db 'Hello, World!\$'
\end{lstlisting}

\myindex{fastcall}
This is quite similar to \myref{fastcall} method.
And also it's very similar to calling syscalls in Linux (\myref{linux_syscall}) and Windows.

\myindex{x86!\Flags!CF}
If a MS-DOS function is going to return a boolean value (i.e., single bit, usually indicating error state),
\TT{CF} flag was often used.

For example:

\begin{lstlisting}[style=customasmx86]
mov ah, 3ch       ; create file
lea dx, filename
mov cl, 1
int 21h
jc  error
mov file_handle, ax
...
error:
...
\end{lstlisting}

In case of error, \TT{CF} flag is raised. Otherwise, handle of newly created file is returned via \TT{AX}.

This method is still used by assembly language programmers.
In Windows Research Kernel source code (which is quite similar to Windows 2003) we can find something like this
(file \emph{base/ntos/ke/i386/cpu.asm}):

\begin{lstlisting}[style=customasmx86]
        public  Get386Stepping
Get386Stepping  proc

        call    MultiplyTest            ; Perform multiplication test
        jnc     short G3s00             ; if nc, muttest is ok
        mov     ax, 0
        ret
G3s00:
        call    Check386B0              ; Check for B0 stepping
        jnc     short G3s05             ; if nc, it's B1/later
        mov     ax, 100h                ; It is B0/earlier stepping
        ret

G3s05:
        call    Check386D1              ; Check for D1 stepping
        jc      short G3s10             ; if c, it is NOT D1
        mov     ax, 301h                ; It is D1/later stepping
        ret

G3s10:
        mov     ax, 101h                ; assume it is B1 stepping
        ret

	...

MultiplyTest    proc

        xor     cx,cx                   ; 64K times is a nice round number
mlt00:  push    cx
        call    Multiply                ; does this chip's multiply work?
        pop     cx
        jc      short mltx              ; if c, No, exit
        loop    mlt00                   ; if nc, YEs, loop to try again
        clc
mltx:
        ret

MultiplyTest    endp
\end{lstlisting}

}
\RU{\subsubsection{Передача параметров функции}

Самый распространенный способ передачи параметров в x86 называется \q{cdecl}:

\begin{lstlisting}[style=customasmx86]
push arg3
push arg2
push arg1
call f
add esp, 12 ; 4*3=12
\end{lstlisting}

Вызываемая функция получает свои параметры также через указатель стека.

Следовательно, так расположены значения в стеке перед исполнением самой первой инструкции функции \ttf{}:

\begin{center}
\begin{tabular}{ | l | l | }
\hline
ESP & адрес возврата \\
\hline
ESP+4 & \argument \#1, \MarkedInIDAAs{} \TT{arg\_0} \\
\hline
ESP+8 & \argument \#2, \MarkedInIDAAs{} \TT{arg\_4} \\
\hline
ESP+0xC & \argument \#3, \MarkedInIDAAs{} \TT{arg\_8} \\
\hline
\dots & \dots \\
\hline
\end{tabular}
\end{center}

См. также в соответствующем разделе о других способах передачи аргументов через стек~(\myref{sec:callingconventions}).

\par Кстати, вызываемая функция не имеет информации о количестве переданных ей аргументов.
Функции Си с переменным количеством аргументов (как \printf) определяют их количество по спецификаторам строки формата (начинающиеся со знака \%).

Если написать что-то вроде:

\begin{lstlisting}
printf("%d %d %d", 1234);
\end{lstlisting}

\printf выведет 1234, затем ещё два случайных числа\footnote{В строгом смысле, они не случайны, скорее, непредсказуемы: \myref{noise_in_stack}}, которые волею случая оказались в стеке рядом.

\par
Вот почему не так уж и важно, как объявлять функцию \main{}:\\
как \main{}, \TT{main(int argc, char *argv[])}\\
либо \TT{main(int argc, char *argv[], char *envp[])}.

В реальности, \ac{CRT}-код вызывает \main примерно так:
	
\begin{lstlisting}[style=customasmx86]
push envp
push argv
push argc
call main
...
\end{lstlisting}

Если вы объявляете \main без аргументов, они, тем не менее, присутствуют в стеке, но не используются.
Если вы объявите \main как \TT{main(int argc, char *argv[])}, 
вы можете использовать два первых аргумента, а третий останется для вашей функции \q{невидимым}.
Более того, можно даже объявить \TT{main(int argc)}, и это будет работать.

\myparagraph{Альтернативные способы передачи аргументов}

Важно отметить, что, в общем, никто не заставляет программистов передавать параметры именно через стек, это не является требованием к исполняемому коду.
Вы можете делать это совершенно иначе, не используя стек вообще.

В каком-то смысле, популярный метод среди начинающих использовать язык ассемблера,
это передавать аргументы в глобальных переменных, например:

\lstinputlisting[caption=Код на ассемблере,style=customasmx86]{patterns/02_stack/global_args.asm}

Но у этого метода есть очевидный недостаток: ф-ция \emph{do\_something()} не сможет вызвать саму себя рекурсивно (либо, через
какую-то стороннюю ф-цию),
потому что тогда придется затереть свои собственные аргументы.
Та же история с локальными переменными: если хранить их в глобальных переменных, ф-ция не сможет вызывать сама себя.
К тому же, этот метод не безопасный для мультитредовой среды\footnote{При корректной реализации,
каждый тред будет иметь свой собственный стек со своими аргументами/переменными.}.
Способ хранения подобной информации в стеке заметно всё упрощает ---
он может хранить столько аргументов ф-ций и/или значений вообще,
сколько в нем есть места.

В \InSqBrackets{\TAOCPvolI{}, 189} можно прочитать про еще более странные схемы передачи аргументов,
которые были очень удобны на IBM System/360.

\myindex{MS-DOS}
\myindex{x86!\Instructions!INT}

В MS-DOS был метод передачи аргументов через регистры, например, этот фрагмент кода для древней 16-битной MS-DOS
выводит ``Hello, world!'':

\begin{lstlisting}[style=customasmx86]
mov  dx, msg      ; адрес сообщения
mov  ah, 9        ; 9 означает ф-цию "вывод строки"
int  21h          ; DOS "syscall"

mov  ah, 4ch      ; ф-ция "закончить программу"
int  21h          ; DOS "syscall"

msg  db 'Hello, World!\$'
\end{lstlisting}

\myindex{fastcall}
Это очень похоже на метод \myref{fastcall}.
И еще на метод вызовов сисколлов в Linux (\myref{linux_syscall}) и Windows.

\myindex{x86!\Flags!CF}
Если ф-ция в MS-DOS возвращает булево значение (т.е., один бит, обычно сигнализирующий об ошибке),
часто использовался флаг \TT{CF}.

Например:

\begin{lstlisting}[style=customasmx86]
mov ah, 3ch       ; создать файл
lea dx, filename
mov cl, 1
int 21h
jc  error
mov file_handle, ax
...
error:
...
\end{lstlisting}

В случае ошибки, флаг \TT{CF} будет выставлен.
Иначе, хэндл только что созданного файла возвращается в \TT{AX}.

Этот метод до сих пор используется программистами на ассемблере.
В исходных кодах Windows Research Kernel (который очень похож на Windows 2003) мы можем найти такое\\
(файл \emph{base/ntos/ke/i386/cpu.asm}):

\begin{lstlisting}[style=customasmx86]
        public  Get386Stepping
Get386Stepping  proc

        call    MultiplyTest            ; Perform multiplication test
        jnc     short G3s00             ; if nc, muttest is ok
        mov     ax, 0
        ret
G3s00:
        call    Check386B0              ; Check for B0 stepping
        jnc     short G3s05             ; if nc, it's B1/later
        mov     ax, 100h                ; It is B0/earlier stepping
        ret

G3s05:
        call    Check386D1              ; Check for D1 stepping
        jc      short G3s10             ; if c, it is NOT D1
        mov     ax, 301h                ; It is D1/later stepping
        ret

G3s10:
        mov     ax, 101h                ; assume it is B1 stepping
        ret

	...

MultiplyTest    proc

        xor     cx,cx                   ; 64K times is a nice round number
mlt00:  push    cx
        call    Multiply                ; does this chip's multiply work?
        pop     cx
        jc      short mltx              ; if c, No, exit
        loop    mlt00                   ; if nc, YEs, loop to try again
        clc
mltx:
        ret

MultiplyTest    endp
\end{lstlisting}


}
\PTBR{\input{patterns/02_stack/02_args_passing_PTBR}}
\DE{\subsubsection{Funktion Argumente übergeben}

Der übliche weg Argumente in x86 zu übergeben ist die \q{cdecl} Methode:

\begin{lstlisting}[style=customasmx86]
push arg3
push arg2
push arg1
call f
add esp, 12 ; 4*3=12
\end{lstlisting}

Die \Gls{callee} Funktionen bekommen ihre Argumente über den Stackpointer. 

So werden die Argumente auf dem Stack gefunden, noch vor der Ausführung der ersten Instruktion der \ttf{} Funktion:

\begin{center}
\begin{tabular}{ | l | l | }
\hline
ESP & return address \\
\hline
ESP+4 & \argument \#1, \MarkedInIDAAs{} \TT{arg\_0} \\
\hline
ESP+8 & \argument \#2, \MarkedInIDAAs{} \TT{arg\_4} \\
\hline
ESP+0xC & \argument \#3, \MarkedInIDAAs{} \TT{arg\_8} \\
\hline
\dots & \dots \\
\hline
\end{tabular}
\end{center}


Für mehr Informationen über andere Aufrufs Konventionen siehe Sektion:~(\myref{sec:callingconventions}).

\par
Übrigens, die \gls{callee} Funktion hat keine Informationen wie viele Argumente übergeben wurden.
C Funktionen mit einer variablen Anzahl an Argumenten (wie z.B \printf) errechnen die zahl der Argumente anhand der 
Formatstring spezifizier-er (alle spezifizier-er die mit dem \% beginnen).

Wenn wir etwas schreiben wie z.B:

\begin{lstlisting}
printf("%d %d %d", 1234);
\end{lstlisting}

\printf wird die Zahlen 1234 und zwei zufällige Werte ausgeben, welche direkt neben 1234 auf
dem Stack lagen\footnote{Nicht zufällig im eigentlichen Sinne sondern eher unvorhersehbar: \myref{noise_in_stack}}.

\par
Das ist auch der Grund warum es nicht wichtig ist wie die \main Funktion definiert ist: Als \main, \\
\TT{main(int argc, char *argv[])} oder \TT{main(int argc, char *argv[], char *envp[])}.

Tatsächlich ruf der \ac{CRT}-Code die \main Funktion um Grunde so auf:
	
\begin{lstlisting}[style=customasmx86]
push envp
push argv
push argc
call main
...
\end{lstlisting}

Wenn man \main als \main Funktion ohne Argumente definiert, dann liegen sie trotzdem auf dem Stack auch wenn sie 
nicht benutzt werden. Wenn man \main als \TT{main(int argc, char *argv[])}, definiert kann man auf die ersten beiden
Argumente der Funktion zugreifen, das dritte bleibt aber weiterhin ``Unsichtbar'' für andere Funktionen.
Es ist aber auch u.a möglich die Main Funktion als \TT{main(int argc)} schreiben und sie wird noch immer funktionieren.

\myparagraph{Alternative Wege Argumente zu übergeben}

Es sollte bemerkt werden das nichts einen Programmierer dazu zwingt Argumente über den Stack zu übergeben. Das ist
keine generelle Anforderung. Jemand könnte auch einfach eine andere Methode implementieren ohne den Stack überhaupt zu benutzen.

Ein ziemlich beliebter Weg Argumente zu übergeben unter Assembler Neulingen ist über globale Variablen wie z.B:

\lstinputlisting[caption=Assembly code,style=customasmx86]{patterns/02_stack/global_args.asm}

Aber diese Methode hat Nachteile: Die \emph{do\_something()} Funktion kann sich selbst nicht rekursiv aufrufen (aber auch keine andere Funktion),
weil sie ihre eigenen Argumente löschen muss.
Die gleiche Geschichte mit lokalen Variablen: Wenn die Werte in globalen Variablen gespeichert sind, kann die Funktion sich nicht selbst aufrufen.
Und das bedeutet wiederum das die Funktion nicht thread-Safe ist.
\footnote{Korrekt implementiert, hat jeder Thread seinen eigenen Stack und seine eigenen Argumente/Variablen}.
Eine Methode solche Informationen auf dem Stack zu speichern macht die Dinge einfacher--- Der Stack kann so viele Funktion Arguemente und/oder Werte speichern,
so viel Speicher wie der Computer hat.

\InSqBrackets{\TAOCPvolI{}, 189} nennt sogar noch verrückter Methoden die speziell auf IBM System/360 benutzt werden.

\myindex{MS-DOS}
\myindex{x86!\Instructions!INT}

Auf MS-DOS gab es einen Weg Funktion Argumente über Register zu übergeben, zum Beispiel dies 
ist ein Stück Code einer veralteten 16-Bit MS-DOS ``Hallo, Welt!'' Funktion:

\begin{lstlisting}[style=customasmx86]
mov  dx, msg      ; Adresse der Naricht
mov  ah, 9        ; 9 bedeutet ``print string''
int  21h          ; DOS "syscall"

mov  ah, 4ch      ; ``Terminiere Programm'' Funktion
int  21h          ; DOS "syscall"

msg  db 'Hello, World!\$' 
\end{lstlisting}

\myindex{fastcall}
Diese Methode ist der \myref{fastcall} Methode sehr ähnlich. Sie ähnelt aber auch der Methode
wie man auf Linux (\myref{linux_syscall}) und Windows syscalls ausführt.

\myindex{x86!\Flags!CF}
Wenn eine MS-DOS Funktion einen Bool'schen Wert zurück gibt (z.B., Single Bit bedeutet ein Fehler ist aufgetreten), wird dafür das \TT{CF} Flag benutzt.

Zum Beispiel:

\begin{lstlisting}[style=customasmx86]
mov ah, 3ch       ; create file
lea dx, filename
mov cl, 1
int 21h
jc  error
mov file_handle, ax
...
error:
...
\end{lstlisting}

Im Falle eines Fehlers, wird das \TT{CF} Flag gesetzt. Anderenfalls wird ein handle für die neu erstellte Datei über \TT{AX} zurück gegeben. 


Diese Methode wird heute immer noch von Assembler Programmierern benutzt.
Im Windows Reseearch Kernel source Code (der sehr ähnlich zum Windows 2003 Kernel ist) können wir folgenden Code
finden (file \emph{base/ntos/ke/i386/cpu.asm}):

% muss noch die kommentare geändert werden
\begin{lstlisting}[style=customasmx86]
        public  Get386Stepping
Get386Stepping  proc

        call    MultiplyTest            ; Muliplikations Test durchführen
        jnc     short G3s00             ; wenn nc, ist muttest ok
        mov     ax, 0
        ret
G3s00:
        call    Check386B0              ; Prüfe das B0 stepping
        jnc     short G3s05             ; wenn nc, ist es B1/later
        mov     ax, 100h                ; It is B0/earlier stepping
        ret

G3s05:
        call    Check386D1              ; Prüfe das D1 stepping
        jc      short G3s10             ; wenn c, iust es NICHT NOT D1
        mov     ax, 301h                ; Es ist das D1/later stepping
        ret

G3s10:
        mov     ax, 101h                ; annahme das es das it is B1 stepping ist
        ret

	...

MultiplyTest    proc

        xor     cx,cx                   ; 64K durchläufe ist eine nette runde Nummer
mlt00:  push    cx
        call    Multiply                ; Funktioniert dis multiplikation auf diesem Chip?
        pop     cx
        jc      short mltx              ; wenn c c, Nein, exit
        loop    mlt00                   ; Wenn nc, Ja, weitere iteration für nächsten versuch
        clc
mltx:
        ret

MultiplyTest    endp
\end{lstlisting}

}
\IT{\subsubsection{Passaggio di argomenti alle funzioni}

Il modo più diffuso per passare parametri in x86 è detto \q{cdecl}:

\begin{lstlisting}[style=customasmx86]
push arg3
push arg2
push arg1
call f
add esp, 12 ; 4*3=12
\end{lstlisting}

La funzioni chiamate, \Gls{callee}, ricevono i propri argomenti tramite lo stack pointer.

Quindi è così che i valori degli argomenti sono posizionati nello stack prima dell'esecuzione della prima istruzione della funzione \ttf{}:

\begin{center}
\begin{tabular}{ | l | l | }
\hline
ESP & return address \\
\hline
ESP+4 & \argument \#1, \MarkedInIDAAs{} \TT{arg\_0} \\
\hline
ESP+8 & \argument \#2, \MarkedInIDAAs{} \TT{arg\_4} \\
\hline
ESP+0xC & \argument \#3, \MarkedInIDAAs{} \TT{arg\_8} \\
\hline
\dots & \dots \\
\hline
\end{tabular}
\end{center}

Per ulteriori informazioni su altri tipi di convenzioni di chiamata (calling conventions), fare riferimento alla sezione~(\myref{sec:callingconventions}).

\par
A proposito, la funzione \gls{callee}{chiamata} non possiede alcuna informazione su quanti argomenti sono stati passati.
Le funzioni C con un numero variabile di argomenti (come \printf) determinano il loro numero attraverso specificatori di formato stringa (che iniziano con il simbolo \%).

Se scriviamo qualcosa come:

\begin{lstlisting}
printf("%d %d %d", 1234);
\end{lstlisting}

\printf scriverà 1234, e successivamente due numeri casuali\footnote{Non casuali in senso stretto, ma piuttosto non predicibili: \myref{noise_in_stack}}, che si trovavano lì vicino nello stack.

\par
Per questo motivo non è molto importante come dichiariamo la funzione \main: come \main, \\
\TT{main(int argc, char *argv[])} oppure \TT{main(int argc, char *argv[], char *envp[])}.

Infatti, il codice \ac{CRT} sta chiamando \main circa in questo modo:

\begin{lstlisting}[style=customasmx86]
push envp
push argv
push argc
call main
...
\end{lstlisting}

Se dichiari \main come \main senza argomenti, questi sono, in ogni caso, ancora presenti nello stack, ma non vengono utilizzati.
Se dichiari \main come \TT{main(int argc, char *argv[])},
sarai in grado di utilizzare i primi due argomenti, ed il terzo rimarrà \q{invisibile} per la tua funzione.
In più, è possibile dichiarare \TT{main(int argc)}, e continuerà a funzionare.

\myparagraph{Metodi alternativi per passare argomenti}

Vale la pena notare che non c'è nulla che obbliga il programmatore a passare gli argomenti attraverso lo stack. Non è un requisito necessario.
Si potrebbe implementare un qualunque altro metodo anche senza usare per niente lo stack.

Un metodo abbastanza popolare tra chi inizia a programmare in linguaggio assembly language è di passare argomenti attraverso variabili globali, in questo modo:

\lstinputlisting[caption=Assembly code,style=customasmx86]{patterns/02_stack/global_args.asm}

Tuttavia questo metodo ha un limite evidente: la funzione \emph{do\_something()} non può richiamare sè stessa in modo ricorsivo (o attraverso un'altra funzione),
perchè deve cancellare i suoi stessi argomenti.
Lo stesso accade con le variabili locali: se le tieni in variabili globali, la funzione non può chiamare se stessa.
Inoltre questo non sarebbe thread-safe
\footnote{Implementato correttamente, ciascun thread avrebbe il suo proprio stack con i suoi argomenti/variabili.}.
Il metodo di memorizzare queste informazioni nello stack rende il tutto più semplice---può mantenere quanti argomenti di funzione e/o valori,
quanto spazio è disponibile.

\InSqBrackets{\TAOCPvolI{}, 189} menziona alcuni schemi ancora più strani e particolarmente convenienti su IBM System/360.

\myindex{MS-DOS}
\myindex{x86!\Instructions!INT}

MS-DOS utilizzava un modo per passare tutti gli argomenti di funzione via registri, ad esempio, in questo pezzo
di codice per MS-DOS a 16 bit scrive ``Hello, world!'':

\begin{lstlisting}[style=customasmx86]
mov  dx, msg      ; indirizzo del messaggio
mov  ah, 9        ; 9 indica la funzione "print string"
int  21h          ; "syscall" (chiamata di sistema) DOS

mov  ah, 4ch      ; funzione "termina il programma"
int  21h          ; "syscall" DOS

msg  db 'Hello, World!\$'
\end{lstlisting}

\myindex{fastcall}
Questo è abbastanza simile al metodo \myref{fastcall}.
Ed è inoltre molto simile alle chiamate syscalls in Linux (\myref{linux_syscall}) e Windows.

\myindex{x86!\Flags!CF}
Se una funzione MS-DOS restituisce un valore di tipo boolean (cioè, un singolo bit, di solito per indicare uno stato di errore),
il flag \TT{CF} era spesso utilizzato.

Ad esempio:

\begin{lstlisting}[style=customasmx86]
mov ah, 3ch       ; crea file
lea dx, filename
mov cl, 1
int 21h
jc  error
mov file_handle, ax
...
error:
...
\end{lstlisting}

In caso di errore, il flag \TT{CF} viene innalzato. Altrimenti, l'handle ad un nuovo file creato viene restituito attraverso \TT{AX}.

Questo metodo viene ancora utilizzato dai programmatori assembly.
Nel codice sorgente del Windows Research Kernel (che è abbastanza simile a Windows 2003) possiamo trovare qualcosa tipo:
(file \emph{base/ntos/ke/i386/cpu.asm}):

\begin{lstlisting}[style=customasmx86]
        public  Get386Stepping
Get386Stepping  proc

        call    MultiplyTest            ; Esegue test di moltiplicazione
        jnc     short G3s00             ; se nc, muttest è ok
        mov     ax, 0
        ret
G3s00:
        call    Check386B0              ; Verifica B0 stepping
        jnc     short G3s05             ; se nc, è B1/later
        mov     ax, 100h                ; è B0/earlier stepping
        ret

G3s05:
        call    Check386D1              ; Verifica D1 stepping
        jc      short G3s10             ; se c, non è D1
        mov     ax, 301h                ; è D1/later stepping
        ret

G3s10:
        mov     ax, 101h                ; suppone che sia B1 stepping
        ret

	...

MultiplyTest    proc

        xor     cx,cx                   ; 64K volte è un bel numero tondo
mlt00:  push    cx
        call    Multiply                ; la moltiplicazione funziona in questo chip?
        pop     cx
        jc      short mltx              ; se c, No, esci
        loop    mlt00                   ; se nc, Si, cicla per riprovare
        clc
mltx:
        ret

MultiplyTest    endp
\end{lstlisting}
}
\FR{\subsubsection{Passage des arguments d'une fonction}

Le moyen le plus utilisé pour passer des arguments en x86 est appelé \q{cdecl}:

\begin{lstlisting}[style=customasmx86]
push arg3
push arg2
push arg1
call f
add esp, 12 ; 4*3=12
\end{lstlisting}

La fonction \glslink{callee}{appelée} reçoit ses arguments par la pile.

Voici donc comment sont stockés les arguments sur la pile avant l'exécution
de la première instruction de la fonction \ttf{}:

\begin{center}
\begin{tabular}{ | l | l | }
\hline
ESP & return address \\
\hline
ESP+4 & \argument \#1, \MarkedInIDAAs{} \TT{arg\_0} \\
\hline
ESP+8 & \argument \#2, \MarkedInIDAAs{} \TT{arg\_4} \\
\hline
ESP+0xC & \argument \#3, \MarkedInIDAAs{} \TT{arg\_8} \\
\hline
\dots & \dots \\
\hline
\end{tabular}
\end{center}

Pour plus d'information sur les conventions d'appel, voir cette section~(\myref{sec:callingconventions}).

\par
À propos, la fonction \glslink{callee}{appelée} n'a aucune d'information sur le
nombre d'arguments qui ont été passés.
Les fonctions C avec un nombre variable d'arguments (comme \printf) déterminent
leur nombre en utilisant les spécificateurs de la chaîne de format (qui commencent
pas le symbole \%).

Si nous écrivons quelque comme:

\begin{lstlisting}
printf("%d %d %d", 1234);
\end{lstlisting}

\printf va afficher 1234, et deux autres nombres aléatoires\footnote{Pas aléatoire
dans le sens strict du terme, mais plutôt imprévisibles: \myref{noise_in_stack}},
qui sont situés à côté dans la pile.

\par
C'est pourquoi la façon dont la fonction \main est déclarée n'est pas très importante:
comme \main, \\\TT{main(int argc, char *argv[])} ou \TT{main(int argc, char *argv[], char *envp[])}.

En fait, le code-\ac{CRT} appelle \main, schématiquement, de cette façon:
	
\begin{lstlisting}[style=customasmx86]
push envp
push argv
push argc
call main
...
\end{lstlisting}

Si vous déclarez \main comme \main sans argument, ils sont néanmoins toujours présents
sur la pile, mais ne sont pas utilisés.
Si vous déclarez \main as comme \TT{main(int argc, char *argv[])},
vous pourrez utiliser les deux premiers arguments, et le troisième restera \q{invisible}
pour votre fonction.
Il est même possible de déclarer \main comme \TT{main(int argc)}, cela fonctionnera.

\myparagraph{Autres façons de passer les arguments}

Il est à noter que rien n'oblige les programmeurs à passer les arguments à travers
la pile. Ce n'est pas une exigence.
On peut implémenter n'importe quelle autre méthode sans utiliser du tout la pile.

Une méthode répandue chez les débutants en assembleur est de passer les arguments
par des variables globales, comme:

\lstinputlisting[caption=Code assembleur,style=customasmx86]{patterns/02_stack/global_args.asm}

Mais cette méthode a un inconvénient évident: la fonction \emph{do\_something()}
ne peut pas s'appeler elle-même récursivement (ou par une autre fonction),
car il faudrait écraser ses propres arguments.
La même histoire avec les variables locales: si vous les stockez dans des variables
globales, la fonction ne peut pas s'appeler elle-même.
Et ce n'est pas thread-safe
\footnote{Correctement implémenté, chaque thread aurait sa propre pile avec ses propres arguments/variables.}.
Une méthode qui stocke ces informations sur la pile rend cela plus facile---elle
peut contenir autant d'arguments de fonctions et/ou de valeurs, que la pile a d'espace.

\InSqBrackets{\TAOCPvolI{}, 189} mentionne un schéma encore plus étrange, particulièrement
pratique sur les IBM System/360.

\myindex{MS-DOS}
\myindex{x86!\Instructions!INT}

MS-DOS a une manière de passer tous les arguments de fonctions via des registres,
par exemple, c'est un morceau de code pour un ancien MS-DOS 16-bit qui affiche
``Hello, world!'':

\begin{lstlisting}[style=customasmx86]
mov  dx, msg      ; address of message
mov  ah, 9        ; 9 means "print string" function
int  21h          ; DOS "syscall"

mov  ah, 4ch      ; "terminate program" function
int  21h          ; DOS "syscall"

msg  db 'Hello, World!\$'
\end{lstlisting}

\myindex{fastcall}
C'est presque similaire à la méthode \myref{fastcall}.
Et c'est aussi très similaire aux appels systèmes sous Linux (\myref{linux_syscall}) et Windows.

\myindex{x86!\Flags!CF}
Si une fonction MS-DOS devait renvoyer une valeur booléenne (i.e., un simple bit,
souvent pour indiquer un état d'erreur), le flag \TT{CF} était souvent utilisé.

Par exemple:

\begin{lstlisting}[style=customasmx86]
mov ah, 3ch       ; create file
lea dx, filename
mov cl, 1
int 21h
jc  error
mov file_handle, ax
...
error:
...
\end{lstlisting}

En cas d'erreur, le flag \TT{CF} est mis. Sinon, le handle du fichier nouvellement
créé est retourné via \TT{AX}.

Cette méthode est encore utilisée par les programmeurs en langage d'assemblage.
Dans le code source de Windows Research Kernel (qui est très similaire à Windows
2003) nous pouvons trouver quelque chose comme ça (file \emph{base/ntos/ke/i386/cpu.asm}):

\begin{lstlisting}[style=customasmx86]
        public  Get386Stepping
Get386Stepping  proc

        call    MultiplyTest            ; Perform multiplication test
        jnc     short G3s00             ; if nc, muttest is ok
        mov     ax, 0
        ret
G3s00:
        call    Check386B0              ; Check for B0 stepping
        jnc     short G3s05             ; if nc, it's B1/later
        mov     ax, 100h                ; It is B0/earlier stepping
        ret

G3s05:
        call    Check386D1              ; Check for D1 stepping
        jc      short G3s10             ; if c, it is NOT D1
        mov     ax, 301h                ; It is D1/later stepping
        ret

G3s10:
        mov     ax, 101h                ; assume it is B1 stepping
        ret

	...

MultiplyTest    proc

        xor     cx,cx                   ; 64K times is a nice round number
mlt00:  push    cx
        call    Multiply                ; does this chip's multiply work?
        pop     cx
        jc      short mltx              ; if c, No, exit
        loop    mlt00                   ; if nc, YEs, loop to try again
        clc
mltx:
        ret

MultiplyTest    endp
\end{lstlisting}

}
\JA{\subsubsection{関数の引数を渡す}

x86でパラメータを渡す最も一般的な方法は、\q{cdecl}です。

\begin{lstlisting}[style=customasmx86]
push arg3
push arg2
push arg1
call f
add esp, 12 ; 4*3=12
\end{lstlisting}

\gls{callee}関数はスタックポインタを介して引数を取得します。

したがって、 \ttf{} 関数の最初の命令が実行される前に、引数の値がスタックにどのように格納されているかがわかります。

\begin{center}
\begin{tabular}{ | l | l | }
\hline
ESP & return address \\
\hline
ESP+4 & \argument \#1, \MarkedInIDAAs{} \TT{arg\_0} \\
\hline
ESP+8 & \argument \#2, \MarkedInIDAAs{} \TT{arg\_4} \\
\hline
ESP+0xC & \argument \#3, \MarkedInIDAAs{} \TT{arg\_8} \\
\hline
\dots & \dots \\
\hline
\end{tabular}
\end{center}

他の呼び出し規約の詳細については、セクション(\myref{sec:callingconventions})も参照してください。

\par
ちなみに、\gls{callee}関数には、渡された引数の数に関する情報はありません。
( \printf のような)可変数の引数を持つC関数は、フォーマット文字列指定子(\%記号で始まる)を使ってその数を決定します。

私たちが次のように書くとします。

\begin{lstlisting}
printf("%d %d %d", 1234);
\end{lstlisting}

\printf は1234を出力し、次にそのスタックの隣にある2つの乱数\footnote{厳密な意味でランダムではなく、むしろ予測不可能: \myref{noise_in_stack}}を出力します。

\par
だから、\main 関数を宣言する方法はあまり重要ではありません: \main
\TT{main(int argc, char *argv[])} または \TT{main(int argc, char *argv[], char *envp[])}のいずれかです。

実際、\ac{CRT}コードは \main を以下のように呼び出しています:
	
\begin{lstlisting}[style=customasmx86]
push envp
push argv
push argc
call main
...
\end{lstlisting}

引数なしで \main を \main として宣言すると、\main はスタックにまだ残っていますが使用されません。 
\main を\TT{main(int argc, char *argv[])}として宣言すると、
最初の2つの引数を使用することができ、3つ目の引数は関数の \q{不可視}のままになります。 
さらに、\TT{main(int argc)}を宣言することも可能です。これは動作します。

\myparagraph{引数を渡す別の方法}

プログラマがスタックを介して引数を渡すことは何も必要ではないことは注目に値する。
それは要件ではありません。 スタックをまったく使用せずに他の方法を実装することもできます。

アセンブリ言語初心者の間でやや普及している方法は、グローバル変数を介して引数を渡すことです

\lstinputlisting[caption=Assembly code,style=customasmx86]{patterns/02_stack/global_args.asm}

しかし、このメソッドには明白な欠点があります。\emph{do\_something()}関数は、独自の引数をzapする必要があるため、
再帰的に(または別の関数を介して)呼び出すことはできません。 
ローカル変数を使った同じ話:グローバル変数でそれらを保持すると、関数は自分自身を呼び出すことができませんでした。 
また、これはスレッドセーフ
\footnote{正しく実装され、各スレッドは独自の引数/変数を持つ独自のスタックを持ちます}
ではありません。このような情報をスタックに格納する方法は、これをより簡単にします。多くの関数の引数や値、スペースを確保できます。

\InSqBrackets{\TAOCPvolI{}, 189}は、IBM System/360上で特に便利な奇妙なスキームについても言及しています。

\myindex{MS-DOS}
\myindex{x86!\Instructions!INT}

MS-DOSには、レジスタを介してすべての関数引数を渡す方法がありました。
たとえば、古代16ビットMS-DOSの ``Hello, world!''コードのコードです。

\begin{lstlisting}[style=customasmx86]
mov  dx, msg      ; address of message
mov  ah, 9        ; 9 means "print string" function
int  21h          ; DOS "syscall"

mov  ah, 4ch      ; "terminate program" function
int  21h          ; DOS "syscall"

msg  db 'Hello, World!\$'
\end{lstlisting}

\myindex{fastcall}
これは、\myref{fastcall}のメソッドと非常によく似ています。 
また、Linuxのsyscalls((\myref{linux_syscall}))とWindowsを呼び出すのと非常によく似ています。

\myindex{x86!\Flags!CF}
MS-DOS関数がブール値(すなわち単一ビット、通常はエラー状態を示す)を返す場合、\TT{CF}フラグがしばしば使用されます。

例えば:

\begin{lstlisting}[style=customasmx86]
mov ah, 3ch       ; create file
lea dx, filename
mov cl, 1
int 21h
jc  error
mov file_handle, ax
...
error:
...
\end{lstlisting}

エラーの場合、\TT{CF}フラグが立てられます。 それ以外の場合は、新しく作成されたファイルのハンドルが\TT{AX}を介して返されます。

このメソッドは、アセンブリ言語プログラマによって引き続き使用されます。 
Windows Research Kernelのソースコード(Windows 2003と非常に似ています)では、次のようなものが見つかります

(ファイル \emph{base/ntos/ke/i386/cpu.asm})

\begin{lstlisting}[style=customasmx86]
        public  Get386Stepping
Get386Stepping  proc

        call    MultiplyTest            ; Perform multiplication test
        jnc     short G3s00             ; if nc, muttest is ok
        mov     ax, 0
        ret
G3s00:
        call    Check386B0              ; Check for B0 stepping
        jnc     short G3s05             ; if nc, it's B1/later
        mov     ax, 100h                ; It is B0/earlier stepping
        ret

G3s05:
        call    Check386D1              ; Check for D1 stepping
        jc      short G3s10             ; if c, it is NOT D1
        mov     ax, 301h                ; It is D1/later stepping
        ret

G3s10:
        mov     ax, 101h                ; assume it is B1 stepping
        ret

	...

MultiplyTest    proc

        xor     cx,cx                   ; 64K times is a nice round number
mlt00:  push    cx
        call    Multiply                ; does this chip's multiply work?
        pop     cx
        jc      short mltx              ; if c, No, exit
        loop    mlt00                   ; if nc, YEs, loop to try again
        clc
mltx:
        ret

MultiplyTest    endp
\end{lstlisting}

}
\PL{\subsubsection{Przekazywanie argumentów funkcji}

Najbardziej powszechny sposób na przekazywanie parametrów funkcji w x86 to \q{cdecl}:

\begin{lstlisting}[style=customasmx86]
push arg3
push arg2
push arg1
call f
add esp, 12 ; 4*3=12
\end{lstlisting}

Wywoływana funkcja otrzymuje swoje parametry również przez wskaźnik stosu.

W konsekwensji tego, zawartość stosu przed wykonaniem pierwszej instrukcji funkcji wygląda w ten sposób \ttf{}:

\begin{center}
\begin{tabular}{ | l | l | }
\hline
ESP & adres powrotu \\
\hline
ESP+4 & \argument \#1, \MarkedInIDAAs{} \TT{arg\_0} \\
\hline
ESP+8 & \argument \#2, \MarkedInIDAAs{} \TT{arg\_4} \\
\hline
ESP+0xC & \argument \#3, \MarkedInIDAAs{} \TT{arg\_8} \\
\hline
\dots & \dots \\
\hline
\end{tabular}
\end{center}

Patrz również w odpowiednim rodziale o innych sposobach przekazywania argumentów przez stos ~(\myref{sec:callingconventions}).

\par A propos, funkcja wywoływana nie posiada informacji o ilości argumentów przekazywanych do niej.
Funkcje w C o zmiennej ilości parametrów (jak np. \printf) wyznaczają ich ilość za pomocą specjalnych specyfikatorów (rozpoczynających się z \%).

Jeśli napisać coś w stylu:

\begin{lstlisting}
printf("%d %d %d", 1234);
\end{lstlisting}

\printf wyprowadzi 1234, następnie jeszcze dwie liczby losowe\footnote{Tak na prawdę nie są one losowe, patrz: \myref{noise_in_stack}}, który przypadkowo okazały się na stosie obok.

\par
Właśnie dlatego nie jest to ważne jak zapiszemy f-cję \main{}:\\
jak \main{}, \TT{main(int argc, char *argv[])}\\
lub \TT{main(int argc, char *argv[], char *envp[])}.

W rzeczywistości, \ac{CRT}-kod wywołuje \main mniej więcej w ten sposób:
	
\begin{lstlisting}[style=customasmx86]
push envp
push argv
push argc
call main
...
\end{lstlisting}

Jeśli zadeklarujecie \main bez argumentów, one, jednak, są obecne na stosie, lecz nie są wykorzystywane.
Jeśli zadeklarujecie \main jako \TT{main(int argc, char *argv[])}, 
to można korzystać z pierwszych dwóch argumentów, a trzeci zostanie dla funkcji \q{niewidocznym}.
Co więcej, można nawet zadeklarować \TT{main(int argc)}, i to zadziała.

\myparagraph{Alternatywne sposoby na przekazywanie argumentów}

Warto zauważyć,że, generalnie, nie ma narzutu na przekazywanie argumentów przez stos, nie jest to wymogiem formalnym.
Można robić to zupełnie inaczej, nie korzystając ze stosu w ogóle.

W pewnym sensie, popularną metodą wśród początkującyh jest przekazywanie argumentów przez zmienne globalne, na przykład:

\lstinputlisting[caption=Kod w asemblerze,style=customasmx86]{patterns/02_stack/global_args.asm}

Ale ta metoda posiada dużą wadę: funkcja \emph{do\_something()} nie może wywołać sama siebie poprzez rekurencję (lub za pomocą innej funkcji),
dlatego że wtedy będzie trzeba wymazać własne argumenty.
Ta sama historia ze zmiennymi lokalnymi: jeśli przechowywać je w zmiennych globalnych, funkcja nie będzie nogła wywołać sama siebie.
Do tego, ta metoda nie jest biezpieczna dla środowiska wielowątkowego\footnote{Przy poprawnej realizacji,
każdy wątek będzie miał własny stos lokalny ze swoimi argumentami/zmiennymi.}.
Metoda przechowywania podobnej informacji na stosie wszystko znacznie upraszcza ---
on może przechowywać tyle argumentów funkcji/zmiennych,
ile się w nim zmieści.

W \InSqBrackets{\TAOCPvolI{}, 189} można przeczytać o jeszcze bardziej dziwnych metodach przekazywania argumentów funkcji, które były bardzo wygodne na
 IBM System/360.

\myindex{MS-DOS}
\myindex{x86!\Instructions!INT}

W MS-DOS istniała metoda przekazywania argumentów przez rejestry, na przykład, ten fragment kodu dla starego 16-bitowego MS-DOS
wyprintuje ``Hello, world!'':

\begin{lstlisting}[style=customasmx86]
mov  dx, msg      ; adres powiadomienia
mov  ah, 9        ; §9 oznacza funkcję "wyprowadzenie linii"§
int  21h          ; DOS "syscall"

mov  ah, 4ch      ; §funkcja zakończenia programu
int  21h          ; DOS "syscall"

msg  db 'Hello, World!\$'
\end{lstlisting}

\myindex{fastcall}
Jest to bardzo podobne do metody \myref{fastcall}.
I jeszcze do sposobu robienia syscall w Linux (\myref{linux_syscall}) i Windows.

\myindex{x86!\Flags!CF}
Jeżeli f-cja w MS-DOS zwraca boolean (tzn jeden bit, zwule sygnalizujący o błędzie wewnętrznym),
to często była wykorzystywana flaga \TT{CF}.

Na przykład:

\begin{lstlisting}[style=customasmx86]
mov ah, 3ch       ; stworzyć plik
lea dx, filename
mov cl, 1
int 21h
jc  error
mov file_handle, ax
...
error:
...
\end{lstlisting}

W przypadku wystąpienia błędu, flaga \TT{CF} zostaje ustawiona.
W innym przypadku, handle stworzonego pliku jest zwracany do \TT{AX}.

Ta metoda dotychczasowo jest wykorzystywana przez programistów asemblera.
W kodach wyjściowych Windows Research Kernel (który jest bardzo podobny do Windows 2003) możemy znaleźć coś takiego\\
(plik \emph{base/ntos/ke/i386/cpu.asm}):

\begin{lstlisting}[style=customasmx86]
        public  Get386Stepping
Get386Stepping  proc

        call    MultiplyTest            ; Perform multiplication test
        jnc     short G3s00             ; if nc, muttest is ok
        mov     ax, 0
        ret
G3s00:
        call    Check386B0              ; Check for B0 stepping
        jnc     short G3s05             ; if nc, it's B1/later
        mov     ax, 100h                ; It is B0/earlier stepping
        ret

G3s05:
        call    Check386D1              ; Check for D1 stepping
        jc      short G3s10             ; if c, it is NOT D1
        mov     ax, 301h                ; It is D1/later stepping
        ret

G3s10:
        mov     ax, 101h                ; assume it is B1 stepping
        ret

	...

MultiplyTest    proc

        xor     cx,cx                   ; 64K times is a nice round number
mlt00:  push    cx
        call    Multiply                ; does this chip's multiply work?
        pop     cx
        jc      short mltx              ; if c, No, exit
        loop    mlt00                   ; if nc, YEs, loop to try again
        clc
mltx:
        ret

MultiplyTest    endp
\end{lstlisting}



}


\EN{\input{patterns/02_stack/03_local_vars_EN}}
\RU{\input{patterns/02_stack/03_local_vars_RU}}
\PTBR{\input{patterns/02_stack/03_local_vars_PTBR}}
\EN{\subsubsection{x86: alloca() function}
\label{alloca}
\myindex{\CStandardLibrary!alloca()}

\newcommand{\AllocaSrcPath}{C:\textbackslash{}Program Files (x86)\textbackslash{}Microsoft Visual Studio 10.0\textbackslash{}VC\textbackslash{}crt\textbackslash{}src\textbackslash{}intel}

It is worth noting the \TT{alloca()} function
\footnote{In MSVC, the function implementation can be found in \TT{alloca16.asm} and \TT{chkstk.asm} in \\
\TT{\AllocaSrcPath{}}}.
This function works like \TT{malloc()}, but allocates memory directly on the stack.
% page break added to prevent "\vref on page boundary" error. it may be dropped in future.
The allocated memory chunk does not have to be freed via a \TT{free()} function call, \\
since the function epilogue (\myref{sec:prologepilog}) returns \ESP back to its initial state and 
the allocated memory is just \emph{dropped}.
It is worth noting how \TT{alloca()} is implemented.
In simple terms, this function just shifts \ESP downwards toward the stack bottom by the number of bytes you need and sets \ESP as a pointer to the \emph{allocated} block.

Let's try:

\lstinputlisting[style=customc]{patterns/02_stack/04_alloca/2_1.c}

\TT{\_snprintf()} function works just like \printf, but instead of dumping the result into \gls{stdout} (e.g., to terminal or 
console), it writes it to the \TT{buf} buffer. Function \puts copies the contents of \TT{buf} to \gls{stdout}. Of course, these two
function calls might be replaced by one \printf call, but we have to illustrate small buffer usage.

\myparagraph{MSVC}

Let's compile (MSVC 2010):

\lstinputlisting[caption=MSVC 2010,style=customasmx86]{patterns/02_stack/04_alloca/2_2_msvc.asm}

\myindex{Compiler intrinsic}
The sole \TT{alloca()} argument is passed via \EAX (instead of pushing it into the stack)
\footnote{It is because alloca() is rather a compiler intrinsic (\myref{sec:compiler_intrinsic}) than a normal function.
One of the reasons we need a separate function instead of just a couple of instructions in the code,
is because the \ac{MSVC} alloca() implementation also has code which reads from the memory just allocated, in order to let the \ac{OS} map
physical memory to this \ac{VM} region.
After the \TT{alloca()} call, \ESP points to the block of 600 bytes and we can use it as memory for the \TT{buf} array.}.

\myparagraph{GCC + \IntelSyntax}

GCC 4.4.1 does the same without calling external functions:

\lstinputlisting[caption=GCC 4.7.3,style=customasmx86]{patterns/02_stack/04_alloca/2_1_gcc_intel_O3_EN.asm}

\myparagraph{GCC + \ATTSyntax}

Let's see the same code, but in AT\&T syntax:

\lstinputlisting[caption=GCC 4.7.3,style=customasmx86]{patterns/02_stack/04_alloca/2_1_gcc_ATT_O3.s}

\myindex{\ATTSyntax}
The code is the same as in the previous listing.

By the way, \INS{movl \$3, 20(\%esp)} corresponds to
\INS{mov DWORD PTR [esp+20], 3} in Intel-syntax.
In the AT\&T syntax, the register+offset format of addressing memory looks like
\TT{offset(\%{register})}.

}
\FR{\subsubsection{x86: alloca() function}
\label{alloca}
\myindex{\CStandardLibrary!alloca()}

\newcommand{\AllocaSrcPath}{C:\textbackslash{}Program Files (x86)\textbackslash{}Microsoft Visual Studio 10.0\textbackslash{}VC\textbackslash{}crt\textbackslash{}src\textbackslash{}intel}

Intéressons-nous à la fonction \TT{alloca()}
\footnote{Avec MSVC, l'implémentation de cette fonction peut être trouvée dans les fichiers \TT{alloca16.asm} et \TT{chkstk.asm} dans \\
\TT{\AllocaSrcPath{}}}

Cette fonction fonctionne comme \TT{malloc()}, mais alloue de la mémoire directement sur la pile.
% page break added to prevent "\vref on page boundary" error. it may be dropped in future.
L'espace de mémoire ne doit pas être libéré via un appel à la fonction \TT{free()},
puisque l'épilogue de fonction (\myref{sec:prologepilog}) remet \ESP à son état initial
ce qui va automatiquement libérer cet espace mémoire.

Intéressons-nous à l'implémentation d'\TT{alloca()}.
Cette fonction décale simplement \ESP du nombre d'octets demandé vers le bas de la
pile et définit \ESP comme un pointeur vers la mémoire \emph{allouée}.

Essayons :

\lstinputlisting[style=customc]{patterns/02_stack/04_alloca/2_1.c}

La fonction \TT{\_snprintf()} fonctionne comme \printf, mais au lieu d'afficher le
résultat sur la \glslink{stdout}{sortie standard} (ex., dans un terminal ou une console), il l'écrit dans
le buffer \TT{buf}. La fonction \puts copie le contenu de \TT{buf} dans la \glslink{stdout}{sortie standard}.
Évidemment, ces deux appels de fonctions peuvent être remplacés par un seul appel à
la fonction \printf, mais nous devons illustrer l'utilisation de petit buffer.

\myparagraph{MSVC}

Compilons (MSVC 2010) :

\lstinputlisting[caption=MSVC 2010,style=customasmx86]{patterns/02_stack/04_alloca/2_2_msvc.asm}

\myindex{Compiler intrinsic}
Le seul argument d'\TT{alloca()} est passé via \EAX (au lieu de le mettre sur la pile)
\footnote{C'est parce que alloca() est plutôt une fonctionnalité intrinsèque du compilateur (\myref{sec:compiler_intrinsic}) qu'une fonction normale. Une des raisons pour laquelle nous avons besoin d'une fonction séparée au lieu de quelques instructions dans le code, est parce que l'implémentation d'alloca() par \ac{MSVC} a également du code qui lit depuis la mémoire récemment allouée pour laisser l'\ac{OS} mapper la mémoire physique vers la \ac{VM}. Aprés l'appel à la fonction \TT{alloca()}, \ESP pointe sur un bloc de 600 octets que nous pouvons utiliser pour le tableau \TT{buf}.}.

\myparagraph{GCC + \IntelSyntax}

GCC 4.4.1 fait la même chose sans effectuer d'appel à des fonctions externes :

\lstinputlisting[caption=GCC 4.7.3,style=customasmx86]{patterns/02_stack/04_alloca/2_1_gcc_intel_O3_FR.asm}

\myparagraph{GCC + \ATTSyntax}

Voyons le même code mais avec la syntaxe AT\&T :

\lstinputlisting[caption=GCC 4.7.3,style=customasmx86]{patterns/02_stack/04_alloca/2_1_gcc_ATT_O3.s}

\myindex{\ATTSyntax}
Le code est le même que le précédent.

Au fait, \INS{movl \$3, 20(\%esp)} correspond à
\INS{mov DWORD PTR [esp+20], 3} avec la syntaxe intel.
Dans la syntaxe AT\&T, le format registre+offset pour l'adressage mémoire
ressemble à \TT{offset(\%{register})}.
}
\RU{\subsubsection{x86: Функция alloca()}
\label{alloca}
\myindex{\CStandardLibrary!alloca()}

\newcommand{\AllocaSrcPath}{C:\textbackslash{}Program Files (x86)\textbackslash{}Microsoft Visual Studio 10.0\textbackslash{}VC\textbackslash{}crt\textbackslash{}src\textbackslash{}intel}

Интересен случай с функцией \TT{alloca()}
\footnote{В MSVC, реализацию функции можно посмотреть в файлах \TT{alloca16.asm} и \TT{chkstk.asm} в \\
\TT{\AllocaSrcPath{}}}. 
Эта функция работает как \TT{malloc()}, но выделяет память прямо в стеке.
Память освобождать через \TT{free()} не нужно, так как эпилог функции~(\myref{sec:prologepilog})
вернет \ESP в изначальное состояние и выделенная память просто \emph{выкидывается}.
Интересна реализация функции \TT{alloca()}.
Эта функция, если упрощенно, просто сдвигает \ESP вглубь стека на столько байт, сколько вам нужно и возвращает \ESP в качестве указателя на выделенный блок.

Попробуем:

\lstinputlisting[style=customc]{patterns/02_stack/04_alloca/2_1.c}

Функция \TT{\_snprintf()} работает так же, как и \printf, только вместо выдачи результата в \gls{stdout} (т.е. на терминал или в консоль),
записывает его в буфер \TT{buf}. Функция \puts выдает содержимое буфера \TT{buf} в \gls{stdout}. Конечно, можно было бы
заменить оба этих вызова на один \printf, но здесь нужно проиллюстрировать использование небольшого буфера.

\myparagraph{MSVC}

Компилируем (MSVC 2010):

\lstinputlisting[caption=MSVC 2010,style=customasmx86]{patterns/02_stack/04_alloca/2_2_msvc.asm}

\myindex{Compiler intrinsic}
Единственный параметр в \TT{alloca()} передается через \EAX, а не как обычно через стек
\footnote{Это потому, что alloca()~--- это не сколько функция, сколько т.н. \emph{compiler intrinsic} (\myref{sec:compiler_intrinsic})
Одна из причин, почему здесь нужна именно функция, а не несколько инструкций прямо в коде в том, что в реализации 
функции alloca() от \ac{MSVC}
есть также код, читающий из только что выделенной памяти, чтобы \ac{OS} подключила физическую память к этому региону \ac{VM}.
После вызова \TT{alloca()} \ESP указывает на блок в 600 байт, который мы можем использовать под \TT{buf}.}.

\myparagraph{GCC + \IntelSyntax}

А GCC 4.4.1 обходится без вызова других функций:

\lstinputlisting[caption=GCC 4.7.3,style=customasmx86]{patterns/02_stack/04_alloca/2_1_gcc_intel_O3_RU.asm}

\myparagraph{GCC + \ATTSyntax}

Посмотрим на тот же код, только в синтаксисе AT\&T:

\lstinputlisting[caption=GCC 4.7.3,style=customasmx86]{patterns/02_stack/04_alloca/2_1_gcc_ATT_O3.s}

\myindex{\ATTSyntax}
Всё то же самое, что и в прошлом листинге.

Кстати, \INS{movl \$3, 20(\%esp)}~--- это аналог \INS{mov DWORD PTR [esp+20], 3} в синтаксисе Intel.
Адресация памяти в виде \emph{регистр+смещение} записывается в синтаксисе AT\&T как \TT{смещение(\%{регистр})}.

}
\PTBR{\mysection{\PrintfSeveralArgumentsSectionName}

Agora vamos extender o nosso exemplo \emph{\HelloWorldSectionName}~(\myref{sec:helloworld}),
trocando \printf no corpo da função \main() por isso:

\lstinputlisting[label=hw_c,style=customc]{patterns/03_printf/1.c}

% sections
\documentclass[a4paper,oneside]{book}

% http://www.tex.ac.uk/FAQ-noroom.html
\usepackage{etex}

\usepackage[table,usenames,dvipsnames]{xcolor}

\usepackage{fontspec}
% fonts
%\setmonofont{DroidSansMono}
%\setmainfont[Ligatures=TeX]{PT Sans}
%\setmainfont{DroidSans}
\setmainfont{DejaVu Sans}
\setmonofont{DejaVu Sans Mono}
\usepackage{polyglossia}
\defaultfontfeatures{Scale=MatchLowercase} % ensure all fonts have the same 1ex
\usepackage{ucharclasses}
\usepackage{csquotes}

\ifdefined\ENGLISH
%\wlog{main ENGLISH defined OK}
\setmainlanguage{english}
\setotherlanguage{russian}
\fi

\ifdefined\RUSSIAN
\setmainlanguage{russian}
%\newfontfamily\cyrillicfont{LiberationSans}
%\newfontfamily\cyrillicfonttt{LiberationMono}
%\newfontfamily\cyrillicfontsf{lmsans10-regular.otf}
\setotherlanguage{english}
\fi

\ifdefined\GERMAN
%\wlog{main GERMAN defined OK}
\setmainlanguage{german}
\setotherlanguage{english}
\fi

\ifdefined\SPANISH
\setmainlanguage{spanish}
\setotherlanguage{english}
\fi

\ifdefined\ITALIAN
\setmainlanguage{italian}
\setotherlanguage{english}
\fi

\ifdefined\BRAZILIAN
\setmainlanguage{portuges}
\setotherlanguage{english}
\fi

\ifdefined\POLISH
\setmainlanguage{polish}
\setotherlanguage{english}
\fi

\ifdefined\DUTCH
\setmainlanguage{dutch}
\setotherlanguage{english}
\fi

\ifdefined\TURKISH
\setmainlanguage{turkish}
\setotherlanguage{english}
\fi

\ifdefined\THAI
\setmainlanguage{thai}
%\usepackage[thai]{babel}
%\usepackage{fonts-tlwg}
\setmainfont[Script=Thai]{TH SarabunPSK}
\newfontfamily{\thaifont}[Script=Thai]{TH SarabunPSK}
\let\thaifonttt\ttfamily
\setotherlanguage{english}
\fi

\ifdefined\FRENCH
\setmainlanguage{french}
\setotherlanguage{english}
\fi

\ifdefined\JAPANESE
\usepackage{xeCJK}
\xeCJKallowbreakbetweenpuncts
\defaultfontfeatures{Ligatures=TeX,Scale=MatchLowercase}
\setCJKmainfont{IPAGothic}
\setCJKsansfont{IPAGothic}
\setCJKmonofont{IPAGothic}
\DeclareQuoteStyle{japanese}
  {「}
  {」}
  {『}
  {』}
\setquotestyle{japanese}
\setmainlanguage{japanese}
\setotherlanguage{english}
\fi

\usepackage{microtype}
\usepackage{fancyhdr}
\usepackage{listings}
%\usepackage{ulem} % used for \sout{}...
\usepackage{url}
\usepackage{graphicx}
\usepackage{makeidx}
\usepackage[cm]{fullpage}
%\usepackage{color}
\usepackage{fancyvrb}
\usepackage{xspace}
\usepackage{tabularx}
\usepackage{framed}
\usepackage{parskip}
\usepackage{epigraph}
\usepackage{ccicons}
\usepackage[nottoc]{tocbibind}
\usepackage{longtable}
\usepackage[footnote,printonlyused,withpage]{acronym}
\usepackage[]{bookmark,hyperref} % must be last
\usepackage[official]{eurosym}
\usepackage[usestackEOL]{stackengine}

% ************** myref
% http://tex.stackexchange.com/questions/228286/how-to-mix-ref-and-pageref#228292
\ifdefined\RUSSIAN
\newcommand{\myref}[1]{%
  \ref{#1}
  (стр.~\pageref{#1})%
  }
% FIXME: I wasn't able to force varioref to output russian text...
\else
\usepackage{varioref}
\newcommand{\myref}[1]{\vref{#1}}
\fi
% ************** myref

\usepackage{glossaries}
\usepackage{tikz}
%\usepackage{fixltx2e}
\usepackage{bytefield}

\usepackage{amsmath}
\usepackage{MnSymbol}
\undef\mathdollar

\usepackage{float}

\usepackage{shorttoc}
\usetikzlibrary{calc,positioning,chains,arrows}
\usepackage[margin=0.5in,headheight=15.5pt]{geometry}
% FIXME would be good without "margin="...
%\usepackage[headheight=15.5pt]{geometry}

%--------------------
% to prevent clashing of numbers and titles in TOC:
% https://tex.stackexchange.com/a/64124
\usepackage{tocloft}% http://ctan.org/pkg/tocloft
\makeatletter
\renewcommand{\numberline}[1]{%
  \@cftbsnum #1\@cftasnum~\@cftasnumb%
}
\makeatother
%--------------------

\newcommand{\footnoteref}[1]{\textsuperscript{\ref{#1}}}

%\definecolor{lstbgcolor}{rgb}{0.94,0.94,0.94}

% I don't know why this voodoo works, but without all-caps, it can't find LIGHT-GRAY color. WTF?
% see also: https://tex.stackexchange.com/questions/64298/error-with-xcolor-package
\definecolor{light-gray}{gray}{0.87}
\definecolor{LIGHT-GRAY}{gray}{0.87}
\definecolor{RED}{rgb}{1,0,0}
\makeindex

\input{macros}
\EN{\input{macro_lang_EN}}
\RU{\input{macro_lang_RU}}
\FR{\input{macro_lang_FR}}
\DE{\input{macro_lang_DE}}
\JA{\input{macro_lang_JP}}
\IT{\input{macro_lang_IT}}
\PL{\input{macro_lang_PL}}
\ES{\input{macro_lang_ES}}
\NL{\input{macro_lang_NL}}
\TR{\input{macro_lang_TR}}
\PTBR{\input{macro_lang_PTBR}}

\ifdefined\UAL
\newcommand{\TitleMain}{\TitleUAL}
\newcommand{\TitleAux}{\TitleRE}
\else
\newcommand{\TitleMain}{\TitleRE}
\newcommand{\TitleAux}{\TitleUAL}
\fi

\EN{\input{glossary_EN}}
\RU{\input{glossary_RU}}
\FR{\input{glossary_FR}}
\DE{\input{glossary_DE}}
\IT{\input{glossary_IT}}
\JA{\input{glossary_JA}}

\makeglossaries

\hypersetup{
    colorlinks=true,
    allcolors=blue,
    pdfauthor={\AUTHOR},
    pdftitle={\TitleMain}
    }

%\ifdefined\RUSSIAN
\newcommand{\LstStyle}{\ttfamily\small}
%\else
%\newcommand{\LstStyle}{\ttfamily}
%\fi

% inspired by http://prismjs.com/
\definecolor{digits}{RGB}{0,0,0}
\definecolor{bg}{RGB}{255,255,255}
%\definecolor{bg}{RGB}{255,252,250}
\definecolor{col1}{RGB}{154,20,150}
\definecolor{col2}{RGB}{112,128,144}
\definecolor{col3}{RGB}{10,120,180}
\definecolor{col4}{RGB}{106,164,108}


\lstset{
    %backgroundcolor=\color{lstbgcolor},
    %backgroundcolor=\color{light-gray},
    backgroundcolor=\color{bg},
    basicstyle=\LstStyle,
    breaklines=true,
    %prebreak=\raisebox{0ex}[0ex][0ex]{->},
    %postbreak=\raisebox{0ex}[0ex][0ex]{->},
    prebreak=\raisebox{0ex}[0ex][0ex]{\ensuremath{\rhookswarrow}},
    postbreak=\raisebox{0ex}[0ex][0ex]{\ensuremath{\rcurvearrowse\space}},
    frame=single,
    columns=fullflexible,keepspaces,
    escapeinside=§§,
    inputencoding=utf8
}

\input{syntax_color}

\ifdefined\RUSSIAN
\renewcommand\lstlistingname{Листинг}
\renewcommand\lstlistlistingname{Листинг}
\fi

\DeclareMathSizes{12}{30}{16}{12}%

% see also:
% http://tex.stackexchange.com/questions/129225/how-can-i-get-get-makeindex-to-ignore-capital-letters
% http://tex.stackexchange.com/questions/18336/correct-sorting-of-index-entries-containing-macros
\def\myindex#1{\expandafter\index\expandafter{#1}}

\begin{document}

% fancyhdr
\pagestyle{fancy}
\setlength{\headheight}{13pt}
\fancyhead[R]{} % suppress chapter name

\VerbatimFootnotes

\frontmatter

%\RU{\include{1st_page_RU}}
%\EN{\include{1st_page_EN}}
%\DE{\include{1st_page_DE}}
%\ES{\include{1st_page_ES}}
%\CN{\include{1st_page_CN}}

\RU{\include{dedication_RU}}
\EN{\include{dedication_EN}}
\FR{\include{dedication_FR}}
\JA{\include{dedication_JA}}
\DE{\include{dedication_DE}}
\IT{\include{dedication_IT}}

\include{page_after_cover}
\include{call_for_translators}

\shorttoc{%
    \RU{Краткое оглавление}%
    \EN{Abridged contents}%
    \ES{Contenidos abreviados}%
    \PTBRph{}%
    \DE{Inhaltsverzeichnis (gekürzt)}%
    \PLph{}%
    \IT{Sommario}%
    \THAph{}\NLph{}%
    \FR{Contenus abrégés}%
    \JA{簡略版}
    \TR{İçindekiler}
}{0}

\tableofcontents
\cleardoublepage

\cleardoublepage
\include{preface}

\mainmatter

\include{parts}

\EN{\include{appendix/appendix}}
\RU{\include{appendix/appendix}}
\DE{\include{appendix/appendix}}
\FR{\include{appendix/appendix}}
\IT{\include{appendix/appendix}}
\include{acronyms}

\bookmarksetup{startatroot}

\clearpage
\phantomsection
\addcontentsline{toc}{chapter}{%
    \RU{Глоссарий}%
    \EN{Glossary}%
    \ES{Glosario}%
    \PTBRph{}%
    \DE{Glossar}%
    \PLph{}%
    \IT{Glossario}%
    \THAph{}\NLph{}%
    \FR{Glossaire}%
    \JA{用語}
    \TR{Bolum}
}
\printglossaries

\clearpage
\phantomsection
\printindex

\end{document}

\subsection{ARM}

\EN{\input{patterns/03_printf/ARM/ARM3_EN}}
\RU{\input{patterns/03_printf/ARM/ARM3_RU}}
\FR{\input{patterns/03_printf/ARM/ARM3_FR}}
\IT{\input{patterns/03_printf/ARM/ARM3_IT}}
\JA{\input{patterns/03_printf/ARM/ARM3_JA}}

\EN{\input{patterns/03_printf/ARM/ARM8_EN}}
\RU{\input{patterns/03_printf/ARM/ARM8_RU}}
\FR{\input{patterns/03_printf/ARM/ARM8_FR}}
\IT{\input{patterns/03_printf/ARM/ARM8_IT}}
\JA{\input{patterns/03_printf/ARM/ARM8_JA}}

\EN{\input{patterns/03_printf/MIPS/main_EN}}
\RU{\input{patterns/03_printf/MIPS/main_RU}}
\IT{\input{patterns/03_printf/MIPS/main_IT}}
\FR{\input{patterns/03_printf/MIPS/main_FR}}
\JA{\input{patterns/03_printf/MIPS/main_JA}}


\subsection{\Conclusion{}}

Aqui está uma estrutura bem rústica da chamada da função

% TODO to be translated to PTBR:
\begin{lstlisting}[caption=x86,style=customasmx86]
...
PUSH 3rd argument
PUSH 2nd argument
PUSH 1st argument
CALL function
; modify stack pointer (if needed)
\end{lstlisting}

\begin{lstlisting}[caption=x64 (MSVC),style=customasmx86]
MOV RCX, 1st argument
MOV RDX, 2nd argument
MOV R8, 3rd argument
MOV R9, 4th argument
...
PUSH 5th, 6th argument, etc. (if needed)
CALL function
; modify stack pointer (if needed)
\end{lstlisting}

\begin{lstlisting}[caption=x64 (GCC),style=customasmx86]
MOV RDI, 1st argument
MOV RSI, 2nd argument
MOV RDX, 3rd argument
MOV RCX, 4th argument
MOV R8, 5th argument
MOV R9, 6th argument
...
PUSH 7th, 8th argument, etc. (if needed)
CALL function
; modify stack pointer (if needed)
\end{lstlisting}

\begin{lstlisting}[caption=ARM,style=customasmARM]
MOV R0, 1st argument
MOV R1, 2nd argument
MOV R2, 3rd argument
MOV R3, 4th argument
; pass 5th, 6th argument, etc., in stack (if needed)
BL function
; modify stack pointer (if needed)
\end{lstlisting}

\begin{lstlisting}[caption=ARM64,style=customasmARM]
MOV X0, 1st argument
MOV X1, 2nd argument
MOV X2, 3rd argument
MOV X3, 4th argument
MOV X4, 5th argument
MOV X5, 6th argument
MOV X6, 7th argument
MOV X7, 8th argument
; pass 9th, 10th argument, etc., in stack (if needed)
BL function
; modify stack pointer (if needed)
\end{lstlisting}

\myindex{MIPS!O32}
\begin{lstlisting}[caption=MIPS (O32 calling convention),style=customasmMIPS]
LI $4, 1st argument ; AKA \$A0
LI $5, 2nd argument ; AKA \$A1
LI $6, 3rd argument ; AKA \$A2
LI $7, 4th argument ; AKA \$A3
; pass 5th, 6th argument, etc., in stack (if needed)
LW temp_reg, address of function
JALR temp_reg
\end{lstlisting}

\subsection{A propósito}

\myindex{fastcall}
A propósito, a diferença entre os argumentos passados em x86, x64, fastcall, ARM e MIPS  é uma boa demonstração do fato de como a CPU é indiferente sobre como os argumentos são passados para as funções.
Também é possível criar um compilador hipotético capaz de passar argumentos por alguma outra estrutura especial sem usar a pilha de nenhuma maneira.

\myindex{MIPS!O32}
\PTBRph{}

A \ac{CPU} não está ciente de convenções de chamada de funções.

Agora nós podemos também relembrar de dos programadores novatos de assembly passando argumentos para outras funções:
geralmente via registradores, sem nenhuma sequência explícita, ou mesmo por variáveis globais. Logicamente, também funciona.

}
\IT{\subsubsection{x86: la funzione alloca() }
\label{alloca}
\myindex{\CStandardLibrary!alloca()}

\newcommand{\AllocaSrcPath}{C:\textbackslash{}Program Files (x86)\textbackslash{}Microsoft Visual Studio 10.0\textbackslash{}VC\textbackslash{}crt\textbackslash{}src\textbackslash{}intel}

Vale la pena esaminare la funzione \TT{alloca()}
\footnote{In MSVC, l'implementazione della funzione si trova in \TT{alloca16.asm} e \TT{chkstk.asm} in \\
\TT{\AllocaSrcPath{}}}.
Questa funzione opera come \TT{malloc()}, ma alloca memoria direttamente nello stack.
% page break added to prevent "\vref on page boundary" error. it may be dropped in future.
Il pezzo di memoria allocato non necessita di essere liberato tramite una chiamata alla funzione \TT{free()} function call, \\
poiche' l'epilogo della funzione (\myref{sec:prologepilog}) ripristina \ESP al suo valore iniziale e la memoria allocata viene semplicemente \emph{abbandonata}.
Vale anche la pena notare come e' implementata la funzione \TT{alloca()}.
In termini semplici, questa funzione shifta \ESP in basso, verso la base dello stack, per il numero di byte necessari e setta \ESP  
per puntare al blocco \emph{allocato}.

Proviamo:

\lstinputlisting[style=customc]{patterns/02_stack/04_alloca/2_1.c}

La funzione \TT{\_snprintf()} opera come \printf, ma invece di inviare il risultato a \gls{stdout} (es. al terminale o console),
lo scrive nel buffer \TT{buf}. La funzione \puts copia il contenuto di \TT{buf} in \gls{stdout}.
Ovviamente questo due chiamate potrebbero essere rimpiazzate da una sola chiamata a \printf, ma in questo caso era necessario per illustrare
l'uso di un piccolo buffer.

\myparagraph{MSVC}

Compiliamo (MSVC 2010):

\lstinputlisting[caption=MSVC 2010,style=customasmx86]{patterns/02_stack/04_alloca/2_2_msvc.asm}

\myindex{Compiler intrinsic}
L'unico argomento di \TT{alloca()} e' passato tramite il registro \EAX (anziche' metterlo nello stack)
\footnote{Questo perche' alloca() e' una "compiler intrinsic" (\myref{sec:compiler_intrinsic}) piuttosto che una funzione normale.
Una delle ragioni per cui abbiamo bisogno di una funzione separata, invece di un paio di istruzioni nel codice, e' che
l'implementazione di alloca() di \ac{MSVC} ha anche del codice che legge dalla memoria appena llocata, per far si che l'\ac{OS} effettui il mapping
della memoria fisica in questa regione della \ac{VM}.
Dopo la chiamata a \TT{alloca()} , \ESP punta al blocco di 600 byte, ed e' possibile utilizzarlo come memoria per l'array \TT{buf}.}.

\myparagraph{GCC + \IntelSyntax}

GCC 4.4.1 fa lo stesso senza chiamare funzioni esterne:

\lstinputlisting[caption=GCC 4.7.3,style=customasmx86]{patterns/02_stack/04_alloca/2_1_gcc_intel_O3_EN.asm}

\myparagraph{GCC + \ATTSyntax}

Esaminiamo lo stesso codice, ma in sintassi AT\&T:

\lstinputlisting[caption=GCC 4.7.3,style=customasmx86]{patterns/02_stack/04_alloca/2_1_gcc_ATT_O3.s}

\myindex{\ATTSyntax}
The code e' uguale a quello del listato precedente.

A proposito, \INS{movl \$3, 20(\%esp)} corrisponde a \INS{mov DWORD PTR [esp+20], 3} in sintassi Intel.
In sintassi AT\&T, il formato registro+offset per indirizzare memoria appare come \TT{offset(\%{register})}.
}
\DE{\subsubsection{x86: alloca() Funktion}
\label{alloca}
\myindex{\CStandardLibrary!alloca()}

\newcommand{\AllocaSrcPath}{C:\textbackslash{}Program Files (x86)\textbackslash{}Microsoft Visual Studio 10.0\textbackslash{}VC\textbackslash{}crt\textbackslash{}src\textbackslash{}intel}

Es macht Sinn einen Blick auf die \TT{alloca()} Funktion zu werfen
\footnote{In MSVC, kann die Funktions Implementierung in \TT{alloca16.asm} und \TT{chkstk.asm} in \\
\TT{\AllocaSrcPath{}}} gefunden werden.
Diese Funktion arbeitet wie \TT{malloc()}, nur das sie Speicher direkt auf dem Stack bereit stellt.

Der allozierte Speicher Chunk muss nicht wieder mit \TT{free()} freigegeben werden, weil
der Funktions Epilog (\myref{sec:prologepilog}) das \ESP Register wieder in seinen ursprünglichen 
Zustand versetzt und der allozierte Speicher wird einfach \IT{verworfen}. 
Es macht Sinn sich anzuschauen wie \TT{alloca()} implementiert ist.
Mit einfachen Begriffen erklärt, diese Funktion verschiebt \ESP in Richtung des Stack ende mit der 
Anzahl der Bytes die alloziert werden müssen und setzt \ESP als einen Zeiger auf den \IT{allozierten} block.

Beispiel:

\lstinputlisting[style=customc]{patterns/02_stack/04_alloca/2_1.c}


Die \TT{\_snprintf()} Funktion arbeitetet genau wie \printf, nur statt die Ergebnisse nach \gls{stdout} aus zu geben ( bsp. auf dem Terminal oder Konsole), schreibt sie in den \TT{buf} buffer. Die Funktion \puts kopiert den Inhalt aus \TT{buf} nach \gls{stdout}. Sicher könnte man die beiden Funktions Aufrufe könnten durch einen \printf Aufruf ersetzt werden, aber wir sollten einen genaueren Blick auf die Benutzung kleiner Buffer anschauen.

\myparagraph{MSVC}

Compilierung mit MSVC 2010: 

\lstinputlisting[caption=MSVC 2010,style=customasmx86]{patterns/02_stack/04_alloca/2_2_msvc.asm}

\myindex{Compiler intrinsisch}
Das einzige \\TT{alloca()} Argument wird über \EAX übergeben (anstatt es erst auf den Stack zu pushen)
\footnote{Das liegt daran, das alloca() Verhalten Compiler intrinsisch bestimmt (\myref{sec:compiler_intrinsic}) im Gegensatz zu einer normalen Funktion. Einer der Gründe dafür das man braucht eine separate Funktion braucht, statt ein paar Code Instruktionen im Code,  ist weil die \ac{MSCV} alloca() Implementierung ebenfalls Code hat welcher aus dem gerade allozierten Speicher gelesen wird. Damit in Folge das \ac{Betriebssystem} physikalischen Speicher in dieser \ac{VM} Region zu allozieren. Nach dem \TT{alloca()} Aufruf, zeigt \ESP auf den Block von 600 Bytes der nun als Speicher für das \TT{buf} Array dienen kann.}.

\myparagraph{GCC + \IntelSyntax}

GCC 4.4.1 macht das selbe, aber ohne externe Funktions aufrufe.

\lstinputlisting[caption=GCC 4.7.3,style=customasmx86]{patterns/02_stack/04_alloca/2_1_gcc_intel_O3_EN.asm}

\myparagraph{GCC + \ATTSyntax}

Nun der gleiche Code, aber in AT\&T Syntax:

\lstinputlisting[caption=GCC 4.7.3,style=customasmx86]{patterns/02_stack/04_alloca/2_1_gcc_ATT_O3.s}

\myindex{\ATTSyntax}
Der Code ist der gleiche wie im vorherigen listig.

Übrigens, \INS{movl \$3, 20(\%esp)} in AT\&T Syntax wird zu \
\INS{mov DWORD PTR [esp+20], 3} in Intel-syntax.
In der AT\&T Syntax, sehen Register+Offset Formatierungen einer Adresse so aus:
\TT{offset(\%{register})}.
}
\PL{\subsubsection{x86: Funkcja alloca()}
\label{alloca}
\myindex{\CStandardLibrary!alloca()}

\newcommand{\AllocaSrcPath}{C:\textbackslash{}Program Files (x86)\textbackslash{}Microsoft Visual Studio 10.0\textbackslash{}VC\textbackslash{}crt\textbackslash{}src\textbackslash{}intel}

Przypadek z funkcją \TT{alloca()} jest całkiem ciekawy
\footnote{W MSVC, implementację funkcji można podejrzeć w plikach \TT{alloca16.asm} i \TT{chkstk.asm} w \\
\TT{\AllocaSrcPath{}}}. 
Ta funkcja działa jak \TT{malloc()}, ale przydziela pamięć od razu na stosie.
Nie potrzebne jest zwalnianie pamięci \TT{free()}, dlatego że epilog funkcji~(\myref{sec:prologepilog})
przywróci \ESP do stanu początkowego i przeznaczona pamięć \emph{zostaje wyrzucona}.
Ciekawa jest również realizacja tej funkcji.
Ona, w skrócie, po prostu przesuwa \ESP wgłąb stosu i zwraca \ESP jako wskaźnik na przydzielony obszar.

Spróbujmy:

\lstinputlisting[style=customc]{patterns/02_stack/04_alloca/2_1.c}

Funkcja \TT{\_snprintf()} działa tak samo, jak i \printf, tylko zamiast wyprowadzenia wyniku na wyjście standardowe \gls{stdout} (czyli do terminalu),
ona go zapisuje do buforu \TT{buf}. Funkcja \puts, z kolei, wyrzuca zawartość buforu \TT{buf} na \gls{stdout}. Oczywiście można by było tutaj zamienić
tę parę instrukcji na \printf, ale tutaj chcielibyśmy zobaczyć wykorzystanie niewielkiego buforu.

\myparagraph{MSVC}

Skompilujmy (MSVC 2010):

\lstinputlisting[caption=MSVC 2010,style=customasmx86]{patterns/02_stack/04_alloca/2_2_msvc.asm}

\myindex{Compiler intrinsic}
Jedyny parametr \TT{alloca()} jest przekazywany przez \EAX, a nie jak zwykle, przez stos
\footnote{To dlatego, że alloca()~--- to nie tyle co funkcja, a raczej \emph{compiler intrinsic} (\myref{sec:compiler_intrinsic})
Jedną z przyczyn, flaczego tu potrzeba funkcji, a nie kilku instrukcji w samym kodzie, polega na tym, że w realizacji
funkcji alloca() w \ac{MSVC}
zawarty również kod, czytający z dopiero co przydzielonej pamięci po to, żeby \ac{OS} zaalokowała pamięć fizyczną dla tego obszaru \ac{VM}.
Po wywołaniu \TT{alloca()} \ESP wskazuje na blok o długości 600 bajtów, z którego możemy korzystać na potrzeby naszego \TT{buf}.}.

\myparagraph{GCC + \IntelSyntax}

GCC 4.4.1 nie wywołuje innych funkcji:

\lstinputlisting[caption=GCC 4.7.3,style=customasmx86]{patterns/02_stack/04_alloca/2_1_gcc_intel_O3_RU.asm}

\myparagraph{GCC + \ATTSyntax}

Spójrzmy na ten sam kod w syntaksie AT\&T:

\lstinputlisting[caption=GCC 4.7.3,style=customasmx86]{patterns/02_stack/04_alloca/2_1_gcc_ATT_O3.s}

\myindex{\ATTSyntax}
Wygląda to tak samo jak i poprzedni listing.

A propos, \INS{movl \$3, 20(\%esp)}~--- jest analogiem do \INS{mov DWORD PTR [esp+20], 3} w syntaksie Intel.
Adresowanie pamięci typu \emph{rejestr+przesunięcie} jest zapisywane w syntaksie AT\&T jako \TT{przesunięcie(\%{rejestr})}.


}
\JPN{\subsubsection{x86: alloca()関数}
\label{alloca}
\myindex{\CStandardLibrary!alloca()}

\newcommand{\AllocaSrcPath}{C:\textbackslash{}Program Files (x86)\textbackslash{}Microsoft Visual Studio 10.0\textbackslash{}VC\textbackslash{}crt\textbackslash{}src\textbackslash{}intel}

\TT{alloca()}関数に注目することは重要です
\footnote{MSVCでは、関数の実装は\TT{\AllocaSrcPath{}}の\TT{alloca16.asm} と \TT{chkstk.asm}にあります}
この関数は\TT{malloc()}のように動作しますが、スタックに直接メモリを割り当てます。 
% page break added to prevent "\vref on page boundary" error. it may be dropped in future.
関数のエピローグ(\myref{sec:prologepilog})は \ESP を初期状態に戻し、割り当てられたメモリは単に\emph{破棄}されるため、
割り当てられたメモリチャンクは\TT{free()}関数呼び出しで解放する必要はありません。 \TT{alloca()}がどのように実装されているかは注目に値する。
簡単に言えば、この関数は必要なバイト数だけスタック底部に向かって \ESP を下にシフトさせ、\emph{割り当てられた}ブロックへのポインタとして \ESP を設定します。

やってみましょう。

\lstinputlisting[style=customc]{patterns/02_stack/04_alloca/2_1.c}

\TT{\_snprintf()}関数は \printf と同じように動作しますが、結果を\gls{stdout}(ターミナルやコンソールなど)
にダンプする代わりに、\TT{buf}バッファに書き込みます。 \puts 関数は\TT{buf}の内容を\gls{stdout}にコピーします。
もちろん、これらの2つの関数呼び出しは1つの \printf 呼び出しで置き換えることができますが、小さなバッファの使用法を説明する必要があります。

\myparagraph{MSVC}

コンパイルしてみましょう(MSVC 2010で)

\lstinputlisting[caption=MSVC 2010,style=customasmx86]{patterns/02_stack/04_alloca/2_2_msvc.asm}

\myindex{Compiler intrinsic}
\TT{alloca()}の唯一の引数は \EAX 経由で(スタックにプッシュするのではなく)渡されます。
\footnote{alloca()はコンパイラ組み込み関数((\myref{sec:compiler_intrinsic}))ではなく、通常の関数です。 
\ac{MSVC}のalloca()の実装には、割り当てられたメモリから読み込むコードが含まれているため、\ac{OS}が物理メモリをVM領域にマップするために、
コード内の命令が数個ではなく別々の関数を必要とする理由の1つです。 \TT{alloca()}呼び出しの後、ESPは600バイトのブロックを指し、\TT{buf}配列のメモリとして使用できます。}

\myparagraph{GCC + \IntelSyntax}

GCC 4.4.1は、外部関数を呼び出すことなく同じことを行います

\lstinputlisting[caption=GCC 4.7.3,style=customasmx86]{patterns/02_stack/04_alloca/2_1_gcc_intel_O3_EN.asm}

\myparagraph{GCC + \ATTSyntax}

同じコードをAT\&T構文で見てみましょう

\lstinputlisting[caption=GCC 4.7.3,style=customasmx86]{patterns/02_stack/04_alloca/2_1_gcc_ATT_O3.s}

\myindex{\ATTSyntax}
コードは前のリストと同じです。

ちなみに、\INS{movl \$3, 20(\%esp)}は、
Intel構文の\INS{mov DWORD PTR [esp+20], 3}に対応しています。 
AT\&Tの構文では、アドレス指定メモリのレジスタ+オフセット形式は
\TT{offset(\%{register})}のように見えます。
}

\input{patterns/02_stack/05_SEH}
\ifdefined\ENGLISH
\subsubsection{Buffer overflow protection}

More about it here~(\myref{subsec:bufferoverflow}).
\fi

\ifdefined\RUSSIAN
\subsubsection{Защита от переполнений буфера}

Здесь больше об этом~(\myref{subsec:bufferoverflow}).
\fi

\ifdefined\BRAZILIAN
\subsubsection{Proteção contra estouro de buffer}

Mais sobre aqui~(\myref{subsec:bufferoverflow}).
\fi

\ifdefined\ITALIAN
\subsubsection{Protezione contro buffer overflow}

Maggiori informazioni qui~(\myref{subsec:bufferoverflow}).
\fi

\ifdefined\FRENCH
\subsubsection{Protection contre les débordements de tampon}

Lire à ce propos~(\myref{subsec:bufferoverflow}).
\fi


\ifdefined\POLISH
\subsubsection{Metody zabiezpieczenia przed przepełnieniem stosu}

Więcej o tym tutaj~(\myref{subsec:bufferoverflow}).
\fi

\ifdefined\JAPANESE
\subsubsection{バッファオーバーフロー保護}

詳細はこちら~(\myref{subsec:bufferoverflow})
\fi


\subsubsection{Автоматическое освобождение данных в стеке}

Возможно, причина хранения локальных переменных и SEH-записей в стеке в том, что после выхода из функции, всё эти данные освобождаются автоматически,
используя только одну инструкцию корректирования указателя стека (часто это \ADD).
Аргументы функций, можно сказать, тоже освобождаются автоматически в конце функции.
А всё что хранится в куче (\emph{heap}) нужно освобождать явно.

% sections
\EN{\input{patterns/02_stack/07_layout_EN}}
\RU{\input{patterns/02_stack/07_layout_RU}}
\PTBR{\input{patterns/02_stack/07_layout_PTBR}}
\EN{\mysection{\MinesweeperWinXPExampleChapterName}
\label{minesweeper_winxp}
\myindex{Windows!Windows XP}

For those who are not very good at playing Minesweeper, we could try to reveal the hidden mines in the debugger.

\myindex{\CStandardLibrary!rand()}
\myindex{Windows!PDB}

As we know, Minesweeper places mines randomly, so there has to be some kind of random number generator or
a call to the standard \TT{rand()} C-function.

What is really cool about reversing Microsoft products is that there are \gls{PDB} 
file with symbols (function names, \etc{}).
When we load \TT{winmine.exe} into \IDA, it downloads the 
\gls{PDB} file exactly for this 
executable and shows all names.

So here it is, the only call to \TT{rand()} is this function:

\lstinputlisting[style=customasmx86]{examples/minesweeper/tmp1.lst}

\IDA named it so, and it was the name given to it by Minesweeper's developers.

The function is very simple:

\begin{lstlisting}[style=customc]
int Rnd(int limit)
{
    return rand() % limit;
};
\end{lstlisting}

(There is no \q{limit} name in the \gls{PDB} file; we manually named this argument like this.)

So it returns 
a random value from 0 to a specified limit.

\TT{Rnd()} is called only from one place, 
a function called \TT{StartGame()}, 
and as it seems, this is exactly 
the code which place the mines:

\begin{lstlisting}[style=customasmx86]
.text:010036C7                 push    _xBoxMac
.text:010036CD                 call    _Rnd@4          ; Rnd(x)
.text:010036D2                 push    _yBoxMac
.text:010036D8                 mov     esi, eax
.text:010036DA                 inc     esi
.text:010036DB                 call    _Rnd@4          ; Rnd(x)
.text:010036E0                 inc     eax
.text:010036E1                 mov     ecx, eax
.text:010036E3                 shl     ecx, 5          ; ECX=ECX*32
.text:010036E6                 test    _rgBlk[ecx+esi], 80h
.text:010036EE                 jnz     short loc_10036C7
.text:010036F0                 shl     eax, 5          ; EAX=EAX*32
.text:010036F3                 lea     eax, _rgBlk[eax+esi]
.text:010036FA                 or      byte ptr [eax], 80h
.text:010036FD                 dec     _cBombStart
.text:01003703                 jnz     short loc_10036C7
\end{lstlisting}

Minesweeper allows you to set the board size, so the X (xBoxMac) and Y (yBoxMac) of the board are global variables.
They are passed to \TT{Rnd()} and random 
coordinates are generated.
A mine is placed by the \TT{OR} instruction at \TT{0x010036FA}. 
And if it has been placed before 
(it's possible if the pair of \TT{Rnd()} 
generates a coordinates pair which has been already 
generated), 
then \TT{TEST} and \TT{JNZ} at \TT{0x010036E6} 
jumps to the generation routine again.

\TT{cBombStart} is the global variable containing total number of mines. So this is loop.

The width of the array is 32 
(we can conclude this by looking at the \TT{SHL} instruction, which multiplies one of the coordinates by 32).

The size of the \TT{rgBlk} 
global array can be easily determined by the difference 
between the \TT{rgBlk} 
label in the data segment and the next known one. 
It is 0x360 (864):

\begin{lstlisting}[style=customasmx86]
.data:01005340 _rgBlk          db 360h dup(?)          ; DATA XREF: MainWndProc(x,x,x,x)+574
.data:01005340                                         ; DisplayBlk(x,x)+23
.data:010056A0 _Preferences    dd ?                    ; DATA XREF: FixMenus()+2
...
\end{lstlisting}

$864/32=27$.

So the array size is $27*32$?
It is close to what we know: when we try to set board size to $100*100$ in Minesweeper settings, it fallbacks to a board of size $24*30$.
So this is the maximal board size here.
And the array has a fixed size for any board size.

So let's see all this in \olly.
We will ran Minesweeper, attaching \olly to it and now we can see the memory dump at the address of the \TT{rgBlk} array (\TT{0x01005340})
\footnote{All addresses here are for Minesweeper for Windows XP SP3 English. 
They may differ for other service packs.}.

So we got this memory dump of the array:

\lstinputlisting[style=customasmx86]{examples/minesweeper/1.lst}

\olly, like any other hexadecimal editor, shows 16 bytes per line.
So each 32-byte array row occupies exactly 2 lines here.

This is beginner level (9*9 board).

There is some square 
structure can be seen visually (0x10 bytes).

We will click \q{Run} in \olly to unfreeze the Minesweeper process, then we'll clicked randomly at the Minesweeper window 
and trapped into mine, but now all mines are visible:

\begin{figure}[H]
\centering
\myincludegraphicsSmall{examples/minesweeper/1.png}
\caption{Mines}
\label{fig:minesweeper1}
\end{figure}

By comparing the mine places and the dump, we can conclude that 0x10 stands for border, 0x0F---empty block, 0x8F---mine.
Perhaps, 0x10 is just a \emph{sentinel value}.

Now we'll add comments and also enclose all 0x8F bytes into square brackets:

\lstinputlisting[style=customasmx86]{examples/minesweeper/2.lst}

Now we'll remove all \emph{border bytes} (0x10) and what's beyond those:

\lstinputlisting[style=customasmx86]{examples/minesweeper/3.lst}

Yes, these are mines, now it can be clearly seen and compared with the screenshot.

\clearpage
What is interesting is that we can modify the array right in \olly.
We can remove all mines by changing all 0x8F bytes by 0x0F, and here is what we'll get in Minesweeper:

\begin{figure}[H]
\centering
\myincludegraphicsSmall{examples/minesweeper/3.png}
\caption{All mines are removed in debugger}
\label{fig:minesweeper3}
\end{figure}

We can also move all of them to the first line: 

\begin{figure}[H]
\centering
\myincludegraphicsSmall{examples/minesweeper/2.png}
\caption{Mines set in debugger}
\label{fig:minesweeper2}
\end{figure}

Well, the debugger is not very convenient for eavesdropping (which is our goal anyway), so we'll write a small utility
to dump the contents of the board:

\lstinputlisting[style=customc]{examples/minesweeper/minesweeper_cheater.c}

Just set the \ac{PID}
\footnote{PID it can be seen in Task Manager 
(enable it in \q{View $\rightarrow$ Select Columns})} 
and the address of the array (\TT{0x01005340} for Windows XP SP3 English) 
and it will dump it
\footnote{The compiled executable is here: 
\href{http://go.yurichev.com/17165}{beginners.re}}.

It attaches itself to a win32 process by \ac{PID} and just reads process memory at the address.

\subsection{Finding grid automatically}

This is kind of nuisance to set address each time when we run our utility.
Also, various Minesweeper versions may have the array on different address.
Knowing the fact that there is always a border (0x10 bytes), we can just find it in memory:

\lstinputlisting[style=customc]{examples/minesweeper/cheater2_fragment.c}

Full source code: \url{https://github.com/DennisYurichev/RE-for-beginners/blob/master/examples/minesweeper/minesweeper_cheater2.c}.

\subsection{\Exercises}

\begin{itemize}

\item 
Why do the \emph{border bytes} (or \emph{sentinel values}) (0x10) exist in the array?

What they are for if they are not visible in Minesweeper's interface?
How could it work without them?

\item 
As it turns out, there are more values possible (for open blocks, for flagged by user, \etc{}).
Try to find the meaning of each one.

\item 
Modify my utility so it can remove all mines or set them in a fixed pattern that you want in the Minesweeper
process currently running.

\end{itemize}
}
\FR{\subsection{MIPS}

\subsubsection{3 arguments}

\myparagraph{GCC 4.4.5 \Optimizing}

La différence principale avec l'exemple \q{\HelloWorldSectionName} est que dans ce cas, \printf est appelée
à la place de \puts et 3 arguments de plus sont passés à travers les registres \$5\dots \$7 (ou \$A0\dots \$A2).
C'est pourquoi ces registres sont préfixés avec A-, ceci sous-entend qu'ils
sont utilisés pour le passage des arguments aux fonctions.

\lstinputlisting[caption=GCC 4.4.5 \Optimizing (\assemblyOutput),style=customasmMIPS]{patterns/03_printf/MIPS/printf3.O3_FR.s}

\lstinputlisting[caption=GCC 4.4.5 \Optimizing (IDA),style=customasmMIPS]{patterns/03_printf/MIPS/printf3.O3.IDA_FR.lst}

\IDA a agrégé la paire d'instructions \INS{LUI} et \INS{ADDIU} en une pseudo instruction \INS{LA}.
C'est pourquoi il n'y a pas d'instruction à l'adresse 0x1C: car \INS{LA} \emph{occupe} 8 octets.

\myparagraph{GCC 4.4.5 \NonOptimizing}

GCC \NonOptimizing est plus verbeux:

\lstinputlisting[caption=GCC 4.4.5 \NonOptimizing (\assemblyOutput),style=customasmMIPS]{patterns/03_printf/MIPS/printf3.O0_FR.s}

\lstinputlisting[caption=GCC 4.4.5 \NonOptimizing (IDA),style=customasmMIPS]{patterns/03_printf/MIPS/printf3.O0.IDA_FR.lst}

\subsubsection{8 arguments}

Utilisons encore l'exemple de la section précédente avec 9 arguments: \myref{example_printf8_x64}.

\lstinputlisting[style=customc]{patterns/03_printf/2.c}

\myparagraph{GCC 4.4.5 \Optimizing}

Seul les 4 premiers arguments sont passés dans les registres \$A0 \dots \$A3,
les autres sont passés par la pile.
\myindex{MIPS!O32}

C'est la convention d'appel O32 (qui est la plus commune dans le monde MIPS).
D'autres conventions d'appel (comme N32) peuvent utiliser les registres à d'autres fins.

\myindex{MIPS!\Instructions!SW}

\INS{SW} est l'abbréviation de \q{Store Word} (depuis un registre vers la mémoire).
En MIPS, il manque une instructions pour stocker une valeur dans la mémoire, donc
une paire d'instruction doit être utilisée à la place (\INS{LI}/\INS{SW}).

\lstinputlisting[caption=GCC 4.4.5 \Optimizing (\assemblyOutput),style=customasmMIPS]{patterns/03_printf/MIPS/printf8.O3_FR.s}

\lstinputlisting[caption=GCC 4.4.5 \Optimizing (IDA),style=customasmMIPS]{patterns/03_printf/MIPS/printf8.O3.IDA_FR.lst}

\myparagraph{GCC 4.4.5 \NonOptimizing}

GCC \NonOptimizing est plus verbeux:

\lstinputlisting[caption=\NonOptimizing GCC 4.4.5 (\assemblyOutput),style=customasmMIPS]{patterns/03_printf/MIPS/printf8.O0_FR.s}

\lstinputlisting[caption=\NonOptimizing GCC 4.4.5 (IDA),style=customasmMIPS]{patterns/03_printf/MIPS/printf8.O0.IDA_FR.lst}

}
\RU{\mysection{Разгон майнера биткоинов Cointerra}
\index{Bitcoin}
\index{BeagleBone}

Был такой майнер биткоинов Cointerra, выглядящий так:

\begin{figure}[H]
\centering
\myincludegraphics{examples/bitcoin_miner/board.jpg}
\caption{Board}
\end{figure}

И была также (возможно утекшая) утилита\footnote{Можно скачать здесь: \url{https://github.com/DennisYurichev/RE-for-beginners/raw/master/examples/bitcoin_miner/files/cointool-overclock}}
которая могла выставлять тактовую частоту платы.
Она запускается на дополнительной плате BeagleBone на ARM с Linux (маленькая плата внизу фотографии).

И у автора (этих строк) однажды спросили, можно ли хакнуть эту утилиту и посмотреть, какие частоты можно выставлять, и какие нет.
И можно ли твикнуть её?

Утилиту нужно запускать так: \TT{./cointool-overclock 0 0 900}, где 900 это частота в МГц.
Если частота слишком большая, утилита выведет ошибку \q{Error with arguments} и закончит работу.

Вот фрагмент кода вокруг ссылки на текстовую строку \q{Error with arguments}:

\begin{lstlisting}[style=customasmARM]

...

.text:0000ABC4         STR      R3, [R11,#var_28]
.text:0000ABC8         MOV      R3, #optind
.text:0000ABD0         LDR      R3, [R3]
.text:0000ABD4         ADD      R3, R3, #1
.text:0000ABD8         MOV      R3, R3,LSL#2
.text:0000ABDC         LDR      R2, [R11,#argv]
.text:0000ABE0         ADD      R3, R2, R3
.text:0000ABE4         LDR      R3, [R3]
.text:0000ABE8         MOV      R0, R3  ; nptr
.text:0000ABEC         MOV      R1, #0  ; endptr
.text:0000ABF0         MOV      R2, #0  ; base
.text:0000ABF4         BL       strtoll
.text:0000ABF8         MOV      R2, R0
.text:0000ABFC         MOV      R3, R1
.text:0000AC00         MOV      R3, R2
.text:0000AC04         STR      R3, [R11,#var_2C]
.text:0000AC08         MOV      R3, #optind
.text:0000AC10         LDR      R3, [R3]
.text:0000AC14         ADD      R3, R3, #2
.text:0000AC18         MOV      R3, R3,LSL#2
.text:0000AC1C         LDR      R2, [R11,#argv]
.text:0000AC20         ADD      R3, R2, R3
.text:0000AC24         LDR      R3, [R3]
.text:0000AC28         MOV      R0, R3  ; nptr
.text:0000AC2C         MOV      R1, #0  ; endptr
.text:0000AC30         MOV      R2, #0  ; base
.text:0000AC34         BL       strtoll
.text:0000AC38         MOV      R2, R0
.text:0000AC3C         MOV      R3, R1
.text:0000AC40         MOV      R3, R2
.text:0000AC44         STR      R3, [R11,#third_argument]
.text:0000AC48         LDR      R3, [R11,#var_28]
.text:0000AC4C         CMP      R3, #0
.text:0000AC50         BLT      errors_with_arguments
.text:0000AC54         LDR      R3, [R11,#var_28]
.text:0000AC58         CMP      R3, #1
.text:0000AC5C         BGT      errors_with_arguments
.text:0000AC60         LDR      R3, [R11,#var_2C]
.text:0000AC64         CMP      R3, #0
.text:0000AC68         BLT      errors_with_arguments
.text:0000AC6C         LDR      R3, [R11,#var_2C]
.text:0000AC70         CMP      R3, #3
.text:0000AC74         BGT      errors_with_arguments
.text:0000AC78         LDR      R3, [R11,#third_argument]
.text:0000AC7C         CMP      R3, #0x31
.text:0000AC80         BLE      errors_with_arguments
.text:0000AC84         LDR      R2, [R11,#third_argument]
.text:0000AC88         MOV      R3, #950
.text:0000AC8C         CMP      R2, R3
.text:0000AC90         BGT      errors_with_arguments
.text:0000AC94         LDR      R2, [R11,#third_argument]
.text:0000AC98         MOV      R3, #0x51EB851F
.text:0000ACA0         SMULL    R1, R3, R3, R2
.text:0000ACA4         MOV      R1, R3,ASR#4
.text:0000ACA8         MOV      R3, R2,ASR#31
.text:0000ACAC         RSB      R3, R3, R1
.text:0000ACB0         MOV      R1, #50
.text:0000ACB4         MUL      R3, R1, R3
.text:0000ACB8         RSB      R3, R3, R2
.text:0000ACBC         CMP      R3, #0
.text:0000ACC0         BEQ      loc_ACEC
.text:0000ACC4
.text:0000ACC4 errors_with_arguments
.text:0000ACC4                                         
.text:0000ACC4         LDR      R3, [R11,#argv]
.text:0000ACC8         LDR      R3, [R3]
.text:0000ACCC         MOV      R0, R3  ; path
.text:0000ACD0         BL       __xpg_basename
.text:0000ACD4         MOV      R3, R0
.text:0000ACD8         MOV      R0, #aSErrorWithArgu ; format
.text:0000ACE0         MOV      R1, R3
.text:0000ACE4         BL       printf
.text:0000ACE8         B        loc_ADD4
.text:0000ACEC ; ------------------------------------------------------------
.text:0000ACEC
.text:0000ACEC loc_ACEC                 ; CODE XREF: main+66C
.text:0000ACEC         LDR      R2, [R11,#third_argument]
.text:0000ACF0         MOV      R3, #499
.text:0000ACF4         CMP      R2, R3
.text:0000ACF8         BGT      loc_AD08
.text:0000ACFC         MOV      R3, #0x64
.text:0000AD00         STR      R3, [R11,#unk_constant]
.text:0000AD04         B        jump_to_write_power
.text:0000AD08 ; ------------------------------------------------------------
.text:0000AD08
.text:0000AD08 loc_AD08                 ; CODE XREF: main+6A4
.text:0000AD08         LDR      R2, [R11,#third_argument]
.text:0000AD0C         MOV      R3, #799
.text:0000AD10         CMP      R2, R3
.text:0000AD14         BGT      loc_AD24
.text:0000AD18         MOV      R3, #0x5F
.text:0000AD1C         STR      R3, [R11,#unk_constant]
.text:0000AD20         B        jump_to_write_power
.text:0000AD24 ; ------------------------------------------------------------
.text:0000AD24
.text:0000AD24 loc_AD24                 ; CODE XREF: main+6C0
.text:0000AD24         LDR      R2, [R11,#third_argument]
.text:0000AD28         MOV      R3, #899
.text:0000AD2C         CMP      R2, R3
.text:0000AD30         BGT      loc_AD40
.text:0000AD34         MOV      R3, #0x5A
.text:0000AD38         STR      R3, [R11,#unk_constant]
.text:0000AD3C         B        jump_to_write_power
.text:0000AD40 ; ------------------------------------------------------------
.text:0000AD40
.text:0000AD40 loc_AD40                 ; CODE XREF: main+6DC
.text:0000AD40         LDR      R2, [R11,#third_argument]
.text:0000AD44         MOV      R3, #999
.text:0000AD48         CMP      R2, R3
.text:0000AD4C         BGT      loc_AD5C
.text:0000AD50         MOV      R3, #0x55
.text:0000AD54         STR      R3, [R11,#unk_constant]
.text:0000AD58         B        jump_to_write_power
.text:0000AD5C ; ------------------------------------------------------------
.text:0000AD5C
.text:0000AD5C loc_AD5C                 ; CODE XREF: main+6F8
.text:0000AD5C         LDR      R2, [R11,#third_argument]
.text:0000AD60         MOV      R3, #1099
.text:0000AD64         CMP      R2, R3
.text:0000AD68         BGT      jump_to_write_power
.text:0000AD6C         MOV      R3, #0x50
.text:0000AD70         STR      R3, [R11,#unk_constant]
.text:0000AD74
.text:0000AD74 jump_to_write_power                     ; CODE XREF: main+6B0
.text:0000AD74                                         ; main+6CC ...
.text:0000AD74         LDR      R3, [R11,#var_28]
.text:0000AD78         UXTB     R1, R3
.text:0000AD7C         LDR      R3, [R11,#var_2C]
.text:0000AD80         UXTB     R2, R3
.text:0000AD84         LDR      R3, [R11,#unk_constant]
.text:0000AD88         UXTB     R3, R3
.text:0000AD8C         LDR      R0, [R11,#third_argument]
.text:0000AD90         UXTH     R0, R0
.text:0000AD94         STR      R0, [SP,#0x44+var_44]
.text:0000AD98         LDR      R0, [R11,#var_24]
.text:0000AD9C         BL       write_power
.text:0000ADA0         LDR      R0, [R11,#var_24]
.text:0000ADA4         MOV      R1, #0x5A
.text:0000ADA8         BL       read_loop
.text:0000ADAC         B        loc_ADD4

...

.rodata:0000B378 aSErrorWithArgu DCB "%s: Error with arguments",0xA,0 ; DATA XREF: main+684

...

\end{lstlisting}

Имена ф-ций присутствовали в отладочной информации в оригинальном исполняемом файле,
такие как \TT{write\_power}, \TT{read\_loop}.
Но имена меткам внутри ф-ции дал я.

\myindex{UNIX!getopt}
\myindex{strtoll()}
Имя \TT{optind} звучит знакомо. Это библиотека \IT{getopt} из *NIX предназначенная для парсинга командной строки ---
и это то, что внутри и происходит.
Затем, третий аргумент (где передается значение частоты) конвертируется из строку в число используя вызов ф-ции \IT{strtoll()}.

Значение затем сравнивается с разными константами.
На 0xACEC есть проверка, меньше ли оно или равно 499, и если это так, то 0x64 будет передано в ф-цию
\TT{write\_power()} (которая посылает команду через USB используя \TT{send\_msg()}).
Если значение больше 499, происходит переход на 0xAD08.

На 0xAD08 есть проверка, меньше ли оно или равно 799. Если это так, то 0x5F передается в ф-цию \TT{write\_power()}.

Есть еще проверки: на 899 на 0xAD24, на 0x999 на 0xAD40, и наконец, на 1099 на 0xAD5C.
Если входная частота меньше или равна 1099, 0x50 (на 0xAD6C) будет передано в ф-цию \TT{write\_power()}.
И тут что-то вроде баги.
Если значение все еще больше 1099, само значение будет передано в ф-цию \TT{write\_power()}.
Но с другой стороны это не бага, потому что мы не можем попасть сюда: значение в начале проверяется с 950 на 0xAC88,
и если оно больше, выводится сообщение об ошибке и утилита заканчивает работу.

Вот таблица между частотами в МГц и значениями передаваемыми в ф-цию \TT{write\_power()}:

\begin{center}
\begin{longtable}{ | l | l | l | }
\hline
\HeaderColor МГц & \HeaderColor шестнадцатеричное представление & \HeaderColor десятичное \\
\hline
499MHz & 0x64 & 100 \\
\hline
799MHz & 0x5f & 95 \\
\hline
899MHz & 0x5a & 90 \\
\hline
999MHz & 0x55 & 85 \\
\hline
1099MHz & 0x50 & 80 \\
\hline
\end{longtable}
\end{center}

Как видно, значение передаваемое в плату постепенно уменьшается с ростом частоты.

Видно что значение в 950МГц это жесткий предел, по крайней мере в этой утилите. Можно ли её обмануть?

Вернемся к этому фрагменту кода:

\begin{lstlisting}[style=customasmARM]
.text:0000AC84      LDR     R2, [R11,#third_argument]
.text:0000AC88      MOV     R3, #950
.text:0000AC8C      CMP     R2, R3
.text:0000AC90      BGT     errors_with_arguments ; Я пропатчил здесь на 00 00 00 00
\end{lstlisting}

Нам нужно как-то запретить инструкцию перехода \INS{BGT} на 0xAC90. И это ARM в режиме ARM, потому что, как мы видим,
все адреса увеличиваются на 4, т.е., длина каждой инструкции это 4 байта.
Инструкция \TT{NOP} (нет операции) в режиме ARM это просто 4 нулевых байта: \TT{00 00 00 00}.
Так что, записывая 4 нуля по адресу 0xAC90 (или по физическому смещению в файле: 0x2C90) мы можем выключить
эту проверку.

Теперь можно выставлять частоты вплоть до 1050МГц. И даже больше, но из-за ошибки, если входное значение больше 1099,
значение в МГц, \IT{как есть}, будет передано в плату, что неправильно.

Дальше я не разбирался, но если бы продолжил, я бы уменьшал значение передаваемое в ф-цию \TT{write\_power()}.

Теперь страшный фрагмент кода, который я в начале пропустил:

\lstinputlisting[style=customasmARM]{examples/bitcoin_miner/tmp1.lst}

Здесь используется деление через умножение, и константа 0x51EB851F.
Я написал для себя простой программистский калькулятор\footnote{\url{https://github.com/DennisYurichev/progcalc}}.
И там есть возможность вычислять обратное число по модулю.

\begin{lstlisting}
modinv32(0x51EB851F)
Warning, result is not integer: 3.125000
(unsigned) dec: 3 hex: 0x3 bin: 11
\end{lstlisting}

Это значит что инструкция \INS{SMULL} на 0xACA0 просто делит 3-й аргумент на 3.125.
На самом деле, все что делает ф-ция \TT{modinv32()} в моем калькуляторе, это:

\[
\frac{1}{\frac{input}{2^{32}}} = \frac{2^{32}}{input}
\]

Потом там есть дополнительные сдвиги и теперь мы видим что 3-й аргумент просто делится на 50.
И затем умножается снова на 50.
Зачем?
Это простейшая проверка, можно ли делить входное значение на 50 без остатка.
Если значение этого выражения ненулевое, $x$ не может быть разделено на 50 без остатка:

\[
x-((\frac{x}{50}) \cdot 50)
\]

На самом деле, это простой способ вычисления остатка от деления.

И затем, если остаток ненулевой, выводится сообщение об ошибке.
Так что эта утилита берет значения частотв вроде 850, 900, 950, 1000, итд, но не 855 или 911.

Вот и всё! Если вы делаете что-то такое, имейте ввиду, что это может испортить вашу плату, как и в случае разгона
чипов вроде \ac{CPU}, \ac{GPU}, итд.
Если у вас есть плата Cointerra, делайте всё это на свой собственный риск!

}
\IT{\subsection{Rumore nello stack}
\label{noise_in_stack}

\epigraph{When one says that something seems random, what one usually
means in practice is that one cannot see any regularities in it.}{Stephen Wolfram, A New Kind of Science.}

In questo libro si fa spesso riferimento a \q{rumore} o \q{spazzatura} (garbage) nello stack o in memoria.
Da dove arrivano?
Si tratta di ciò che resta dopo l'esecuzione di altre funzioni.
Un piccolo esempio:

\lstinputlisting[style=customc]{patterns/02_stack/08_noise/st.c}

Compilando si ottiene:

\lstinputlisting[caption=\NonOptimizing MSVC 2010,style=customasmx86]{patterns/02_stack/08_noise/st.asm}

Il compilatore si lamenterà un pochino\dots

\begin{lstlisting}
c:\Polygon\c>cl st.c /Fast.asm /MD
Microsoft (R) 32-bit C/C++ Optimizing Compiler Version 16.00.40219.01 for 80x86
Copyright (C) Microsoft Corporation.  All rights reserved.

st.c
c:\polygon\c\st.c(11) : warning C4700: uninitialized local variable 'c' used
c:\polygon\c\st.c(11) : warning C4700: uninitialized local variable 'b' used
c:\polygon\c\st.c(11) : warning C4700: uninitialized local variable 'a' used
Microsoft (R) Incremental Linker Version 10.00.40219.01
Copyright (C) Microsoft Corporation.  All rights reserved.

/out:st.exe
st.obj
\end{lstlisting}

Ma quando avvieremo il programma \dots

\begin{lstlisting}
c:\Polygon\c>st
1, 2, 3
\end{lstlisting}

Oh, che cosa strana! Non abbiamo impostato il valore di alcuna variabile in \TT{f2()}.
Si tratta di valori \q{fantasma}, che si trovano ancora nello stack.

\clearpage
Carichiamo l'esempio in \olly:

\begin{figure}[H]
\centering
\myincludegraphics{patterns/02_stack/08_noise/olly1.png}
\caption{\olly: \TT{f1()}}
\label{fig:stack_noise_olly1}
\end{figure}

Quando \TT{f1()} assegna le variabili $a$, $b$ e $c$, i loro valori sono memorizzati all'indirizzo \TT{0x1FF860} e seguenti.

\clearpage
E quando viene eseguita \TT{f2()}:

\begin{figure}[H]
\centering
\myincludegraphics{patterns/02_stack/08_noise/olly2.png}
\caption{\olly: \TT{f2()}}
\label{fig:stack_noise_olly2}
\end{figure}

... $a$, $b$ e $c$ di \TT{f2()} si trovano agli stessi indirizzi!
Nessuno ha ancora sovrascritto quei valori, quindi in quel punto sono ancora intatti.
Quindi, affinchè questa strana situazione si verifichi, più funzioni devono essere chiamate una dopo l'altra e
\ac{SP} deve essere uguale ad ogni ingresso nella funzione (ovvero le funzioni devono avere lo stesso numero di argomenti).
A quel punto le variabili locali si troveranno nelle stesse posizioni nello stack.
Per riassumere, tutti i valori nello stack (e nelle celle di memoria in generale) hanno valori lasciati lì dall'esecuzione di funzioni precedenti.
Non sono letteralmente casuali, piuttosto hanno valori non predicibili.
C'è un'altra opzione?
Sarebbe possibile ripulire porzioni dello stack prima di ogni esecuzione di una funzione, ma sarebbe un lavoro extra probabilmente inutile.

\subsubsection{MSVC 2013}

L'esempio è stato compilato con MSVC 2010.
Un lettore di questo libro ha provato a compilare l'esempio con MSVC 2013, lo ha eseguito, ed ha ottenuto i 3 numeri in ordine inverso:%

\begin{lstlisting}
c:\Polygon\c>st
3, 2, 1
\end{lstlisting}

Perchè?
Ho compilato anche io l'esempio in MSVC 2013 ed ho visto questo:


\begin{lstlisting}[caption=MSVC 2013,style=customasmx86]
_a$ = -12	; size = 4
_b$ = -8	; size = 4
_c$ = -4	; size = 4
_f2	PROC

...

_f2	ENDP

_c$ = -12	; size = 4
_b$ = -8	; size = 4
_a$ = -4	; size = 4
_f1	PROC

...

_f1	ENDP
\end{lstlisting}

Contrariamente a MSVC 2010, MSVC 2013 ha allocato le variabili a/b/c nella funzione \TT{f2()} in ordine inverso.%
E ciò è del tutto corretto, perchè lo standard \CCpp non ha una regola che definisce in quale ordine le variabili locali devono essere allocate nello stack.
La ragione per cui si presenta questa differenza è che MSVC 2010 lo fa in un certo modo, mentre MSVC 2013 ha probabilmente subito modifiche all'interno del compilatore, e si comporta quindi in modo leggermente diverso.
}
\DE{\mysection{\Stack}
\label{sec:stack}
\myindex{\Stack}

Der Stack ist eine der fundamentalen Datenstrukturen in der Informatik.
\footnote{\href{http://go.yurichev.com/17119}{wikipedia.org/wiki/Call\_Stack}}.
\ac{AKA} \ac{LIFO}.

Technisch betrachtet ist es ein Stapelspeicher innerhalb des Prozessspeichers der zusammen mit den \ESP (x86), \RSP (x64) oder dem \ac{SP} (ARM) Register als ein Zeiger in diesem Speicherblock fungiert.

\myindex{ARM!\Instructions!PUSH}
\myindex{ARM!\Instructions!POP}
\myindex{x86!\Instructions!PUSH}
\myindex{x86!\Instructions!POP}

Die häufigsten Stack-Zugriffsinstruktionen sind die \PUSH- und \POP-Instruktionen (in beidem x86 und ARM Thumb-Modus). \PUSH subtrahiert vom \ESP/\RSP/\ac{SP} 4 Byte im 32-Bit Modus (oder 8 im 64-Bit Modus) und schreibt dann den Inhalt des Zeigers an die Adresse auf die von \ESP/\RSP/\ac{SP} gezeigt wird.

\POP ist die umgekehrte Operation: Die Daten des Zeigers für die Speicherregion auf die von \ac{SP}
gezeigt wird werden ausgelesen und die Inhalte in den Instruktionsoperanden geschreiben (oft ist das ein Register). Dann werden 4 (beziehungsweise 8) Byte zum \gls{stack pointer} addiert.

Nach der Stackallokation, zeigt der \gls{stack pointer} auf den Boden des Stacks.
\PUSH verringert den \gls{stack pointer} und \POP erhöht ihn.
Der Boden des Stacks ist eigentlich der Anfang der Speicherregion die für den Stack reserviert wurde.
Das wirkt zunächst seltsam, aber so funktioniert es.

ARM unterstützt beides, aufsteigende und absteigende Stacks.

\myindex{ARM!\Instructions!STMFD}
\myindex{ARM!\Instructions!LDMFD}
\myindex{ARM!\Instructions!STMED}
\myindex{ARM!\Instructions!LDMED}
\myindex{ARM!\Instructions!STMFA}
\myindex{ARM!\Instructions!LDMFA}
\myindex{ARM!\Instructions!STMEA}
\myindex{ARM!\Instructions!LDMEA}

Zum Beispiel die \ac{STMFD}/\ac{LDMFD} und \ac{STMED}/\ac{LDMED} Instruktionen sind alle dafür gedacht mit einem absteigendem Stack zu arbeiten ( wächst nach unten, fängt mit hohen Adressen an und entwickelt sich zu niedrigeren Adressen). Die \ac{STMFA}/\ac{LDMFA} und \ac{STMEA}/\ac{LDMEA} Instruktionen sind dazu gedacht mit einem aufsteigendem Stack zu arbeiten (wächst nach oben und fängt mit niedrigeren Adressen an und wächst nach oben).

% It might be worth mentioning that STMED and STMEA write first,
% and then move the pointer, and that LDMED and LDMEA move the pointer first, and then read.
% In other words, ARM not only lets the stack grow in a non-standard direction,
% but also in a non-standard order.
% Maybe this can be in the glossary, which would explain why E stands for "empty".

\subsection{Warum wächst der Stack nach unten?}
\label{stack_grow_backwards}

Intuitiv, würden man annehmen das der Stack nach oben wächst z.B Richtung höherer Adressen, so wie bei jeder anderen Datenstruktur.

Der Grund das der Stack rückwärts wächst ist wohl historisch bedingt. Als Computer so groß waren das sie einen ganzen Raum beansprucht haben war es einfach Speicher in zwei Sektionen zu unterteilen, einen Teil für den \gls{heap} und einen Teil für den Stack. Sicher war zu dieser Zeit nicht bekannt wie groß der \gls{heap} und der Stack wachsen würden, während der Programm Laufzeit, also war die Lösung die einfachste mögliche.

\input{patterns/02_stack/stack_and_heap}

In \RitchieThompsonUNIX können wir folgendes lesen:

\begin{framed}
\begin{quotation}
Der user-core eines Programm Images wird in drei logische Segmente unterteilt. Das Programm-Text Segment beginnt bei 0 im virtuellen Adress Speicher. Während der Ausführung wird das Segment als schreibgeschützt markiert und eine einzelne Kopie des Segments wird unter allen Prozessen geteilt die das Programm ausführen. An der ersten 8K grenze über dem Programm Text Segment im Virtuellen Speicher, fängt der ``nonshared'' Bereich an, der nach Bedarf von Syscalls erweitert werden kann. Beginnend bei der höchsten Adresse im Virtuellen Speicher ist das Stack Segment, das Automatisch nach unten wächst während der Hardware Stackpointer sich ändert.
\end{quotation}
\end{framed}

Das erinnert daran wie manche Schüler Notizen zu  zwei Vorträgen in einem Notebook dokumentieren:
Notizen für den ersten Vortrag werden normal notiert, und Notizen zur zum zweiten Vortrag werden 
ans Ende des Notizbuches geschrieben, indem man das Notizbuch umdreht. Die Notizen treffen sich irgendwann
im Notizbuch aufgrund des fehlenden Freien Platzes.

% I think if we want to expand on this analogy,
% one might remember that the line number increases as as you go down a page.
% So when you decrease the address when pushing to the stack, visually,
% the stack does grow upwards.
% Of course, the problem is that in most human languages,
% just as with computers,
% we write downwards, so this direction is what makes buffer overflows so messy.

\subsection{Für was wird der Stack benutzt?}

% subsections
\EN{\input{patterns/02_stack/01_saving_ret_addr_EN}}
\RU{\input{patterns/02_stack/01_saving_ret_addr_RU}}
\DE{\input{patterns/02_stack/01_saving_ret_addr_DE}}
\FR{\input{patterns/02_stack/01_saving_ret_addr_FR}}
\PTBR{\input{patterns/02_stack/01_saving_ret_addr_PTBR}}
\IT{\input{patterns/02_stack/01_saving_ret_addr_IT}}
\PL{\input{patterns/02_stack/01_saving_ret_addr_PL}}
\JPN{\input{patterns/02_stack/01_saving_ret_addr_JPN}}

\EN{\input{patterns/02_stack/02_args_passing_EN}}
\RU{\input{patterns/02_stack/02_args_passing_RU}}
\PTBR{\input{patterns/02_stack/02_args_passing_PTBR}}
\DE{\input{patterns/02_stack/02_args_passing_DE}}
\IT{\input{patterns/02_stack/02_args_passing_IT}}
\FR{\input{patterns/02_stack/02_args_passing_FR}}
\JA{\input{patterns/02_stack/02_args_passing_JA}}
\PL{\input{patterns/02_stack/02_args_passing_PL}}


\EN{\input{patterns/02_stack/03_local_vars_EN}}
\RU{\input{patterns/02_stack/03_local_vars_RU}}
\DE{\input{patterns/02_stack/03_local_vars_DE}}
\PTBR{\input{patterns/02_stack/03_local_vars_PTBR}}
\EN{\input{patterns/02_stack/04_alloca/main_EN}}
\FR{\input{patterns/02_stack/04_alloca/main_FR}}
\RU{\input{patterns/02_stack/04_alloca/main_RU}}
\PTBR{\input{patterns/02_stack/04_alloca/main_PTBR}}
\IT{\input{patterns/02_stack/04_alloca/main_IT}}
\DE{\input{patterns/02_stack/04_alloca/main_DE}}
\PL{\input{patterns/02_stack/04_alloca/main_PL}}
\JPN{\input{patterns/02_stack/04_alloca/main_JPN}}

\input{patterns/02_stack/05_SEH}
\ifdefined\ENGLISH
\subsubsection{Buffer overflow protection}

More about it here~(\myref{subsec:bufferoverflow}).
\fi

\ifdefined\RUSSIAN
\subsubsection{Защита от переполнений буфера}

Здесь больше об этом~(\myref{subsec:bufferoverflow}).
\fi

\ifdefined\BRAZILIAN
\subsubsection{Proteção contra estouro de buffer}

Mais sobre aqui~(\myref{subsec:bufferoverflow}).
\fi

\ifdefined\ITALIAN
\subsubsection{Protezione contro buffer overflow}

Maggiori informazioni qui~(\myref{subsec:bufferoverflow}).
\fi

\ifdefined\FRENCH
\subsubsection{Protection contre les débordements de tampon}

Lire à ce propos~(\myref{subsec:bufferoverflow}).
\fi


\ifdefined\POLISH
\subsubsection{Metody zabiezpieczenia przed przepełnieniem stosu}

Więcej o tym tutaj~(\myref{subsec:bufferoverflow}).
\fi

\ifdefined\JAPANESE
\subsubsection{バッファオーバーフロー保護}

詳細はこちら~(\myref{subsec:bufferoverflow})
\fi


\subsubsection{Automatisches deallokieren der Daten auf dem Stack}

Vielleicht ist der Grund warum man lokale Variablen und SEH Einträge auf dem Stack speichert, weil sie beim 
verlassen der Funktion automatisch aufgeräumt werden. Man braucht dabei nur eine Instruktion um die Position
des Stackpointers zu korrigieren (oftmals ist es die \ADD Instruktion). Funktions Argumente, könnte man sagen 
werden auch am Ende der Funktion deallokiert. Im Kontrast dazu, alles was auf dem \emph{heap} gespeichert wird muss
explizit deallokiert werden. 

% sections
\EN{\input{patterns/02_stack/07_layout_EN}}
\RU{\input{patterns/02_stack/07_layout_RU}}
\DE{\input{patterns/02_stack/07_layout_DE}}
\PTBR{\input{patterns/02_stack/07_layout_PTBR}}
\EN{\input{patterns/02_stack/08_noise/main_EN}}
\FR{\input{patterns/02_stack/08_noise/main_FR}}
\RU{\input{patterns/02_stack/08_noise/main_RU}}
\IT{\input{patterns/02_stack/08_noise/main_IT}}
\DE{\input{patterns/02_stack/08_noise/main_DE}}
\PL{\input{patterns/02_stack/08_noise/main_PL}}
\JA{\input{patterns/02_stack/08_noise/main_JA}}

\input{patterns/02_stack/exercises}
}
\PL{\mysection{\Stack}
\label{sec:stack}
\myindex{\Stack}

Stos w informatyce jest jedną z najbardziej fundamentalnych struktur danych
\footnote{\href{http://go.yurichev.com/17119}{wikipedia.org/wiki/Call\_stack}}.
\ac{AKA} \ac{LIFO}.

Technicznie rzecz biorąc, jest to tylko blok pamięci w pamięci procesora + rejestr \ESP w x86 lub \RSP w x64, lub \ac{SP} w ARM, który wskazuje obszar gdzieś w granicach tego bloku.

\myindex{ARM!\Instructions!PUSH}
\myindex{ARM!\Instructions!POP}
\myindex{x86!\Instructions!PUSH}
\myindex{x86!\Instructions!POP}
Najczęsciej wykorzystywanymi instrukcjami do operowania na stosie są \PUSH i \POP (w x86 i Thumb-trybie ARM). 
\PUSH zmniejsza \ESP/\RSP/\ac{SP} o 4 w trybie 32-bitowym (lub o 8 w 64-bitowym),
następnie zapisuje pod adres, na który wskazuje \ESP/\RSP/\ac{SP}, zawartość swojego operandu.

\POP jest odwrotną operacją- najpierw zdejmuje ze \glslink{stack pointer}{wskaźnika stosu} wartość i umieszcza ją do operandu 
(który często jest rejestrem), a następnie zwiększa wskaźnik o 4 (lub 8).

Przed alokacją pamięci na stosie \glslink{stack pointer}{rejestr-wskaźnik} wskazuje na koniec stosu.
Koniec stosu znajduje się na początku zaalokowanego bloku pamięci, przeznaczonego na stos. Może zabrzmieć to dziwnie, ale tak to działa.
\PUSH zmniejsza \glslink{stack pointer}{rejestr-wskaźnik}, а \POP~--- zwiększa.

W procesorze ARM jest wsparcie dla stosów zarówno rosnących w dół, jak i rosnącyh w górę.

\myindex{ARM!\Instructions!STMFD}
\myindex{ARM!\Instructions!LDMFD}
\myindex{ARM!\Instructions!STMED}
\myindex{ARM!\Instructions!LDMED}
\myindex{ARM!\Instructions!STMFA}
\myindex{ARM!\Instructions!LDMFA}
\myindex{ARM!\Instructions!STMEA}
\myindex{ARM!\Instructions!LDMEA}

Na przykład, instrukcje \ac{STMFD}/\ac{LDMFD}, \ac{STMED}/\ac{LDMED} są przeznaczone dla stosu malejącego (rośnie w dół, zaczynając od adresów wysokich, do adresów niskich).\\
Natomiast instrukcje \ac{STMFA}/\ac{LDMFA}, \ac{STMEA}/\ac{LDMEA} są przeznaczone dla stosu rosnącego (rośnie w górę, zaczynając od niskich adresów, kończąc na adresach wysokich).

% It might be worth mentioning that STMED and STMEA write first,
% and then move the pointer,
% and that LDMED and LDMEA move the pointer first, and then read.
% In other words, ARM not only lets the stack grow in a non-standard direction,
% but also in a non-standard order.
% Maybe this can be in the glossary, which would explain why E stands for "empty".

\subsection{Dlaczego stos rośnie wstecznie?}
\label{stack_grow_backwards}

Intuicyjnie moglibyśmy pomyśleć, że, jak i każda inna struktura danych, stos mogłby rosnąć w górę, tzn. w kierunku zwiększenia adresów.

Powód, dlaczego stos rośnie w dół, jest najprawdobodobniej historyczny.
Kiedy komputery były duże i zajmowały cały pokój, można było bardzo łatwo rozdzielić segment na dwa obszary: dla \glslink{heap}{kopca} i dla stosu.
Z góry nie było wiadomo, jak duża może być \glslink{heap}{sterta} lub stos, dlatego takie rozwiązanie było najbardziej logiczne.

\input{patterns/02_stack/stack_and_heap}

W \RitchieThompsonUNIX można przeczytać:

\begin{framed}
\begin{quotation}
The user-core part of an image is divided into three logical segments. The program text segment begins at location 0 in the virtual address space. During execution, this segment is write-protected and a single copy of it is shared among all processes executing the same program. At the first 8K byte boundary above the program text segment in the virtual address space begins a nonshared, writable data segment, the size of which may be extended by a system call. Starting at the highest address in the virtual address space is a stack segment, which automatically grows downward as the hardware's stack pointer fluctuates.
\end{quotation}
\end{framed}

To trochę przypomina podejście studenta,
który pisze dwie osobne lektury w jednym zeszycie:
pierwsza lektura jest pisana jak zwykle od początku zeszytu, a druga jest pisana od końca zeszytu.
Lektury mogą się "spotkać" gdzieś na środku zeszytu, jeśli zabraknie miejsca.

% I think if we want to expand on this analogy,
% one might remember that the line number increases as as you go down a page.
% So when you decrease the address when pushing to the stack, visually,
% the stack does grow upwards.
% Of course, the problem is that in most human languages,
% just as with computers,
% we write downwards, so this direction is what makes buffer overflows so messy.

\subsection{Do jakich celów służy stos?}

% subsections
\EN{\input{patterns/02_stack/01_saving_ret_addr_EN}}
\RU{\input{patterns/02_stack/01_saving_ret_addr_RU}}
\DE{\input{patterns/02_stack/01_saving_ret_addr_DE}}
\FR{\input{patterns/02_stack/01_saving_ret_addr_FR}}
\PTBR{\input{patterns/02_stack/01_saving_ret_addr_PTBR}}
\IT{\input{patterns/02_stack/01_saving_ret_addr_IT}}
\PL{\input{patterns/02_stack/01_saving_ret_addr_PL}}
\JPN{\input{patterns/02_stack/01_saving_ret_addr_JPN}}

\EN{\input{patterns/02_stack/02_args_passing_EN}}
\RU{\input{patterns/02_stack/02_args_passing_RU}}
\PTBR{\input{patterns/02_stack/02_args_passing_PTBR}}
\DE{\input{patterns/02_stack/02_args_passing_DE}}
\IT{\input{patterns/02_stack/02_args_passing_IT}}
\FR{\input{patterns/02_stack/02_args_passing_FR}}
\JA{\input{patterns/02_stack/02_args_passing_JA}}
\PL{\input{patterns/02_stack/02_args_passing_PL}}


\EN{\input{patterns/02_stack/03_local_vars_EN}}
\RU{\input{patterns/02_stack/03_local_vars_RU}}
\PTBR{\input{patterns/02_stack/03_local_vars_PTBR}}
\PL{\input{patterns/02_stack/03_local_vars_PL}}
\EN{\input{patterns/02_stack/04_alloca/main_EN}}
\FR{\input{patterns/02_stack/04_alloca/main_FR}}
\RU{\input{patterns/02_stack/04_alloca/main_RU}}
\PTBR{\input{patterns/02_stack/04_alloca/main_PTBR}}
\IT{\input{patterns/02_stack/04_alloca/main_IT}}
\DE{\input{patterns/02_stack/04_alloca/main_DE}}
\PL{\input{patterns/02_stack/04_alloca/main_PL}}
\JPN{\input{patterns/02_stack/04_alloca/main_JPN}}

\input{patterns/02_stack/05_SEH}
\ifdefined\ENGLISH
\subsubsection{Buffer overflow protection}

More about it here~(\myref{subsec:bufferoverflow}).
\fi

\ifdefined\RUSSIAN
\subsubsection{Защита от переполнений буфера}

Здесь больше об этом~(\myref{subsec:bufferoverflow}).
\fi

\ifdefined\BRAZILIAN
\subsubsection{Proteção contra estouro de buffer}

Mais sobre aqui~(\myref{subsec:bufferoverflow}).
\fi

\ifdefined\ITALIAN
\subsubsection{Protezione contro buffer overflow}

Maggiori informazioni qui~(\myref{subsec:bufferoverflow}).
\fi

\ifdefined\FRENCH
\subsubsection{Protection contre les débordements de tampon}

Lire à ce propos~(\myref{subsec:bufferoverflow}).
\fi


\ifdefined\POLISH
\subsubsection{Metody zabiezpieczenia przed przepełnieniem stosu}

Więcej o tym tutaj~(\myref{subsec:bufferoverflow}).
\fi

\ifdefined\JAPANESE
\subsubsection{バッファオーバーフロー保護}

詳細はこちら~(\myref{subsec:bufferoverflow})
\fi


\subsubsection{Automatyczne zwalnianie danych na stosie}
Możliwym powodem przechowywania zmiennych lokalnych i rekordów SEH na stosie jest to, że kiedy funkcja zakończy działanie są one automatycznie zwalniane ze stosu używając tylko jednej instrukcji w celu przywrócenia poprzedniego stanu stosu (często jest to instrukcja ADD). Argumenty funkcji także są automatycznie zwalniane z pamięci pod koniec funkcji. Natomiast wszystko co jest przechowywane na stercie(\emph{heap}) trzeba zwalniać jawnie.

% sections
\EN{\input{patterns/02_stack/07_layout_EN}}
\RU{\input{patterns/02_stack/07_layout_RU}}
\PTBR{\input{patterns/02_stack/07_layout_PTBR}}
\PL{\input{patterns/02_stack/07_layout_PL}}
\EN{\input{patterns/02_stack/08_noise/main_EN}}
\FR{\input{patterns/02_stack/08_noise/main_FR}}
\RU{\input{patterns/02_stack/08_noise/main_RU}}
\IT{\input{patterns/02_stack/08_noise/main_IT}}
\DE{\input{patterns/02_stack/08_noise/main_DE}}
\PL{\input{patterns/02_stack/08_noise/main_PL}}
\JA{\input{patterns/02_stack/08_noise/main_JA}}

\input{patterns/02_stack/exercises}


}
\JA{\subsection{スタックのノイズ}
\label{noise_in_stack}

\epigraph{ある人が何かがランダムに見えると言うとき、実際には、その中に何らかの規則性を見ることができないということです}{Stephen Wolfram, A New Kind of Science.}

多くの場合、この本では\q{ノイズ}や\q{ガベージ}の値がスタックやメモリに記述されています。
彼らはどこから来たのか?
これらは、他の関数の実行後にそこに残っているものです。 
短い例:

\lstinputlisting[style=customc]{patterns/02_stack/08_noise/st.c}

コンパイルすると \dots

\lstinputlisting[caption=\NonOptimizing MSVC 2010,style=customasmx86]{patterns/02_stack/08_noise/st.asm}

コンパイラは少し不満そうです\dots

\begin{lstlisting}
c:\Polygon\c>cl st.c /Fast.asm /MD
Microsoft (R) 32-bit C/C++ Optimizing Compiler Version 16.00.40219.01 for 80x86
Copyright (C) Microsoft Corporation.  All rights reserved.

st.c
c:\polygon\c\st.c(11) : warning C4700: uninitialized local variable 'c' used
c:\polygon\c\st.c(11) : warning C4700: uninitialized local variable 'b' used
c:\polygon\c\st.c(11) : warning C4700: uninitialized local variable 'a' used
Microsoft (R) Incremental Linker Version 10.00.40219.01
Copyright (C) Microsoft Corporation.  All rights reserved.

/out:st.exe
st.obj
\end{lstlisting}

しかし、コンパイルされたプログラムを実行すると \dots

\begin{lstlisting}
c:\Polygon\c>st
1, 2, 3
\end{lstlisting}

ああ、なんて奇妙なんでしょう! 我々は\TT{f2()}に変数を設定しませんでした。
これらは\q{ゴースト}値であり、まだスタックに入っています。

\clearpage
サンプルを \olly にロードしましょう。

\begin{figure}[H]
\centering
\myincludegraphics{patterns/02_stack/08_noise/olly1.png}
\caption{\olly: \TT{f1()}}
\label{fig:stack_noise_olly1}
\end{figure}

\TT{f1()}が変数$a$、$b$、$c$を代入すると、その値はアドレス\TT{0x1FF860}に格納されます。

\clearpage
そして\TT{f2()}が実行されるとき:

\begin{figure}[H]
\centering
\myincludegraphics{patterns/02_stack/08_noise/olly2.png}
\caption{\olly: \TT{f2()}}
\label{fig:stack_noise_olly2}
\end{figure}

... \TT{f2()}の$a$、$b$、$c$は同じアドレスにあります!
誰もまだ値を上書きしていないので、その時点でまだ変更はありません。
したがって、この奇妙な状況が発生するためには、いくつかの関数を次々と呼び出さなければならず、
\ac{SP}は各関数エントリで同じでなければならない(すなわち、それらは同じ数の引数を有する)。 
次に、ローカル変数はスタック内の同じ位置に配置されます。 
要約すると、スタック(およびメモリセル)内のすべての値は、以前の関数実行から残った値を持ちます。 
彼らは厳密な意味でランダムではなく、むしろ予測不可能な値を持っています。 
別のオプションがありますか? 
各関数の実行前にスタックの一部をクリアすることはおそらく可能ですが、余計な(そして不要な)作業です。

\subsubsection{MSVC 2013}

この例はMSVC 2010によってコンパイルされました。
しかし、この本の読者は、このサンプルをMSVC 2013でコンパイルして実行し、3つの数字がすべて逆の結果になるでしょう。

\begin{lstlisting}
c:\Polygon\c>st
3, 2, 1
\end{lstlisting}

どうして?
私もMSVC 2013でこの例をコンパイルし、見てみました。

\begin{lstlisting}[caption=MSVC 2013,style=customasmx86]
_a$ = -12	; size = 4
_b$ = -8	; size = 4
_c$ = -4	; size = 4
_f2	PROC

...

_f2	ENDP

_c$ = -12	; size = 4
_b$ = -8	; size = 4
_a$ = -4	; size = 4
_f1	PROC

...

_f1	ENDP
\end{lstlisting}

MSVC 2010とは異なり、MSVC 2013は関数\TT{f2()}のa/b/c変数を逆順に割り当てました。
これは完全に正しい動作です。 \CCpp 標準にはルールがありません。ローカル変数をローカルスタックに割り当てる必要があれば、どういう順番でもよいのです。 
理由の違いは、MSVC 2010にはその方法があり、MSVC 2013はおそらくコンパイラの心臓部で何かが変わったと考えられるからです。
}

\input{patterns/02_stack/exercises}

