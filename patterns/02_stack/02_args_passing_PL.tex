\subsubsection{Przekazywanie argumentów funkcji}

Najbardziej powszechny sposób na przekazywanie parametrów funkcji w x86 to \q{cdecl}:

\begin{lstlisting}[style=customasmx86]
push arg3
push arg2
push arg1
call f
add esp, 12 ; 4*3=12
\end{lstlisting}

Wywoływana funkcja otrzymuje swoje parametry również przez wskaźnik stosu.

W konsekwensji tego, zawartość stosu przed wykonaniem pierwszej instrukcji funkcji wygląda w ten sposób \ttf{}:

\begin{center}
\begin{tabular}{ | l | l | }
\hline
ESP & adres powrotu \\
\hline
ESP+4 & \argument \#1, \MarkedInIDAAs{} \TT{arg\_0} \\
\hline
ESP+8 & \argument \#2, \MarkedInIDAAs{} \TT{arg\_4} \\
\hline
ESP+0xC & \argument \#3, \MarkedInIDAAs{} \TT{arg\_8} \\
\hline
\dots & \dots \\
\hline
\end{tabular}
\end{center}

Patrz również w odpowiednim rodziale o innych sposobach przekazywania argumentów przez stos ~(\myref{sec:callingconventions}).

\par A propos, funkcja wywoływana nie posiada informacji o ilości argumentów przekazywanych do niej.
Funkcje w C o zmiennej ilości parametrów (jak np. \printf) wyznaczają ich ilość za pomocą specjalnych specyfikatorów (rozpoczynających się z \%).

Jeśli napisać coś w stylu:

\begin{lstlisting}
printf("%d %d %d", 1234);
\end{lstlisting}

\printf wyprowadzi 1234, następnie jeszcze dwie liczby losowe\footnote{Tak na prawdę nie są one losowe, patrz: \myref{noise_in_stack}}, który przypadkowo okazały się na stosie obok.

\par
Właśnie dlatego nie jest to ważne jak zapiszemy f-cję \main{}:\\
jak \main{}, \TT{main(int argc, char *argv[])}\\
lub \TT{main(int argc, char *argv[], char *envp[])}.

W rzeczywistości, \ac{CRT}-kod wywołuje \main mniej więcej w ten sposób:
	
\begin{lstlisting}[style=customasmx86]
push envp
push argv
push argc
call main
...
\end{lstlisting}

Jeśli zadeklarujecie \main bez argumentów, one, jednak, są obecne na stosie, lecz nie są wykorzystywane.
Jeśli zadeklarujecie \main jako \TT{main(int argc, char *argv[])}, 
to można korzystać z pierwszych dwóch argumentów, a trzeci zostanie dla funkcji \q{niewidocznym}.
Co więcej, można nawet zadeklarować \TT{main(int argc)}, i to zadziała.

\myparagraph{Alternatywne sposoby na przekazywanie argumentów}

Warto zauważyć,że, generalnie, nie ma narzutu na przekazywanie argumentów przez stos, nie jest to wymogiem formalnym.
Można robić to zupełnie inaczej, nie korzystając ze stosu w ogóle.

W pewnym sensie, popularną metodą wśród początkującyh jest przekazywanie argumentów przez zmienne globalne, na przykład:

\lstinputlisting[caption=Kod w asemblerze,style=customasmx86]{patterns/02_stack/global_args.asm}

Ale ta metoda posiada dużą wadę: funkcja \emph{do\_something()} nie może wywołać sama siebie poprzez rekurencję (lub za pomocą innej funkcji),
dlatego że wtedy będzie trzeba wymazać własne argumenty.
Ta sama historia ze zmiennymi lokalnymi: jeśli przechowywać je w zmiennych globalnych, funkcja nie będzie nogła wywołać sama siebie.
Do tego, ta metoda nie jest biezpieczna dla środowiska wielowątkowego\footnote{Przy poprawnej realizacji,
każdy wątek będzie miał własny stos lokalny ze swoimi argumentami/zmiennymi.}.
Metoda przechowywania podobnej informacji na stosie wszystko znacznie upraszcza ---
on może przechowywać tyle argumentów funkcji/zmiennych,
ile się w nim zmieści.

W \InSqBrackets{\TAOCPvolI{}, 189} można przeczytać o jeszcze bardziej dziwnych metodach przekazywania argumentów funkcji, które były bardzo wygodne na
 IBM System/360.

\myindex{MS-DOS}
\myindex{x86!\Instructions!INT}

W MS-DOS istniała metoda przekazywania argumentów przez rejestry, na przykład, ten fragment kodu dla starego 16-bitowego MS-DOS
wyprintuje ``Hello, world!'':

\begin{lstlisting}[style=customasmx86]
mov  dx, msg      ; adres powiadomienia
mov  ah, 9        ; §9 oznacza funkcję "wyprowadzenie linii"§
int  21h          ; DOS "syscall"

mov  ah, 4ch      ; §funkcja zakończenia programu
int  21h          ; DOS "syscall"

msg  db 'Hello, World!\$'
\end{lstlisting}

\myindex{fastcall}
Jest to bardzo podobne do metody \myref{fastcall}.
I jeszcze do sposobu robienia syscall w Linux (\myref{linux_syscall}) i Windows.

\myindex{x86!\Flags!CF}
Jeżeli f-cja w MS-DOS zwraca boolean (tzn jeden bit, zwule sygnalizujący o błędzie wewnętrznym),
to często była wykorzystywana flaga \TT{CF}.

Na przykład:

\begin{lstlisting}[style=customasmx86]
mov ah, 3ch       ; stworzyć plik
lea dx, filename
mov cl, 1
int 21h
jc  error
mov file_handle, ax
...
error:
...
\end{lstlisting}

W przypadku wystąpienia błędu, flaga \TT{CF} zostaje ustawiona.
W innym przypadku, handle stworzonego pliku jest zwracany do \TT{AX}.

Ta metoda dotychczasowo jest wykorzystywana przez programistów asemblera.
W kodach wyjściowych Windows Research Kernel (który jest bardzo podobny do Windows 2003) możemy znaleźć coś takiego\\
(plik \emph{base/ntos/ke/i386/cpu.asm}):

\begin{lstlisting}[style=customasmx86]
        public  Get386Stepping
Get386Stepping  proc

        call    MultiplyTest            ; Perform multiplication test
        jnc     short G3s00             ; if nc, muttest is ok
        mov     ax, 0
        ret
G3s00:
        call    Check386B0              ; Check for B0 stepping
        jnc     short G3s05             ; if nc, it's B1/later
        mov     ax, 100h                ; It is B0/earlier stepping
        ret

G3s05:
        call    Check386D1              ; Check for D1 stepping
        jc      short G3s10             ; if c, it is NOT D1
        mov     ax, 301h                ; It is D1/later stepping
        ret

G3s10:
        mov     ax, 101h                ; assume it is B1 stepping
        ret

	...

MultiplyTest    proc

        xor     cx,cx                   ; 64K times is a nice round number
mlt00:  push    cx
        call    Multiply                ; does this chip's multiply work?
        pop     cx
        jc      short mltx              ; if c, No, exit
        loop    mlt00                   ; if nc, YEs, loop to try again
        clc
mltx:
        ret

MultiplyTest    endp
\end{lstlisting}



