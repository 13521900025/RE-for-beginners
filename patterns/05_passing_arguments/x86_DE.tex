\subsection{x86}

\subsubsection{MSVC}

Das ist das Ergebnis nach dem kompilieren (MSVC 2010 Express):

\lstinputlisting[label=src:passing_arguments_ex_MSVC_cdecl,caption=MSVC 2010 Express,style=customasmx86]{patterns/05_passing_arguments/msvc_DE.asm}

\myindex{x86!\Registers!EBP}

Was wir hier sehen ist das die \main Funktion drei Zahlen auf den Stack schiebt und \TT{f(int,int,int).} aufruft

Der Argument zugriff innerhalb von \ttf wird organisiert mit der Hilfe von Makros wie zum Beispiel:\\
\TT{\_a\$ = 8}, 
auf die gleiche weise wie Lokale Variablen allerdings mit positiven Offsets (adressiert mit \IT{plus}).

Also adressieren wir die \IT{äussere} Seite des \glslink{stack frame}{Stack frame} indem wir \TT{\_a\$} Makros zum Wert des \EBP Registers addieren  

\myindex{x86!\Instructions!IMUL}
\myindex{x86!\Instructions!ADD}

Dann wird der Wert von $a$ in \EAX gespeichert. Nachdem die \IMUL Instruktion ausgeführt wurde, ist
der Wert in \EAX ein Produkt des Wertes aus \EAX und dem Inhalt von \TT{\_b}.

Nun addiert \ADD den Wert in \TT{\_c} auf \EAX

Der Wert in \EAX muss nicht verschoben werden: Der Wert von \EAX befindet sich schon wo er sein muss

Beim zurück kehren zur \gls{caller} Funktion, wird der Wert aus \EAX genommen und als Argument 
für den \printf Aufruf benutzt.


\clearpage
\myparagraph{\Optimizing MSVC + \olly}
\myindex{\olly}

Wir untersuchen das (optimierte) Beispiel in \olly. Hier ist der erste
Durchlauf:

\begin{figure}[H]
\centering
\myincludegraphics{patterns/10_strings/1_strlen/olly1.png}
\caption{\olly: Beginn erster Durchlauf}
\label{fig:strlen_olly_1}
\end{figure}

Wir sehen, dass \olly eine Schleife gefunden hat, und zur Verbesserung der
Lesbarkeit, diese in eckige Klammern \emph{eingeschlossen} hat.
Nach Rechtsklick auf \EAX wählen wir \q{Follow in Dump} und das Speicherfenster
scrollt an die passende Stelle. 
Hier sehen wir den String \q{hello!} im Speicher. 
Dahinter befindet sich mindestens ein Nullbyte und im Anschluss Zufallsbits.

Wenn \olly ein Register mit einer gültigen Adresse, die auf einen String zeigt,
findet, wird dieser String angezeigt.

\clearpage
Wir drücken einige Male F8 (\stepover) um zum Anfang der Schleifenkörpers zu
gelangen:

\begin{figure}[H]
\centering
\myincludegraphics{patterns/10_strings/1_strlen/olly2.png}
\caption{\olly: Beginn zweiter Durchlauf}
\label{fig:strlen_olly_2}
\end{figure}

Wir sehen, dass \EAX nun die Adresse des zweiten Zeichens des Strings enthält.

\clearpage

Durch hinreichend häufiges Drücken von F8 verlassen wir schließlich die
Schleife:

\begin{figure}[H]
\centering
\myincludegraphics{patterns/10_strings/1_strlen/olly3.png}
\caption{\olly: Pointer Differenz wird berechnet}
\label{fig:strlen_olly_3}
\end{figure}

% FIXME:
Wir sehen, dass \EAX jetzt die Adresse des Nullbytes direkt hinter dem String
enthält. In der Zwischenzeit hat sich \EDX nicht verändert, es zeigt also immer
noch auf den Anfang des Strings. 

Die Differenz zwischen den beiden Adressen wird jetzt berechnet. 

\clearpage
Der \SUB Befehl wurde gerade ausgeführt:

\begin{figure}[H]
\centering
\myincludegraphics{patterns/10_strings/1_strlen/olly4.png}
\caption{\olly: \EAX muss dekrementiert werden}
\label{fig:strlen_olly_4}
\end{figure}

Die Differenz der Pointer im \EAX Register beträgt nun--7.
Tatsächlich beträgt die Länge des \q{hello!} Strings 6 Zeichen, aber mit dem
Nullbyte am Ende dazugezählt sind es 7.
Die Funktion \TT{strlen()} soll aber die Anzahl der Nicht-Null-Zeichen im String
zurückliefert, also wird einmal dekrementiert und der Funktionsaufruf
anschließend beendet.


\subsubsection{GCC}


Lasst uns das gleiche in GCC kompilieren und die Ergebnisse in \IDA betrachten:

\lstinputlisting[caption=GCC 4.4.1,style=customasmx86]{patterns/05_passing_arguments/gcc_DE.asm}

Das Ergebnis ist fast das gleiche aber mit kleineren Unterschieden die wir bereits früher
besprochen haben.

Der \gls{stack pointer} wird nicht zurück gesetzt nach den beiden Funktion aufrufen (f und printf),
weil die vorletzte \TT{LEAVE} Instruktion (\myref{x86_ins:LEAVE}) sich um das zurück setzen kümmert.
