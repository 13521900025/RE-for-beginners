\subsection{x86}

\subsubsection{MSVC}

Ecco il risultato della compilazione ocn MSVC 2010 Express:

\lstinputlisting[label=src:passing_arguments_ex_MSVC_cdecl,caption=MSVC 2010 Express,style=customasmx86]{patterns/05_passing_arguments/msvc_EN.asm}

\myindex{x86!\Registers!EBP}

Vediamo che la funzione \main fa il push di 3 numeri sullo stack e chiama \TT{f(int,int,int).} 

L'accesso agli argomenti all'interno della funzione \ttf e' gestito con l'aiuto di macro come: \TT{\_a\$ = 8}, 
allo stesso modo delle variabili locali, ma con offset positivi.
Si sta quindi indirizzando il lato \emph{esterno} dello \gls{stack frame} sommando la macro \TT{\_a\$} al valore contenuto nel registro \EBP.

\myindex{x86!\Instructions!IMUL}
\myindex{x86!\Instructions!ADD}

Successivamente il valore di $a$ e' memorizzato in \EAX. A seguito dell'esecuzione dell'istruzione \IMUL, il valore in \EAX e' 
il \gls{prodotto} del valore in \EAX e del contenuto di \TT{\_b}.

Infine, \ADD aggiunge il valore in \TT{\_c} a \EAX.

Il valore \EAX non necessita di essere spostato: si trova gia' nel posto giusto.
Al termine, la funzione chiamante (\gls{caller}) prende il valore di \EAX e lo usa come argomento di \printf.

\clearpage
\subsubsection{MSVC + \olly}
\myindex{\olly}

Proviamo ad analizzare l'esempio con \olly.
Carichiamo l'eseguibile e premiamo F8 (\stepover) fino a raggiungere il nostro eseguibile invece che \TT{ntdll.dll}.
Scorriamo verso l'alto finche' appare \main .

Clicchiamo sulla prima istruzione (\TT{PUSH EBP}), premiamo F2 (\emph{set a breakpoint}), e quindi F9 (\emph{Run}).
Il breakpoint sara' scatenato all'inizio della funzione \main .

Tracciamo adesso fino al punto in cui viene calcolato l'indirizzo della variabile $x$:

\begin{figure}[H]
\centering
\myincludegraphics{patterns/04_scanf/1_simple/ex1_olly_1.png}
\caption{\olly: The address of the local variable is calculated}
\label{fig:scanf_ex1_olly_1}
\end{figure}


Click di destra su \EAX nella finestra dei registri e selezioniamo \q{Follow in stack}.

Questo indirizzo apparira' nella finestra dello stack.
La freccia rossa aggiunta punta alla variabile nello stack locale.
Al momento questa locazione contiene un po' di immondizia (garbage) (\TT{0x6E494714}).
Con l'aiuto dell'istruzione \PUSH l'indirizzo di questo elemento dello stack sara' memorizzato nello stesso stack alla posizione successiva.
Tracciamo con F8 finche' non viene completata l'esecuzione della funzione \scanf.
Durante l'esecuzione di \scanf, diamo in input un valore nella console. Ad esempio 123:

\lstinputlisting{patterns/04_scanf/1_simple/console.txt}

\clearpage
\scanf ha gia' completato la sua esecuzione:

\begin{figure}[H]
\centering
\myincludegraphics{patterns/04_scanf/1_simple/ex1_olly_3.png}
\caption{\olly: \scanf executed}
\label{fig:scanf_ex1_olly_3}
\end{figure}

\scanf restituisce 1 in \EAX, e cio' implica che ha letto con successo un valore.
Se guardiamo nuovamente l'elemento nello stack corrispondente alla variabile locale, adesso contiente \TT{0x7B} (123).

\clearpage

Successivamente questo valore viene copiato dallo stack al registro \ECX e passato a \printf:

\begin{figure}[H]
\centering
\myincludegraphics{patterns/04_scanf/1_simple/ex1_olly_4.png}
\caption{\olly: preparing the value for passing to \printf}
\label{fig:scanf_ex1_olly_4}
\end{figure}


\subsubsection{GCC}

Compiliamo lo stesso esempio con GCC 4.4.1 ed osserviamo il risultato con \IDA:

\lstinputlisting[caption=GCC 4.4.1,style=customasmx86]{patterns/05_passing_arguments/gcc_EN.asm}

Il risultato e' pressoche' identico, a meno di piccole differenze gia' discusse in precedenza.

Lo \gls{stack pointer} non viene ripristinato dopo le due chiamate a funzione(f and printf), 
poiche' se ne occupa la penultima istruzione \TT{LEAVE} (\myref{x86_ins:LEAVE}) alla fine della funizone.
