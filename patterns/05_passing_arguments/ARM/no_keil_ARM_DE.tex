\subsubsection{\NonOptimizingKeilVI (\ARMMode)}

\lstinputlisting[style=customasmARM]{patterns/05_passing_arguments/ARM/tmp1.asm}

Die \main Funktion ruft einfach zwei weitere Funktionen auf, mit diesen drei werten die dann
der ersten Funktion übergeben werden.

Wie bereits angemerkt, auf ARM werden die erste 4 Werte in den ersten vier Registern übergeben (\Reg{0}-\Reg{3}).

Die \ttf Funktion, benutzt augenscheinlich die ersten drei Register (\Reg{0}-\Reg{2}) als Argumente

\myindex{ARM!\Instructions!MLA}

Die \TT{MLA} (\IT{Multiplikation Akkumulierung})
Instruktion multipliziert den ersten der beiden Operanden (\Reg{3} und \Reg{1}), addiert den dritten Operanden
(\Reg{2}) zum Produkt und speichert das Ergebnis in das nullte Register (\Reg{0}), wohin per Standard definiert
Funktionen ihre Rückgabe werte speichern.

\myindex{Fused multiply–add}
Multiplikation und Addition in einem (\IT{Fused multiply-add})
ist ist eine sehr nützliche Instruktion. Nebenbei bemerkt gab es eine solche Instruktion 
auf x86 nicht, bis FMA-Instruktionen in SIMD implementiert wurden.
\footnote{\href{http://go.yurichev.com/17103}{wikipedia}}.

Die erste Instruktion \TT{MOV R3, R0},
ist anscheinen redundant (es hätte anstatt eine einzelne \TT{MLA} Instruktion benutzt werden können).
Der Compiler hat die Instruktion nicht weg optimiert, da das Programm ohne Optimierungen compiliert wurde.

\myindex{ARM!Mode switching}
\myindex{ARM!\Instructions!BX}

Die \TT{BX} Instruktion gibt die Kontrolle an die Adresse zurück die im \ac{LR} Register gespeichert ist.
Falls nötig switcht die Instruktion den Prozessor Modus von Thumb zu ARM oder umgekehrt.
Das kann nötig sein da die \ttf Funktion nicht weiß von welcher Art Code sie vielleicht aufgerufen wird,
ARM oder Thumb.
Wenn die Funktion von Thumb Code aufgerufen wird gibt \TT{BX} nicht nur 
die Kontrolle an die aufrufende Funktion zurück, sondern switcht auch den Prozessor Modus auf 
Thumb. Es wird kein switch ausgeführt, falls die Funktion von ARM Code aufgerufen wurde \InSqBrackets{\ARMSevenRef A2.3.2} 
