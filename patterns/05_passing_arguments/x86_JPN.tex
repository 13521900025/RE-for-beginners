\subsection{x86}

\subsubsection{MSVC}

コンパイルして得られるものを次に示します(MSVC 2010 Express)。

\lstinputlisting[label=src:passing_arguments_ex_MSVC_cdecl,caption=MSVC 2010 Express,style=customasmx86]{patterns/05_passing_arguments/msvc_JPN.asm}

\myindex{x86!\Registers!EBP}

\main 関数は3つの数値をスタックにプッシュし、\TT{f(int,int,int)}を呼び出すことがわかります。

\ttf 内の引数アクセスは、ローカル変数と同じ方法で
\TT{\_a\$ = 8}
のようなマクロの助けを借りて構成されますが、正のオフセット(\emph{プラス}で扱われます)を持ちます。 
したがって、\TT{\_a\$}マクロを \EBP レジスタの値に追加することによって\gls{stack frame}の\emph{外側}を処理しています。

\myindex{x86!\Instructions!IMUL}
\myindex{x86!\Instructions!ADD}

次に、 $a$ の値が \EAX に格納されます。 \IMUL 命令実行後、 
\EAX の値は \EAX の値と\TT{\_b}の内容の\gls{product}です。

その後、 \ADD は\TT{\_c}の値を \EAX に追加します。

\EAX の値は移動する必要はありません。すでに存在している必要があります。 
\gls{caller}に戻ると、 \EAX 値をとり、 \printf の引数として使用します。

\clearpage
\myparagraph{\Optimizing MSVC + \olly}
\myindex{\olly}

\olly でこの(最適化された)例を試すことができます。ここに最初のイテレーションがあります:

\begin{figure}[H]
\centering
\myincludegraphics{patterns/10_strings/1_strlen/olly1.png}
\caption{\olly: 最初のイテレーションの開始}
\label{fig:strlen_olly_1}
\end{figure}

\olly がループを見つけ、便宜上、その指示を括弧で\emph{囲んだ}ことがわかります。 
\EAX の右ボタンをクリックして、\q{Follow in Dump}を選択すると、メモリウィンドウが正しい場所にスクロールします。
ここではメモリ内に \q{hello!}という文字列があります。
その後に少なくとも1つのゼロバイトがあり、次にランダムなごみがあります。

\olly が有効なアドレスを持つレジスタを見ると、それは文字列を指しており、
文字列として表示されます。

\clearpage
F8(\stepover)を数回押して、ループの本体の先頭に移動しましょう:

\begin{figure}[H]
\centering
\myincludegraphics{patterns/10_strings/1_strlen/olly2.png}
\caption{\olly: 2回目のイテレーションの開始}
\label{fig:strlen_olly_2}
\end{figure}

\EAX には文字列中の2番目の文字のアドレスが含まれていることがわかります。

\clearpage

我々はループから脱出するためにF8を十分な回数押す必要があります:

\begin{figure}[H]
\centering
\myincludegraphics{patterns/10_strings/1_strlen/olly3.png}
\caption{\olly: 計算すべきポインタの差}
\label{fig:strlen_olly_3}
\end{figure}

% FIXME:
\EAX には文字列の直後に0バイトのアドレスが含まれることがわかりました。一方、
\EDX は変更されていないので、まだ文字列の先頭を指しています。

これらの2つのアドレスの差がここで計算されています。

\clearpage
\SUB 命令が実行されました。

\begin{figure}[H]
\centering
\myincludegraphics{patterns/10_strings/1_strlen/olly4.png}
\caption{\olly: \EAX がデクリメントされる}
\label{fig:strlen_olly_4}
\end{figure}

ポインタの違いは \EAX レジスタにあります(値は7)。
確かに、 \q{hello!}文字列の長さは6ですが、7にはゼロバイトが含まれています。
しかし、\TT{strlen()}は文字列中のゼロ以外の文字数を返さなければなりません。
したがって、デクリメントが実行され、関数が戻ります。


\subsubsection{GCC}

GCC 4.4.1で同じものをコンパイルし、 \IDA の結果を見てみましょう。

\lstinputlisting[caption=GCC 4.4.1,style=customasmx86]{patterns/05_passing_arguments/gcc_JPN.asm}

結果はほぼ同じで、以前に説明したいくつかの小さな違いがあります。

\gls{stack pointer}は2つの関数呼び出し(fとprintf)の後にセットバックされません。
最後から2番目の\TT{LEAVE}命令(\myref{x86_ins:LEAVE})
命令が最後にこれを処理するためです。
