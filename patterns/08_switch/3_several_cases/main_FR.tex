\subsection{Lorsqu'il y a quelques déclarations \emph{case} dans un bloc}

Voici une construction très répandue: quelques déclarations \emph{case} pour un seul bloc:

\lstinputlisting[style=customc]{patterns/08_switch/3_several_cases/several_cases.c}

C'est souvent du gaspillage de générer un bloc pour chaque cas possible, c'est
pourquoi ce qui se fait d'habitude, c'est de générer un bloc et une sorte de répartiteur.

\subsubsection{MSVC}

\lstinputlisting[caption=MSVC 2010 \Optimizing,numbers=left,style=customasmx86]{patterns/08_switch/3_several_cases/several_cases_MSVC_2010_Ox_FR.asm}

Nous voyons deux tables ici: la première (\TT{\$LN10@f}) est une table d'index,
et la seconde (\TT{\$LN11@f}) est un tableau de pointeurs sur les blocs.

Tout d'abord, la valeur entrée est utilisée comme un index dans la table d'index
(ligne 13).

Voici un petit récapitulatif pour les valeurs dans la table:
0 est le premier bloc \emph{case} (pour les valeurs 1, 2, 7, 10),
1 est le second (pour les valeurs 3, 4, 5),
2 est le troisième (pour les valeurs 8, 9, 21),
3 est le quatrième (pour la valeur 22),
4 est pour le bloc par défaut.

Ici, nous obtenons un index pour la seconde table de pointeurs sur du code et nous
y sautons (ligne 14).

Il est intéressant de remarquer qu'il n'y a pas de cas pour une valeur d'entrée
de 0.

C'est pourquoi nous voyons l'instruction \DEC à la ligne 10, et la table commence
à $a=1$, car il n'y a pas besoin d'allouer un élément dans la table pour $a=0$.

C'est un pattern très répandu.

Donc, pourquoi est-ce que c'est économique ?
Pourquoi est-ce qu'il n'est pas possible de faire comme avant (\myref{switch_lot_GCC}),
avec une seule table consistant en des pointeurs vers les blocs?
La raison est que les index des éléments de la table sont 8-bit, donc c'est plus
compact.

\subsubsection{GCC}

GCC génère du code de la façon dont nous avons déjà discuté (\myref{switch_lot_GCC}),
en utilisant juste une table de pointeurs.

\subsubsection{ARM64: GCC 4.9.1 \Optimizing}

Il n'y a pas de code à exécuter si la valeur entrée est 0, c'est pourquoi GCC
essaye de rendre la table des sauts plus compacte et donc il commence avec la
valeur d'entrée 1.

GCC 4.9.1 pour ARM64 utilise un truc encore plus astucieux.
Il est capable d'encoder tous les offsets en octets 8-bit.

Rappelons-nous que toutes les instructions ARM64 ont une taille de 4 octets.

GCC utilise le fait que tous les offsets de mon petit exemple sont tous proche l'un
de l'autre. Donc la table des sauts consiste en de simple octets.

\lstinputlisting[caption=\Optimizing GCC 4.9.1 ARM64,style=customasmARM]{patterns/08_switch/3_several_cases/ARM64_GCC491_O3_FR.s}

Compilons cet exemple en un fichier objet et ouvrons-le dans \IDA. Voici la table
des sauts:

\lstinputlisting[caption=jumptable in IDA,style=customasmARM]{patterns/08_switch/3_several_cases/ARM64_GCC491_O3_IDA.lst}

Donc dans le cas de 1, 9 est multiplié par 4 et ajouté à l'adresse
du label \TT{Lrtx4}.

Dans le cas de 22, 0 est multiplié par 4, ce qui donne 0.

Juste après le label \TT{Lrtx4} se trouve le label \TT{L7}, où se trouve le code
qui affiche \q{22}.

Il n'y a pas de table des sauts dans le segment de code, elle est allouée dans
la section .rodata (il n'y a pas de raison de l'allouer dans le segment de code).

Il y a aussi des octets négatifs (0xF7), ils sont utilisés pour sauter en arrière
dans le code qui affiche la chaîne \q{default} (en \TT{.L2}).

