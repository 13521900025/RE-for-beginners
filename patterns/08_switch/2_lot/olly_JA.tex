\clearpage
\mysubparagraph{\olly}
\myindex{\olly}

\olly でこの例を試してみましょう。
関数への入力値2が \EAX にロードされます。

\begin{figure}[H]
\centering
\myincludegraphics{patterns/08_switch/2_lot/olly1.png}
\caption{\olly: 関数への入力値が \EAX にロードされる}
\label{fig:switch_lot_olly1}
\end{figure}

\clearpage
入力値が4より大きいかチェックされます。
そうでなければ、\q{default} ジャンプは実行されません。

\begin{figure}[H]
\centering
\myincludegraphics{patterns/08_switch/2_lot/olly2.png}
\caption{\olly: 2は4より大きいか:ジャンプは実行されない}
\label{fig:switch_lot_olly2}
\end{figure}

\clearpage
ここで、ジャンプテーブルを見ることができます。

\begin{figure}[H]
\centering
\myincludegraphics{patterns/08_switch/2_lot/olly3.png}
\caption{\olly: ジャンプテーブルを用いて行先のアドレスを計算する}
\label{fig:switch_lot_olly3}
\end{figure}

ここで、\q{Follow in Dump} $\rightarrow$ \q{Address constant} をチェックします。そして、データウィンドウに\emph{jumptable}が見えます。
5つの32ビット値があります。\footnote{これらはまた要修正\myref{subsec:relocs}であるため、 \olly で下線が引かれています。後でそれらに戻ってくるつもりです}
\ECX は2になりました。したがって、テーブルの3番目の要素(2として\footnote{インデックスについては以下を参照: \ref{arrays_at_one}}索引付けできます)が使用されます。
\q{Follow in Dump} $\rightarrow$ \q{Memory address}をクリックすることができ、
\olly は \JMP 命令で指示された要素を表示します。
それは\TT{0x010B103A}です。

\clearpage
ジャンプの後、\TT{0x010B103A}にいます。\q{two}を表示するコードが実行されます。

\begin{figure}[H]
\centering
\myincludegraphics{patterns/08_switch/2_lot/olly4.png}
\caption{\olly: 今や \emph{case:} ラベルにいます}
\label{fig:switch_lot_olly4}
\end{figure}
