\subsubsection{x86}

\myparagraph{\NonOptimizing MSVC}

Wir erhalten (MSVC 2010):

\lstinputlisting[caption=MSVC 2010,style=customasmx86]{patterns/08_switch/2_lot/lot_msvc_DE.asm}

\myindex{jumptable}
Was wir hier sehen ist eine Ansammlung von Aufrufen von \printf mit diversen Argumenten.
Alle haben nicht Adressen im Speicher des Prozesses, sondern auch interne symbolische Labels, die ihnen vom Compiler
zugewiesen werden.
Alle diese Labels werden auch in der internen Tabelle \TT{\$LN11@f} aufgeführt. 

Zu Beginn der Funktion wird der Control Flow an das Label \TT{\$LN1@f} abgegeben, wenn $a$ größer ist als 4. An diesem
Label wird \printf mit dem Argument \TT{'something unknown'} aufgerufen.

Wenn aber der Wert von $a$ kleiner gleich 4 ist, dann wird dieser mit 4 multipliziert und mit der Tabellenadresse
\TT{\$LN11@f} addiert. Auf diese Weise wird die Adresse innerhalb der Tabelle konstruiert und zeigt genau auf das
gewünschte Element. Nehmen wir zum Beispiel an, dass $a$ gleich 2 ist. $2\cdot 4=8$ (alle Tabellenelemente sind
Adressen in einem 32-Bit-Prozess und haben daher eine Breite von 4 Bytes).
Die Adresse an der Stelle \TT{\$LN11@f} + 8 ist das Tabellenelement, an dem das Label \TT{\$LN4@f} gespeichert ist.
\JMP holt die Adresse \TT{\$LN4@f} aus der Tabelle und springt dorthin.

Diese Tabelle wird manchmal \emph{Jumptable} oder \emph{Verzweigungstabelle} genannt\footnote{Die ganze Methode wurde
in früheren Versionen von Fortran \emph{berechnetes GOTO} genannt:
\href{http://go.yurichev.com/17122}{wikipedia}.
Heutzutage zwar nicht mehr relevant, aber welch ein Ausdruck!}.

Dann wird das zugehörige \printf mit dem Argument \TT{'two'} aufgerufen.\\
Der Befehl TT{jmp DWORD PTR \$LN11@f[ecx*4]} bedeutet dabei \emph{springe zum an dieser Stelle gespeicherten
DWORD}\TT{\$LN11@f + ecx * 4}.

\TT{npad} (\myref{sec:npad}) ist ein Assemblermakro, dass das nächste Label so angeordnet, dass es an einer 4 Byte
(oder 16 Bit) Adressgrenze gespeichert wird. Das ist für den Prozessor sehr praktisch, da er die 32-Bit-Werte aus dem
Speicher durch den Speicherbus, den Cache, etc. in effektiverer Weise laden kann.

\clearpage
\mysubparagraph{\olly}
\myindex{\olly}

Untersuchen wir das Beispiel in \olly.
Der Eingabewert der Funktion (2) wird nach \EAX geladen: 

\begin{figure}[H]
\centering
\myincludegraphics{patterns/08_switch/2_lot/olly1.png}
\caption{\olly: das Funktionsargument wird nach \EAX geladen}
\label{fig:switch_lot_olly1}
\end{figure}

\clearpage
Es wird geprüft, ob der Eingabewert größer als 4 ist.
Falls nicht, wird der \q{default} Sprung nicht ausgeführt:
\begin{figure}[H]
\centering
\myincludegraphics{patterns/08_switch/2_lot/olly2.png}
\caption{\olly: 2 ist nicht größer als 4: kein Sprung wird ausgeführt}
\label{fig:switch_lot_olly2}
\end{figure}

\clearpage
Hier sehen wir eine Jumptable:

\begin{figure}[H]
\centering
\myincludegraphics{patterns/08_switch/2_lot/olly3.png}
\caption{\olly: Zieladresse mit Jumptable berechnen}
\label{fig:switch_lot_olly3}
\end{figure}
Wir haben \q{Follow in Dump}$\rightarrow$ \q{Address constant} geklickt, sodass wir jetzt die \emph{Jumptable} im
Datenfenster sehen. Hier sind 5 32-Bit-Werte\footnote{Diese werden von \olly unterstrichen, da
es auch FIXUPs sind: \myref{subsec:relocs}, wir kommen später darauf zurück}.
\ECX ist jetzt 2, sodass das zweite Element (beginnend bei null) der Tabelle verwendet wird.
% TBT
Es ist auch möglich durch Klicken auf q{Follow in Dump} $\rightarrow$ 
\q{Memory address} in \olly das Element, das durch den \JMP Befehl angesteuert wird, anzeigen zu lassen. Dieses Element
ist hier \TT{0x010B103A}.

\clearpage
Nach dem Sprung sind wir an der Stelle \TT{0x010B103A}: der Code zur Ausgabe von \q{two} wird jetzt ausgeführt:

\begin{figure}[H]
\centering
\myincludegraphics{patterns/08_switch/2_lot/olly4.png}
\caption{\olly: jetzt sind wir am \emph{case:} Label}
\label{fig:switch_lot_olly4}
\end{figure}


\myparagraph{\NonOptimizing GCC}
\label{switch_lot_GCC}

Schauen wir was GCC 4.4.1 erzeugt:

\lstinputlisting[caption=GCC 4.4.1,style=customasmx86]{patterns/08_switch/2_lot/lot_gcc.asm}

\myindex{x86!\Registers!JMP}
Es ist bis auf eine Nuance das gleiche: das Argument \TT{arg\_0} wird mit 4 multipliziert durch eine Verschiebung von 2
Bits nach links (dies entspricht einer Multiplikation mit 4)~(\myref{SHR}).
Dann wird die Adresse des Labels vom \TT{off\_804855C} genommen, die in \EAX gespeichert wird, und dann wird mit
\TT{JMP EAX} der eigentliche Sprung durchgeführt.


