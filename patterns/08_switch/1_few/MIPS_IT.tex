\subsubsection{MIPS}

\lstinputlisting[caption=\Optimizing GCC 4.4.5 (IDA),style=customasmMIPS]{patterns/08_switch/1_few/MIPS_O3_IDA_EN.lst}

\myindex{MIPS!\Instructions!JR}

La funzione finisce sempre col chiamare \puts, e quindi qui vediamo un jump a \puts (\INS{JR}: \q{Jump Register}) invece di \q{jump and link}.
Abbiamo già discusso questo argomento qui: \myref{JMP_instead_of_RET}.

\myindex{MIPS!Load delay slot}
Vediamo anche spesso delle istruzioni \INS{NOP} dopo le istruzioni \INS{LW}.
Si tratta di \q{load delay slot}: un altro tipo di \emph{delay slot} in MIPS.
\myindex{MIPS!\Instructions!LW}

Un'istruzione immediatamente successiva a \INS{LW} potrebbe essere eseguita mentre \INS{LW} carica il valore dalla memoria.
Questa istruzione, comunque, non deve usare il risultato di \INS{LW}.

Le moderne CPU MIPS hanno una funzionalità che consente di attendere, nel caso in cui l'istruzione successiva usi il risultato di 
\INS{LW}, peranto questo tipo codice è piuttosto antiquato e può essere ignorato. GCC continua ad aggiungere i NOP a favore delle cpu MIPS più vecchie.

