\subsubsection{x86}

\myparagraph{\NonOptimizing MSVC}

結果 (MSVC 2010):

\lstinputlisting[caption=MSVC 2010,style=customasmx86]{patterns/08_switch/1_few/few_msvc.asm}

実際、switch()でいくつかのcaseを持つ私たちの関数は、この構造に似ています。

\lstinputlisting[label=switch_few_ifelse,style=customc]{patterns/08_switch/1_few/few_analogue.c}

\myindex{\CLanguageElements!switch}
\myindex{\CLanguageElements!if}

いくつかのcaseでswitch()を使用する場合、ソースコード内の実際のswitch()か、
単にif文の組であるかどうかを確認することは不可能です。
\myindex{\SyntacticSugar}

これはswitch()が多段にネストされたif文との糖衣構文のようなものであることを意味します。

コンパイラが入力変数 $a$ を一時的なローカル変数\TT{tv64}に移動することを除いて、
生成されたコードには特に新しいことはありません。
\footnote{スタック内のローカル変数には接頭辞\TT{tv}が付きます。MSVCが内部変数として使用するために命名しています。}

これをGCC 4.4.1でコンパイルすると、最大限の最適化(\Othree option)を有効にしても
ほぼ同じ結果になります。

\myparagraph{\Optimizing MSVC}

% TODO separate various kinds of \TT
% idea: enclose command lines in a specific environment, like \cmdline{} 
% assembly instructions in \asm{} (now both \TT and \q{} are used),
% variables in,  like \var{}
% messages (string constants) in something else, like \strconst
% to separate them all. Now they all use \TT, which is not best
% \INS{} for all instructions including operands? --DY
では、MSVC(\Ox)の最適化を有効にしましょう:\TT{cl 1.c /Fa1.asm /Ox}

\label{JMP_instead_of_RET}
\lstinputlisting[caption=MSVC,style=customasmx86]{patterns/08_switch/1_few/few_msvc_Ox.asm}

ここで、汚いハックを見ることができます。

\myindex{x86!\Instructions!JZ}
\myindex{x86!\Instructions!JE}
\myindex{x86!\Instructions!SUB}

最初に、 $a$ の値を \EAX に置き、0を引きます。 EAXの値が0かどうかを確認するために行われますが、
そうであれば、 \ZF フラグがセットされます(例えば、0からの減算は0)
最初の条件ジャンプ \JE (\emph{Jump if Equal} またはあ同義語 \JZ~---\emph{Jump if Zero})は実行され、
制御フローは\TT{\$LN4@f}ラベルに渡されます。ここでは、 \TT{'zero'}メッセージが出力されます。
最初のジャンプが実行されない場合は、入力値から1が減算され、結果が0の場合、対応するジャンプが実行されます。

また、ジャンプが全く実行されない場合、制御フローは文字列引数\TT{'something unknown'}を \printf に渡します。

\label{jump_to_last_printf}
\myindex{\Stack}

次に、文字列ポインタが $a$ 変数に置かれ、 \printf が \CALL ではなく \JMP を介して呼び出されます。 
簡単に説明するとこうなります:
\gls{caller} は値をスタックにプッシュし、 \CALL 経由で関数を呼び出します。
\CALL 自体は戻りアドレス(\ac{RA})をスタックにプッシュし、関数アドレスへの無条件ジャンプを行います。
スタックポインタを移動させる命令が含まれていないため、任意の実行時点での関数は、次のスタックレイアウトを持ちます。

\begin{itemize}
\item\ESP---points to \ac{RA}
\item\TT{ESP+4}---points to the $a$ variable 
\end{itemize}

反対に、\printf をここで呼び出さなければならないときは、文字列を指し示す必要がある最初の\printf 引数を除いて、
全く同じスタックレイアウトが必要です。それが私たちのコードがすることです。

ファンクションの最初の引数を文字列のアドレスに置き換え、
関数 \ttf を直接呼び出しずに直接 \printf を呼び出すかのように、 \printf にジャンプします。
\printf は文字列を \gls{stdout} に出力し、 \RET 命令を実行します。スタックから\ac{RA}を取り出し、
制御フローは \ttf ではなく \ttf 関数の終りをバイパスして、 \ttf の \gls{caller} です。

\myindex{\CStandardLibrary!longjmp()}
\newcommand{\URLSJ}{\href{http://go.yurichev.com/17121}{wikipedia}}

% TODO \myref{}
\printf はすべての場合に  \ttf 関数の終わりで右に呼ばれるので、これはすべて可能です。
ある意味では、\TT{longjmp()}\footnote{\URLSJ}関数に似ています。
そしてもちろん、それはスピードのためにすべて行われます。

ARMコンパイラと同様のケースは、\q{\PrintfSeveralArgumentsSectionName}セクションに記載されています。
こちら:~(\myref{ARM_B_to_printf})

\clearpage
\myparagraph{\Optimizing MSVC + \olly}
\myindex{\olly}

\olly でこの(最適化された)例を試すことができます。ここに最初のイテレーションがあります:

\begin{figure}[H]
\centering
\myincludegraphics{patterns/10_strings/1_strlen/olly1.png}
\caption{\olly: 最初のイテレーションの開始}
\label{fig:strlen_olly_1}
\end{figure}

\olly がループを見つけ、便宜上、その指示を括弧で\emph{囲んだ}ことがわかります。 
\EAX の右ボタンをクリックして、\q{Follow in Dump}を選択すると、メモリウィンドウが正しい場所にスクロールします。
ここではメモリ内に \q{hello!}という文字列があります。
その後に少なくとも1つのゼロバイトがあり、次にランダムなごみがあります。

\olly が有効なアドレスを持つレジスタを見ると、それは文字列を指しており、
文字列として表示されます。

\clearpage
F8(\stepover)を数回押して、ループの本体の先頭に移動しましょう:

\begin{figure}[H]
\centering
\myincludegraphics{patterns/10_strings/1_strlen/olly2.png}
\caption{\olly: 2回目のイテレーションの開始}
\label{fig:strlen_olly_2}
\end{figure}

\EAX には文字列中の2番目の文字のアドレスが含まれていることがわかります。

\clearpage

我々はループから脱出するためにF8を十分な回数押す必要があります:

\begin{figure}[H]
\centering
\myincludegraphics{patterns/10_strings/1_strlen/olly3.png}
\caption{\olly: 計算すべきポインタの差}
\label{fig:strlen_olly_3}
\end{figure}

% FIXME:
\EAX には文字列の直後に0バイトのアドレスが含まれることがわかりました。一方、
\EDX は変更されていないので、まだ文字列の先頭を指しています。

これらの2つのアドレスの差がここで計算されています。

\clearpage
\SUB 命令が実行されました。

\begin{figure}[H]
\centering
\myincludegraphics{patterns/10_strings/1_strlen/olly4.png}
\caption{\olly: \EAX がデクリメントされる}
\label{fig:strlen_olly_4}
\end{figure}

ポインタの違いは \EAX レジスタにあります(値は7)。
確かに、 \q{hello!}文字列の長さは6ですが、7にはゼロバイトが含まれています。
しかし、\TT{strlen()}は文字列中のゼロ以外の文字数を返さなければなりません。
したがって、デクリメントが実行され、関数が戻ります。


