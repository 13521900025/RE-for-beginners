\subsubsection{x86}

\myparagraph{\NonOptimizing MSVC}

Это дает в итоге (MSVC 2010):

\lstinputlisting[caption=MSVC 2010,style=customasmx86]{patterns/08_switch/1_few/few_msvc.asm}

Наша функция с оператором switch(), с небольшим количеством вариантов, 
это практически аналог подобной конструкции:

\lstinputlisting[label=switch_few_ifelse,style=customc]{patterns/08_switch/1_few/few_analogue.c}

\myindex{\CLanguageElements!switch}
\myindex{\CLanguageElements!if}
Когда вариантов немного и мы видим подобный код, невозможно сказать с уверенностью, был ли
в оригинальном исходном коде switch(), либо просто набор операторов if().

\myindex{\SyntacticSugar}
То есть, switch() это синтаксический сахар для большого количества вложенных проверок 
при помощи if().

В самом выходном коде ничего особо нового, 
за исключением того, что компилятор зачем-то 
перекладывает входящую переменную ($a$) во временную в локальном стеке \TT{v64}\footnote{Локальные переменные в стеке с префиксом \TT{tv}~--- 
так MSVC называет внутренние переменные для своих нужд}.

Если скомпилировать это при помощи GCC 4.4.1, то будет почти то же самое, даже с максимальной оптимизацией (ключ \Othree).

\myparagraph{\Optimizing MSVC}

% TODO separate various kinds of \TT
% idea: enclose command lines in a specific environment, like \cmdline{} 
% assembly instructions in \asm{} (now both \TT and \q{} are used),
% variables in,  like \var{}
% messages (string constants) in something else, like \strconst
% to separate them all. Now they all use \TT, which is not best
% \INS{} for all instructions including operands? --DY

Попробуем включить оптимизацию кодегенератора MSVC (\Ox): \TT{cl 1.c /Fa1.asm /Ox}

\label{JMP_instead_of_RET}
\lstinputlisting[caption=MSVC,style=customasmx86]{patterns/08_switch/1_few/few_msvc_Ox.asm}

Вот здесь уже всё немного по-другому, причем не без грязных трюков.

\myindex{x86!\Instructions!JZ}
\myindex{x86!\Instructions!JE}
\myindex{x86!\Instructions!SUB}
Первое: \TT{а} помещается в \EAX и от него отнимается 0. Звучит абсурдно, но нужно это для того, чтобы проверить, 
0 ли в \EAX был до этого? Если да, то выставится флаг \ZF (что означает, что результат вычитания 0 от числа 
стал 0) и первый условный переход \JE (\emph{Jump if Equal} или его синоним \JZ~--- \emph{Jump if Zero}) 
сработает на метку \TT{\$LN4@f}, где выводится сообщение \TT{'zero'}.
Если первый переход не сработал, от значения отнимается по единице, 
и если на какой-то стадии в результате образуется 0, то сработает соответствующий переход.

И в конце концов, если ни один из условных переходов не сработал, управление передается \printf
со строковым аргументом \TT{'something unknown'}.

\label{jump_to_last_printf}
\myindex{\Stack}
Второе: мы видим две, мягко говоря, необычные вещи: указатель на сообщение помещается в переменную $a$, 
и затем \printf вызывается не через \CALL, а через \JMP. Объяснение этому простое. 
Вызывающая функция заталкивает в стек некоторое значение и через \CALL вызывает нашу функцию. 
\CALL в свою очередь заталкивает в стек адрес возврата (\ac{RA}) и делает безусловный переход на адрес нашей функции. 
Наша функция в самом начале (да и в любом её месте, потому что в теле функции нет ни одной инструкции, 
которая меняет что-то в стеке или в \ESP) имеет следующую разметку стека:

\begin{itemize}
\item\ESP --- хранится \ac{RA}
\item\TT{ESP+4} --- хранится значение $a$ 
\end{itemize}

С другой стороны, чтобы вызвать \printf, нам нужна почти такая же разметка стека, 
только в первом аргументе нужен указатель на строку. Что, собственно, этот код и делает.

Он заменяет свой первый аргумент на адрес строки, и затем передает управление \printf, как если бы вызвали не 
нашу функцию \ttf, а сразу \printf. 
\printf выводит некую строку на \gls{stdout}, затем исполняет инструкцию \RET, 
которая из стека достает \ac{RA} и управление передается в ту функцию, 
которая вызывала \ttf, минуя при этом конец функции \ttf.

\myindex{\CStandardLibrary!longjmp()}
\newcommand{\URLSJ}{\href{http://go.yurichev.com/17121}{wikipedia}}
% TODO \myref{}
Всё это возможно, потому что \printf вызывается в \ttf в самом конце. 
Всё это чем-то даже похоже на \TT{longjmp()}\footnote{\URLSJ}.
И всё это, разумеется, сделано для экономии времени исполнения.

Похожая ситуация с компилятором для ARM описана в секции \q{\PrintfSeveralArgumentsSectionName}~(\myref{ARM_B_to_printf}).

\clearpage
\subsubsection{MSVC + \olly}
\myindex{\olly}

Попробуем этот же пример в \olly.
Загружаем, нажимаем F8 (\stepover) до тех пор, пока не окажемся в своем исполняемом файле,
а не в \TT{ntdll.dll}.
Прокручиваем вверх до тех пор, пока не найдем \main.
Щелкаем на первой инструкции (\TT{PUSH EBP}), нажимаем F2 (\emph{set a breakpoint}), 
затем F9 (\emph{Run}) и точка останова срабатывает на начале \main.

Трассируем до того места, где готовится адрес переменной $x$:

\begin{figure}[H]
\centering
\myincludegraphics{patterns/04_scanf/1_simple/ex1_olly_1.png}
\caption{\olly: вычисляется адрес локальной переменной}
\label{fig:scanf_ex1_olly_1}
\end{figure}

На \EAX в окне регистров можно нажать правой кнопкой и далее выбрать \q{Follow in stack}.
Этот адрес покажется в окне стека.

Смотрите, это переменная в локальном стеке. Там дорисована красная стрелка.
И там сейчас какой-то мусор (\TT{0x6E494714}).
Адрес этого элемента стека сейчас, при помощи \PUSH запишется в этот же стек рядом.
Трассируем при помощи F8 вплоть до конца исполнения \scanf.
А пока \scanf исполняется, в консольном окне, вводим, например, 123:

\lstinputlisting{patterns/04_scanf/1_simple/console.txt}

\clearpage
Вот тут \scanf отработал:

\begin{figure}[H]
\centering
\myincludegraphics{patterns/04_scanf/1_simple/ex1_olly_3.png}
\caption{\olly: \scanf исполнилась}
\label{fig:scanf_ex1_olly_3}
\end{figure}

\scanf вернул 1 в \EAX, что означает, что он успешно прочитал одно значение.
В наблюдаемом нами элементе стека теперь \TT{0x7B} (123).

\clearpage
Чуть позже это значение копируется из стека в регистр \ECX и передается в \printf{}:

\begin{figure}[H]
\centering
\myincludegraphics{patterns/04_scanf/1_simple/ex1_olly_4.png}
\caption{\olly: готовим значение для передачи в \printf}
\label{fig:scanf_ex1_olly_4}
\end{figure}


