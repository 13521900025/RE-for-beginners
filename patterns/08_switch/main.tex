\mysection{\SwitchCaseDefaultSectionName}
\myindex{\CLanguageElements!switch}

% sections
\subsection{\RU{Если вариантов мало}\EN{Small number of cases}\DE{Kleine Anzahl von Fällen}\FR{Petit nombre de cas}\IT{Pochi casi}\JA{小さな数のcase}}

\lstinputlisting[style=customc]{patterns/08_switch/1_few/few.c}

\EN{\subsubsection{x86}

\myparagraph{\NonOptimizing MSVC}

Result (MSVC 2010):

\lstinputlisting[caption=MSVC 2010,style=customasmx86]{patterns/08_switch/1_few/few_msvc.asm}

Our function with a few cases in switch() is in fact analogous to this construction:

\lstinputlisting[label=switch_few_ifelse,style=customc]{patterns/08_switch/1_few/few_analogue.c}

\myindex{\CLanguageElements!switch}
\myindex{\CLanguageElements!if}

If we work with switch() with a few cases it is impossible to be sure if it was
a real switch() in the source code, or just a pack of if() statements.
\myindex{\SyntacticSugar}

This implies that switch() is like syntactic sugar for a large number of nested if()s.

There is nothing especially new to us in the generated code,
with the exception of the compiler moving input variable $a$ to a temporary local variable \TT{tv64}
\footnote{Local variables in stack are prefixed with \TT{tv}---that's how MSVC names internal variables for its needs}.

If we compile this in GCC 4.4.1, we'll get almost the same result, even with maximal optimization
turned on (\Othree option).

\myparagraph{\Optimizing MSVC}

% TODO separate various kinds of \TT
% idea: enclose command lines in a specific environment, like \cmdline{} 
% assembly instructions in \asm{} (now both \TT and \q{} are used),
% variables in,  like \var{}
% messages (string constants) in something else, like \strconst
% to separate them all. Now they all use \TT, which is not best
% \INS{} for all instructions including operands? --DY
Now let's turn on optimization in MSVC (\Ox): \TT{cl 1.c /Fa1.asm /Ox}

\label{JMP_instead_of_RET}
\lstinputlisting[caption=MSVC,style=customasmx86]{patterns/08_switch/1_few/few_msvc_Ox.asm}

Here we can see some dirty hacks.

\myindex{x86!\Instructions!JZ}
\myindex{x86!\Instructions!JE}
\myindex{x86!\Instructions!SUB}

First: the value of $a$ is placed in \EAX and 0 is subtracted from it. Sounds absurd, but it is done to check if 
the value in \EAX is 0. If yes, the \ZF flag is to be set (e.g. subtracting from 0 is 0) 
and the first conditional jump \JE (\emph{Jump if Equal} or synonym \JZ~---\emph{Jump if Zero}) is to be triggered 
and control flow is to be passed to the \TT{\$LN4@f} label, where the \TT{'zero'} message is being printed. 
If the first jump doesn't get triggered, 1 is subtracted from the input value and if at some stage the result is 0, 
the corresponding jump is to be triggered.

And if no jump gets triggered at all, the control flow passes to \printf with string argument \\
\TT{'something unknown'}.

\label{jump_to_last_printf}
\myindex{\Stack}

Second: we see something unusual for us: a string pointer is placed into the $a$ variable, and 
then \printf is called not via \CALL, but via \JMP. There is a simple explanation for that: 
the \gls{caller} pushes a value to the stack and calls our function via \CALL. 
\CALL itself pushes the return address (\ac{RA}) to the stack and does an unconditional jump to our function address. 
Our function at any point of execution (since it do not contain any instruction that moves the stack 
pointer) has the following stack layout:

\begin{itemize}
\item\ESP---points to \ac{RA}
\item\TT{ESP+4}---points to the $a$ variable 
\end{itemize}

On the other side, when we have to call \printf here we need exactly the same stack 
layout, except for the first \printf argument, which needs to point to the string. 
And that is what our code does.

It replaces the function's first argument with the address of the string and 
jumps to \printf, as if we didn't call our function \ttf, but directly \printf.
\printf prints a string to \gls{stdout} and then executes the \RET instruction, which POPs 
\ac{RA} from the stack and control flow is returned not to \ttf but rather to \ttf's \gls{caller}, 
bypassing the end of the \ttf function.

\myindex{\CStandardLibrary!longjmp()}
\newcommand{\URLSJ}{\href{http://go.yurichev.com/17121}{wikipedia}}

% TODO \myref{}
All this is possible because \printf is called right at the end of the \ttf function in all cases. 
In some way, it is similar to the \TT{longjmp()}\footnote{\URLSJ} function.
And of course, it is all done for the sake of speed.

A similar case with the ARM compiler is described in \q{\PrintfSeveralArgumentsSectionName}
section, here~(\myref{ARM_B_to_printf}).

\clearpage
\subsubsection{MSVC: x86 + \olly}
\myindex{\olly}

Things are even simpler here:

\begin{figure}[H]
\centering
\myincludegraphics{patterns/04_scanf/2_global/ex2_olly_1.png}
\caption{\olly: after \scanf execution}
\label{fig:scanf_ex2_olly_1}
\end{figure}

The variable is located in the data segment.
After the \PUSH instruction (pushing the address of $x$) gets executed, 
the address appears in the stack window. Right-click on that row and select \q{Follow in dump}.
The variable will appear in the memory window on the left.
After we have entered 123 in the console, 
\TT{0x7B} appears in the memory window (see the highlighted screenshot regions).

But why is the first byte \TT{7B}?
Thinking logically, \TT{00 00 00 7B} must be there.
The cause for this is referred as  \gls{endianness}, and x86 uses \emph{little-endian}.
This implies that the lowest byte is written first, and the highest written last.
Read more about it at: \myref{sec:endianness}.
Back to the example, the 32-bit value is loaded from this memory address into \EAX and passed to \printf.

The memory address of $x$ is \TT{0x00C53394}.

\clearpage
In \olly we can review the process memory map (Alt-M)
and we can see that this address is inside the \TT{.data} PE-segment of our program:

\label{olly_memory_map_example}
\begin{figure}[H]
\centering
\myincludegraphics{patterns/04_scanf/2_global/ex2_olly_2.png}
\caption{\olly: process memory map}
\label{fig:scanf_ex2_olly_2}
\end{figure}



}
\RU{\subsubsection{x86}

\myparagraph{\NonOptimizing MSVC}

Это дает в итоге (MSVC 2010):

\lstinputlisting[caption=MSVC 2010,style=customasmx86]{patterns/08_switch/1_few/few_msvc.asm}

Наша функция с оператором switch(), с небольшим количеством вариантов, 
это практически аналог подобной конструкции:

\lstinputlisting[label=switch_few_ifelse,style=customc]{patterns/08_switch/1_few/few_analogue.c}

\myindex{\CLanguageElements!switch}
\myindex{\CLanguageElements!if}
Когда вариантов немного и мы видим подобный код, невозможно сказать с уверенностью, был ли
в оригинальном исходном коде switch(), либо просто набор операторов if().

\myindex{\SyntacticSugar}
То есть, switch() это синтаксический сахар для большого количества вложенных проверок 
при помощи if().

В самом выходном коде ничего особо нового, 
за исключением того, что компилятор зачем-то 
перекладывает входящую переменную ($a$) во временную в локальном стеке \TT{v64}\footnote{Локальные переменные в стеке с префиксом \TT{tv}~--- 
так MSVC называет внутренние переменные для своих нужд}.

Если скомпилировать это при помощи GCC 4.4.1, то будет почти то же самое, даже с максимальной оптимизацией (ключ \Othree).

\myparagraph{\Optimizing MSVC}

% TODO separate various kinds of \TT
% idea: enclose command lines in a specific environment, like \cmdline{} 
% assembly instructions in \asm{} (now both \TT and \q{} are used),
% variables in,  like \var{}
% messages (string constants) in something else, like \strconst
% to separate them all. Now they all use \TT, which is not best
% \INS{} for all instructions including operands? --DY

Попробуем включить оптимизацию кодегенератора MSVC (\Ox): \TT{cl 1.c /Fa1.asm /Ox}

\label{JMP_instead_of_RET}
\lstinputlisting[caption=MSVC,style=customasmx86]{patterns/08_switch/1_few/few_msvc_Ox.asm}

Вот здесь уже всё немного по-другому, причем не без грязных трюков.

\myindex{x86!\Instructions!JZ}
\myindex{x86!\Instructions!JE}
\myindex{x86!\Instructions!SUB}
Первое: \TT{а} помещается в \EAX и от него отнимается 0. Звучит абсурдно, но нужно это для того, чтобы проверить, 
0 ли в \EAX был до этого? Если да, то выставится флаг \ZF (что означает, что результат вычитания 0 от числа 
стал 0) и первый условный переход \JE (\emph{Jump if Equal} или его синоним \JZ~--- \emph{Jump if Zero}) 
сработает на метку \TT{\$LN4@f}, где выводится сообщение \TT{'zero'}.
Если первый переход не сработал, от значения отнимается по единице, 
и если на какой-то стадии в результате образуется 0, то сработает соответствующий переход.

И в конце концов, если ни один из условных переходов не сработал, управление передается \printf
со строковым аргументом \TT{'something unknown'}.

\label{jump_to_last_printf}
\myindex{\Stack}
Второе: мы видим две, мягко говоря, необычные вещи: указатель на сообщение помещается в переменную $a$, 
и затем \printf вызывается не через \CALL, а через \JMP. Объяснение этому простое. 
Вызывающая функция заталкивает в стек некоторое значение и через \CALL вызывает нашу функцию. 
\CALL в свою очередь заталкивает в стек адрес возврата (\ac{RA}) и делает безусловный переход на адрес нашей функции. 
Наша функция в самом начале (да и в любом её месте, потому что в теле функции нет ни одной инструкции, 
которая меняет что-то в стеке или в \ESP) имеет следующую разметку стека:

\begin{itemize}
\item\ESP --- хранится \ac{RA}
\item\TT{ESP+4} --- хранится значение $a$ 
\end{itemize}

С другой стороны, чтобы вызвать \printf, нам нужна почти такая же разметка стека, 
только в первом аргументе нужен указатель на строку. Что, собственно, этот код и делает.

Он заменяет свой первый аргумент на адрес строки, и затем передает управление \printf, как если бы вызвали не 
нашу функцию \ttf, а сразу \printf. 
\printf выводит некую строку на \gls{stdout}, затем исполняет инструкцию \RET, 
которая из стека достает \ac{RA} и управление передается в ту функцию, 
которая вызывала \ttf, минуя при этом конец функции \ttf.

\myindex{\CStandardLibrary!longjmp()}
\newcommand{\URLSJ}{\href{http://go.yurichev.com/17121}{wikipedia}}
% TODO \myref{}
Всё это возможно, потому что \printf вызывается в \ttf в самом конце. 
Всё это чем-то даже похоже на \TT{longjmp()}\footnote{\URLSJ}.
И всё это, разумеется, сделано для экономии времени исполнения.

Похожая ситуация с компилятором для ARM описана в секции \q{\PrintfSeveralArgumentsSectionName}~(\myref{ARM_B_to_printf}).

\clearpage
\subsubsection{MSVC + \olly}
\myindex{\olly}

Попробуем этот же пример в \olly.
Загружаем, нажимаем F8 (\stepover) до тех пор, пока не окажемся в своем исполняемом файле,
а не в \TT{ntdll.dll}.
Прокручиваем вверх до тех пор, пока не найдем \main.
Щелкаем на первой инструкции (\TT{PUSH EBP}), нажимаем F2 (\emph{set a breakpoint}), 
затем F9 (\emph{Run}) и точка останова срабатывает на начале \main.

Трассируем до того места, где готовится адрес переменной $x$:

\begin{figure}[H]
\centering
\myincludegraphics{patterns/04_scanf/1_simple/ex1_olly_1.png}
\caption{\olly: вычисляется адрес локальной переменной}
\label{fig:scanf_ex1_olly_1}
\end{figure}

На \EAX в окне регистров можно нажать правой кнопкой и далее выбрать \q{Follow in stack}.
Этот адрес покажется в окне стека.

Смотрите, это переменная в локальном стеке. Там дорисована красная стрелка.
И там сейчас какой-то мусор (\TT{0x6E494714}).
Адрес этого элемента стека сейчас, при помощи \PUSH запишется в этот же стек рядом.
Трассируем при помощи F8 вплоть до конца исполнения \scanf.
А пока \scanf исполняется, в консольном окне, вводим, например, 123:

\lstinputlisting{patterns/04_scanf/1_simple/console.txt}

\clearpage
Вот тут \scanf отработал:

\begin{figure}[H]
\centering
\myincludegraphics{patterns/04_scanf/1_simple/ex1_olly_3.png}
\caption{\olly: \scanf исполнилась}
\label{fig:scanf_ex1_olly_3}
\end{figure}

\scanf вернул 1 в \EAX, что означает, что он успешно прочитал одно значение.
В наблюдаемом нами элементе стека теперь \TT{0x7B} (123).

\clearpage
Чуть позже это значение копируется из стека в регистр \ECX и передается в \printf{}:

\begin{figure}[H]
\centering
\myincludegraphics{patterns/04_scanf/1_simple/ex1_olly_4.png}
\caption{\olly: готовим значение для передачи в \printf}
\label{fig:scanf_ex1_olly_4}
\end{figure}


}
\DE{\subsubsection{x86}

\myparagraph{\NonOptimizing MSVC}

Ergebnis (MSVC 2010):

\lstinputlisting[caption=MSVC 2010,style=customasmx86]{patterns/08_switch/1_few/few_msvc.asm}
Unsere Funktionen mit ein paar Fällen in switch() ist analog zu dieser Konstruktion:

\lstinputlisting[label=switch_few_ifelse,style=customc]{patterns/08_switch/1_few/few_analogue.c}

\myindex{\CLanguageElements!switch}
\myindex{\CLanguageElements!if}
Wenn wir mit einem switch() mit einigen wenigen Fällen arbeiten, ist es unmöglich sicher zu sein, dass es sich
tatsächlich im Quellcode um ein switch() handelt und nicht um eine Reihe von if()-Anweisungen.

\myindex{\SyntacticSugar}
Das bedeutet, dass switch() ein syntaktischer Zucker für eine große Anzahl von verschachtelten if()-Anweisungen ist.

Hier ist nichts Neues für uns im erzeugten Code, mit der Ausnahme, dass der Compiler die Eingabevariable $a$ in einer
temporären Variable \TT{tv64} speichert\footnote{Lokale Variablen im Stack haben den Präfix \TT{tv}---so benennt MSVC
interne Variablen für seine Zwecke}.

Wenn wir diesen Code mit GCC 4.4.1 kompilieren, erhalten wir fast das gleiche Ergebnis, sogar unter Verwendung maximaler
Optimierung (Option \Othree).

\myparagraph{\Optimizing MSVC}

% TODO separate various kinds of \TT
% idea: enclose command lines in a specific environment, like \cmdline{} 
% assembly instructions in \asm{} (now both \TT and \q{} are used),
% variables in,  like \var{}
% messages (string constants) in something else, like \strconst
% to separate them all. Now they all use \TT, which is not best
% \INS{} for all instructions including operands? --DY
Aktivieren wir nun Optimierung in MSVC (\Ox): \TT{cl 1.c /Fa1.asm /Ox}

\label{JMP_instead_of_RET}
\lstinputlisting[caption=MSVC,style=customasmx86]{patterns/08_switch/1_few/few_msvc_Ox.asm}
Hier finden wir ein paar schmutzige Tricks.

\myindex{x86!\Instructions!JZ}
\myindex{x86!\Instructions!JE}
\myindex{x86!\Instructions!SUB}
Zunächst: der Wert von $a$ wird in \EAX abgelegt und 0 wird davon abgezogen. Klingt absurd, muss aber getan werden, um
zu prüfen, ob der Wert in \EAX null ist. Falls ja, wird das \ZF Flag gesetzt (denn 0 minus 0 ist 0) und der erste
bedingte Sprung \JE (\emph{Jump if Equal}) oder synonym \JZ~---\emph{Jump if Zero}) wird ausgeführt und der Control Flow
wird an das Label \TT{\$LN4@f} übergeben, an dem die Nachricht \TT{'zero'} ausgegeben wird.
Wenn der erste Sprung nicht ausgeführt wird, wird 1 vom Eingabewert abgezogen und wenn an irgendeinem Punkt der
Ausführung ein Ergebnis von 0 auftritt, wird der zugehörige Sprung ausgeführt. 

Falls aber keiner der Sprünge ausgeführt wird, wird der Control Flow mit dem String Argument \TT{'something unknown'}
an \printf übergeben.

\label{jump_to_last_printf}
\myindex{\Stack}
Zweitens: wir sehen etwas für uns Ungewohntes: ein Pointer auf einen String wird in der Variablen $a$ abgelegt und
anschließend wird \printf nicht über \CALL, sondern per \JMP aufgerufen. Es gibt hierfür eine einfache Erklärung:
Der \glslink{caller}{Aufrufer} legt einen Wert auf dem Stack ab und ruft unsere Funktion über \CALL auf.
\CALL selbst legt die Rücksprungadresse (\ac{RA}) auf dem Stack ab und macht einen unbedingten Sprung zur Adresse
unserer Funktion. Unsere Funktion hat an jedem Punkt der Ausführung (da sie keine Befehle enthält, die den Stackpointer
verändern) das folgende Stacklayout:

\begin{itemize}
\item\ESP---zeigt auf \ac{RA}
\item\TT{ESP+4}---zeigt auf die Variable $a$ 
\end{itemize}
Wenn wir andererseits \printf aufrufen benötigen wir hier exakt das gleiche Stacklayout mit dem Unterschied des ersten
Arguments von \printf, welches auf den auszugebenden String zeigt. Unser Code tut hier das Folgende:

Er ersetzt das erste Argument der Funktion mit der Adresse (d.i. dem Pointer) auf den String und springt zu \printf als
ob wir nicht unsere Funktion \ttf, sondern direkt \printf aufrufen würden. Die Funktion \printf gibt einen String auf
\gls{stdout} aus und führt dann den \RET Befehl aus, der die \ac{RA} vom Stack holt. Der Control Flow wird nicht an \ttf
übergeben, sondern an den Aufrufer von \ttf, womit effektiv die Funktion \ttf umgangen wird.

\myindex{\CStandardLibrary!longjmp()}
\newcommand{\URLSJ}{\href{http://go.yurichev.com/17121}{wikipedia}}

% TODO \myref{}
All dies ist möglich, da \printf in allen Fällen ganz am Ende der Funktion \ttf aufgerufen wird. In gewisser Weise
besteht Ähnlichkeit zur Funktion \TT{longjmp()}\footnote{\URLSJ}. Natürlich geschieht all dies um die
Ausführungsgeschwindigkeit zu erhöhen.

Ein ähnlicher Fall mit dem ARM Compiler wird im Abschnitt \q{\PrintfSeveralArgumentsSectionName}
beschrieben:~(\myref{ARM_B_to_printf}).


\clearpage
\myparagraph{\Optimizing MSVC + \olly}
\myindex{\olly}

Wir untersuchen das (optimierte) Beispiel in \olly. Hier ist der erste
Durchlauf:

\begin{figure}[H]
\centering
\myincludegraphics{patterns/10_strings/1_strlen/olly1.png}
\caption{\olly: Beginn erster Durchlauf}
\label{fig:strlen_olly_1}
\end{figure}

Wir sehen, dass \olly eine Schleife gefunden hat, und zur Verbesserung der
Lesbarkeit, diese in eckige Klammern \emph{eingeschlossen} hat.
Nach Rechtsklick auf \EAX wählen wir \q{Follow in Dump} und das Speicherfenster
scrollt an die passende Stelle. 
Hier sehen wir den String \q{hello!} im Speicher. 
Dahinter befindet sich mindestens ein Nullbyte und im Anschluss Zufallsbits.

Wenn \olly ein Register mit einer gültigen Adresse, die auf einen String zeigt,
findet, wird dieser String angezeigt.

\clearpage
Wir drücken einige Male F8 (\stepover) um zum Anfang der Schleifenkörpers zu
gelangen:

\begin{figure}[H]
\centering
\myincludegraphics{patterns/10_strings/1_strlen/olly2.png}
\caption{\olly: Beginn zweiter Durchlauf}
\label{fig:strlen_olly_2}
\end{figure}

Wir sehen, dass \EAX nun die Adresse des zweiten Zeichens des Strings enthält.

\clearpage

Durch hinreichend häufiges Drücken von F8 verlassen wir schließlich die
Schleife:

\begin{figure}[H]
\centering
\myincludegraphics{patterns/10_strings/1_strlen/olly3.png}
\caption{\olly: Pointer Differenz wird berechnet}
\label{fig:strlen_olly_3}
\end{figure}

% FIXME:
Wir sehen, dass \EAX jetzt die Adresse des Nullbytes direkt hinter dem String
enthält. In der Zwischenzeit hat sich \EDX nicht verändert, es zeigt also immer
noch auf den Anfang des Strings. 

Die Differenz zwischen den beiden Adressen wird jetzt berechnet. 

\clearpage
Der \SUB Befehl wurde gerade ausgeführt:

\begin{figure}[H]
\centering
\myincludegraphics{patterns/10_strings/1_strlen/olly4.png}
\caption{\olly: \EAX muss dekrementiert werden}
\label{fig:strlen_olly_4}
\end{figure}

Die Differenz der Pointer im \EAX Register beträgt nun--7.
Tatsächlich beträgt die Länge des \q{hello!} Strings 6 Zeichen, aber mit dem
Nullbyte am Ende dazugezählt sind es 7.
Die Funktion \TT{strlen()} soll aber die Anzahl der Nicht-Null-Zeichen im String
zurückliefert, also wird einmal dekrementiert und der Funktionsaufruf
anschließend beendet.


}
\FR{\subsubsection{x86}

\myparagraph{MSVC \NonOptimizing}

Résultat (MSVC 2010):

\lstinputlisting[caption=MSVC 2010,style=customasmx86]{patterns/08_switch/1_few/few_msvc.asm}

Notre fonction avec quelques cas dans switch() est en fait analogue à cette construction:

\lstinputlisting[label=switch_few_ifelse,style=customc]{patterns/08_switch/1_few/few_analogue.c}

\myindex{\CLanguageElements!switch}
\myindex{\CLanguageElements!if}

Si nous utilisons switch() avec quelques cas, il est impossible de savoir si il y
avait un vrai switch() dans le code source, ou un ensemble de directives if().
\myindex{\SyntacticSugar}

Ceci indique que switch() est comme un sucre syntaxique pour un grand nombre de if()
imbriqués.

Il n'y a rien de particulièrement nouveau pour nous dans le code généré, à l'exception
que le compilateur déplace la variable d'entrée $a$ dans une variable locale temporaire
\TT{tv64} \footnote{Les variables locales sur la pile sont préfixées avec \TT{tv}---c'est
ainsi que MSVC appelle les variables internes dont il a besoin.}.

Si nous compilons ceci avec GCC 4.4.1, nous obtenons presque le même résultat,
même avec le niveau d'optimisation le plus élevé (\Othree option).

\myparagraph{MSVC \Optimizing}

% TODO separate various kinds of \TT
% idea: enclose command lines in a specific environment, like \cmdline{} 
% assembly instructions in \asm{} (now both \TT and \q{} are used),
% variables in,  like \var{}
% messages (string constants) in something else, like \strconst
% to separate them all. Now they all use \TT, which is not best
% \INS{} for all instructions including operands? --DY
Maintenant compilons dans MSVC avec l'optimisation (\Ox): \TT{cl 1.c /Fa1.asm /Ox}

\label{JMP_instead_of_RET}
\lstinputlisting[caption=MSVC,style=customasmx86]{patterns/08_switch/1_few/few_msvc_Ox.asm}

Ici, nous voyons quelques hacks moches.

\myindex{x86!\Instructions!JZ}
\myindex{x86!\Instructions!JE}
\myindex{x86!\Instructions!SUB}

Premièrement: la valeurs de $a$ est mise dans \EAX et 0 en est soustrait. Ça semble
absurde, mais cela est fait pour vérifier si la valeur dans \EAX est 0. Si oui, le
flag \ZF est mis (e.g. soustraire de 0 est 0) et le premier saut conditionnel \JE
(\emph{Jump if Equal} saut si égal ou synonyme \JZ~---\emph{Jump if Zero} saut si zéro)
va être effectué et le déroulement du programme passera au label \TT{\$LN4@f}, où
le message \TT{'zero'} est affiché.
Si le premier saut n'est pas effectué, 1 est soustrait de la valeur d'entrée et si
à une étape le résultat est 0, le saut correspondant sera effectué.

Et si aucun saut n'est exécuté, l'exécution passera au \printf avec comme argument
la chaîne\\ \TT{'something unknown'}.

\label{jump_to_last_printf}
\myindex{\Stack}

Deuxièmement: nous voyons quelque chose d'inhabituel pour nous: un pointeur sur une
chaîne est mis dans la variable $a$ et ensuite \printf est appelé non pas par \CALL,
mais par \JMP. Il y a une explication simple à cela: l'\glslink{caller}{appelant}
pousse une valeur sur la pile et appelle notre fonction via \CALL.
\CALL lui même pousse l'adresse de retour (\ac{RA}) sur la pile et fait un saut
inconditionnel à l'adresse de notre fonction.
Notre fonction, à tout moment de son exécution (car elle ne contient pas d'instruction
qui modifie le pointeur de pile) à le schéma suivant pour sa pile:

\begin{itemize}
\item\ESP---pointe sur \ac{RA}
\item\TT{ESP+4}---pointe sur la variable $a$
\end{itemize}

D'un autre côté, lorsque nous appelons \printf ici nous avons exactement la même
disposition de pile, excepté pour le premier argument de \printf, qui doit pointer
sur la chaîne. Et c'est ce que fait notre code.

Il remplace le premier argument de la fonction par l'adresse de la chaîne et saute
à \printf, comme si nous n'avions pas appelé notre fonction \ttf, mais directement
\printf.
\printf affiche la chaîne sur la \glslink{stdout}{sortie standard} et ensuite exécute
l'instruction \RET qui POPs \ac{RA} de la pile et l'exécution est renvoyée non pas
à \ttf mais plutôt à l'\glslink{caller}{appelant} de \ttf, ignorant la fin de la
fonction \ttf.

\myindex{\CStandardLibrary!longjmp()}
\newcommand{\URLSJ}{\href{http://go.yurichev.com/17121}{Wikipédia}}

% TODO \myref{}
Tout ceci est possible car \printf est appelée, dans tous les cas, tout à la fin
de la fonction \ttf.
Dans un certain sens, c'est similaire à la fonction \TT{longjmp()}\footnote{\URLSJ}.
Et bien sûr, c'est fait dans un but de vitesse d'exécution.

Un cas similaire avec le compilateur ARM est décrit dans la section \q{\PrintfSeveralArgumentsSectionName},
ici~(\myref{ARM_B_to_printf}).

\clearpage
\mysubparagraph{\olly}

Comme cet exemple est compliqué, traçons-le dans \olly.

\olly peut détecter des constructions avec switch(), et ajoute des commentaires utiles.
\EAX contient 2 au début, c'est la valeur du paramètre de la fonction:

\begin{figure}[H]
\centering
\myincludegraphics{patterns/08_switch/1_few/olly1.png}
\caption{\olly: \EAX 
contient maintenant le premier (et unique) argument de la fonction}
\label{fig:switch_few_olly1}
\end{figure}

\clearpage
0 est soustrait de 2 dans \EAX.
Bien sûr, \EAX contient toujours 2.
Mais le flag \ZF est maintenant à 0, indiquant que le résultat est différent de
zéro:

\begin{figure}[H]
\centering
\myincludegraphics{patterns/08_switch/1_few/olly2.png}
\caption{\olly: \SUB exécuté}
\label{fig:switch_few_olly2}
\end{figure}

\clearpage
\DEC est exécuté et \EAX contient maintenant 1.
Mais 1 est différent de zéro, donc le flag \ZF est toujours à 0:

\begin{figure}[H]
\centering
\myincludegraphics{patterns/08_switch/1_few/olly3.png}
\caption{\olly: premier \DEC exécuté}
\label{fig:switch_few_olly3}
\end{figure}

\clearpage
Le \DEC suivant est exécuté.
\EAX contient maintenant 0 et le flag \ZF est mis, car le résultat devient zéro:

\begin{figure}[H]
\centering
\myincludegraphics{patterns/08_switch/1_few/olly4.png}
\caption{\olly: second \DEC exécuté}
\label{fig:switch_few_olly4}
\end{figure}

\olly montre que le saut va être effectué (\emph{Jump is taken}).

\clearpage
Un pointeur sur la chaîne \q{two} est maintenant écrit sur la pile:

\begin{figure}[H]
\centering
\myincludegraphics{patterns/08_switch/1_few/olly5.png}
\caption{\olly: 
pointeur sur la chaîne qui va être écrite à la place du premier argument}
\label{fig:switch_few_olly5}
\end{figure}

% TODO: homogenize numbers
% now they are inconsistent: sometimes plain text, sometimes in math mode
% some kind of \expr{} both for numbers and expressions? --DY
Veuillez noter: l'argument de la fonction courante est 2, et 2 est maintenant
sur la pile, à l'adresse \TT{0x001EF850}.

\clearpage
\MOV écrit le pointeur sur la chaîne à l'adresse \TT{0x001EF850} (voir la fenêtre de la pile).
Puis, le saut est effectué.
Ceci est la première instruction de la fonction \printf dans MSVCR100.DLL (Cet exemple
a été compilé avec le switch /MD):

\begin{figure}[H]
\centering
\myincludegraphics{patterns/08_switch/1_few/olly6.png}
\caption{\olly: première instruction de \printf dans MSVCR100.DLL}
\label{fig:switch_few_olly6}
\end{figure}

Maintenant \printf traite la chaîne à l'adresse \TT{0x00FF3010} comme c'est son
seul argument et l'affiche.

\clearpage
Ceci est la dernière instruction de \printf:

\begin{figure}[H]
\centering
\myincludegraphics{patterns/08_switch/1_few/olly7.png}
\caption{\olly: dernière instruction de \printf dans MSVCR100.DLL}
\label{fig:switch_few_olly7}
\end{figure}

La chaîne \q{two} vient juste d'être affichée dans la fenêtre console.

\clearpage
Maintenant, appuyez sur F7 ou F8 (\stepover) et le retour ne se fait pas sur \ttf,
mais sur \main:

\begin{figure}[H]
\centering
\myincludegraphics{patterns/08_switch/1_few/olly8.png}
\caption{\olly: retourne à \main}
\label{fig:switch_few_olly8}
\end{figure}

Oui, le saut a été direct, depuis les entrailles de \printf vers \main.
Car \ac{RA} dans la pile pointe non pas quelque part dans \ttf, mais en fait sur
\main.
Et \CALL \TT{0x00FF1000} a été l'instruction qui a appelé \ttf.


}
\IT{\subsubsection{x86}

\myparagraph{\NonOptimizing MSVC}

Risultato (MSVC 2010):

\lstinputlisting[caption=MSVC 2010,style=customasmx86]{patterns/08_switch/1_few/few_msvc.asm}

La nostra funzione con pochi casi nello switch() è praticamente analaco a questa costruzione:

\lstinputlisting[label=switch_few_ifelse,style=customc]{patterns/08_switch/1_few/few_analogue.c}

\myindex{\CLanguageElements!switch}
\myindex{\CLanguageElements!if}

Nel caso di switch() con un piccolo numero di casi, è impossibile essere sicuri che nel sorgente originale ci fosse
veramente la funzione switch() o soltanto una serie di costrutti if().


\myindex{\SyntacticSugar}

Ciò vuol dire che switch() si comporta come "zucchero sintattico", equivalente ad un alto numero di if() annidati.

Dal nostro punto di vista non c'è niente di particolarmente nuovo nel codice generato,
con l'eccezione dello spostamento (da parte del compilatore) della variabile in input $a$ in una variabile locale temporanea \TT{tv64}
\footnote{Le variabili locali nello stack hanno il prefisso \TT{tv}--- MSVC nomina così le varibili locali per i suoi scopi}.

Se compiliamo questo codice con 4.4.1 otteniamo pressoché lo stesso risultato, anche con il maggior livello di ottimizzazione (\Othree option).

\myparagraph{\Optimizing MSVC}

% TODO separate various kinds of \TT
% idea: enclose command lines in a specific environment, like \cmdline{} 
% assembly instructions in \asm{} (now both \TT and \q{} are used),
% variables in,  like \var{}
% messages (string constants) in something else, like \strconst
% to separate them all. Now they all use \TT, which is not best
% \INS{} for all instructions including operands? --DY
Now let's turn on optimization in MSVC (\Ox): \TT{cl 1.c /Fa1.asm /Ox}

\label{JMP_instead_of_RET}
\lstinputlisting[caption=MSVC,style=customasmx86]{patterns/08_switch/1_few/few_msvc_Ox.asm}

Qui possiamo vedere un po' di trucchetti.

\myindex{x86!\Instructions!JZ}
\myindex{x86!\Instructions!JE}
\myindex{x86!\Instructions!SUB}

Primo: il valore di $a$ è messo in \EAX, e gli viene sottratto 0. Sembra assurdo, ma è fatto per controllare se il valore 
in \EAX è 0. Se si, il flag \ZF viene settato (i.e. sottrarre 0 a 0 da 0)
e il primo jump condizionale \JE (\emph{Jump if Equal} o il sinonimo \JZ~---\emph{Jump if Zero}) è innescato e il controllo del flusso 
passato alla label \TT{\$LN4@f}, dove il messaggio \TT{'zero'} viene stampato. 
Se il primo jump non viene innescato, 1 viene sottratto dal valore in input e se ad un certo punto il risultato è 0, il jump corrispondente
viene innescato.

Se nessun jump viene innescato, il controllo del flusso passa a \printf con l'argomento \\
\TT{'something unknown'}.

\label{jump_to_last_printf}
\myindex{\Stack}

Secondo: notiamo qualcosa di inusuale per noi: un puntatore a stringa viene messo nella variabile $a$, e 
successivamente viene chiamata \printf non tramite \CALL, ma via \JMP. C'è una spiegazione semplice dietro ciò: 
il \gls{caller} mette un valore sullo stack e chiama la nostra funzione tramite \CALL. 
La stessa \CALL fa il push del return address (\ac{RA}) sullo stack e fa un salto non condizionale all'indirizzo della nostra funzione. 
La nostra funzione, in qualunque punto della sua esecuzione (poiché non contiene istruzioni che spostano lo stack pointer)
ha il seguente layout dello stack:

\begin{itemize}
\item\ESP---punta a \ac{RA}
\item\TT{ESP+4}---punta alla variabile $a$ 
\end{itemize}

Dall'altro lato, quando dobbiamo chiamare \printf qui abbiamo bisogno esattamente dello stesso layout dello stack,
eccetto per il primo argomento di \printf, che deve puntare alla stringa. 
E questo è ciò che fa il codice.

Sostituisce il primo argomento della funzione con l'indirizzo della stringa, e salta a \printf, come se non avessimo chiamato
la nostra funzione \ttf, ma direttamente \printf.
\printf stampa una stringa su \gls{stdout} e successivamente esegue l'istruzione \RET , che fa il POP del 
\ac{RA} dallo stack, e il controllo del flusso viene restituito non a \ttf ma al \gls{caller} di \ttf, 
bypassando la fine della funzione \ttf.

\myindex{\CStandardLibrary!longjmp()}
\newcommand{\URLSJ}{\href{http://go.yurichev.com/17121}{wikipedia}}

% TODO \myref{}
Tutto ciò è possibile perchè \printf è chiamata proprio alla fine della funzione \ttf in tutti i casi.
In qualche modo è simile alla funzione \TT{longjmp()}\footnote{\URLSJ} function.
E, ovviamente, tutto ciò viene fatto a favore della velocità di esecuzione.

Un simile caso con il compilatore ARM è descritto nella sezione \q{\PrintfSeveralArgumentsSectionName}, qui~(\myref{ARM_B_to_printf}).

\clearpage
\subsubsection{MSVC: x86 + \olly}
\myindex{\olly}

Things are even simpler here:

\begin{figure}[H]
\centering
\myincludegraphics{patterns/04_scanf/2_global/ex2_olly_1.png}
\caption{\olly: after \scanf execution}
\label{fig:scanf_ex2_olly_1}
\end{figure}

The variable is located in the data segment.
After the \PUSH instruction (pushing the address of $x$) gets executed, 
the address appears in the stack window. Right-click on that row and select \q{Follow in dump}.
The variable will appear in the memory window on the left.
After we have entered 123 in the console, 
\TT{0x7B} appears in the memory window (see the highlighted screenshot regions).

But why is the first byte \TT{7B}?
Thinking logically, \TT{00 00 00 7B} must be there.
The cause for this is referred as  \gls{endianness}, and x86 uses \emph{little-endian}.
This implies that the lowest byte is written first, and the highest written last.
Read more about it at: \myref{sec:endianness}.
Back to the example, the 32-bit value is loaded from this memory address into \EAX and passed to \printf.

The memory address of $x$ is \TT{0x00C53394}.

\clearpage
In \olly we can review the process memory map (Alt-M)
and we can see that this address is inside the \TT{.data} PE-segment of our program:

\label{olly_memory_map_example}
\begin{figure}[H]
\centering
\myincludegraphics{patterns/04_scanf/2_global/ex2_olly_2.png}
\caption{\olly: process memory map}
\label{fig:scanf_ex2_olly_2}
\end{figure}



}
\JA{\subsubsection{x86}

\myparagraph{\NonOptimizing MSVC}

結果 (MSVC 2010):

\lstinputlisting[caption=MSVC 2010,style=customasmx86]{patterns/08_switch/1_few/few_msvc.asm}

実際、switch()でいくつかのcaseを持つ私たちの関数は、この構造に似ています。

\lstinputlisting[label=switch_few_ifelse,style=customc]{patterns/08_switch/1_few/few_analogue.c}

\myindex{\CLanguageElements!switch}
\myindex{\CLanguageElements!if}

いくつかのcaseでswitch()を使用する場合、ソースコード内の実際のswitch()か、
単にif文の組であるかどうかを確認することは不可能です。
\myindex{\SyntacticSugar}

これはswitch()が多段にネストされたif文との糖衣構文のようなものであることを意味します。

コンパイラが入力変数 $a$ を一時的なローカル変数\TT{tv64}に移動することを除いて、
生成されたコードには特に新しいことはありません。
\footnote{スタック内のローカル変数には接頭辞\TT{tv}が付きます。MSVCが内部変数として使用するために命名しています。}

これをGCC 4.4.1でコンパイルすると、最大限の最適化(\Othree option)を有効にしても
ほぼ同じ結果になります。

\myparagraph{\Optimizing MSVC}

% TODO separate various kinds of \TT
% idea: enclose command lines in a specific environment, like \cmdline{} 
% assembly instructions in \asm{} (now both \TT and \q{} are used),
% variables in,  like \var{}
% messages (string constants) in something else, like \strconst
% to separate them all. Now they all use \TT, which is not best
% \INS{} for all instructions including operands? --DY
では、MSVC(\Ox)の最適化を有効にしましょう:\TT{cl 1.c /Fa1.asm /Ox}

\label{JMP_instead_of_RET}
\lstinputlisting[caption=MSVC,style=customasmx86]{patterns/08_switch/1_few/few_msvc_Ox.asm}

ここで、汚いハックを見ることができます。

\myindex{x86!\Instructions!JZ}
\myindex{x86!\Instructions!JE}
\myindex{x86!\Instructions!SUB}

最初に、 $a$ の値を \EAX に置き、0を引きます。 EAXの値が0かどうかを確認するために行われますが、
そうであれば、 \ZF フラグがセットされます(例えば、0からの減算は0)
最初の条件ジャンプ \JE (\emph{Jump if Equal} またはあ同義語 \JZ~---\emph{Jump if Zero})は実行され、
制御フローは\TT{\$LN4@f}ラベルに渡されます。ここでは、 \TT{'zero'}メッセージが出力されます。
最初のジャンプが実行されない場合は、入力値から1が減算され、結果が0の場合、対応するジャンプが実行されます。

また、ジャンプが全く実行されない場合、制御フローは文字列引数\TT{'something unknown'}を \printf に渡します。

\label{jump_to_last_printf}
\myindex{\Stack}

次に、文字列ポインタが $a$ 変数に置かれ、 \printf が \CALL ではなく \JMP を介して呼び出されます。 
簡単に説明するとこうなります:
\gls{caller} は値をスタックにプッシュし、 \CALL 経由で関数を呼び出します。
\CALL 自体は戻りアドレス(\ac{RA})をスタックにプッシュし、関数アドレスへの無条件ジャンプを行います。
スタックポインタを移動させる命令が含まれていないため、任意の実行時点での関数は、次のスタックレイアウトを持ちます。

\begin{itemize}
\item\ESP---points to \ac{RA}
\item\TT{ESP+4}---points to the $a$ variable 
\end{itemize}

反対に、\printf をここで呼び出さなければならないときは、文字列を指し示す必要がある最初の\printf 引数を除いて、
全く同じスタックレイアウトが必要です。それが私たちのコードがすることです。

ファンクションの最初の引数を文字列のアドレスに置き換え、
関数 \ttf を直接呼び出しずに直接 \printf を呼び出すかのように、 \printf にジャンプします。
\printf は文字列を \gls{stdout} に出力し、 \RET 命令を実行します。スタックから\ac{RA}を取り出し、
制御フローは \ttf ではなく \ttf 関数の終りをバイパスして、 \ttf の \gls{caller} です。

\myindex{\CStandardLibrary!longjmp()}
\newcommand{\URLSJ}{\href{http://go.yurichev.com/17121}{wikipedia}}

% TODO \myref{}
\printf はすべての場合に  \ttf 関数の終わりで右に呼ばれるので、これはすべて可能です。
ある意味では、\TT{longjmp()}\footnote{\URLSJ}関数に似ています。
そしてもちろん、それはスピードのためにすべて行われます。

ARMコンパイラと同様のケースは、\q{\PrintfSeveralArgumentsSectionName}セクションに記載されています。
こちら:~(\myref{ARM_B_to_printf})

\clearpage
\mysubparagraph{\olly}

この例は扱いにくいので、 \olly でトレースしてみましょう。

\olly はそのようなswitch()構文を検出することができ、有用なコメントを追加することができます。
\EAX の値は最初は2で、それは関数への入力値です:

\begin{figure}[H]
\centering
\myincludegraphics{patterns/08_switch/1_few/olly1.png}
\caption{\olly: \EAX 
は最初の(そして唯一の)関数への引数を含んでいます}
\label{fig:switch_few_olly1}
\end{figure}

\clearpage
0は \EAX から2を引いた値です。
もちろん、 \EAX にはまだ2が入っています。
しかし、 \ZF フラグは0になり、結果の値がゼロでないことを示します。

\begin{figure}[H]
\centering
\myincludegraphics{patterns/08_switch/1_few/olly2.png}
\caption{\olly: \SUB の実行}
\label{fig:switch_few_olly2}
\end{figure}

\clearpage
\DEC が実行され、 \EAX には1が入ります。
しかし1はゼロではないので、 \ZF フラグはまだ0です:

\begin{figure}[H]
\centering
\myincludegraphics{patterns/08_switch/1_few/olly3.png}
\caption{\olly: 最初の \DEC 実行}
\label{fig:switch_few_olly3}
\end{figure}

\clearpage
次の \DEC が実行されます。
\EAX は最終的に0になり、結果がゼロであるため \ZF フラグが設定されます。

\begin{figure}[H]
\centering
\myincludegraphics{patterns/08_switch/1_few/olly4.png}
\caption{\olly: 2回目の \DEC 実行}
\label{fig:switch_few_olly4}
\end{figure}

\olly は、このジャンプが今行われることを示しています。

\clearpage
\q{two}という文字列へのポインタが今スタックに書き込まれます:

\begin{figure}[H]
\centering
\myincludegraphics{patterns/08_switch/1_few/olly5.png}
\caption{\olly: 
文字列へのポインタは、最初の引数の場所に書き込まれる}
\label{fig:switch_few_olly5}
\end{figure}

% TODO: homogenize numbers
% now they are inconsistent: sometimes plain text, sometimes in math mode
% some kind of \expr{} both for numbers and expressions? --DY
注意:関数の現在の引数は2であり、2はスタックに\TT{0x001EF850}のアドレスにあります。

\clearpage
\MOV はアドレス\TT{0x001EF850}の文字列にポインタを書き込みます(スタックウィンドウを参照)。
その後、ジャンプが発生します。
これはMSVCR100.DLLの \printf 関数の最初の命令です(この例は/MDスイッチでコンパイルされています)。

\begin{figure}[H]
\centering
\myincludegraphics{patterns/08_switch/1_few/olly6.png}
\caption{\olly: MSVCR100.DLLでの \printf の最初の命令}
\label{fig:switch_few_olly6}
\end{figure}

今や \printf は\TT{0x00FF3010}の文字列を唯一の引数として扱い、文字列を出力します。

\clearpage
これが \printf の最後の命令です。

\begin{figure}[H]
\centering
\myincludegraphics{patterns/08_switch/1_few/olly7.png}
\caption{\olly: MSVCR100.DLLの \printf の最後の命令}
\label{fig:switch_few_olly7}
\end{figure}

文字列 \q{two} はコンソールウィンドウに表示されます。

\clearpage
F7またはF8を押して(\stepover) リターンすると\dots \ttf ではなく、 \main にいきます。

\begin{figure}[H]
\centering
\myincludegraphics{patterns/08_switch/1_few/olly8.png}
\caption{\olly: \main へのリターン}
\label{fig:switch_few_olly8}
\end{figure}

はい、 \printf の中心から \main に直接ジャンプしました。
なぜならスタックの\ac{RA}は \ttf ではなく、 \main の場所を指しているからです。
\CALL \TT{0x00FF1000}は \ttf を呼び出した実際の命令です。


}

\EN{\subsubsection{ARM: \OptimizingKeilVI (\ARMMode)}
\myindex{\CLanguageElements!switch}

\lstinputlisting[style=customasmARM]{patterns/08_switch/1_few/few_ARM_ARM_O3.asm}

Again, by investigating this code we cannot say if it was a switch() in the original source code, 
or just a pack of if() statements.

\myindex{ARM!\Instructions!ADRcc}

Anyway, we see here predicated instructions again (like \ADREQ (\emph{Equal}))
which is triggered only in case $R0=0$, and then loads the address of the string \emph{<<zero\textbackslash{}n>>}
into \Reg{0}.
\myindex{ARM!\Instructions!BEQ}
The next instruction \ac{BEQ} redirects control flow to \TT{loc\_170}, if $R0=0$.

An astute reader may ask, will \ac{BEQ} trigger correctly since \ADREQ it
has already filled the \Reg{0} register before with another value?

Yes, it will since \ac{BEQ} checks the flags set by the \CMP instruction, 
and \ADREQ does not modify any flags at all.

The rest of the instructions are already familiar to us. 
There is only one call to \printf , at the end, and we have already examined this trick here~(\myref{ARM_B_to_printf}).
At the end, there are three paths to \printf{}.

\myindex{ARM!\Instructions!ADRcc}
\myindex{ARM!\Instructions!CMP}
The last instruction, \TT{CMP R0, \#2}, is needed to check if $a=2$.

If it is not true, then \ADRNE loads a pointer to the string \emph{<<something unknown \textbackslash{}n>>}
into \Reg{0}, since $a$ has already been checked to be equal to 0 or 1,
and we can sure that the $a$ variable is not equal to these numbers at this point.
And if $R0=2$, 
a pointer to the string \emph{<<two\textbackslash{}n>>}
will be loaded by \ADREQ into \Reg{0}.

\subsubsection{ARM: \OptimizingKeilVI (\ThumbMode)}

\lstinputlisting[style=customasmARM]{patterns/08_switch/1_few/few_ARM_thumb_O3.asm}

% FIXME а каким можно? к каким нельзя? \myref{} ->

As was already mentioned, it is not possible to add conditional predicates to most instructions in Thumb
mode, so the Thumb-code here is somewhat similar to the easily understandable x86 \ac{CISC}-style code.

\subsubsection{ARM64: \NonOptimizing GCC (Linaro) 4.9}

\lstinputlisting[style=customasmARM]{patterns/08_switch/1_few/ARM64_GCC_O0_EN.lst}

The type of the input value is \Tint, hence register \RegW{0} is used to hold it instead of the whole
\RegX{0} register.

The string pointers are passed to \puts using an \INS{ADRP}/\INS{ADD} instructions pair just like it was demonstrated in the 
\q{\HelloWorldSectionName} example:~\myref{pointers_ADRP_and_ADD}.

\subsubsection{ARM64: \Optimizing GCC (Linaro) 4.9}

\lstinputlisting[style=customasmARM]{patterns/08_switch/1_few/ARM64_GCC_O3_EN.lst}

Better optimized piece of code.
\TT{CBZ} (\emph{Compare and Branch on Zero}) instruction does jump if \RegW{0} is zero.
There is also a direct jump to \puts instead of calling it, like it was explained before:~\myref{JMP_instead_of_RET}.

}
\RU{\subsubsection{ARM: \OptimizingKeilVI (\ARMMode)}
\myindex{\CLanguageElements!switch}

\lstinputlisting[style=customasmARM]{patterns/08_switch/1_few/few_ARM_ARM_O3.asm}

Мы снова не сможем сказать, глядя на этот код, был ли в оригинальном исходном коде switch() 
либо же несколько операторов if().

\myindex{ARM!\Instructions!ADRcc}
Так или иначе, мы снова видим здесь инструкции с предикатами, например, \ADREQ (\emph{(Equal)}), 
которая будет исполняться только
если $R0=0$, и тогда в \Reg{0} будет загружен адрес строки \emph{<<zero\textbackslash{}n>>}.

\myindex{ARM!\Instructions!BEQ}
Следующая инструкция \ac{BEQ} перенаправит исполнение на \TT{loc\_170}, если $R0=0$.

Кстати, наблюдательный читатель может спросить, сработает ли \ac{BEQ} нормально,
ведь \ADREQ перед ним уже заполнила регистр \Reg{0} чем-то другим?

Сработает, потому что \ac{BEQ} проверяет флаги, установленные инструкцией \CMP, 
а \ADREQ флаги никак не модифицирует.

Далее всё просто и знакомо. 
Вызов \printf один, и в самом конце, мы уже рассматривали подобный трюк~(\myref{ARM_B_to_printf}).
К вызову функции \printf{} в конце ведут три пути.

\myindex{ARM!\Instructions!ADRcc}
\myindex{ARM!\Instructions!CMP}
Последняя инструкция \TT{CMP R0, \#2} здесь нужна, чтобы узнать $a=2$ или нет.

Если это не так, то при помощи \ADRNE (\emph{Not Equal}) в \Reg{0} будет загружен указатель на 
строку \emph{<<something unknown \textbackslash{}n>>}, ведь $a$ уже было проверено на 0 и 1 до этого, 
и здесь $a$ точно не попадает под эти константы.

Ну а если $R0=2$, в \Reg{0} будет загружен указатель на строку \emph{<<two\textbackslash{}n>>} при помощи инструкции \ADREQ.

\subsubsection{ARM: \OptimizingKeilVI (\ThumbMode)}

\lstinputlisting[style=customasmARM]{patterns/08_switch/1_few/few_ARM_thumb_O3.asm}

% FIXME а каким можно? к каким нельзя? \myref{} ->
Как уже было отмечено, в Thumb-режиме нет возможности добавлять условные предикаты к большинству инструкций,
так что Thumb-код вышел похожим на код x86 в стиле \ac{CISC}, вполне понятный.

\subsubsection{ARM64: \NonOptimizing GCC (Linaro) 4.9}

\lstinputlisting[style=customasmARM]{patterns/08_switch/1_few/ARM64_GCC_O0_RU.lst}

Входное значение имеет тип \Tint, поэтому для него используется регистр \RegW{0},
а не целая часть регистра \RegX{0}.

Указатели на строки передаются в \puts при помощи пары инструкций ADRP/ADD, как было показано в примере
\q{\HelloWorldSectionName}:~\myref{pointers_ADRP_and_ADD}.

\subsubsection{ARM64: \Optimizing GCC (Linaro) 4.9}

\lstinputlisting[style=customasmARM]{patterns/08_switch/1_few/ARM64_GCC_O3_RU.lst}

Фрагмент кода более оптимизированный.
Инструкция \TT{CBZ} (\emph{Compare and Branch on Zero}~--- сравнить и перейти если ноль) совершает переход если \RegW{0} ноль.
Здесь также прямой переход на \puts вместо вызова, как уже было описано:~\myref{JMP_instead_of_RET}.

}
\DE{\subsubsection{ARM: \OptimizingKeilVI (\ARMMode)}
\myindex{\CLanguageElements!switch}

\lstinputlisting[style=customasmARM]{patterns/08_switch/1_few/few_ARM_ARM_O3.asm}
Auch hier können wir bei Untersuchung des Code nicht sagen, ob im Quellcode ein switch() oder eine Folge von
if()-Ausdrücken vorliegt.


\myindex{ARM!\Instructions!ADRcc}
Wir finden hier Befehle mit Prädikaten wieder (wie \ADREQ (\emph{Equal})), welcher nur dann ausgeführt wird, wenn $R0=0$
und dann die Adresse des Strings IT{<<zero\textbackslash{}n>>} nach \Reg{0} lädt.

\myindex{ARM!\Instructions!BEQ}
Der folgende \ac{BEQ} Befehl übergibt den Control Flow an \TT{loc\_170}, falls $R0=0$.
Ein aufmerksamer Leser könnte sich fragen, ob \ac{BEQ} korrekt ausgelöst wird, da \ADREQ das \Reg{0} Register bereits
mit einem anderen Wert befüllt hat.
Es wird korrekt ausgelöst, da \ac{BEQ} die Flags, die vom \CMP Befehl gesetzt wurden, prüft und \ADREQ die Flags nicht
verändert.

Die übrigen Befehle kennen wir bereits.
Es gibt nur einen Aufruf von \printf am Ende und wir haben diesen Trick bereits hier
kennengelernt~(\myref{ARM_B_to_printf}). Am Ende gibt es drei Wege zur Ausführung von \printf.

\myindex{ARM!\Instructions!ADRcc}
\myindex{ARM!\Instructions!CMP}
Der letzte Befehl, \TT{CMP R0, \#2}, wird benötigt, um zu prüfen, ob $a=2$.
Wenn dies nicht der Fall ist, lädt \ADRNE einen Pointer auf den String \emph{<<something unknown \textbackslash{}n>>} nach
\Reg{0}, da $a$ bereits auf Gleichheit mit 0 oder 1 geprüft wurde und wir können sicher sein, dass die Variable $a$ an
dieser Stelle keinen dieser beiden Werte enthält.
Falls $R0=2$ ist, lädt \ADREQ einen Pointer auf den String \emph{<<two\textbackslash{}n>>} nach \Reg{0}. 

\subsubsection{ARM: \OptimizingKeilVI (\ThumbMode)}

\lstinputlisting[style=customasmARM]{patterns/08_switch/1_few/few_ARM_thumb_O3.asm}

% FIXME а каким можно? к каким нельзя? \myref{} ->
Wie bereits erwähnt ist es bei den meisten Befehlen im Thumb mode nicht möglich Prädikate für Bedingungen hinzuzufügen,
sodass der Thumb-Code hier dem leicht verständlichen x86 \ac{CISC}-style Code sehr ähnlich ist.

\subsubsection{ARM64: \NonOptimizing GCC (Linaro) 4.9}

\lstinputlisting[style=customasmARM]{patterns/08_switch/1_few/ARM64_GCC_O0_DE.lst}
Der Datentyp des Eingabewertes ist \Tint, deshalb wird das Register \RegW{0} anstatt des \RegX{0} Registers verwendet,
um ihn aufzunehmen.

Die Pointer auf die Strings werden an \puts mit einem \INS{ADRP}/\INS{ADD} Befehlspaar übergeben, genauso wie wir es im
\q{\HelloWorldSectionName} Beispiel gezeigt haben:~\myref{pointers_ADRP_and_ADD}.

\subsubsection{ARM64: \Optimizing GCC (Linaro) 4.9}

\lstinputlisting[style=customasmARM]{patterns/08_switch/1_few/ARM64_GCC_O3_DE.lst}
Ein besser optimiertes Stück Code. 
Der Befehl \TT{CBZ} (\emph{Compare and Branch on Zero}) springt, falls \RegW{0} gleich null ist.
Es gibt auch einen direkten Sprung zu \puts anstelle eines Aufrufs, so wie bereits hier
erklärt:~\myref{JMP_instead_of_RET}.

}
\FR{\subsubsection{ARM: \OptimizingKeilVI (\ARMMode)}
\myindex{\CLanguageElements!switch}

\lstinputlisting[style=customasmARM]{patterns/08_switch/1_few/few_ARM_ARM_O3.asm}

A nouveau, en investiguant ce code, nous ne pouvons pas dire si il y avait un switch()
dans le code source d'origine ou juste un ensemble de déclarations if().

\myindex{ARM!\Instructions!ADRcc}

En tout cas, nous voyons ici des instructions conditionnelles (comme \ADREQ (\emph{Equal}))
qui ne sont exécutées que si $R0=0$, et qui chargent ensuite l'adresse de la chaîne
\emph{<<zero\textbackslash{}n>>} dans \Reg{0}.
\myindex{ARM!\Instructions!BEQ}
L'instruction suivante \ac{BEQ} redirige le flux d'exécution en \TT{loc\_170}, si $R0=0$.

Le lecteur attentif peut se demander si \ac{BEQ} s'exécute correctement puisque \ADREQ
a déjà mis une autre valeur dans le registre \Reg{0}.

Oui, elle s'exécutera correctement, car \ac{BEQ} vérifie les flags mis par l'instruction
\CMP et \ADREQ ne modifie aucun flag.

Les instructions restantes nous sont déjà familières.
Il y a seulement un appel à \printf, à la fin, et nous avons déjà examiné cette
astuce ici~(\myref{ARM_B_to_printf}).
A la fin, il y a trois chemins vers \printf{}.

\myindex{ARM!\Instructions!ADRcc}
\myindex{ARM!\Instructions!CMP}
La dernière instruction, \TT{CMP R0, \#2}, est nécessaire pour vérifier si $a=2$.

Si ce n'est pas vrai, alors \ADRNE charge un pointeur sur la chaîne \emph{<<something unknown \textbackslash{}n>>}
dans \Reg{0}, puisque $a$ a déjà été comparée pour savoir s'elle est égale
à 0 ou 1, et nous sommes sûrs que la variable $a$ n'est pas égale à l'un de
ces nombres, à ce point.
Et si $R0=2$, un pointeur sur la chaîne \emph{<<two\textbackslash{}n>>} sera chargé
par \ADREQ dans \Reg{0}.

\subsubsection{ARM: \OptimizingKeilVI (\ThumbMode)}

\lstinputlisting[style=customasmARM]{patterns/08_switch/1_few/few_ARM_thumb_O3.asm}

% FIXME а каким можно? к каким нельзя? \myref{} ->

Comme il y déjà été dit, il n'est pas possible d'ajouter un prédicat conditionnel
à la plupart des instructions en mode Thumb, donc ce dernier est quelque peu similaire
au code \ac{CISC}-style x86, facilement compréhensible.

\subsubsection{ARM64: GCC (Linaro) 4.9 \NonOptimizing}

\lstinputlisting[style=customasmARM]{patterns/08_switch/1_few/ARM64_GCC_O0_FR.lst}

Le type de la valeur d'entrée est \Tint, par conséquent le registre \RegW{0} est
utilisé pour garder la valeur au lieu du registre complet \RegX{0}.

Les pointeurs de chaîne sont passés à \puts en utilisant la paire d'instructions
\INS{ADRP}/\INS{ADD} comme expliqué dans l'exemple \q{\HelloWorldSectionName}:~\myref{pointers_ADRP_and_ADD}.

\subsubsection{ARM64: GCC (Linaro) 4.9 \Optimizing}

\lstinputlisting[style=customasmARM]{patterns/08_switch/1_few/ARM64_GCC_O3_FR.lst}

Ce morceau de code est mieux optimisé.
L'instruction \TT{CBZ} (\emph{Compare and Branch on Zero} comparer et sauter si zéro)
effectue un saut si \RegW{0} vaut zéro.
Il y a alors un saut direct à \puts au lieu de l'appeler, comme cela a été expliqué
avant:~\myref{JMP_instead_of_RET}.
}
\IT{\subsubsection{ARM: \OptimizingKeilVI (\ARMMode)}
\myindex{\CLanguageElements!switch}

\lstinputlisting[style=customasmARM]{patterns/08_switch/1_few/few_ARM_ARM_O3.asm}

Nuovamente, analizzando questo codice, non possiamo dire con certezza se originariamente nel sorgente ci fosse uno vero e proprio switch
o una serie di istruzioni if().

\myindex{ARM!\Instructions!ADRcc}

Ad ogni modo vediamo istruzioni condizionali (dette anche \emph{predicated instructions}) come \ADREQ (\emph{Equal})
che viene eseguita solo nel caso $R0=0$, e carica l'indirizzo della stringa \emph{<<zero\textbackslash{}n>>}
into \Reg{0}.
\myindex{ARM!\Instructions!BEQ}
La successiva istruzione \ac{BEQ} redirige il controllo del flusso a \TT{loc\_170}, se $R0=0$.

Un lettore attento potrebbe chiedersi se \ac{BEQ} sarà attivata correttamente, dal momento che \ADREQ
ha riempito prima il registro \Reg{0} con un altro valore.

Si, sarà eseguita correttamente perchè \ac{BEQ} controlla i flag settati dall'istruzione \CMP, 
e \ADREQ non modifica alcun flag.

Il resto delle istruzioni ci sono già familiari. C'è solo una chiamata a \printf, alla fine, ed abbiamo già esaminato
questo trucco qui ~(\myref{ARM_B_to_printf}).
A conti fatti, ci sono tre percorsi che portano alla \printf{}.

\myindex{ARM!\Instructions!ADRcc}
\myindex{ARM!\Instructions!CMP}
L'ultima istruzione, \TT{CMP R0, \#2}, è necessria per controllare se $a=2$.

Se la condizione non è vera, allora \ADRNE carica un puntatore alla stringa \emph{<<something unknown \textbackslash{}n>>}
nel registro \Reg{0}, poiché la variabile $a$ è stata già confrontata 0 e 1 e siamo certi, a questo punto, 
che non sia uguale a tali valori.
E se $R0=2$, il puntatore alla stringa \emph{<<two\textbackslash{}n>>} sarà caricato in \Reg{0} da \ADREQ.

\subsubsection{ARM: \OptimizingKeilVI (\ThumbMode)}

\lstinputlisting[style=customasmARM]{patterns/08_switch/1_few/few_ARM_thumb_O3.asm}

% FIXME а каким можно? к каким нельзя? \myref{} ->

Come già detto in precedenza, non è possibile aggiungere predicati condizionali alla maggior parte di istruzioni in modalità
Thumb, pertanto il codice Thumb qui mostrato è piuttosto simile a quello x86 \ac{CISC}-style facilmente comprensibile.

\subsubsection{ARM64: \NonOptimizing GCC (Linaro) 4.9}

\lstinputlisting[style=customasmARM]{patterns/08_switch/1_few/ARM64_GCC_O0_EN.lst}

Il tipo di valore in input è \Tint, perciò per memorizzarlo viene usato il registro \RegW{0} anziché l'intero registro \RegX{0}.

I puntatori alle stringhe sono passati a \puts tramite una coppia di istruzioni \INS{ADRP}/\INS{ADD} secondo quanto già dimostrato
nell'esempio \q{\HelloWorldSectionName}:~\myref{pointers_ADRP_and_ADD}.

\subsubsection{ARM64: \Optimizing GCC (Linaro) 4.9}

\lstinputlisting[style=customasmARM]{patterns/08_switch/1_few/ARM64_GCC_O3_EN.lst}

Codice maggiormente ottimizzato.
L'istruzione \TT{CBZ} (\emph{Compare and Branch on Zero}) salta se \RegW{0} è zero.
C'è anche un salto diretto a \puts invece di una chiamata, come spiegato in precedenza:~\myref{JMP_instead_of_RET}.

}
\JA{\subsubsection{ARM: \OptimizingKeilVI (\ARMMode)}
\myindex{\CLanguageElements!switch}

\lstinputlisting[style=customasmARM]{patterns/08_switch/1_few/few_ARM_ARM_O3.asm}

繰り返しますが、このコードを調べることで、元のソースコードのswitch()か
単なるif()文の集合かどうかはわかりません。

\myindex{ARM!\Instructions!ADRcc}

とにかく、ここでは、 $R0=0$ の場合にのみトリガされる \ADREQ(\emph{Equal})のような述語命令を再度参照し、
文字列 \emph{<<zero\textbackslash{}n>>} を \Reg{0} にコピーします。
\myindex{ARM!\Instructions!BEQ}
$R0=0$ の場合、次の命令\ac{BEQ}は制御フローを\TT{loc\_170}にリダイレクトします。

巧妙な読者は、\Reg{0}レジスタに既に値を埋め込んでいるので、
\ac{BEQ}が正しくトリガされるかどうかを尋ねるかもしれません。

はい、\ac{BEQ}は\CMP 命令で設定されたフラグをチェックし、
\ADREQ はフラグをまったく変更しません。

命令の残りの部分は既に慣れ親しんでいます。
最後に \printf を1回呼び出すだけですが、ここではこのトリック~(\myref{ARM_B_to_printf})を既に調べています。
最後に \printf{} には3つのパスがあります。

\myindex{ARM!\Instructions!ADRcc}
\myindex{ARM!\Instructions!CMP}
$a=2$ かどうかを確認するには、最後の命令\TT{CMP R0, \#2}が必要です。

それが真でない場合、\ADRNE は $a$ がすでに0に等しいとチェックされているので、
\emph{<<something unknown \textbackslash{}n>>}文字列へのポインタを\Reg{0}にロードします 
または1であり、この時点で $a$ 変数がこれらの数値と等しくないことがわかります。
$R0=2$ の場合、文字列 \emph{<<two\textbackslash{}n>>} へのポインタは \ADREQ によって\Reg{0}にロードされます。

\subsubsection{ARM: \OptimizingKeilVI (\ThumbMode)}

\lstinputlisting[style=customasmARM]{patterns/08_switch/1_few/few_ARM_thumb_O3.asm}

% FIXME а каким можно? к каким нельзя? \myref{} ->

既に言及したように、Thumbモードのほとんどの命令に条件付き述語を追加することはできないため、
ここのThumbコードは、わかりやすいx86 \ac{CISC}スタイルコードと多少似ています。

\subsubsection{ARM64: \NonOptimizing GCC (Linaro) 4.9}

\lstinputlisting[style=customasmARM]{patterns/08_switch/1_few/ARM64_GCC_O0_JA.lst}

入力値のタイプは \Tint なので、 \RegX{0} レジスタ全体ではなくレジスタ \RegW{0} が使用されます。

文字列ポインタは\INS{ADRP}/\INS{ADD}命令ペアを使用して\q{\HelloWorldSectionName}
の例~\myref{pointers_ADRP_and_ADD}と同じように \puts に渡されます。

\subsubsection{ARM64: \Optimizing GCC (Linaro) 4.9}

\lstinputlisting[style=customasmARM]{patterns/08_switch/1_few/ARM64_GCC_O3_JA.lst}

より最適化されたコード。
\Reg{0}がゼロの場合、\TT{CBZ}(\emph{Compare and Branch on Zero})命令はジャンプします。
また、\myref{JMP_instead_of_RET}の前に説明したように、 \puts を呼び出す代わりに直接ジャンプすることもできます。
}

\EN{\subsubsection{MIPS}

\lstinputlisting[caption=\Optimizing GCC 4.4.5 (IDA),style=customasmMIPS]{patterns/08_switch/1_few/MIPS_O3_IDA_EN.lst}

\myindex{MIPS!\Instructions!JR}

The function always ends with calling \puts, so here we see a jump to \puts (\INS{JR}: \q{Jump Register}) instead of \q{jump and link}.
We talked about this earlier: \myref{JMP_instead_of_RET}.

\myindex{MIPS!Load delay slot}
We also often see \INS{NOP} instructions after \INS{LW} ones.
This is \q{load delay slot}: another \emph{delay slot} in MIPS.
\myindex{MIPS!\Instructions!LW}

An instruction next to \INS{LW} may execute at the moment while \INS{LW} loads value from memory. 

However, the next instruction must not use the result of \INS{LW}.

Modern MIPS CPUs have a feature to wait if the next instruction uses result of \INS{LW}, so this is somewhat outdated,
but GCC still adds NOPs for older MIPS CPUs.
In general, it can be ignored.
}
\RU{\subsubsection{MIPS}

\lstinputlisting[caption=\Optimizing GCC 4.4.5 (IDA),style=customasmMIPS]{patterns/08_switch/1_few/MIPS_O3_IDA_RU.lst}

\myindex{MIPS!\Instructions!JR}

Функция всегда заканчивается вызовом \puts, так что здесь мы видим переход на \puts (\INS{JR}: \q{Jump Register})
вместо перехода с сохранением \ac{RA} (\q{jump and link}).

Мы говорили об этом ранее: \myref{JMP_instead_of_RET}.

\myindex{MIPS!Load delay slot}
Мы также часто видим NOP-инструкции после \INS{LW}.
Это \q{load delay slot}: ещё один \emph{delay slot} в MIPS.
\myindex{MIPS!\Instructions!LW}
Инструкция после \INS{LW} может исполняться в тот момент, когда \INS{LW} загружает значение из памяти.

Впрочем, следующая инструкция не должна использовать результат \INS{LW}.

Современные MIPS-процессоры ждут, если следующая инструкция использует результат \INS{LW}, так что всё это уже
устарело, но GCC всё еще добавляет NOP-ы для более старых процессоров.

Вообще, это можно игнорировать.

}
\DE{\subsubsection{MIPS}

\lstinputlisting[caption=\Optimizing GCC 4.4.5 (IDA),style=customasmMIPS]{patterns/08_switch/1_few/MIPS_O3_IDA_DE.lst}

\myindex{MIPS!\Instructions!JR}
Die Funktion endet stets mit einem Aufruf von \puts, weshalb wir hier einen Sprung zu \puts (\INS{JR}: \q{Jump
Register}) anstelle von \q{jump and link} finden.
Dieses Feature haben wir bereit in \myref{JMP_instead_of_RET} besprochen.

\myindex{MIPS!Load delay slot}
Wir finden auch oft \INS{NOP} Befehle nach \INS{LW} Befehlen.
Dies ist \q{load delay slot}: ein anderer \emph{delay slot} in MIPS.
\myindex{MIPS!\Instructions!LW}
Ein Befehl neben \INS{LW} kann in dem Moment ausgeführt werden, in dem \INS{LW} Werte aus dem Speicher lädt.
Der nächste Befehl muss aber nicht das Ergebnis von \INS{LW} verwenden.
Moderne MIPS CPUs haben die Eigenschaft abwarten zu können, ob der folgende Befehl das Ergebnis von \INS{LW} verwendet,
sodass dieses Vorgehen überholt wirkt, aber GCC fügt für ältere MIPS CPUs immer noch NOPs hinzu.
Im Allgemeinen können diese aber ignoriert werden.}
\FR{\subsubsection{MIPS}

\lstinputlisting[caption=GCC 4.4.5 \Optimizing (IDA),style=customasmMIPS]{patterns/08_switch/1_few/MIPS_O3_IDA_FR.lst}

\myindex{MIPS!\Instructions!JR}

La fonction se termine toujours en appelant \puts, donc nous voyons un saut à \puts
(\INS{JR}: \q{Jump Register}) au lieu de \q{jump and link}.
Nous avons parlé de ceci avant: \myref{JMP_instead_of_RET}.

\myindex{MIPS!Load delay slot}
Nous voyons aussi souvent l'instruction \INS{NOP} après \INS{LW}.
Ceci est le slot de délai de chargement (\q{load delay slot}): un autre slot de
délai (\emph{delay slot}) en MIPS.
\myindex{MIPS!\Instructions!LW}

Une instruction suivant \INS{LW} peut s'exécuter pendant que \INS{LW} charge une
valeur depuis la mémoire.

Toutefois, l'instruction suivante ne doit pas utiliser le résultat de \INS{LW}.

Les CPU MIPS modernes ont la capacité d'attendre si l'instruction suivante utilise
le résultat de \INS{LW}, donc ceci est un peu démodé, mais GCC ajoute toujours
des NOPs pour les anciens CPU MIPS.
En général, ça peut être ignoré.
}
\IT{\subsubsection{MIPS}

\lstinputlisting[caption=\Optimizing GCC 4.4.5 (IDA),style=customasmMIPS]{patterns/08_switch/1_few/MIPS_O3_IDA_EN.lst}

\myindex{MIPS!\Instructions!JR}

La funzione finisce sempre col chiamare \puts, e quindi qui vediamo un jump a \puts (\INS{JR}: \q{Jump Register}) invece di \q{jump and link}.
Abbiamo già discusso questo argomento qui: \myref{JMP_instead_of_RET}.

\myindex{MIPS!Load delay slot}
Vediamo anche spesso delle istruzioni \INS{NOP} dopo le istruzioni \INS{LW}.
Si tratta di \q{load delay slot}: un altro tipo di \emph{delay slot} in MIPS.
\myindex{MIPS!\Instructions!LW}

Un'istruzione immediatamente successiva a \INS{LW} potrebbe essere eseguita mentre \INS{LW} carica il valore dalla memoria.
Questa istruzione, comunque, non deve usare il risultato di \INS{LW}.

Le moderne CPU MIPS hanno una funzionalità che consente di attendere, nel caso in cui l'istruzione successiva usi il risultato di 
\INS{LW}, peranto questo tipo codice è piuttosto antiquato e può essere ignorato. GCC continua ad aggiungere i NOP a favore delle cpu MIPS più vecchie.

}
\JA{\subsubsection{MIPS}

\lstinputlisting[caption=\Optimizing GCC 4.4.5 (IDA),style=customasmMIPS]{patterns/08_switch/1_few/MIPS_O3_IDA_JA.lst}

\myindex{MIPS!\Instructions!JR}

関数は常に \puts を呼び出すことで終了するので、\puts(\INS{JR}:\q{Jump Register})へのジャンプは\q{jump and link}ではなく、ここにあります。
私たちは以前これについて話しました:\myref{JMP_instead_of_RET}

\myindex{MIPS!Load delay slot}
\INS{LW}命令の後に\INS{NOP}命令も表示されることがよくあります。
これは\q{load delay slot}:MIPSの別の\emph{delay slot}です。
\myindex{MIPS!\Instructions!LW}

\INS{LW}がメモリから値をロードする間に、\INS{LW}の次の命令が実行されることがあります。

ただし、次の命令は\INS{LW}の結果を使用してはなりません。

現代のMIPS CPUは、次の命令が\INS{LW}の結果を使用するのを待つ機能を持っているので、これは幾分時代遅れですが、
GCCは古いMIPS CPU用にNOPを追加します。
一般に、無視することができます。
}

\subsubsection{\Conclusion{}}

\EN{A \emph{switch()} with few cases is indistinguishable from an \emph{if/else} construction, for example:}
\RU{Оператор \emph{switch()} с малым количеством вариантов трудно отличим от применения конструкции \emph{if/else}:}
\DEph{}
\FR{Un \emph{switch()} avec peu de cas est indistinguable d'une construction avec \emph{if/else}, par exemple:}
\IT{Uno \emph{switch()} con un piccolo numero di casi è indistinguibile da un costrutto \emph{if/else} , per esempio:}
\JA{ほとんどの場合switch()はif / else構造と区別できません:}
\lstref{switch_few_ifelse}.



\EN{\subsection{A lot of cases}

If a \TT{switch()} statement contains a lot of cases, it is not very convenient for the compiler to emit too large code
with a lot \JE/\JNE instructions.

\lstinputlisting[label=switch_lot_c,style=customc]{patterns/08_switch/2_lot/lot.c}

\subsubsection{x86}

\myparagraph{\NonOptimizing MSVC}

We get (MSVC 2010):

\lstinputlisting[caption=MSVC 2010,style=customasmx86]{patterns/08_switch/2_lot/lot_msvc_EN.asm}

\myindex{jumptable}

What we see here is a set of \printf calls with various arguments. 
All they have not only addresses in the memory of the process, but also internal symbolic labels assigned 
by the compiler. 
All these labels are also mentioned in the \TT{\$LN11@f} internal table.

At the function start, if $a$ is greater than 4, control flow is passed to label 
\TT{\$LN1@f}, where \printf with argument \TT{'something unknown'} is called.

But if the value of $a$ is less or equals to 4, then it gets multiplied by 4 and added with the \TT{\$LN11@f} 
table address. That is how an address inside the table is constructed, pointing exactly to the 
element we need. For example, let's say $a$ is equal to 2. $2*4 = 8$ (all table elements 
are addresses in a 32-bit process and that is why all elements are 4 bytes wide). 
The address of the \TT{\$LN11@f} table + 8 is the table element where the \TT{\$LN4@f} label is stored.
\JMP fetches the \TT{\$LN4@f} address from the table and jumps to it.

This table is sometimes called \emph{jumptable} or \emph{branch table}\footnote{The whole method was once called 
\emph{computed GOTO} in early versions of Fortran:
\href{http://go.yurichev.com/17122}{wikipedia}.
Not quite relevant these days, but what a term!}.

Then the corresponding \printf is called with argument \TT{'two'}.\\
Literally, the \TT{jmp DWORD PTR \$LN11@f[ecx*4]} instruction implies
\emph{jump to the DWORD that is stored at address} \TT{\$LN11@f + ecx * 4}.

\TT{npad} (\myref{sec:npad}) is an assembly language macro that align the next label so that it will be stored at an address aligned on a 4 bytes
(or 16 bytes) boundary.
This is very suitable for the processor since it is able to fetch 32-bit values from memory through the memory bus,
cache memory, etc., in a more effective way if it is aligned.

\input{patterns/08_switch/2_lot/olly_EN}

\myparagraph{\NonOptimizing GCC}
\label{switch_lot_GCC}

Let's see what GCC 4.4.1 generates:

\lstinputlisting[caption=GCC 4.4.1,style=customasmx86]{patterns/08_switch/2_lot/lot_gcc.asm}

\myindex{x86!\Registers!JMP}

It is almost the same, with a little nuance: argument \TT{arg\_0} is multiplied by 4 by
shifting it to left by 2 bits (it is almost the same as multiplication by 4)~(\myref{SHR}).
Then the address of the label is taken from the \TT{off\_804855C} array, stored in 
\EAX, and then \TT{JMP EAX} does the actual jump.


\subsubsection{ARM: \OptimizingKeilVI (\ARMMode)}
\label{sec:SwitchARMLot}

\lstinputlisting[caption=\OptimizingKeilVI (\ARMMode),style=customasmARM]{patterns/08_switch/2_lot/lot_ARM_ARM_O3.asm}

This code makes use of the ARM mode feature in which all instructions have a fixed size of 4 bytes.

Let's keep in mind that the maximum value for $a$ is 4 and any greater value will cause
\emph{<<something unknown\textbackslash{}n>>} string to be printed.

\myindex{ARM!\Instructions!CMP}
\myindex{ARM!\Instructions!ADDCC}
The first \TT{CMP R0, \#5} instruction compares the input value of $a$ with 5.

\footnote{ADD---addition}
The next \TT{ADDCC PC, PC, R0,LSL\#2} instruction is being executed only if $R0 < 5$ (\emph{CC=Carry clear / Less than}). 
Consequently, if \TT{ADDCC} does not trigger (it is a $R0 \geq 5$ case), a jump to \emph{default\_case} label will occur.

But if $R0 < 5$ and \TT{ADDCC} triggers, the following is to be happen:

The value in \Reg{0} is multiplied by 4.
In fact, \TT{LSL\#2} at the instruction's suffix stands for \q{shift left by 2 bits}.
But as we will see later~(\myref{division_by_shifting}) in section \q{\ShiftsSectionName}, 
shift left by 2 bits is equivalent to multiplying by 4.

Then we add $R0*4$ to the current value in \ac{PC}, 
thus jumping to one of the \TT{B} (\emph{Branch}) instructions located below.

At the moment of the execution of \TT{ADDCC}, the value in \ac{PC} is 8 bytes ahead (\TT{0x180})
than the address at which the \TT{ADDCC} instruction is located (\TT{0x178}), 
or, in other words, 2 instructions ahead.

\myindex{ARM!Pipeline}

This is how the pipeline in ARM processors works: when \TT{ADDCC} is executed,
the processor at the moment
is beginning to process the instruction after the next one,
so that is why \ac{PC} points there. This has to be memorized.

If $a=0$, then is to be added to the value in \ac{PC},
and the actual value of the \ac{PC} will be written into \ac{PC} (which is 8 bytes ahead)
and a jump to the label \emph{loc\_180} will happen,
which is 8 bytes ahead of the point where the \TT{ADDCC} instruction is.

If $a=1$, then $PC+8+a*4 = PC+8+1*4 = PC+12 = 0x184$ will be written to \ac{PC},
which is the address of the \emph{loc\_184} label.

With every 1 added to $a$, the resulting \ac{PC} is increased by 4.

4 is the instruction length in ARM mode and also, the length of each \TT{B} instruction,
of which there are 5 in row.

Each of these five \TT{B} instructions passes control further, to what was programmed in the \emph{switch()}.

Pointer loading of the corresponding string occurs there, etc.

\subsubsection{ARM: \OptimizingKeilVI (\ThumbMode)}

\lstinputlisting[caption=\OptimizingKeilVI (\ThumbMode),style=customasmARM]{patterns/08_switch/2_lot/lot_ARM_thumb_O3.asm}

\myindex{ARM!\ThumbMode}
\myindex{ARM!\ThumbTwoMode}

One cannot be sure that all instructions in Thumb and Thumb-2 modes has the same size.
It can even be said that in these modes the instructions have variable lengths, just like in x86.

\myindex{jumptable}

So there is a special table added that contains information about how much cases are there (not including 
default-case), and an offset for each with a label to which control must be passed in 
the corresponding case.

\myindex{ARM!Mode switching}
\myindex{ARM!\Instructions!BX}

A special function is present here in order to deal with the table and pass control, \\
named \emph{\_\_ARM\_common\_switch8\_thumb}. 
It starts with \TT{BX PC}, whose function is to switch the processor to ARM-mode.
Then you see the function for table processing. 

It is too advanced to describe it here now, so let's omit it.
% TODO explain it...

\myindex{ARM!\Registers!Link Register}

It is interesting to note that the function uses the \ac{LR} register as a pointer to the table.

Indeed, after calling of this function, \ac{LR} contains the address after\\
\TT{BL \_\_ARM\_common\_switch8\_thumb} instruction, where the table starts.

It is also worth noting that the code is generated as a separate function in order to reuse it, 
so the compiler doesn't generate the same code for every switch() statement.

\IDA successfully perceived it as a service function and a table, and added comments to the labels like\\
\TT{jumptable 000000FA case 0}.


\subsubsection{MIPS}

\lstinputlisting[caption=\Optimizing GCC 4.4.5 (IDA),style=customasmMIPS]{patterns/08_switch/2_lot/MIPS_O3_IDA_EN.lst}

\myindex{MIPS!\Instructions!SLTIU}

The new instruction for us is \INS{SLTIU} (\q{Set on Less Than Immediate Unsigned}).
\myindex{MIPS!\Instructions!SLTU}

This is the same as \INS{SLTU} (\q{Set on Less Than Unsigned}), but \q{I} stands for \q{immediate}, 
i.e., a number has to be specified in the instruction itself.

\myindex{MIPS!\Instructions!BNEZ}
\INS{BNEZ} is \q{Branch if Not Equal to Zero}.

Code is very close to the other \ac{ISA}s.
\myindex{MIPS!\Instructions!SLL}
\INS{SLL} (\q{Shift Word Left Logical}) does multiplication by 4.

MIPS is a 32-bit CPU after all, so all addresses in the \emph{jumptable} are 32-bit ones.



\subsubsection{\Conclusion{}}

Rough skeleton of \emph{switch()}:

% TODO: ARM, MIPS skeleton
\lstinputlisting[caption=x86,style=customasmx86]{patterns/08_switch/2_lot/skel1_EN.lst}

The jump to the address in the jump table may also be implemented using this instruction: \\
\TT{JMP jump\_table[REG*4]}.
Or \TT{JMP jump\_table[REG*8]} in x64.

A \emph{jumptable} is just array of pointers, like the one described later: \myref{array_of_pointers_to_strings}.
}
\RU{\subsection{И если много}

Если ветвлений слишком много, то генерировать слишком длинный код с многочисленными \JE/\JNE 
уже не так удобно.

\lstinputlisting[label=switch_lot_c,style=customc]{patterns/08_switch/2_lot/lot.c}

\subsubsection{x86}

\myparagraph{\NonOptimizing MSVC}

Рассмотрим пример, скомпилированный в (MSVC 2010):

\lstinputlisting[caption=MSVC 2010,style=customasmx86]{patterns/08_switch/2_lot/lot_msvc_RU.asm}

\myindex{jumptable}
Здесь происходит следующее: в теле функции есть набор вызовов \printf с разными аргументами. 
Все они имеют, конечно же, адреса, а также внутренние символические метки, которые присвоил им компилятор.
Также все эти метки указываются во внутренней таблице \TT{\$LN11@f}.

В начале функции, если $a$ больше 4, то сразу происходит переход на метку \TT{\$LN1@f}, 
где вызывается \printf с аргументом \TT{'something unknown'}.

А если $a$ меньше или равно 4, то это значение умножается на 4 и прибавляется адрес таблицы 
с переходами (\TT{\$LN11@f}). 
Таким образом, получается адрес внутри таблицы, где лежит нужный адрес внутри тела функции. 
Например, возьмем $a$ равным 2. $2*4 = 8$ (ведь все элементы таблицы~--- это адреса внутри 32-битного процесса, 
таким образом, каждый элемент занимает 4 байта). 8 прибавить к \TT{\$LN11@f}~--- это будет элемент таблицы,
где лежит \TT{\$LN4@f}. \JMP вытаскивает из таблицы адрес \TT{\$LN4@f} и делает безусловный переход туда.

Эта таблица иногда называется \emph{jumptable} или
\emph{branch table}\footnote{Сам метод раньше назывался 
\emph{computed GOTO} В ранних версиях Фортрана:
\href{http://go.yurichev.com/17122}{wikipedia}.
Не очень-то и полезно в наше время, но каков термин!}.

А там вызывается \printf с аргументом \TT{'two'}. 
Дословно, инструкция \TT{jmp DWORD PTR \$LN11@f[ecx*4]} 
означает \emph{перейти по DWORD, который лежит по адресу} \TT{\$LN11@f + ecx * 4}.

\TT{npad} (\myref{sec:npad}) это макрос ассемблера, выравнивающий начало таблицы, 
чтобы она располагалась по адресу кратному 4 (или 16).
Это нужно для того, чтобы процессор мог эффективнее загружать 32-битные 
значения из памяти через шину с памятью, кэш-память, итд.

\input{patterns/08_switch/2_lot/olly_RU}

\myparagraph{\NonOptimizing GCC}
\label{switch_lot_GCC}

Посмотрим, что сгенерирует GCC 4.4.1:

\lstinputlisting[caption=GCC 4.4.1,style=customasmx86]{patterns/08_switch/2_lot/lot_gcc.asm}

\myindex{x86!\Registers!JMP}
Практически то же самое, за исключением мелкого нюанса: аргумент из \TT{arg\_0} умножается на 4 
при помощи сдвига влево на 2 бита (это почти то же самое что и умножение на 4)~(\myref{SHR}).
Затем адрес метки внутри функции берется из массива \TT{off\_804855C} и адресуется при помощи 
вычисленного индекса.


\subsubsection{ARM: \OptimizingKeilVI (\ARMMode)}
\label{sec:SwitchARMLot}

\lstinputlisting[caption=\OptimizingKeilVI (\ARMMode),style=customasmARM]{patterns/08_switch/2_lot/lot_ARM_ARM_O3.asm}

В этом коде используется та особенность режима ARM, 
что все инструкции в этом режиме имеют фиксированную длину 4 байта.

Итак, не будем забывать, что максимальное значение для $a$ это 4: всё что выше, должно вызвать
вывод строки \emph{<<something unknown\textbackslash{}n>>}.

\myindex{ARM!\Instructions!CMP}
\myindex{ARM!\Instructions!ADDCC}
Самая первая инструкция \TT{CMP R0, \#5} сравнивает входное значение в $a$ c 5.

\footnote{ADD---складывание чисел}
Следующая инструкция \TT{ADDCC PC, PC, R0,LSL\#2} сработает только в случае если $R0 < 5$ (\emph{CC=Carry clear / Less than}). 
Следовательно, если \TT{ADDCC} не сработает (это случай с $R0 \geq 5$), выполнится переход на метку 
\emph{default\_case}.

Но если $R0 < 5$ и \TT{ADDCC} сработает, то произойдет следующее.

Значение в \Reg{0} умножается на 4.
Фактически, \TT{LSL\#2} в суффиксе инструкции означает \q{сдвиг влево на 2 бита}.

Но как будет видно позже~(\myref{division_by_shifting}) в секции \q{\ShiftsSectionName}, 
сдвиг влево на 2 бита, это эквивалентно его умножению на 4.

Затем полученное $R0*4$ прибавляется к текущему значению \ac{PC}, 
совершая, таким образом, переход на одну из расположенных ниже инструкций \TT{B} (\emph{Branch}).

На момент исполнения \TT{ADDCC},
содержимое \ac{PC} на 8 байт больше (\TT{0x180}), чем адрес по которому расположена сама инструкция \TT{ADDCC} (\TT{0x178}), 
либо, говоря иным языком, на 2 инструкции больше.

\myindex{ARM!Конвейер}
Это связано с работой конвейера процессора ARM:
пока исполняется инструкция \TT{ADDCC}, процессор уже начинает обрабатывать инструкцию после следующей, 
поэтому \ac{PC} указывает туда. Этот факт нужно запомнить.

Если $a=0$, тогда к \ac{PC} ничего не будет прибавлено и 
в \ac{PC} запишется актуальный на тот момент \ac{PC} (который больше на 8) 
и произойдет переход на метку \emph{loc\_180}. 
Это на 8 байт дальше места, где находится инструкция \TT{ADDCC}.

Если $a=1$, тогда в \ac{PC} запишется 
$PC+8+a*4 = PC+8+1*4 = PC+12 = 0x184$. Это адрес метки \emph{loc\_184}.

При каждой добавленной к $a$ единице итоговый \ac{PC} увеличивается на 4.

4 это длина инструкции в режиме ARM и одновременно с этим, 
длина каждой инструкции \TT{B}, их здесь следует 5 в ряд.

Каждая из этих пяти инструкций \TT{B} передает управление дальше, где собственно и происходит то, 
что запрограммировано в операторе \emph{switch()}.
Там происходит загрузка указателя на свою строку, итд.

\subsubsection{ARM: \OptimizingKeilVI (\ThumbMode)}

\lstinputlisting[caption=\OptimizingKeilVI (\ThumbMode),style=customasmARM]{patterns/08_switch/2_lot/lot_ARM_thumb_O3.asm}

\myindex{ARM!\ThumbMode}
\myindex{ARM!\ThumbTwoMode}
В режимах Thumb и Thumb-2 уже нельзя надеяться на то, что все инструкции имеют одну длину.

Можно даже сказать, что в этих режимах инструкции переменной длины, как в x86.

\myindex{jumptable}
Так что здесь добавляется специальная таблица, содержащая информацию о том, как много вариантов здесь,
не включая варианта по умолчанию, и смещения, для каждого варианта. Каждое смещение кодирует метку, куда нужно передать
управление в соответствующем случае.

\myindex{ARM!Переключение режимов}
\myindex{ARM!\Instructions!BX}
Для того чтобы работать с таблицей и совершить переход, вызывается служебная функция

\emph{\_\_ARM\_common\_switch8\_thumb}. 
Она начинается с инструкции \TT{BX PC}, чья функция~--- переключить процессор в ARM-режим.

Далее функция, работающая с таблицей. 
Она слишком сложная для рассмотрения в данном месте, так что пропустим это.

% TODO explain it...

\myindex{ARM!\Registers!Link Register}
Но можно отметить, что эта функция использует регистр \ac{LR} как указатель на таблицу.

Действительно, после вызова этой функции, в \ac{LR} был записан адрес после инструкции

\TT{BL \_\_ARM\_common\_switch8\_thumb}, а там как раз и начинается таблица.

Ещё можно отметить, что код для этого выделен в отдельную функцию для того, 
чтобы не нужно было каждый раз генерировать 
точно такой же фрагмент кода для каждого выражения switch().

\IDA распознала эту служебную функцию и таблицу автоматически дописала комментарии к меткам вроде \\
\TT{jumptable 000000FA case 0}.


\subsubsection{MIPS}

\lstinputlisting[caption=\Optimizing GCC 4.4.5 (IDA),style=customasmMIPS]{patterns/08_switch/2_lot/MIPS_O3_IDA_RU.lst}

\myindex{MIPS!\Instructions!SLTIU}
Новая для нас инструкция здесь это \INS{SLTIU} (\q{Set on Less Than Immediate Unsigned}~--- установить,
если меньше чем значение, беззнаковое сравнение).

\myindex{MIPS!\Instructions!SLTU}
На самом деле, это то же что и \INS{SLTU} (\q{Set on Less Than Unsigned}), но \q{I} означает \q{immediate},
т.е. число может быть задано в самой инструкции.

\myindex{MIPS!\Instructions!BNEZ}
\INS{BNEZ} это \q{Branch if Not Equal to Zero} (переход если не равно нулю).

Код очень похож на код для других \ac{ISA}.
\myindex{MIPS!\Instructions!SLL}
\INS{SLL} (\q{Shift Word Left Logical}~--- логический сдвиг влево) совершает умножение на 4.
MIPS всё-таки это 32-битный процессор, так что все адреса в таблице переходов (\emph{jumptable}) 32-битные.



\subsubsection{\Conclusion{}}

Примерный скелет оператора \emph{switch()}:

% TODO: ARM, MIPS skeleton
\lstinputlisting[caption=x86,style=customasmx86]{patterns/08_switch/2_lot/skel1_RU.lst}

Переход по адресу из таблицы переходов может быть также реализован такой инструкцией: \\
\TT{JMP jump\_table[REG*4]}. Или \TT{JMP jump\_table[REG*8]} в x64.

Таблица переходов (\emph{jumptable}) это просто массив указателей, как это будет вскоре описано: \myref{array_of_pointers_to_strings}.
}
\DE{\subsection{Viele Fälle}
Wenn ein \TT{switch()} Ausdruck viele Fälle enthält, ist es für den Compiler nicht günstig sehr großen Code mit vielen
\JE/\JNE Befehlen zu erzeugen.

\lstinputlisting[label=switch_lot_c,style=customc]{patterns/08_switch/2_lot/lot.c}

\subsubsection{x86}

\myparagraph{\NonOptimizing MSVC}

Wir erhalten (MSVC 2010):

\lstinputlisting[caption=MSVC 2010,style=customasmx86]{patterns/08_switch/2_lot/lot_msvc_DE.asm}

\myindex{jumptable}
Was wir hier sehen ist eine Ansammlung von Aufrufen von \printf mit diversen Argumenten.
Alle haben nicht Adressen im Speicher des Prozesses, sondern auch interne symbolische Labels, die ihnen vom Compiler
zugewiesen werden.
Alle diese Labels werden auch in der internen Tabelle \TT{\$LN11@f} aufgeführt. 

Zu Beginn der Funktion wird der Control Flow an das Label \TT{\$LN1@f} abgegeben, wenn $a$ größer ist als 4. An diesem
Label wird \printf mit dem Argument \TT{'something unknown'} aufgerufen.

Wenn aber der Wert von $a$ kleiner gleich 4 ist, dann wird dieser mit 4 multipliziert und mit der Tabellenadresse
\TT{\$LN11@f} addiert. Auf diese Weise wird die Adresse innerhalb der Tabelle konstruiert und zeigt genau auf das
gewünschte Element. Nehmen wir zum Beispiel an, dass $a$ gleich 2 ist. $2\cdot 4=8$ (alle Tabellenelemente sind
Adressen in einem 32-Bit-Prozess und haben daher eine Breite von 4 Bytes).
Die Adresse an der Stelle \TT{\$LN11@f} + 8 ist das Tabellenelement, an dem das Label \TT{\$LN4@f} gespeichert ist.
\JMP holt die Adresse \TT{\$LN4@f} aus der Tabelle und springt dorthin.

Diese Tabelle wird manchmal \emph{Jumptable} oder \emph{Verzweigungstabelle} genannt\footnote{Die ganze Methode wurde
in früheren Versionen von Fortran \emph{berechnetes GOTO} genannt:
\href{http://go.yurichev.com/17122}{wikipedia}.
Heutzutage zwar nicht mehr relevant, aber welch ein Ausdruck!}.

Dann wird das zugehörige \printf mit dem Argument \TT{'two'} aufgerufen.\\
Der Befehl TT{jmp DWORD PTR \$LN11@f[ecx*4]} bedeutet dabei \emph{springe zum an dieser Stelle gespeicherten
DWORD}\TT{\$LN11@f + ecx * 4}.

\TT{npad} (\myref{sec:npad}) ist ein Assemblermakro, dass das nächste Label so angeordnet, dass es an einer 4 Byte
(oder 16 Bit) Adressgrenze gespeichert wird. Das ist für den Prozessor sehr praktisch, da er die 32-Bit-Werte aus dem
Speicher durch den Speicherbus, den Cache, etc. in effektiverer Weise laden kann.

\input{patterns/08_switch/2_lot/olly_DE}

\myparagraph{\NonOptimizing GCC}
\label{switch_lot_GCC}

Schauen wir was GCC 4.4.1 erzeugt:

\lstinputlisting[caption=GCC 4.4.1,style=customasmx86]{patterns/08_switch/2_lot/lot_gcc.asm}

\myindex{x86!\Registers!JMP}
Es ist bis auf eine Nuance das gleiche: das Argument \TT{arg\_0} wird mit 4 multipliziert durch eine Verschiebung von 2
Bits nach links (dies entspricht einer Multiplikation mit 4)~(\myref{SHR}).
Dann wird die Adresse des Labels vom \TT{off\_804855C} genommen, die in \EAX gespeichert wird, und dann wird mit
\TT{JMP EAX} der eigentliche Sprung durchgeführt.



\subsubsection{ARM: \OptimizingKeilVI (\ARMMode)}
\label{sec:SwitchARMLot}

\lstinputlisting[caption=\OptimizingKeilVI (\ARMMode),style=customasmARM]{patterns/08_switch/2_lot/lot_ARM_ARM_O3.asm}
Dieser Code verwendet das ARM mode Feature, das alle Befehle eine feste Länge von 4 Byte haben.

Vergessen wir nicht, dass der Maximalwert für $a$ 4 beträgt und jeder größere Wert zur Ausgabe des \emph{<<something
unknown\textbackslash{}n>>} Strings führt.

\myindex{ARM!\Instructions!CMP}
\myindex{ARM!\Instructions!ADDCC}
Der erste \TT{CMP R0, \#5} Befehl vergleich den Eingabewert $a$ mit 5.

\footnote{ADD---Addition}
Der nächste \TT{ADDCC PC, PC, R0,LSL\#2} Befehl wird nur ausgeführt, falls $R0 < 5$ (\emph{CC=Carry clear / kleiner als}).
Wenn \TT{ADDCC} nicht ausgeführt wird (d.h. $R0\geq 5$), wird ein Sprung zum \emph{default\_case} Label ausgeführt.

Aber wenn $R0 < 5$ und \TT{ADDCC} ausgeführt wird, wird das Folgende geschehen:

Der Wert in \Reg{0} wird mit 4 multipliziert.
Der Suffix \TT{LSL2} am Befehl steht dabei für \q{shift left by 2 bits}.
Aber wie wir später~(\myref{division_by_shifting}) im Abschnitt \q{\ShiftsSectionName} sehen werden, ist eine
Verschiebung um 2 Bits nach links äquivalent zu einer Multiplikation mit 4.

Danach addieren wir $R0\cdot 4$ zum aktuellen Wert in \ac{PC} und springen dadurch zu einem der unteren \TT{B}
(\emph{Branch}) Befehle.

Im Moment der Ausführung von\TT{ADDCC} ist der Wert von \ac{PC} (\TT{0x180}) 8 Bytes - oder mit anderen Worten: 2
Befehle - größer als die Adresse, an der sich der \TT{ADDCC} Befehl befindet (\TT{0x178})

\myindex{ARM!Pipeline}
So funktioniert die Pipeline in ARM Prozessoren: wenn \TT{ADDCC} ausgeführt wird, beginnt der Prozessor den Befehl
nach dem nächsten abzuarbeiten und deshalb zeigt \ac{PC} hierher. Das müssen wir im Kopf behalten.

Wenn $a=0$, dann wird dies zum Wert in \ac{PC} addiert und der aktuelle Wert des \ac{PC} wird nach \ac{PC} geschrieben
(welcher 8 Byte größer ist) und es wird zum Label \emph{loc\_180} gesprungen, welches 8 Byte größer ist als die Adresse
des \TT{ADDCC} Befehls.

Wenn $a=1$, dann wird $PC+8+a\cdot 4 = PC+8+1\cdot 4 = PC+12 = 0x184$ nach \ac{PC} geschrieben,was der Adresse des
\emph{loc\_184} Labels entspricht.

Jedes Mal wenn $a$ um 1 erhöht wird, erhöht sich der \ac{PC} um 4.

Dabei ist 4 die Länge eines Befehls im ARM mode und auch die Länge jedes \TT{B} Befehls, von denen sich hier 5 befinden.

Jeder dieser fünf \TT{B} Befehle gibt den Control Flow weiter so wie es im \emph{switch()} Ausdruck programmiert wurde.

Hier werden jeweils die Pointer auf die zugehörigen Strings geladen, etc.

\subsubsection{ARM: \OptimizingKeilVI (\ThumbMode)}

\lstinputlisting[caption=\OptimizingKeilVI (\ThumbMode),style=customasmARM]{patterns/08_switch/2_lot/lot_ARM_thumb_O3.asm}

\myindex{ARM!\ThumbMode}
\myindex{ARM!\ThumbTwoMode}
Man kann sich nicht sicher sein, dass alle Befehle im Thumb und Thumb-2 mode dieselbe Größe haben.
Man kann sogar sagen, dass die Befehle hier genau wie in x86 variable Längen haben.

\myindex{jumptable}
Deshalb wird hier eine spezielle Tabelle verwendet, die Informationen darüber enthält wie viele Fälle vorliegen (ohne
den Default-Case) und es wird für jeden Fall ein Label mit einem Offset für den Control Flow im zugehörigen Fall
angegeben.


\myindex{ARM!Mode switching}
\myindex{ARM!\Instructions!BX}
Hier taucht eine spezielle Funktion namens \emph{\_\_ARM\_common\_switch8\_thumb} auf, die mit der Tabelle und der
Übergabe des Control Flows umgeht.
Sie beginnt mit \TT{BX PC}, dessen Aufgabe es ist, den Prozessor in den ARM mode zu versetzen.
Danach finden wir die Funktion für den Umgang mit der Tabelle.

Es ist hier zu fortgeschritten um weiter ins Details zu gehen, daher lassen wir es für den Moment hierbei bewenden. 

% TODO explain it...

\myindex{ARM!\Registers!Link Register}
Ist ist interessant festzustellen, dass die Funktion das \ac{LR} Register als Pointer auf die Tabelle verwendet.

Tatsächlich enthält \ac{LR} nach dem Aufruf der Funktion die Adresse nach dem Befehl\\
\TT{BL \_\_ARM\_common\_switch8\_thumb}, an dem die Tabelle beginnt.

Es ist auch bemerkenswert, dass der Code als eine separate Funktion erzeugt wird, um wiederverwendet werden zu können,
sodass der Compiler nicht für jeden switch() Ausdruck den gleichen Code erzeugen muss.

\IDA hat erfolgreich ermittelt, dass es sich um eine Servicefunktion und eine Tabelle handelt und hat Kommentare wie
etwa \TT{jumptable 000000FA case 0} zu den Labels hinzugefügt.



\subsubsection{MIPS}

\lstinputlisting[caption=\Optimizing GCC 4.4.5 (IDA),style=customasmMIPS]{patterns/08_switch/2_lot/MIPS_O3_IDA_DE.lst}

\myindex{MIPS!\Instructions!SLTIU}
Der für uns neue Befehl ist \INS{SLTIU} (\q{Set on Less Than Immediate Unsigned}).

\myindex{MIPS!\Instructions!SLTU}
Dies ist das gleiche wie \INS{SLTU} (\q{Set on Less Than Unsigned}); das \q{I} steht dabei für \q{immediate}, d.h.
für den Befehl muss eine Zahl angegeben werden. 

\myindex{MIPS!\Instructions!BNEZ}
\INS{BNEZ} ist \q{Branch if Not Equal to Zero}.

Der Code ist den anderen \ac{ISA}s sehr ähnlich.
\myindex{MIPS!\Instructions!SLL}
\INS{SLL} (\q{Shift Word Left Logical}) führt eine Multiplikation mit 4 durch.
Da MIPS eine 32-Bit CPU ist, sind auch die Adressen in der \emph{Jumptable} 32 Bit groß.

\subsubsection{\Conclusion{}}

Das grobe Gerüst eines \emph{switch()}:

% TODO: ARM, MIPS skeleton
\lstinputlisting[caption=x86,style=customasmx86]{patterns/08_switch/2_lot/skel1_DE.lst}
Der Sprung zur Adresse in der Jumptable kann auch durch den folgenden Befehl realisiert werden:\\
\TT{JMP jump\_table[REG*4]}
oder \TT{JMP jump\_table[REG*8]} in x64.

Eine \emph{Jumptable} ist nur ein Array von Pointern, genau wie das hier beschriebene:
\myref{array_of_pointers_to_strings}.
}
\FR{\subsection{De nombreux cas}

Si une déclaration \TT{switch()} contient beaucoup de cas, il n'est pas très pratique
pour le compilateur de générer un trop gros code avec de nombreuses instructions
\JE/\JNE.

\lstinputlisting[label=switch_lot_c,style=customc]{patterns/08_switch/2_lot/lot.c}

\subsubsection{x86}

\myparagraph{MSVC \NonOptimizing}

Nous obtenons (MSVC 2010):

\lstinputlisting[caption=MSVC 2010,style=customasmx86]{patterns/08_switch/2_lot/lot_msvc_FR.asm}

\myindex{jumptable}

Ce que nous voyons ici est un ensemble d'appels à \printf avec des arguments variés.
Ils ont tous, non seulement des adresses dans la mémoire du processus, mais aussi
des labels symboliques internes assignés par le compilateur.
Tous ces labels ont aussi mentionnés dans la table interne \TT{\$LN11@f}.

Au début de la fonctions, si $a$ est supérieur à 4, l'exécution est passée au
labal \TT{\$LN1@f}, où \printf est appelé avec l'argument \TT{'something unknown'}.

Mais si la valeur de $a$ est inférieure ou égale à 4, elle est alors multipliée
par 4 et ajoutée à l'adresse de la table \TT{\$LN11@f}. C'est ainsi qu'une adresse
à l'intérieur de la table est construite, pointant exactement sur l'élément dont
nous avons besoin. Par exemple, supposons que $a$ soit égale à 2. $2*4 = 8$ (tous
les éléments de la table sont adressés dans un processus 32-bit, c'est pourquoi les
éléments ont une taille de 4 octets).
L'adresse de la table \TT{\$LN11@f} + 8 est celle de l'élément de la table où
le label \TT{\$LN4@f} est stocké.
\JMP prend l'adresse de \TT{\$LN4@f} dans la table et y saute.

Cette table est quelquefois appelée \emph{jumptable} (table de saut) ou \emph{branch table}
(table de branchement)\footnote{L'ensemble de la méthode était appelé \emph{computed
GOTO} (GOTO calculés) dans les premières versions de ForTran:
\href{http://go.yurichev.com/17122}{Wikipédia}.
Pas très pertinent de nos jours, mais quel terme!}.

Le \printf correspondant est appelé avec l'argument \TT{'two'}.\\
Littéralement, l'instruction \TT{jmp DWORD PTR \$LN11@f[ecx*4]} signifie
\emph{sauter au DWORD qui est stocké à l'adresse} \TT{\$LN11@f + ecx * 4}.

\TT{npad} (\myref{sec:npad}) est une macro du langage d'assemblage qui aligne le
label suivant de telle sorte qu'il soit stocké à une adresse alignée sur une limite
de 4 octets (ou 16 octets).
C'est très adapté pour le processeur puisqu'il est capable d'aller chercher des
valeurs 32-bit dans la mémoire à travers le bus mémoire, la mémoire cache, etc.,
de façons beaucoup plus efficace si c'est aligné.

\input{patterns/08_switch/2_lot/olly_FR}

\myparagraph{GCC \NonOptimizing}
\label{switch_lot_GCC}

Voyons ce que GCC 4.4.1 génère:

\lstinputlisting[caption=GCC 4.4.1,style=customasmx86]{patterns/08_switch/2_lot/lot_gcc.asm}

\myindex{x86!\Registers!JMP}

C'est presque la même chose, avec une petite nuance: l'argument \TT{arg\_0} est multiplié
par 4 en décalant de 2 bits vers la gauche (c'est presque comme multiplier par 4)~(\myref{SHR}).
Ensuite l'adresse du label est prise depuis le tableau \TT{off\_804855C}, stockée
dans \EAX, et ensuite \TT{JMP EAX} effectue le saut réel.


\subsubsection{ARM: \OptimizingKeilVI (\ARMMode)}
\label{sec:SwitchARMLot}

\lstinputlisting[caption=\OptimizingKeilVI (\ARMMode),style=customasmARM]{patterns/08_switch/2_lot/lot_ARM_ARM_O3.asm}

Ce code utilise les caractéristiques du mode ARM dans lequel toutes les instructions
ont une taille fixe de 4 octets.

Gardons à l'esprit que la valeur maximale de $a$ est 4 et que toute autre valeur
supérieure provoquera l'affichage de la chaîne \emph{<<something unknown\textbackslash{}n>>}

\myindex{ARM!\Instructions!CMP}
\myindex{ARM!\Instructions!ADDCC}
La première instruction \TT{CMP R0, \#5} compare la valeur entrée dans $a$ avec 5.

\footnote{ADD---addition}
L'instruction suivante, \TT{ADDCC PC, PC, R0,LSL\#2}, est exécutée si et seulement
si $R0 < 5$ (\emph{CC=Carry clear / Less than} retenue vide, inférieur à).
Par conséquent, si \TT{ADDCC} n'est pas exécutée (c'est le cas $R0 \geq 5$), un
saut au label \emph{default\_case} se produit.

Mais si $R0 < 5$ et que \TT{ADDCC} est exécuté, voici ce qui se produit:

La valeur dans \Reg{0} est multipliée par 4.
En fait, le suffixe de l'instruction \TT{LSL\#2} signifie \q{décalage à gauche de 2 bits}.
Mais comme nous le verrons plus tard~(\myref{division_by_shifting}) dans la section
\q{\ShiftsSectionName}, décaler de 2 bits vers la gauche est équivalent à multiplier
par 4.

Puis, nous ajoutons $R0*4$ à la valeur courante du \ac{PC}, et sautons à l'une
des instructions \TT{B} (\emph{Branch}) situées plus bas.

Au moment de l'exécution de \TT{ADDCC}, la valeur du \ac{PC} est en avance de 8
octets (\TT{0x180}) sur l'adresse à laquelle l'instruction \TT{ADDCC} se trouve
(\TT{0x178}), ou, autrement dit, en avance de 2 instructions.

\myindex{ARM!Pipeline}

C'est ainsi que le pipeline des processeurs ARM fonctionne: lorsque \TT{ADDCC} est
exécutée, le processeur, à ce moment, commence à préparer les instructions après
la suivante, c'est pourquoi \ac{PC} pointe ici. Cela doit être mémorisé.

Si $a=0$, elle sera ajoutée à la valeur de \ac{PC}, et la valeur courante de \ac{PC}
sera écrite dans \ac{PC} (qui est 8 octets en avant) et un saut au label \emph{loc\_180}
sera effectué, qui est 8 octets en avant du point où l'instruction se trouve.

Si $a=1$, alors $PC+8+a*4 = PC+8+1*4 = PC+12 = 0x184$ sera écrit dans \ac{PC}, qui
est l'adresse du label \emph{loc\_184}.

A chaque fois que l'on ajoute 1 à $a$, le \ac{PC} résultant est incrémenté de
4.

4 est la taille des instructions en mode ARM, et donc, la longueur de chaque instruction
\TT{B} desquelles il y a 5 à la suite.

Chacune de ces cinq instructions \TT{B} passe le contrôle plus loin, à ce qui a
été programmé dans le \emph{switch()}.

Le chargement du pointeur sur la chaîne correspondante se produit ici, etc.

\subsubsection{ARM: \OptimizingKeilVI (\ThumbMode)}

\lstinputlisting[caption=\OptimizingKeilVI (\ThumbMode),style=customasmARM]{patterns/08_switch/2_lot/lot_ARM_thumb_O3.asm}

\myindex{ARM!\ThumbMode}
\myindex{ARM!\ThumbTwoMode}

On ne peut pas être sûr que toutes ces instructions en mode Thumb et Thumb-2 ont
la même taille. On peut même dire que les instructions dans ces modes ont une longueur
variable, tout comme en x86.

\myindex{jumptable}

Donc, une table spéciale est ajoutée, qui contient des informations sur le nombre
de cas (sans inclure celui par défaut), et un offset pour chaque label auquel le
contrôle doit être passé dans chaque cas.

\myindex{ARM!Mode switching}
\myindex{ARM!\Instructions!BX}

Une fonction spéciale est présente ici qui s'occupe de la table et du passage du
contrôle, appelée\\ \emph{\_\_ARM\_common\_switch8\_thumb}.
Elle commence avec \TT{BX PC}, dont la fonction est de passer le mode du processeur
en ARM.
Ensuite, vous voyez la fonction pour le traitement de la table.

C'est trop avancé pour être détaillé ici, donc passons cela.
% TODO explain it...

\myindex{ARM!\Registers!Link Register}

Il est intéressant de noter que la fonction utilise le registre \ac{LR} comme un
pointeur sur la table.

En effet, après l'appel de cette fonction, \ac{LR} contient l'adresse après l'instruction\\
\TT{BL \_\_ARM\_common\_switch8\_thumb}, où la table commence.

Il est intéressant de noter que le code est généré comme une fonction indépendante
afin de la ré-utiliser, donc le compilateur ne générera pas le même code pour chaque
déclaration switch().

\IDA l'a correctement identifié comme une fonction de service et une table, et a
ajouté un commentaire au label comme \TT{jumptable 000000FA case 0}.


\subsubsection{MIPS}

\lstinputlisting[caption=GCC 4.4.5 \Optimizing (IDA),style=customasmMIPS]{patterns/08_switch/2_lot/MIPS_O3_IDA_FR.lst}

\myindex{MIPS!\Instructions!SLTIU}

La nouvelle instruction pour nous est \INS{SLTIU} (\q{Set on Less Than Immediate Unsigned}
Mettre si inférieur à la valeur immédiate non signée).
\myindex{MIPS!\Instructions!SLTU}

Ceci est la même que \INS{SLTU} (\q{Set on Less Than Unsigned}), mais \q{I} signifie
\q{immediate}, i.e., un nombre doit être spécifié dans l'instruction elle-même.

\myindex{MIPS!\Instructions!BNEZ}
\INS{BNEZ} est \q{Branch if Not Equal to Zero}.

Le code est très proche de l'autre \ac{ISA}s.
\myindex{MIPS!\Instructions!SLL}
\INS{SLL} (\q{Shift Word Left Logical}) effectue une multiplication par 4.

MIPS est un CPU 32-bit après tout, donc toutes les adresses de la \emph{jumtable}
sont 32-bits.



\subsubsection{\Conclusion{}}

Squelette grossier d'un \emph{switch()}:

% TODO: ARM, MIPS skeleton
\lstinputlisting[caption=x86,style=customasmx86]{patterns/08_switch/2_lot/skel1_FR.lst}

Le saut à une adresse de la table de saut peut aussi être implémenté en utilisant
cette instruction: \\
\TT{JMP jump\_table[REG*4]}.
Ou \TT{JMP jump\_table[REG*8]} en x64.

Une table de saut est juste un tableau de pointeurs, comme celle décrite plus
loin: \myref{array_of_pointers_to_strings}. 
}
\IT{\subsection{Molti casi}

Se uno statement \TT{switch()} contiene molti casi, per il compilatore non è molto conveniente enettere codice troppo lungo con un sacco di 
istruzioni \JE/\JNE.

\lstinputlisting[label=switch_lot_c,style=customc]{patterns/08_switch/2_lot/lot.c}

\subsubsection{x86}

\myparagraph{\NonOptimizing MSVC}

Con MSVC 2010 otteniamo:

\lstinputlisting[caption=MSVC 2010,style=customasmx86]{patterns/08_switch/2_lot/lot_msvc_EN.asm}

\myindex{jumptable}

Vediamo una serie di chiamate a \printf con vari argomenti. Hanno tutte non solo indirizzi nella memoria del processo, ma anche etichette
simboliche assegnate dal compilatore. Queste label sono anche menzionate nalle tabella interna \TT{\$LN11@f}.

All'inizio della funzione, se $a$ è maggiore di 4, il controllo del flusso è passato alla label 
\TT{\$LN1@f}, dove viene chiamata \printf con argomento \TT{'something unknown'}.

Se invece il valore di $a$ è minore o uguale a 4, viene moltiplicato per 4 e sommato all'indirizzo della tabella \TT{\$LN11@f}. 
In questo modo vengono costruiti gli indirizzi della tabell, facendo puntare esattamente all'elemento giusto per ogni caso,

Poniamo ad esempio che $a$ sia uguale a 2. $2*4 = 8$ (tutti gli elementi della tabella sono indirizzi in un processo a 32-bit, perciò tutti gli elementi sono larghi 4 byte).
L'indirizzo della tabella \TT{\$LN11@f} + 8 corrisponde all'elemento della tabella in cui è memorizzata la label \TT{\$LN4@f}.
L'istruzione \JMP recupera quindi l'indirizzo di \TT{\$LN4@f} dalla tabella e salta.

Questa tabella è talvolta detta \emph{jumptable} o \emph{branch table}\footnote{L'intero metodo una volta era noto come 
\emph{computed GOTO} nelle prime versioni di Fortran:
\href{http://go.yurichev.com/17122}{wikipedia}.
Non è molto rilevante oggigiorno, ma che termine!}.

Successivamente la corrispondente \printf viene chiamata con argomento \TT{'two'}.\\
Letteralmente, l'istruzione \TT{jmp DWORD PTR \$LN11@f[ecx*4]} corrisponde a 
\emph{salta alla DWORD che è memorizzata all'indirizzo} \TT{\$LN11@f + ecx * 4}.

\TT{npad} (\myref{sec:npad}) è una macro del linguaggio assembly che allinea la prossima label in modo tale che sia memorizzata 
ad un indirizzo allineato a 4 byte (or a 16 byte).

Ciò è molto utile in termini di performance poiché così il processore è in grado di recuperare valori a 32-bit dalla memoria attraverso il memory bus, 
la cache, etc., in maniera più efficiente.y if it is aligned.

\input{patterns/08_switch/2_lot/olly_IT}

\myparagraph{\NonOptimizing GCC}
\label{switch_lot_GCC}

Vediamo il codice generato da GCC 4.4.1:

\lstinputlisting[caption=GCC 4.4.1,style=customasmx86]{patterns/08_switch/2_lot/lot_gcc.asm}

\myindex{x86!\Registers!JMP}

E' pressoché identico, con una leggera variazione: l'argomento \TT{arg\_0} è moltiplicato per 4
effettuando uno shift a sinistra di 2 bit (quasi identico alla moltiplicazione per 4)~(\myref{SHR}).
Successivamente l'indirizzo della label è preso dall'array \TT{off\_804855C}, memorizzato in 
\EAX, e infine \TT{JMP EAX} effettua il salto.


\subsubsection{ARM: \OptimizingKeilVI (\ARMMode)}
\label{sec:SwitchARMLot}

\lstinputlisting[caption=\OptimizingKeilVI (\ARMMode),style=customasmARM]{patterns/08_switch/2_lot/lot_ARM_ARM_O3.asm}

Il codice fa uso della modalità ARM in cui tutte le istruzioni hanno dimensione fissa di 4 byte.
Ricordiamoci che il massimo valore previsto per $a$ è 4 e ogni valore maggiore causerà la stampa della stringa 
\emph{<<something unknown\textbackslash{}n>>}.

\myindex{ARM!\Instructions!CMP}
\myindex{ARM!\Instructions!ADDCC}
La prima istruzione \TT{CMP R0, \#5} confronta il valore di $a$ con 5.

\footnote{ADD---addition}
La successiva \TT{ADDCC PC, PC, R0,LSL\#2} viene eseguita solo se $R0 < 5$ (\emph{CC=Carry clear / Less than}). 
Di conseguenza, se \TT{ADDCC} non viene innescata (è il caso $R0 \geq 5$), si verificherà un jump a \emph{default\_case}.

Se invece $R0 < 5$ e \TT{ADDCC} viene innescata, succede quanto segue:

Il valore in \Reg{0} viene moltiplicato per 4.
Infatti \TT{LSL\#2} nel suffisso dell'istruzione sta per \q{shift left by 2 bits} (shift a sinistra di 2 bit).
Come vedremo più avanti ~(\myref{division_by_shifting}) nella sezione \q{\ShiftsSectionName}, uno shift a sinistra
di 2 bit equivale a moltiplicare per 4.

In seguito viene aggiunto $R0*4$ all'attuale valore in \ac{PC}, saltando quindi ad una delle istruzioni \TT{B} (\emph{Branch}) poste sotto.

Al momento dell'esecuzione di \TT{ADDCC}, il valore in \ac{PC} si trova 8 byte più avanti (\TT{0x180})
rispetto all'indirizzo a cui si trova l'istruzione \TT{ADDCC} (\TT{0x178}), 
o, in altre parole, 2 istruzioni più avanti.

\myindex{ARM!Pipeline}

La pipeline nei processori ARM funziona così: nel momento in cui \TT{ADDCC} viene eseguita,
il processore sta iniziando a processare l'istruzione successiva, e questo è il motivo per cui \ac{PC} punta a quella.
Bisogna ricordarsi di ciò e tenerne conto.

Se $a=0$, viene aggiunta al valore in \ac{PC},
e l'attuale valore del \ac{PC} sarà scritto in \ac{PC} (che è 8 byte più avanti)
e si verificherà un salto alla label \emph{loc\_180},
che si trova 8 byte più avanti rispetto al punto in cui si trova l'istruzione \TT{ADDCC}.

Se $a=1$, allora $PC+8+a*4 = PC+8+1*4 = PC+12 = 0x184$ sarà scritto in \ac{PC}, ovvero l'indirizzo della label \emph{loc\_184}.

Ogni volta che si aggiunge 1 ad $a$, il risultante \ac{PC} è incrementato di 4.

4 è la lunghezza delle istruzioni in modalità ARM, comprese le istruzioni \TT{B} di cui ne abbiamo 5.

Ognuna di queste istruzioni \TT{B} passa il controllo più avanti, a quello che era previsto nello \emph{switch()}.
Li' avviene il caricamento del puntatore alla stringa corrispondente al caso, etc.

\subsubsection{ARM: \OptimizingKeilVI (\ThumbMode)}

\lstinputlisting[caption=\OptimizingKeilVI (\ThumbMode),style=customasmARM]{patterns/08_switch/2_lot/lot_ARM_thumb_O3.asm}

\myindex{ARM!\ThumbMode}
\myindex{ARM!\ThumbTwoMode}

Non possiamo essere certi che tutte le istruzioni in modalità Thumb e Thumb-2 siano della stessa lunghezza. Si può dire
che in queste modalità le istruzioni hanno lunghezza variabile, proprio come in x86.

\myindex{jumptable}

E' stata aggiunta una speciale tabella che contiene informazioni su quanti casi sono previsti (escluso il default-case),
ed un offset per ciascuno di essi, con una label a cui deve essere passato il controllo per il caso corrispondente.


\myindex{ARM!Mode switching}
\myindex{ARM!\Instructions!BX}

E' anche presente una funzione speciale per gestire la tabella e passare il controllo, \\
chiamata \emph{\_\_ARM\_common\_switch8\_thumb}. 
Inizia con l'istruzione \TT{BX PC}, la cui funzione è quella di mettere il processore in ARM-mode.
Subito dopo c'è la funzione per il processamento della tabella.

E' troppo avanzata per essere analizzata adesso, e per il momento la saltiamo.
% TODO explain it...

\myindex{ARM!\Registers!Link Register}

E' interessante notare ch ela funzione usa il registro \ac{LR} come puntatore alla tabella.

Infatti, dopo la chiamata a questa funzione, \ac{LR} contiene l'indirizzo subito dopo l'istruzione\\
\TT{BL \_\_ARM\_common\_switch8\_thumb}, dove inizia appunto la tabella.

Vale anche la pena notare che il codice è generato come una funzione separata, così che possa essere riutilizzata, e il compilatore
debba generare lo stesso codice per ogni istruzione switch().

\IDA ha correttamente capito che si tratta di una funzione di servizio e di una tabella, ed ha aggiunto i commenti alle label come\\
\TT{jumptable 000000FA case 0}.


\subsubsection{MIPS}

\lstinputlisting[caption=\Optimizing GCC 4.4.5 (IDA),style=customasmMIPS]{patterns/08_switch/2_lot/MIPS_O3_IDA_EN.lst}

\myindex{MIPS!\Instructions!SLTIU}

La nuova istruzione che incontriamo è \INS{SLTIU} (\q{Set on Less Than Immediate Unsigned}).
\myindex{MIPS!\Instructions!SLTU}

E' uguale a \INS{SLTU} (\q{Set on Less Than Unsigned}), e la \q{I} sta per \q{immediate}, 
e prevede cioè che un valore sia specificato nell'istruzione stessa.

\myindex{MIPS!\Instructions!BNEZ}
\INS{BNEZ} is \q{Branch if Not Equal to Zero}.

Il codice è molto simile a quello di altre \ac{ISA}.
\myindex{MIPS!\Instructions!SLL}
\INS{SLL} (\q{Shift Word Left Logical}) moltiplica per 4.

MIPS è una CPU a 32-bit, e tutti gli indirizzi contenuti nella \emph{jumptable} sono quindi a 32-bit.



\subsubsection{\Conclusion{}}

Stuttura approssimativa di \emph{switch()}:

% TODO: ARM, MIPS skeleton
\lstinputlisting[caption=x86,style=customasmx86]{patterns/08_switch/2_lot/skel1_IT.lst}

Il salto agli indirizzi nella tabella di jump può anche essere implementato usando questa istruzione: \\
\TT{JMP jump\_table[REG*4]}.
oppure \TT{JMP jump\_table[REG*8]} in x64.

Una \emph{jumptable} è semplicemente un array di puntatori, come quello descritto più avanti: \myref{array_of_pointers_to_strings}.
}
\JA{\subsection{A lot of cases}

\TT{switch()}ステートメントに大量のケースが含まれている場合、コンパイラが多くの \JE/\JNE 命令で大きすぎるコードを
出力することはあまり便利ではありません。

\lstinputlisting[label=switch_lot_c,style=customc]{patterns/08_switch/2_lot/lot.c}

\subsubsection{x86}

\myparagraph{\NonOptimizing MSVC}

We get (MSVC 2010):

\lstinputlisting[caption=MSVC 2010,style=customasmx86]{patterns/08_switch/2_lot/lot_msvc_JA.asm}

\myindex{jumptable}

ここでは、さまざまな引数を持つ \printf 呼び出しのセットを見ていきます。
すべては、プロセスのメモリだけでなく、コンパイラによって割り当てられた内部シンボリックラベルも持っています。
これらのラベルはすべて \TT{\$LN11@f} 内部テーブルにも記載されています。

関数の開始時に、 $a$ が4より大きい場合、制御フローはラベル \TT{\$LN1@f}に渡されます。
引数 \TT{'something unknown'}をとって \printf が呼び出されます。

しかし、 $a$ の値が4以下の場合は、4を乗算して \TT{\$LN11@f}テーブルアドレスで加算します。 
これはテーブル内のアドレスがどのように構築され、必要な要素を正確に指し示すものです。 
たとえば、 $a$ が2に等しいとしましょう。$2*4 = 8$(すべてのテーブル要素は
32ビットプロセスのアドレスなので、すべての要素が4バイト幅です)。
\TT{\$LN11@f}テーブルのアドレス+ 8は\TT{\$LN4@f}ラベルが格納されているテーブル要素です。
\JMP はテーブルから\TT{\$LN4@f}アドレスを取り出し、それにジャンプします。

このテーブルはしばしば\emph{jumptable} または \emph{branch table}\footnote{The whole method was once called 
\emph{computed GOTO} in early versions of Fortran:
\href{http://go.yurichev.com/17122}{wikipedia}.
Not quite relevant these days, but what a term!}と呼ばれます。

それから、対応する \printf は引数 \TT{'two'}で呼び出されます。
実際、\TT{jmp DWORD PTR \$LN11@f[ecx*4]}命令は\emph{jump to the DWORD that is stored at address} \TT{\$LN11@f + ecx * 4}

\TT{npad}(\myref{sec:npad})は、4バイト(または16バイト)の境界に整列したアドレスに格納されるように次のラベルを整列するアセンブリ言語マクロです。
これは、メモリバス、キャッシュメモリなどを介してメモリから32ビット値をフェッチすることができるため、
プロセッサが整列している場合にはより効果的な方法でプロセッサに非常に適しています。

\input{patterns/08_switch/2_lot/olly_JA}

\myparagraph{\NonOptimizing GCC}
\label{switch_lot_GCC}

GCC 4.4.1が生成するものを見てみましょう:

\lstinputlisting[caption=GCC 4.4.1,style=customasmx86]{patterns/08_switch/2_lot/lot_gcc.asm}

\myindex{x86!\Registers!JMP}

引数\TT{arg\_0}は2ビット左にシフトすることで4倍されます
(これは4倍の乗算とほぼ同じです)。~(\myref{SHR})
\TT{off\_804855C}配列からラベルのアドレスを取り出し、 \EAX に格納してから、\TT{JMP EAX}が実際のジャンプを行います。


\subsubsection{ARM: \OptimizingKeilVI (\ARMMode)}
\label{sec:SwitchARMLot}

\lstinputlisting[caption=\OptimizingKeilVI (\ARMMode),style=customasmARM]{patterns/08_switch/2_lot/lot_ARM_ARM_O3.asm}

このコードでは、すべての命令の固定サイズが4バイトのARMモード機能を使用しています。

$a$ の最大値は4で、それ以上の値を指定すると、\emph{<<something unknown\textbackslash{}n>>}文字列が
出力されることに注意しましょう。

\myindex{ARM!\Instructions!CMP}
\myindex{ARM!\Instructions!ADDCC}
最初の\TT{CMP R0, \#5}命令は、 $a$ の入力値を5と比較します。

\footnote{ADD---addition}
次の\TT{ADDCC PC, PC,R0, LSL \#2}命令は、$R0 < 5$(\emph{CC=Carry clear / Less than})の場合にのみ実行されます。
したがって、\TT{ADDCC}がトリガしない場合($R0 \geq 5$の場合)、\emph{default\_case}ラベルにジャンプします。

しかし$R0 < 5$と\TT{ADDCC}がトリガされた場合、次のことが起こります:

\Reg{0}の値には4が掛けられます。
実際、命令のサフィックスの\TT{LSL \#2}は\q{2ビット左シフト}の略です。
しかし、セクション\q{\ShiftsSectionName}の~(\myref{division_by_shifting})で後で見るように、2ビット左シフトは4を乗算するのと同じです。

次に、$R0*4$を\ac{PC}の現在の値に追加し、下にある\TT{B}(\emph{Branch})命令の1つにジャンプします。

\TT{ADDCC}命令の実行時に、\ac{PC}の値は\TT{ADDCC}命令が置かれているアドレス(\TT{0x178})よりも8バイト先(\TT{0x180})であり、
言い換えれば2命令先にあります。

\myindex{ARM!Pipeline}

これはARMプロセッサのパイプラインがどのように動作するかを示しています。
\TT{ADDCC}が実行されると、現時点でプロセッサは次の命令の後に命令を処理し始めているので、
\ac{PC}がそこを指しているのはそのためです。 
これは覚えておく必要があります。

$a=0$ の場合、\ac{PC}の値に加算され、\ac{PC}の実際の値は\ac{PC}(8バイト先)に書き込まれ、
\emph{loc\_180}というラベルへのジャンプが起こります。これは、\TT{ADDCC}命令の先の8バイト先です。

$a=1$ の場合、\ac{PC}には $PC+8+a*4 = PC+8+1*4 = PC+12 = 0x184$ が書き込まれます。
\emph{loc\_184}というラベルが付いたアドレスです。

1を $a$ に加えるごとに、結果の\ac{PC}は4ずつ増加します。

4はARMモードの命令長であり、各\TT{B}命令の長さ4でそれらは5つあります。

これらの5つの\TT{B}命令のそれぞれは、制御を\emph{switch()}にプログラムされたものにさらに渡します。

対応する文字列のポインタローディングが発生します。

\subsubsection{ARM: \OptimizingKeilVI (\ThumbMode)}

\lstinputlisting[caption=\OptimizingKeilVI (\ThumbMode),style=customasmARM]{patterns/08_switch/2_lot/lot_ARM_thumb_O3.asm}

\myindex{ARM!\ThumbMode}
\myindex{ARM!\ThumbTwoMode}

ThumbモードとThumb-2モードのすべての命令が同じサイズであることを確認することはできません。
これらのモードでは、x86の場合と同様に、命令の長さが可変であるといえます。

\myindex{jumptable}

したがって、そこにあるケースの数(デフォルトケースを含まない)に関する情報と、
対応するケースでコントロールを渡す必要があるラベルを持つそれぞれのオフセットが含まれている特別なテーブルが追加されています。

\myindex{ARM!Mode switching}
\myindex{ARM!\Instructions!BX}

\emph{\_\_ARM\_common\_switch8\_thumb}という名前のテーブルと
パスコントロールを扱うために特別な関数がここにあります。
\TT{BX PC}で始まり、その機能はプロセッサをARMモードに切り替えることです。
次に、テーブル処理の機能が表示されます。

今ここで説明するにはあまりにも進んでいるので、省略しましょう。
% TODO explain it...

\myindex{ARM!\Registers!Link Register}

関数が\ac{LR}レジスタをテーブルへのポインタとして使用することは興味深いことです。

実際、この関数を呼び出した後、\ac{LR}にはテーブルが始まる\TT{BL \_\_ARM\_common\_switch8\_thumb}命令の後のアドレスが入ります。

また、コードを再利用するために別の関数として生成されるので、
コンパイラはすべてのswitch()文に対して同じコードを生成しないことにも注意してください。

\IDA はそれをサービス関数とテーブルとして認識し、\TT{jumptable 000000FA case 0}のような
ラベルのコメントを追加します。

\subsubsection{MIPS}

\lstinputlisting[caption=\Optimizing GCC 4.4.5 (IDA),style=customasmMIPS]{patterns/08_switch/2_lot/MIPS_O3_IDA_JA.lst}

\myindex{MIPS!\Instructions!SLTIU}

私たちの新しい命令は \INS{SLTIU} です(\q{Set on Less Than Immediate Unsigned})。
\myindex{MIPS!\Instructions!SLTU}

\INS{SLTU}と同じですが、\q{I}は\q{immediate}を表します。
つまり、命令自体に数値を指定する必要があります。

\myindex{MIPS!\Instructions!BNEZ}
\INS{BNEZ} は \q{Branch if Not Equal to Zero}です。

コードは他の\ac{ISA}に似ています。
\myindex{MIPS!\Instructions!SLL}
\INS{SLL} (\q{Shift Word Left Logical})は4を掛けます。

結局のところ、MIPSは32ビットCPUなので、\emph{jumptable}のすべてのアドレスは32ビットのものです。


\subsubsection{\Conclusion{}}

\emph{switch()} の大まかなスケルトン:

% TODO: ARM, MIPS skeleton
\lstinputlisting[caption=x86,style=customasmx86]{patterns/08_switch/2_lot/skel1_JA.lst}

ジャンプテーブルのアドレスへのジャンプはこの命令で用いて実装されるでしょう:\TT{JMP jump\_table[REG*4]}
もしくはx64では\TT{JMP jump\_table[REG*8]} 。

\emph{jumptable}は単にポインタの配列で、後で説明します:\myref{array_of_pointers_to_strings}
}

% TODO What's the difference between 3 and 4? Seems to be the same...
% it is fallthrough from 3 to 4 :) --DY
\EN{\subsection{When there are several \emph{case} statements in one block}

Here is a very widespread construction: several \emph{case} statements for a single block:

\lstinputlisting[style=customc]{patterns/08_switch/3_several_cases/several_cases.c}

It's too wasteful to generate a block for each possible case,
so what is usually done is to generate each block plus some kind of dispatcher.

\subsubsection{MSVC}

\lstinputlisting[caption=\Optimizing MSVC 2010,numbers=left,style=customasmx86]{patterns/08_switch/3_several_cases/several_cases_MSVC_2010_Ox_EN.asm}

We see two tables here: the first table (\TT{\$LN10@f}) is an index table, and the second one (\TT{\$LN11@f}) is an array of pointers to blocks.

First, the input value is used as an index in the index table (line 13). 

Here is a short legend for the values in the table: 
0 is the first \emph{case} block (for values 1, 2, 7, 10),
1 is the second one (for values 3, 4, 5),
2 is the third one (for values 8, 9, 21),
3 is the fourth one (for value 22),
4 is for the default block.

There we get an index for the second table of code pointers and we jump to it (line 14).

What is also worth noting is that there is no case for input value 0.

That's why we see the \DEC instruction at line 10, and the table starts at $a=1$, 
because there is no need to allocate a table element for $a=0$.

This is a very widespread pattern.

So why is this economical?
Why isn't it possible to make it as before
(\myref{switch_lot_GCC}), just with one table consisting of block pointers?
The reason is that the elements in index table are 8-bit, hence it's all more compact.

\subsubsection{GCC}

GCC does the job in the way we already discussed (\myref{switch_lot_GCC}), using just one table of pointers.

\subsubsection{ARM64: \Optimizing GCC 4.9.1}

There is no code to be triggered if the input value is 0, so GCC tries to make the jump table more compact
and so it starts at 1 as an input value.

GCC 4.9.1 for ARM64 uses an even cleverer trick.
It's able to encode all offsets as 8-bit bytes.

Let's recall that all ARM64 instructions have a size of 4 bytes.

GCC is uses the fact that all offsets in my tiny example are in close proximity to each other.
So the jump table consisting of single bytes.

\lstinputlisting[caption=\Optimizing GCC 4.9.1 ARM64,style=customasmARM]{patterns/08_switch/3_several_cases/ARM64_GCC491_O3_EN.s}

Let's compile this example to object file and open it in \IDA. Here is the jump table:

\lstinputlisting[caption=jumptable in IDA,style=customasmARM]{patterns/08_switch/3_several_cases/ARM64_GCC491_O3_IDA.lst}

So in case of 1, 9 is to be multiplied by 4 and added to the address of \TT{Lrtx4} label.

In case of 22, 0 is to be multiplied by 4, resulting in 0. 

Right after the \TT{Lrtx4} label is the \TT{L7} label, where you can find the code that prints \q{22}.

There is no jump table in the code segment, it's allocated in a separate .rodata section 
(there is no special necessity to place it in the code section).

There are also negative bytes (0xF7), they are used for jumping back to the code that prints the \q{default} string (at \TT{.L2}).

}
\RU{\subsection{Когда много \emph{case} в одном блоке}

Вот очень часто используемая конструкция: несколько \emph{case} может быть использовано в одном блоке:

\lstinputlisting[style=customc]{patterns/08_switch/3_several_cases/several_cases.c}

Слишком расточительно генерировать каждый блок для каждого случая, поэтому обычно
генерируется каждый блок плюс некий диспетчер.

\subsubsection{MSVC}

\lstinputlisting[caption=\Optimizing MSVC 2010,numbers=left,style=customasmx86]{patterns/08_switch/3_several_cases/several_cases_MSVC_2010_Ox_RU.asm}

Здесь видим две таблицы: первая таблица (\TT{\$LN10@f}) это таблица индексов,
а вторая таблица (\TT{\$LN11@f}) это массив указателей на блоки.

В начале, входное значение используется как индекс в таблице индексов (строка 13). 

Вот краткое описание значений в таблице: 
0 это первый блок \emph{case} (для значений 1, 2, 7, 10),
1 это второй (для значений 3, 4, 5),
2 это третий (для значений 8, 9, 21),
3 это четвертый (для значений 22),
4 это для блока по умолчанию.

Мы получаем индекс для второй таблицы указателей на блоки и переходим туда (строка 14).

Ещё нужно отметить то, что здесь нет случая для нулевого входного значения.

Поэтому мы видим инструкцию \DEC на строке 10 и таблица начинается с $a=1$.
Потому что незачем выделять в таблице элемент для $a=0$.

Это очень часто используемый шаблон.

В чем же экономия?
Почему нельзя сделать так, как уже обсуждалось (\myref{switch_lot_GCC}), используя только одну таблицу, содержащую указатели на блоки?
Причина в том, что элементы в таблице индексов занимают только по 8-битному байту, поэтому всё это более компактно.

\subsubsection{GCC}

GCC делает так, как уже обсуждалось (\myref{switch_lot_GCC}), используя просто таблицу указателей.

\subsubsection{ARM64: \Optimizing GCC 4.9.1}

Во-первых, здесь нет кода, срабатывающего в случае если входное значение~--- 0, так что GCC пытается
сделать таблицу переходов более компактной и начинает со случая, когда входное значение~--- 1.

GCC 4.9.1 для ARM64 использует даже более интересный трюк.
Он может закодировать все смещения как 8-битные байты.
Вспомним, что все инструкции в ARM64 имеют размер в 4 байта.

GCC также использует тот факт, что все смещения в моем крохотном примере находятся достаточно близко друг от друга.

Так что таблица переходов состоит из байт.

\lstinputlisting[caption=\Optimizing GCC 4.9.1 ARM64,style=customasmARM]{patterns/08_switch/3_several_cases/ARM64_GCC491_O3_RU.s}

Скомпилируем этот пример как объектный файл и откроем его в \IDA. Вот таблица переходов:

\lstinputlisting[caption=jumptable in IDA,style=customasmARM]{patterns/08_switch/3_several_cases/ARM64_GCC491_O3_IDA.lst}

В случае 1, 9 будет умножено на 9 и прибавлено к адресу метки \TT{Lrtx4}.

В случае 22, 0 будет умножено на 4, в результате это 0.

Место сразу за меткой \TT{Lrtx4} это метка L7, где находится код, выводящий \q{22}.

В сегменте кода нет таблицы переходов, место для нее выделено в отдельной секции .rodata
(нет особой нужды располагать её в сегменте кода).

Там есть также отрицательные байты (0xF7). Они используются для перехода назад, на код, выводящий
строку \q{default} (на \TT{.L2}).

}
\DE{\subsection{Wenn es mehrere \emph{case} Ausdrücke in einem Block gibt}
Hier ist eine weit verbreitete Konstruktion: mehrere \emph{case} Ausdrücke für einen einzigen Block:

\lstinputlisting[style=customc]{patterns/08_switch/3_several_cases/several_cases.c}
Es ist zu verschwenderisch einen Block für jeden möglichen Fall zu erzeugen, sodass normalerweise ein Block und eine Art
Dispatcher erzeugt werden.

\subsubsection{MSVC}

\lstinputlisting[caption=\Optimizing MSVC
2010,numbers=left,style=customasmx86]{patterns/08_switch/3_several_cases/several_cases_MSVC_2010_Ox_DE.asm}
Wir sehen hier zwei Tabellen: die erste Tabelle (\TT{\$LN10@f}) ist eine Indextabelle und die zweite (\TT{\$LN11@f}) ist
ein Array von Pointern auf Blöcke.

Zuerst wird der Eingabewert als Index in der Indextabelle verwendet (Zeile 13).

Hier ist eine kurze Legende für die Werte in der Tabelle:
0 ist der erste \emph{case} Block (für die Werte 1, 2, 7, 10),
1 ist der zweite (für die Werte 3, 4, 5),
2 ist der dritte (für die Werte 8, 9, 21),
3 ist der vierte (für die Werte 22),
4 ist der Defaultblock.
Hier erhalten wir einen Index für die zweite Tabelle aus Pointern und springen zu einem solchen (Zeile 14).

Bemerkenswert ist auch, dass es keinen Case für den Eingabewert 0 gibt. 

Aus diesem Grund haben wir den \DEC Befehl in Zeile 10 und die Tabelle beginnt bei $a=1$, da kein Tabellenelement für
$a=0$ angelegt werden muss.

Dies ist ein weitverbreitetes Muster.

Warum ist dieses Vorgehen so ökonomisch?
Warum ist es nicht möglich wie vorher in (\myref{switch_lot_GCC}) vorzugehen mit nur einer Tabelle aus Blockpointern?
Der Grund hierfür ist, dass die Elemente in der Indextabelle nur 8 Bit groß sind und alles deshalb deutlich kompakter
ist.

\subsubsection{GCC}

GCC erledigt seinen Job unter Verwendung von nur einer Pointertabelle wie bereits hier besprochen
(\myref{switch_lot_GCC}).

\subsubsection{ARM64: \Optimizing GCC 4.9.1}
Es wird kein Code ausgeführt, wenn der Eingabewert 0 ist, weshalb GCC versucht, die Jumptable kleiner zu machen und erst
beim Eingabewert 1 zu beginnen.

GCC 4.9.1 für ARM64 verwendeten einen noch ausgefeilteren Trick.
Es ist möglich alle Offsets als 8 Bit Werte zu kodieren.

Erinnern wir uns, dass alle ARM64 Befehle eine Größe von 4 Bytes haben.

GCC verwendet den Umstand, dass alle Offsets in unserem Minimalbeispiel in der Nähe voneinander liegen. Daher kann die
Jumptable aus einzelnen Bytes bestehen.

\lstinputlisting[caption=\Optimizing GCC 4.9.1
ARM64,style=customasmARM]{patterns/08_switch/3_several_cases/ARM64_GCC491_O3_DE.s}
Kompilieren wir dieses Beispiel in eine Object-Datei und öffnen es ist \IDA. Hier ist die Jumptable:

\lstinputlisting[caption=jumptable in IDA,style=customasmARM]{patterns/08_switch/3_several_cases/ARM64_GCC491_O3_IDA.lst}

Im Fall von 1, 9 wird also mit 4 multipliziert und zur Adresse des Labels \TT{Lrtx4} addiert.

Im Fall von 22, wird 0 mit 4 multipliziert; mit dem Ergebnis 0. 

Direkt hinter dem \TT{Lrtx4} Label befindet sich das \TT{L7} Label, an dem sich der Code befindet, der \q{22} ausgibt.

Es gibt keine Jumptable im Codesegment; sie wird in einem getrennten .rodata Segment angelegt (es besteht keine
Notwendigkeit, die Tabelle im Codesegment anzulegen).

Hier befinden sich auch negative Bytes (0xF7), die für das Zurückspringen im Code verwendet werden, um den \q{Default}
String am Label \TT{.L2} auszugeben.

}
\FR{\subsection{Lorsqu'il y a quelques déclarations \emph{case} dans un bloc}

Voici une construction très répandue: quelques déclarations \emph{case} pour un seul bloc:

\lstinputlisting[style=customc]{patterns/08_switch/3_several_cases/several_cases.c}

C'est souvent du gaspillage de générer un bloc pour chaque cas possible, c'est
pourquoi ce qui se fait d'habitude, c'est de générer un bloc et une sorte de répartiteur.

\subsubsection{MSVC}

\lstinputlisting[caption=MSVC 2010 \Optimizing,numbers=left,style=customasmx86]{patterns/08_switch/3_several_cases/several_cases_MSVC_2010_Ox_FR.asm}

Nous voyons deux tables ici: la première (\TT{\$LN10@f}) est une table d'index,
et la seconde (\TT{\$LN11@f}) est un tableau de pointeurs sur les blocs.

Tout d'abord, la valeur entrée est utilisée comme un index dans la table d'index
(ligne 13).

Voici un petit récapitulatif pour les valeurs dans la table:
0 est le premier bloc \emph{case} (pour les valeurs 1, 2, 7, 10),
1 est le second (pour les valeurs 3, 4, 5),
2 est le troisième (pour les valeurs 8, 9, 21),
3 est le quatrième (pour la valeur 22),
4 est pour le bloc par défaut.

Ici, nous obtenons un index pour la seconde table de pointeurs sur du code et nous
y sautons (ligne 14).

Il est intéressant de remarquer qu'il n'y a pas de cas pour une valeur d'entrée
de 0.

C'est pourquoi nous voyons l'instruction \DEC à la ligne 10, et la table commence
à $a=1$, car il n'y a pas besoin d'allouer un élément dans la table pour $a=0$.

C'est un pattern très répandu.

Donc, pourquoi est-ce que c'est économique ?
Pourquoi est-ce qu'il n'est pas possible de faire comme avant (\myref{switch_lot_GCC}),
avec une seule table consistant en des pointeurs vers les blocs?
La raison est que les index des éléments de la table sont 8-bit, donc c'est plus
compact.

\subsubsection{GCC}

GCC génère du code de la façon dont nous avons déjà discuté (\myref{switch_lot_GCC}),
en utilisant juste une table de pointeurs.

\subsubsection{ARM64: GCC 4.9.1 \Optimizing}

Il n'y a pas de code à exécuter si la valeur entrée est 0, c'est pourquoi GCC
essaye de rendre la table des sauts plus compacte et donc il commence avec la
valeur d'entrée 1.

GCC 4.9.1 pour ARM64 utilise un truc encore plus astucieux.
Il est capable d'encoder tous les offsets en octets 8-bit.

Rappelons-nous que toutes les instructions ARM64 ont une taille de 4 octets.

GCC utilise le fait que tous les offsets de mon petit exemple sont tous proche l'un
de l'autre. Donc la table des sauts consiste en de simple octets.

\lstinputlisting[caption=\Optimizing GCC 4.9.1 ARM64,style=customasmARM]{patterns/08_switch/3_several_cases/ARM64_GCC491_O3_FR.s}

Compilons cet exemple en un fichier objet et ouvrons-le dans \IDA. Voici la table
des sauts:

\lstinputlisting[caption=jumptable in IDA,style=customasmARM]{patterns/08_switch/3_several_cases/ARM64_GCC491_O3_IDA.lst}

Donc dans le cas de 1, 9 est multiplié par 4 et ajouté à l'adresse
du label \TT{Lrtx4}.

Dans le cas de 22, 0 est multiplié par 4, ce qui donne 0.

Juste après le label \TT{Lrtx4} se trouve le label \TT{L7}, où se trouve le code
qui affiche \q{22}.

Il n'y a pas de table des sauts dans le segment de code, elle est allouée dans
la section .rodata (il n'y a pas de raison de l'allouer dans le segment de code).

Il y a aussi des octets négatifs (0xF7), ils sont utilisés pour sauter en arrière
dans le code qui affiche la chaîne \q{default} (en \TT{.L2}).

}
\IT{\subsection{Ancora più istruzioni \emph{case} in un unico blocco}

Ecco un costrutto molto diffuso: molti casi in un singolo blocco:
\lstinputlisting[style=customc]{patterns/08_switch/3_several_cases/several_cases.c}

Generare un blocco per ciascun caso possibile risulta poco efficiente, perciò solitamente quello che viene fatto è generare un
ogni blocco più una sorta di smistatore (dispatcher).

\subsubsection{MSVC}

\lstinputlisting[caption=\Optimizing MSVC 2010,numbers=left,style=customasmx86]{patterns/08_switch/3_several_cases/several_cases_MSVC_2010_Ox_EN.asm}

Notiamo due tabelle: la prima, (\TT{\$LN10@f}), è una tabella di indici. La seconda, (\TT{\$LN11@f}), è un array di puntatori ai blocchi.

Per cominciare, il valore di input è usato come indice nella tabella di indici (riga 13).

Ecco una piccola legenda di valori in questa tabella:
0 è il blocco relativo al primo \emph{case} (per i valori 1, 2, 7, 10),
1 al secondo (per i valori 3, 4, 5),
2 al terzo (per i valori 8, 9, 21),
3 al quarto (per i valori 22),
4 è relativo al blocco default.

Da qui ottieniamo un indice per la seconda tabella di puntatori al codice, e vi saltiamo (riga 14).

Vale anche la pena notare che non c'è alcun caso per il valore 0 in input.

Per questo motivo vediamo l'istruzione \DEC a riga 10, e la tabella inizia da $a=1$, 
non c'è bisogno di allocare un elemento nella tabella per $a=0$.

Questo è un pattern molto diffuso

Perchè è economico? Perchè non è possibile avere lo stesso risultato con le tecniche viste prima 
(\myref{switch_lot_GCC}), solo con una tabella di puntatori ai blocchi?
La risposta sta nel fatto che gli elementi nella tabella di indici sono di 8-bit, dunque è tutto più compatto.

\subsubsection{GCC}

GCC 
GCC utilizza la tecnica già vista (\myref{switch_lot_GCC}), usando solo una tabella di puntatori.

\subsubsection{ARM64: \Optimizing GCC 4.9.1}

Non c'è codice da eseguire se il valore in input è 0, perciò GCC prova a rendere la jump table più compatta iniziando da 1 come valore di input.

GCC 4.9.1 per ARM64 usa un trucco ancora migliore.
Riesce a codificare tutti gli offset con byte (8-bit).

Ricordiamoci che tutte le istruzioni ARM64 sono lunghe 4 byte.

GCC sfrutta il fatto che tutti gli offset nel nostro piccolo esempio si trovano molto vicini tra di loro.
In questo modo la jump table consiste di singoli byte.

\lstinputlisting[caption=\Optimizing GCC 4.9.1 ARM64,style=customasmARM]{patterns/08_switch/3_several_cases/ARM64_GCC491_O3_EN.s}

Compiliamo questo esempio in un file oggetto e apriamolo con \IDA. Questa è la jump table:

\lstinputlisting[caption=jumptable in IDA,style=customasmARM]{patterns/08_switch/3_several_cases/ARM64_GCC491_O3_IDA.lst}

Riassumendo, nel caso 1, 9 viene moltiplicato per 4 e aggiunto all'indirizzo della label \TT{Lrtx4}.
Nel caso 22, 0 viene moltiplicato per 4, risultando 0. 

Subito dopo la label \TT{Lrtx4} si trova label \TT{L7}, dove c'è il codice che stampa \q{22}.

Non c'è alcuna jump table nel code segnment, è allocata in una sezione .rodata separata (non vi è alcun motivo particolare
per cui sarebbe necessario metterla nella sezione del codice).

Ci sono anche byte negativi (0xF7), usati per saltare indietro al codice che stampa la stringa \q{default} (a \TT{.L2}).

}
\JA{\subsection{あるブロックに複数の\emph{case}文があるとき}

よく用いられる構成があります:単一ブロックにいくつか\emph{case}ステートメントがあります:

\lstinputlisting[style=customc]{patterns/08_switch/3_several_cases/several_cases.c}

可能性のあるケースごとにブロックを生成するのは無駄です。
通常は、各ブロックに何らかのディスパッチャーを加えたものを生成します。

\subsubsection{MSVC}

\lstinputlisting[caption=\Optimizing MSVC 2010,numbers=left,style=customasmx86]{patterns/08_switch/3_several_cases/several_cases_MSVC_2010_Ox_JA.asm}

最初のテーブル(\TT{\$LN10@f})はインデックステーブルで、2番目のテーブル(\TT{\$LN11@f})はブロックへのポインタの配列です。

まず、入力値がインデックステーブルのインデックスとして使用されます(13行目)。

表の値の短い凡例は次のとおりです。
0は最初の\emph{case}ブロックです(値1,2,7,10の場合)。
1は2番目の値(値3,4,5)です。
2は3番目の値(値8,9,21)です。
3は4番目の値(値22)です。
4はデフォルトブロック用です。

コードポインタの2番目のテーブルのインデックスを取得し、それにジャンプします(14行目)。

注目すべき点は、入力値0の場合がないことです。

そのため、10行目の \DEC 命令が表示され、$a=1$にテーブル要素を割り当てる必要がないため、
$a=1$でテーブルが開始されます。

これはよく用いられるパターンです。

それでなぜこれが経済的なのでしょうか?
以前はブロックポインタで構成された1つのテーブルだけで、
それを作ることができないのはなぜですか?(\myref{switch_lot_GCC})
その理由は、インデックステーブルの要素が8ビットで、よりコンパクトなためです。

\subsubsection{GCC}

GCCはすでに述べた方法で(\myref{switch_lot_GCC})、ポインタのテーブルを1つだけ使用して仕事をしています。

\subsubsection{ARM64: \Optimizing GCC 4.9.1}

入力値が0の場合にトリガされるコードはないので、GCCはジャンプテーブルをよりコンパクトにしようとし、
入力値として1から開始します。

ARM64用のGCC 4.9.1は、より巧妙なトリックを使用します。
すべてのオフセットを8ビットのバイトとしてエンコードできます。

すべてのARM64命令のサイズが4バイトであることを思い出してみましょう。

GCCは、私の小さな例のすべてのオフセットがお互いに非常に近いという事実を利用しています。
ジャンプテーブルは1バイトで構成されています。

\lstinputlisting[caption=\Optimizing GCC 4.9.1 ARM64,style=customasmARM]{patterns/08_switch/3_several_cases/ARM64_GCC491_O3_JA.s}

この例をオブジェクトファイルにコンパイルし、 \IDA で開きましょう。 ここにジャンプテーブルがあります:

\lstinputlisting[caption=jumptable in IDA,style=customasmARM]{patterns/08_switch/3_several_cases/ARM64_GCC491_O3_IDA.lst}

したがって、1の場合、9は4で乗算され、\TT{Lrtx4}ラベルのアドレスに追加されます。

22の場合、0には4が掛けられ、結果は0になります。

\TT{Lrtx4}ラベルの直後に\TT{L7}ラベルがあります。このラベルでは、\q{22}を出力するコードを見つけることができます。

コードセグメントにはジャンプテーブルはありません。別の.rodataセクションに割り当てられています
(コードセクションに配置する特別な必要はありません)。

負のバイト(0xF7)もあり、\q{default}文字列(\TT{.L2})を出力するコードにジャンプするために使用されます。
}


\EN{\subsection{Fall-through}

Another popular usage of \TT{switch()} operator is so-called \q{fallthrough}.
Here is simple example\footnote{Copypasted from \url{https://github.com/azonalon/prgraas/blob/master/prog1lib/lecture_examples/is_whitespace.c}}:

\lstinputlisting[numbers=left,style=customc]{patterns/08_switch/4_fallthrough/fallthrough1.c}

Slightly harder, from Linux kernel\footnote{Copypasted from \url{https://github.com/torvalds/linux/blob/master/drivers/media/dvb-frontends/lgdt3306a.c}}:

\lstinputlisting[numbers=left,style=customc]{patterns/08_switch/4_fallthrough/fallthrough2.c}

\lstinputlisting[caption=Optimizing GCC 5.4.0 x86,numbers=left,style=customasmx86]{patterns/08_switch/4_fallthrough/fallthrough2.s}

We can get to \TT{.L5} label if there is number 3250 at function's input.
But we can get to this label from the other side:
we see that there are no jumps between \printf call and \TT{.L5} label.

Now we can understand why \emph{switch()} statement is sometimes a source of bugs:
one forgotten \emph{break} will transform your
\emph{switch()} statement into \emph{fallthrough} one, and several blocks will be executed instead of single one.

}
\RU{\subsection{Fall-through}

Ещё одно популярное использование оператора \TT{switch()} это т.н. \q{fallthrough} (\q{провал}).
Вот простой пример\footnote{Взято отсюда: \url{https://github.com/azonalon/prgraas/blob/master/prog1lib/lecture_examples/is_whitespace.c}}:

\lstinputlisting[numbers=left,style=customc]{patterns/08_switch/4_fallthrough/fallthrough1.c}

Немного сложнее, из ядра Linux\footnote{\url{https://github.com/torvalds/linux/blob/master/drivers/media/dvb-frontends/lgdt3306a.c}}:

\lstinputlisting[numbers=left,style=customc]{patterns/08_switch/4_fallthrough/fallthrough2.c}

\lstinputlisting[caption=Оптимизирующий GCC 5.4.0 x86,numbers=left,style=customc]{patterns/08_switch/4_fallthrough/fallthrough2.s}

На метку \TT{.L5} управление может перейти если на входе ф-ции число 3250.
Но на эту метку можно попасть и с другой стороны:
мы видим что между вызовом \printf и меткой \TT{.L5} нет никаких пероходов.

Теперь мы можем понять, почему иногда \emph{switch()} является источником ошибок: если забыть дописать \emph{break},
это прекратит выражение \emph{switch()} в \emph{fallthrough}, и вместо одного блока для каждого условия,
будет исполняться сразу несколько.

}
\DE{\subsection{Fallthrough}
Eine andere übliche Verwendung des \TT{switch()} Operators ist der sogenannte \q{Fallthrough}.
Hier ist ein einfaches Beispiel\footnote{Kopiert von
\url{https://github.com/azonalon/prgraas/blob/master/prog1lib/lecture_examples/is_whitespace.c}}:

\lstinputlisting[numbers=left,style=customc]{patterns/08_switch/4_fallthrough/fallthrough1.c}

Ein etwas komplizierteres Beispiel aus dem Linux Kernel\footnote{Kopiert von
\url{https://github.com/torvalds/linux/blob/master/drivers/media/dvb-frontends/lgdt3306a.c}}:

\lstinputlisting[numbers=left,style=customc]{patterns/08_switch/4_fallthrough/fallthrough2.c}

\lstinputlisting[caption=Optimizing GCC 5.4.0 x86,numbers=left,style=customasmx86]{patterns/08_switch/4_fallthrough/fallthrough2.s}
Wir gelangen zum \TT{.L5} Label, wenn die Eingabe der Funktion die Zahl 3250 ist.
Aber wir können dieses Label von der anderen Seite erreichen: wir sehen, dass es keine Sprünge zwischen dem Aufruf von
\printf und dem \TT{.L5} Label gibt.

Jetzt verstehen wir auch, warum der \emph{switch()} Ausdruck manchmal eine Quelle von Bugs ist:
ein einziges vergessenes \emph{break} verändert einen \emph{switch()} Ausdruck in einen \emph{Fallthrough} und mehrere Blöcke
anstelle eine einzigen werden ausgeführt.
}
\FR{\subsection{Fall-through}

Un autre usage très répandu de l'opérateur \TT{switch()} est ce qu'on appelle
un \q{fallthrough} (passer à travers).
Voici un exemple simple\footnote{Copié/collé depuis \url{https://github.com/azonalon/prgraas/blob/master/prog1lib/lecture_examples/is_whitespace.c}}:

\lstinputlisting[numbers=left,style=customc]{patterns/08_switch/4_fallthrough/fallthrough1.c}

Légèrement plus difficile, tiré du noyau Linux\footnote{Copié/collé depuis \url{https://github.com/torvalds/linux/blob/master/drivers/media/dvb-frontends/lgdt3306a.c}}:

\lstinputlisting[numbers=left,style=customc]{patterns/08_switch/4_fallthrough/fallthrough2.c}

\lstinputlisting[caption=GCC 5.4.0 x86 \Optimizing,numbers=left,style=customasmx86]{patterns/08_switch/4_fallthrough/fallthrough2.s}

Nous atteignons le label \TT{.L5} si la fonction a reçue le nombre 3250 en entrée.
Mais nous pouvons atteindre ce label d'une autre façon:
nous voyons qu'il n'y a pas de saut entre l'appel à \printf et le label \TT{.L5}.

Nous comprenons maintenant pourquoi la déclaration \emph{switch()} est parfois une
source de bug:
un \emph{break} oublié va transformer notre déclaration \emph{switch()} en un \emph{fallthrough},
et plusieurs blocs seront exécutés au lieu d'un seul.

}
\IT{\subsection{Fall-through}

Un altro uso diffuso dell'operatore \TT{switch()} è il cosiddetto \q{fallthrough}.
Ecco un semplice esempio \footnote{Preso da \url{https://github.com/azonalon/prgraas/blob/master/prog1lib/lecture_examples/is_whitespace.c}}:

\lstinputlisting[numbers=left,style=customc]{patterns/08_switch/4_fallthrough/fallthrough1.c}

Uno leggermente più difficile, dal kernel di Linux \footnote{Preso da \url{https://github.com/torvalds/linux/blob/master/drivers/media/dvb-frontends/lgdt3306a.c}}:

\lstinputlisting[numbers=left,style=customc]{patterns/08_switch/4_fallthrough/fallthrough2.c}

\lstinputlisting[caption=Optimizing GCC 5.4.0 x86,numbers=left,style=customasmx86]{patterns/08_switch/4_fallthrough/fallthrough2.s}

Possiamo arrivare alla label \TT{.L5} se all'input della funzione viene dato il valore 3250.
Ma si può anche giungere allo stesso punto da un altro percorso:
notiamo che non ci sono jump tra la chiamata a \printf e la label \TT{.L5}.

Questo spiega facilmente perchè i costrutti con \emph{switch()} sono spesso fonte di bug:
è sufficiente dimenticare un \emph{break} per trasformare il costrutto \emph{switch()} in un \emph{fallthrough} , in cui vengono eseguiti
più blocchi invece di uno solo.
}
\JA{\subsection{フォールスルー}

\TT{switch()}演算子の別のポピュラーな使い方は\q{フォールスルー}です。
単純なサンプルがあります。\footnote{\url{https://github.com/azonalon/prgraas/blob/master/prog1lib/lecture_examples/is_whitespace.c}からコピーペースト}:

\lstinputlisting[numbers=left,style=customc]{patterns/08_switch/4_fallthrough/fallthrough1.c}

やや難しいものをLinuxカーネル\footnote{\url{https://github.com/torvalds/linux/blob/master/drivers/media/dvb-frontends/lgdt3306a.c}からコピーペースト}:

\lstinputlisting[numbers=left,style=customc]{patterns/08_switch/4_fallthrough/fallthrough2.c}

\lstinputlisting[caption=Optimizing GCC 5.4.0 x86,numbers=left,style=customasmx86]{patterns/08_switch/4_fallthrough/fallthrough2.s}

関数の入力に3250という数字がある場合、\TT{.L5}ラベルを得ることができます。
しかし、我々は反対側からこのラベルに行くことができます:
\printf 呼び出しと\TT{.L5}ラベルの間にはジャンプがないことがわかります。

\emph{switch()}文がバグの原因となることが理解できます。
\emph{break}を1つ忘れるとはあなたの\emph{switch()}文を\emph{フォールスルー}に変換し、1つのブロックの代わりにいくつかのブロックが実行されます。
}


\subsection{\Exercises}

\subsubsection{\Exercise \#1}
\label{exercise_switch_1}

\RU{Вполне возможно переделать пример на Си в листинге \myref{switch_lot_c} так, чтобы при компиляции
получалось даже ещё меньше кода, но работать всё будет точно так же.
Попробуйте этого добиться.}
\EN{It's possible to rework the C example in \myref{switch_lot_c} in such way that the compiler
can produce even smaller code, but will work just the same.
Try to achieve it.}
\DE{Der C-Code des Beispiels in \myref{switch_lot_c} soll so neu geschrieben werden, dass der Compiler die gleiche
Funktionalität in noch kürzerem Code erreichen kann.}
\FR{Il est possible de modifier l'exemple en C de \myref{switch_lot_c} de telle sorte
que le compilateur produise un code plus concis, mais qui fonctionne toujours pareil.}
\IT{E' possibile riscrivere l'esempio C da \myref{switch_lot_c} in modo tale che il compilatore riesca a produrre codice ancora più breve e che funzioni allo stesso modo. Prova a farlo.}
\PLph{}
\JPN{コンパイラがより小さなコードを生成することができるように\myref{switch_lot_c}のCの例を
修正することは可能ですが、まったく同じように動作します。
やってみてください。}


% \RU{Подсказка}\EN{Hint}: \printf \EN{may be called only from a single place}\RU{вполне может 
% вызываться только из одного места}.
