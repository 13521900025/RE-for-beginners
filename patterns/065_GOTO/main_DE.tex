\mysection{GOTO Operator}

Der GOTO Operator wird für gewöhnlich als Anti-Pattern angesehen, vgl. dazu
[Edgar Dijkstra, \emph{Go To Statement Considered Harmful} (1968)\footnote{\url{http://yurichev.com/mirrors/Dijkstra68.pdf}}].
Nichtsdestotrotz kann es er auch sinnvoll eingesetzt werden, siehe
[Donald E. Knuth, \emph{Strukt Programming with go to Statements} (1974)\footnote{\url{http://yurichev.com/mirrors/KnuthStructuredProgrammingGoTo.pdf}}]
\footnote{\InSqBrackets{\CNotes} enthält auch einige Beispiele.}.

Hier ist ein sehr einfaches Beispiel:

\lstinputlisting[style=customc]{patterns/065_GOTO/goto.c}

Dies ehalten wir in MSVC 2012:

\lstinputlisting[caption=MSVC 2012,style=customasmx86]{patterns/065_GOTO/MSVC_goto.asm}
Der \emph{Goto} Ausdruck wurde einfach durch einen \JMP Befehl ersetzt, der den gleichen Effekt hat: einen unbedingten Sprung an eine andere Stelle.
Das zweite \printf kann nur durch menschlichen Eingriff ausgeführt werden, z.B. durch einen Debugger oder Patchen des Codes.

\par

\clearpage
Dies könnte auch als einfache Übung zum Patchen von Nutzen sein. Öffnen wir die Executable in Hiew:

\begin{figure}[H]
\centering
\myincludegraphics{patterns/065_GOTO/hiew1.png}
\caption{Hiew}
\label{fig:goto_hiew1}
\end{figure}

\clearpage
Wir setzen den Cursor auf die Adresse \JMP (\TT{0x410}), drücken F3 (edit), drücken zweimal Null, sodass der Opcode
zu \TT{EB 00} verändert wird:

\begin{figure}[H]
\centering
\myincludegraphics{patterns/065_GOTO/hiew2.png}
\caption{Hiew}
\label{fig:goto_hiew2}
\end{figure}
Das zweite Byte des \JMP Opcodes gibt den relativen Offset für den Sprung an; 0 entspricht dabei der Stelle direkt
hinter dem aktuellen Befehl.

Auf diese Weise wird \JMP den zweiten Aufruf von \printf nicht überspringen.

Wir drücken F9 (save) und verlassen den Editor. Wenn wir die Executable jetzt ausführen, sehen wir dies:

\lstinputlisting[caption=Patched executable output]{patterns/065_GOTO/console.txt}
Das gleiche Ergebnis kann erreicht werden, wenn der \JMP Befehl durch 2 \NOP Befehle ersetzt wird.

\NOP hat einen Opcode von \TT{0x90} und die Länge 1 Byte, sodass wir 2 Befehle als Ersatz für \JMP, welcher eine Größe
von 2 Byte hat, benötigen.

\subsection{Dead code}
Der zweite Aufruf von \printf wird fachsprachlich auch \q{dead code} genannt.
Dies bedeutet, dass der Code nie ausgeführt wird. Wenn wir dieses Beispiel mit aktivierter Optimierung kompilieren,
entfernt der Compiler den \q{dead code} und es gibt keine Spur mehr von ihm:

\lstinputlisting[caption=\Optimizing MSVC 2012,style=customasmx86]{patterns/065_GOTO/MSVC_goto_Ox.asm}

Trotzdem hat der Compiler vergessen, den \q{skip me!} String ebenfalls zu entfernen.

%Note: cl "/Ox" option for maximum optimisation does get rid of "skip me" string as well

\subsection{\Exercise}

% TODO debugger example can fit here
Versuchen Sie das gleiche Ergebnis mit ihrem bevorzugten Compiler und Debugger zu erzielen.

