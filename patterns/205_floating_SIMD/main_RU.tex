% FIXME1 divide this file into separate ones...
\mysection{Работа с числами с плавающей запятой (SIMD)}

\label{floating_SIMD}
\myindex{IEEE 754}
\myindex{SIMD}
\myindex{SSE}
\myindex{SSE2}
Разумеется, FPU остался в x86-совместимых процессорах в то время, когда ввели расширения \ac{SIMD}.

\ac{SIMD}-расширения (SSE2) позволяют удобнее работать с числами с плавающей запятой.

Формат чисел остается тот же (IEEE 754).

\myindex{x86-64}
Так что современные компиляторы (включая те, что компилируют под x86-64) 
обычно используют \ac{SIMD}-инструкции вместо FPU-инструкций.

Это, можно сказать, хорошая новость, потому что работать с ними легче.

Примеры будем использовать из секции о FPU: \myref{sec:FPU}.

\subsection{Простой пример}

\lstinputlisting[style=customc]{patterns/12_FPU/1_simple/simple.c}

\subsubsection{x64}

\lstinputlisting[caption=\Optimizing MSVC 2012 x64,style=customasmx86]{patterns/205_floating_SIMD/simple_MSVC_2012_x64_Ox.asm}

Собственно, входные значения с плавающей запятой передаются через регистры \XMM{0}-\XMM{3}, 
а остальные --- через стек
\footnote{\href{http://go.yurichev.com/17263}{MSDN: Parameter Passing}}.

$a$ передается через \XMM{0}, $b$ --- через \XMM{1}.
Но XMM-регистры (как мы уже знаем из секции о \ac{SIMD}: \myref{SIMD_x86}) 128-битные, 
а значения типа \Tdouble --- 64-битные,
так что используется только младшая половина регистра.

\myindex{x86!\Instructions!DIVSD}
\TT{DIVSD} это SSE-инструкция, означает 
\q{Divide Scalar Double-Precision Floating-Point Values}, 
и просто делит значение типа \Tdouble на другое, лежащие в младших половинах операндов.

Константы закодированы компилятором в формате IEEE 754.

\myindex{x86!\Instructions!MULSD}
\myindex{x86!\Instructions!ADDSD}
\TT{MULSD} и \TT{ADDSD} работают так же, только производят умножение и сложение.

Результат работы функции типа \Tdouble функция оставляет в регистре \XMM{0}.\\
\\
Как работает неоптимизирующий MSVC:

\lstinputlisting[caption=MSVC 2012 x64,style=customasmx86]{patterns/205_floating_SIMD/simple_MSVC_2012_x64.asm}

\myindex{Shadow space}
Чуть более избыточно. 
Входные аргументы сохраняются в \q{shadow space} (\myref{shadow_space}), 
причем, только младшие половины регистров, т.е. только 64-битные значения типа \Tdouble{}.
Результат работы компилятора GCC точно такой же.

\subsubsection{x86}

Скомпилируем этот пример также и под x86. MSVC 2012 даже генерируя под x86, использует SSE2-инструкции:

\lstinputlisting[caption=\NonOptimizing MSVC 2012 x86,style=customasmx86]{patterns/205_floating_SIMD/simple_MSVC_2012_x86.asm}

\lstinputlisting[caption=\Optimizing MSVC 2012 x86,style=customasmx86]{patterns/205_floating_SIMD/simple_MSVC_2012_x86_Ox.asm}

Код почти такой же, правда есть пара отличий связанных с соглашениями о вызовах:

1) аргументы передаются не в XMM-регистрах, а через стек, как и прежде, в примерах с FPU (\myref{sec:FPU});

2) результат работы функции возвращается через \ST{0} --- для этого он через стек
(через локальную переменную \TT{tv}) копируется из XMM-регистра в \ST{0}.

\clearpage
Попробуем соптимизированный пример в \olly:

\begin{figure}[H]
\centering
\myincludegraphics{patterns/205_floating_SIMD/simple_olly1.png}
\caption{\olly: \TT{MOVSD} загрузила значение $a$ в \XMM{1}}
\label{fig:FPU_SIMD_simple_olly1}
\end{figure}

\clearpage
\begin{figure}[H]
\centering
\myincludegraphics{patterns/205_floating_SIMD/simple_olly2.png}
\caption{\olly: \TT{DIVSD} вычислила \gls{quotient} 
и оставила его в \XMM{1}}
\label{fig:FPU_SIMD_simple_olly2}
\end{figure}

\clearpage
\begin{figure}[H]
\centering
\myincludegraphics{patterns/205_floating_SIMD/simple_olly3.png}
\caption{\olly: \TT{MULSD} вычислила \gls{product} и оставила его в \XMM{0}}
\label{fig:FPU_SIMD_simple_olly3}
\end{figure}

\clearpage
\begin{figure}[H]
\centering
\myincludegraphics{patterns/205_floating_SIMD/simple_olly4.png}
\caption{\olly: \TT{ADDSD} прибавила значение в \XMM{0} к \XMM{1}}
\label{fig:FPU_SIMD_simple_olly4}
\end{figure}

\clearpage
\begin{figure}[H]
\centering
\myincludegraphics{patterns/205_floating_SIMD/simple_olly5.png}
\caption{\olly: \FLD оставляет результат функции в \ST{0}}
\label{fig:FPU_SIMD_simple_olly5}
\end{figure}

Видно, что \olly показывает XMM-регистры как пары чисел в формате \Tdouble,
но используется только \emph{младшая} часть.

Должно быть, \olly показывает их именно так, потому что сейчас исполняются SSE2-инструкции
с суффиксом \TT{-SD}.

Но конечно же, можно переключить отображение значений в регистрах и посмотреть содержимое
как 4 \Tfloat{}-числа или просто как 16 байт.

\clearpage
\subsection{Передача чисел с плавающей запятой в аргументах}

\lstinputlisting[style=customc]{patterns/12_FPU/2_passing_floats/pow.c}

Они передаются в младших половинах регистров \XMM{0}-\XMM{3}.

\lstinputlisting[caption=\Optimizing MSVC 2012 x64,style=customasmx86]{patterns/205_floating_SIMD/pow_MSVC_2012_x64_Ox.asm}

\myindex{x86!\Instructions!MOVSD}
\myindex{x86!\Instructions!MOVSDX}
Инструкции \TT{MOVSDX} нет в документации от Intel и AMD  (\myref{x86_manuals}), там она называется просто \TT{MOVSD}.
Таким образом, в процессорах x86 две инструкции с одинаковым именем (о второй: \myref{REP_MOVSx}).
Возможно, в Microsoft решили избежать путаницы и переименовали инструкцию в \TT{MOVSDX}.
Она просто загружает значение в младшую половину XMM-регистра.

Функция \TT{pow()} берет аргументы из \XMM{0} и \XMM{1}, 
и возвращает результат в \XMM{0}.
Далее он перекладывается в \RDX для \printf. 
Почему? 
Может быть, это потому что 
\printf --- функция с переменным количеством аргументов?

\lstinputlisting[caption=\Optimizing GCC 4.4.6 x64,style=customasmx86]{patterns/205_floating_SIMD/pow_GCC446_x64_O3_RU.s}

GCC работает понятнее. 
Значение для \printf передается в \XMM{0}. 
Кстати, вот тот случай, когда в \EAX
для \printf записывается 1 --- это значит, что будет передан один аргумент в векторных регистрах, 
так того требует стандарт \SysVABI.

\subsection{Пример со сравнением}

\lstinputlisting[style=customc]{patterns/12_FPU/3_comparison/d_max.c}

\subsubsection{x64}

\lstinputlisting[caption=\Optimizing MSVC 2012 x64,style=customasmx86]{patterns/205_floating_SIMD/d_max_MSVC_2012_x64_Ox.asm}

\Optimizing MSVC генерирует очень понятный код.

\myindex{x86!\Instructions!COMISD}
Инструкция \TT{COMISD} это \q{Compare Scalar Ordered Double-Precision Floating-Point 
Values and Set EFLAGS}. Собственно, это она и делает.\\
\\
\NonOptimizing MSVC генерирует более избыточно, но тоже всё понятно:

\lstinputlisting[caption=MSVC 2012 x64,style=customasmx86]{patterns/205_floating_SIMD/d_max_MSVC_2012_x64.asm}

\myindex{x86!\Instructions!MAXSD}
А вот GCC 4.4.6 дошел в оптимизации дальше и применил инструкцию \TT{MAXSD} (\q{Return Maximum Scalar 
Double-Precision Floating-Point Value}), которая просто выбирает максимальное значение!

\lstinputlisting[caption=\Optimizing GCC 4.4.6 x64,style=customasmx86]{patterns/205_floating_SIMD/d_max_GCC446_x64_O3.s}

\clearpage
\subsubsection{x86}

Скомпилируем этот пример в MSVC 2012 с включенной оптимизацией:

\lstinputlisting[caption=\Optimizing MSVC 2012 x86,style=customasmx86]{patterns/205_floating_SIMD/d_max_MSVC_2012_x86_Ox.asm}

Всё то же самое, только значения $a$ и $b$ 
берутся из стека, а результат функции оставляется в \ST{0}.

Если загрузить этот пример в \olly, 
увидим, как инструкция \TT{COMISD} сравнивает значения и устанавливает/сбрасывает
флаги \CF и \PF:

\begin{figure}[H]
\centering
\myincludegraphics{patterns/205_floating_SIMD/d_max_olly.png}
\caption{\olly: \TT{COMISD} изменила флаги \CF и \PF}
\label{fig:FPU_SIMD_d_max_olly}
\end{figure}

\subsection{Вычисление машинного эпсилона: x64 и SIMD}
\label{machine_epsilon_x64_and_SIMD}

Вернемся к примеру \q{вычисление машинного эпсилона} для \Tdouble \lstref{machine_epsilon_double_c}.

Теперь скомпилируем его для x64:

\lstinputlisting[caption=\Optimizing MSVC 2012 x64,style=customasmx86]{patterns/205_floating_SIMD/epsilon_double_MSVC_2012_x64_Ox.asm}

Нет способа прибавить 1 к значению в 128-битном XMM-регистре, так что его нужно в начале поместить в память.

Впрочем, есть инструкция \INS{ADDSD} (\emph{Add Scalar Double-Precision Floating-Point Values}),
которая может прибавить значение к младшей 64-битной части XMM-регистра игнорируя старшую половину,
но наверное MSVC 2012 пока недостаточно хорош для этого

\footnote{В качестве упражнения, вы можете попробовать переработать этот код, чтобы избавиться 
от использования локального стека.}.

Так или иначе, значение затем перезагружается в XMM-регистр и происходит вычитание.

\INS{SUBSD} это \q{Subtract Scalar Double-Precision Floating-Point Values}, 
т.е. операция производится над младшей 64-битной частью 128-битного XMM-регистра.
Результат возвращается в регистре XMM0.

\mysection{Разгон майнера биткоинов Cointerra}
\index{Bitcoin}
\index{BeagleBone}

Был такой майнер биткоинов Cointerra, выглядящий так:

\begin{figure}[H]
\centering
\myincludegraphics{examples/bitcoin_miner/board.jpg}
\caption{Board}
\end{figure}

И была также (возможно утекшая) утилита\footnote{Можно скачать здесь: \url{https://github.com/DennisYurichev/RE-for-beginners/raw/master/examples/bitcoin_miner/files/cointool-overclock}}
которая могла выставлять тактовую частоту платы.
Она запускается на дополнительной плате BeagleBone на ARM с Linux (маленькая плата внизу фотографии).

И у автора (этих строк) однажды спросили, можно ли хакнуть эту утилиту и посмотреть, какие частоты можно выставлять, и какие нет.
И можно ли твикнуть её?

Утилиту нужно запускать так: \TT{./cointool-overclock 0 0 900}, где 900 это частота в МГц.
Если частота слишком большая, утилита выведет ошибку \q{Error with arguments} и закончит работу.

Вот фрагмент кода вокруг ссылки на текстовую строку \q{Error with arguments}:

\begin{lstlisting}[style=customasmARM]

...

.text:0000ABC4         STR      R3, [R11,#var_28]
.text:0000ABC8         MOV      R3, #optind
.text:0000ABD0         LDR      R3, [R3]
.text:0000ABD4         ADD      R3, R3, #1
.text:0000ABD8         MOV      R3, R3,LSL#2
.text:0000ABDC         LDR      R2, [R11,#argv]
.text:0000ABE0         ADD      R3, R2, R3
.text:0000ABE4         LDR      R3, [R3]
.text:0000ABE8         MOV      R0, R3  ; nptr
.text:0000ABEC         MOV      R1, #0  ; endptr
.text:0000ABF0         MOV      R2, #0  ; base
.text:0000ABF4         BL       strtoll
.text:0000ABF8         MOV      R2, R0
.text:0000ABFC         MOV      R3, R1
.text:0000AC00         MOV      R3, R2
.text:0000AC04         STR      R3, [R11,#var_2C]
.text:0000AC08         MOV      R3, #optind
.text:0000AC10         LDR      R3, [R3]
.text:0000AC14         ADD      R3, R3, #2
.text:0000AC18         MOV      R3, R3,LSL#2
.text:0000AC1C         LDR      R2, [R11,#argv]
.text:0000AC20         ADD      R3, R2, R3
.text:0000AC24         LDR      R3, [R3]
.text:0000AC28         MOV      R0, R3  ; nptr
.text:0000AC2C         MOV      R1, #0  ; endptr
.text:0000AC30         MOV      R2, #0  ; base
.text:0000AC34         BL       strtoll
.text:0000AC38         MOV      R2, R0
.text:0000AC3C         MOV      R3, R1
.text:0000AC40         MOV      R3, R2
.text:0000AC44         STR      R3, [R11,#third_argument]
.text:0000AC48         LDR      R3, [R11,#var_28]
.text:0000AC4C         CMP      R3, #0
.text:0000AC50         BLT      errors_with_arguments
.text:0000AC54         LDR      R3, [R11,#var_28]
.text:0000AC58         CMP      R3, #1
.text:0000AC5C         BGT      errors_with_arguments
.text:0000AC60         LDR      R3, [R11,#var_2C]
.text:0000AC64         CMP      R3, #0
.text:0000AC68         BLT      errors_with_arguments
.text:0000AC6C         LDR      R3, [R11,#var_2C]
.text:0000AC70         CMP      R3, #3
.text:0000AC74         BGT      errors_with_arguments
.text:0000AC78         LDR      R3, [R11,#third_argument]
.text:0000AC7C         CMP      R3, #0x31
.text:0000AC80         BLE      errors_with_arguments
.text:0000AC84         LDR      R2, [R11,#third_argument]
.text:0000AC88         MOV      R3, #950
.text:0000AC8C         CMP      R2, R3
.text:0000AC90         BGT      errors_with_arguments
.text:0000AC94         LDR      R2, [R11,#third_argument]
.text:0000AC98         MOV      R3, #0x51EB851F
.text:0000ACA0         SMULL    R1, R3, R3, R2
.text:0000ACA4         MOV      R1, R3,ASR#4
.text:0000ACA8         MOV      R3, R2,ASR#31
.text:0000ACAC         RSB      R3, R3, R1
.text:0000ACB0         MOV      R1, #50
.text:0000ACB4         MUL      R3, R1, R3
.text:0000ACB8         RSB      R3, R3, R2
.text:0000ACBC         CMP      R3, #0
.text:0000ACC0         BEQ      loc_ACEC
.text:0000ACC4
.text:0000ACC4 errors_with_arguments
.text:0000ACC4                                         
.text:0000ACC4         LDR      R3, [R11,#argv]
.text:0000ACC8         LDR      R3, [R3]
.text:0000ACCC         MOV      R0, R3  ; path
.text:0000ACD0         BL       __xpg_basename
.text:0000ACD4         MOV      R3, R0
.text:0000ACD8         MOV      R0, #aSErrorWithArgu ; format
.text:0000ACE0         MOV      R1, R3
.text:0000ACE4         BL       printf
.text:0000ACE8         B        loc_ADD4
.text:0000ACEC ; ------------------------------------------------------------
.text:0000ACEC
.text:0000ACEC loc_ACEC                 ; CODE XREF: main+66C
.text:0000ACEC         LDR      R2, [R11,#third_argument]
.text:0000ACF0         MOV      R3, #499
.text:0000ACF4         CMP      R2, R3
.text:0000ACF8         BGT      loc_AD08
.text:0000ACFC         MOV      R3, #0x64
.text:0000AD00         STR      R3, [R11,#unk_constant]
.text:0000AD04         B        jump_to_write_power
.text:0000AD08 ; ------------------------------------------------------------
.text:0000AD08
.text:0000AD08 loc_AD08                 ; CODE XREF: main+6A4
.text:0000AD08         LDR      R2, [R11,#third_argument]
.text:0000AD0C         MOV      R3, #799
.text:0000AD10         CMP      R2, R3
.text:0000AD14         BGT      loc_AD24
.text:0000AD18         MOV      R3, #0x5F
.text:0000AD1C         STR      R3, [R11,#unk_constant]
.text:0000AD20         B        jump_to_write_power
.text:0000AD24 ; ------------------------------------------------------------
.text:0000AD24
.text:0000AD24 loc_AD24                 ; CODE XREF: main+6C0
.text:0000AD24         LDR      R2, [R11,#third_argument]
.text:0000AD28         MOV      R3, #899
.text:0000AD2C         CMP      R2, R3
.text:0000AD30         BGT      loc_AD40
.text:0000AD34         MOV      R3, #0x5A
.text:0000AD38         STR      R3, [R11,#unk_constant]
.text:0000AD3C         B        jump_to_write_power
.text:0000AD40 ; ------------------------------------------------------------
.text:0000AD40
.text:0000AD40 loc_AD40                 ; CODE XREF: main+6DC
.text:0000AD40         LDR      R2, [R11,#third_argument]
.text:0000AD44         MOV      R3, #999
.text:0000AD48         CMP      R2, R3
.text:0000AD4C         BGT      loc_AD5C
.text:0000AD50         MOV      R3, #0x55
.text:0000AD54         STR      R3, [R11,#unk_constant]
.text:0000AD58         B        jump_to_write_power
.text:0000AD5C ; ------------------------------------------------------------
.text:0000AD5C
.text:0000AD5C loc_AD5C                 ; CODE XREF: main+6F8
.text:0000AD5C         LDR      R2, [R11,#third_argument]
.text:0000AD60         MOV      R3, #1099
.text:0000AD64         CMP      R2, R3
.text:0000AD68         BGT      jump_to_write_power
.text:0000AD6C         MOV      R3, #0x50
.text:0000AD70         STR      R3, [R11,#unk_constant]
.text:0000AD74
.text:0000AD74 jump_to_write_power                     ; CODE XREF: main+6B0
.text:0000AD74                                         ; main+6CC ...
.text:0000AD74         LDR      R3, [R11,#var_28]
.text:0000AD78         UXTB     R1, R3
.text:0000AD7C         LDR      R3, [R11,#var_2C]
.text:0000AD80         UXTB     R2, R3
.text:0000AD84         LDR      R3, [R11,#unk_constant]
.text:0000AD88         UXTB     R3, R3
.text:0000AD8C         LDR      R0, [R11,#third_argument]
.text:0000AD90         UXTH     R0, R0
.text:0000AD94         STR      R0, [SP,#0x44+var_44]
.text:0000AD98         LDR      R0, [R11,#var_24]
.text:0000AD9C         BL       write_power
.text:0000ADA0         LDR      R0, [R11,#var_24]
.text:0000ADA4         MOV      R1, #0x5A
.text:0000ADA8         BL       read_loop
.text:0000ADAC         B        loc_ADD4

...

.rodata:0000B378 aSErrorWithArgu DCB "%s: Error with arguments",0xA,0 ; DATA XREF: main+684

...

\end{lstlisting}

Имена ф-ций присутствовали в отладочной информации в оригинальном исполняемом файле,
такие как \TT{write\_power}, \TT{read\_loop}.
Но имена меткам внутри ф-ции дал я.

\myindex{UNIX!getopt}
\myindex{strtoll()}
Имя \TT{optind} звучит знакомо. Это библиотека \IT{getopt} из *NIX предназначенная для парсинга командной строки ---
и это то, что внутри и происходит.
Затем, третий аргумент (где передается значение частоты) конвертируется из строку в число используя вызов ф-ции \IT{strtoll()}.

Значение затем сравнивается с разными константами.
На 0xACEC есть проверка, меньше ли оно или равно 499, и если это так, то 0x64 будет передано в ф-цию
\TT{write\_power()} (которая посылает команду через USB используя \TT{send\_msg()}).
Если значение больше 499, происходит переход на 0xAD08.

На 0xAD08 есть проверка, меньше ли оно или равно 799. Если это так, то 0x5F передается в ф-цию \TT{write\_power()}.

Есть еще проверки: на 899 на 0xAD24, на 0x999 на 0xAD40, и наконец, на 1099 на 0xAD5C.
Если входная частота меньше или равна 1099, 0x50 (на 0xAD6C) будет передано в ф-цию \TT{write\_power()}.
И тут что-то вроде баги.
Если значение все еще больше 1099, само значение будет передано в ф-цию \TT{write\_power()}.
Но с другой стороны это не бага, потому что мы не можем попасть сюда: значение в начале проверяется с 950 на 0xAC88,
и если оно больше, выводится сообщение об ошибке и утилита заканчивает работу.

Вот таблица между частотами в МГц и значениями передаваемыми в ф-цию \TT{write\_power()}:

\begin{center}
\begin{longtable}{ | l | l | l | }
\hline
\HeaderColor МГц & \HeaderColor шестнадцатеричное представление & \HeaderColor десятичное \\
\hline
499MHz & 0x64 & 100 \\
\hline
799MHz & 0x5f & 95 \\
\hline
899MHz & 0x5a & 90 \\
\hline
999MHz & 0x55 & 85 \\
\hline
1099MHz & 0x50 & 80 \\
\hline
\end{longtable}
\end{center}

Как видно, значение передаваемое в плату постепенно уменьшается с ростом частоты.

Видно что значение в 950МГц это жесткий предел, по крайней мере в этой утилите. Можно ли её обмануть?

Вернемся к этому фрагменту кода:

\begin{lstlisting}[style=customasmARM]
.text:0000AC84      LDR     R2, [R11,#third_argument]
.text:0000AC88      MOV     R3, #950
.text:0000AC8C      CMP     R2, R3
.text:0000AC90      BGT     errors_with_arguments ; Я пропатчил здесь на 00 00 00 00
\end{lstlisting}

Нам нужно как-то запретить инструкцию перехода \INS{BGT} на 0xAC90. И это ARM в режиме ARM, потому что, как мы видим,
все адреса увеличиваются на 4, т.е., длина каждой инструкции это 4 байта.
Инструкция \TT{NOP} (нет операции) в режиме ARM это просто 4 нулевых байта: \TT{00 00 00 00}.
Так что, записывая 4 нуля по адресу 0xAC90 (или по физическому смещению в файле: 0x2C90) мы можем выключить
эту проверку.

Теперь можно выставлять частоты вплоть до 1050МГц. И даже больше, но из-за ошибки, если входное значение больше 1099,
значение в МГц, \IT{как есть}, будет передано в плату, что неправильно.

Дальше я не разбирался, но если бы продолжил, я бы уменьшал значение передаваемое в ф-цию \TT{write\_power()}.

Теперь страшный фрагмент кода, который я в начале пропустил:

\lstinputlisting[style=customasmARM]{examples/bitcoin_miner/tmp1.lst}

Здесь используется деление через умножение, и константа 0x51EB851F.
Я написал для себя простой программистский калькулятор\footnote{\url{https://github.com/DennisYurichev/progcalc}}.
И там есть возможность вычислять обратное число по модулю.

\begin{lstlisting}
modinv32(0x51EB851F)
Warning, result is not integer: 3.125000
(unsigned) dec: 3 hex: 0x3 bin: 11
\end{lstlisting}

Это значит что инструкция \INS{SMULL} на 0xACA0 просто делит 3-й аргумент на 3.125.
На самом деле, все что делает ф-ция \TT{modinv32()} в моем калькуляторе, это:

\[
\frac{1}{\frac{input}{2^{32}}} = \frac{2^{32}}{input}
\]

Потом там есть дополнительные сдвиги и теперь мы видим что 3-й аргумент просто делится на 50.
И затем умножается снова на 50.
Зачем?
Это простейшая проверка, можно ли делить входное значение на 50 без остатка.
Если значение этого выражения ненулевое, $x$ не может быть разделено на 50 без остатка:

\[
x-((\frac{x}{50}) \cdot 50)
\]

На самом деле, это простой способ вычисления остатка от деления.

И затем, если остаток ненулевой, выводится сообщение об ошибке.
Так что эта утилита берет значения частотв вроде 850, 900, 950, 1000, итд, но не 855 или 911.

Вот и всё! Если вы делаете что-то такое, имейте ввиду, что это может испортить вашу плату, как и в случае разгона
чипов вроде \ac{CPU}, \ac{GPU}, итд.
Если у вас есть плата Cointerra, делайте всё это на свой собственный риск!



\subsection{Итог}

Во всех приведенных примерах, в XMM-регистрах используется только младшая половина регистра, там
хранится значение в формате IEEE 754.

Собственно, все инструкции с суффиксом 
\TT{-SD} (\q{Scalar Double-Precision}) --- это инструкции для работы с числами с плавающей 
запятой в формате IEEE 754, 
хранящиеся в младшей 64-битной половине XMM-регистра.

Всё удобнее чем это было в FPU, видимо, сказывается тот факт, что расширения 
SIMD развивались не так стихийно, как FPU в прошлом.

Стековая модель регистров не используется.

\myindex{x86!\Instructions!ADDSS}
\myindex{x86!\Instructions!MOVSS}
\myindex{x86!\Instructions!COMISS}
% TODO1: do this!
% FIXME1 ... but their -SS versions
Если вы попробуете заменить в этих примерах \Tdouble на \Tfloat{}, 
то инструкции будут использоваться те же, только с суффиксом \TT{-SS}
(\q{Scalar Single-Precision}), например, \TT{MOVSS}, \TT{COMISS}, \TT{ADDSS}, итд.

\q{Scalar} означает, что SIMD-регистр будет хранить только одно значение, вместо нескольких.

Инструкции, работающие с несколькими значениями в регистре одновременно, имеют \q{Packed} в названии.

Нужно также обратить внимание, что SSE2-инструкции работают с 64-битными числами (\Tdouble) в формате IEEE 754,
в то время как внутреннее представление в FPU --- 80-битные числа.

Поэтому ошибок округления (\emph{round-off error}) в FPU может быть меньше чем в SSE2,
как следствие, можно сказать, работа с FPU может давать более точные результаты вычислений.

