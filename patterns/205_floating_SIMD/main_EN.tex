% FIXME1 divide this file into separate ones...
\mysection{Working with floating point numbers using SIMD}

\label{floating_SIMD}
\myindex{IEEE 754}
\myindex{SIMD}
\myindex{SSE}
\myindex{SSE2}

Of course, the \ac{FPU} has remained in x86-compatible processors when the \ac{SIMD} extensions were added.

The \ac{SIMD} extensions (SSE2) offer an easier way to work with floating-point numbers.

The number format remains the same (IEEE 754).

\myindex{x86-64}
So, modern compilers (including those generating
for x86-64) usually use \ac{SIMD} instructions instead of FPU ones.

It can be said that it's good news, because it's easier to work with them.

We are going to reuse the examples from the FPU section here: \myref{sec:FPU}.

\subsection{Simple example}

\lstinputlisting[style=customc]{patterns/12_FPU/1_simple/simple.c}

\subsubsection{x64}

\lstinputlisting[caption=\Optimizing MSVC 2012 x64,style=customasmx86]{patterns/205_floating_SIMD/simple_MSVC_2012_x64_Ox.asm}

The input floating point values are passed in the \XMM{0}-\XMM{3} registers,
all the rest---via the stack
\footnote{\href{http://go.yurichev.com/17263}{MSDN: Parameter Passing}}.

$a$ is passed in \XMM{0}, $b$---via \XMM{1}.

The XMM-registers are 128-bit (as we know from the section about \ac{SIMD}: \myref{SIMD_x86}), 
but the \Tdouble values are 64 bit, so only lower register half is used.

\myindex{x86!\Instructions!DIVSD}
\TT{DIVSD} is an SSE-instruction that stands for 
\q{Divide Scalar Double-Precision Floating-Point Values}, 
it just divides
one value of type \Tdouble by another, stored in the lower halves of operands.

The constants are encoded by compiler in IEEE 754 format.

\myindex{x86!\Instructions!MULSD}
\myindex{x86!\Instructions!ADDSD}
\TT{MULSD} and \TT{ADDSD} work just as the same, but do multiplication and addition.

The result of the function's execution in type \Tdouble is left in the in \XMM{0} register.\\
\\
That is how non-optimizing MSVC works:

\lstinputlisting[caption=MSVC 2012 x64,style=customasmx86]{patterns/205_floating_SIMD/simple_MSVC_2012_x64.asm}

\myindex{Shadow space}
Slightly redundant. 
The input arguments are saved in the \q{shadow space} (\myref{shadow_space}), 
but only their lower register halves, i.e., only 64-bit values of type \Tdouble.
GCC produces the same code.

\subsubsection{x86}

Let's also compile this example for x86. Despite the fact it's generating for x86, MSVC 2012 uses SSE2 instructions:

\lstinputlisting[caption=\NonOptimizing MSVC 2012 x86,style=customasmx86]{patterns/205_floating_SIMD/simple_MSVC_2012_x86.asm}

\lstinputlisting[caption=\Optimizing MSVC 2012 x86,style=customasmx86]{patterns/205_floating_SIMD/simple_MSVC_2012_x86_Ox.asm}

It's almost the same code, however, there are some differences related to calling conventions:
1) the arguments are passed not in XMM registers, but in the stack, like in the FPU examples (\myref{sec:FPU});
2) the result of the function is returned in \ST{0} --- in order to do so, it's copied
(through local variable \TT{tv}) from one of the XMM registers to \ST{0}.

\clearpage
Let's try the optimized example in \olly:

\begin{figure}[H]
\centering
\myincludegraphics{patterns/205_floating_SIMD/simple_olly1.png}
\caption{\olly: \TT{MOVSD} loads the value of $a$ into \XMM{1}}
\label{fig:FPU_SIMD_simple_olly1}
\end{figure}

\clearpage
\begin{figure}[H]
\centering
\myincludegraphics{patterns/205_floating_SIMD/simple_olly2.png}
\caption{\olly: \TT{DIVSD} calculated \gls{quotient} 
and stored it in \XMM{1}}
\label{fig:FPU_SIMD_simple_olly2}
\end{figure}

\clearpage
\begin{figure}[H]
\centering
\myincludegraphics{patterns/205_floating_SIMD/simple_olly3.png}
\caption{\olly: \TT{MULSD} calculated \gls{product} and stored it
in \XMM{0}}
\label{fig:FPU_SIMD_simple_olly3}
\end{figure}

\clearpage
\begin{figure}[H]
\centering
\myincludegraphics{patterns/205_floating_SIMD/simple_olly4.png}
\caption{\olly: \TT{ADDSD} adds value in \XMM{0} to \XMM{1}}
\label{fig:FPU_SIMD_simple_olly4}
\end{figure}

\clearpage
\begin{figure}[H]
\centering
\myincludegraphics{patterns/205_floating_SIMD/simple_olly5.png}
\caption{\olly: \FLD left function result in \ST{0}}
\label{fig:FPU_SIMD_simple_olly5}
\end{figure}

We see that \olly shows the XMM registers as pairs of \Tdouble numbers,
but only the \emph{lower} part is used.

Apparently, \olly shows them in that format because the SSE2 instructions (suffixed with \TT{-SD}) 
are executed right now.

But of course, it's possible to switch the register format and to see their contents as
4 \Tfloat{}-numbers or just as 16 bytes.

\clearpage
\subsection{Passing floating point number via arguments}

\lstinputlisting[style=customc]{patterns/12_FPU/2_passing_floats/pow.c}

They are passed in the lower halves
of the \XMM{0}-\XMM{3} registers.

\lstinputlisting[caption=\Optimizing MSVC 2012 x64,style=customasmx86]{patterns/205_floating_SIMD/pow_MSVC_2012_x64_Ox.asm}

\myindex{x86!\Instructions!MOVSD}
\myindex{x86!\Instructions!MOVSDX}
There is no \TT{MOVSDX} instruction in Intel and AMD  manuals (\myref{x86_manuals}), there it is called just \TT{MOVSD}.
So there are two instructions sharing the same name in x86 (about the other see: \myref{REP_MOVSx}).
Apparently, Microsoft developers wanted to get rid of the mess, so they renamed it to \TT{MOVSDX}.
It just loads a value into the lower half of a XMM register.

\TT{pow()} takes arguments from \XMM{0} and \XMM{1}, and returns result in \XMM{0}.
It is then moved to \RDX for \printf. 
Why? 
Maybe because 
\printf{}---is a variable arguments function?

\lstinputlisting[caption=\Optimizing GCC 4.4.6 x64,style=customasmx86]{patterns/205_floating_SIMD/pow_GCC446_x64_O3_EN.s}

GCC generates clearer output. 
The value for \printf is passed in \XMM{0}. 
By the way, here is a case when 1 is written into \EAX
for \printf ---this implies that one argument will be passed in vector registers,
just as the standard requires \SysVABI.

\subsection{Comparison example}

\lstinputlisting[style=customc]{patterns/12_FPU/3_comparison/d_max.c}

\subsubsection{x64}

\lstinputlisting[caption=\Optimizing MSVC 2012 x64,style=customasmx86]{patterns/205_floating_SIMD/d_max_MSVC_2012_x64_Ox.asm}

\Optimizing MSVC generates a code very easy to understand.

\myindex{x86!\Instructions!COMISD}
\TT{COMISD} is \q{Compare Scalar Ordered Double-Precision Floating-Point 
Values and Set EFLAGS}. Essentially, that is what it does.\\
\\
\NonOptimizing MSVC generates more redundant code,
but it is still not hard to understand:

\lstinputlisting[caption=MSVC 2012 x64,style=customasmx86]{patterns/205_floating_SIMD/d_max_MSVC_2012_x64.asm}

\myindex{x86!\Instructions!MAXSD}
However, GCC 4.4.6 
did more optimizations and used the \TT{MAXSD} (\q{Return Maximum Scalar 
Double-Precision Floating-Point Value}) instruction,
which just choose the maximum value!

\lstinputlisting[caption=\Optimizing GCC 4.4.6 x64,style=customasmx86]{patterns/205_floating_SIMD/d_max_GCC446_x64_O3.s}

\clearpage
\subsubsection{x86}

Let's compile this example in MSVC 2012 with optimization turned on:

\lstinputlisting[caption=\Optimizing MSVC 2012 x86,style=customasmx86]{patterns/205_floating_SIMD/d_max_MSVC_2012_x86_Ox.asm}

Almost the same, but the values of $a$ and $b$ are taken from the stack and the function result 
is left in \ST{0}.

If we load this example in \olly, 
we can see how the \TT{COMISD} instruction compares values and sets/clears the \CF and \PF flags:

\begin{figure}[H]
\centering
\myincludegraphics{patterns/205_floating_SIMD/d_max_olly.png}
\caption{\olly: \TT{COMISD} changed \CF and \PF flags}
\label{fig:FPU_SIMD_d_max_olly}
\end{figure}

\subsection{Calculating machine epsilon: x64 and SIMD}
\label{machine_epsilon_x64_and_SIMD}

Let's revisit the \q{calculating machine epsilon} example for \Tdouble \lstref{machine_epsilon_double_c}.

Now we compile it for x64:

\lstinputlisting[caption=\Optimizing MSVC 2012 x64,style=customasmx86]{patterns/205_floating_SIMD/epsilon_double_MSVC_2012_x64_Ox.asm}

There is no way to add 1 to a value in 128-bit XMM register, so it must be placed into memory.

There is, however, the \INS{ADDSD} instruction (\emph{Add Scalar Double-Precision Floating-Point Values}) 
which can add a value to the lowest 64-bit half of a XMM register while ignoring the higher one, 
but MSVC 2012 probably is not that good yet
\footnote{As an exercise, you may try to rework this code to 
eliminate the usage of the local stack.}.

Nevertheless, the value is then reloaded to a XMM register and subtraction occurs.
\INS{SUBSD} is \q{Subtract Scalar Double-Precision Floating-Point Values}, 
i.e., it operates on the lower 64-bit part of 128-bit XMM register.
The result is returned in the XMM0 register.

\mysection{\MinesweeperWinXPExampleChapterName}
\label{minesweeper_winxp}
\myindex{Windows!Windows XP}

For those who are not very good at playing Minesweeper, we could try to reveal the hidden mines in the debugger.

\myindex{\CStandardLibrary!rand()}
\myindex{Windows!PDB}

As we know, Minesweeper places mines randomly, so there has to be some kind of random number generator or
a call to the standard \TT{rand()} C-function.

What is really cool about reversing Microsoft products is that there are \gls{PDB} 
file with symbols (function names, \etc{}).
When we load \TT{winmine.exe} into \IDA, it downloads the 
\gls{PDB} file exactly for this 
executable and shows all names.

So here it is, the only call to \TT{rand()} is this function:

\lstinputlisting[style=customasmx86]{examples/minesweeper/tmp1.lst}

\IDA named it so, and it was the name given to it by Minesweeper's developers.

The function is very simple:

\begin{lstlisting}[style=customc]
int Rnd(int limit)
{
    return rand() % limit;
};
\end{lstlisting}

(There is no \q{limit} name in the \gls{PDB} file; we manually named this argument like this.)

So it returns 
a random value from 0 to a specified limit.

\TT{Rnd()} is called only from one place, 
a function called \TT{StartGame()}, 
and as it seems, this is exactly 
the code which place the mines:

\begin{lstlisting}[style=customasmx86]
.text:010036C7                 push    _xBoxMac
.text:010036CD                 call    _Rnd@4          ; Rnd(x)
.text:010036D2                 push    _yBoxMac
.text:010036D8                 mov     esi, eax
.text:010036DA                 inc     esi
.text:010036DB                 call    _Rnd@4          ; Rnd(x)
.text:010036E0                 inc     eax
.text:010036E1                 mov     ecx, eax
.text:010036E3                 shl     ecx, 5          ; ECX=ECX*32
.text:010036E6                 test    _rgBlk[ecx+esi], 80h
.text:010036EE                 jnz     short loc_10036C7
.text:010036F0                 shl     eax, 5          ; EAX=EAX*32
.text:010036F3                 lea     eax, _rgBlk[eax+esi]
.text:010036FA                 or      byte ptr [eax], 80h
.text:010036FD                 dec     _cBombStart
.text:01003703                 jnz     short loc_10036C7
\end{lstlisting}

Minesweeper allows you to set the board size, so the X (xBoxMac) and Y (yBoxMac) of the board are global variables.
They are passed to \TT{Rnd()} and random 
coordinates are generated.
A mine is placed by the \TT{OR} instruction at \TT{0x010036FA}. 
And if it has been placed before 
(it's possible if the pair of \TT{Rnd()} 
generates a coordinates pair which has been already 
generated), 
then \TT{TEST} and \TT{JNZ} at \TT{0x010036E6} 
jumps to the generation routine again.

\TT{cBombStart} is the global variable containing total number of mines. So this is loop.

The width of the array is 32 
(we can conclude this by looking at the \TT{SHL} instruction, which multiplies one of the coordinates by 32).

The size of the \TT{rgBlk} 
global array can be easily determined by the difference 
between the \TT{rgBlk} 
label in the data segment and the next known one. 
It is 0x360 (864):

\begin{lstlisting}[style=customasmx86]
.data:01005340 _rgBlk          db 360h dup(?)          ; DATA XREF: MainWndProc(x,x,x,x)+574
.data:01005340                                         ; DisplayBlk(x,x)+23
.data:010056A0 _Preferences    dd ?                    ; DATA XREF: FixMenus()+2
...
\end{lstlisting}

$864/32=27$.

So the array size is $27*32$?
It is close to what we know: when we try to set board size to $100*100$ in Minesweeper settings, it fallbacks to a board of size $24*30$.
So this is the maximal board size here.
And the array has a fixed size for any board size.

So let's see all this in \olly.
We will ran Minesweeper, attaching \olly to it and now we can see the memory dump at the address of the \TT{rgBlk} array (\TT{0x01005340})
\footnote{All addresses here are for Minesweeper for Windows XP SP3 English. 
They may differ for other service packs.}.

So we got this memory dump of the array:

\lstinputlisting[style=customasmx86]{examples/minesweeper/1.lst}

\olly, like any other hexadecimal editor, shows 16 bytes per line.
So each 32-byte array row occupies exactly 2 lines here.

This is beginner level (9*9 board).

There is some square 
structure can be seen visually (0x10 bytes).

We will click \q{Run} in \olly to unfreeze the Minesweeper process, then we'll clicked randomly at the Minesweeper window 
and trapped into mine, but now all mines are visible:

\begin{figure}[H]
\centering
\myincludegraphicsSmall{examples/minesweeper/1.png}
\caption{Mines}
\label{fig:minesweeper1}
\end{figure}

By comparing the mine places and the dump, we can conclude that 0x10 stands for border, 0x0F---empty block, 0x8F---mine.
Perhaps, 0x10 is just a \emph{sentinel value}.

Now we'll add comments and also enclose all 0x8F bytes into square brackets:

\lstinputlisting[style=customasmx86]{examples/minesweeper/2.lst}

Now we'll remove all \emph{border bytes} (0x10) and what's beyond those:

\lstinputlisting[style=customasmx86]{examples/minesweeper/3.lst}

Yes, these are mines, now it can be clearly seen and compared with the screenshot.

\clearpage
What is interesting is that we can modify the array right in \olly.
We can remove all mines by changing all 0x8F bytes by 0x0F, and here is what we'll get in Minesweeper:

\begin{figure}[H]
\centering
\myincludegraphicsSmall{examples/minesweeper/3.png}
\caption{All mines are removed in debugger}
\label{fig:minesweeper3}
\end{figure}

We can also move all of them to the first line: 

\begin{figure}[H]
\centering
\myincludegraphicsSmall{examples/minesweeper/2.png}
\caption{Mines set in debugger}
\label{fig:minesweeper2}
\end{figure}

Well, the debugger is not very convenient for eavesdropping (which is our goal anyway), so we'll write a small utility
to dump the contents of the board:

\lstinputlisting[style=customc]{examples/minesweeper/minesweeper_cheater.c}

Just set the \ac{PID}
\footnote{PID it can be seen in Task Manager 
(enable it in \q{View $\rightarrow$ Select Columns})} 
and the address of the array (\TT{0x01005340} for Windows XP SP3 English) 
and it will dump it
\footnote{The compiled executable is here: 
\href{http://go.yurichev.com/17165}{beginners.re}}.

It attaches itself to a win32 process by \ac{PID} and just reads process memory at the address.

\subsection{Finding grid automatically}

This is kind of nuisance to set address each time when we run our utility.
Also, various Minesweeper versions may have the array on different address.
Knowing the fact that there is always a border (0x10 bytes), we can just find it in memory:

\lstinputlisting[style=customc]{examples/minesweeper/cheater2_fragment.c}

Full source code: \url{https://github.com/DennisYurichev/RE-for-beginners/blob/master/examples/minesweeper/minesweeper_cheater2.c}.

\subsection{\Exercises}

\begin{itemize}

\item 
Why do the \emph{border bytes} (or \emph{sentinel values}) (0x10) exist in the array?

What they are for if they are not visible in Minesweeper's interface?
How could it work without them?

\item 
As it turns out, there are more values possible (for open blocks, for flagged by user, \etc{}).
Try to find the meaning of each one.

\item 
Modify my utility so it can remove all mines or set them in a fixed pattern that you want in the Minesweeper
process currently running.

\end{itemize}


\subsection{Summary}

Only the lower half of XMM registers is used in all examples here, 
to store number in IEEE 754 format.

Essentially, all instructions prefixed by 
\TT{-SD} (\q{Scalar Double-Precision})---are instructions working with floating point numbers
in IEEE 754 format, stored in the lower 64-bit half of a XMM register.

And it is easier than in the FPU, probably because the SIMD extensions 
were evolved in a less chaotic way than the FPU ones in the past.
The stack register model is not used.

\myindex{x86!\Instructions!ADDSS}
\myindex{x86!\Instructions!MOVSS}
\myindex{x86!\Instructions!COMISS}
% TODO1: do this!
If you would try to replace \Tdouble with \Tfloat

% FIXME1 ... but their -SS versions
in these examples, the same instructions will be used, but prefixed with \TT{-SS} 
(\q{Scalar Single-Precision}), for example, \TT{MOVSS}, \TT{COMISS}, \TT{ADDSS}, etc.

\q{Scalar} 
implies that the SIMD register containing only one value instead of several.

Instructions working with several values in a register simultaneously have \q{Packed} in their name.

Needless to say, the SSE2 instructions work with 64-bit IEEE 754 numbers (\Tdouble),
while the internal representation of the floating-point numbers in FPU is 80-bit numbers.

Hence, the FPU may produce less round-off errors and as a consequence, FPU may give more precise
calculation results.
