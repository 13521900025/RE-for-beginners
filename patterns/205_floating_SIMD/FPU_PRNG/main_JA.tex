\subsection{疑似乱数値の生成例を再訪}
\label{FPU_PRNG_SIMD}

\q{疑似乱数値の生成例}を再訪しましょう:\lstref{FPU_PRNG}

MSVC 2012でコンパイルすると、FPU用にSIMD命令を使用します。

\lstinputlisting[caption=\Optimizing MSVC 2012,style=customasmx86]{patterns/205_floating_SIMD/FPU_PRNG/MSVC2012_Ox_Ob0_JA.asm}

% FIXME1 rewrite!

命令はすべて-SS接尾辞がついています。これは、\q{Scalar Single}を表します。

\q{Scalar} は、1つの値だけがレジスタに格納されることを意味します。

\q{Single}\footnote{すなわち、単精度}は \Tfloat データ型を表します。
