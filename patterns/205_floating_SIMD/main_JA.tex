% FIXME1 divide this file into separate ones...
\mysection{SIMDを使用した浮動小数点数の取り扱い}

\label{floating_SIMD}
\myindex{IEEE 754}
\myindex{SIMD}
\myindex{SSE}
\myindex{SSE2}

もちろん、\ac{SIMD}拡張機能が追加されたとき、\ac{FPU}はx86互換プロセッサに残っていました。

\ac{SIMD}拡張(SSE2)は、浮動小数点数を扱うためのより簡単な方法を提供します。

数値のフォーマットは変わりません(IEEE 754)。

\myindex{x86-64}
そのため、現代のコンパイラ(x86-64用に生成されたものも含む)
は通常、FPUの代わりに\ac{SIMD}命令を使用します。

彼らと一緒に動作する方が簡単なので、それは良いニュースだと言えます。

ここではFPUセクションの例を再利用します:\myref{sec:FPU}

\subsection{単純な例}

\lstinputlisting[style=customc]{patterns/12_FPU/1_simple/simple.c}

\subsubsection{x64}

\lstinputlisting[caption=\Optimizing MSVC 2012 x64,style=customasmx86]{patterns/205_floating_SIMD/simple_MSVC_2012_x64_Ox.asm}

入力浮動小数点値は\XMM{0}-\XMM{3}レジスタに渡され、
残りはすべてスタックを介して渡されます。
\footnote{\href{http://go.yurichev.com/17263}{MSDN: Parameter Passing}}.

$a$ は\XMM{0}に渡され、$b$は\XMM{1}を介して渡されます。

XMMレジスタは128ビットです(SIMDに関するセクションからわかるように:\myref{SIMD_x86})。
しかし \Tdouble 値は64ビットですので、下位半分のレジスタだけが使用されます。

\myindex{x86!\Instructions!DIVSD}
\TT{DIVSD}は\q{Divide Scalar Double-Precision Floating-Point Values}
を表すSSE命令です。
これは、
オペランドの下半分に格納された \Tdouble 型の値を別の値で除算するだけです。

定数は、コンパイラによってIEEE 754形式でエンコードされています。

\myindex{x86!\Instructions!MULSD}
\myindex{x86!\Instructions!ADDSD}
\TT{MULSD}と\TT{ADDSD}はまったく同じように機能しますが、乗算と加算を行います。

関数が \Tdouble 型で実行された結果は、\XMM{0}レジスタに残ります。\\
\\
これが、最適化されていないMSVCの仕組みです。

\lstinputlisting[caption=MSVC 2012 x64,style=customasmx86]{patterns/205_floating_SIMD/simple_MSVC_2012_x64.asm}

\myindex{Shadow space}
少し冗長です。 
入力引数は\q{シャドースペース}(\myref{shadow_space})に保存されますが、
それらの下位レジスタのみが半分になります。つまり、 \Tdouble 型の64ビット値だけです。 
GCCは同じコードを生成します。

\subsubsection{x86}

この例もx86用にコンパイルしましょう。 x86用に生成されているという事実にもかかわらず、MSVC 2012はSSE2命令を使用します。

\lstinputlisting[caption=\NonOptimizing MSVC 2012 x86,style=customasmx86]{patterns/205_floating_SIMD/simple_MSVC_2012_x86.asm}

\lstinputlisting[caption=\Optimizing MSVC 2012 x86,style=customasmx86]{patterns/205_floating_SIMD/simple_MSVC_2012_x86_Ox.asm}

これはほぼ同じコードですが、呼び出し規約に関していくつかの違いがあります。
1)引数はXMMレジスタではなくスタックに渡されます(FPUの例(\myref{sec:FPU})のように)。 
2)関数の結果が\ST{0}に返されます。 そのためには、
XMMレジスタの1つから\ST{0}に(ローカル変数\TT{tv}を介して)コピーします。

\clearpage
最適化された例を \olly で試してみましょう

\begin{figure}[H]
\centering
\myincludegraphics{patterns/205_floating_SIMD/simple_olly1.png}
\caption{\olly: \TT{MOVSD} は $a$ の値を\XMM{1}にロード}
\label{fig:FPU_SIMD_simple_olly1}
\end{figure}

\clearpage
\begin{figure}[H]
\centering
\myincludegraphics{patterns/205_floating_SIMD/simple_olly2.png}
\caption{\olly: \TT{DIVSD} \gls{quotient} を計算し、
結果を\XMM{1}に保存する}
\label{fig:FPU_SIMD_simple_olly2}
\end{figure}

\clearpage
\begin{figure}[H]
\centering
\myincludegraphics{patterns/205_floating_SIMD/simple_olly3.png}
\caption{\olly: \TT{MULSD} calculated \gls{product}を計算し、
\XMM{0}に保存する}
\label{fig:FPU_SIMD_simple_olly3}
\end{figure}

\clearpage
\begin{figure}[H]
\centering
\myincludegraphics{patterns/205_floating_SIMD/simple_olly4.png}
\caption{\olly: \TT{ADDSD} は値を\XMM{0}と\XMM{1}に追加する}
\label{fig:FPU_SIMD_simple_olly4}
\end{figure}

\clearpage
\begin{figure}[H]
\centering
\myincludegraphics{patterns/205_floating_SIMD/simple_olly5.png}
\caption{\olly: \FLD は関数の結果を\ST{0}に残す}
\label{fig:FPU_SIMD_simple_olly5}
\end{figure}

\olly はXMMレジスタを \Tdouble 型の数のペアとして示していますが、
使用されているのはその低位の部分だけです。

SSE2命令(接尾辞\TT{-SD})が現在実行されているため、明らかに、
\olly はそれらをその形式で表示します。

しかし、もちろん、レジスタフォーマットを切り替えて、その内容を4つの \Tfloat{} 数
またはちょうど16バイトとして表示することは可能です。

\clearpage
\subsection{引数を介して浮動小数点数を渡す}

\lstinputlisting[style=customc]{patterns/12_FPU/2_passing_floats/pow.c}

それらは\XMM{0}-\XMM{3}レジスタの
下位半分に渡されます。

\lstinputlisting[caption=\Optimizing MSVC 2012 x64,style=customasmx86]{patterns/205_floating_SIMD/pow_MSVC_2012_x64_Ox.asm}

\myindex{x86!\Instructions!MOVSD}
\myindex{x86!\Instructions!MOVSDX}
IntelおよびAMDのマニュアルには\TT{MOVSDX}命令がなく(\myref{x86_manuals})、単に\TT{MOVSD}と呼ばれています。 
そのため、x86では同じ名前を共有する2つの命令があります(他のものについては \myref{REP_MOVSx} を参照)。
どうやら、Microsoftの開発者たちはこの混乱を取り除きたかったので、\TT{MOVSDX}に改名しました。 
XMMレジスタの下半分に値をロードするだけです。

\TT{pow()}は\XMM{0}と\XMM{1}から引数を取り、結果を\XMM{0}に返します。 
その後 \printf のために \RDX に移動されます。 
なぜでしょうか?
おそらく \
printf が可変引数関数だからでしょう。

\lstinputlisting[caption=\Optimizing GCC 4.4.6 x64,style=customasmx86]{patterns/205_floating_SIMD/pow_GCC446_x64_O3_JA.s}

GCCはより明確な出力を生成します。 
\printf の値は\XMM{0}に渡されます。 
ところで、 \printf のために \EAX に1が書かれている場合は、
標準で要求されているように \SysVABI 、1つの引数がベクトルレジスタに渡されることを意味します。

\subsection{Comparison example}

\lstinputlisting[style=customc]{patterns/12_FPU/3_comparison/d_max.c}

\subsubsection{x64}

\lstinputlisting[caption=\Optimizing MSVC 2012 x64,style=customasmx86]{patterns/205_floating_SIMD/d_max_MSVC_2012_x64_Ox.asm}

\Optimizing MSVCはとても理解しやすいコードを生成します。

\myindex{x86!\Instructions!COMISD}
\TT{COMISD} は \q{Compare Scalar Ordered Double-Precision Floating-Point 
Values and Set EFLAGS}です。 本質的に、命令が示すそのもののことをします。\\
\\
\NonOptimizing MSVC はもっと冗長なコードを生成します。
しかし、これもそんなに理解するのが難しくないです。

\lstinputlisting[caption=MSVC 2012 x64,style=customasmx86]{patterns/205_floating_SIMD/d_max_MSVC_2012_x64.asm}

\myindex{x86!\Instructions!MAXSD}
しかしながら、GCC 4.4.6は
もっと最適化して\TT{MAXSD}(\q{Return Maximum Scalar 
Double-Precision Floating-Point Value})命令を使用します。
これは単に最大値を選択します!

\lstinputlisting[caption=\Optimizing GCC 4.4.6 x64,style=customasmx86]{patterns/205_floating_SIMD/d_max_GCC446_x64_O3.s}

\clearpage
\subsubsection{x86}

この例をMSVC 2012の最適化オプションをONにしてコンパイルしてみましょう。

\lstinputlisting[caption=\Optimizing MSVC 2012 x86,style=customasmx86]{patterns/205_floating_SIMD/d_max_MSVC_2012_x86_Ox.asm}

ほとんど同じですが、 $a$ と $b$ の値はスタックからとられて、関数の結果は
\ST{0}に残ります。

\olly でこの例をロードした場合、
\TT{COMISD}命令が値を比較して \CF と \PF フラグをどうやってセット/クリアするのかみることができます。

\begin{figure}[H]
\centering
\myincludegraphics{patterns/205_floating_SIMD/d_max_olly.png}
\caption{\olly: \TT{COMISD} は \CF と \PF フラグを変更}
\label{fig:FPU_SIMD_d_max_olly}
\end{figure}

\subsection{計算機イプシロンを計算する: x64 と SIMD}
\label{machine_epsilon_x64_and_SIMD}

\q{計算機イプシロンを計算する}の例を \Tdouble 型で再訪しましょう:\lstref{machine_epsilon_double_c}

x64でコンパイルします。

\lstinputlisting[caption=\Optimizing MSVC 2012 x64,style=customasmx86]{patterns/205_floating_SIMD/epsilon_double_MSVC_2012_x64_Ox.asm}

128ビットXMMレジスタに値を1加える方法がないので、メモリ上に配置しなければなりません。

しかしながら、\INS{ADDSD}命令(\IT{Add Scalar Double-Precision Floating-Point Values})があります。
高位64ビットを無視しつつ、これは低位64ビットのXMMレジスタに値を加えることができます。
しかし、MSVC 2012はおそらくそれほど良くないです。
\footnote{練習として、ローカルスタックの使用を取り除く
ためにこのコードを書き直してもいいかもしれません。}

それでも、値はXMMレジスタに再ロードされ、減算が行われます。
\INS{SUBSD}は\q{Subtract Scalar Double-Precision Floating-Point Values}です。
つまり、128ビットXMMレジスタの下位64ビット部に対して動作します。
結果はXMM0レジスタに返されます。

\subsection{疑似乱数値の生成例を再訪}
\label{FPU_PRNG_SIMD}

\q{疑似乱数値の生成例}を再訪しましょう:\lstref{FPU_PRNG}

MSVC 2012でコンパイルすると、FPU用にSIMD命令を使用します。

\lstinputlisting[caption=\Optimizing MSVC 2012,style=customasmx86]{patterns/205_floating_SIMD/FPU_PRNG/MSVC2012_Ox_Ob0_JA.asm}

% FIXME1 rewrite!

命令はすべて-SS接尾辞がついています。これは、\q{Scalar Single}を表します。

\q{Scalar} は、1つの値だけがレジスタに格納されることを意味します。

\q{Single}\footnote{すなわち、単精度}は \Tfloat データ型を表します。


\subsection{概要}

ここに示すすべての例では、数値をIEEE 754形式で格納するために、
XMMレジスタの下半分だけが使用されています。

本質的に、\TT{-SD}(\q{Scalar Double-Precision})
の接頭語がついた命令はすべて、IEEE 754形式の浮動小数点数として動作します。
そして、XMMレジスタの下位64ビット半分に格納されます。

そしてそれは、おそらくSIMD拡張が過去のFPU拡張よりも混沌としていない
方法で進化したために、FPUよりも簡単です。
スタックレジスタモデルは使用されません。

\myindex{x86!\Instructions!ADDSS}
\myindex{x86!\Instructions!MOVSS}
\myindex{x86!\Instructions!COMISS}
% TODO1: do this!
\Tdouble を \Tfloat に置き換えようとした場合

% FIXME1 ... but their -SS versions
これらの例では、同じ命令が使用されますが、\TT{-SS}(\q{Scalar Single-Precision})接頭語が
つきます。例えば、\TT{MOVSS}、\TT{COMISS}、\TT{ADDSS}などです。

\q{Scalar} 
は、SIMDレジスタに複数の値ではなく1つの値しか含まれていないことを意味します。

レジスタ内の複数の値を同時に扱う命令は、それらの名前に\q{パック}されています。

言うまでもなく、SSE2命令は64ビットのIEEE 754形式の数( \Tdouble )で機能しますが、
FPUの浮動小数点数の内部表現は80ビットの数値です。

したがって、FPUは丸め誤差を少なくすることができ、その結果、FPUはより正確な計算結果を
得られます。
