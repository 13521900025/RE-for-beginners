\mysection{Prologo ed epilogo delle funzioni}
\label{sec:prologepilog}
\myindex{Function epilogue}
\myindex{Function prologue}

Il prologo (o preambolo) di una funzione e' una sequenza di istruzioni all'inizio della funzione stessa.
Spesso ha una forma simile al seguente frammento di codice:

\begin{lstlisting}[style=customasmx86]
    push    ebp
    mov     ebp, esp
    sub     esp, X
\end{lstlisting}

Cosa fanno queste istruzioni: salvano il valore del registro \EBP,
impostano il valore del registro \EBP con il valore di \ESP e allocano spazio sullo stack per le variabili locali.

Il valore di \EBP resta costante durante il periodo di esecuzione della funzione, ed e' usato per accede a variabili locali e argomenti.
Per lo stesso scopo si puo' usare \ESP, ma siccome cambia nel tempo non e' un approccio molto conveniente.

L'epilogo della funzione libera lo spazio allocato nello stack, ripristina il valore nel registro \EBP al suo stato iniziale e restituisce
il controllo al \glslink{caller}{chiamante}:

\begin{lstlisting}[style=customasmx86]
    mov    esp, ebp
    pop    ebp
    ret    0
\end{lstlisting}

% what about calling convention?
Prologo ed epilogo di funzioni sono solitamente identificati nei disassemblatori per delimitare le funzioni.

\subsection{\Recursion}

\myindex{\Recursion}
Epiloghi e prologhi possono avere un effetto negativo sulla performance in caso di ricorsione.

Maggiori informazioni sulla ricorsione in questo libro: \myref{Recursion_and_tail_call}.
