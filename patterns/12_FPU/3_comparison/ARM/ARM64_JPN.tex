\subsubsection{ARM64}

\myparagraph{\Optimizing GCC (Linaro) 4.9}

\lstinputlisting[style=customasmARM]{patterns/12_FPU/3_comparison/ARM/ARM64_GCC_O3_JPN.lst}

ARM64 \ac{ISA}には、便宜上、
\ac{FPSCR}の代わりにCPUフラグを\ac{APSR}に設定するFPU命令があります。
\ac{FPU}はもはや別個のデバイスではありません。(少なくとも論理的には)
\myindex{ARM!\Instructions!FCMPE}
ここでは\INS{FCMPE}を参照してください。 \RegD{0}と\RegD{1}(関数の第1引数と第2引数)で渡された2つの値を比較し、
\ac{APSR}フラグ(N, Z, C, V)を設定します。

\myindex{ARM!\Instructions!FCSEL}
\INS{FCSEL}(\emph{Floating Conditional Select})は、条件(\GTT{GT}---Greater Than)に応じて\RegD{0}または\RegD{1}の値を\RegD{0}にコピーし、
再び\ac{FPSCR}の代わりに\ac{APSR}レジスタのフラグを使用します。

これは、古いCPUの命令セットに比べてはるかに便利です。

条件が真(\GTT{GT})の場合、\RegD{0}の値が
\RegD{0}にコピーされます。(つまり何も起こりません)
条件が真でない場合、\RegD{1}の値が\RegD{0}にコピーされます。

\myparagraph{\NonOptimizing GCC (Linaro) 4.9}

\lstinputlisting[style=customasmARM]{patterns/12_FPU/3_comparison/ARM/ARM64_GCC_JPN.lst}

非最適化GCCはより冗長です。

まず、関数は入力引数の値をローカルスタック(\emph{Register Save Area})に保存します。
次に、これらの値をレジスタ
\RegD{0}/\RegD{1}にリロードし、
最終的に\RegX{0}/\RegX{1}にコピーして\INS{FCMPE}を使用して比較します。
冗長なコードがたくさんありますが、
最適化されていないコンパイラの仕組みです。 
\INS{FCMPE}は値を比較し、\ac{APSR}フラグを設定します。
現時点では、コンパイラは、より便利な\INS{FCSEL}命令についてはまだ考えていないため、古いメソッドを使用して処理を進めます。
つまり、\INS{BLE}命令を使用します(\emph{Branch if Less than or Equal})。
最初のケース($a>b$)では、 $a$ の値が\RegX{0}にロードされます。
それ以外の場合($a<=b$)、 $b$ の値は\RegX{0}にロードされます。
最後に、戻り値がこのレジスタにある必要があるため、
\RegX{0}からの値が\RegD{0}にコピーされます。

\mysubparagraph{\Exercise}

練習として、冗長な命令を削除し、
新しい命令(\INS{FCSEL}を含む)を導入しないで手動でこのコードを最適化することができます。

\myparagraph{\Optimizing GCC (Linaro) 4.9---float}

\Tdouble の代わりに \Tfloat を使うようにこの例を書き直しましょう。

\begin{lstlisting}[style=customc]
float f_max (float a, float b)
{
	if (a>b)
		return a;

	return b;
};
\end{lstlisting}

\lstinputlisting[style=customasmARM]{patterns/12_FPU/3_comparison/ARM/ARM64_GCC_O3_float_JPN.lst}

これは同じコードですが、D-レジスタの代わりにS-レジスタが使用されています。 
これは、浮動小数点数が32ビットSレジスタ(実際には64ビットDレジスタの下位部分)に渡されるためです。
