\myparagraph{GCC 4.4.1}

\lstinputlisting[caption=GCC 4.4.1,style=customasmx86]{patterns/12_FPU/3_comparison/x86/GCC_JPN.asm}

\myindex{x86!\Instructions!FUCOMPP}

\FUCOMPP{} は \FCOM に似ていますが、スタックから両方の値をポップし、
\q{非数値} を異なる方法で処理します。

\myindex{Non-a-numbers (NaNs)}
\q{非数値} について少々。

\newcommand{\NANFN}{\footnote{\href{http://go.yurichev.com/17130}{wikipedia.org/wiki/NaN}}}

FPUは、数字でない特別な\emph{非数値} や \gls{NaN}\NANFN を扱うことができます。 
これらは無限大で、0で除算した結果です。
非数値は\q{寡黙}であることも\q{シグナルを発する}こともできます。\q{寡黙な} NaNで
作業を続行することは可能ですが、\q{シグナルを発する} NaNで何らかの操作を試みる場合は例外が発生します。

\myindex{x86!\Instructions!FCOM}
\myindex{x86!\Instructions!FUCOM}

オペランドが\gls{NaN}の場合、 \FCOM は例外を送出します。 
\FUCOM は、オペランドがシグナルを発する \gls{NaN}(SNaN)である場合にのみ例外を送出します。

\myindex{x86!\Instructions!SAHF}
\label{SAHF}

次の命令は \SAHF (\emph{AHをフラグにストア})です。これはFPUに
関連しないコードではまれな命令です。
AHからの8ビットは、次の順序でCPUフラグの下位8ビットに移動します。

\input{SAHF_LAHF}

\myindex{x86!\Instructions!FNSTSW}

\FNSTSW が関心のあるビット(\CThreeBits)を \AH に移動し、
\AH レジスタの位置6,2,0にあることを思い出してください。

\input{C3_in_AH}

言い換えると、\INS{fnstsw  ax / sahf}命令ペアは、 \CThreeBits を \ZF 、 \PF 、および \CF に移動します。

異なる条件で \CThreeBits の値を思い出してみましょう。

\begin{itemize}
\item この例で $a$ が $b$ より大きい場合、 \CThreeBits は0,0,0に設定されます。
\item $a$ が $b$ より小さければ、ビットは0,0,1に設定されます。
\item $a=b$ の場合は、1,0,0に設定されます。
\end{itemize}
% TODO: table?

言い換えれば、これらのCPUフラグの状態は、
3つの \FUCOMPP/\FNSTSW/\SAHF 命令の後に可能になります。

\begin{itemize}
\item $a>b$ の場合、CPUフラグは、\GTT{ZF=0, PF=0, CF=0}として設定されます。
\item $a<b$ の場合、フラグは、\GTT{ZF=0, PF=0, CF=1}として設定されます。
\item そして、 $a=b$ ならば、\GTT{ZF=1, PF=0, CF=0}になります。
\end{itemize}
% TODO: table?

\myindex{x86!\Instructions!SETcc}
\myindex{x86!\Instructions!JNBE}

CPUのフラグと条件に応じて、 \SETNBE はALに1または0を格納します。 
これはほぼ \JNBE のものですが、 \SETcc
\footnote{\emph{cc} は \emph{条件コード} です}
はALに1または0を格納しますが、 \Jcc は実際にジャンプするかどうかは異なります。
\SETNBE は、\GTT{CF=0}および\GTT{ZF=0}の場合にのみ1を格納します。 
真でない場合は、0が \AL に格納されます。

$a>b$ の場合でのみ、\CF と \ZF の両方が0になります。

その後、1が \AL に格納され、後続の \JZ は実行されず、関数は{\_a}を返します。 
それ以外の場合は{\_b}が返されます。
