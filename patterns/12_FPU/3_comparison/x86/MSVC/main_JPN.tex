\myparagraph{\NonOptimizing MSVC}

MSVC 2010は以下のコードを生成します。

\lstinputlisting[caption=\NonOptimizing MSVC 2010,style=customasmx86]{patterns/12_FPU/3_comparison/x86/MSVC/MSVC_JPN.asm}

\myindex{x86!\Instructions!FLD}

\FLD は\GTT{\_b}を\ST{0}にロードします。

\label{Czero_etc}
\newcommand{\Czero}{\GTT{C0}\xspace}
\newcommand{\Ctwo}{\GTT{C2}\xspace}
\newcommand{\Cthree}{\GTT{C3}\xspace}
\newcommand{\CThreeBits}{\Cthree/\Ctwo/\Czero}

\myindex{x86!\Instructions!FCOMP}

\FCOMP は\ST{0}の値と\GTT{\_a}の値を比較し、
それに応じてFPUステータスワードレジスタの \CThreeBits ビットを設定します。
これは、FPUの現在の状態を反映する16ビットのレジスタです。

ビットがセットされると、 \FCOMP 命令はスタックから1つの変数もポップします。
これは、値を比較してスタックを同じ状態にしておく \FCOM とは区別されます。

残念ながら、インテルP6 
\footnote{インテルP6はPentium Pro、Pentium IIなどです。}
より前のCPUには、 \CThreeBits ビットをチェックする条件付きジャンプ命令はありません。
おそらく、それは歴史の問題です。(思い起こしてみてください:FPUは過去に別のチップでした)

インテルP6で始まる最新のCPUは、\FCOMI/\FCOMIP/\FUCOMI/\FUCOMIP 命令を持っていて、
同じことをしますが、 \ZF/\PF/\CF CPUフラグを変更します。

\myindex{x86!\Instructions!FNSTSW}

\FNSTSW 命令は状態レジスタであるFPUを \AX にコピーします。 
\CThreeBits ビットは14/10/8の位置に配置され、
\AX レジスタの同じ位置にあり、 \AX{}~---\AH{} の上位部分に配置されます。

\begin{itemize}
\item この例では $b>a$ の場合、 \CThreeBits ビットは0,0,0と設定します。
\item $a>b$ の場合、ビットは0,0,1です。
\item $a=b$ の場合、ビットは1,0,0です。
\item 結果が順序付けられていない場合(エラーの場合)、セットされたビットは1,1,1,1です。
\end{itemize}
% TODO: table here?

これは、 \CThreeBits ビットが \AX レジスタにどのように配置されるかを示しています。

\input{C3_in_AX}

これは、 \CThreeBits ビットが \AH レジスタにどのように配置されるかを示しています。

\input{C3_in_AH}

\INS{test ah, 5}\footnote{5=101b}の実行後、
\Czero と \Ctwo ビット(0と2の位置)のみが考慮され、他のビットはすべて無視されます。

\label{parity_flag}
\myindex{x86!\Registers!\Flags!Parity flag}

さて、\IT{パリティーフラグ}と注目すべきもう1つの歴史的基礎についてお話しましょう。

このフラグは、最後の計算結果の1の数が偶数の場合は1に設定され、奇数の場合は0に設定されます。

Wikipedia\footnote{\href{http://go.yurichev.com/17131}{wikipedia.org/wiki/Parity\_flag}}を見てみましょう:

\begin{framed}
\begin{quotation}
パリティフラグをテストする一般的な理由の1つに、無関係なFPUフラグをチェックすることがあります。 FPUには4つの条件フラグ
(C0~C3)がありますが、直接テストすることはできず、最初にフラグレジスタにコピーする必要があります。 
これが起こると、C0はキャリーフラグに、C2はパリティフラグに、C3はゼロフラグに置かれます。 
C2フラグは、例えば比較できない浮動小数点値(NaNまたはサポートされていない形式)がFUCOM命令と比較されます。
\end{quotation}
\end{framed}

Wikipediaで述べられているように、パリティフラグはFPUコードで使用されることがあります。

\myindex{x86!\Instructions!JP}

\Czero と \Ctwo の両方が0に設定されている場合、 \PF フラグは1に設定されます。その場合、
後続の \JP (\IT{jump if PF==1})が実行されます。 
いろいろな場合の \CThreeBits の値を思い出すと、
条件ジャンプ \JP は、 $b>a$ または $a=b$ の場合に実行されます。
(\INS{test ah, 5}命令によってクリアされているので、 \Cthree ビットはここでは考慮されていません)

それ以降はすべて簡単です。 
条件付きジャンプが実行された場合、
\FLD は\ST{0}の\GTT{\_b}の値をロードし、
実行されていなければ\GTT{\_a}の値をロードします。

\mysubparagraph{\Ctwo? のチェックは?}

\Ctwo フラグはエラー(\gls{NaN}など)の場合に設定されますが、コードではチェックされません。

プログラマがFPUエラーを気にする場合は、チェックを追加する必要があります。

\clearpage
\myparagraph{\Optimizing MSVC + \olly}
\myindex{\olly}

\olly でこの(最適化された)例を試すことができます。ここに最初のイテレーションがあります:

\begin{figure}[H]
\centering
\myincludegraphics{patterns/10_strings/1_strlen/olly1.png}
\caption{\olly: 最初のイテレーションの開始}
\label{fig:strlen_olly_1}
\end{figure}

\olly がループを見つけ、便宜上、その指示を括弧で\emph{囲んだ}ことがわかります。 
\EAX の右ボタンをクリックして、\q{Follow in Dump}を選択すると、メモリウィンドウが正しい場所にスクロールします。
ここではメモリ内に \q{hello!}という文字列があります。
その後に少なくとも1つのゼロバイトがあり、次にランダムなごみがあります。

\olly が有効なアドレスを持つレジスタを見ると、それは文字列を指しており、
文字列として表示されます。

\clearpage
F8(\stepover)を数回押して、ループの本体の先頭に移動しましょう:

\begin{figure}[H]
\centering
\myincludegraphics{patterns/10_strings/1_strlen/olly2.png}
\caption{\olly: 2回目のイテレーションの開始}
\label{fig:strlen_olly_2}
\end{figure}

\EAX には文字列中の2番目の文字のアドレスが含まれていることがわかります。

\clearpage

我々はループから脱出するためにF8を十分な回数押す必要があります:

\begin{figure}[H]
\centering
\myincludegraphics{patterns/10_strings/1_strlen/olly3.png}
\caption{\olly: 計算すべきポインタの差}
\label{fig:strlen_olly_3}
\end{figure}

% FIXME:
\EAX には文字列の直後に0バイトのアドレスが含まれることがわかりました。一方、
\EDX は変更されていないので、まだ文字列の先頭を指しています。

これらの2つのアドレスの差がここで計算されています。

\clearpage
\SUB 命令が実行されました。

\begin{figure}[H]
\centering
\myincludegraphics{patterns/10_strings/1_strlen/olly4.png}
\caption{\olly: \EAX がデクリメントされる}
\label{fig:strlen_olly_4}
\end{figure}

ポインタの違いは \EAX レジスタにあります(値は7)。
確かに、 \q{hello!}文字列の長さは6ですが、7にはゼロバイトが含まれています。
しかし、\TT{strlen()}は文字列中のゼロ以外の文字数を返さなければなりません。
したがって、デクリメントが実行され、関数が戻ります。

