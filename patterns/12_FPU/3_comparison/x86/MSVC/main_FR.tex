\myparagraph{MSVC \NonOptimizing}

MSVC 2010 génère ce qui suit:

\lstinputlisting[caption=MSVC 2010 \NonOptimizing,style=customasmx86]{patterns/12_FPU/3_comparison/x86/MSVC/MSVC_FR.asm}

\myindex{x86!\Instructions!FLD}

Ainsi, \FLD charge \GTT{\_b} dans \ST{0}.

\label{Czero_etc}
\newcommand{\Czero}{\GTT{C0}\xspace}
\newcommand{\Ctwo}{\GTT{C2}\xspace}
\newcommand{\Cthree}{\GTT{C3}\xspace}
\newcommand{\CThreeBits}{\Cthree/\Ctwo/\Czero}

\myindex{x86!\Instructions!FCOMP}

\FCOMP compare la valeur dans \ST{0} avec ce qui est dans \GTT{\_a} et met les bits
\CThreeBits du mot registre d'état du FPU, suivant le résultat.
Ceci est un registre 16-bit qui reflète l'état courant du FPU.

Après que les bits ont été mis, l'instruction \FCOMP dépile une variable depuis la
pile.
C'est ce qui la différencie de \FCOM, qui compare juste les valeurs, laissant la
pile dans le même état.

Malheureusement, les CPUs avant les Intel P6\footnote{Intel P6 comprend les Pentium
Pro, Pentium II, etc.} ne possèdent aucune instruction de saut conditionnel qui teste
les bits \CThreeBits.
Peut-être est-ce une raison historique (rappel: le FPU était une puce séparée dans
le passé).\\Les CPU modernes, à partir des Intel P6 possèdent les instructions \FCOMI/\FCOMIP/\FUCOMI/\FUCOMIP~---qui
font la même chose, mais modifient les flags \ZF/\PF/\CF du CPU.

\myindex{x86!\Instructions!FNSTSW}

L'instruction \FNSTSW copie le le mot du registre d'état du FPU dans \AX.
Les bits \CThreeBits sont placés aux positions 14/10/8, ils sont à la même position
dans le registre \AX et tous sont placés dans la partie haute de \AX{}~---\AH{}.

\begin{itemize}
\item Si $b>a$ dans notre exemple, alors les bits \CThreeBits sont mis comme ceci: 0, 0, 0.
\item Si $a>b$, alors les bits sont: 0, 0, 1.
\item Si $a=b$, alors les bits sont: 1, 0, 0.

Si le résultat n'est pas ordonné (en cas d'erreur), alors les bits sont: 1, 1, 1.
\end{itemize}
% TODO: table here?

Voici comment les bits \CThreeBits sont situés dans le registre \AX:

\input{C3_in_AX}

Voici comment les bits \CThreeBits sont situés dans le registre \AH:

\input{C3_in_AH}

Après l'exécution de \INS{test ah, 5}\footnote{5=101b}, seul les bits \Czero et \Ctwo
(en position 0 et 2) sont considérés, tous les autres bits sont simplement ignorés.

\label{parity_flag}
\myindex{x86!\Registers!\Flags!Parity flag}

Parlons maintenant du \emph{parity flag} (flag de parité), un autre rudiment historique
remarquable.

Ce flag est mis à 1 si le nombre de un dans le résultat du dernier calcul est pair,
et à 0 s'il est impair.

Regardons sur Wikipédia\footnote{\href{http://go.yurichev.com/17131}{wikipedia.org/wiki/Parity\_flag}}:

\begin{framed}
\begin{quotation}
Une raison commune de tester le bit de parité n'a rien à voir avec la parité. Le FPU
possède quatre flags de condition (C0 à C3), mais ils ne peuvent pas être testés
directement, et doivent d'abord être copiés dans le registre d'états.
Lorsque ça se produit, C0 est mis dans le flag de retenue, C2 dans le flag
de parité et C3 dans le flag de zéro.
Le flag C2 est mis lorsque e.g. des valeurs en virgule flottantes incomparable
(NaN ou format non supporté) sont comparées avec l'instruction \FUCOM.
\end{quotation}
\end{framed}

Comme indiqué dans Wikipédia, le flag de parité est parfois utilisé dans du code
FPU, voyons comment.

\myindex{x86!\Instructions!JP}

Le flag \PF est mis à 1 si à la fois \Czero et \Ctwo sont mis à 0 ou si les deux
sont à 1, auquel cas le \JP (\emph{jump if PF==1}) subséquent est déclenché.
Si l'on se rappelle les valeurs de \CThreeBits pour différents cas, nous pouvons
voir que le saut conditionnel \JP est déclenché dans deux cas: si $b>a$ ou $a=b$
(le bit \Cthree n'est pris en considération ici, puisqu'il a été mis à 0 par l'instruction
\INS{test ah, 5}).

C'est très simple ensuite.
Si le saut conditionnel a été déclenché, \FLD charge la valeur de \GTT{\_b} dans
\ST{0}, et sinon, la valeur de \GTT{\_a} est chargée ici.

\mysubparagraph{Et à propos du test de \Ctwo?}

Le flag \Ctwo est mis en cas d'erreur (\gls{NaN}, etc.), mais notre code ne le teste
pas.

Si le programmeur veut prendre en compte les erreurs FPU, il doit ajouter des tests
supplémentaires.

\clearpage
\myparagraph{MSVC + \olly}
\myindex{\olly}

2 paires de mots 32-bit sont marquées en rouge sur la pile.
Chaque paire est un double au format IEEE 754 et est passée depuis \main.

Nous voyons comment le premier \FLD charge une valeur ($1.2$) depuis la pile et la
stocke dans \ST{0}:

\begin{figure}[H]
\centering
\myincludegraphics{patterns/12_FPU/1_simple/olly1.png}
\caption{\olly: le premier \FLD a été exécuté}
\label{fig:FPU_simple_olly_1}
\end{figure}

À cause des inévitables erreurs de conversion depuis un nombre flottant 64-bit au
format IEEE 754 vers 80-bit (utilisé en interne par le FPU), ici nous voyons 1.199\ldots,
qui est proche de 1.2.

\EIP pointe maintenant sur l'instruction suivante (\FDIV), qui charge un double
(une constante) depuis la mémoire.
Par commodité, \olly affiche sa valeur: 3.14

\clearpage
Continuons l'exécution pas à pas.
\FDIV a été exécuté, maintenant \ST{0} contient 0.382\ldots
(\gls{quotient}):

\begin{figure}[H]
\centering
\myincludegraphics{patterns/12_FPU/1_simple/olly2.png}
\caption{\olly: \FDIV a été exécuté}
\label{fig:FPU_simple_olly_2}
\end{figure}

\clearpage
Troisième étape: le \FLD suivant a été exécuté, chargeant 3.4 dans \ST{0} (ici nous
voyons la valeur approximative 3.39999\ldots):

\begin{figure}[H]
\centering
\myincludegraphics{patterns/12_FPU/1_simple/olly3.png}
\caption{\olly: le second \FLD a été exécuté}
\label{fig:FPU_simple_olly_3}
\end{figure}

En même temps, le \gls{quotient} \emph{est poussé} dans \ST{1}.
Exactement maintenant, \EIP pointe sur la prochaine instruction: \FMUL.
Ceci charge la constante 4.1 depuis la mémoire, ce que montre \olly.

\clearpage
Suivante: \FMUL a été exécutée, donc maintenant le \glslink{product}{produit} est dans \ST{0}:

\begin{figure}[H]
\centering
\myincludegraphics{patterns/12_FPU/1_simple/olly4.png}
\caption{\olly: \FMUL a été exécuté}
\label{fig:FPU_simple_olly_4}
\end{figure}

\clearpage
Suivante: \FADDP a été exécutée, maintenant le résultat de l'addition est dans \ST{0},
et \ST{1} est vidé.

\begin{figure}[H]
\centering
\myincludegraphics{patterns/12_FPU/1_simple/olly5.png}
\caption{\olly: \FADDP a été exécuté}
\label{fig:FPU_simple_olly_5}
\end{figure}

Le résultat est laissé dans \ST{0}, car la fonction renvoie son résultat dans \ST{0}.

\main prend cette valeur depuis le registre plus loin.

Nous voyons quelque chose d'inhabituel: la valeur 13.93\ldots se trouve maintenant
dans \ST{7}.
Pourquoi?

\label{FPU_is_rather_circular_buffer}

Comme nous l'avons lu il y a quelque temps dans ce livre, les registres \ac{FPU} sont
une pile: \myref{FPU_is_stack}.
Mais ceci est une simplification.

Imaginez si cela était implémenté \emph{en hardware} comme cela est décrit, alors
tout le contenu des 7 registres devrait être déplacé (ou copié) dans les registres
adjacents lors d'un push ou d'un pop, et ceci nécessite beaucoup de travail.

En réalité, le \ac{FPU} a seulement 8 registres et un pointeur (appelé \GTT{TOP})
qui contient un numéro de registre, qui est le \q{haut de la pile} courant.

Lorsqu'une valeur est poussée sur la pile, \GTT{TOP} est déplacé sur le registre
disponible suivant, et une valeur est écrite dans ce registre.

La procédure est inversée si la valeur est lue, toutefois, le registre qui a été
libéré n'est pas vidé (il serait possible de le vider, mais ceci nécessite plus de
travail qui peut dégrader les performances).
Donc, c'est ce que nous voyons ici.

On peut dire que \FADDP sauve la somme sur la pile, et y supprime un élément.

Mais en fait, cette instruction sauve la somme et ensuite décale \GTT{TOP}.

Plus précisément, les registres du  \ac{FPU} sont un tampon circulaire.

