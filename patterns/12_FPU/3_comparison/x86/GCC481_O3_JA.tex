\myparagraph{GCC 4.8.1 with \Othree optimization turned on}
\label{gcc481_o3}

いくつかの新しいFPU命令がP6インテルファミリ\footnote{Pentium Pro、Pentium-IIなどに始まる}に追加されました。 
\myindex{x86!\Instructions!FUCOMI}
これらは \INS{FUCOMI} (メインCPUのオペランドとフラグの比較)と
\myindex{x86!\Instructions!FCMOVcc}
\INS{FCMOVcc} (FPOレジスタ上の\INS{CMOVcc}のように機能します)です。

どうやら、GCCのメンテナは、P6以前のインテルCPU(初期のPentium、80486など)のサポートを中止することに決めました。

また、FPUはP6インテルファミリではもはや別個のユニットではなくなったので、FPUからメインCPUのフラグを変更/チェックすることが可能になりました。

つまり私たちが得るものは次のとおりです。

\lstinputlisting[caption=\Optimizing GCC 4.8.1,style=customasmx86]{patterns/12_FPU/3_comparison/x86/GCC481_O3_JA.s}

\INS{FXCH} (スワップオペランド)がどうしてここにあるのかを推測するのは難しいです。

最初の2つの \FLD 命令を交換するか、 \INS{FCMOVBE} (\emph{below or equal})を \INS{FCMOVA} (\emph{above})に
置き換えることで、簡単に取り除くことができます。 
おそらく、それはコンパイラが不正確なためです。

そのため、 \INS{FUCOMI} は \ST{0} ( $a$ )と \ST{1}( $b$ )を比較し、
メインCPUにいくつかのフラグを設定します。 
\INS{FCMOVBE}はフラグをチェックし、 $ST0 (a) <= ST1 (b)$ なら
\ST{1}(ここでは $b$ )を\ST{0}(ここでは $a$ )にコピーします。
そうでなければ( $a>b$ )、\ST{0}に $a$ を残します。

最後の \FSTP は、\ST{1}の内容を破棄してスタックの上に\ST{0}を残します。

GDBでこの関数をトレースしましょう:

\lstinputlisting[caption=\Optimizing GCC 4.8.1 and GDB,numbers=left]{patterns/12_FPU/3_comparison/x86/gdb.txt}

\q{ni}を使って、
最初の \FLD 命令を2つ実行してみましょう。

FPUレジスタを確認してみましょう。(33行目)

以前書いたように、FPUレジスタのセットはスタックではなく循環バッファです。(\myref{FPU_is_rather_circular_buffer})
そしてGDBは\GTT{STx}レジスタを表示しませんが、FPUレジスタの内部を表示します。(\GTT{Rx})
(35行目の)矢印は現在のスタックのトップを示しています。

\emph{Status Word}(44行目)に\GTT{TOP}レジスタの内容を見ることができます。今は6で、
スタックのトップは内部レジスタ6を示しています。

$a$ および $b$ の値は\INS{FXCH}が実行されると交換されます。(54行目)

\INS{FUCOMI}は実行されます。(83行目)
フラグを見てみましょう: \CF がセットされます。(95行目)

\INS{FCMOVBE} は $b$ の値をコピーします。(104行目)

\FSTP はスタックのトップの値を1つ残します。(136行目)
\GTT{TOP}の値は7で、FPUスタックのトップは内部レジスタ7を示しています。
