\myparagraph{GCC 4.8.1 avec l'option d'optimisation \Othree}
\label{gcc481_o3}

De nouvelles instructions FPU ont été ajoutées avec la famille Intel P6\footnote{À partir du Pentium Pro, Pentium-II, etc.}.
\myindex{x86!\Instructions!FUCOMI}
Ce sont \INS{FUCOMI} (comparer les opérandes et positionner les flags du CPU principal)
et \myindex{x86!\Instructions!FCMOVcc}
\INS{FCMOVcc} (fonctionne comme \INS{CMOVcc}, mais avec les registres du FPU).

Apparemment, les mainteneurs de GCC ont décidé de supprimer le support des CPUs Intel
pré-P6 (premier Pentium, 80486, etc.).

Et donc, le FPU n'est plus une unité séparée dans la famille Intel P6, ainsi il est
possible de modifier/vérifier un flag du CPU principal depuis le FPU.

Voici ce que nous obtenons:

\lstinputlisting[caption=GCC 4.8.1 \Optimizing,style=customasmx86]{patterns/12_FPU/3_comparison/x86/GCC481_O3_FR.s}

Difficile de deviner pourquoi \INS{FXCH} (échange les opérandes) est ici.

Il est possible de s'en débarrasser facilement en échangeant les deux premières instructions
\FLD ou en remplaçant \INS{FCMOVBE} (\emph{below or equal} inférieur ou égal) par
\INS{FCMOVA} (\emph{above}).
Il s'agit probablement d'une imprécision du compilateur.

Donc \INS{FUCOMI} compare \ST{0} ($a$) et \ST{1} ($b$) et met certains flags
dans le CPU principal.
\INS{FCMOVBE} vérifie les flags et copie \ST{1} ($b$ ici à ce moment) dans \ST{0}
($a$ ici) si $ST0 (a) <= ST1 (b)$.
Autrement ($a>b$), $a$ est laissé dans \ST{0}.

Le dernier \FSTP laisse \ST{0} sur le sommet de la pile, supprimant le contenu de \ST{1}.

Exécutons pas à pas cette fonction dans GDB:

\lstinputlisting[caption=GCC 4.8.1 \Optimizing and GDB,numbers=left]{patterns/12_FPU/3_comparison/x86/gdb.txt}

En utilisant \q{ni}, exécutons les deux premières instructions \FLD.

Examinons les registres du FPU (ligne 33).

Comme cela a déjà été mentionné, l'ensemble des registres FPU est un buffeur
circulaire plutôt qu'une pile (\myref{FPU_is_rather_circular_buffer}).
Et GDB ne montre pas les registres \GTT{STx}, mais les registre internes du FPU (\GTT{Rx}).
La flèche (à la ligne 35) pointe sur le haut courant de la pile.

Vous pouvez voir le contenu du registre \GTT{TOP} dans le \emph{Status Word} (ligne 44)---c'est
6 maintenant, donc le haut de la pile pointe maintenant sur le registre interne 6.

Les valeurs de $a$ et $b$ sont échangées après l'exécution de \INS{FXCH} (ligne 54).

\INS{FUCOMI} est exécuté (ilgne 83).
Regardons les flags: \CF est mis (ligne 95).

\INS{FCMOVBE} a copié la valeur de $b$ (voir ligne 104).

\FSTP dépose une valeur au sommet de la pile (ligne 136).
La valeur de \GTT{TOP} est maintenant 7, donc le sommet de la pile du FPU pointe
sur le registre interne 7.

