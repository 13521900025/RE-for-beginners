%TODO
\subsubsection{ARM: \OptimizingXcodeIV (\ARMMode)}
Bis die Unterstützung für Fließkommaarithmetik in ARM standardisiert wurde,
fügten einige Hersteller von Prozessoren ihre eigenen Befehlserweiterungen
hinzu. 
Schließlich wurde VFP (\emph{Vector Floating Point}) standardisiert.

Ein wichtiger Unterschied zum x86 ist, dass in es in ARM keinen Stack gibt,
sondern man nur mit den Registern arbeitet.

\lstinputlisting[label=ARM_leaf_example10,caption=\OptimizingXcodeIV (\ARMMode),style=customasmARM]{patterns/12_FPU/1_simple/ARM/Xcode_ARM_O3_DE.asm}

\myindex{ARM!D-\registers{}}
\myindex{ARM!S-\registers{}}

Hier sehen wir, dass einige neue Register mit einem D als Präfix verwendet
werden.

Bei diesen handelt es sich um 64-bit-Register; es gibt 32 von ihnen und sie
können sowohl für Fließkommazahlen (doppelte Genauigkeit (double)) als auch für
SIMD (heißt hier in ARM NEON) benutzt werden.

Es gibt also 32 32-bit-S-Register vorgesehen für Fließkommazahlen in einfacher
Genauigkeit (float).

Es ist leicht zu merken: D-Register sind für Zahlen in doppelter Genauigkeit,
während S-Register für einfache Genauigkeit (engl. single) vorgesehen sind.
Mehr dazu hier:\myref{ARM_VFP_registers}

Beide Konstanten (3.14 und 4.1) werden im IEEE 754 Format im Speicher abgelegt.

\myindex{ARM!\Instructions!VLDR}
\myindex{ARM!\Instructions!VMOV}
Wie man leicht sieht sind \INS{VLDR} und \INS{VMOV} analog zu den \INS{LDR} und
\MOV Befehlen, aber arbeiten auf D-Registern.

Es muss angemerkt werden, dass diese Befehle genau wie die D-Register nicht nur
für Fließkommazahlen vorgesehen sind, sondern ebenfalls für SIMD (NEON)
Operationen verwendet werden können, was wir im folgenden zeigen werden.

Die Paraemter werden der Funktion auf übliche Weise über die R-Register
übergeben, aber da jede Zahl in doppelter Genauigkeit eine Größe von 64 Bit hat
werden jeweils zwei R-Register benötigt, um eine Zahl zu übergeben.

Der Befehl \INS{VMOV D17, R0, R1} zu Beginn, fasst zwei 32-Bit-Werte aus
\Reg{0} und \Reg{1} zu einem 64-Bit-Wert zusammen und speichert diesen in
\GTT{D17}.

\INS{VMOV R0, R1, D16} ist die umgekehrte Operation: was vorher in \GTT{D16}
war, wird in zwei Register, \Reg{0} und \Reg{1} aufgeteilt, denn eine Zahl in
doppelter Genauigkeit, die 64 Bit Speicherplatz benötigt, wird über \Reg{0} und
\Reg{1} zurückgegeben.

\myindex{ARM!\Instructions!VDIV}
\myindex{ARM!\Instructions!VMUL}
\myindex{ARM!\Instructions!VADD}
\INS{VDIV}, \INS{VMUL} und \INS{VADD} sind Befehle zur Verarbeitung von
Fließkommazahlen, die \glslink{quotient}{Quotient}, \glslink{product}{Produkt} bzw. Summe berechnen.

Der Code für Thumb-2 ist identisch.

\subsubsection{ARM: \OptimizingKeilVI (\ThumbMode)}

\lstinputlisting[style=customasmARM]{patterns/12_FPU/1_simple/ARM/Keil_O3_thumb_DE.asm}

Keil erzeugte Code für einen Prozessor ohne FPU oder NEON Unterstützung.

Die Fließkommazahlen in doppelter Genauigkeit werden über die üblichen
R-Register übergeben und anstelle von FPU-Befehlen werden Programmbibliotheken
(wie z.B. \GTT{\_\_aeabi\_dmul}, \GTT{\_\_aeabi\_ddiv}, \GTT{\_\_aeabi\_dadd})
aufgerufen, welche Multiplikation, Division und Addition auf Fließkommazahlen
emulieren. 

Diese Vorgehensweise ist natürlich langsamer als der FPU-Koprozessor, aber es
ist besser als nichts.

Übrigens waren ähnliche FPU-emulierende Programmbibliotheken auch in der
x86-Welt sehr beliebt als Koprozessoren selten und teuer waren und nur auf
wertvollen Computern installiert waren.

\myindex{ARM!soft float}
\myindex{ARM!armel}
\myindex{ARM!armhf}
\myindex{ARM!hard float}
Die Emulation des FPU-Koprozessors wird \emph{soft float} oder \emph{armel} (in der
ARM-Welt) genannt, wohingegen die FPU-Befehle des Koprozessors \emph{hard float}
oder \emph{armhf} genannt werden.

\iffalse
% TODO разобраться...
\myindex{Raspberry Pi}
Der Linux Kernel des Raspberry Pi beispielsweise wird in zwei Varianten
kompiliert.

Im Falle von \emph{soft float} werden Parameter über R-Register übergeben und im
Falle von \emph{hard float} über D-Register.

Diese Tatsache hält einen davon ab armhf-Programmbibliotheken für armel-Code
oder umgekehrt zu verwenden und dies ist der Grund warum der gesamte Code in
Linux-Distributionen speziell für eine der beiden Konventionen kompiliert wird.
\fi

\subsubsection{ARM64: \Optimizing GCC (Linaro) 4.9}

Sehr kompakter Code:

\lstinputlisting[caption=\Optimizing GCC (Linaro) 4.9,style=customasmARM]{patterns/12_FPU/1_simple/ARM/ARM64_GCC_O3_DE.s}

\subsubsection{ARM64: \NonOptimizing GCC (Linaro) 4.9}

\lstinputlisting[caption=\NonOptimizing GCC (Linaro) 4.9,style=customasmARM]{patterns/12_FPU/1_simple/ARM/ARM64_GCC_O0_DE.s}

\NonOptimizing GCC ist geschwätziger.
Hier findet eine Menge unnützes Verschieben von Werten statt, inklusive einigem
eindeutig redundantem Code (die letzten beiden \INS{FMOV} Befehle). Vermutlich
ist GCC 4.9 noch nicht besonders gut im Erzeugen von ARM64 Code.

Bemerkenswert ist, dass ARM64 64-Bit-Register besitzt und die D-Register
ebenfalls 64 Bit breit sind.

Dadurch steht es dem Compiler frei Werte von Typ \Tdouble in \ac{GPR}s anstelle
auf dem lokalen Stack zu speichern. Dies ist in 32-bit-CPUs nicht möglich.

Wiederum kann man als Übung versuchen diese Funktion manuell zu optimieren ohne
neue Befehl wie \INS{FMADD} einzuführen. 
