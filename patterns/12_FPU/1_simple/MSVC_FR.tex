\myparagraph{MSVC}

Compilons-le avec MSVC 2010:

\lstinputlisting[caption=MSVC 2010: \ttf{},style=customasmx86]{patterns/12_FPU/1_simple/MSVC_FR.asm}

\FLD prend 8 octets depuis la pile et charge le nombre dans le registre \ST{0}, en
le convertissant automatiquement dans le format interne sur 80-bit (\emph{précision
étendue}):

\myindex{x86!\Instructions!FDIV}

\FDIV divise la valeur dans \ST{0} par le nombre stocké à l'adresse \\
\GTT{\_\_real@40091eb851eb851f}~---la valeur 3.14 est encodée ici.
La syntaxe assembleur ne supporte pas les nombres à virgule flottante, donc ce que
l'on voit ici est la représentation hexadécimale de 3.14 au format 64-bit IEEE 754.

Après l'exécution de \FDIV, \ST{0} contient le \gls{quotient}.

\myindex{x86!\Instructions!FDIVP}

À propos, il y a aussi l'instruction \FDIVP, qui divise \ST{1} par \ST{0}, prenant
ces deux valeurs dans la pile et poussant le résultant.
Si vous connaissez le langage Forth\FNURLFORTH, vous pouvez comprendre rapidement
que ceci est une machine à pile\FNURLSTACK.

L'instruction \FLD subséquente pousse la valeur de $b$ sur la pile.

Après cela, le quotient est placé dans \ST{1}, et \ST{0} a la valeur de $b$.

\myindex{x86!\Instructions!FMUL}

L'instruction suivante effectue la multiplication: $b$ de \ST{0} est multiplié par
la valeur en\\
\GTT{\_\_real@4010666666666666} (le nombre 4.1 est là) et met le résultat
dans le registre \ST{0}.

\myindex{x86!\Instructions!FADDP}

La dernière instruction \FADDP ajoute les deux valeurs au sommet de la pile, stockant
le résultat dans \ST{1} et supprimant la valeur de \ST{0}, laissant ainsi le résultat
au sommet de la pile, dans \ST{0}.

La fonction doit renvoyer son résultat dans le registre \ST{0}, donc il n'y a aucune
autre instruction après \FADDP, excepté l'épilogue de la fonction.

\clearpage
\myparagraph{MSVC + \olly}
\myindex{\olly}

2 paires de mots 32-bit sont marquées en rouge sur la pile.
Chaque paire est un double au format IEEE 754 et est passée depuis \main.

Nous voyons comment le premier \FLD charge une valeur ($1.2$) depuis la pile et la
stocke dans \ST{0}:

\begin{figure}[H]
\centering
\myincludegraphics{patterns/12_FPU/1_simple/olly1.png}
\caption{\olly: le premier \FLD a été exécuté}
\label{fig:FPU_simple_olly_1}
\end{figure}

À cause des inévitables erreurs de conversion depuis un nombre flottant 64-bit au
format IEEE 754 vers 80-bit (utilisé en interne par le FPU), ici nous voyons 1.199\ldots,
qui est proche de 1.2.

\EIP pointe maintenant sur l'instruction suivante (\FDIV), qui charge un double
(une constante) depuis la mémoire.
Par commodité, \olly affiche sa valeur: 3.14

\clearpage
Continuons l'exécution pas à pas.
\FDIV a été exécuté, maintenant \ST{0} contient 0.382\ldots
(\gls{quotient}):

\begin{figure}[H]
\centering
\myincludegraphics{patterns/12_FPU/1_simple/olly2.png}
\caption{\olly: \FDIV a été exécuté}
\label{fig:FPU_simple_olly_2}
\end{figure}

\clearpage
Troisième étape: le \FLD suivant a été exécuté, chargeant 3.4 dans \ST{0} (ici nous
voyons la valeur approximative 3.39999\ldots):

\begin{figure}[H]
\centering
\myincludegraphics{patterns/12_FPU/1_simple/olly3.png}
\caption{\olly: le second \FLD a été exécuté}
\label{fig:FPU_simple_olly_3}
\end{figure}

En même temps, le \gls{quotient} \emph{est poussé} dans \ST{1}.
Exactement maintenant, \EIP pointe sur la prochaine instruction: \FMUL.
Ceci charge la constante 4.1 depuis la mémoire, ce que montre \olly.

\clearpage
Suivante: \FMUL a été exécutée, donc maintenant le \glslink{product}{produit} est dans \ST{0}:

\begin{figure}[H]
\centering
\myincludegraphics{patterns/12_FPU/1_simple/olly4.png}
\caption{\olly: \FMUL a été exécuté}
\label{fig:FPU_simple_olly_4}
\end{figure}

\clearpage
Suivante: \FADDP a été exécutée, maintenant le résultat de l'addition est dans \ST{0},
et \ST{1} est vidé.

\begin{figure}[H]
\centering
\myincludegraphics{patterns/12_FPU/1_simple/olly5.png}
\caption{\olly: \FADDP a été exécuté}
\label{fig:FPU_simple_olly_5}
\end{figure}

Le résultat est laissé dans \ST{0}, car la fonction renvoie son résultat dans \ST{0}.

\main prend cette valeur depuis le registre plus loin.

Nous voyons quelque chose d'inhabituel: la valeur 13.93\ldots se trouve maintenant
dans \ST{7}.
Pourquoi?

\label{FPU_is_rather_circular_buffer}

Comme nous l'avons lu il y a quelque temps dans ce livre, les registres \ac{FPU} sont
une pile: \myref{FPU_is_stack}.
Mais ceci est une simplification.

Imaginez si cela était implémenté \emph{en hardware} comme cela est décrit, alors
tout le contenu des 7 registres devrait être déplacé (ou copié) dans les registres
adjacents lors d'un push ou d'un pop, et ceci nécessite beaucoup de travail.

En réalité, le \ac{FPU} a seulement 8 registres et un pointeur (appelé \GTT{TOP})
qui contient un numéro de registre, qui est le \q{haut de la pile} courant.

Lorsqu'une valeur est poussée sur la pile, \GTT{TOP} est déplacé sur le registre
disponible suivant, et une valeur est écrite dans ce registre.

La procédure est inversée si la valeur est lue, toutefois, le registre qui a été
libéré n'est pas vidé (il serait possible de le vider, mais ceci nécessite plus de
travail qui peut dégrader les performances).
Donc, c'est ce que nous voyons ici.

On peut dire que \FADDP sauve la somme sur la pile, et y supprime un élément.

Mais en fait, cette instruction sauve la somme et ensuite décale \GTT{TOP}.

Plus précisément, les registres du  \ac{FPU} sont un tampon circulaire.

