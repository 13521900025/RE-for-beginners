\subsubsection{MIPS}

MIPSはいくつかのコプロセッサ(最大4個)をサポートすることができます。
そのうちの0番目\footnote{0から始まる}は特別な制御コプロセッサであり、最初のコプロセッサはFPUです。

ARMと同様に、MIPSコプロセッサはスタックマシンではなく、32個の32ビットレジスタ(\$F0-\$F31)を持ちます。
\myref{MIPS_FPU_registers}.

64ビットの \Tdouble 値を扱う必要がある場合、32ビットのFレジスタのペアが使用されます。

\lstinputlisting[caption=\Optimizing GCC 4.4.5 (IDA),style=customasmMIPS]{patterns/12_FPU/1_simple/MIPS_O3_IDA_JA.lst}

新しい命令は以下です。

\myindex{MIPS!\Instructions!LWC1}
\myindex{MIPS!\Instructions!DIV.D}
\myindex{MIPS!\Instructions!MUL.D}
\myindex{MIPS!\Instructions!ADD.D}
\begin{itemize}

\item \INS{LWC1}は32ビットワードを第1コプロセッサのレジスタにロードします(命令名は\q{1})。
\myindex{MIPS!\Pseudoinstructions!L.D}

一対の\INS{LWC1}命令を組み合わせて\INS{L.D}疑似命令にすることができます。

\item \INS{DIV.D}、 \INS{MUL.D}、 \INS{ADD.D}はそれぞれ除算、乗算、加算を行います
(接尾辞の\q{.D}は倍精度、\q{.S}は単精度を表します)

\end{itemize}

\myindex{MIPS!\Instructions!LUI}
\myindex{\CompilerAnomaly}
\label{MIPS_FPU_LUI}

また、奇妙なコンパイラの例外があります\INS{LUI}命令に疑問符がついています。 
\$V0 レジスタに64ビット定数の \Tdouble 型の一部をロードする理由を理解することは難しいです。 
これらの命令は何の効果もありません。 
% TODO did you try checking out compiler source code?
これについて何か知っているなら、著者に電子メール\footnote{\EMAIL}を送ってください。
