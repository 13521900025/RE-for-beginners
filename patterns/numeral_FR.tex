\subsection{Systèmes de numération}

\epigraph{Nowadays octal numbers seem to be
used for exactly one purpose---file permissions on POSIX systems---but hexadecimal numbers
are widely used to emphasize the bit pattern of a number over its numeric value.}
{Alan A. A. Donovan, Brian W. Kernighan ---  The Go Programming Language}

Les Hommes ont probablement pris l'habitude d'utiliser la numérotation décimale parce
qu'ils ont 10 doigts. Néanmoins, le nombre 10 n'a pas de signification particulière
en science et en mathématiques. En électronique, le système de numérotation est le
binaire : 0 pour l'absence de courant dans un fil et 1 s'il y en a. 10 en binaire
est 2 en décimal; 100 en binaire est 4 en décimal et ainsi de suite.

Si le système de numération a 10 chiffres, il est en \emph{base} 10. % \emph{radix} ?
Le système binaire est en \emph{base} 2.

Choses importantes à retenir:

1) Un \emph{nombre} est un nombre, tandis qu'un \emph{chiffre} est
un élément d'un système d'écriture et est généralement un caractère

2) Un nombre ne change pas lorsqu'on le convertit dans une autre base; seule sa
représentation écrite change (et donc la façon de le représenter en \ac{RAM}).

\subsection{Conversion d'une base à une autre}

La notation positionnelle est utilisée dans presque tous les systèmes de numération,
cela signifie qu'un chiffre a un poids dépendant de sa position dans la représentation
du nombre. Si 2 se situe le plus à droite, c'est 2. S'il est placé un chiffre avant
celui le plus à droite, c'est 20.

Que représente $1234$ ?

$10^3 \cdot 1 + 10^2 \cdot 2 + 10^1 \cdot 3 + 1 \cdot 4$ = 1234 ou $1000 \cdot 1 +
100 \cdot 2 + 10 \cdot 3 + 4 = 1234$

De même pour les nombres binaires, mais la base est 2 au lieu de 10. Que représente
0b101011 ?

$2^5 \cdot 1 + 2^4 \cdot 0 + 2^3 \cdot 1 + 2^2 \cdot 0 + 2^1 \cdot 1 + 2^0 \cdot 1 =
43$ ou $32 \cdot 1 + 16 \cdot 0 + 8 \cdot 1 + 4 \cdot 0 + 2 \cdot 1 + 1 = 43$

Il existe aussi la notation non-positionnelle comme la numération romaine\footnote{À
propos de l'évolution du système de numération, voir \InSqBrackets{\TAOCPvolII{},
195--213.}}.
Peut-être que l'humanité a choisi le système de numération positionnelle parce qu'il
était plus simple pour les opérations basiques (addition, multiplication, etc.) à
la main sur papier.

En effet, les nombres binaires peuvent être ajoutés, soustraits et ainsi de suite de
la même manière que c'est enseigné à l'école, mais seulement 2 chiffres sont
disponibles.

Les nombres binaires sont volumineux lorsqu'ils sont représentés dans le code source
et les dumps, c'est pourquoi le système hexadécimal peut être utilisé.
La base hexadécimale utilise les nombres 0..9 et aussi 6 caractères latins : A..F.
Chaque chiffre hexadécimal prend 4 bits ou 4 chiffres binaires, donc c'est très simple
de convertir un nombre binaire vers l'hexadécimal et inversement, même manuellement,
de tête.

\begin{center} \begin{longtable}{ | l | l | l | }
\hline
\HeaderColor hexadécimal & \HeaderColor binaire & \HeaderColor décimal \\
\hline
0	&0000	&0 \\
1	&0001	&1 \\
2	&0010	&2 \\
3	&0011	&3 \\
4	&0100	&4 \\
5	&0101	&5 \\
6	&0110	&6 \\
7	&0111	&7 \\
8	&1000	&8 \\
9	&1001	&9 \\
A	&1010	&10 \\
B	&1011	&11 \\
C	&1100	&12 \\
D	&1101	&13 \\
E	&1110	&14 \\
F	&1111	&15 \\
\hline
\end{longtable}
\end{center}

Comment savoir quelle est la base utilisée dans un cas particulier?

Les nombres décimaux sont d'ordinaire écrits tels quels, i.e, 1234. Certains
assembleurs permettent d'accentuer la base décimale, et les nombres peuvent alors
s'écrire avec le suffixe "d" : 1234d.

Les nombres binaires sont parfois préfixés avec "0b" : 0b100110111 (\ac{GCC} a une
extension de langage non-standard pour ça
\footnote{\url{https://gcc.gnu.org/onlinedocs/gcc/Binary-constants.html}}). Il y a
aussi un autre moyen : le suffixe "b", par exemple : 100110111b. J'essaierai de
garder le préfixe "0b" tout le long du livre pour les nombres binaires.

Les nombres hexadécimaux sont préfixés avec "0x" en \CCpp et autres \ac{PL}s :
0x1234ABCD. Ou ils ont le suffixe "h" : 1234ABCDh - c'est une manière commune de les
représenter dans les assembleurs et les débogueurs. Si le nombre commence par un
A..F, un 0 est ajouté au début : 0ABCDEFh. Il y avait une convention répandue à
l'ère des ordinateurs personnels 8-bit, utilisant le préfixe \$, comme \$ABCD.
J'essaierai de garder le préfixe "0x" tout le long du livre pour les nombres
hexadécimaux.

Faut-il apprendre à convertir les nombres de tête? La table des nombres hexadécimaux
de 1 chiffre peut facilement être mémorisée. Pour les nombres plus gros, ce n'est pas
la peine de se tourmenter.

Peut-être que les nombres hexadécimaux les plus visibles sont dans les \ac{URL}s.
C'est la façon d'encoder les caractères non-Latin.
Par exemple:
\url{https://en.wiktionary.org/wiki/na\%C3\%AFvet\%C3\%A9} est l'\ac{URL} de l'article
de Wiktionary à propos du mot \q{naïveté}.

\subsubsection{Base octale}

Un autre système de numération a été largement utilisé en informatique est la
représentation octale. Elle comprend 8 chiffres (0..7), et chacun occupe 3 bits,
donc c'est facile de convertir un nombre d'une base à l'autre. Il est maintenant
remplacé par le système hexadécimal quasiment partout mais, chose surprenante, il
y a encore une commande sur *NIX, utilisée par beaucoup de personnes, qui a un nombre
octal comme argument : \TT{chmod}.

\myindex{UNIX!chmod}
Comme beaucoup d'utilisateurs *NIX le savent, l'argument de
\TT{chmod} peut être un nombre à 3 chiffres. Le premier correspond aux droits du
propriétaire du fichier, le second correspond aux droits pour le groupe (auquel le
fichier appartient), le troisième est pour tous les autres. Et chaque chiffre peut
être représenté en binaire:

\begin{center}
\begin{longtable}{ | l | l | l | }
\hline
\HeaderColor décimal &
\HeaderColor binaire & \HeaderColor signification \\
\hline
7	&111	&\textbf{rwx} \\
6	&110	&\textbf{rw-} \\
5	&101	&\textbf{r-x} \\
4	&100	&\textbf{r-{}-} \\
3	&011	&\textbf{-wx} \\
2	&010	&\textbf{-w-} \\
1	&001	&\textbf{-{}-x} \\
0	&000	&\textbf{-{}-{}-} \\
\hline
\end{longtable}
\end{center}

Ainsi chaque bit correspond à un droit: lecture (r) / écriture (w) / exécution (x).

L'importance de \TT{chmod} est que le nombre entier en argument peut être écrit
comme un nombre octal. Prenons par exemple, 644. Quand vous tapez \TT{chmod 644
file}, vous définissez les droits de lecture/écriture pour le propriétaire, les
droits de lecture pour le groupe et encore les droits de lecture pour tous les autres.
Convertissons le nombre octal 644 en binaire, ça donne \TT{110100100}, ou (par groupe
de 3 bits) \TT{110 100 100}.

Maintenant que nous savons que chaque triplet sert à décrire les permissions pour le
propriétaire/groupe/autres : le premier est \TT{rw-}, le second est \TT{r--} et le
troisième est \TT{r--}.

Le système de numération octal était aussi populaire sur les vieux ordinateurs comme
le PDP-8 parce que les mots pouvaient être de 12, 24 ou de 36 bits et ces nombres
sont divisibles par 3, donc la représentation octale était naturelle dans cet
environnement. Aujourd'hui, tous les ordinateurs populaires utilisent des
mots/taille d'adresse de 16, 32 ou de 64 bits et ces nombres sont divisibles par 4,
donc la représentation hexadécimale était plus naturelle ici.

Le système de numération octal est supporté par tous les compilateurs \CCpp
standards. C'est parfois une source de confusion parce que les nombres octaux sont
notés avec un zéro au début. Par exemple, 0377 est 255. Et parfois, vous faites une
faute de frappe et écrivez "09" au lieu de 9, et le compilateur renvoie une erreur.
GCC peut renvoyer quelque chose comme ça:\\ \TT{erreur: chiffre  9  invalide
dans la constante en base 8}.

De même, le système octal est assez populaire en Java. Lorsque \IDA affiche des
chaînes Java avec des caractères non-imprimables, ils sont encodés dans le système
octal au lieu d'hexadécimal.
\myindex{JAD}
Le décompilateur Java JAD se comporte de la même façon.

\subsubsection{Divisibilité}

Quand vous voyez un nombre décimal comme 120, vous en déduisez immédiatement qu'il
est divisible par 10, parce que le dernier chiffre est zéro.
De la même façon, 123400 est divisible par 100 parce que les deux derniers chiffres
sont zéros.

Pareillement, le nombre hexadécimal 0x1230 est divisible par 0x10 (ou 16),
0x123000 est divisible par 0x1000 (ou 4096), etc.

Un nombre binaire 0b1000101000 est divisible par 0b1000 (8), etc.

Cette propriété peut être souvent utilisée pour déterminer rapidement si l'adresse
ou la taille d'un bloc mémoire correspond à une limite.
Par exemple, les sections dans les fichiers \ac{PE} commencent quasiment toujours
à une adresse finissant par 3 zéros hexadécimaux: 0x41000, 0x10001000, etc. La raison
sous-jacente est que la plupart des sections \ac{PE} sont alignées sur une limite
de 0x1000 (4096) octets.

\subsubsection{Arithmétique multi-précision et base}

\myindex{RSA}
L'arithmétique multi-précision utilise des nombres très grands et chacun peut être
stocké sur plusieurs octets. Par exemple, les clés RSA, tant publique que privée,
utilisent jusqu'à 4096 bits et parfois plus encore.

Dans \InSqBrackets{\TAOCPvolII, 265} nous trouvons l'idée suivante: quand vous
stockez un nombre multi-précision dans plusieurs octets, le nombre complet peut être
représenté dans une base de $2^8=256$, et chacun des chiffres correspond à un octet.
De la même manière, si vous sauvegardez un nombre multi-précision sur plusieurs
entiers de 32 bits, chaque chiffre est associé à l'emplacement de 32 bits et vous
pouvez penser à ce nombre comme étant stocké dans une base $2^{32}$.

\subsubsection{Comment prononcer les nombres non-décimaux}

Les nombres dans une base non décimale sont généralement prononcés un chiffre à
la fois : ``un-zéro-zéro-un-un-...''. Les mots comme ``dix``, ``mille``, etc, ne
sont généralement pas prononcés, pour éviter d'être confondus avec ceux en
base décimale.

\subsubsection{Nombres à virgule flottante}

Pour distinguer les nombres à virgule flottante des entiers, ils sont souvent écrits
avec avec un ``.0`` à la fin, comme $0.0$, $123.0$, etc.
