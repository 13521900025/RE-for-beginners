\subsection{Loading a 32-bit constant into register}
\label{MIPS_big_constants}

\begin{lstlisting}[style=customc]
unsigned int f()
{
	return 0x12345678;
};
\end{lstlisting}

All instructions in MIPS, just like ARM, have a size of 32-bit, so it's not possible to
embed a 32-bit constant into one instruction.
\myindex{MIPS!\Instructions!LI}
\myindex{MIPS!\Instructions!ORI}

So one have to use at least two instructions: 
the first loads the high part of the 32-bit number and the second
one applies an OR operation, which effectively sets the low 16-bit part of the target register:

\begin{lstlisting}[caption=GCC 4.4.5 -O3 (\assemblyOutput),style=customasmMIPS]
        li      $2,305397760  # 0x12340000
        j       $31
        ori     $2,$2,0x5678 ; branch delay slot
\end{lstlisting}

\IDA is fully aware of such frequently encountered code patterns, 
so, for convenience it shows the last \INS{ORI} instruction as the \INS{LI} pseudo instruction,
which allegedly loads a full 32-bit number into the \$V0 register.

\myindex{MIPS!\Instructions!LUI}

\begin{lstlisting}[caption=GCC 4.4.5 -O3 (IDA),style=customasmMIPS]
         lui     $v0, 0x1234
         jr      $ra
         li      $v0, 0x12345678 ; branch delay slot
\end{lstlisting}

The GCC assembly output has the LI pseudo instruction, but in fact, \INS{LUI} (\q{Load Upper Immediate}) is there,
which stores a 16-bit value into the high part of the register.

Let's see in \IT{objdump} output:

\begin{lstlisting}[caption=objdump,style=customasmMIPS]
00000000 <f>:
   0:   3c021234        lui     v0,0x1234
   4:   03e00008        jr      ra
   8:   34425678        ori     v0,v0,0x5678
\end{lstlisting}

\subsubsection{Loading a 32-bit global variable into register}

\begin{lstlisting}[style=customc]
unsigned int global_var=0x12345678;

unsigned int f2()
{
        return global_var;
};
\end{lstlisting}

\myindex{MIPS!\Instructions!LW}

This is slightly different: \INS{LUI} loads upper 16-bit from \IT{global\_var} into \$2 (or \$V0) and then \INS{LW} loads lower 16-bits summing it with the contents of \$2:

\begin{lstlisting}[caption=GCC 4.4.5 -O3 (\assemblyOutput),style=customasmMIPS]
f2:
        lui     $2,%hi(global_var)
        lw      $2,%lo(global_var)($2)
        j       $31
        nop	; branch delay slot

	...

global_var:
        .word   305419896
\end{lstlisting}

\IDA is fully aware of often used \INS{LUI}/\INS{LW} instruction pair, so it coalesces both into a single \INS{LW} instruction:

\begin{lstlisting}[caption=GCC 4.4.5 -O3 (IDA),style=customasmMIPS]
_f2:
                lw      $v0, global_var
                jr      $ra
                or      $at, $zero	; branch delay slot

		...

                .data
                .globl global_var
global_var:     .word 0x12345678         # DATA XREF: \_f2
\end{lstlisting}

\IT{objdump}'s output is the same as GCC's assembly output.
Let's also dump relocs of the object file:

\begin{lstlisting}[caption=objdump,style=customasmMIPS]
objdump -D filename.o

...

0000000c <f2>:
   c:   3c020000        lui     v0,0x0
  10:   8c420000        lw      v0,0(v0)
  14:   03e00008        jr      ra
  18:   00200825        move    at,at	; branch delay slot
  1c:   00200825        move    at,at

Disassembly of section .data:

00000000 <global_var>:
   0:   12345678        beq     s1,s4,159e4 <f2+0x159d8>

...

objdump -r filename.o

...

RELOCATION RECORDS FOR [.text]:
OFFSET   TYPE              VALUE
0000000c R_MIPS_HI16       global_var
00000010 R_MIPS_LO16       global_var

...

\end{lstlisting}

We can see that address of \IT{global\_var} is to be written right into \INS{LUI} and \INS{LW} instructions during executable file loading:
high 16-bit part of \IT{global\_var} goes into the first one (\INS{LUI}), lower 16-bit part goes into the second one (\INS{LW}).

