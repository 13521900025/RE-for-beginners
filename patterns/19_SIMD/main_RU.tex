\mysection{SIMD}

\label{SIMD_x86}
\ac{SIMD} это акроним: \emph{Single Instruction, Multiple Data}.

Как можно судить по названию, это обработка множества данных исполняя только одну инструкцию.

Как и \ac{FPU}, эта подсистема процессора выглядит так же отдельным процессором внутри x86.

\myindex{x86!MMX}
SIMD в x86 начался с MMX. Появилось 8 64-битных регистров MM0-MM7.

Каждый MMX-регистр может содержать 2 32-битных значения, 4 16-битных или же 8 байт. 
Например, складывая значения двух MMX-регистров, можно складывать одновременно 8 8-битных значений.

Простой пример, это некий графический редактор, который хранит открытое изображение как двумерный массив. 
Когда пользователь меняет яркость изображения, редактору нужно, например, прибавить некий коэффициент 
ко всем пикселям, или отнять. 
Для простоты можно представить, что изображение у нас бело-серо-черное и каждый пиксель занимает один байт, 
то с помощью MMX можно менять яркость сразу у восьми пикселей.

Кстати, вот причина почему в SIMD присутствуют инструкции с \emph{насыщением} (\emph{saturation}).

Когда пользователь в графическом редакторе изменяет яркость, переполнение и антипереполнение (\emph{underflow})
не нужны, так что в SIMD имеются, например, инструкции сложения, которые ничего не будут прибавлять
если максимальное значение уже достигнуто, итд.

Когда MMX только появилось, эти регистры на самом деле располагались в FPU-регистрах. 
Можно было использовать 
либо FPU либо MMX в одно и то же время. Можно подумать, что Intel решило немного сэкономить на транзисторах, 
но на самом деле причина такого симбиоза проще ~--- более старая \ac{OS} не знающая о дополнительных 
регистрах процессора не будет сохранять их во время переключения задач, а вот регистры FPU сохранять будет. 
Таким образом, процессор с MMX + старая \ac{OS} + задача, использующая возможности MMX = все 
это может работать вместе.

\myindex{x86!SSE}
\myindex{x86!SSE2}
SSE --- это расширение регистров до 128 бит, теперь уже отдельно от FPU.

\myindex{x86!AVX}
AVX --- расширение регистров до 256 бит.

Немного о практическом применении.

Конечно же, это копирование блоков в памяти (\TT{memcpy}), сравнение (\TT{memcmp}), и подобное.

\myindex{DES}
Еще пример: имеется алгоритм шифрования DES, который берет 64-битный блок, 56-битный ключ, 
шифрует блок с ключом и образуется 64-битный результат.
Алгоритм DES можно легко представить в виде очень большой электронной цифровой схемы, 
с проводами, элементами И, ИЛИ, НЕ.

\label{bitslicedes}
\newcommand{\URLBS}{\url{http://go.yurichev.com/17329}}

Идея bitslice DES\footnote{\URLBS} ~--- это обработка сразу группы блоков и ключей одновременно. 
Скажем, на x86 переменная типа \emph{unsigned int} вмещает в себе 32 бита, так что там можно хранить 
промежуточные результаты сразу для 32-х блоков-ключей, используя 64+56 переменных типа \emph{unsigned int}.

\myindex{\oracle}
Существует утилита для перебора паролей/хешей \oracle (которые основаны на алгоритме DES), 
реализующая алгоритм bitslice DES для SSE2 и AVX --- и теперь возможно шифровать одновременно 
128 или 256 блоков-ключей:

\url{http://go.yurichev.com/17313}

% sections
\subsection{Векторизация}

\newcommand{\URLVEC}{\href{http://go.yurichev.com/17080}{Wikipedia: vectorization}}

Векторизация\footnote{\URLVEC} это когда у вас есть цикл, который берет на вход несколько массивов и выдает, 
например, один массив данных. 
Тело цикла берет некоторые элементы из входных массивов, что-то делает с ними и помещает в выходной. 
%Важно, что операция, применяемая ко всем элементам одна и та же. 
Векторизация ~--- это обрабатывать несколько элементов одновременно.

Векторизация ~--- это не самая новая технология: автор сих строк видел её по крайней мере на 
линейке суперкомпьютеров Cray Y-MP от 1988, когда работал на его версии-\q{лайт} Cray Y-MP EL
\footnote{Удаленно. Он находится в музее суперкомпьютеров: \url{http://go.yurichev.com/17081}}.

% FIXME! add assembly listing!
Например:

\begin{lstlisting}[style=customc]
for (i = 0; i < 1024; i++)
{
    C[i] = A[i]*B[i];
}
\end{lstlisting}

Этот фрагмент кода берет элементы из A и B, перемножает и сохраняет результат в C.

\myindex{x86!\Instructions!PLMULLD}
\myindex{x86!\Instructions!PLMULHW}
\newcommand{\PMULLD}{\emph{PMULLD} (\emph{Перемножить запакованные знаковые DWORD и сохранить младшую часть результата})}
\newcommand{\PMULHW}{\TT{PMULHW} (\emph{Перемножить запакованные знаковые DWORD и сохранить старшую часть результата})}

Если представить, что каждый элемент массива ~--- это 32-битный \Tint, то их можно загружать сразу 
по 4 из А в 128-битный XMM-регистр, 
из B в другой XMM-регистр и выполнив инструкцию \PMULLD{} и \PMULHW{}, можно получить 4 64-битных 
\glslink{product}{произведения} сразу.

Таким образом, тело цикла исполняется $1024/4$ раза вместо 1024, что в 4 раза меньше, и, конечно, быстрее.

\newcommand{\URLINTELVEC}{\href{http://go.yurichev.com/17082}{Excerpt: Effective Automatic Vectorization}}

\subsubsection{Пример сложения}

\myindex{Intel C++}
Некоторые компиляторы умеют делать автоматическую векторизацию в простых случаях, 
например, Intel C++\footnote{Еще о том, как Intel C++ умеет автоматически векторизовать циклы: \URLINTELVEC}.

Вот очень простая функция:

\begin{lstlisting}[style=customc]
int f (int sz, int *ar1, int *ar2, int *ar3)
{
	for (int i=0; i<sz; i++)
		ar3[i]=ar1[i]+ar2[i];

	return 0;
};
\end{lstlisting}

\myparagraph{Intel C++}

Компилируем её при помощи Intel C++ 11.1.051 win32:

\begin{verbatim}
icl intel.cpp /QaxSSE2 /Faintel.asm /Ox
\end{verbatim}

Имеем такое (в \IDA):

\lstinputlisting[style=customasmx86]{patterns/19_SIMD/18_1_RU.asm}

Инструкции, имеющие отношение к SSE2 это:
\myindex{x86!\Instructions!MOVDQA}
\myindex{x86!\Instructions!MOVDQU}
\myindex{x86!\Instructions!PADDD}
\begin{itemize}
\item
\MOVDQU (\emph{Move Unaligned Double Quadword}) --- она просто загружает 16 байт из памяти в XMM-регистр.

\item
\PADDD (\emph{Add Packed Integers}) --- складывает сразу 4 пары 32-битных чисел и оставляет в первом операнде результат. 
Кстати, если произойдет переполнение, то исключения не произойдет и никакие флаги не установятся, 
запишутся просто младшие 32 бита результата. 
Если один из операндов \PADDD ~--- адрес значения в памяти, 
то требуется чтобы адрес был выровнен по 16-байтной границе. Если он не выровнен, произойдет исключение
\footnote{О выравнивании данных см. также: \URLWPDA}.

\item
\MOVDQA (\emph{Move Aligned Double Quadword}) --- тоже что и \MOVDQU, только подразумевает 
что адрес в памяти выровнен по 16-байтной границе. 
Если он не выровнен, произойдет исключение. 
\MOVDQA работает быстрее чем \MOVDQU, но требует вышеозначенного.

\end{itemize}

Итак, эти SSE2-инструкции исполнятся только в том случае если еще осталось просуммировать 
4 пары переменных типа \Tint плюс если указатель \TT{ar3} выровнен по 16-байтной границе.

Более того, если еще и \TT{ar2} выровнен по 16-байтной границе, 
то будет выполняться этот фрагмент кода:

\begin{lstlisting}[style=customasmx86]
movdqu  xmm0, xmmword ptr [ebx+edi*4] ; ar1+i*4
paddd   xmm0, xmmword ptr [esi+edi*4] ; ar2+i*4
movdqa  xmmword ptr [eax+edi*4], xmm0 ; ar3+i*4
\end{lstlisting}

А иначе, значение из \TT{ar2} загрузится в \XMM{0} используя инструкцию \MOVDQU, 
которая не требует выровненного указателя, зато может работать чуть медленнее:

\lstinputlisting[style=customasmx86]{patterns/19_SIMD/18_1_excerpt_RU.asm}

А во всех остальных случаях, будет исполняться код, который был бы, как если бы не была 
включена поддержка SSE2.

\myparagraph{GCC}

\newcommand{\URLGCCVEC}{\url{http://go.yurichev.com/17083}}

Но и GCC умеет кое-что векторизировать\footnote{Подробнее о векторизации в GCC: \URLGCCVEC}, 
если компилировать с опциями \Othree и включить поддержку SSE2: \TT{-msse2}.

Вот что вышло (GCC 4.4.1):

\lstinputlisting[style=customasmx86]{patterns/19_SIMD/18_2_gcc_O3.asm}

Почти то же самое, хотя и не так дотошно, как Intel C++.

\subsubsection{Пример копирования блоков}
\label{vec_memcpy}

Вернемся к простому примеру memcpy() (\myref{loop_memcpy}):

\lstinputlisting[style=customc]{memcpy.c}

И вот что делает оптимизирующий GCC 4.9.1:

\lstinputlisting[caption=\Optimizing GCC 4.9.1 x64,style=customasmx86]{patterns/19_SIMD/memcpy_GCC49_x64_O3_RU.s}

\subsection{Реализация \strlen при помощи SIMD}
\label{SIMD_strlen}

\newcommand{\URLMSDNSSE}{\href{http://go.yurichev.com/17262}{MSDN: MMX, SSE, and SSE2 Intrinsics}}

Прежде всего, следует заметить, что SIMD-инструкции можно вставлять в \CCpp код при помощи специальных 
макросов\footnote{\URLMSDNSSE}. В MSVC, часть находится в файле \TT{intrin.h}.

\newcommand{\URLSTRLEN}{http://go.yurichev.com/17330}

\myindex{\CStandardLibrary!strlen()}
Имеется возможность реализовать функцию \strlen\footnote{strlen() ~--- стандартная функция Си 
для подсчета длины строки} при помощи SIMD-инструкций, работающий в 2-2.5 раза быстрее обычной реализации. 
Эта функция будет загружать в XMM-регистр сразу 16 байт и проверять каждый на ноль

\footnote{Пример базируется на исходнике отсюда: \url{\URLSTRLEN}.}.

\lstinputlisting[style=customc]{patterns/19_SIMD/18_3.c}

Компилируем в MSVC 2010 с опцией \Ox:

\lstinputlisting[caption=\Optimizing MSVC 2010,style=customasmx86]{patterns/19_SIMD/18_4_msvc_Ox_RU.asm}

Как это работает?
Прежде всего, нужно определиться с целью этой ф-ции.
Она вычисляет длину Си-строки, но можно сказать иначе --- её задача это поиск нулевого байта, а затем вычисление его позиции относительно начала строки.

Итак, прежде всего, мы проверяем указатель \TT{str}, выровнен ли он по 16-байтной границе. 
Если нет, то мы вызовем обычную реализацию \strlen.

Далее мы загружаем по 16 байт в регистр \XMM{1} при помощи команды \MOVDQA.

Наблюдательный читатель может спросить, почему в этом месте мы не можем использовать \MOVDQU, 
которая может загружать откуда угодно невзирая на факт, выровнен ли указатель?

Да, можно было бы сделать вот как: если указатель выровнен, загружаем используя \MOVDQA, 
иначе используем работающую чуть медленнее \MOVDQU.

Однако здесь кроется не сразу заметная проблема, которая проявляется вот в чем:

\myindex{Page (memory)}
\newcommand{\URLPAGE}{\href{http://go.yurichev.com/17136}{wikipedia}}

В \ac{OS} линии \gls{Windows NT} (и не только), память выделяется страницами по 4 KiB (4096 байт). 
Каждый win32-процесс якобы имеет в наличии 4 GiB, но на самом деле, 
только некоторые части этого адресного пространства присоединены к реальной физической памяти. 
Если процесс обратится к блоку памяти, которого не существует, сработает исключение. 
Так работает \ac{VM}\footnote{\URLPAGE}.

Так вот, функция, читающая сразу по 16 байт, имеет возможность нечаянно вылезти за границу 
выделенного блока памяти. 
Предположим, \ac{OS} выделила программе 8192 (0x2000) байт по адресу 0x008c0000. 
Таким образом, блок занимает байты с адреса 0x008c0000 по 0x008c1fff включительно.

За этим блоком, то есть начиная с адреса 0x008c2000 нет вообще ничего, т.е. \ac{OS} не выделяла там память. 
Обращение к памяти начиная с этого адреса вызовет исключение.

И предположим, что программа хранит некую строку из, скажем, пяти символов почти в самом конце блока, 
что не является преступлением:

\begin{center}
  \begin{tabular}{ | l | l | }
    \hline
        0x008c1ff8 & 'h' \\
        0x008c1ff9 & 'e' \\
        0x008c1ffa & 'l' \\
        0x008c1ffb & 'l' \\
        0x008c1ffc & 'o' \\
        0x008c1ffd & '\textbackslash{}x00' \\
        0x008c1ffe & здесь случайный мусор \\
        0x008c1fff & здесь случайный мусор \\
    \hline
  \end{tabular}
\end{center}

В обычных условиях, программа вызывает \strlen передав ей указатель на строку \TT{'hello'} 
лежащую по адресу 0x008c1ff8. 
\strlen будет читать по одному байту до 0x008c1ffd, где ноль, и здесь она закончит работу.

Теперь, если мы напишем свою реализацию \strlen читающую сразу по 16 байт, с любого адреса, 
будь он выровнен по 16-байтной границе или нет, 
\MOVDQU попытается загрузить 16 байт с адреса 0x008c1ff8 по 0x008c2008, и произойдет исключение. 
Это ситуация которой, конечно, хочется избежать.

Поэтому мы будем работать только с адресами, выровненными по 16 байт, что в сочетании со знанием 
что размер страницы \ac{OS} также, как правило, выровнен по 16 байт, 
даст некоторую гарантию что наша функция не будет пытаться читать из мест в невыделенной памяти.

Вернемся к нашей функции.

\myindex{x86!\Instructions!PXOR}
\verb|_mm_setzero_si128()| --- это макрос, генерирующий \TT{pxor xmm0, xmm0} ~--- инструкция просто обнуляет регистр \XMM{0}.

\verb|_mm_load_si128()| --- это макрос для \MOVDQA, он просто загружает 16 байт по адресу из указателя в \XMM{1}.

\myindex{x86!\Instructions!PCMPEQB}
\verb|_mm_cmpeq_epi8()| --- это макрос для \PCMPEQB, это инструкция, которая 
побайтово сравнивает значения из двух XMM регистров. 

И если какой-то из байт равен другому, 
то в результирующем значении будет выставлено на месте этого 
байта \TT{0xff}, либо 0, если байты не были равны.

Например:

\begin{verbatim}
XMM1: 0x11223344556677880000000000000000
XMM0: 0x11ab3444007877881111111111111111
\end{verbatim}

После исполнения \TT{pcmpeqb xmm1, xmm0}, регистр \XMM{1} содержит:

\begin{verbatim}
XMM1: 0xff0000ff0000ffff0000000000000000
\end{verbatim}

Эта инструкция в нашем случае, сравнивает каждый 16-байтный блок с блоком состоящим из 16-и нулевых байт, 
выставленным в \XMM{0} при помощи \TT{pxor xmm0, xmm0}.

\myindex{x86!\Instructions!PMOVMSKB}
Следующий макрос \TT{\_mm\_movemask\_epi8()} ~--- это инструкция \TT{PMOVMSKB}.

Она очень удобна как раз для использования в паре с \PCMPEQB.

\TT{pmovmskb eax, xmm1}

Эта инструкция выставит самый первый бит \EAX в единицу, если старший бит первого байта в 
регистре \XMM{1} является единицей. 
Иными словами, если первый байт в регистре \XMM{1} является \TT{0xff}, то первый бит в \EAX будет также единицей, 
иначе нулем.

Если второй байт в регистре \XMM{1} является \TT{0xff}, то второй бит в \EAX также будет единицей. 
Иными словами, инструкция отвечает на вопрос, \q{какие из байт в \XMM{1} имеют старший бит равный 1, или больше 0x7f?}
В результате приготовит 16 бит и запишет в \EAX. Остальные биты в \EAX обнулятся.

Кстати, не забывайте также вот о какой особенности нашего алгоритма.

На вход может прийти 16 байт вроде:

\input{patterns/19_SIMD/strlen_hello_and_garbage}

Это строка \TT{'hello'}, после нее терминирующий ноль, затем немного мусора в памяти.

Если мы загрузим эти 16 байт в \XMM{1} и сравним с нулевым \XMM{0}, то в итоге получим такое 
\footnote{Здесь используется порядок с \ac{MSB} до \ac{LSB}.}:

\begin{verbatim}
XMM1: 0x0000ff00000000000000ff0000000000
\end{verbatim}

Это означает, что инструкция сравнения обнаружила два нулевых байта, что и не удивительно.

\TT{PMOVMSKB} в нашем случае подготовит \EAX вот так:\\
\emph{0b0010000000100000}.

Совершенно очевидно, что далее наша функция должна учитывать только первый встретившийся
нулевой бит и игнорировать все остальное.

\myindex{x86!\Instructions!BSF}
\label{instruction_BSF}
Следующая инструкция --- \TT{BSF} (\emph{Bit Scan Forward}). 
Это инструкция находит самый младший бит во втором операнде и записывает его позицию в первый операнд.

\begin{verbatim}
EAX=0b0010000000100000
\end{verbatim}

После исполнения этой инструкции \TT{bsf eax, eax}, в \EAX будет 5, что означает, 
что единица найдена в пятой позиции (считая с нуля).

Для использования этой инструкции, в MSVC также имеется макрос \TT{\_BitScanForward}.

А дальше все просто. Если нулевой байт найден, его позиция прибавляется к тому что 
мы уже насчитали и возвращается результат.

Почти всё.

Кстати, следует также отметить, что компилятор MSVC сгенерировал два тела цикла сразу, для оптимизации.

Кстати, в SSE 4.2 (который появился в Intel Core i7) все эти манипуляции со строками могут быть еще проще:
 \url{http://go.yurichev.com/17331}


