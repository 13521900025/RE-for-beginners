\subsection{ベクトル化}

\newcommand{\URLVEC}{\href{http://go.yurichev.com/17080}{Wikipedia: vectorization}}

ベクトル化\footnote{\URLVEC}は、たとえば、入力用に2つの配列を取り、1つの配列を生成するループがある場合です。
ループ本体は入力配列から値を受け取り、何かを実行して結果を出力配列に入れます。
%It is important that there is only a single operation applied to each element.
ベクトル化は、いくつかの要素を同時に処理することです。

ベクトル化はそれほど新鮮なテクノロジではありません。この教科書の作者は、少なくとも
1988年のCray Y-MPスーパーコンピュータラインで、その\q{ライト}バージョンのCray Y-MP EL 179を使ったことを見ました。
\footnote{リモートから。それはスーパーコンピュータの博物館に設置されています: \url{http://go.yurichev.com/17081}}.

% FIXME! add assembly listing!
例えば:

\begin{lstlisting}[style=customc]
for (i = 0; i < 1024; i++)
{
    C[i] = A[i]*B[i];
}
\end{lstlisting}

このコード片は、AとBから要素を取り出し、それらを乗算して結果をCに保存します。

\myindex{x86!\Instructions!PLMULLD}
\myindex{x86!\Instructions!PLMULHW}
\newcommand{\PMULLD}{\emph{PMULLD} (\emph{Multiply Packed Signed Dword Integers and Store Low Result})}
\newcommand{\PMULHW}{\TT{PMULHW} (\emph{Multiply Packed Signed Integers and Store High Result})}

各配列要素が32ビット \Tint の場合、Aから128ビットのXMMレジスタへ、Bから別のXMMレジスタへ、
\PMULLD{} および \PMULHW{}を実行することで、一度に4つの64ビット\glspl{product}を取得できます。

したがって、ループ本体の実行数は1024ではなく$1024/4$となり、4倍少なくなり、もちろん高速になります。

\newcommand{\URLINTELVEC}{\href{http://go.yurichev.com/17082}{Excerpt: Effective Automatic Vectorization}}

\subsubsection{加算の例}

\myindex{Intel C++}

Intel C++\footnote{インテルC ++自動ベクトル化についての詳細: \URLINTELVEC}のように、
単純な場合には自動的にベクトル化を実行できるコンパイラーもあります。

これが小さな機能です。

\begin{lstlisting}[style=customc]
int f (int sz, int *ar1, int *ar2, int *ar3)
{
	for (int i=0; i<sz; i++)
		ar3[i]=ar1[i]+ar2[i];

	return 0;
};
\end{lstlisting}

\myparagraph{Intel C++}

それをIntel C++ 11.1.051 win32でコンパイルしましょう。

\begin{verbatim}
icl intel.cpp /QaxSSE2 /Faintel.asm /Ox
\end{verbatim}

(\IDA で)次の結果を得ました。

\lstinputlisting[style=customasmx86]{patterns/19_SIMD/18_1_JA.asm}

SSE2関連の命令は以下のとおりです。
\myindex{x86!\Instructions!MOVDQA}
\myindex{x86!\Instructions!MOVDQU}
\myindex{x86!\Instructions!PADDD}
\begin{itemize}
\item
\MOVDQU (\emph{Move Unaligned Double Quadword})---メモリから16バイトをXMMレジスタにロードします

\item
\PADDD (\emph{Add Packed Integers})---4対の32ビット数を加算し、その結果を最初のオペランドに残します。
ちなみに、オーバーフローが発生しても例外は発生せず、フラグも設定されません。
結果の下位32ビットだけが格納されます。 
\PADDD のオペランドの1つがメモリ内の値のアドレスである場合、
そのアドレスは16バイト境界に揃えられている必要があります。
整列されていない場合は、例外が発生します。
\footnote{データアラインメントについての詳細は: \URLWPDA}.

\item
\MOVDQA (\emph{Move Aligned Double Quadword})
はMOVDQUと同じですが、メモリ内の値のアドレスを16ビット境界に揃える必要があります。
整列されていないと、例外が発生します。 
\MOVDQA は \MOVDQU よりも高速に動作しますが、前述のものが必要です。

\end{itemize}

そのため、これらのSSE2命令は、作業するペアが4つ以上あり、
ポインタ\TT{ar3}が16バイト境界に整列している場合にのみ実行されます。

また、\TT{ar2}が16バイト境界にも揃えられている場合は、
次のコードが実行されます。

\begin{lstlisting}[style=customasmx86]
movdqu  xmm0, xmmword ptr [ebx+edi*4] ; ar1+i*4
paddd   xmm0, xmmword ptr [esi+edi*4] ; ar2+i*4
movdqa  xmmword ptr [eax+edi*4], xmm0 ; ar3+i*4
\end{lstlisting}

そうでなければ、\TT{ar2}からの値は、 \MOVDQU を使用して\XMM{0}にロードされます。
これは、位置合わせされたポインターを必要としませんが、遅くなる可能性があります。

\lstinputlisting[style=customasmx86]{patterns/19_SIMD/18_1_excerpt_JA.asm}

それ以外の場合は、SSE2以外のコードが実行されます。

\myparagraph{GCC}

\newcommand{\URLGCCVEC}{\url{http://go.yurichev.com/17083}}

\Othree オプションが使用され、SSE2サポートがオンになっている場合、
GCCは単純な場合にもベクトル化することがあります\footnote{GCCベクトル化サポートについての詳細は: \URLGCCVEC}

以下を得ます(GCC 4.4.1)。

\lstinputlisting[style=customasmx86]{patterns/19_SIMD/18_2_gcc_O3.asm}

しかし、ほぼ同じですが、Intel C++ほど細心の注意を払っていません。

\subsubsection{メモリコピーの例}
\label{vec_memcpy}

簡単なmemcpy()の例をもう一度見てみましょう。
(\myref{loop_memcpy}):

\lstinputlisting[style=customc]{memcpy.c}

GCC 4.9.1の最適化によるものです。

\lstinputlisting[caption=\Optimizing GCC 4.9.1 x64,style=customasmx86]{patterns/19_SIMD/memcpy_GCC49_x64_O3_JA.s}
