\subsubsection{x86}

\myindex{x86!\Instructions!LOOP}

Il y a une instruction \LOOP spéciale en x86 qui teste le contenu du registre \ECX
et si il est différent de 0, le \glslink{decrement}{décrémente} et continue l'exécution
au label de l'opérande \LOOP.
Probablement que cette instruction n'est pas très pratique, et il n'y a aucun compilateur
moderne qui la génère automatiquement.
Donc, si vous la rencontrez dans du code, il est probable qu'il s'agisse de code
assembleur écrit manuellement.

\par

En \CCpp les boucles sont en général construites avec une déclaration \TT{for()},
\TT{while()} ou \TT{do/while()}.

Commençons avec \TT{for()}.
\myindex{\CLanguageElements!for}

Cette déclaration définit l'initialisation de la boucle (met le compteur à sa valeur
initiale), la condition de boucle (est-ce que le compteur est plus grand qu'une limite?),
qu'est-ce qui est fait à chaque itération (\glslink{increment}{incrémenter}/\glslink{decrement}{décrémenter})
et bien sûr le corps de la boucle.

\lstinputlisting[style=customc]{patterns/09_loops/simple/loops_1_FR.c}

Le code généré consiste également en quatre parties.

Commençons avec un exemple simple:

\lstinputlisting[label=loops_src,style=customc]{patterns/09_loops/simple/loops_2.c}

Résultat (MSVC 2010):

\lstinputlisting[caption=MSVC 2010,style=customasmx86]{patterns/09_loops/simple/1_MSVC_FR.asm}

Comme nous le voyons, rien de spécial.

GCC 4.4.1 génère presque le même code, avec une différence subtile:

\lstinputlisting[caption=GCC 4.4.1,style=customasmx86]{patterns/09_loops/simple/1_GCC_FR.asm}

Maintenant, regardons ce que nous obtenons avec l'optimisation  (\TT{\Ox}):

\lstinputlisting[caption=\Optimizing MSVC,style=customasmx86]{patterns/09_loops/simple/1_MSVC_Ox.asm}

Ce qui se passe alors, c'est que l'espace pour la variable $i$ n'est plus alloué
sur la pile locale, mais utilise un registre individuel pour cela, \ESI. Ceci est
possible pour ce genre de petites fonctions, où il n'y a pas beaucoup de variables
locales.

Il est très important que la fonction \ttf ne modifie pas la valeur de \ESI.
Notre compilateur en est sûr ici.
Et si le compilateur décide d'utiliser le registre \ESI aussi dans la fonction \ttf,
sa valeur devra être sauvegardée lors du prologue de la fonction et restaurée lors
de son épilogue, presque comme dans notre listing: notez les \TT{PUSH ESI/POP ESI}
au début et à la fin de la fonction.

Essayons GCC 4.4.1 avec l'optimisation la plus performante (option \Othree):

\lstinputlisting[caption=GCC 4.4.1 \Optimizing,style=customasmx86]{patterns/09_loops/simple/1_GCC_O3.asm}

\myindex{Loop unwinding}

Hé, GCC a juste déroulé notre boucle.

Le \glslink{loop unwinding}{déroulement de boucle} est un avantage lorsqu'il n'y a pas
beaucoup d'itérations et que nous pouvons économiser du temps d'exécution en supprimant
les instructions de gestion de la boucle.
D'un autre côté, le code est étonnement plus gros.

Dérouler des grandes boucles n'est pas recommandé de nos jours, car les grosses fonctions
ont une plus grande empreinte sur le cache%
%
\footnote{Un très bon article à ce sujet: \DrepperMemory.
D'autres recommandations sur l'expansion des boucles d'Intel sont ici:
\InSqBrackets{\IntelOptimization 3.4.1.7}.}.

Ok, augmentons la valeur maximale de la variable $i$ à 100 et essayons à nouveau.
GCC donne:

\lstinputlisting[caption=GCC,style=customasmx86]{patterns/09_loops/simple/2_GCC_FR.asm}

C'est assez similaire à ce que MSVC 2010 génère avec l'optimisation (\Ox), avec l'exception
que le registre \EBX est utilisé pour la variable $i$.

GCC est sûr que ce registre ne sera pas modifié à l'intérieur de la fonction \ttf,
et si il l'était, il serait sauvé dans le prologue de la fonction et restauré dans
l'épilogue, tout comme dans la fonction \main.

\clearpage
\subsubsection{x86: \olly}
\myindex{\olly}

Compilons notre exemple dans MSVC 2010 avec les options \Ox et \Obzero, puis chargeons
le dans \olly.

Il semble qu'\olly soit capable de détecter des boucles simples et les affiche entre
parenthèses, par commodité.

\begin{figure}[H]
\centering
\myincludegraphics{patterns/09_loops/simple/olly1.png}
\caption{\olly: début de \main}
\label{fig:loops_olly_1}
\end{figure}

En traçant (F8~--- \stepover) nous voyons \ESI 
s'\glslink{increment}{incrémenter}.
Ici, par exemple, $ESI=i=6$:

\begin{figure}[H]
\centering
\myincludegraphics{patterns/09_loops/simple/olly2.png}
\caption{\olly: le corps de la boucle vient de s'exécuter avec $i=6$}
\label{fig:loops_olly_2}
\end{figure}

9 est la dernière valeur de la boucle.
C'est pourquoi \JL ne s'exécute pas après l'\glslink{increment}{incrémentation},
et que le fonction se termine.

\begin{figure}[H]
\centering
\myincludegraphics{patterns/09_loops/simple/olly3.png}
\caption{\olly: $ESI=10$, fin de la boucle}
\label{fig:loops_olly_3}
\end{figure}

\subsubsection{x86: tracer}
\myindex{tracer}

Comme nous venons de le voir, il n'est pas très commode de tracer manuellement dans
le débogueur.
C'est pourquoi nous allons essayer \tracer.

Nous ouvrons dans \IDA l'exemple compilé, trouvons l'adresse de l'instruction \INS{PUSH ESI}
(qui passe le seul argument à \ttf), qui est \TT{0x401026} dans ce cas et nous lançons
le \tracer:

\begin{lstlisting}
tracer.exe -l:loops_2.exe bpx=loops_2.exe!0x00401026
\end{lstlisting}

\TT{BPX} met juste un point d'arrêt à l'adresse et \tracer va alors afficher l'état
des registres.

Voici ce que l'on voit dans \TT{tracer.log}:

\lstinputlisting{patterns/09_loops/simple/tracer.log}

Nous voyons comment la valeur du registre \ESI change de 2 à 9.

Encore plus que ça, \tracer peut collecter les valeurs des registres pour toutes
les adresses dans la fonction. C'est appelé \emph{trace} ici. Chaque instruction est
tracée, toutes les valeurs intéressantes des registres sont enregistrées.

Ensuite, un script \IDA\ .idc est généré, qui ajoute des commentaires.
Donc, dans \IDA, nous avons appris que l'adresse de la fonction \main est \TT{0x00401020}
et nous lançons:

\begin{lstlisting}
tracer.exe -l:loops_2.exe bpf=loops_2.exe!0x00401020,trace:cc
\end{lstlisting}

\TT{BPF} signifie mettre un point d'arrêt sur la fonction.

Comme résultat, nous obtenons les scripts \TT{loops\_2.exe.idc} et \TT{loops\_2.exe\_clear.idc}.

\clearpage
Nous chargeons \TT{loops\_2.exe.idc} dans \IDA et voyons:

\begin{figure}[H]
\centering
\myincludegraphics{patterns/09_loops/simple/IDA_tracer_cc.png}
\caption{\IDA avec le script .idc chargé}
\label{fig:loops_IDA_tracer}
\end{figure}

Nous voyons que \ESI varie de 2 à 9 au début du corps de boucle, mais de 3 à
0xA (10) après l'incrément.
Nous voyons aussi que \main se termine avec 0 dans \EAX.

\tracer génère également \TT{loops\_2.exe.txt}, qui contient des informations sur
le nombre de fois qu'une instruction a été exécutée et les valeurs du registre:

\lstinputlisting[caption=loops\_2.exe.txt]{patterns/09_loops/simple/loops_2.exe.txt}
\myindex{\GrepUsage}
Nous pouvons utiliser grep ici.

