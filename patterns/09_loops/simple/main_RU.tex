\subsection{Простой пример}

% subsections
\EN{\subsubsection{x86}

\myindex{x86!\Instructions!LOOP}

There is a special \LOOP instruction in x86 instruction set for checking the value in register \ECX and 
if it is not 0, to \gls{decrement} \ECX
and pass control flow to the label in the \LOOP operand. 
Probably this instruction is not very convenient, and there are no any modern compilers which emit it automatically.
So, if you see this instruction somewhere in code, it is most likely that this is a manually written piece 
of assembly code.

\par

In \CCpp loops are usually constructed using \TT{for()}, \TT{while()} or \TT{do/while()} statements.

Let's start with \TT{for()}.
\myindex{\CLanguageElements!for}

This statement defines loop initialization (set loop counter to initial value), 
loop condition (is the counter bigger than a limit?), what is performed at each iteration (\gls{increment}/\gls{decrement})
and of course loop body.

\lstinputlisting[style=customc]{patterns/09_loops/simple/loops_1_EN.c}

The generated code is consisting of four parts as well.

Let's start with a simple example:

\lstinputlisting[label=loops_src,style=customc]{patterns/09_loops/simple/loops_2.c}

The result (MSVC 2010):

\lstinputlisting[caption=MSVC 2010,style=customasmx86]{patterns/09_loops/simple/1_MSVC_EN.asm}

As we see, nothing special.

GCC 4.4.1 emits almost the same code, with one subtle difference:

\lstinputlisting[caption=GCC 4.4.1,style=customasmx86]{patterns/09_loops/simple/1_GCC_EN.asm}

Now let's see what we get with optimization turned on (\TT{\Ox}):

\lstinputlisting[caption=\Optimizing MSVC,style=customasmx86]{patterns/09_loops/simple/1_MSVC_Ox.asm}

What happens here is that space for the $i$ variable is not allocated in the local stack anymore,
but uses an individual register for it, \ESI.
This is possible in such small functions where there aren't many local variables.

One very important thing is that the \ttf function must not change the value in \ESI.
Our compiler is sure here. 
And if the compiler decides to use the \ESI register in \ttf too, its value would have to be saved 
at the function's prologue and restored at the function's epilogue,
almost like in our listing: please note \TT{PUSH ESI/POP ESI}
at the function start and end.

Let's try GCC 4.4.1 with maximal optimization turned on (\Othree option):

\lstinputlisting[caption=\Optimizing GCC 4.4.1,style=customasmx86]{patterns/09_loops/simple/1_GCC_O3.asm}

\myindex{Loop unwinding}

Huh, GCC just unwound our loop.

\Gls{loop unwinding} has an advantage in the cases when there aren't much iterations and 
we could cut some execution time by removing all loop support instructions. 
On the other side, the resulting code is obviously larger.

Big unrolled loops are not recommended in modern times, because bigger functions
may require bigger cache footprint%
%
\footnote{A very good article about it: \DrepperMemory.
Another recommendations about loop unrolling from Intel are here: 
\InSqBrackets{\IntelOptimization 3.4.1.7}.}.

OK, let's increase the maximum value of the $i$ variable to 100 and try again. GCC does:

\lstinputlisting[caption=GCC,style=customasmx86]{patterns/09_loops/simple/2_GCC_EN.asm}

It is quite similar to what MSVC 2010 with optimization (\Ox) produce, 
with the exception that the \EBX register is allocated for the $i$ variable.

GCC is sure this register will not be modified inside of the \ttf function, 
and if it will, it will be saved at the function prologue and restored at epilogue, 
just like here in the \main function.

\clearpage
\subsubsection{x86: \olly}
\myindex{\olly}

Let's compile our example in MSVC 2010 with \Ox and \Obzero 
options and load it into \olly.

It seems that \olly is able to detect simple loops and show them in square brackets, for convenience:

\begin{figure}[H]
\centering
\myincludegraphics{patterns/09_loops/simple/olly1.png}
\caption{\olly: \main begin}
\label{fig:loops_olly_1}
\end{figure}

By tracing (F8~--- \stepover) we see \ESI 
\glslink{increment}{incrementing}.
Here, for instance, $ESI=i=6$:

\begin{figure}[H]
\centering
\myincludegraphics{patterns/09_loops/simple/olly2.png}
\caption{\olly: loop body just executed with $i=6$}
\label{fig:loops_olly_2}
\end{figure}

9 is the last loop value.
That's why \JL is not triggering after the \gls{increment}, and the function will finish:

\begin{figure}[H]
\centering
\myincludegraphics{patterns/09_loops/simple/olly3.png}
\caption{\olly: $ESI=10$, loop end}
\label{fig:loops_olly_3}
\end{figure}

\subsubsection{x86: tracer}
\myindex{tracer}

As we might see, it is not very convenient to trace manually in the debugger.
That's a reason we will try \tracer.

We open compiled example in \IDA, find the address of the instruction \INS{PUSH ESI}
(passing the sole argument to \ttf), which is \TT{0x401026} for this case and we run the \tracer:

\begin{lstlisting}
tracer.exe -l:loops_2.exe bpx=loops_2.exe!0x00401026
\end{lstlisting}

\TT{BPX} just sets a breakpoint at the address and \tracer will then print the state of the registers.

In the \TT{tracer.log}, this is what we see:

\lstinputlisting{patterns/09_loops/simple/tracer.log}

We see how the value of \ESI register changes from 2 to 9.

Even more than that, the \tracer can collect register values for all addresses within the function.
This is called \IT{trace} there.
Every instruction gets traced, all interesting register values are recorded.

Then, an \IDA .idc-script is generated, that adds comments.
So, in the \IDA we've learned that the \main function address is \TT{0x00401020} and we run:

\begin{lstlisting}
tracer.exe -l:loops_2.exe bpf=loops_2.exe!0x00401020,trace:cc
\end{lstlisting}

\TT{BPF} stands for set breakpoint on function.

As a result, we get the \TT{loops\_2.exe.idc} and \TT{loops\_2.exe\_clear.idc} scripts.

\clearpage
We load \TT{loops\_2.exe.idc} into \IDA and see:

\begin{figure}[H]
\centering
\myincludegraphics{patterns/09_loops/simple/IDA_tracer_cc.png}
\caption{\IDA with .idc-script loaded}
\label{fig:loops_IDA_tracer}
\end{figure}

We see that \ESI can be from 2 to 9 at the start of the loop body,
but from 3 to 0xA (10) after the increment.
We can also see that \main is finishing with 0 in \EAX.

\tracer also generates \TT{loops\_2.exe.txt}, 
that contains information about how many times each instruction has been executed and
register values:

\lstinputlisting[caption=loops\_2.exe.txt]{patterns/09_loops/simple/loops_2.exe.txt}
\myindex{\GrepUsage}
We can use grep here.

}
\RU{\subsubsection{x86}

\myindex{x86!\Instructions!LOOP}
Для организации циклов в архитектуре x86 есть старая инструкция \LOOP. 
Она проверяет значение регистра \ECX и если оно не 0, делает \glslink{decrement}{декремент} \ECX 
и переход по метке, указанной в операнде. 
Возможно, эта инструкция не слишком удобная, потому что уже почти не бывает современных компиляторов, 
которые использовали бы её. Так что если вы видите где-то \LOOP, то с большой вероятностью это 
вручную написанный код на ассемблере.

Обычно, циклы на \CCpp создаются при помощи \TT{for()}, \TT{while()}, \TT{do/while()}.
Начнем с \TT{for()}.
\myindex{\CLanguageElements!for}
Это выражение описывает инициализацию, условие, операцию после каждой итерации\\
(\glslink{increment}{инкремент}/\glslink{decrement}{декремент}) и тело цикла.

\lstinputlisting[style=customc]{patterns/09_loops/simple/loops_1_RU.c}

Примерно так же, генерируемый код и будет состоять из этих четырех частей.
Возьмем пример:

\lstinputlisting[label=loops_src,style=customc]{patterns/09_loops/simple/loops_2.c}

Имеем в итоге (MSVC 2010):

\lstinputlisting[caption=MSVC 2010,style=customasmx86]{patterns/09_loops/simple/1_MSVC_RU.asm}

В принципе, ничего необычного.

GCC 4.4.1 выдает примерно такой же код, с небольшой разницей:

\lstinputlisting[caption=GCC 4.4.1,style=customasmx86]{patterns/09_loops/simple/1_GCC_RU.asm}

Интересно становится, если скомпилируем этот же код при помощи MSVC 2010 с включенной оптимизацией (\TT{\Ox}):

\lstinputlisting[caption=\Optimizing MSVC,style=customasmx86]{patterns/09_loops/simple/1_MSVC_Ox.asm}

Здесь происходит следующее: переменную $i$ компилятор не выделяет в локальном стеке, 
а выделяет целый регистр под нее: \ESI. 
Это возможно для маленьких функций, где мало локальных переменных.

В принципе, всё то же самое, только теперь одна важная особенность: 
\ttf не должна менять значение \ESI. 
Наш компилятор уверен в этом, а если бы и была необходимость использовать регистр \ESI в функции \ttf, 
то её значение сохранялось бы в стеке. Примерно так же как и в нашем листинге: 
обратите внимание на \TT{PUSH ESI/POP ESI} в начале и конце функции.

Попробуем GCC 4.4.1 с максимальной оптимизацией (\Othree):

\lstinputlisting[caption=\Optimizing GCC 4.4.1,style=customasmx86]{patterns/09_loops/simple/1_GCC_O3.asm}

\myindex{Loop unwinding}
Однако GCC просто \emph{развернул} цикл\footnote{\gls{loop unwinding} в англоязычной литературе}.

Делается это в тех случаях, когда итераций не слишком много (как в нашем примере)
и можно немного сэкономить время, убрав все инструкции, обеспечивающие цикл. 
В качестве обратной стороны медали, размер кода увеличился.

Использовать большие развернутые циклы в наше время не рекомендуется, потому что большие
функции требуют больше кэш-памяти%
\footnote{Очень хорошая статья об этом: \DrepperMemory.
А также о рекомендациях о развернутых циклах от Intel можно прочитать здесь: 
\InSqBrackets{\IntelOptimization 3.4.1.7}.}.

Увеличим максимальное значение $i$ в цикле до 100 и попробуем снова. GCC выдает:

\lstinputlisting[caption=GCC,style=customasmx86]{patterns/09_loops/simple/2_GCC_RU.asm}

Это уже похоже на то, что сделал MSVC 2010 в режиме оптимизации (\TT{\Ox}).
За исключением того, что под переменную $i$ будет выделен регистр \EBX.

GCC уверен, что этот регистр не будет 
модифицироваться внутри \ttf, а если вдруг это и придётся там сделать, то его значение будет сохранено 
в начале функции, прямо как в \main.

\clearpage
\subsubsection{x86: \olly}
\myindex{\olly}

Скомпилируем наш пример в MSVC 2010 с \Ox и \Obzero и загрузим в \olly.

Оказывается, \olly может обнаруживать простые циклы и показывать их в квадратных скобках, 
для удобства:

\begin{figure}[H]
\centering
\myincludegraphics{patterns/09_loops/simple/olly1.png}
\caption{\olly: начало \main}
\label{fig:loops_olly_1}
\end{figure}

Трассируя (F8~--- \stepover) мы видим, как \ESI увеличивается на 1.

Например, здесь $ESI=i=6$:

\begin{figure}[H]
\centering
\myincludegraphics{patterns/09_loops/simple/olly2.png}
\caption{\olly: тело цикла только что отработало с $i=6$}
\label{fig:loops_olly_2}
\end{figure}

9 это последнее значение цикла.
Поэтому \JL после \glslink{increment}{инкремента} не срабатывает и функция заканчивается:

\begin{figure}[H]
\centering
\myincludegraphics{patterns/09_loops/simple/olly3.png}
\caption{\olly: $ESI=10$, конец цикла}
\label{fig:loops_olly_3}
\end{figure}

\subsubsection{x86: tracer}
\myindex{tracer}

Как видно, трассировать вручную цикл в отладчике --- это не очень удобно.
Поэтому попробуем \tracer.
Открываем скомпилированный пример в \IDA, находим там адрес инструкции \INS{PUSH ESI}
(передающей единственный аргумент в \ttf),
а это \TT{0x401026} в нашем случае и запускаем \tracer:

\begin{lstlisting}
tracer.exe -l:loops_2.exe bpx=loops_2.exe!0x00401026
\end{lstlisting}

Опция \TT{BPX} просто ставит точку останова по адресу и затем tracer будет выдавать состояние регистров.
В \TT{tracer.log} после запуска я вижу следующее:

\lstinputlisting{patterns/09_loops/simple/tracer.log}

Видно, как значение \ESI последовательно изменяется от 2 до 9.
И даже более того, в \tracer можно собирать значения регистров по всем адресам внутри функции.

Там это называется \emph{trace}.
Каждая инструкция трассируется, значения самых интересных регистров запоминаются.
Затем генерируется .idc-скрипт для \IDA, который добавляет комментарии.
Итак, в \IDA я узнал что адрес \main это \TT{0x00401020} и запускаю:

\begin{lstlisting}
tracer.exe -l:loops_2.exe bpf=loops_2.exe!0x00401020,trace:cc
\end{lstlisting}

\TT{BPF} означает установить точку останова на функции.

Получаю в итоге скрипты \TT{loops\_2.exe.idc} и \TT{loops\_2.exe\_clear.idc}.

\clearpage
Загружаю \TT{loops\_2.exe.idc} в \IDA и увижу следующее:

\begin{figure}[H]
\centering
\myincludegraphics{patterns/09_loops/simple/IDA_tracer_cc.png}
\caption{\IDA с загруженным .idc-скриптом}
\label{fig:loops_IDA_tracer}
\end{figure}

Видно, что \ESI меняется от 2 до 9 в начале тела цикла, но после 
\glslink{increment}{инкремента} он в пределах [3..0xA].
Видно также, что функция \main заканчивается с 0 в \EAX.

\tracer также генерирует \TT{loops\_2.exe.txt}, 
содержащий адреса инструкций, сколько раз была исполнена
каждая и значения регистров:

\lstinputlisting[caption=loops\_2.exe.txt]{patterns/09_loops/simple/loops_2.exe.txt}
\myindex{\GrepUsage}
Так можно использовать grep.
}
\DE{\subsubsection{x86}

\myindex{x86!\Instructions!LOOP}
Es gibt einen speziellen \LOOP Befehl im x86 Befehlssatz, der den Wert des
Registers \ECX prüft und falls dieser ungleich 0 ist, dekrementiert und danach
die den control flow wieder an das Label des \LOOP Operanden übergibt.
Vermutlich ist dieser Befehl nicht allzu geläufig und es gibt keine modernen
Compiler, welche ihn automatisch erzeugen. Wenn man also diesen Befehl irgendwo
im Code entdeckt, dann ist es äußerst wahrscheinlich, dass es sich um ein
handgeschriebenes Stück Assemblercode handelt.

\par

In \CCpp werden Schleifen normalerweise mittels \TT{for()}-, \TT{while()}- oder
\TT{do/while()}-Ausdrücken erzeugt.

Starten wir mit \TT{for()}.

\myindex{\CLanguageElements!for}

Dieser Ausdruck definiert eine Schleifeninitialisierung (setzt den Zähler auf
einen Startwert), definiert eine Schleifenbedingung (ist der Zähler größer als
ein Grenzwert?), legt fest, was in jedem Durchlauf
(\glslink{increment}{Inkrement}/\glslink{decrement}{Dekrement}) geschieht und umschließt einen
Schleifenkörper.


\lstinputlisting[style=customc]{patterns/09_loops/simple/loops_1_EN.c}

Der erzeugte Code besteht ebenfalls aus vier Teilen.

Beginnen wir mit einem einfachen Beispiel:


\lstinputlisting[label=loops_src,style=customc]{patterns/09_loops/simple/loops_2.c}

Ergebnis (MSVC 2010):

\lstinputlisting[caption=MSVC 2010,style=customasmx86]{patterns/09_loops/simple/1_MSVC_EN.asm}

Hier gibt es nichts Besonderes zu sehen.

GCC 4.4.1 erzeugt einen fast identischen Code mit nur einen kleinen Unterschied:

\lstinputlisting[caption=GCC 4.4.1,style=customasmx86]{patterns/09_loops/simple/1_GCC_DE.asm}

Schauen wir uns nun an, was wir erhalten, wenn wir die Optimierung aktivieren
(\TT{\Ox}):

\lstinputlisting[caption=\Optimizing MSVC,style=customasmx86]{patterns/09_loops/simple/1_MSVC_Ox.asm}

Was hier passiert ist, dass der Speicherplatz für die $i$ Variable nicht mehr
auf dem lokalen Stack bereitgestellt wird, sondern das extra ein Register, \ESI,
hierfür verwendet wird. Dies ist bei derartig kleinen Funktionen möglich, wenn
nicht zu viele lokalen Variablen existieren.

Wichtig ist, dass die \ttf Funktion den Wert im Register \ESI nicht verändern
darf. Unser Compiler ist sich dieser Sache hier sicher. 
Und falls der Compiler entscheidet, das \ESI auch innerhalb der Funktion \ttf zu
verwenden, würde der Wert des Registers im Funktionsprolog gesichert und im
Funktionsepilog wiederhergestellt werden; fast genauso wie im folgenden Listing.
Man beachte das \TT{PUSH ESI/POP ESI} bei Funktionsbeginn und -ende. 
 
Probieren wir aus, was GCC 4.4.1 mit maximaler Optimierung (\Othree option)
liefert:


\lstinputlisting[caption=\Optimizing GCC 4.4.1,style=customasmx86]{patterns/09_loops/simple/1_GCC_O3.asm}

\myindex{Loop unwinding}

Aha, GCC hat unsere Schleife unrolled (d.h. ausgerollt).

\Gls{loop unwinding} hat Vorteile in Fällen, in denen es nicht viele
Schleifendurchläufe gibt und Ausführungszeit durch das Weglassen der
Befehle für die Kontrollstrukturen der Schleife gewonnen werden kann. 
Andererseits ist der erzeugte Code natürlich deutlich länger.

Große Schleifen zu \textit{unrollen} ist heutzutage nicht empfehlenswert, denn
größere Funktionen erfordern einen größeren Cache-Fußabdruck.%
%
\footnote{Ein hervorragender Artikel zum Thema: \DrepperMemory.
Für weitere Empfehlungen von Intel zum Unrolling siehe hier: 
\InSqBrackets{\IntelOptimization 3.4.1.7}.}.

Gut, nun wollen wir den Höchstwert der Variable $i$ auf 100 setzen und
kompilieren erneut. GCC liefert:

\lstinputlisting[caption=GCC,style=customasmx86]{patterns/09_loops/simple/2_GCC_DE.asm}

Das Ergebnis ist sehr ähnlich dem, das MSVC 2010 mit Optimierung (\Ox) erzeugt,
mit der Ausnahme, dass das \EBX Register für die Variable $i$ verwendet wird.

GCC ist sicher, dass das Register innerhalb der \ttf Funktion nicht verändert
wird und sollte dies doch der Fall sein, dass es im Funktionsprolog gesichert
und im Funktionsepilog wiederhergestellt werden wird, genau wie hier in der
\main Funktion.

\clearpage
\subsubsection{x86: \olly}
\myindex{\olly}

Wir kompilieren unser Beispiel in MSVC 2010 mit den Optionen \Ox und \Obzero und
laden es in \olly.

Es scheint, dass \olly in der Lage ist, einfache Schleifen zu erkennen und in
eckigen Klammern darzustellen, um die Übersichtlichkeit zu erhöhen:

\begin{figure}[H]
\centering
\myincludegraphics{patterns/09_loops/simple/olly1.png}
\caption{\olly: \main Einstieg}
\label{fig:loops_olly_1}
\end{figure}

Verfolgen mit (F8~--- \stepover) zeigt \ESI 
\glslink{increment}{incrementing}.
Hier ist zum Beispiel, $ESI=i=6$:

\begin{figure}[H]
\centering
\myincludegraphics{patterns/09_loops/simple/olly2.png}
\caption{\olly: Schleifenkörper wird gerade ausgeführt für $i=6$}
\label{fig:loops_olly_2}
\end{figure}

9 ist der letzte Wert in der Schleife
Deshalb triggert \JL nach der \glslink{increment}{inkrement}-Anweisung nicht und die Funktion
wird beendet:

\begin{figure}[H]
\centering
\myincludegraphics{patterns/09_loops/simple/olly3.png}
\caption{\olly: $ESI=10$, Ende der Schleife}
\label{fig:loops_olly_3}
\end{figure}

\subsubsection{x86: tracer}
\myindex{tracer}

Wie wir bemerken ist es nicht sonderlich komfortabel, Werte im Debugger manuell
nachzuverfolgen. Aus diesem Grund probieren wir \tracer aus.

Wir öffnen das kompilierte Beispiel in \IDA, finden die Adresse mit dem Befehl
\INS{PUSH ESI} (das einzige Argument an \ttf übergebend), welche hier
\TT{0x401026} ist und aktivieren den \tracer:


\begin{lstlisting}
tracer.exe -l:loops_2.exe bpx=loops_2.exe!0x00401026
\end{lstlisting}

\TT{BPX} setzt einen Breakpoint an der Adresse und der Tracer zeigt uns den
momentanen Status der Register an. In \TT{tracer.log} sehen wir das Folgende:

\lstinputlisting{patterns/09_loops/simple/tracer.log}

Wir sehen wie der Wert des \ESI Registers sich schrittweise von 2 zu 9
verändert. 

Mehr noch, der \tracer kann alle Registerwerte für alle Adressen innerhalb der
Funktion zusammensammeln. Dies wird hier mit \emph{trace} (dt. Nachverfolgung)
bezeichnet. Jeder Befehl wird verfolgt, alle interessanten Registerwerte werden
aufgezeichnet.

Danach die ein \IDA .idc-script erzeugt, das Kommentare hinzufügt. Wir haben
also herausgefunden, dass die Adresse der \main Funktion \TT{0x00401020} ist und
wir führen nun das Folgende aus:

\begin{lstlisting}
tracer.exe -l:loops_2.exe bpf=loops_2.exe!0x00401020,trace:cc
\end{lstlisting}

\TT{BPF} setzt einen Breakpoint auf eine Funktion.

Als Ergebnis erahlten wir die Skripte \TT{loops\_2.exe.idc} und
\TT{loops\_2.exe\_clear.idc}.

\clearpage
Wir laden \TT{loops\_2.exe.idc} in \IDA und erhalten:

\begin{figure}[H]
\centering
\myincludegraphics{patterns/09_loops/simple/IDA_tracer_cc.png}
\caption{\IDA mit geladenem .idc-script}
\label{fig:loops_IDA_tracer}
\end{figure}

Wir sehen, dass der Wert von \ESI zu Beginn der Schleife zwischen 2 und 9 und
nach dem Inkrement zwischen 3 und 0xA (10) liegt. Wir sehen auch, dass die
Funktion \main mit dem Rückgabewert 0 in \EAX terminiert.

\tracer erzeugt ebenfalls die Datei \TT{loops\_2.exe.txt}, welche Informationen
darüber enthält, welcher Befehl wie oft ausgeführt wurde, sowie zugehörige
Registerwerte:

\lstinputlisting[caption=loops\_2.exe.txt]{patterns/09_loops/simple/loops_2.exe.txt}
\myindex{\GrepUsage}
An dieser Stelle können wir grep verwenden.

}
\FR{\subsubsection{x86}

\myindex{x86!\Instructions!LOOP}

Il y a une instruction \LOOP spéciale en x86 qui teste le contenu du registre \ECX
et si il est différent de 0, le \glslink{decrement}{décrémente} et continue l'exécution
au label de l'opérande \LOOP.
Probablement que cette instruction n'est pas très pratique, et il n'y a aucun compilateur
moderne qui la génère automatiquement.
Donc, si vous la rencontrez dans du code, il est probable qu'il s'agisse de code
assembleur écrit manuellement.

\par

En \CCpp les boucles sont en général construites avec une déclaration \TT{for()},
\TT{while()} ou \TT{do/while()}.

Commençons avec \TT{for()}.
\myindex{\CLanguageElements!for}

Cette déclaration définit l'initialisation de la boucle (met le compteur à sa valeur
initiale), la condition de boucle (est-ce que le compteur est plus grand qu'une limite?),
qu'est-ce qui est fait à chaque itération (\glslink{increment}{incrémenter}/\glslink{decrement}{décrémenter})
et bien sûr le corps de la boucle.

\lstinputlisting[style=customc]{patterns/09_loops/simple/loops_1_FR.c}

Le code généré consiste également en quatre parties.

Commençons avec un exemple simple:

\lstinputlisting[label=loops_src,style=customc]{patterns/09_loops/simple/loops_2.c}

Résultat (MSVC 2010):

\lstinputlisting[caption=MSVC 2010,style=customasmx86]{patterns/09_loops/simple/1_MSVC_FR.asm}

Comme nous le voyons, rien de spécial.

GCC 4.4.1 génère presque le même code, avec une différence subtile:

\lstinputlisting[caption=GCC 4.4.1,style=customasmx86]{patterns/09_loops/simple/1_GCC_FR.asm}

Maintenant, regardons ce que nous obtenons avec l'optimisation  (\TT{\Ox}):

\lstinputlisting[caption=\Optimizing MSVC,style=customasmx86]{patterns/09_loops/simple/1_MSVC_Ox.asm}

Ce qui se passe alors, c'est que l'espace pour la variable $i$ n'est plus alloué
sur la pile locale, mais utilise un registre individuel pour cela, \ESI. Ceci est
possible pour ce genre de petites fonctions, où il n'y a pas beaucoup de variables
locales.

Il est très important que la fonction \ttf ne modifie pas la valeur de \ESI.
Notre compilateur en est sûr ici.
Et si le compilateur décide d'utiliser le registre \ESI aussi dans la fonction \ttf,
sa valeur devra être sauvegardée lors du prologue de la fonction et restaurée lors
de son épilogue, presque comme dans notre listing: notez les \TT{PUSH ESI/POP ESI}
au début et à la fin de la fonction.

Essayons GCC 4.4.1 avec l'optimisation la plus performante (option \Othree):

\lstinputlisting[caption=GCC 4.4.1 \Optimizing,style=customasmx86]{patterns/09_loops/simple/1_GCC_O3.asm}

\myindex{Loop unwinding}

Hé, GCC a juste déroulé notre boucle.

Le \glslink{loop unwinding}{déroulement de boucle} est un avantage lorsqu'il n'y a pas
beaucoup d'itérations et que nous pouvons économiser du temps d'exécution en supprimant
les instructions de gestion de la boucle.
D'un autre côté, le code est étonnement plus gros.

Dérouler des grandes boucles n'est pas recommandé de nos jours, car les grosses fonctions
ont une plus grande empreinte sur le cache%
%
\footnote{Un très bon article à ce sujet: \DrepperMemory.
D'autres recommandations sur l'expansion des boucles d'Intel sont ici:
\InSqBrackets{\IntelOptimization 3.4.1.7}.}.

Ok, augmentons la valeur maximale de la variable $i$ à 100 et essayons à nouveau.
GCC donne:

\lstinputlisting[caption=GCC,style=customasmx86]{patterns/09_loops/simple/2_GCC_FR.asm}

C'est assez similaire à ce que MSVC 2010 génère avec l'optimisation (\Ox), avec l'exception
que le registre \EBX est utilisé pour la variable $i$.

GCC est sûr que ce registre ne sera pas modifié à l'intérieur de la fonction \ttf,
et si il l'était, il serait sauvé dans le prologue de la fonction et restauré dans
l'épilogue, tout comme dans la fonction \main.

\clearpage
\subsubsection{x86: \olly}
\myindex{\olly}

Compilons notre exemple dans MSVC 2010 avec les options \Ox et \Obzero, puis chargeons
le dans \olly.

Il semble qu'\olly soit capable de détecter des boucles simples et les affiche entre
parenthèses, par commodité.

\begin{figure}[H]
\centering
\myincludegraphics{patterns/09_loops/simple/olly1.png}
\caption{\olly: début de \main}
\label{fig:loops_olly_1}
\end{figure}

En traçant (F8~--- \stepover) nous voyons \ESI 
s'\glslink{increment}{incrémenter}.
Ici, par exemple, $ESI=i=6$:

\begin{figure}[H]
\centering
\myincludegraphics{patterns/09_loops/simple/olly2.png}
\caption{\olly: le corps de la boucle vient de s'exécuter avec $i=6$}
\label{fig:loops_olly_2}
\end{figure}

9 est la dernière valeur de la boucle.
C'est pourquoi \JL ne s'exécute pas après l'\glslink{increment}{incrémentation},
et que le fonction se termine.

\begin{figure}[H]
\centering
\myincludegraphics{patterns/09_loops/simple/olly3.png}
\caption{\olly: $ESI=10$, fin de la boucle}
\label{fig:loops_olly_3}
\end{figure}

\subsubsection{x86: tracer}
\myindex{tracer}

Comme nous venons de le voir, il n'est pas très commode de tracer manuellement dans
le débogueur.
C'est pourquoi nous allons essayer \tracer.

Nous ouvrons dans \IDA l'exemple compilé, trouvons l'adresse de l'instruction \INS{PUSH ESI}
(qui passe le seul argument à \ttf), qui est \TT{0x401026} dans ce cas et nous lançons
le \tracer:

\begin{lstlisting}
tracer.exe -l:loops_2.exe bpx=loops_2.exe!0x00401026
\end{lstlisting}

\TT{BPX} met juste un point d'arrêt à l'adresse et \tracer va alors afficher l'état
des registres.

Voici ce que l'on voit dans \TT{tracer.log}:

\lstinputlisting{patterns/09_loops/simple/tracer.log}

Nous voyons comment la valeur du registre \ESI change de 2 à 9.

Encore plus que ça, \tracer peut collecter les valeurs des registres pour toutes
les adresses dans la fonction. C'est appelé \emph{trace} ici. Chaque instruction est
tracée, toutes les valeurs intéressantes des registres sont enregistrées.

Ensuite, un script \IDA\ .idc est généré, qui ajoute des commentaires.
Donc, dans \IDA, nous avons appris que l'adresse de la fonction \main est \TT{0x00401020}
et nous lançons:

\begin{lstlisting}
tracer.exe -l:loops_2.exe bpf=loops_2.exe!0x00401020,trace:cc
\end{lstlisting}

\TT{BPF} signifie mettre un point d'arrêt sur la fonction.

Comme résultat, nous obtenons les scripts \TT{loops\_2.exe.idc} et \TT{loops\_2.exe\_clear.idc}.

\clearpage
Nous chargeons \TT{loops\_2.exe.idc} dans \IDA et voyons:

\begin{figure}[H]
\centering
\myincludegraphics{patterns/09_loops/simple/IDA_tracer_cc.png}
\caption{\IDA avec le script .idc chargé}
\label{fig:loops_IDA_tracer}
\end{figure}

Nous voyons que \ESI varie de 2 à 9 au début du corps de boucle, mais de 3 à
0xA (10) après l'incrément.
Nous voyons aussi que \main se termine avec 0 dans \EAX.

\tracer génère également \TT{loops\_2.exe.txt}, qui contient des informations sur
le nombre de fois qu'une instruction a été exécutée et les valeurs du registre:

\lstinputlisting[caption=loops\_2.exe.txt]{patterns/09_loops/simple/loops_2.exe.txt}
\myindex{\GrepUsage}
Nous pouvons utiliser grep ici.

}
\JA{\subsubsection{x86}

\myindex{x86!\Instructions!LOOP}

x86命令セットには \ECX というレジスタが値をチェックする特別な \LOOP 命令になります。そして
0でなければ、\gls{decrement} \ECX し、 \LOOP オペランドのラベルに制御フローを渡します。
おそらくこの命令はあまり便利ではなく、自動的にそれを発行する最新のコンパイラはありません。
したがって、コードのどこかでこの命令を見ると、これは手作業で書かれたアセンブリコードである可能性が高いです。

\par

\CCpp のループでは通常\TT{for()}、\TT{while()} または \TT{do/while()}文を使用して構成されます。

\TT{for()}を使ってみましょう。
\myindex{\CLanguageElements!for}

この文はループの初期化を定義し(ループカウンタを初期値にセット)、
ループ条件(がリミットより大きいか?)が各イテレーション(\gls{increment}/\gls{decrement})で実行され、
ループボディも当然実行されます。

\lstinputlisting[style=customc]{patterns/09_loops/simple/loops_1_JA.c}

生成されたコードは4つの部分で構成されています。

簡単な例から始めましょう:

\lstinputlisting[label=loops_src,style=customc]{patterns/09_loops/simple/loops_2.c}

結果(MSVC 2010):

\lstinputlisting[caption=MSVC 2010,style=customasmx86]{patterns/09_loops/simple/1_MSVC_JA.asm}

我々が見るように、特別なものはありません。

GCC 4.4.1はほぼ同じコードを出力しますが、微妙な違いが1つあります:

\lstinputlisting[caption=GCC 4.4.1,style=customasmx86]{patterns/09_loops/simple/1_GCC_JA.asm}

最適化を有効にして(\TT{\Ox})取得した内容を見てみましょう。

\lstinputlisting[caption=\Optimizing MSVC,style=customasmx86]{patterns/09_loops/simple/1_MSVC_Ox.asm}

ここで起こるのは、 $i$ 変数のスペースがローカルスタックにはもう割り当てられず、
\ESI のための個別のレジスタを使用するということです。 
これは、ローカル変数があまりないような小さな関数で可能です。

とても重要なことは、 \ttf 関数が \ESI の値を変更してはならないことです。 
私たちのコンパイラは確かにそうしています。
コンパイラが \ttf で \ESI レジスタを使用することを決定した場合、
その値は関数のプロローグに保存され、関数のエピローグで復元されなければなりません。
リストのようになります。関数の開始と終了での\TT{PUSH ESI/POP ESI}に注意してください。

最適化を最大にしてGCC 4.4.1を試してみましょう( \Othree オプション):

\lstinputlisting[caption=\Optimizing GCC 4.4.1,style=customasmx86]{patterns/09_loops/simple/1_GCC_O3.asm}

\myindex{Loop unwinding}

えっ、GCCは単に私たちのループを巻き戻してしまいました。

\Gls{loop unwinding} は反復回数が多くなく、
すべてのループサポート命令を削除することで実行時間を短縮できる場合に利点があります。 
逆の場合では、明らかにコードが大きくなります。

大規模な関数は大量のキャッシュフットプリント%
%
\footnote{非常によい記事: \DrepperMemory 
インテルのループアンローリングに関するその他の推奨事項はこちら:
\InSqBrackets{\IntelOptimization 3.4.1.7}}
を必要とする可能性があるため、大きなアンロールループは現代では推奨されません。

OK、 $i$ 変数の最大値を100に増やして、もう一度試してみましょう。 GCCの結果は以下の通り:

\lstinputlisting[caption=GCC,style=customasmx86]{patterns/09_loops/simple/2_GCC_JA.asm}

これは、EBXレジスタが $i$ 変数に割り当てられていることを除いて、
最適化ありのMSVC 2010(\Ox)と非常によく似ています。

GCCはこのレジスタが \ttf 関数の内部で変更されないことをわかっています。
もしそうであれば、 \main 関数のように関数プロローグに保存され、エピローグで復元されます。

\clearpage
\subsubsection{x86: \olly}
\myindex{\olly}

\Ox と \Obzero オプションを使用してMSVC 2010のサンプルをコンパイルし、 
\olly にロードしてみましょう。

\olly は単純なループを検出し、便宜上、角カッコで表示してくれます。

\begin{figure}[H]
\centering
\myincludegraphics{patterns/09_loops/simple/olly1.png}
\caption{\olly: \main 開始}
\label{fig:loops_olly_1}
\end{figure}

By tracing (F8~--- \stepover) we see \ESI 
\glslink{increment}{incrementing}.
Here, for instance, $ESI=i=6$:

トレースすることにより(F8。ステップオーバ)、ESIが増加することがわかります。 ここで、例えば、ESI = i = 6:

\begin{figure}[H]
\centering
\myincludegraphics{patterns/09_loops/simple/olly2.png}
\caption{\olly: ループボディが $i=6$ で実行}
\label{fig:loops_olly_2}
\end{figure}

9は最後のループ値です。
そのため、 \JL は\gls{increment}後に実行されず、関数は終了します。

\begin{figure}[H]
\centering
\myincludegraphics{patterns/09_loops/simple/olly3.png}
\caption{\olly: $ESI=10$、ループ終了}
\label{fig:loops_olly_3}
\end{figure}

\subsubsection{x86: tracer}
\myindex{tracer}

見てきたように、デバッガで手動でトレースするのはあまり便利ではありません。 
それがトレーサを試みる理由です。

コンパイルされたサンプルを \IDA で開き、命令\INS{PUSH ESI}( \ttf へ1つ引数を渡す)
のアドレスを見つけます。この場合は \TT{0x401026} で、トレーサを実行してみます。

\begin{lstlisting}
tracer.exe -l:loops_2.exe bpx=loops_2.exe!0x00401026
\end{lstlisting}

\TT{BPX} はアドレスにブレークポイントを設定するだけで、 \tracer はレジスタの状態を出力します。

\TT{tracer.log}では、このようになります。

\lstinputlisting{patterns/09_loops/simple/tracer.log}

\ESI レジスタの値が2から9に変化する様子を見ています。

\tracer はそれ以上にも、関数内のすべてのアドレスのレジスタ値を収集できます。 
これを\emph{trace}といいます。 
すべての命令がトレースされ、興味深いレジスタ値がすべて記録されます。

次に、コメントを追加する \IDA .idcスクリプトが生成されます。 
したがって、 \IDA では、 \main 関数のアドレスは\TT{0x00401020}であり、次のように実行されます。

\begin{lstlisting}
tracer.exe -l:loops_2.exe bpf=loops_2.exe!0x00401020,trace:cc
\end{lstlisting}

\TT{BPF}は、関数にブレークポイントを設定します。

その結果、\TT{loops\_2.exe.idc}および\TT{loops\_2.exe\_clear.idc}スクリプトが取得されます。

\clearpage
\TT{loops\_2.exe.idc}を \IDA にロードすると次のようになります。

\begin{figure}[H]
\centering
\myincludegraphics{patterns/09_loops/simple/IDA_tracer_cc.png}
\caption{.idc-scriptを \IDA でロードした}
\label{fig:loops_IDA_tracer}
\end{figure}

\ESI はループ本体の開始時には2から9、
インクリメント後は3から0xA(10)になります。
\main が \EAX 0で終了していることもわかります。

\tracer はまた、各命令が何回実行されたかに
関する情報とレジスタ値を含む\TT{loops\_2.exe.txt}も生成します。

\lstinputlisting[caption=loops\_2.exe.txt]{patterns/09_loops/simple/loops_2.exe.txt}
\myindex{\GrepUsage}
ここではgrepを使うことができます。
}


\EN{\subsubsection{ARM}

\myparagraph{\NonOptimizingKeilVI (\ARMMode)}

\lstinputlisting[label=Keil_number_sign,style=customasmARM]{patterns/09_loops/simple/ARM/Keil_ARM_O0.asm}

Iteration counter $i$ is to be stored in the \Reg{4} register.
The \INS{MOV R4, \#2} instruction just initializes $i$.
The \INS{MOV R0, R4} and \INS{BL printing\_function} instructions
compose the body of the loop, the first instruction preparing the argument for 
\ttf function and the second calling the function.
\myindex{ARM!\Instructions!ADD}
The \INS{ADD R4, R4, \#1} instruction just adds 1 to the $i$ variable at each iteration.
\myindex{ARM!\Instructions!CMP}
\myindex{ARM!\Instructions!BLT}
\INS{CMP R4, \#0xA} compares $i$ with \TT{0xA} (10). 
The next instruction \INS{BLT} (\emph{Branch Less Than}) jumps if $i$ is less than 10.
Otherwise, 0 is to be written into \Reg{0} (since our function returns 0)
and function execution finishes.

\myparagraph{\OptimizingKeilVI (\ThumbMode)}

\lstinputlisting[style=customasmARM]{patterns/09_loops/simple/ARM/Keil_thumb_O3.asm}

Practically the same.

\myparagraph{\OptimizingXcodeIV (\ThumbTwoMode)}
\label{ARM_unrolled_loops}

\lstinputlisting[style=customasmARM]{patterns/09_loops/simple/ARM/xcode_thumb_O3.asm}

In fact, this was in my \ttf function:

\begin{lstlisting}[style=customc]
void printing_function(int i)
{
    printf ("%d\n", i);
};
\end{lstlisting}

\myindex{Unrolled loop}
\myindex{Inline code}
So, LLVM not just \emph{unrolled} the loop, 
but also \emph{inlined} my 
very simple function \ttf,
and inserted its body 8 times instead of calling it. 

This is possible when the function is so simple (like mine) and when it is not called too much (like here).

\myparagraph{ARM64: \Optimizing GCC 4.9.1}

\lstinputlisting[caption=\Optimizing GCC 4.9.1,style=customasmARM]{patterns/09_loops/simple/ARM/ARM64_GCC491_O3_EN.s}

\myparagraph{ARM64: \NonOptimizing GCC 4.9.1}

\lstinputlisting[caption=\NonOptimizing GCC 4.9.1 -fno-inline,style=customasmARM]{patterns/09_loops/simple/ARM/ARM64_GCC491_O0_EN.s}
}
\RU{\subsubsection{ARM}

\myparagraph{\NonOptimizingKeilVI (\ARMMode)}

\lstinputlisting[label=Keil_number_sign,style=customasmARM]{patterns/09_loops/simple/ARM/Keil_ARM_O0.asm}

Счетчик итераций $i$ будет храниться в регистре \Reg{4}.
Инструкция \INS{MOV R4, \#2} просто инициализирует $i$.
Инструкции \INS{MOV R0, R4} и \INS{BL printing\_function} составляют тело цикла. 
Первая инструкция готовит аргумент для функции, \ttf а вторая вызывает её.
\myindex{ARM!\Instructions!ADD}
Инструкция \INS{ADD R4, R4, \#1} прибавляет единицу к $i$ при каждой итерации.
\myindex{ARM!\Instructions!CMP}
\myindex{ARM!\Instructions!BLT}
\INS{CMP R4, \#0xA} сравнивает $i$ с \TT{0xA} (10). 
Следующая за ней инструкция \INS{BLT} (\IT{Branch Less Than}) совершит переход, если $i$ меньше чем 10.
В противном случае в \Reg{0} запишется 0 (потому что наша функция возвращает 0) 
и произойдет выход из функции.

\myparagraph{\OptimizingKeilVI (\ThumbMode)}

\lstinputlisting[style=customasmARM]{patterns/09_loops/simple/ARM/Keil_thumb_O3.asm}

Практически всё то же самое.

\myparagraph{\OptimizingXcodeIV (\ThumbTwoMode)}
\label{ARM_unrolled_loops}

\lstinputlisting[style=customasmARM]{patterns/09_loops/simple/ARM/xcode_thumb_O3.asm}

На самом деле, в моей функции \ttf было такое:

\begin{lstlisting}[style=customc]
void printing_function(int i)
{
    printf ("%d\n", i);
};
\end{lstlisting}

\myindex{Unrolled loop}
\myindex{Inline code}
Так что LLVM не только \IT{развернул} цикл, 
но также и представил мою очень простую функцию \ttf как \IT{inline-функцию},
и вставил её тело вместо цикла 8 раз. 
Это возможно, когда функция очень простая (как та что у меня) и когда
она вызывается не очень много раз, как здесь.

\myparagraph{ARM64: \Optimizing GCC 4.9.1}

\lstinputlisting[caption=\Optimizing GCC 4.9.1,style=customasmARM]{patterns/09_loops/simple/ARM/ARM64_GCC491_O3_RU.s}

\myparagraph{ARM64: \NonOptimizing GCC 4.9.1}

\lstinputlisting[caption=\NonOptimizing GCC 4.9.1 -fno-inline,style=customasmARM]{patterns/09_loops/simple/ARM/ARM64_GCC491_O0_RU.s}
}
\DE{\subsubsection{ARM}

\myparagraph{\NonOptimizingKeilVI (\ARMMode)}

\lstinputlisting[label=Keil_number_sign,style=customasmARM]{patterns/09_loops/simple/ARM/Keil_ARM_O0.asm}

Der Zähler $i$ wird im Register \Reg{4} gespeichert. 
Der Befehl \INS{MOV R4, \#2} initialisiert $i$.
Die Befehle \INS{MOV R0, R4} und \INS{BL printing\_function} bilden den Körper
der Schleife; der erste Befehl bereitet das Argument für die \ttf Funktion vor
und der zweite ruft die Funktion auf.

\myindex{ARM!\Instructions!ADD}
Der Befehl \INS{ADD R4, R4, \#1} erhöht die $i$ Variable in jedem Durchlauf um
1.
\myindex{ARM!\Instructions!CMP}
\myindex{ARM!\Instructions!BLT}
\INS{CMP R4, \#0xA} vergleicht $i$ mit \TT{0xA} (10). 
Der nächste Befehl \INS{BLT} (\emph{Branch Less Than}) springt, falls $i$ kleiner
als 10 ist. Sonst wird 0 in das Register \Reg{0} geschrieben (unsere Funktion
liefert den Wert 0 zurück) und die Funktionsausführung wird beendet.

\myparagraph{\OptimizingKeilVI (\ThumbMode)}

\lstinputlisting[style=customasmARM]{patterns/09_loops/simple/ARM/Keil_thumb_O3.asm}

Praktisch das gleiche.

\myparagraph{\OptimizingXcodeIV (\ThumbTwoMode)}
\label{ARM_unrolled_loops}

\lstinputlisting[style=customasmARM]{patterns/09_loops/simple/ARM/xcode_thumb_O3.asm}

In meiner \ttf Funktion befand sich tatsächlich Folgendes:

\begin{lstlisting}[style=customc]
void printing_function(int i)
{
    printf ("%d\n", i);
};
\end{lstlisting}

\myindex{Unrolled loop}
\myindex{Inline code}
Also hat LLVM die Schleife nicht nur \emph{unrolled} sondern auch die einfache
Funktion \ttf \emph{inlined} und den Körper der Schleife acht Mal generiert,
anstatt die Schleife aufzurufen. 

Dies ist möglich, wenn eine Funktion sehr einfach ist (wie meine) und wenn sie
nicht allzu oft aufgerufen wird (wie hier).


\myparagraph{ARM64: \Optimizing GCC 4.9.1}

\lstinputlisting[caption=\Optimizing GCC 4.9.1,style=customasmARM]{patterns/09_loops/simple/ARM/ARM64_GCC491_O3_DE.s}

\myparagraph{ARM64: \NonOptimizing GCC 4.9.1}

\lstinputlisting[caption=\NonOptimizing GCC 4.9.1 -fno-inline,style=customasmARM]{patterns/09_loops/simple/ARM/ARM64_GCC491_O0_DE.s}

}
\FR{\subsubsection{ARM}

\myparagraph{\NonOptimizingKeilVI (\ARMMode)}

\lstinputlisting[label=Keil_number_sign,style=customasmARM]{patterns/09_loops/simple/ARM/Keil_ARM_O0.asm}

Le compteur de boucle $i$ est stocké dans le registre \Reg{4}.
L'instruction \INS{MOV R4, \#2} initialise $i$.
Les instructions \INS{MOV R0, R4} et \INS{BL printing\_function} composent le corps
de la boucle, la première instruction préparant l'argument pour la fonction \ttf
et la seconde l'appelant.
\myindex{ARM!\Instructions!ADD}
L'instruction \INS{ADD R4, R4, \#1} ajoute 1 à la variable $i$ à chaque itération.
\myindex{ARM!\Instructions!CMP}
\myindex{ARM!\Instructions!BLT}
\INS{CMP R4, \#0xA} compare $i$ avec \TT{0xA} (10). 
L'instruction suivante, \INS{BLT} (\IT{Branch Less Than}) saute si $i$ est inférieur
à 10.
Autrement, 0 est écrit dans \Reg{0} (puisque notre fonction renvoie 0) et l'exécution
de la fonction se termine.

\myparagraph{\OptimizingKeilVI (\ThumbMode)}

\lstinputlisting[style=customasmARM]{patterns/09_loops/simple/ARM/Keil_thumb_O3.asm}

Pratiquement la même chose.

\myparagraph{\OptimizingXcodeIV (\ThumbTwoMode)}
\label{ARM_unrolled_loops}

\lstinputlisting[style=customasmARM]{patterns/09_loops/simple/ARM/xcode_thumb_O3.asm}

En fait, il y avait ceci dans ma fonction \ttf:

\begin{lstlisting}[style=customc]
void printing_function(int i)
{
    printf ("%d\n", i);
};
\end{lstlisting}

\myindex{Unrolled loop}
\myindex{Inline code}
Donc, non seulement LLVM \IT{déroule} la boucle, mais aussi \IT{inline} ma fonction
très simple et insère son corps 8 fois au lieu de l'appeler.

Ceci est possible lorsque la fonction est très simple (comme la mienne) et lorsqu'elle
n'est pas trop appelée (comme ici).

\myparagraph{ARM64: GCC 4.9.1 \Optimizing}

\lstinputlisting[caption=GCC 4.9.1 \Optimizing,style=customasmARM]{patterns/09_loops/simple/ARM/ARM64_GCC491_O3_FR.s}

\myparagraph{ARM64: GCC 4.9.1 \NonOptimizing}

\lstinputlisting[caption=GCC 4.9.1 -fno-inline \NonOptimizing,style=customasmARM]{patterns/09_loops/simple/ARM/ARM64_GCC491_O0_FR.s}
}
\JA{\subsubsection{ARM}

\myparagraph{\NonOptimizingKeilVI (\ARMMode)}

\lstinputlisting[label=Keil_number_sign,style=customasmARM]{patterns/09_loops/simple/ARM/Keil_ARM_O0.asm}

ループカウンタ $i$ は\Reg{4}レジスタに格納されます。
\INS{MOV R4, \#2}命令は $i$ をちょうど初期化します。 
\INS{MOV R0, R4}、および\INS{BL printing\_function}命令は、
\ttf 関数の引数を準備する最初の命令と2番目の関数を呼び出すループの本体を構成します。 
\myindex{ARM!\Instructions!ADD}
\INS{ADD R4, R4, \#1}命令は、各繰り返しで $i$ 変数に1を加算するだけです。 
\INS{CMP R4, \#0xA}は $i$ と\TT{0xA}(10)を比較します。 
次の命令\INS{BLT}(\emph{Branch Less Than})は、 $i$ が10未満の場合にジャンプします。
それ以外の場合は、\Reg{0}に0が書き込まれます(関数が0を返すため)。
そして関数の実行が終了します。

\myparagraph{\OptimizingKeilVI (\ThumbMode)}

\lstinputlisting[style=customasmARM]{patterns/09_loops/simple/ARM/Keil_thumb_O3.asm}

実質的に同じです。

\myparagraph{\OptimizingXcodeIV (\ThumbTwoMode)}
\label{ARM_unrolled_loops}

\lstinputlisting[style=customasmARM]{patterns/09_loops/simple/ARM/xcode_thumb_O3.asm}

実際、これは私の \ttf 関数にありました:

\begin{lstlisting}[style=customc]
void printing_function(int i)
{
    printf ("%d\n", i);
};
\end{lstlisting}

\myindex{Unrolled loop}
\myindex{Inline code}
だから、LLVMはループを\emph{展開}しただけでなく、
私の非常に単純な関数 \ttf を\emph{インライン}化し、
呼び出すのではなく本体を8回挿入しました。

これは、関数が(私の例のように)とても簡単で、(ここのように)あまり呼び出されないときに可能です。

\myparagraph{ARM64: \Optimizing GCC 4.9.1}

\lstinputlisting[caption=\Optimizing GCC 4.9.1,style=customasmARM]{patterns/09_loops/simple/ARM/ARM64_GCC491_O3_JA.s}

\myparagraph{ARM64: \NonOptimizing GCC 4.9.1}

\lstinputlisting[caption=\NonOptimizing GCC 4.9.1 -fno-inline,style=customasmARM]{patterns/09_loops/simple/ARM/ARM64_GCC491_O0_JA.s}
}


\subsubsection{MIPS}

\ifdefined\RUSSIAN
\lstinputlisting[caption=\NonOptimizing GCC 4.4.5 (IDA),style=customasmMIPS]{patterns/09_loops/simple/MIPS_O0_IDA_RU.lst}

\myindex{MIPS!\Pseudoinstructions!B}
Новая для нас инструкция это \INS{B}. Вернее, это псевдоинструкция (\INS{BEQ}).
\fi

\ifdefined\ENGLISH
\lstinputlisting[caption=\NonOptimizing GCC 4.4.5 (IDA),style=customasmMIPS]{patterns/09_loops/simple/MIPS_O0_IDA_EN.lst}

\myindex{MIPS!\Pseudoinstructions!B}
The instruction that's new to us is \TT{B}. It is actually the pseudo instruction (\INS{BEQ}).
\fi

\ifdefined\FRENCH
\lstinputlisting[caption=GCC 4.4.5 \NonOptimizing (IDA),style=customasmMIPS]{patterns/09_loops/simple/MIPS_O0_IDA_FR.lst}

\myindex{MIPS!\Pseudoinstructions!B}
L'instruction qui est nouvelle pour nous est \TT{B}. C'est la pseudo instruction (\INS{BEQ}).
\fi

\ifdefined\JAPANESE
\lstinputlisting[caption=GCC 4.4.5 \NonOptimizing (IDA),style=customasmMIPS]{patterns/09_loops/simple/MIPS_O0_IDA_JA.lst}

\myindex{MIPS!\Pseudoinstructions!B}
新しい命令は\TT{B}です。 実際には擬似命令(\INS{BEQ})です。
\fi



\subsubsection{Ещё кое-что}

По генерируемому коду мы видим следующее: после инициализации $i$, тело цикла не исполняется.
В начале проверяется условие $i$, а лишь затем исполняется тело цикла.
И это правильно, потому что если условие в самом начале не истинно, тело цикла исполнять нельзя.

Так может быть, например, в таком случае:

\lstinputlisting[style=customc]{patterns/09_loops/simple/loops_3_RU.c}

Если \emph{количество\_элементов\_для\_обработки} равно 0,
тело цикла не должно исполниться ни разу.
Поэтому проверка условия происходит перед тем как исполнить само тело.

Впрочем, оптимизирующий компилятор может переставить проверку условия и тело цикла местами, если он уверен,
что описанная здесь ситуация невозможна, как в случае с нашим простейшим примером и с применением компиляторов
Keil, Xcode (LLVM), MSVC и GCC в режиме оптимизации.

