\subsection{Vérification de condition}

Il est important de garder à l'esprit que dans une boucle \emph{for()}, la condition est vérifiée 
préalablement à l'itération du corps de la boucle et non pas après. Cela étant il est souvent plus 
pratique pour le compilateur de placer les instructions qui effectuent le test après le corps de 
la boucle. Il arrive aussi qu'il rajoute des vérifications au début du corps de la boucle.

Par exemple:

\lstinputlisting[style=customc]{patterns/09_loops/cond_check/1.c}

GCC 5.4.0 x64 en mode optimisé:

\lstinputlisting[style=customasmx86]{patterns/09_loops/cond_check/1.s}

Nous constatons la présence de deux vérifications.

\myindex{Hex-Rays}
Le code décompilé produit par Hex-Rays (dans sa version 2.2.0) est celui-ci:

\lstinputlisting[style=customc]{patterns/09_loops/cond_check/hexrays.c}

Dans le cas présent, il ne fait aucun doute que la structure \emph{do/while()} peut être remplacée par 
une construction \emph{for()}, et que le premier contrôle peut être supprimé.

