\subsection{\Conclusion{}}

Примерный скелет цикла от 2 до 9 включительно:

\lstinputlisting[caption=x86,style=customasmx86]{patterns/09_loops/skeleton_x86_2_9_optimized_RU.lst}

Операция инкремента может быть представлена как 3 инструкции в неоптимизированном коде:

\lstinputlisting[caption=x86,style=customasmx86]{patterns/09_loops/skeleton_x86_2_9_RU.lst}

Если тело цикла короткое, под переменную счетчика можно выделить целый регистр:

\lstinputlisting[caption=x86,style=customasmx86]{patterns/09_loops/skeleton_x86_2_9_reg_RU.lst}

Некоторые части цикла могут быть сгенерированы компилятором в другом порядке:

\lstinputlisting[caption=x86,style=customasmx86]{patterns/09_loops/skeleton_x86_2_9_order_RU.lst}

Обычно условие проверяется \emph{перед} телом цикла, но компилятор может перестроить цикл так, 
что условие проверяется \emph{после} тела цикла.

Это происходит тогда, когда компилятор уверен, что условие всегда будет \emph{истинно} на первой итерации,
так что тело цикла исполнится как минимум один раз:

\lstinputlisting[caption=x86,style=customasmx86]{patterns/09_loops/skeleton_x86_2_9_reorder_RU.lst}

\myindex{x86!\Instructions!LOOP}
Используя инструкцию \TT{LOOP}. Это редкость, компиляторы не используют её.
Так что если вы её видите, это верный знак, что этот фрагмент кода написан вручную:

\lstinputlisting[caption=x86,style=customasmx86]{patterns/09_loops/skeleton_x86_loop_RU.lst}

ARM. 
В этом примере регистр \Reg{4} выделен для переменной счетчика:


\lstinputlisting[caption=ARM,style=customasmARM]{patterns/09_loops/skeleton_ARM_RU.lst}

% TODO MIPS
