\mysection{Blague avec le gestionnaire de tâche (Windows Vista)}
\myindex{Windows!Windows Vista}

Voyons s'il est possible de légèrement modifier le gestionnaire de tâches pour qu'il
détecte plus de c\oe{}urs \ac{CPU}.

\myindex{Windows!NTAPI}

Demandons-nous d'abord, comment est-ce que le gestionnaire de tâches connait le nombre
de c\oe{}urs?

Il y a une fonction win32 \TT{GetSystemInfo()} présente dans l'espace utilisateur
win32 qui peut nous dire ceci.
Mais elle n'est pas importées dans \TT{taskmgr.exe}.

Il y en a, toutefois, une autre dans \gls{NTAPI}, \TT{NtQuerySystemInformation()},
qui est utilisée dans \TT{taskmgr.exe} à plusieurs endroits.

Pour obtenir le nombre de c\oe{}urs,  il faut appeler cette fonction avec la constante
\TT{SystemBasicInformation} comme premier argument (qui vaut zéro\footnote{\href{http://go.yurichev.com/17251}{MSDN}}).

Le second argument doit pointer vers le buffer qui va recevoir toute l'information.

Donc nous devons trouver tous les appels à la fonction \\
\TT{NtQuerySystemInformation(0, ?, ?, ?)}.
Ouvrons \TT{taskmgr.exe} dans \IDA.
\myindex{Windows!PDB}

Ce qui est toujours bien avec les exécutables Microsoft, c'est que \IDA peut télécharger
le fichier \gls{PDB} correspondant à cet exécutable et afficher les noms de toutes
les fonctions. 

Il est visible que le gestionnaire des tâches est écrit en \Cpp et certains noms
de fonction et classes sont vraiment parlants.
Il y a des classes CAdapter, CNetPage, CPerfPage, CProcInfo, CProcPage, CSvcPage, 
CTaskPage, CUserPage.

Il semble que chaque onglet du gestionnaire de tâches ait une classe correspondante.

Regardons chaque appel et ajoutons un commentaire avec la valeur qui est passée comme
premier argument de la fonction.
Nous allons écrire \q{not zero} à certains endroits, car la valeur n'est clairement
pas zéro, mais quelque chose de vraiment différent (plus à ce propos dans la seconde
partie de ce chapitre).

Et nous cherchons les zéros passés comme argument après tout.

\begin{figure}[H]
\centering
\myincludegraphics{examples/taskmgr/IDA_xrefs.png}
\caption{IDA: références croisées vers NtQuerySystemInformation()}
\end{figure}

Oui, les noms parlent vraiment d'eux-même.

Nous allons examiner précisément les endroits où\\
\TT{NtQuerySystemInformation(0, ?, ?, ?)} est appelée,
nous trouvons rapidement ce que nous cherchons dans la fonction \TT{InitPerfInfo()}:

\lstinputlisting[caption=taskmgr.exe (Windows Vista),style=customasmx86]{examples/taskmgr/taskmgr.lst}

\TT{g\_cProcessors} est une variable globale, et ce nom a été assigné par \IDA suivant
le symbole \gls{PDB} chargé depuis le serveur de Microsoft.

L'octet est pris de \TT{var\_C20}.
Et \TT{var\_C58} est passée à\\
\TT{NtQuerySystemInformation()} comme un pointeur sur le buffer de réception.
La différence entre 0xC20 et 0xC58 est 0x38 (56).

Regardons le format de la structure renvoyée, que nous pouvons trouver dans MSDN:

\begin{lstlisting}[style=customc]
typedef struct _SYSTEM_BASIC_INFORMATION {
    BYTE Reserved1[24];
    PVOID Reserved2[4];
    CCHAR NumberOfProcessors;
} SYSTEM_BASIC_INFORMATION;
\end{lstlisting}

Ceci est un système x64, donc chaque PVOID occupe 8 octets.

Tous les champs réservés dans la structure occupent $24+4*8=56$ octets.

Oh oui, ceci implique que \TT{var\_C20} dans la pile locale est exactement le champ
\TT{NumberOfProcessors} de la structure \TT{SYSTEM\_BASIC\_INFORMATION}.

Vérifions notre hypothèse.
Copier \TT{taskmgr.exe} depuis \TT{C:\textbackslash{}Windows\textbackslash{}System32}
dans un autre répertoire (ainsi le \emph{Windows Resource Protection} ne va pas essayer
de restaurer l'ancienne version du \TT{taskmgr.exe} modifié).

Ouvrons-le dans Hiew et trouvons l'endroit:

\begin{figure}[H]
\centering
\myincludegraphics{examples/taskmgr/hiew2.png}
\caption{Hiew: trouver l'endroit à modifier}
\end{figure}

Remplaçons l'instruction \TT{MOVZX}  par la notre.
Prétendons avoir un CPU 64 c\oe{}urs.

Ajouter un \ac{NOP} additionnel (car notre instruction est plus courte que l'originale):

\begin{figure}[H]
\centering
\myincludegraphics{examples/taskmgr/hiew1.png}
\caption{Hiew: modification effectuée}
\end{figure}

Et ça fonctionne!
Bien sûr, les données dans les graphes ne sont pas correctes.

À certains moments, le gestionnaire de tâches montre même une charge globale du CPU
de plus de 100\%.

\begin{figure}[H]
\centering
\myincludegraphics{examples/taskmgr/taskmgr_64cpu_crop.png}
\caption{Gestionnaire de tâches Windows fou}
\end{figure}

Le plus grand nombre avec lequel le gestionnaire de tâches ne plante pas est 64.

Il semble que le gestionnaire de tâche de Windows Vista n'a pas été testé sur des
ordinateurs avec un grand nombre de c\oe{}urs.

Il doit y avoir une sorte de structure de données dedans. limitée à 64 c\oe{}urs
(ou plusieurs).

\subsection{Utilisation de LEA pour charger des valeurs}
\label{TaskMgr_LEA}

Parfois, \TT{LEA} est utilisée dans \TT{taskmgr.exe} au lieu de \TT{MOV} pour définir
le premier argument de \\
\TT{NtQuerySystemInformation()}:

\lstinputlisting[caption=taskmgr.exe (Windows Vista),style=customasmx86]{examples/taskmgr/taskmgr2.lst}

\myindex{x86!\Instructions!LEA}

Peut-être que \ac{MSVC} fit ainsi car le code machine de \INS{LEA} est plus court
que celui de \INS{MOV REG, 5} (il serait de 5 au lieu de 4).

\INS{LEA} avec un offset dans l'intervalle $-128..127$ (l'offset occupe 1 octet dans
l'opcode) avec des registres 32-bit est encore plus court (faute de préfixe REX )---3
octets.

Un autre exemple d'une telle chose: \myref{using_MOV_and_pack_of_LEA_to_load_values}.

