\subsection{10 PRINT CHR\$(205.5+RND(1)); : GOTO 10}

Tous les exemples sont des fichier MS-DOS .COM.
%FIXME1 -> about .COM files

\myindex{MS-DOS}
Dans [Nick Montfort et al, \emph{10 PRINT CHR\$(205.5+RND(1)); : GOTO 10}, (The MIT Press:2012)]
\footnote{\AlsoAvailableAs \url{http://go.yurichev.com/17286}}

nous pouvons nous renseigner sur l'un des générateurs de labyrinthe le plus simple
possible.

Il affiche simplement un caractère slash ou backslash aléatoirement et indéfiniment,
donnant quelque chose comme ceci:

\begin{figure}[H]
\centering
\includegraphics[width=0.6\textwidth]{examples/demos/10print/10print.png}
\end{figure}

Il y a quelques implémentations connues en x86 16-bit.

\subsubsection{Version de Trixter en 42 octets}

\newcommand{\FNURLTRIXTER}{\footnote{\url{http://go.yurichev.com/17305}}}

Le listing provient du site web\FNURLTRIXTER, mais les commentaires sont miens.

\lstinputlisting[style=customasmx86]{examples/demos/10print/10print_42_FR.lst}

\myindex{Intel!8253}
La valeur pseudo aléatoire ici est en fait le temps qui a passé depuis le démarrage
du système, pris depuis le temps du chip 8253, dont la valeur est incrémentée 18,2
fois par seconde.

En écrivant zéro sur le port \TT{43h},
nous envoyons la commande \q{select counter 0},
"counter latch", 
"binary counter" (pas une valeur \ac{BCD}).

\myindex{x86!\Instructions!POPF}
Les interruptions sont ré-autorisées avec l'instruction \TT{POPF}, qui restaure aussi
le flag \TT{IF}.

\myindex{x86!\Instructions!IN}
Il n'est pas possible d'utiliser l'instruction \TT{IN} avec des registres autres
que \TT{AL}, d'où le mélange.

\subsubsection{
Ma tentative de réduire la version de Trixter: 27 octets}

Nous pouvons dire que puisque nous utilisons le timer non pas pour avoir une valeur
précise, mais une valeur pseudo aléatoire, nous n'avons pas besoin de passer du temps
(et du code) pour interdire les interruptions.

Une autre chose que l'on peut dire est que nous n'avons besoin que d'un bit de la
partie basse 8-bit, donc lisons-le seulement.

Nous pouvons réduire légèrement le code et obtenons 27 octets:

\lstinputlisting[style=customasmx86]{examples/demos/10print/10print_27_FR.lst}

\subsubsection{
Prendre le contenu résiduel de la mémoire comme source d'aléas}

Puisqu'il s'agit de MS-DOS, il n'y a pas du tout de protection de la mémoire, nous
pouvons lire l'adresse que nous voulons.
\myindex{x86!\Instructions!LODSB}
Encore mieux que ça: une seule instruction \TT{LODSB} lit un octet depuis l'adresse
\TT{DS:SI}, mais ce n'est pas un problème si les valeurs des registres ne sont pas
définies, lisons 1) des octets aléatoires; 2) depuis un endroit aléatoire de la mémoire! 

Il est suggéré dans la page web de Trixter\FNURLTRIXTER d'utiliser \TT{LODSB} sans
aucune initialisation.

\myindex{x86!\Instructions!SCASB}
Il est aussi suggéré que l'on utilisé à la place l'instruction \TT{SCASB}, car elle
met les flags suivant l'octet qu'elle lit.

Une autre idée pour minimiser le code est d'utiliser l'appel système DOS \TT{INT 29h},
qui affiche simplement le caractère stocké dans le registre \TT{AL}.

C'est ce que Peter Ferrie a fait
\footnote{\url{http://go.yurichev.com/17087}}:

\lstinputlisting[caption=Peter Ferrie: 10 octets,style=customasmx86]{examples/demos/10print/ferrie_10_FR.lst}

Donc il est possible de se passer complètement de saut conditionnel.
Le code \ac{ASCII} du backslash (\q{\textbackslash{}}) est \TT{0x5C} et \TT{0x2F}
pour le slash (\q{/}).
Dons nous devons convertir un bit (pseudo aléatoire) dans le flag \TT{CF} en une
valeur \TT{0x5C} ou \TT{0x2F}.

Ceci est fait facilement: en \TT{AND}-ant tous les bits de \TT{AL} (où tous les 8
bits sont mis ou effacés) nous obtenons 0 ou \TT{0x2D}.

En ajoutant \TT{0x2F} à cette valeur, nous obtenons \TT{0x5C} ou \TT{0x2F}.

Puis il suffit de l'afficher sur l'écran.

\subsubsection{\Conclusion{}}

\myindex{DOSBox}
Il vaut la peine de mentionner que le résultat peut être différent dans DOSBox, \gls{Windows NT}
et même MS-DOS,

à cause de conditions différentes: le chip timer peut être émulé différemment et le
contenu initial du registre peut être différent aussi.
