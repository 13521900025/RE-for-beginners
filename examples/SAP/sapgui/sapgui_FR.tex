\subsection{À propos de la compression du trafic réseau par le client SAP}
\label{sec:SAPGUI}

\newcommand{\TDWNC}{TDW\_NOCOMPRESS\xspace}

(Tracer la connexion entre la variable d'environnement \TDWNC{} SAPGUI\footnote{client SAP GUI}
et la fenêtre pop-up gênante et ennuyeuse et la routine de compression de données actuelle.)
 
On sait que le trafic réseau entre le SAPGUI et SAP n'est pas chiffré par défaut,
mais compressé (voir ici\footnote{\url{http://go.yurichev.com/17221}} et
ici\footnote{\href{http://go.yurichev.com/17225}{blog.yurichev.com}}).

Il est aussi connue que mettre la variable d'environnement \emph{\TDWNC} à 1,
permet d'arrêter la compression des paquets réseau.

Mais vous verrez toujours l'ennuyeuse fenêtre pop-up, qui ne peut pas être fermée:

\begin{figure}[H]
\centering
\myincludegraphics{examples/SAP/sapgui/sapgui720.png}
\caption{Screenshot}
\end{figure}

Voyons si nous pouvons supprimer cette fenêtre.

Mais avant, voyons ce que nous savons déjà:

Premièrement: nous savons que la variable d'environnement \emph{\TDWNC} est vérifiée
quelque part dans le client SAPGUI.

Deuxièmement: une chaîne comme \q{data compression switched off} doit s'y trouver
quelque part.
\newcommand{\FNURLFAR}{\footnote{\url{http://go.yurichev.com/17347}}}

Avec l'aide du gestionnaire de fichier FAR\FNURLFAR nous pouvons trouver que deux
de ces chaînes sont stockées dans le fichier SAPguilib.dll.

Donc ouvrons SAPguilib.dll dans \IDA et cherchons la chaîne \emph{\TDWNC}.
Oui, elle s'y trouve et il n'y a qu'une référence vers elle.

Nous voyons le morceau de code suivant (tous les offsets de fichiers sont valables
pour SAPGUI 720 win32, fichier SAPguilib.dll version 7200,1,0,9009):

\lstinputlisting[style=customasmx86]{examples/SAP/sapgui/0.lst}

\myindex{\CStandardLibrary!atoi()}

La chaîne renvoyée par \TT{chk\_env()} via son second argument est ensuite traitée
par la fonction de chaîne MFC et ensuite \TT{atoi()}\footnote{fonction C standard
qui convertit les chiffres d'une chaîne en un nombre} est appelée.
Après ça, la valeur numérique est stockée en \TT{edi+15h}.

Jetons aussi un \oe{}il à la fonction \TT{chk\_env()} (nous avons donné ce nom manuellement):

\lstinputlisting[style=customasmx86]{examples/SAP/sapgui/01.lst}

\myindex{\CStandardLibrary!getenv()}
Oui. La fonction \TT{getenv\_s()}\footnote{\href{http://go.yurichev.com/17250}{MSDN}}

est une version de Microsoft à la sécurité avancée de \TT{getenv()}\footnote{Fonction
de la bibliothèque C standard renvoyant une variable d'environnement}.

\myindex{MFC}

Il y a quelques manipulation de chaîne MFC.

De nombreuses autres variables d'environnement sont également testées.
Voici une liste de toutes les variables qui sont testé et ce que SAPGUI écrirait
dans son fichier de log, lorsque les traces sont activées:

\small
\begin{center}
\begin{tabular}{ | l | l | }
\hline                        
DPTRACE                  & ``GUI-OPTION: Trace set to \%d'' \\
TDW\_HEXDUMP             & ``GUI-OPTION: Hexdump enabled'' \\
TDW\_WORKDIR             & ``GUI-OPTION: working directory `\%s\''' \\
TDW\_SPLASHSRCEENOFF     & ``GUI-OPTION: Splash Screen Off''\\
                         & ``GUI-OPTION: Splash Screen On'' \\
TDW\_REPLYTIMEOUT        & ``GUI-OPTION: reply timeout \%d milliseconds'' \\
TDW\_PLAYBACKTIMEOUT     & ``GUI-OPTION: PlaybackTimeout  set to \%d milliseconds'' \\ 
TDW\_NOCOMPRESS          & ``GUI-OPTION: no compression read'' \\
TDW\_EXPERT              & ``GUI-OPTION: expert mode'' \\
TDW\_PLAYBACKPROGRESS    & ``GUI-OPTION: PlaybackProgress'' \\
TDW\_PLAYBACKNETTRAFFIC  & ``GUI-OPTION: PlaybackNetTraffic'' \\
TDW\_PLAYLOG             & ``GUI-OPTION: /PlayLog is YES, file \%s'' \\
TDW\_PLAYTIME            & ``GUI-OPTION: /PlayTime set to \%d milliseconds'' \\
TDW\_LOGFILE             & ``GUI-OPTION: TDW\_LOGFILE `\%s\''' \\
TDW\_WAN                 & ``GUI-OPTION: WAN - low speed connection enabled'' \\
TDW\_FULLMENU            & ``GUI-OPTION: FullMenu enabled'' \\
SAP\_CP / SAP\_CODEPAGE  & ``GUI-OPTION: SAP\_CODEPAGE `\%d\''' \\
UPDOWNLOAD\_CP           & ``GUI-OPTION: UPDOWNLOAD\_CP `\%d\''' \\
SNC\_PARTNERNAME         & ``GUI-OPTION: SNC name `\%s\''' \\
SNC\_QOP                 & ``GUI-OPTION: SNC\_QOP `\%s\''' \\
SNC\_LIB                 & ``GUI-OPTION: SNC is set to: \%s'' \\ 
SAPGUI\_INPLACE          & ``GUI-OPTION: environment variable SAPGUI\_INPLACE is on'' \\
\hline  
\end{tabular}
\end{center}
\normalsize


La configuration de chaque variable est écrit dans le tableau via le pointeur dans
le registre \EDI. \EDI est renseigné avant l'appel à la fonction:

\begin{lstlisting}[style=customasmx86]
.text:6440EE00                 lea     edi, [ebp+2884h+var_2884] ; options here like +0x15...
.text:6440EE03                 lea     ecx, [esi+24h]
.text:6440EE06                 call    load_command_line
.text:6440EE0B                 mov     edi, eax
.text:6440EE0D                 xor     ebx, ebx
.text:6440EE0F                 cmp     edi, ebx
.text:6440EE11                 jz      short loc_6440EE42
.text:6440EE13                 push    edi
.text:6440EE14                 push    offset aSapguiStoppedA ; "Sapgui stopped after commandline interp"...
.text:6440EE19                 push    dword_644F93E8
.text:6440EE1F                 call    FEWTraceError
\end{lstlisting}

Maintenant, pouvons-nous trouver la chaîne \emph{data record mode switched on}?

Oui, et la seule référence est dans
\par \TT{CDwsGui::PrepareInfoWindow()}.

Comment connaissons-nous les noms de classe/méthode? Il y a beaucoup d'appels spéciaux
de débogage qui écrivent dans les fichiers de log, comme:

\lstinputlisting[style=customasmx86]{examples/SAP/sapgui/02.lst}

\dots ou:

\begin{lstlisting}[style=customasmx86]
.text:6440237A                 push    eax
.text:6440237B                 push    offset aCclientStart_6 ; "CClient::Start: set shortcut user to '\%"...
.text:64402380                 push    dword ptr [edi+4]
.text:64402383                 call    dbg
.text:64402388                 add     esp, 0Ch
\end{lstlisting}

C'est \emph{très} utile.

Voyons le contenu de la fonction de cette fenêtre pop-up ennuyeuse:

\lstinputlisting[style=customasmx86]{examples/SAP/sapgui/03.lst}

Au début de la fonction, \ECX a un pointeur sur l'objet (puisque c'est une fonction
avec le type d'appel thiscall~(\myref{thiscall})).
Dans notre cas, l'objet a étonnement un type de classe de \emph{CDwsGui}.
En fonction de l'option mise dans l'objet, un message spécifique est concaténé au
message résultant.

Si la valeur à l'adresse \TT{this+0x3D} n'est pas zéro, la compression est désactivée:

\lstinputlisting[style=customasmx86]{examples/SAP/sapgui/04.lst}

Finalement, il est intéressant de noter que l'état de la variable \emph{var\_10}
défini si le message est affiché:

\lstinputlisting[style=customasmx86]{examples/SAP/sapgui/1_FR.lst}

Vérifions notre théorie en pratique.

\JNZ à cette ligne \dots

\lstinputlisting[style=customasmx86]{examples/SAP/sapgui/2_FR.lst}

\dots 
remplaçons-le par un \JMP, et nous obtenons SAPGUI fonctionnant sans que l'ennuyeuse
fenêtre pop-up n'apparaisse!

Maintenant approfondissons et trouvons la relation entre l'offset $0x15$ dans la
fonction \TT{load\_command\_line()} (nous lui avons donné ce nom) et la variable
\TT{this+0x3D} dans \emph{CDwsGui::PrepareInfoWindow}.
Sommes-nous sûrs que la valeur est la même?

Nous commençons par chercher toutes les occurrences de la valeur \TT{0x15} dans le
code.
Pour un petit programme comme SAPGUI, cela fonctionne parfois.a Voici la première
occurrence que nous obtenons:

\lstinputlisting[style=customasmx86]{examples/SAP/sapgui/3_FR.lst}

La fonction a été appelée depuis la fonction appelée \emph{CDwsGui::CopyOptions}!
Encore merci pour les informations de débogage.

Mais la vraie réponse est dans \emph{CDwsGui::Init()}:

\lstinputlisting[style=customasmx86]{examples/SAP/sapgui/4_FR.lst}

Enfin, nous comprennons: le tableau rempli dans la fonction \TT{load\_command\_line()}
est stocké dans la classe \emph{CDwsGui} mais à l'adresse \TT{this+0x28}. 0x15 + 0x28
vaut exactement 0x3D. OK, nous avons trouvé le point où la valeur y est copiée.

Trouvons les autres endroits où l'offset 0x3D est utilisé.
Voici l'un d'entre eux dans la fonction \emph{CDwsGui::SapguiRun} (à nouveau, merci
aux appels de débogage):

\lstinputlisting[style=customasmx86]{examples/SAP/sapgui/5_FR.lst}

Vérifions nos découvertes. \\
Remplaçons \TT{setz al} par les instructions \TT{xor eax, eax / nop},
effaçons la variable d'environnement \TDWNC et lançons SAPGUI. Ouah! La fenêtre ennuyeuse
n'est plus là (comme nous l'attendions, puisque la variable d'environnement n'est
pas mise) mais dans Wireshark nous pouvons voir que les paquets réseau ne sont plus
compressés!
Visiblement, c'est le point où le flag de compression doit être défini dans l'objet
\emph{CConnectionContext}.

Donc, le flag de compression est passé dans le 5ème argument de \emph{CConnectionContext::CreateNetwork}.
À l'intérieur de la fonction, une autre est appelée:

\begin{lstlisting}[style=customasmx86]
...
.text:64403476                 push    [ebp+compression]
.text:64403479                 push    [ebp+arg_C]
.text:6440347C                 push    [ebp+arg_8]
.text:6440347F                 push    [ebp+arg_4]
.text:64403482                 push    [ebp+arg_0]
.text:64403485                 call    CNetwork__CNetwork
\end{lstlisting}

Le flag de compression est passé ici dans le 5ème argument au constructer \emph{CNetwork::CNetwork}.

Et voici comment le constructeur \emph{CNetwork} défini le flag dans l'objet \emph{CNetwork}
suivant son 5ème argument \emph{et} une autre variable qui peut probablement aussi
affecter la compression des paquets réseau.

\lstinputlisting[style=customasmx86]{examples/SAP/sapgui/6_FR.lst}

À ce point, nous savons que le flag de compression est stocké dans la classe
\emph{CNetwork} à l'adresse \emph{this+0x3A4}.

Plongeons-nous maintenant dans SAPguilib.dll à la recherche de la valeur \TT{0x3A4}.
Et il y a une seconde occurrence dans \emph{CDwsGui::OnClientMessageWrite} (Merci
infiniment pour les informations de débogage):

\lstinputlisting[style=customasmx86]{examples/SAP/sapgui/7_FR.lst}

Jetons un \oe{}il dans \emph{sub\_644055C5}. Nous y voyons seulement l'appel à memcpy()
et une autre fonction appelée (par \IDA) \emph{sub\_64417440}.

Et, regardons dans \emph{sub\_64417440}. Nous y voyons:

\lstinputlisting[style=customasmx86]{examples/SAP/sapgui/8.lst}

Voilà! Nous avons trouvé la fonction qui effectue la compression des données.
Comme cela a été décrit dans le passé
\footnote{\url{http://go.yurichev.com/17312}},

cette fonction est utilisé dans SAP et aussi dans le projet open-source MaxDB.
Donc, elle est disponible sous forme de code source.

La dernière vérification est faite ici:

\begin{lstlisting}[style=customasmx86]
.text:64406F79                 cmp     dword ptr [ecx+3A4h], 1
.text:64406F80                 jnz     compression_flag_is_zero
\end{lstlisting}

Remplaçons ici \JNZ par un \JMP inconditionnel. Supprimons la variable d'environnement
\TDWNC. Voilà!

Dans Wireshark nous voyons que les messages du client ne sont pas compressés. Les
réponses du serveur, toutefois, le sont.

Donc nous avons trouvé le lien entre la variable d'environnement et le point où la
routine de compression peut être appelée ou non.
