\subsection{About SAP client network traffic compression}
\label{sec:SAPGUI}

\newcommand{\TDWNC}{TDW\_NOCOMPRESS\xspace}

(Tracing the connection between the \TDWNC{}
SAPGUI\footnote{SAP GUI client} environment variable and 
the pesky annoying pop-up window and the actual data compression routine.)
 
It is known that the network traffic between SAPGUI and SAP is not encrypted by default, but compressed
(see here\footnote{\url{http://go.yurichev.com/17221}} 
and here\footnote{\href{http://go.yurichev.com/17225}{blog.yurichev.com}}). 

It is also known that by setting the environment variable \emph{\TDWNC} to 1, it is possible to turn the network packet compression off.

But you will see an annoying pop-up window that cannot be closed:

\begin{figure}[H]
\centering
\myincludegraphics{examples/SAP/sapgui/sapgui720.png}
\caption{Screenshot}
\end{figure}

Let's see if we can remove the window somehow.

But before this, let's see what we already know.

First: we know that the environment variable \emph{\TDWNC} is checked somewhere inside the SAPGUI client.

Second: a string like \q{data compression switched off} must be present somewhere in it.
\newcommand{\FNURLFAR}{\footnote{\url{http://go.yurichev.com/17347}}}

With the help of the FAR file manager\FNURLFAR we can found that both of these strings are stored in the SAPguilib.dll file.

So let's open SAPguilib.dll in \IDA and search for the \emph{\TDWNC} string. 
Yes, it is present and there is only one reference to it.

We see the following fragment of code 
(all file offsets are valid for SAPGUI 720 win32, SAPguilib.dll file version 7200,1,0,9009):

\lstinputlisting[style=customasmx86]{examples/SAP/sapgui/0.lst}

\myindex{\CStandardLibrary!atoi()}

The string returned by \TT{chk\_env()} via its second argument is then handled by the MFC string functions and then 
\TT{atoi()}\footnote{standard C library function that converts the digits in a string to a number} is called. 
After that, the numerical value is stored in \TT{edi+15h}.

Also take a look at the \TT{chk\_env()} function (we gave this name to it manually):

\lstinputlisting[style=customasmx86]{examples/SAP/sapgui/01.lst}

\myindex{\CStandardLibrary!getenv()}
Yes. The \TT{getenv\_s()}\footnote{\href{http://go.yurichev.com/17250}{MSDN}} 

function is a Microsoft security-enhanced version of \TT{getenv()}\footnote{Standard C library returning environment variable}.

\myindex{MFC}

There are also some MFC string manipulations.

Lots of other environment variables are checked as well. 
Here is a list of all variables that are being checked and what SAPGUI would write to its trace log when logging is turned on:

\small
\begin{center}
\begin{tabular}{ | l | l | }
\hline                        
DPTRACE                  & ``GUI-OPTION: Trace set to \%d'' \\
TDW\_HEXDUMP             & ``GUI-OPTION: Hexdump enabled'' \\
TDW\_WORKDIR             & ``GUI-OPTION: working directory `\%s\''' \\
TDW\_SPLASHSRCEENOFF     & ``GUI-OPTION: Splash Screen Off''\\
                         & ``GUI-OPTION: Splash Screen On'' \\
TDW\_REPLYTIMEOUT        & ``GUI-OPTION: reply timeout \%d milliseconds'' \\
TDW\_PLAYBACKTIMEOUT     & ``GUI-OPTION: PlaybackTimeout  set to \%d milliseconds'' \\ 
TDW\_NOCOMPRESS          & ``GUI-OPTION: no compression read'' \\
TDW\_EXPERT              & ``GUI-OPTION: expert mode'' \\
TDW\_PLAYBACKPROGRESS    & ``GUI-OPTION: PlaybackProgress'' \\
TDW\_PLAYBACKNETTRAFFIC  & ``GUI-OPTION: PlaybackNetTraffic'' \\
TDW\_PLAYLOG             & ``GUI-OPTION: /PlayLog is YES, file \%s'' \\
TDW\_PLAYTIME            & ``GUI-OPTION: /PlayTime set to \%d milliseconds'' \\
TDW\_LOGFILE             & ``GUI-OPTION: TDW\_LOGFILE `\%s\''' \\
TDW\_WAN                 & ``GUI-OPTION: WAN - low speed connection enabled'' \\
TDW\_FULLMENU            & ``GUI-OPTION: FullMenu enabled'' \\
SAP\_CP / SAP\_CODEPAGE  & ``GUI-OPTION: SAP\_CODEPAGE `\%d\''' \\
UPDOWNLOAD\_CP           & ``GUI-OPTION: UPDOWNLOAD\_CP `\%d\''' \\
SNC\_PARTNERNAME         & ``GUI-OPTION: SNC name `\%s\''' \\
SNC\_QOP                 & ``GUI-OPTION: SNC\_QOP `\%s\''' \\
SNC\_LIB                 & ``GUI-OPTION: SNC is set to: \%s'' \\ 
SAPGUI\_INPLACE          & ``GUI-OPTION: environment variable SAPGUI\_INPLACE is on'' \\
\hline  
\end{tabular}
\end{center}
\normalsize


The settings for each variable are written in the array via a pointer in the \EDI register.
\EDI is set before the function call:

\begin{lstlisting}[style=customasmx86]
.text:6440EE00                 lea     edi, [ebp+2884h+var_2884] ; options here like +0x15...
.text:6440EE03                 lea     ecx, [esi+24h]
.text:6440EE06                 call    load_command_line
.text:6440EE0B                 mov     edi, eax
.text:6440EE0D                 xor     ebx, ebx
.text:6440EE0F                 cmp     edi, ebx
.text:6440EE11                 jz      short loc_6440EE42
.text:6440EE13                 push    edi
.text:6440EE14                 push    offset aSapguiStoppedA ; "Sapgui stopped after commandline interp"...
.text:6440EE19                 push    dword_644F93E8
.text:6440EE1F                 call    FEWTraceError
\end{lstlisting}

Now, can we find the \emph{data record mode switched on} string?

Yes, and the only reference is in
\par \TT{CDwsGui::PrepareInfoWindow()}.

How do we get know the class/method names? There are a lot of special debugging calls that write to the log files, like:

\lstinputlisting[style=customasmx86]{examples/SAP/sapgui/02.lst}

\dots or:

\begin{lstlisting}[style=customasmx86]
.text:6440237A                 push    eax
.text:6440237B                 push    offset aCclientStart_6 ; "CClient::Start: set shortcut user to '\%"...
.text:64402380                 push    dword ptr [edi+4]
.text:64402383                 call    dbg
.text:64402388                 add     esp, 0Ch
\end{lstlisting}

It is \emph{very} useful.

So let's see the contents of this pesky annoying pop-up window's function:

\lstinputlisting[style=customasmx86]{examples/SAP/sapgui/03.lst}

At the start of the function \ECX has a pointer to the object (since it is a thiscall~(\myref{thiscall})-type of function).
In our case, the object obviously has class type of \emph{CDwsGui}. 
Depending on the option turned on in the object, a specific message part is to be concatenated with the resulting message.

If the value at address \TT{this+0x3D} is not zero, the compression is off:

\lstinputlisting[style=customasmx86]{examples/SAP/sapgui/04.lst}

It is interesting that finally the \emph{var\_10} variable state defines whether the message is to be shown at all:

\lstinputlisting[style=customasmx86]{examples/SAP/sapgui/1_EN.lst}

Let's check our theory on practice.

\JNZ at this line \dots

\lstinputlisting[style=customasmx86]{examples/SAP/sapgui/2_EN.lst}

\dots 
replace it with just \JMP, and we get SAPGUI working without the pesky annoying pop-up window appearing!

Now let's dig deeper and find a connection between the $0x15$ offset in the \TT{load\_command\_line()} 
(we gave it this name) function and the \TT{this+0x3D} variable in \emph{CDwsGui::PrepareInfoWindow}. 
Are we sure the value is the same?

We are starting to search for all occurrences of the \TT{0x15} value in code. 
For a small programs like SAPGUI, it sometimes works. Here is the first occurrence we've got:

\lstinputlisting[style=customasmx86]{examples/SAP/sapgui/3_EN.lst}

The function has been called from the function named \emph{CDwsGui::CopyOptions}! And thanks again for debugging information.

But the real answer is in \emph{CDwsGui::Init()}:

\lstinputlisting[style=customasmx86]{examples/SAP/sapgui/4_EN.lst}

Finally, we understand: the array filled in the \TT{load\_command\_line()} function is actually placed in the \emph{CDwsGui} class, 
but at address \TT{this+0x28}. 0x15 + 0x28 is exactly 0x3D. OK, we found the point where the value is copied to.

Let's also find the rest of the places where the 0x3D offset is used. 
Here is one of them in the \emph{CDwsGui::SapguiRun} function (again, thanks to the debugging calls):

\lstinputlisting[style=customasmx86]{examples/SAP/sapgui/5_EN.lst}

Let's check our findings. \\
Replace the \TT{setz al} here with the \TT{xor eax, eax / nop} instructions,
clear the \TDWNC 
environment variable and run SAPGUI. Wow! There pesky annoying window is no more (just as expected, 
because the variable is not set) but in Wireshark we can see that the network packets are not compressed anymore! 
Obviously, this is the point where the compression flag is to be set in the \emph{CConnectionContext} object.

So, the compression flag is passed in the 5th argument of \emph{CConnectionContext::CreateNetwork}. 
Inside the function, another one is called:

\begin{lstlisting}[style=customasmx86]
...
.text:64403476                 push    [ebp+compression]
.text:64403479                 push    [ebp+arg_C]
.text:6440347C                 push    [ebp+arg_8]
.text:6440347F                 push    [ebp+arg_4]
.text:64403482                 push    [ebp+arg_0]
.text:64403485                 call    CNetwork__CNetwork
\end{lstlisting}

The compression flag is passed here in the 5th argument to the \emph{CNetwork::CNetwork} constructor.

And here is how the \emph{CNetwork} constructor sets the flag in the \emph{CNetwork} object according to its 5th argument \emph{and}
another variable which probably could also affect network packets compression.

\lstinputlisting[style=customasmx86]{examples/SAP/sapgui/6_EN.lst}

At this point we know the compression flag is stored in the \emph{CNetwork} class at address \emph{this+0x3A4}.

Now let's dig through SAPguilib.dll for the \TT{0x3A4} value. And here is the second occurrence in 
\emph{CDwsGui::OnClientMessageWrite} (endless thanks for the debugging information):

\lstinputlisting[style=customasmx86]{examples/SAP/sapgui/7_EN.lst}

Let's take a look in \emph{sub\_644055C5}. In it we can only see the call to memcpy() and another function named 
(by \IDA) \emph{sub\_64417440}.

And, let's take a look inside \emph{sub\_64417440}. What we see is:

\lstinputlisting[style=customasmx86]{examples/SAP/sapgui/8_EN.lst}

Voilà! We've found the function that actually compresses the data.
As it was shown in past
\footnote{\url{http://go.yurichev.com/17312}},

this function is used in SAP and also the open-source MaxDB project. 
So it is available in source form.

Doing the last check here:

\begin{lstlisting}[style=customasmx86]
.text:64406F79                 cmp     dword ptr [ecx+3A4h], 1
.text:64406F80                 jnz     compression_flag_is_zero
\end{lstlisting}

Replace \JNZ here for an unconditional \JMP. Remove the environment variable \TDWNC. Voilà!

In Wireshark we see that the client messages are not compressed. The server responses, however, are compressed.

So we found exact connection between the environment variable and the point where data compression 
routine can be called or bypassed.

