\subsection{GCC}
\myindex{GCC}

\RU{Некоторые полезные опции, которые были использованы в книге.}%
\EN{Some useful options which were used through this book.}%
\DE{Einige nützliche Optionen die in diesem Buch genutzt werden.}%
\FR{Quelques options utiles qui ont été utilisées dans ce livre.}

\begin{center}
\begin{tabular}{ | l | l | }
\hline
\HeaderColor \RU{опция}\EN{option}\DE{Option}\FR{option} & 
\HeaderColor \RU{значение}\EN{meaning}\DE{Bedeutung}\FR{signification} \\
\hline
-Os		& \RU{оптимизация по размеру кода}\EN{code size optimization}\DE{Optimierung der Code-Größe}%
\FR{optimiser la taille du code} \\
-O3		& \RU{максимальная оптимизация}\EN{maximum optimization}\DE{maximale Optimierung}%
\FR{optimisation maximale} \\
-regparm=	& \RU{как много аргументов будет передаваться через регистры}
			\EN{how many arguments are to be passed in registers}\DE{Anzahl der in Registern übergebenen Argumente}%
			\FR{nombre d'arguments devant être passés dans les registres} \\
-o file		& \RU{задать имя выходного файла}\EN{set name of output file}\DE{Name der Ausgabedatei}%
			\FR{définir le nom du fichier de sortie} \\
-g		& \RU{генерировать отладочную информацию в итоговом исполняемом файле}%
			\EN{produce debugging information in resulting executable}%
			\DE{Debug-Informationen in der ausführbaren Datei erzeugen}%
			\FR{mettre l'information de débogage dans l'exécutable généré} \\
-S		& \RU{генерировать листинг на ассемблере}%
			\EN{generate assembly listing file}%
			\DE{Assembler-Quellcode erstellen}%
			\FR{générer un fichier assembleur} \\
-masm=intel	& \RU{генерировать листинг в синтаксисе Intel}%
			\EN{produce listing in Intel syntax}%
			\DE{Quellcode im Intel-Syntax erstellen}%
			\FR{construire le code source en syntaxe Intel} \\
-fno-inline	& \RU{не вставлять тело функции там, где она вызывается}%
			\EN{do not inline functions}%
			\DE{keine Inline-Funktionen verwenden}%
			\FR{ne pas mettre les fonctions en ligne} \\
\hline
\end{tabular}
\end{center}


