\subsection{Registres de débogage}

Ils sont utilisés pour le contrôle des points d'arrêt matériel (hardware breakpoints).

\begin{itemize}
	\item DR0 --- adresse du point d'arrêt \#1
	\item DR1 --- adresse du point d'arrêt \#2
	\item DR2 --- adresse du point d'arrêt \#3
	\item DR3 --- adresse du point d'arrêt \#4
	\item DR6 --- la cause de l'arrêt est indiquée ici
	\item DR7 --- les types de point d'arrêt sont mis ici
\end{itemize}

\subsubsection{DR6}
\myindex{x86!\Registers!DR6}

\begin{center}
\begin{tabular}{ | l | l | }
\hline
\headercolor\ Bit (masque) &
\headercolor\ Description \\
\hline
0 (1)       &  B0 --- le point d'arrêt \#1 a été déclenché \\
\hline
1 (2)       &  B1 --- le point d'arrêt \#2 a été déclenché \\
\hline
2 (4)       &  B2 --- le point d'arrêt \#3 a été déclenché \\
\hline
3 (8)       &  B3 --- le point d'arrêt \#4 a été déclenché \\
\hline
13 (0x2000) &  BD --- tentative de modification d'un des registres DRx.\\
            &  peut être déclenché si GD est activé \\
\hline
14 (0x4000) &  BS --- point d'arrêt simple (le flag TF a été mis dans EFLAGS). \\
	    &  La plus haute priorité. D'autres bits peuvent être mis aussi. \\
\hline
% TODO: describe BT
15 (0x8000) &  BT (task switch flag) \\
\hline
\end{tabular}
\end{center}

N.B.  Un point d'arrêt simple est un point d'arrêt qui se produit après chaque instruction.
Il peut être enclenché en mettant le flag TF dans EFLAGS (\myref{EFLAGS}).

\subsubsection{DR7}
\myindex{x86!\Registers!DR7}

Les types de point d'arrêt sont mis ici.

\small
\begin{center}
\begin{tabular}{ | l | l | }
\hline
\headercolor\ Bit (masque) &
\headercolor\ Description \\
\hline
0 (1)       &  L0 --- activer le point d'arrêt \#1 pour la tâche courante \\
\hline
1 (2)       &  G0 --- activer le point d'arrêt \#1 pour toutes les tâches \\
\hline
2 (4)       &  L1 --- activer le point d'arrêt \#2 pour la tâche courante \\
\hline
3 (8)       &  G1 --- activer le point d'arrêt \#2 pour toutes les tâches \\
\hline
4 (0x10)    &  L2 --- activer le point d'arrêt \#3 pour la tâche courante \\
\hline
5 (0x20)    &  G2 --- activer le point d'arrêt \#3 pour toutes les tâches \\
\hline
6 (0x40)    &  L3 --- activer le point d'arrêt \#4 pour la tâche courante \\
\hline
7 (0x80)    &  G3 --- activer le point d'arrêt \#4 pour toutes les tâches \\
\hline
8 (0x100)   &  LE --- non supporté depuis P6 \\
\hline
9 (0x200)   &  GE --- non supporté depuis P6 \\
\hline
13 (0x2000) &  GD --- exception déclenchée si une instruction MOV \\
            &  essaye de modifier un des registres DRx \\
\hline
16,17 (0x30000)    &  point d'arrêt \#1: R/W --- type \\
\hline
18,19 (0xC0000)    &  point d'arrêt \#1: LEN --- longueur \\
\hline
20,21 (0x300000)   &  point d'arrêt \#2: R/W --- type \\
\hline
22,23 (0xC00000)   &  point d'arrêt \#2: LEN --- longueur \\
\hline
24,25 (0x3000000)  &  point d'arrêt \#3: R/W --- type \\
\hline
26,27 (0xC000000)  &  point d'arrêt \#3: LEN --- longueur \\
\hline
28,29 (0x30000000) &  point d'arrêt \#4: R/W --- type \\
\hline
30,31 (0xC0000000) &  point d'arrêt \#4: LEN --- longueur \\
\hline
\end{tabular}
\end{center}
\normalsize

Le type de point d'arrêt doit être mis comme suit (R/W):

\begin{itemize}
\item 00 --- exécution de l'instruction
\item 01 --- écriture de données
\item 10 --- lecture ou écriture I/O (non disponible en mode user)
\item 11 --- à la lecture ou l'écriture de données
\end{itemize}

N.B.: le type de point d'arrêt est absent pour la lecture de données, en effet. \\
\\
La longueur du point d'arrêt est mise comme suit (LEN):

\begin{itemize}
\item 00 --- un octet
\item 01 --- deux octets
\item 10 --- non défini pour le mode 32-bit, huit octets en mode 64-bit
\item 11 --- quatre octets
\end{itemize}
