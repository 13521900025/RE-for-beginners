\subsection{\RU{Инструкции}\EN{Instructions}}
\label{sec:x86_instructions}

\RU{Инструкции, отмеченные как (M) обычно не генерируются компилятором: если вы видите её, очень может быть
это вручную написанный фрагмент кода, либо это т.н. compiler intrinsic}
\EN{Instructions marked as (M) are not usually generated by the compiler: if you see one of them, it is probably
a hand-written piece of assembly code, or a compiler intrinsic} (\myref{sec:compiler_intrinsic}).

% TODO ? обратные инструкции

\RU{Только наиболее используемые инструкции перечислены здесь}
\EN{Only the most frequently used instructions are listed here}.
\EN{You can read \myref{x86_manuals} for a full documentation.}%
\RU{Обращайтесь к \myref{x86_manuals} для полной документации.}

\RU{Нужно ли заучивать опкоды инструкций на память?}\EN{Do you have to know all instruction's opcodes by heart?}
\RU{Нет, только те, которые часто используются для модификации кода}\EN{No, only those
which are used for code patching} (\myref{x86_patching}).
\RU{Остальные запоминать нет смысла.}\EN{All the rest of the opcodes don't need to be memorized.}

\subsubsection{\RU{Префиксы}\EN{Prefixes}}

\myindex{x86!\Prefixes!LOCK}
\myindex{x86!\Prefixes!REP}
\myindex{x86!\Prefixes!REPE/REPNE}
\begin{description}
\label{x86_lock}
\item[LOCK] \RU{используется чтобы предоставить эксклюзивный доступ к памяти в многопроцессорной среде}
\EN{forces CPU to make exclusive access to the RAM in multiprocessor environment}.
\RU{Для упрощения, можно сказать, что когда исполняется инструкция с этим префиксом, остальные процессоры
в системе останавливаются}\EN{For the sake of simplification, it can be said that when an instruction
with this prefix is executed, all other CPUs in a multiprocessor system are stopped}.
\RU{Чаще все это используется для критических секций, семафоров, мьютексов}\EN{Most often
it is used for critical sections, semaphores, mutexes}.
\RU{Обычно используется с}\EN{Commonly used with} ADD, AND, BTR, BTS, CMPXCHG, OR, XADD, XOR.
\RU{Читайте больше о критических секциях}\EN{You can read more about critical sections here} (\myref{critical_sections}).

\item[REP] \RU{используется с инструкциями}\EN{is used with the} MOVSx \AndENRU STOSx\EN{ instructions}:
\RU{инструкция будет исполняться в цикле, счетчик расположен в регистре CX/ECX/RCX}
\EN{execute the instruction in a loop, the counter is located in the CX/ECX/RCX register}.
\RU{Для более детального описания, читайте больше об инструкциях}
\EN{For a detailed description, read more about the} MOVSx (\myref{REP_MOVSx}) 
\AndENRU STOSx (\myref{REP_STOSx})\EN{ instructions}.

\RU{Работа инструкций с префиксом REP зависит от флага DF, он задает направление}
\EN{The instructions prefixed by REP are sensitive to the DF flag, which is used to set the direction}.

\item[REPE/REPNE] (\ac{AKA} REPZ/REPNZ) \RU{используется с инструкциями}\EN{used with} CMPSx \AndENRU
SCASx\EN{ instructions}:
\RU{инструкция будет исполняться в цикле, счетчик расположен в регистре \TT{CX}/\TT{ECX}/\TT{RCX}}
\EN{execute the last instruction in a loop, the count is set in the \TT{CX}/\TT{ECX}/\TT{RCX} register}. 
\RU{Выполнение будет прервано если ZF будет 0 (REPE) либо если ZF будет 1 (REPNE)}
\EN{It terminates prematurely if ZF is 0 (REPE) or if ZF is 1 (REPNE)}.

\RU{Для более детального описания, читайте больше об инструкциях}
\EN{For a detailed description, you can read more about the} CMPSx (\myref{REPE_CMPSx}) 
\AndENRU SCASx (\myref{REPNE_SCASx})\EN{ instructions}.

\RU{Работа инструкций с префиксами REPE/REPNE зависит от флага DF, он задает направление}
\EN{Instructions prefixed by REPE/REPNE are sensitive to the DF flag, which is used to set the direction}.

\end{description}

\subsubsection{\RU{Наиболее часто используемые инструкции}\EN{Most frequently used instructions}}

\RU{Их можно заучить в первую очередь}\EN{These can be memorized in the first place}.

\begin{description}
% in order to keep them easily sorted...
\myindex{x86!\Instructions!ADC}
\myindex{x86!\Flags!CF}
  \item[ADC] (\emph{add with carry}) \RU{сложить два значения, \glslink{increment}{инкремент} 
  если выставлен флаг CF.
  ADC часто используется для складывания больших значений, например, складывания двух 64-битных
  значений в 32-битной среде используя две инструкции ADD и ADC, например:}
  \EN{add values, \gls{increment} the result if the CF flag is set.
  ADC is often used for the addition of large values, for example, 
  to add two 64-bit values in a 32-bit environment using two ADD and ADC instructions. For example:}

\EN{\lstinputlisting[style=customasmx86]{appendix/x86/instructions/ADC_example_EN.lst}}
\RU{\lstinputlisting[style=customasmx86]{appendix/x86/instructions/ADC_example_RU.lst}}

\RU{Еще один пример}\EN{One more example}: \myref{sec:64bit_in_32_env}.

\input{appendix/x86/instructions/ADD}
\input{appendix/x86/instructions/AND}
\input{appendix/x86/instructions/CALL}
\input{appendix/x86/instructions/CMP}
\myindex{x86!\Instructions!DEC}
\myindex{x86!\Flags!CF}
  \item[DEC] \gls{decrement}.
\RU{В отличие от других арифметических инструкций, \TT{DEC} не модифицирует флаг CF.}%
\EN{Unlike other arithmetic instructions, \TT{DEC} doesn't modify CF flag.}%
\FR{Contrairement aux autres instructions arithmétiques, \TT{DEC} ne modifie pas
le flag CF.}

\input{appendix/x86/instructions/IMUL}
\input{appendix/x86/instructions/INC}
\input{appendix/x86/instructions/JCXZ}
\input{appendix/x86/instructions/JMP}
\item[Jcc] (\RU{где}\EN{where} cc --- condition code)

\myindex{x86!\Instructions!JAE}
\myindex{x86!\Instructions!JA}
\myindex{x86!\Instructions!JBE}
\myindex{x86!\Instructions!JB}
\myindex{x86!\Instructions!JC}
\myindex{x86!\Instructions!JE}
\myindex{x86!\Instructions!JGE}
\myindex{x86!\Instructions!JG}
\myindex{x86!\Instructions!JLE}
\myindex{x86!\Instructions!JL}
\myindex{x86!\Instructions!JNAE}
\myindex{x86!\Instructions!JNA}
\myindex{x86!\Instructions!JNBE}
\myindex{x86!\Instructions!JNB}
\myindex{x86!\Instructions!JNC}
\myindex{x86!\Instructions!JNE}
\myindex{x86!\Instructions!JNGE}
\myindex{x86!\Instructions!JNG}
\myindex{x86!\Instructions!JNLE}
\myindex{x86!\Instructions!JNL}
\myindex{x86!\Instructions!JNO}
\myindex{x86!\Instructions!JNS}
\myindex{x86!\Instructions!JNZ}
\myindex{x86!\Instructions!JO}
\myindex{x86!\Instructions!JPO}
\myindex{x86!\Instructions!JP}
\myindex{x86!\Instructions!JS}
\myindex{x86!\Instructions!JZ}

\RU{Немало этих инструкций имеют синонимы (отмечены с AKA), это сделано для удобства}
\EN{A lot of these instructions have synonyms (denoted with AKA), this was done for convenience}.
\RU{Синонимичные инструкции транслируются в один и тот же опкод}
\EN{Synonymous instructions are translated into the same opcode}.
\RU{Опкод имеет т.н.}\EN{The opcode has a} \gls{jump offset}.

\label{Jcc}
\begin{description}
\item[JAE] \ac{AKA} JNC: \RU{переход если больше или равно (беззнаковый)}\EN{jump if above or equal (unsigned)}: CF=0
\EN{\item[JA] \ac{AKA} JNBE: jump if greater (unsigned): CF=0 and ZF=0}
\RU{\item[JA] \ac{AKA} JNBE: переход если больше (беззнаковый): CF=0 и ZF=0}
\EN{\item[JBE] jump if lesser or equal (unsigned): CF=1 or ZF=1}
\RU{\item[JBE] переход если меньше или равно (беззнаковый): CF=1 или ZF=1}
\item[JB] \ac{AKA} JC: \RU{переход если меньше (беззнаковый)}\EN{jump if below (unsigned)}: CF=1
\item[JC] \ac{AKA} JB: \RU{переход если CF=1}\EN{jump if CF=1}
\item[JE] \ac{AKA} JZ: \RU{переход если равно или ноль}\EN{jump if equal or zero}: ZF=1
\item[JGE] \RU{переход если больше или равно (знаковый)}\EN{jump if greater or equal (signed)}: SF=OF
\EN{\item[JG] jump if greater (signed): ZF=0 and SF=OF}
\RU{\item[JG] переход если больше (знаковый): ZF=0 и SF=OF}
\EN{\item[JLE] jump if lesser or equal (signed): ZF=1 or SF$\neq$OF}
\RU{\item[JLE] переход если меньше или равно (знаковый): ZF=1 или SF$\neq$OF}
\item[JL] \RU{переход если меньше (знаковый)}\EN{jump if lesser (signed)}: SF$\neq$OF
\item[JNAE] \ac{AKA} JC: \RU{переход если не больше или равно (беззнаковый)}\EN{jump if not above or equal (unsigned)} CF=1
\item[JNA] \RU{переход если не больше (беззнаковый)}\EN{jump if not above (unsigned)} CF=1 \AndENRU ZF=1
\item[JNBE] \RU{переход если не меньше или равно (беззнаковый)}\EN{jump if not below or equal (unsigned)}: CF=0 \AndENRU ZF=0
\item[JNB] \ac{AKA} JNC: \RU{переход если не меньше (беззнаковый)}\EN{jump if not below (unsigned)}: CF=0
\item[JNC] \ac{AKA} JAE: \RU{переход если CF=0, синонимично}\EN{jump CF=0 synonymous to} JNB.
\item[JNE] \ac{AKA} JNZ: \RU{переход если не равно или не ноль}\EN{jump if not equal or not zero}: ZF=0
\item[JNGE] \RU{переход если не больше или равно (знаковый)}\EN{jump if not greater or equal (signed)}: SF$\neq$OF
\EN{\item[JNG] jump if not greater (signed): ZF=1 or SF$\neq$OF}
\RU{\item[JNG] переход если не больше (знаковый): ZF=1 или SF$\neq$OF}
\item[JNLE] \RU{переход если не меньше (знаковый)}\EN{jump if not lesser (signed)}: ZF=0 \AndENRU SF=OF
\item[JNL] \RU{переход если не меньше (знаковый)}\EN{jump if not lesser (signed)}: SF=OF
\item[JNO] \RU{переход если не переполнение}\EN{jump if not overflow}: OF=0
\item[JNS] \RU{переход если флаг SF сброшен}\EN{jump if SF flag is cleared}
\item[JNZ] \ac{AKA} JNE: \RU{переход если не равно или не ноль}\EN{jump if not equal or not zero}: ZF=0
\item[JO] \RU{переход если переполнение}\EN{jump if overflow}: OF=1
\item[JPO] \RU{переход если сброшен флаг PF}\EN{jump if PF flag is cleared} (Jump Parity Odd)
\item[JP] \ac{AKA} \ac{JPE}: \RU{переход если выставлен флаг PF}\EN{jump if PF flag is set}
\item[JS] \RU{переход если выставлен флаг SF}\EN{jump if SF flag is set}
\item[JZ] \ac{AKA} JE: \RU{переход если равно или ноль}\EN{jump if equal or zero}: ZF=1
\end{description}


\input{appendix/x86/instructions/LAHF}
\myindex{x86!\Instructions!LEAVE}
\label{x86_ins:LEAVE}
\item[LEAVE] \RU{аналог команд \TT{MOV ESP, EBP} и \TT{POP EBP} --- 
то есть возврат \glslink{stack pointer}{указателя стека} и регистра \EBP в первоначальное состояние.}%
\EN{equivalent of the \TT{MOV ESP, EBP} and \TT{POP EBP} instruction
pair --- in other words, this instruction sets the \gls{stack pointer} (\ESP) back and restores
the \EBP register to its initial state.}%
\FR{équivalente à la paire d'instructions  \TT{MOV ESP, EBP} et \TT{POP EBP}  --- autrement dit,
cette instruction remet le \glslink{stack pointer}{pointeur de pile} et restaure le registre
\EBP à l'état initial.}


\myindex{x86!\Instructions!LEA}
\item[LEA] (\emph{Load Effective Address}) \RU{сформировать адрес}\EN{form an address}

\label{sec:LEA}

\newcommand{\URLAM}{\href{http://go.yurichev.com/17109}{wikipedia}}

\RU{Это инструкция, которая задумывалась вовсе не для складывания 
и умножения чисел, 
а для формирования адреса например, из указателя на массив и прибавления индекса к нему
\footnote{См. также: \URLAM}.}
\EN{This instruction was intended not for summing values and multiplication 
but for forming an address, 
e.g., for calculating the address of an array element by adding the array address, element index, with 
multiplication of element size\footnote{See also: \URLAM}.}
\par
\RU{То есть, разница между \MOV и \LEA в том, что \MOV формирует адрес в памяти 
и загружает значение из памяти, либо записывает его туда, а \LEA только формирует адрес.}
\EN{So, the difference between \MOV and \LEA is that \MOV forms a memory address and loads a value
from memory or stores it there, but \LEA just forms an address.}
\par
\RU{Тем не менее, её можно использовать для любых других вычислений}
\EN{But nevertheless, it is can be used for any other calculations}.
\par
\RU{\LEA удобна тем, что производимые ею вычисления не модифицируют флаги \ac{CPU}. 
Это может быть очень важно для \ac{OOE} процессоров (чтобы было меньше зависимостей между данными).}
\EN{\LEA is convenient because the computations performed by it does not alter \ac{CPU} flags.
This may be very important for \ac{OOE} processors (to create less data dependencies).}

\RU{Помимо всего прочего, начиная минимум с Pentium, инструкция LEA исполняется за 1 такт.}%
\EN{Aside from this, starting at least at Pentium, LEA instruction is executed in 1 cycle.}

\begin{lstlisting}[style=customc]
int f(int a, int b)
{
	return a*8+b;
};
\end{lstlisting}

\begin{lstlisting}[caption=\Optimizing MSVC 2010,style=customasmx86]
_a$ = 8		; size = 4
_b$ = 12	; size = 4
_f	PROC
	mov	eax, DWORD PTR _b$[esp-4]
	mov	ecx, DWORD PTR _a$[esp-4]
	lea	eax, DWORD PTR [eax+ecx*8]
	ret	0
_f	ENDP
\end{lstlisting}

\myindex{Intel C++}
Intel C++ \RU{использует LEA даже больше}\EN{uses LEA even more}:

\begin{lstlisting}[style=customc]
int f1(int a)
{
	return a*13;
};
\end{lstlisting}

\begin{lstlisting}[caption=Intel C++ 2011,style=customasmx86]
_f1	PROC NEAR 
        mov       ecx, DWORD PTR [4+esp]      ; ecx = a
	lea       edx, DWORD PTR [ecx+ecx*8]  ; edx = a*9
	lea       eax, DWORD PTR [edx+ecx*4]  ; eax = a*9 + a*4 = a*13
        ret                                
\end{lstlisting}

\RU{Эти две инструкции вместо одной IMUL будут работать быстрее.}
\EN{These two instructions performs faster than one IMUL.}


\input{appendix/x86/instructions/MOVSB_W_D_Q}
\myindex{x86!\Instructions!MOVSX}
  \item[MOVSX] \RU{загрузить с расширением знака}\EN{load with sign extension}\FR{charger avec extension du signe} %
  \RU{см. также}\EN{see also}\FR{voir aussi}: (\myref{MOVSX})

\input{appendix/x86/instructions/MOVZX}
\input{appendix/x86/instructions/MOV}
\input{appendix/x86/instructions/MUL}
\input{appendix/x86/instructions/NEG}
\input{appendix/x86/instructions/NOP}
\input{appendix/x86/instructions/NOT}
\input{appendix/x86/instructions/OR}
\input{appendix/x86/instructions/POP}
\input{appendix/x86/instructions/PUSH}
\myindex{x86!\Instructions!RET}
\myindex{MS-DOS}
\item[RET] \RU{возврат из процедуры}\EN{return from subroutine}: \TT{POP tmp; JMP tmp}.

\RU{В реальности}\EN{In fact}, RET 
\RU{это макрос ассемблера, в среде Windows и *NIX транслирующийся в}
\EN{is an assembly language macro, in Windows and *NIX environment it is translated into}
RETN (\q{return near}) 
\RU{ибо, во времена MS-DOS, где память адресовалась немного иначе}
\EN{or, in MS-DOS times, where the memory was addressed differently}
(\myref{8086_memory_model}), \RU{в}\EN{into} RETF (\q{return far}).

\RU{\TT{RET} может иметь операнд.}\EN{\TT{RET} can have an operand.}
\RU{Тогда его работа будет такой}\EN{Then it works like this}:\\
\TT{POP tmp; ADD ESP op1; JMP tmp}.
\TT{RET} \RU{с операндом обычно завершает функции с соглашением о вызовах \emph{stdcall}, см. также}
\EN{with an operand usually ends functions in the \emph{stdcall} calling convention, see also}: \myref{sec:stdcall}.


\input{appendix/x86/instructions/SAHF}
\myindex{x86!\Instructions!SBB}
\myindex{x86!\Flags!CF}
  \item[SBB] (\emph{subtraction with borrow}) 
  \RU{вычесть одно значение из другого, \glslink{decrement}{декремент} результата если флаг CF выставлен.
  SBB часто используется для вычитания больших значений, например, для вычитания двух 64-битных
  значений в 32-битной среде используя инструкции SUB и SBB, например:}
  \EN{subtract values, \gls{decrement} the result if the CF flag is set.
  SBB is often used for subtraction of large values, for example,
  to subtract two 64-bit values in 32-bit environment using two SUB and SBB instructions. For example:}
  \FR{soustrait les valeurs, \glslink{decrement}{décrémente} le résultat si le flag
  CF est mis. SBB est souvent utilisé pour la soustraction de grandes valeurs, par
  exemple:}

\EN{\lstinputlisting[style=customasmx86]{appendix/x86/instructions/SBB_example_EN.lst}}
\RU{\lstinputlisting[style=customasmx86]{appendix/x86/instructions/SBB_example_RU.lst}}
\FR{\lstinputlisting[style=customasmx86]{appendix/x86/instructions/SBB_example_FR.lst}}

\RU{Еще один пример}\EN{One more example}\FR{Un autre exemple}: \myref{sec:64bit_in_32_env}.

\input{appendix/x86/instructions/SCASB_W_D_Q}
\input{appendix/x86/instructions/SHx}
\input{appendix/x86/instructions/SHRD}
\input{appendix/x86/instructions/STOSB_W_D_Q}
\myindex{x86!\Instructions!SUB}
  \item[SUB] \RU{вычесть одно значение из другого. 
  часто встречающийся вариант \TT{SUB reg,reg} означает обнуление \emph{reg}.}
  \EN{subtract values. 
  A frequently occurring pattern is \TT{SUB reg,reg}, which implies zeroing of \emph{reg}.}
  \FR{soustrait des valeurs.
  Une utilisation fréquente est \TT{SUB reg,reg}, qui met \emph{reg} à zéro.}

\input{appendix/x86/instructions/TEST}
\myindex{x86!\Instructions!XOR}
  \item[XOR] op1, op2: \ac{XOR} \RU{значений}\EN{values}\FR{valeurs}. $op1=op1\oplus{}op2$.
  \RU{Часто встречающийся вариант \TT{XOR reg,reg} означает обнуление регистра \emph{reg}.}
  \EN{A frequently occurring pattern is \TT{XOR reg,reg}, which implies zeroing of \emph{reg}.}
  \FR{Un schéma récurrent est \TT{XOR reg,reg}, qui met \emph{reg} à zéro.}
  \EN{See also}\RU{См.также}\FR{Voir aussi}: \myref{XOR_property}.


\end{description}

\subsubsection{\RU{Реже используемые инструкции}\EN{Less frequently used instructions}}

\begin{description}
\myindex{x86!\Instructions!BSF}
  \item[BSF] \emph{bit scan forward}, \RU{см. также}\EN{see also}\FR{voir aussi}: \myref{instruction_BSF}

\myindex{x86!\Instructions!BSR}
  \item[BSR] \emph{bit scan reverse}

\myindex{x86!\Instructions!BSWAP}
  \item[BSWAP] \emph{(byte swap)}, \RU{смена \glslink{endianness}{порядка байт} в значении}\EN{change value \gls{endianness}}%
  \FR{change le \glslink{endianness}{boutisme} de la valeur}.

\input{appendix/x86/instructions/BTC}
\input{appendix/x86/instructions/BTR}
\input{appendix/x86/instructions/BTS}
\input{appendix/x86/instructions/BT}
\myindex{x86!\Instructions!CBW}
\myindex{x86!\Instructions!CWD}
\myindex{x86!\Instructions!CDQ}
\myindex{x86!\Instructions!CWDE}
\myindex{x86!\Instructions!CDQE}
\label{ins:CBW_CWD_etc}
\item[CBW/CWD/CWDE/CDQ/CDQE]

\RU{Расширить значение учитывая его знак}\EN{Sign-extend value}:

\begin{description}
\item[CBW] \RU{конвертировать байт в AL в слово в AX}\EN{convert byte in AL to word in AX}
\item[CWD] \RU{конвертировать слово в AX в двойное слово в DX:AX}\EN{convert word in AX to doubleword in DX:AX} 
\item[CWDE] \RU{конвертировать слово в AX в двойное слово в EAX}\EN{convert word in AX to doubleword in EAX} 
\item[CDQ] \RU{конвертировать двойное слово в EAX в четверное слово в EDX:EAX}\EN{convert doubleword in EAX to quadword in EDX:EAX}
\item[CDQE] (x64) \RU{конвертировать двойное слово в EAX в четверное слово в RAX}\EN{convert doubleword in EAX to quadword in RAX}
\end{description}

\RU{Эти инструкции учитывают знак значения, расширяя его в старшую часть выходного
значения. См. также:}
\EN{These instructions consider the value's sign, extending it to high part of the newly constructed 
value. See also:} \myref{subsec:sign_extending_32_to_64}.

\newcommand{\StephenMorse}{[Stephen P. Morse, \emph{The 8086 Primer}, (1980)]\footnote{\AlsoAvailableAs \url{https://archive.org/details/The8086Primer}}}

\EN{Interestingly to know these instructions was initially named as \TT{SEX} (\emph{Sign EXtend}), 
as Stephen P. Morse (one of Intel 8086 CPU designers) wrote in \StephenMorse:}
\RU{Интересно узнать, что эти инструкции назывались \TT{SEX} (\emph{Sign EXtend}),
как Stephen P. Morse (один из создателей Intel 8086 CPU) пишет в \StephenMorse:}

\begin{framed}
\begin{quotation}
The process of stretching numbers by extending the sign bit is called sign extension. 
The 8086 provides instructions (Fig. 3.29) to facilitate the task of sign extension. 
These instructions were initially named SEX (sign extend) but were later renamed to the more 
conservative CBW (convert byte to word) and CWD (convert word to double word).
\end{quotation}
\end{framed}

\myindex{x86!\Instructions!CLD}
\myindex{x86!\Flags!DF}
  \item[CLD] \RU{сбросить флаг DF}\EN{clear DF flag}\FR{efface le flag DF}.

\myindex{x86!\Instructions!CLI}
\myindex{x86!\Flags!IF}
  \item[CLI] (M) \RU{сбросить флаг IF}\EN{clear IF flag}\FR{efface le flag IF}.

\input{appendix/x86/instructions/CMC}
\input{appendix/x86/instructions/CMOVcc}
\input{appendix/x86/instructions/CMPSB_W_D_Q}
\input{appendix/x86/instructions/CPUID}
\input{appendix/x86/instructions/DIV}
\input{appendix/x86/instructions/IDIV}
\myindex{x86!\Instructions!INT}
\myindex{MS-DOS}

\item[INT] (M): \INS{INT x} \RU{аналогична}\EN{is analogous to} \INS{PUSHF; CALL dword ptr [x*4]} 
\RU{в 16-битной среде}\EN{in 16-bit environment}.
  \RU{Она активно использовалась в MS-DOS, работая как сисколл. Аргументы записывались в регистры
  AX/BX/CX/DX/SI/DI и затем происходил переход на таблицу векторов прерываний (расположенную в самом
  начале адресного пространства)}
  \EN{It was widely used in MS-DOS, functioning as a syscall vector. The registers AX/BX/CX/DX/SI/DI were filled
  with the arguments and then the flow jumped to the address in the Interrupt Vector Table 
  (located at the beginning of the address space)}.
  \RU{Она была очень популярна потому что имела короткий опкод (2 байта) и программе использующая
  сервисы MS-DOS не нужно было заморачиваться узнавая адреса всех функций этих сервисов}
  \EN{It was popular because INT has a short opcode (2 bytes) and the program which needs
  some MS-DOS services is not bother to determine the address of the service's entry point}.
\myindex{x86!\Instructions!IRET}
  \RU{Обработчик прерываний возвращал управление назад при помощи инструкции IRET}
  \EN{The interrupt handler returns the control flow to caller using the IRET instruction}.

  \RU{Самое используемое прерывание в MS-DOS было 0x21, там была основная часть его \ac{API}}
  \EN{The most busy MS-DOS interrupt number was 0x21, serving a huge part of its \ac{API}}.
  \RU{См. также}\EN{See also}: [Ralf Brown \emph{Ralf Brown's Interrupt List}], 
  \RU{самый крупный список всех известных прерываний и вообще там много информации о MS-DOS}
  \EN{for the most comprehensive interrupt lists and other MS-DOS information}.

\myindex{x86!\Instructions!SYSENTER}
\myindex{x86!\Instructions!SYSCALL}
  \RU{Во времена после MS-DOS, эта инструкция все еще использовалась как сискол, и в Linux
  и в Windows (\myref{syscalls}), но позже была заменена инструкцией SYSENTER или SYSCALL}
  \EN{In the post-MS-DOS era, this instruction was still used as syscall both in Linux and 
  Windows (\myref{syscalls}), but was later replaced by the SYSENTER or SYSCALL instructions}.

\item[INT 3] (M): \RU{эта инструкция стоит немного в стороне от}\EN{this instruction is somewhat close to} 
\INS{INT}, \RU{она имеет собственный 1-байтный опкод}\EN{it has its own 1-byte opcode} (\GTT{0xCC}), 
\RU{и активно используется в отладке}\EN{and is actively used while debugging}.
\RU{Часто, отладчик просто записывает байт}\EN{Often, the debuggers just write the} \GTT{0xCC} 
\RU{по адресу в памяти где устанавливается точка останова, и когда исключение поднимается, оригинальный байт
будет восстановлен и оригинальная инструкция по этому адресу исполнена заново}\EN{byte at the address of 
the breakpoint to be set, and when an exception is raised,
the original byte is restored and the original instruction at this address is re-executed}. \\
\RU{В}\EN{As of} \gls{Windows NT}, \RU{исключение}\EN{an} \GTT{EXCEPTION\_BREAKPOINT} \RU{поднимается,
когда \ac{CPU} исполняет эту инструкцию}\EN{exception is to be raised when the \ac{CPU} executes this instruction}.
\RU{Это отладочное событие может быть перехвачено и обработано отладчиком, если он загружен}
\EN{This debugging event may be intercepted and handled by a host debugger, if one is loaded}.
\RU{Если он не загружен, Windows предложит запустить один из зарегистрированных в системе отладчиков}
\EN{If it is not loaded, Windows offers to run one of the registered system debuggers}.
\RU{Если}\EN{If} \ac{MSVS} \RU{установлена, его отладчик может быть загружен и подключен к процессу}\EN{is installed, 
its debugger may be loaded and connected to the process}.
\RU{В целях защиты от}\EN{In order to protect from} \gls{reverse engineering}, \RU{множество анти-отладочных методов
проверяют целостность загруженного кода}\EN{a lot of anti-debugging methods check integrity of the loaded code}.

\RU{В }\ac{MSVC} \RU{есть}\EN{has} \gls{compiler intrinsic} \RU{для этой инструкции}\EN{for the instruction}:
\GTT{\_\_debugbreak()}\footnote{\href{http://go.yurichev.com/17226}{MSDN}}.

\RU{В win32 также имеется функция в}\EN{There is also a win32 function in} kernel32.dll \RU{с названием}\EN{named}
\GTT{DebugBreak()}\footnote{\href{http://go.yurichev.com/17227}{MSDN}},
\RU{которая также исполняет}\EN{which also executes} \GTT{INT 3}.


\myindex{x86!\Instructions!IN}
\myindex{MS-DOS}
  \item[IN] (M) \RU{получить данные из порта}\EN{input data from port}\FR{lire des données depuis le port}.
	  \RU{Эту инструкцию обычно можно найти в драйверах OS либо в старом коде для MS-DOS,
	  например}%
	  \EN{The instruction usually can be seen in OS drivers or in old MS-DOS code,
	  for example}%
	  \FR{On trouve cette instruction dans les drivers de l'OS ou dans de l'ancien
	  code MS-DOS, par exemple} (\myref{IN_example}).

\input{appendix/x86/instructions/IRET}
\input{appendix/x86/instructions/LOOP}
\input{appendix/x86/instructions/OUT}
\input{appendix/x86/instructions/POPA}
\input{appendix/x86/instructions/POPCNT}
\input{appendix/x86/instructions/POPF}
\input{appendix/x86/instructions/PUSHA}
\input{appendix/x86/instructions/PUSHF}
\input{appendix/x86/instructions/RCx}
\myindex{x86!\Instructions!ROL}
\myindex{x86!\Instructions!ROR}
\label{ROL_ROR}
\item[ROL/ROR] (M) \RU{циклический сдвиг}\EN{cyclic shift}
  
ROL: \RU{вращать налево}\EN{rotate left}:

\input{rotate_left}

ROR: \RU{вращать направо}\EN{rotate right}:

\input{rotate_right}

\RU{Не смотря на то что многие \ac{CPU} имеют эти инструкции, в \CCpp нет соответствующих операций,
так что компиляторы с этих \ac{PL} обычно не генерируют код использующий эти инструкции}\EN{Despite the 
fact that almost all \ac{CPU}s have these instructions, there are no corresponding
operations in \CCpp, so the compilers of these \ac{PL}s usually do not generate these 
instructions}.

\RU{Чтобы программисту были доступны эти инструкции, в \ac{MSVC} есть псевдофункции}
\EN{For the programmer's convenience, at least \ac{MSVC} has the pseudofunctions} (compiler intrinsics)
\emph{\_rotl()} \AndENRU \emph{\_rotr()}\FNMSDNROTxURL{},
\RU{которые транслируются компилятором напрямую в эти инструкции}
\EN{which are translated by the compiler directly to these instructions}.


\myindex{x86!\Instructions!SAL}
  \item[SAL] \RU{Арифметический сдвиг влево}\EN{Arithmetic shift left}\FR{Décalage
  arithmétique à gauche}, \RU{синонимично}\EN{synonymous to}\FR{synonyme de} \TT{SHL}

\input{appendix/x86/instructions/SAR}
\input{appendix/x86/instructions/SETcc}
\input{appendix/x86/instructions/STC}
\input{appendix/x86/instructions/STD}
\input{appendix/x86/instructions/STI}
\input{appendix/x86/instructions/SYSCALL}
\input{appendix/x86/instructions/SYSENTER}
\input{appendix/x86/instructions/UD2}
\input{appendix/x86/instructions/XCHG}
\end{description}

\subsubsection{\RU{Инструкции FPU}\EN{FPU instructions}}

\RU{Суффикс \TT{-R} в названии инструкции обычно означает, что операнды поменяны местами, суффикс \TT{-P} означает
что один элемент выталкивается из стека после исполнения инструкции, суффикс \TT{-PP} означает, что
выталкиваются два элемента}%
\EN{\TT{-R} suffix in the mnemonic usually implies that the operands are reversed,
\TT{-P} suffix implies that one element is popped
from the stack after the instruction's execution, \TT{-PP} suffix implies that two elements are popped}.

\TT{-P} \RU{инструкции часто бывают полезны, когда нам уже больше не нужно хранить значение в 
FPU-стеке после операции.}%
\EN{instructions are often useful when we do not need the value in the FPU stack to be 
present anymore after the operation.}

\begin{description}
\input{appendix/x86/instructions/FABS}
\input{appendix/x86/instructions/FADD} % + FADDP
\input{appendix/x86/instructions/FCHS}
\input{appendix/x86/instructions/FCOM} % + FCOMP + FCOMPP
\input{appendix/x86/instructions/FDIVR} % + FDIVRP
\input{appendix/x86/instructions/FDIV} % + FDIVP
\input{appendix/x86/instructions/FILD}
\input{appendix/x86/instructions/FIST} % + FISTP
\input{appendix/x86/instructions/FLD1}
\input{appendix/x86/instructions/FLDCW}
\input{appendix/x86/instructions/FLDZ}
\input{appendix/x86/instructions/FLD}
\input{appendix/x86/instructions/FMUL} % + FMULP
\input{appendix/x86/instructions/FSINCOS}
\input{appendix/x86/instructions/FSQRT}
\input{appendix/x86/instructions/FSTCW} % + FNSTCW
\input{appendix/x86/instructions/FSTSW} % + FNSTSW
\input{appendix/x86/instructions/FST}
\input{appendix/x86/instructions/FSUBR} % + FSUBRP
\input{appendix/x86/instructions/FSUB} % + FSUBP
\input{appendix/x86/instructions/FUCOM} % + FUCOMP + FUCOMPP
\input{appendix/x86/instructions/FXCH}
\end{description}

%\subsubsection{\RU{SIMD-инструкции}\EN{SIMD instructions}}

% TODO

%\begin{description}
%\input{appendix/x86/instructions/DIVSD}
%\input{appendix/x86/instructions/MOVDQA}
%\input{appendix/x86/instructions/MOVDQU}
%\input{appendix/x86/instructions/PADDD}
%\input{appendix/x86/instructions/PCMPEQB}
%\input{appendix/x86/instructions/PLMULHW}
%\input{appendix/x86/instructions/PLMULLD}
%\input{appendix/x86/instructions/PMOVMSKB}
%\input{appendix/x86/instructions/PXOR}
%\end{description}

% SHLD !
% SHRD !
% BSWAP !
% CMPXCHG
% XADD !
% CMPXCHG8B
% RDTSC !
% PAUSE!

% xsave
% fnclex, fnsave
% movsxd, movaps, wait, sfence, lfence, pushfq
% prefetchw
% REP RETN
% REP BSF
% movnti, movntdq, rdmsr, wrmsr
% ldmxcsr, stmxcsr, invlpg
% swapgs
% movq, movd
% mulsd
% POR
% IRETQ
% pslldq
% psrldq
% cqo, fxrstor, comisd, xrstor, wbinvd, movntq
% fprem
% addsb, subsd, frndint

% rare:
%\item[ENTER]
%\item[LES]
% LDS
% XLAT

\subsubsection{\RU{Инструкции с печатаемым ASCII-опкодом}\EN{Instructions having printable ASCII opcode}}

(\RU{В 32-битном режиме}\EN{In 32-bit mode}).

\label{printable_x86_opcodes}
\myindex{Shellcode}
\RU{Это может пригодиться для создания шеллкодов}\EN{These can be suitable for shellcode construction}.
\RU{См. также}\EN{See also}: \myref{subsec:EICAR}.

% FIXME: start at 0x20...
\begin{center}
\begin{longtable}{ | l | l | l | }
\hline
\HeaderColor ASCII\RU{-символ}\EN{ character} & 
\HeaderColor \RU{шестнадцатеричный код}\EN{hexadecimal code} & 
\HeaderColor x86\RU{-инструкция}\EN{ instruction} \\
\hline
0	 &30	 &XOR \\
1	 &31	 &XOR \\
2	 &32	 &XOR \\
3	 &33	 &XOR \\
4	 &34	 &XOR \\
5	 &35	 &XOR \\
7	 &37	 &AAA \\
8	 &38	 &CMP \\
9	 &39	 &CMP \\
:	 &3a	 &CMP \\
;	 &3b	 &CMP \\
<	 &3c	 &CMP \\
=	 &3d	 &CMP \\
?	 &3f	 &AAS \\
@	 &40	 &INC \\
A	 &41	 &INC \\
B	 &42	 &INC \\
C	 &43	 &INC \\
D	 &44	 &INC \\
E	 &45	 &INC \\
F	 &46	 &INC \\
G	 &47	 &INC \\
H	 &48	 &DEC \\
I	 &49	 &DEC \\
J	 &4a	 &DEC \\
K	 &4b	 &DEC \\
L	 &4c	 &DEC \\
M	 &4d	 &DEC \\
N	 &4e	 &DEC \\
O	 &4f	 &DEC \\
P	 &50	 &PUSH \\
Q	 &51	 &PUSH \\
R	 &52	 &PUSH \\
S	 &53	 &PUSH \\
T	 &54	 &PUSH \\
U	 &55	 &PUSH \\
V	 &56	 &PUSH \\
W	 &57	 &PUSH \\
X	 &58	 &POP \\
Y	 &59	 &POP \\
Z	 &5a	 &POP \\
\lbrack{}	 &5b	 &POP \\
\textbackslash{}	 &5c	 &POP \\
\rbrack{}	 &5d	 &POP \\
\verb|^|	 &5e	 &POP \\
\_	 &5f	 &POP \\
\verb|`|	 &60	 &PUSHA \\
a	 &61	 &POPA \\

h	 &68	 &PUSH\\
i	 &69	 &IMUL\\
j	 &6a	 &PUSH\\
k	 &6b	 &IMUL\\
p	 &70	 &JO\\
q	 &71	 &JNO\\
r	 &72	 &JB\\
s	 &73	 &JAE\\
t	 &74	 &JE\\
u	 &75	 &JNE\\
v	 &76	 &JBE\\
w	 &77	 &JA\\
x	 &78	 &JS\\
y	 &79	 &JNS\\
z	 &7a	 &JP\\
\hline
\end{longtable}
\end{center}

\RU{А также}\EN{Also}:

\begin{center}
\begin{longtable}{ | l | l | l | }
\hline
\HeaderColor ASCII\RU{-символ}\EN{ character} & 
\HeaderColor \RU{шестнадцатеричный код}\EN{hexadecimal code} & 
\HeaderColor x86\RU{-инструкция}\EN{ instruction} \\
\hline
f	 &66	 &\RU{(в 32-битном режиме) переключиться на}\EN{(in 32-bit mode) switch to}\\
   & & \RU{16-битный размер операнда}\EN{16-bit operand size} \\
g	 &67	 &\RU{(в 32-битном режиме) переключиться на}\EN{in 32-bit mode) switch to}\\
   & & \RU{16-битный размер адреса}\EN{16-bit address size} \\
\hline
\end{longtable}
\end{center}

\myindex{x86!\Instructions!AAA}
\myindex{x86!\Instructions!AAS}
\myindex{x86!\Instructions!CMP}
\myindex{x86!\Instructions!DEC}
\myindex{x86!\Instructions!IMUL}
\myindex{x86!\Instructions!INC}
\myindex{x86!\Instructions!JA}
\myindex{x86!\Instructions!JAE}
\myindex{x86!\Instructions!JB}
\myindex{x86!\Instructions!JBE}
\myindex{x86!\Instructions!JE}
\myindex{x86!\Instructions!JNE}
\myindex{x86!\Instructions!JNO}
\myindex{x86!\Instructions!JNS}
\myindex{x86!\Instructions!JO}
\myindex{x86!\Instructions!JP}
\myindex{x86!\Instructions!JS}
\myindex{x86!\Instructions!POP}
\myindex{x86!\Instructions!POPA}
\myindex{x86!\Instructions!PUSH}
\myindex{x86!\Instructions!PUSHA}
\myindex{x86!\Instructions!XOR}

\RU{В итоге}\EN{In summary}:
AAA, AAS, CMP, DEC, IMUL, INC, JA, JAE, JB, JBE, JE, JNE, JNO, JNS, JO, JP, JS, POP, POPA, PUSH, PUSHA, 
XOR.

