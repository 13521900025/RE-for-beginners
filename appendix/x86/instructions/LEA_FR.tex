\myindex{x86!\Instructions!LEA}
\item[LEA] (\emph{Load Effective Address}) forme une adresse

\label{sec:LEA}

\newcommand{\URLAM}{\href{http://go.yurichev.com/17109}{Wikipédia}}

Cette instruction n'a pas été conçue pour sommer des valeurs et/ou les multiplier,
mais pour former une adresse, e.g., pour calculer l'adresse d'un élément d'un tableau
en ajoutant l'adresse du tableau, l'index de l'élément multiplié par la taille de
l'élément\footnote{Voir aussi: \URLAM}.
\par
Donc, la différence entre \MOV et \LEA est que \MOV forme une adresse mémoire et
charge une valeur depuis la mémoire ou l'y stocke, alors que \LEA forme simplement une adresse.
\par
Mais néanmoins, elle peut être utilisée pour tout autre calcul.
\par
\LEA est pratique car le calcul qu'elle effectue n'altère pas les flags du \ac{CPU}.
Ceci peut être très important pour les processeurs \ac{OOE} (afin de créer moins de dépendances).
A part ça, au moins à partir du Pentium, l'instruction \LEA est exécutée en 1 cycle.

\begin{lstlisting}[style=customc]
int f(int a, int b)
{
	return a*8+b;
};
\end{lstlisting}

\begin{lstlisting}[caption=MSVC 2010 \Optimizing,style=customasmx86]
_a$ = 8		; size = 4
_b$ = 12	; size = 4
_f	PROC
	mov	eax, DWORD PTR _b$[esp-4]
	mov	ecx, DWORD PTR _a$[esp-4]
	lea	eax, DWORD PTR [eax+ecx*8]
	ret	0
_f	ENDP
\end{lstlisting}

\myindex{Intel C++}
Intel C++ utilise encore plus LEA:

\begin{lstlisting}[style=customc]
int f1(int a)
{
	return a*13;
};
\end{lstlisting}

\begin{lstlisting}[caption=Intel C++ 2011,style=customasmx86]
_f1	PROC NEAR
        mov       ecx, DWORD PTR [4+esp]      ; ecx = a
	lea       edx, DWORD PTR [ecx+ecx*8]  ; edx = a*9
	lea       eax, DWORD PTR [edx+ecx*4]  ; eax = a*9 + a*4 = a*13
        ret
\end{lstlisting}

Ces deux instructions sont plus rapide qu'un IMUL.

