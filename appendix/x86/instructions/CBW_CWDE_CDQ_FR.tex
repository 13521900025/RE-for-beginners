\myindex{x86!\Instructions!CBW}
\myindex{x86!\Instructions!CWD}
\myindex{x86!\Instructions!CDQ}
\myindex{x86!\Instructions!CWDE}
\myindex{x86!\Instructions!CDQE}
\label{ins:CBW_CWD_etc}
\item[CBW/CWD/CWDE/CDQ/CDQE]

Étendre le signe de la valeur:

\begin{description}
\item[CBW] Convertit l'octet dans AL en un mot dans AX
\item[CWD] Convertit le mot dans AX en double-mot dans DX:AX
\item[CWDE] Convertit le mot dans AX en double-mot dans EAX
\item[CDQ] Convertit le double-mot dans EAX en quadruple-mot dans EDX:EAX
\item[CDQE] (x64) Convertit le double-mot dans EAX en quadruple-mot dans RAX
\end{description}

Cette instruction examine le signe de la valeur, l'étend à la partie haute de la
valeur nouvellement construite. Voir aussi: \myref{subsec:sign_extending_32_to_64}.

\newcommand{\StephenMorse}{[Stephen P. Morse, \emph{The 8086 Primer}, (1980)]\footnote{\AlsoAvailableAs \url{https://archive.org/details/The8086Primer}}}

Il est intéressant de savoir que ces instructions furent initialement appelées \TT{SEX}
(\emph{Sign EXtend}), comme l'écrit Stephen P. Morse (un des concepteurs du CPU 8086)
dans \StephenMorse:

\begin{framed}
\begin{quotation}
The process of stretching numbers by extending the sign bit is called sign extension.
The 8086 provides instructions (Fig. 3.29) to facilitate the task of sign extension.
These instructions were initially named SEX (sign extend) but were later renamed to the more
conservative CBW (convert byte to word) and CWD (convert word to double word).
\end{quotation}
\end{framed}
