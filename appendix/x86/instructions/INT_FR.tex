\myindex{x86!\Instructions!INT}
\myindex{MS-DOS}

\item[INT] (M): \INS{INT x} est similaire à \INS{PUSHF; CALL dword ptr [x*4]}
en environnement 16-bit.
  Elle était énormément utilisée dans MS-DOS, fonctionnant comme un vecteur syscall.
  Les registres AX/BX/CX/DX/SI/DI étaient remplis avec les arguments et le flux sautait
  à l'adresse dans la table des vecteurs d'interruption (Interrupt Vector Table,
  située au début de l'espace d'adressage).
  Elle était répandue car INT a un opcode court (2 octets) et le programme qui a
  besoin d'un service MS-DOS ne doit pas déterminer l'adresse du point d'entrée de
  ce service.
\myindex{x86!\Instructions!IRET}
  Le gestionnaire d'interruption renvoie le contrôle du flux à l'appelant en utilisant
  l'instruction IRET.

  Le numéro d'interruption les plus utilisé était 0x21, servant une grande partie
  de on \ac{API}.
  Voir aussi: [Ralf Brown \emph{Ralf Brown's Interrupt List}],
  pour les listes d'interruption plus exhaustives et d'autres informations sur MS-DOS.

\myindex{x86!\Instructions!SYSENTER}
\myindex{x86!\Instructions!SYSCALL}
  Durant l'ère post-MS-DOS, cette instruction était toujours utilisée comme un syscall
  à la fois dans Linux et Windows (\myref{syscalls}), mais fût remplacée plus tard
  par les instructions SYSENTER ou SYSCALL.

\item[INT 3] (M): cette instruction est proche de
\INS{INT}, elle a son propre opcode d'1 octet (\GTT{0xCC}),
et est très utilisée pour le débogage.
Souvent, les débogueurs écrivent simplement l'octet \GTT{0xCC} à l'adresse du point
d'arrêt à mettre, et lorsqu'une exception est levée, l'octet original est restauré
et l'instruction originale à cette adresse est ré-exécutée. \\
Depuis \gls{Windows NT}, une exception \GTT{EXCEPTION\_BREAKPOINT} est déclenchée
lorsque le \ac{CPU} exécute cette instruction.
Cet évènement de débogage peut être intercepté et géré par un débogueur hôte, si
il y en a un de chargé.
S'il n'y en a pas de charger, Windows propose de lancer un des débogueurs enregistré
dans le système.
Si \ac{MSVS} est installé, son débogueur peut être chargé et connecté au processus.
Afin de protéger contre le \gls{reverse engineering}, de nombreuses méthodes anti-débogage
vérifient l'intégrité du code chargé.

\ac{MSVC} possède une \glslink{compiler intrinsic}{fonction intrinsèque} pour l'instruction:
\GTT{\_\_debugbreak()}\footnote{\href{http://go.yurichev.com/17226}{MSDN}}.


Il y a aussi une fonction win32 dans kernel32.dll appelée
\GTT{DebugBreak()}\footnote{\href{http://go.yurichev.com/17227}{MSDN}},
qui exécute aussi \GTT{INT 3}.

