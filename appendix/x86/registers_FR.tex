\subsection{Registres à usage général}

Il est possible d'accéder à de nombreux registres par octet ou par mot
de 16-bit.

Les vieux CPUs 8-bit (8080) avaient des registres de 16-bit divisés en deux.

Les programmes écrits pour 8080 pouvaient accéder à l'octet bas des registres de
16-bit, à l'octet haut ou au registre 16-bit en entier.

Peut-être que cette caractéristique a été conservée dans le 8086 pour faciliter le
portage.

Cette caractéristiques n'est en général pas présente sur les CPUs \ac{RISC}.

\myindex{x86-64}
Les registres préfixés par \TT{R-} sont apparus en x86-64, et ceux préfixés par \TT{E-}---dans le 80386.

Ainsi, les R-registres sont 64-bit, et les E-registres---32-bit.

8 \ac{GPR} ont été ajoutés en x86-64: R8-R15.

N.B.: 
Dans les manuels Intel, les parties octet de ces registres sont préfixées par \IT{L},
e.g.: \IT{R8L}, mais \ac{IDA} nomme ces registres en ajoutant le suffixe \IT{B},
e.g.: \IT{R8B}.

\subsubsection{RAX/EAX/AX/AL}
\RegTableOne{RAX}{EAX}{AX}{AH}{AL}

\ac{AKA} accumulateur
Le résultat d'une fonction est en général renvoyé via ce registre.

\subsubsection{RBX/EBX/BX/BL}
\RegTableOne{RBX}{EBX}{BX}{BH}{BL}

\subsubsection{RCX/ECX/CX/CL}
\RegTableOne{RCX}{ECX}{CX}{CH}{CL}

\ac{AKA} compteur:
il est utilisé dans ce rôle avec les instructions préfixées par REP et aussi dans
les instructions de décalage
(SHL/SHR/RxL/RxR).

\subsubsection{RDX/EDX/DX/DL}
\RegTableOne{RDX}{EDX}{DX}{DH}{DL}

\subsubsection{RSI/ESI/SI/SIL}
\RegTableTwo{RSI}{ESI}{SI}{SIL}

\ac{AKA} \q{source index}. Utilisé comme source dans les instructions
REP MOVSx, REP CMPSx.

\subsubsection{RDI/EDI/DI/DIL}
\RegTableTwo{RDI}{EDI}{DI}{DIL}

\ac{AKA} \q{destination index}. Utilisé comme un pointeur sur la destination dans
les instructions REP MOVSx, REP STOSx.

% TODO навести тут порядок
\subsubsection{R8/R8D/R8W/R8L}
\RegTableFour{R8}{R8D}{R8W}{R8L}

\subsubsection{R9/R9D/R9W/R9L}
\RegTableFour{R9}{R9D}{R9W}{R9L}

\subsubsection{R10/R10D/R10W/R10L}
\RegTableFour{R10}{R10D}{R10W}{R10L}

\subsubsection{R11/R11D/R11W/R11L}
\RegTableFour{R11}{R11D}{R11W}{R11L}

\subsubsection{R12/R12D/R12W/R12L}
\RegTableFour{R12}{R12D}{R12W}{R12L}

\subsubsection{R13/R13D/R13W/R13L}
\RegTableFour{R13}{R13D}{R13W}{R13L}

\subsubsection{R14/R14D/R14W/R14L}
\RegTableFour{R14}{R14D}{R14W}{R14L}

\subsubsection{R15/R15D/R15W/R15L}
\RegTableFour{R15}{R15D}{R15W}{R15L}

\subsubsection{RSP/ESP/SP/SPL}
\RegTableFour{RSP}{ESP}{SP}{SPL}

\ac{AKA} \glslink{stack pointer}{pointeur de pile}. Pointe en général sur la pile
courante excepté dans le cas où il n'est pas encore initialisé.

\subsubsection{RBP/EBP/BP/BPL}
\RegTableFour{RBP}{EBP}{BP}{BPL}

\ac{AKA} frame pointer. Utilisé d'habitude pour les variables locales et accéder
aux arguments de la fonction. En lire plus ici: (\myref{stack_frame}).

\subsubsection{RIP/EIP/IP}

\begin{center}
\begin{tabular}{ | l | l | l | l | l | l | l | l | l |}
\hline
\RegHeaderTop \\
\hline
\RegHeader \\
\hline
\multicolumn{8}{ | c | }{RIP\textsuperscript{x64}} \\
\hline
\multicolumn{4}{ | c | }{} & \multicolumn{4}{ c | }{EIP} \\
\hline
\multicolumn{6}{ | c | }{} & \multicolumn{2}{ c | }{IP} \\
\hline
\end{tabular}
\end{center}

\ac{AKA} \q{instruction pointer}
\footnote{Parfois appelé \q{program counter}}.
En général, il pointe toujours sur l'instruction en cours d'exécution.
Il ne peut pas être modifié, toutefois, il est possible de faire ceci (ce qui est
équivalent):

\begin{lstlisting}
MOV EAX, ...
JMP EAX
\end{lstlisting}

Ou:

\begin{lstlisting}
PUSH value
RET
\end{lstlisting}

\subsubsection{CS/DS/ES/SS/FS/GS}

Les registres 16-bit contiennent le sélecteur de code (CS), le sélecteur de données
(DS), le sélecteur de pile (SS).\\
\\
\myindex{TLS}
\myindex{Windows!TIB}
FS \InENRU win32 pointe sur \ac{TLS}, GS prend ce rôle dans Linux.
C'est fait pour accèder plus au \ac{TLS} et autres structures comme le \ac{TIB}.
\\
Dans le passé, ces registres étaient utilisés comme registres de segments (\myref{8086_memory_model}).

\subsubsection{Registre de flags}
\myindex{x86!\Registers!\Flags}
\label{EFLAGS}
\ac{AKA} EFLAGS.

\small
\begin{center}
\begin{tabular}{ | l | l | l | }
\hline
\headercolor{} Bit (masque) &
\headercolor{} Abréviation (signification) &
\headercolor{} Description \\
\hline
0 (1) & CF (Carry) &  \\
      &            & Les instructions CLC/STC/CMC sont utilisées \\
      &            & pour mettre/effacer/changer ce flag \\
\hline
2 (4) & PF (Parity) & (\myref{parity_flag}). \\
\hline
4 (0x10) & AF (Adjust) & Existe seulement pour travailler avec les nombres \ac{BCD} \\
\hline
6 (0x40) & ZF (Zero) & Mettre à 0 \\
         &           & si le résultat de la dernière opération est égal à 0. \\
\hline
7 (0x80) & SF (Sign) &  \\
\hline
8 (0x100) & TF (Trap) & Utilisé pour le débogage. \\
&         &             S'il est mis, une exception est générée \\
&         &             après l'exécution de chaque instruction. \\
\hline
9 (0x200) & IF (Interrupt enable) & Est-ce que les interruptions sont activées \\
          &                       & Les instructions CLI/STI sont utilisées \\
	  &                       & pour activer/désactiver le flag \\
\hline
10 (0x400) & DF (Direction) & Une direction est défini pour les \\
           &                & instructions REP MOVSx/CMPSx/LODSx/SCASx \\
           &                & Les instructions CLD/STD sont utilisées \\
	   &                & pour activer/désactiver le flag \\
	   &                & Voir aussi \myref{memmove_and_DF}. \\
\hline
11 (0x800) & OF (Overflow) &  \\
\hline
12, 13 (0x3000) & IOPL (I/O privilege level)\textsuperscript{i286} & \\
\hline
14 (0x4000) & NT (Nested task)\textsuperscript{i286} & \\
\hline
16 (0x10000) & RF (Resume)\textsuperscript{i386} & Utilisé pour le débogage. \\
             &                  & Le CPU ignore le point d'arrêt \\
	     &                  & matériel dans DRx si le flag est mis. \\
\hline
17 (0x20000) & VM (Virtual 8086 mode)\textsuperscript{i386} & \\
\hline
18 (0x40000) & AC (Alignment check)\textsuperscript{i486} & \\
\hline
19 (0x80000) & VIF (Virtual interrupt)\textsuperscript{i586} & \\
\hline
20 (0x100000) & VIP (Virtual interrupt pending)\textsuperscript{i586} & \\
\hline
21 (0x200000) & ID (Identification)\textsuperscript{i586} & \\
\hline
\end{tabular}
\end{center}
\normalsize

Tous les autres flags sont réservés.

\subsection{\registers{} FPU}

\myindex{x86!FPU}
8 registres de 80-bit fonctionnant comme une pile: ST(0)-ST(7).
N.B.: \ac{IDA} nomme ST(0) simplement en ST.
Les nombres sont stockés au format IEEE 754.

Format d'une valeur \IT{long double}:

\bigskip
% a hack used here! http://tex.stackexchange.com/questions/73524/bytefield-package
\begin{center}
\begingroup
\makeatletter
\let\saved@bf@bitformatting\bf@bitformatting
\renewcommand*{\bf@bitformatting}{%
	\ifnum\value{header@val}=21 %
	\value{header@val}=62 %
	\else\ifnum\value{header@val}=22 %
	\value{header@val}=63 %
	\else\ifnum\value{header@val}=23 %
	\value{header@val}=64 %
	\else\ifnum\value{header@val}=30 %
	\value{header@val}=78 %
	\else\ifnum\value{header@val}=31 %
	\value{header@val}=79 %
	\fi\fi\fi\fi\fi
	\saved@bf@bitformatting
}%
\begin{bytefield}[bitwidth=0.03\linewidth]{32}
	\bitheader[endianness=big]{0,21,22,23,30,31} \\
	\bitbox{1}{S} &
	\bitbox{8}{exposant} &
	\bitbox{1}{I} &
	\bitbox{22}{mantisse ou fraction}
\end{bytefield}
\endgroup
\end{center}

\begin{center}
( S --- signe, I --- partie entière )
\end{center}

\label{FPU_control_word}
\subsubsection{Mot de Contrôle}

Registre contrôlant le comportement du
\ac{FPU}.

\small
\begin{center}
\begin{tabular}{ | l | l | l | }
\hline
Bit &
Abréviation (signification) &
Description \\
\hline
0   & IM (Invalid operation Mask) & \\
\hline
1   & DM (Denormalized operand Mask) & \\
\hline
2   & ZM (Zero divide Mask) & \\
\hline
3   & OM (Overflow Mask) & \\
\hline
4   & UM (Underflow Mask) & \\
\hline
5   & PM (Precision Mask) & \\
\hline
7   & IEM (Interrupt Enable Mask) & Exceptions activées, 1 par défaut (désactivées) \\
\hline
8, 9 & PC (Precision Control) &  \\
     &                        & 00 ~--- 24 bits (REAL4) \\
     &                        & 10 ~--- 53 bits (REAL8) \\
     &                        & 11 ~--- 64 bits (REAL10) \\
\hline
10, 11 & RC (Rounding Control) &  \\
       &                       & 00 ~--- (par défaut) arrondir au plus proche \\
       &                       & 01 ~--- arrondir vers $-\infty$ \\
       &                       & 10 ~--- arrondir vers $+\infty$ \\
       &                       & 11 ~--- arrondir vers 0 \\
\hline
12 & IC (Infinity Control) & 0 ~--- (par défaut) traite $+\infty$ \AndENRU $-\infty$ comme non signé \\
   &                       & 1 ~--- respecte à la fois $+\infty$ \AndENRU $-\infty$ \\
\hline
\end{tabular}
\end{center}
\normalsize

Les flags PM, UM, OM, ZM, DM, IM 
définissent si une exception est générée en cas d'erreur correspondante.

\subsubsection{Mot d'état}

\label{FPU_status_word}
Registre en lecture seule.

\small
\begin{center}
\begin{tabular}{ | l | l | l | }
\hline
Bit &
Abréviation (signification) &
Description \\
\hline
15   & B (Busy) & Est-ce que le FPU fait quelque chose (1)
ou des résultats sont prêts (0) \\
\hline
14   & C3 & \\
\hline
13, 12, 11 & TOP & pointe sur le registre zéro actuel \\
\hline
10 & C2 & \\
\hline
9  & C1 & \\
\hline
8  & C0 & \\
\hline
7  & IR (Interrupt Request) & \\
\hline
6  & SF (Stack Fault) & \\
\hline
5  & P (Precision) & \\
\hline
4  & U (Underflow) & \\
\hline
3  & O (Overflow) & \\
\hline
2  & Z (Zero) & \\
\hline
1  & D (Denormalized) & \\
\hline
0  & I (Invalid operation) & \\
\hline
\end{tabular}
\end{center}
\normalsize

Les bits SF, P, U, O, Z, D, I indiquent les exceptions.

Vous trouverez des précisions à propos de C3, C2, C1, C0 ici: (\myref{Czero_etc}).

N.B.: Lorsque ST(x) est utilisé, le FPU ajoute $x$ à TOP (modulo 8) et c'est ainsi
qu'il obtient le numéro du registre interne.

\subsubsection{Mot Tag}

Le registre possède l'information actuelle à propos de l'utilisation des registres
de nombres.

\begin{center}
\begin{tabular}{ | l | l | l | }
\hline
Bit & Abréviation (signification) \\
\hline
15, 14 & Tag(7) \\
\hline
13, 12 & Tag(6) \\
\hline
11, 10 & Tag(5) \\
\hline
9, 8 & Tag(4) \\
\hline
7, 6 & Tag(3) \\
\hline
5, 4 & Tag(2) \\
\hline
3, 2 & Tag(1) \\
\hline
1, 0 & Tag(0) \\
\hline
\end{tabular}
\end{center}

Chaque tag contient l'information à propos d'un registre FPU physique, pas logique (ST(x)).

Pour chaque tag:

\begin{itemize}
\item 00 ~--- Le registre contient une valeur non-zéro
\item 01 ~--- Le registre contient 0
\item 10 ~--- Le registre contient une valeur particulière (\ac{NAN}, $\infty$, ou anormale)
\item 11 ~--- Le registre est vide
\end{itemize}

\subsection{\registers{} SIMD}

\subsubsection{\registers{} MMX}

8 registres 64-bit: MM0..MM7.

\subsubsection{\registers{} SSE \AndENRU AVX}

\myindex{x86-64}
SSE: 8 registres 128-bit: XMM0..XMM7.
En x86-64 8 autres registres ont été ajoutés: XMM8..XMM15.

AVX est l'extension de tous ces registres à 256 bits.

\subsection{Registres de débogage}

Ils sont utilisés pour le contrôle des points d'arrêt matériel (hardware breakpoints).

\begin{itemize}
	\item DR0 --- adresse du point d'arrêt \#1
	\item DR1 --- adresse du point d'arrêt \#2
	\item DR2 --- adresse du point d'arrêt \#3
	\item DR3 --- adresse du point d'arrêt \#4
	\item DR6 --- la cause de l'arrêt est indiquée ici
	\item DR7 --- les types de point d'arrêt sont mis ici
\end{itemize}

\subsubsection{DR6}
\myindex{x86!\Registers!DR6}

\begin{center}
\begin{tabular}{ | l | l | }
\hline
\headercolor\ Bit (masque) &
\headercolor\ Description \\
\hline
0 (1)       &  B0 --- le point d'arrêt \#1 a été déclenché \\
\hline
1 (2)       &  B1 --- le point d'arrêt \#2 a été déclenché \\
\hline
2 (4)       &  B2 --- le point d'arrêt \#3 a été déclenché \\
\hline
3 (8)       &  B3 --- le point d'arrêt \#4 a été déclenché \\
\hline
13 (0x2000) &  BD --- tentative de modification d'un des registres DRx.\\
            &  peut être déclenché si GD est activé \\
\hline
14 (0x4000) &  BS --- point d'arrêt simple (le flag TF a été mis dans EFLAGS). \\
	    &  La plus haute priorité. D'autres bits peuvent être mis aussi. \\
\hline
% TODO: describe BT
15 (0x8000) &  BT (task switch flag) \\
\hline
\end{tabular}
\end{center}

N.B.  Un point d'arrêt simple est un point d'arrêt qui se produit après chaque instruction.
Il peut être enclenché en mettant le flag TF dans EFLAGS (\myref{EFLAGS}).

\subsubsection{DR7}
\myindex{x86!\Registers!DR7}

Les types de point d'arrêt sont mis ici.

\small
\begin{center}
\begin{tabular}{ | l | l | }
\hline
\headercolor\ Bit (masque) &
\headercolor\ Description \\
\hline
0 (1)       &  L0 --- activer le point d'arrêt \#1 pour la tâche courante \\
\hline
1 (2)       &  G0 --- activer le point d'arrêt \#1 pour toutes les tâches \\
\hline
2 (4)       &  L1 --- activer le point d'arrêt \#2 pour la tâche courante \\
\hline
3 (8)       &  G1 --- activer le point d'arrêt \#2 pour toutes les tâches \\
\hline
4 (0x10)    &  L2 --- activer le point d'arrêt \#3 pour la tâche courante \\
\hline
5 (0x20)    &  G2 --- activer le point d'arrêt \#3 pour toutes les tâches \\
\hline
6 (0x40)    &  L3 --- activer le point d'arrêt \#4 pour la tâche courante \\
\hline
7 (0x80)    &  G3 --- activer le point d'arrêt \#4 pour toutes les tâches \\
\hline
8 (0x100)   &  LE --- non supporté depuis P6 \\
\hline
9 (0x200)   &  GE --- non supporté depuis P6 \\
\hline
13 (0x2000) &  GD --- exception déclenchée si une instruction MOV \\
            &  essaye de modifier un des registres DRx \\
\hline
16,17 (0x30000)    &  point d'arrêt \#1: R/W --- type \\
\hline
18,19 (0xC0000)    &  point d'arrêt \#1: LEN --- longueur \\
\hline
20,21 (0x300000)   &  point d'arrêt \#2: R/W --- type \\
\hline
22,23 (0xC00000)   &  point d'arrêt \#2: LEN --- longueur \\
\hline
24,25 (0x3000000)  &  point d'arrêt \#3: R/W --- type \\
\hline
26,27 (0xC000000)  &  point d'arrêt \#3: LEN --- longueur \\
\hline
28,29 (0x30000000) &  point d'arrêt \#4: R/W --- type \\
\hline
30,31 (0xC0000000) &  point d'arrêt \#4: LEN --- longueur \\
\hline
\end{tabular}
\end{center}
\normalsize

Le type de point d'arrêt doit être mis comme suit (R/W):

\begin{itemize}
\item 00 --- exécution de l'instruction
\item 01 --- écriture de données
\item 10 --- lecture ou écriture I/O (non disponible en mode user)
\item 11 --- à la lecture ou l'écriture de données
\end{itemize}

N.B.: le type de point d'arrêt est absent pour la lecture de données, en effet. \\
\\
La longueur du point d'arrêt est mise comme suit (LEN):

\begin{itemize}
\item 00 --- un octet
\item 01 --- deux octets
\item 10 --- non défini pour le mode 32-bit, huit octets en mode 64-bit
\item 11 --- quatre octets
\end{itemize}


% TODO: control registers
