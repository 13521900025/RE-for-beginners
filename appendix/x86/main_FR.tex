\mysection{x86}

\subsection{Terminologie}

Commun en 16-bit (8086/80286), 32-bit (80386, etc.), 64-bit.

\myindex{IEEE 754}
\myindex{MS-DOS}
\begin{description}
	\item[octet] 8-bit.
		La directive d'assembleur DB est utilisée pour définir les variables et les tableaux d'octets.
		Les octets sont passés dans les parties 8-bit des registres: \TT{AL/BL/CL/DL/AH/BH/CH/DH/SIL/DIL/R*L}.
	\item[mot] 16-bit. 
		directive assembleur DW \dittoclosing.
		Les mots sont passés dans la partie 16-bit des registres:\\
			\TT{AX/BX/CX/DX/SI/DI/R*W}.
	\item[double mot] (\q{dword}) 32-bit.
		directive assembleur DD \dittoclosing.
		Les double mots sont passés dans les registres (x86) ou dans la partie 32-bit de registres (x64).
		Dans du code 16-bit, les double mots sont passés dans une paire de registres 16-bit.
	\item[quadruple mot] (\q{qword}) 64-bit.
		directive assembleur DQ \dittoclosing.
		En environnement 32-bit, les quadruple mots sont passés dans une paire de registres 32-bit.
	\item[tbyte] (10 bytes) 80-bit ou 10 octets (utilisé pour les registres FPU IEEE 754).
	\item[paragraph] (16 bytes)---le terme était répandu dans l'environnement MS-DOS.
\end{description}

\myindex{Windows!API}

Des types de données de même taille (BYTE, WORD, DWORD) existent aussi dans l'\ac{API} Windows.

\subsection{Registres à usage général}

Il est possible d'accéder à de nombreux registres par octet ou par mot
de 16-bit.

Les vieux CPUs 8-bit (8080) avaient des registres de 16-bit divisés en deux.

Les programmes écrits pour 8080 pouvaient accéder à l'octet bas des registres de
16-bit, à l'octet haut ou au registre 16-bit en entier.

Peut-être que cette caractéristique a été conservée dans le 8086 pour faciliter le
portage.

Cette caractéristiques n'est en général pas présente sur les CPUs \ac{RISC}.

\myindex{x86-64}
Les registres préfixés par \TT{R-} sont apparus en x86-64, et ceux préfixés par \TT{E-}---dans le 80386.

Ainsi, les R-registres sont 64-bit, et les E-registres---32-bit.

8 \ac{GPR} ont été ajoutés en x86-64: R8-R15.

N.B.: 
Dans les manuels Intel, les parties octet de ces registres sont préfixées par \IT{L},
e.g.: \IT{R8L}, mais \ac{IDA} nomme ces registres en ajoutant le suffixe \IT{B},
e.g.: \IT{R8B}.

\subsubsection{RAX/EAX/AX/AL}
\RegTableOne{RAX}{EAX}{AX}{AH}{AL}

\ac{AKA} accumulateur
Le résultat d'une fonction est en général renvoyé via ce registre.

\subsubsection{RBX/EBX/BX/BL}
\RegTableOne{RBX}{EBX}{BX}{BH}{BL}

\subsubsection{RCX/ECX/CX/CL}
\RegTableOne{RCX}{ECX}{CX}{CH}{CL}

\ac{AKA} compteur:
il est utilisé dans ce rôle avec les instructions préfixées par REP et aussi dans
les instructions de décalage
(SHL/SHR/RxL/RxR).

\subsubsection{RDX/EDX/DX/DL}
\RegTableOne{RDX}{EDX}{DX}{DH}{DL}

\subsubsection{RSI/ESI/SI/SIL}
\RegTableTwo{RSI}{ESI}{SI}{SIL}

\ac{AKA} \q{source index}. Utilisé comme source dans les instructions
REP MOVSx, REP CMPSx.

\subsubsection{RDI/EDI/DI/DIL}
\RegTableTwo{RDI}{EDI}{DI}{DIL}

\ac{AKA} \q{destination index}. Utilisé comme un pointeur sur la destination dans
les instructions REP MOVSx, REP STOSx.

% TODO навести тут порядок
\subsubsection{R8/R8D/R8W/R8L}
\RegTableFour{R8}{R8D}{R8W}{R8L}

\subsubsection{R9/R9D/R9W/R9L}
\RegTableFour{R9}{R9D}{R9W}{R9L}

\subsubsection{R10/R10D/R10W/R10L}
\RegTableFour{R10}{R10D}{R10W}{R10L}

\subsubsection{R11/R11D/R11W/R11L}
\RegTableFour{R11}{R11D}{R11W}{R11L}

\subsubsection{R12/R12D/R12W/R12L}
\RegTableFour{R12}{R12D}{R12W}{R12L}

\subsubsection{R13/R13D/R13W/R13L}
\RegTableFour{R13}{R13D}{R13W}{R13L}

\subsubsection{R14/R14D/R14W/R14L}
\RegTableFour{R14}{R14D}{R14W}{R14L}

\subsubsection{R15/R15D/R15W/R15L}
\RegTableFour{R15}{R15D}{R15W}{R15L}

\subsubsection{RSP/ESP/SP/SPL}
\RegTableFour{RSP}{ESP}{SP}{SPL}

\ac{AKA} \glslink{stack pointer}{pointeur de pile}. Pointe en général sur la pile
courante excepté dans le cas où il n'est pas encore initialisé.

\subsubsection{RBP/EBP/BP/BPL}
\RegTableFour{RBP}{EBP}{BP}{BPL}

\ac{AKA} frame pointer. Utilisé d'habitude pour les variables locales et accéder
aux arguments de la fonction. En lire plus ici: (\myref{stack_frame}).

\subsubsection{RIP/EIP/IP}

\begin{center}
\begin{tabular}{ | l | l | l | l | l | l | l | l | l |}
\hline
\RegHeaderTop \\
\hline
\RegHeader \\
\hline
\multicolumn{8}{ | c | }{RIP\textsuperscript{x64}} \\
\hline
\multicolumn{4}{ | c | }{} & \multicolumn{4}{ c | }{EIP} \\
\hline
\multicolumn{6}{ | c | }{} & \multicolumn{2}{ c | }{IP} \\
\hline
\end{tabular}
\end{center}

\ac{AKA} \q{instruction pointer}
\footnote{Parfois appelé \q{program counter}}.
En général, il pointe toujours sur l'instruction en cours d'exécution.
Il ne peut pas être modifié, toutefois, il est possible de faire ceci (ce qui est
équivalent):

\begin{lstlisting}
MOV EAX, ...
JMP EAX
\end{lstlisting}

Ou:

\begin{lstlisting}
PUSH value
RET
\end{lstlisting}

\subsubsection{CS/DS/ES/SS/FS/GS}

Les registres 16-bit contiennent le sélecteur de code (CS), le sélecteur de données
(DS), le sélecteur de pile (SS).\\
\\
\myindex{TLS}
\myindex{Windows!TIB}
FS \InENRU win32 pointe sur \ac{TLS}, GS prend ce rôle dans Linux.
C'est fait pour accèder plus au \ac{TLS} et autres structures comme le \ac{TIB}.
\\
Dans le passé, ces registres étaient utilisés comme registres de segments (\myref{8086_memory_model}).

\subsubsection{Registre de flags}
\myindex{x86!\Registers!\Flags}
\label{EFLAGS}
\ac{AKA} EFLAGS.

\small
\begin{center}
\begin{tabular}{ | l | l | l | }
\hline
\headercolor{} Bit (masque) &
\headercolor{} Abréviation (signification) &
\headercolor{} Description \\
\hline
0 (1) & CF (Carry) &  \\
      &            & Les instructions CLC/STC/CMC sont utilisées \\
      &            & pour mettre/effacer/changer ce flag \\
\hline
2 (4) & PF (Parity) & (\myref{parity_flag}). \\
\hline
4 (0x10) & AF (Adjust) & Existe seulement pour travailler avec les nombres \ac{BCD} \\
\hline
6 (0x40) & ZF (Zero) & Mettre à 0 \\
         &           & si le résultat de la dernière opération est égal à 0. \\
\hline
7 (0x80) & SF (Sign) &  \\
\hline
8 (0x100) & TF (Trap) & Utilisé pour le débogage. \\
&         &             S'il est mis, une exception est générée \\
&         &             après l'exécution de chaque instruction. \\
\hline
9 (0x200) & IF (Interrupt enable) & Est-ce que les interruptions sont activées \\
          &                       & Les instructions CLI/STI sont utilisées \\
	  &                       & pour activer/désactiver le flag \\
\hline
10 (0x400) & DF (Direction) & Une direction est défini pour les \\
           &                & instructions REP MOVSx/CMPSx/LODSx/SCASx \\
           &                & Les instructions CLD/STD sont utilisées \\
	   &                & pour activer/désactiver le flag \\
	   &                & Voir aussi \myref{memmove_and_DF}. \\
\hline
11 (0x800) & OF (Overflow) &  \\
\hline
12, 13 (0x3000) & IOPL (I/O privilege level)\textsuperscript{i286} & \\
\hline
14 (0x4000) & NT (Nested task)\textsuperscript{i286} & \\
\hline
16 (0x10000) & RF (Resume)\textsuperscript{i386} & Utilisé pour le débogage. \\
             &                  & Le CPU ignore le point d'arrêt \\
	     &                  & matériel dans DRx si le flag est mis. \\
\hline
17 (0x20000) & VM (Virtual 8086 mode)\textsuperscript{i386} & \\
\hline
18 (0x40000) & AC (Alignment check)\textsuperscript{i486} & \\
\hline
19 (0x80000) & VIF (Virtual interrupt)\textsuperscript{i586} & \\
\hline
20 (0x100000) & VIP (Virtual interrupt pending)\textsuperscript{i586} & \\
\hline
21 (0x200000) & ID (Identification)\textsuperscript{i586} & \\
\hline
\end{tabular}
\end{center}
\normalsize

Tous les autres flags sont réservés.

\subsection{\registers{} FPU}

\myindex{x86!FPU}
8 registres de 80-bit fonctionnant comme une pile: ST(0)-ST(7).
N.B.: \ac{IDA} nomme ST(0) simplement en ST.
Les nombres sont stockés au format IEEE 754.

Format d'une valeur \IT{long double}:

\bigskip
% a hack used here! http://tex.stackexchange.com/questions/73524/bytefield-package
\begin{center}
\begingroup
\makeatletter
\let\saved@bf@bitformatting\bf@bitformatting
\renewcommand*{\bf@bitformatting}{%
	\ifnum\value{header@val}=21 %
	\value{header@val}=62 %
	\else\ifnum\value{header@val}=22 %
	\value{header@val}=63 %
	\else\ifnum\value{header@val}=23 %
	\value{header@val}=64 %
	\else\ifnum\value{header@val}=30 %
	\value{header@val}=78 %
	\else\ifnum\value{header@val}=31 %
	\value{header@val}=79 %
	\fi\fi\fi\fi\fi
	\saved@bf@bitformatting
}%
\begin{bytefield}[bitwidth=0.03\linewidth]{32}
	\bitheader[endianness=big]{0,21,22,23,30,31} \\
	\bitbox{1}{S} &
	\bitbox{8}{exposant} &
	\bitbox{1}{I} &
	\bitbox{22}{mantisse ou fraction}
\end{bytefield}
\endgroup
\end{center}

\begin{center}
( S --- signe, I --- partie entière )
\end{center}

\label{FPU_control_word}
\subsubsection{Mot de Contrôle}

Registre contrôlant le comportement du
\ac{FPU}.

\small
\begin{center}
\begin{tabular}{ | l | l | l | }
\hline
Bit &
Abréviation (signification) &
Description \\
\hline
0   & IM (Invalid operation Mask) & \\
\hline
1   & DM (Denormalized operand Mask) & \\
\hline
2   & ZM (Zero divide Mask) & \\
\hline
3   & OM (Overflow Mask) & \\
\hline
4   & UM (Underflow Mask) & \\
\hline
5   & PM (Precision Mask) & \\
\hline
7   & IEM (Interrupt Enable Mask) & Exceptions activées, 1 par défaut (désactivées) \\
\hline
8, 9 & PC (Precision Control) &  \\
     &                        & 00 ~--- 24 bits (REAL4) \\
     &                        & 10 ~--- 53 bits (REAL8) \\
     &                        & 11 ~--- 64 bits (REAL10) \\
\hline
10, 11 & RC (Rounding Control) &  \\
       &                       & 00 ~--- (par défaut) arrondir au plus proche \\
       &                       & 01 ~--- arrondir vers $-\infty$ \\
       &                       & 10 ~--- arrondir vers $+\infty$ \\
       &                       & 11 ~--- arrondir vers 0 \\
\hline
12 & IC (Infinity Control) & 0 ~--- (par défaut) traite $+\infty$ \AndENRU $-\infty$ comme non signé \\
   &                       & 1 ~--- respecte à la fois $+\infty$ \AndENRU $-\infty$ \\
\hline
\end{tabular}
\end{center}
\normalsize

Les flags PM, UM, OM, ZM, DM, IM 
définissent si une exception est générée en cas d'erreur correspondante.

\subsubsection{Mot d'état}

\label{FPU_status_word}
Registre en lecture seule.

\small
\begin{center}
\begin{tabular}{ | l | l | l | }
\hline
Bit &
Abréviation (signification) &
Description \\
\hline
15   & B (Busy) & Est-ce que le FPU fait quelque chose (1)
ou des résultats sont prêts (0) \\
\hline
14   & C3 & \\
\hline
13, 12, 11 & TOP & pointe sur le registre zéro actuel \\
\hline
10 & C2 & \\
\hline
9  & C1 & \\
\hline
8  & C0 & \\
\hline
7  & IR (Interrupt Request) & \\
\hline
6  & SF (Stack Fault) & \\
\hline
5  & P (Precision) & \\
\hline
4  & U (Underflow) & \\
\hline
3  & O (Overflow) & \\
\hline
2  & Z (Zero) & \\
\hline
1  & D (Denormalized) & \\
\hline
0  & I (Invalid operation) & \\
\hline
\end{tabular}
\end{center}
\normalsize

Les bits SF, P, U, O, Z, D, I indiquent les exceptions.

Vous trouverez des précisions à propos de C3, C2, C1, C0 ici: (\myref{Czero_etc}).

N.B.: Lorsque ST(x) est utilisé, le FPU ajoute $x$ à TOP (modulo 8) et c'est ainsi
qu'il obtient le numéro du registre interne.

\subsubsection{Mot Tag}

Le registre possède l'information actuelle à propos de l'utilisation des registres
de nombres.

\begin{center}
\begin{tabular}{ | l | l | l | }
\hline
Bit & Abréviation (signification) \\
\hline
15, 14 & Tag(7) \\
\hline
13, 12 & Tag(6) \\
\hline
11, 10 & Tag(5) \\
\hline
9, 8 & Tag(4) \\
\hline
7, 6 & Tag(3) \\
\hline
5, 4 & Tag(2) \\
\hline
3, 2 & Tag(1) \\
\hline
1, 0 & Tag(0) \\
\hline
\end{tabular}
\end{center}

Chaque tag contient l'information à propos d'un registre FPU physique, pas logique (ST(x)).

Pour chaque tag:

\begin{itemize}
\item 00 ~--- Le registre contient une valeur non-zéro
\item 01 ~--- Le registre contient 0
\item 10 ~--- Le registre contient une valeur particulière (\ac{NAN}, $\infty$, ou anormale)
\item 11 ~--- Le registre est vide
\end{itemize}

\subsection{\registers{} SIMD}

\subsubsection{\registers{} MMX}

8 registres 64-bit: MM0..MM7.

\subsubsection{\registers{} SSE \AndENRU AVX}

\myindex{x86-64}
SSE: 8 registres 128-bit: XMM0..XMM7.
En x86-64 8 autres registres ont été ajoutés: XMM8..XMM15.

AVX est l'extension de tous ces registres à 256 bits.

\subsection{Registres de débogage}

Ils sont utilisés pour le contrôle des points d'arrêt matériel (hardware breakpoints).

\begin{itemize}
	\item DR0 --- adresse du point d'arrêt \#1
	\item DR1 --- adresse du point d'arrêt \#2
	\item DR2 --- adresse du point d'arrêt \#3
	\item DR3 --- adresse du point d'arrêt \#4
	\item DR6 --- la cause de l'arrêt est indiquée ici
	\item DR7 --- les types de point d'arrêt sont mis ici
\end{itemize}

\subsubsection{DR6}
\myindex{x86!\Registers!DR6}

\begin{center}
\begin{tabular}{ | l | l | }
\hline
\headercolor\ Bit (masque) &
\headercolor\ Description \\
\hline
0 (1)       &  B0 --- le point d'arrêt \#1 a été déclenché \\
\hline
1 (2)       &  B1 --- le point d'arrêt \#2 a été déclenché \\
\hline
2 (4)       &  B2 --- le point d'arrêt \#3 a été déclenché \\
\hline
3 (8)       &  B3 --- le point d'arrêt \#4 a été déclenché \\
\hline
13 (0x2000) &  BD --- tentative de modification d'un des registres DRx.\\
            &  peut être déclenché si GD est activé \\
\hline
14 (0x4000) &  BS --- point d'arrêt simple (le flag TF a été mis dans EFLAGS). \\
	    &  La plus haute priorité. D'autres bits peuvent être mis aussi. \\
\hline
% TODO: describe BT
15 (0x8000) &  BT (task switch flag) \\
\hline
\end{tabular}
\end{center}

N.B.  Un point d'arrêt simple est un point d'arrêt qui se produit après chaque instruction.
Il peut être enclenché en mettant le flag TF dans EFLAGS (\myref{EFLAGS}).

\subsubsection{DR7}
\myindex{x86!\Registers!DR7}

Les types de point d'arrêt sont mis ici.

\small
\begin{center}
\begin{tabular}{ | l | l | }
\hline
\headercolor\ Bit (masque) &
\headercolor\ Description \\
\hline
0 (1)       &  L0 --- activer le point d'arrêt \#1 pour la tâche courante \\
\hline
1 (2)       &  G0 --- activer le point d'arrêt \#1 pour toutes les tâches \\
\hline
2 (4)       &  L1 --- activer le point d'arrêt \#2 pour la tâche courante \\
\hline
3 (8)       &  G1 --- activer le point d'arrêt \#2 pour toutes les tâches \\
\hline
4 (0x10)    &  L2 --- activer le point d'arrêt \#3 pour la tâche courante \\
\hline
5 (0x20)    &  G2 --- activer le point d'arrêt \#3 pour toutes les tâches \\
\hline
6 (0x40)    &  L3 --- activer le point d'arrêt \#4 pour la tâche courante \\
\hline
7 (0x80)    &  G3 --- activer le point d'arrêt \#4 pour toutes les tâches \\
\hline
8 (0x100)   &  LE --- non supporté depuis P6 \\
\hline
9 (0x200)   &  GE --- non supporté depuis P6 \\
\hline
13 (0x2000) &  GD --- exception déclenchée si une instruction MOV \\
            &  essaye de modifier un des registres DRx \\
\hline
16,17 (0x30000)    &  point d'arrêt \#1: R/W --- type \\
\hline
18,19 (0xC0000)    &  point d'arrêt \#1: LEN --- longueur \\
\hline
20,21 (0x300000)   &  point d'arrêt \#2: R/W --- type \\
\hline
22,23 (0xC00000)   &  point d'arrêt \#2: LEN --- longueur \\
\hline
24,25 (0x3000000)  &  point d'arrêt \#3: R/W --- type \\
\hline
26,27 (0xC000000)  &  point d'arrêt \#3: LEN --- longueur \\
\hline
28,29 (0x30000000) &  point d'arrêt \#4: R/W --- type \\
\hline
30,31 (0xC0000000) &  point d'arrêt \#4: LEN --- longueur \\
\hline
\end{tabular}
\end{center}
\normalsize

Le type de point d'arrêt doit être mis comme suit (R/W):

\begin{itemize}
\item 00 --- exécution de l'instruction
\item 01 --- écriture de données
\item 10 --- lecture ou écriture I/O (non disponible en mode user)
\item 11 --- à la lecture ou l'écriture de données
\end{itemize}

N.B.: le type de point d'arrêt est absent pour la lecture de données, en effet. \\
\\
La longueur du point d'arrêt est mise comme suit (LEN):

\begin{itemize}
\item 00 --- un octet
\item 01 --- deux octets
\item 10 --- non défini pour le mode 32-bit, huit octets en mode 64-bit
\item 11 --- quatre octets
\end{itemize}


% TODO: control registers
 % subsection
\subsection{Instructions}
\label{sec:x86_instructions}

Les instructions marquées avec un (M) ne sont généralement pas générées par le compilateur:
si vous rencontrez l'une d'entre elles, il s'agit probablement de code assembleur
écrit à la main, ou de fonctions intrinsèques (\myref{sec:compiler_intrinsic}).

% TODO ? обратные инструкции

Seules les instructions les plus fréquemment utilisées sont listées ici.
Vous pouvez lire \myref{x86_manuals} pour une documentation complète.

Devez-vous connaître tous les opcodes des instructions par c\oe{}ur?
Non, seulement ceux qui sont utilisés pour patcher du code
(\myref{x86_patching}).
Tout le reste des opcodes n'a pas besoin d'être mémorisé.

\subsubsection{Préfixes}

\myindex{x86!\Prefixes!LOCK}
\myindex{x86!\Prefixes!REP}
\myindex{x86!\Prefixes!REPE/REPNE}
\begin{description}
\label{x86_lock}
\item[LOCK] force le CPU à faire un accès exclusif à la RAM dans un environnement multi-processeurs.
Par simplification, on peut dire que lorsqu'une instruction avec ce préfixe est exécutée,
tous les autres CPU dans un système multi-processeur sont stoppés.
Le plus souvent, c'est utilisé pour les sections critiques, les sémaphores et les mutex.
Couramment utilisé avec ADD, AND, BTR, BTS, CMPXCHG, OR, XADD, XOR.
Vous pouvez en lire plus sur les sections critiques ici (\myref{critical_sections}).

\item[REP] est utilisé avec les instructions MOVSx \AndENRU STOSx:
exécute l'instruction dans une boucle, le compteur est situé dans le registre CX/ECX/RCX.
Pour une description plus détaillée de ces instructions, voir MOVSx (\myref{REP_MOVSx})
\AndENRU STOSx (\myref{REP_STOSx}).

Les instructions préfixées par REP sont sensibles au flag DF, qui est utilisé pour définir la direction.

\item[REPE/REPNE] (\ac{AKA} REPZ/REPNZ) utilisé avec les instructions CMPSx \AndENRU SCASx:
exécute la dernière instruction dans une boucle, le compteur est mis dans le registre \TT{CX}/\TT{ECX}/\TT{RCX}.
Elle s'arrête prématurément si ZF vaut 0 (REPE) ou si ZF vaut 1 (REPNE).

Pour une description plus détaillée de ces instructions, voir CMPSx (\myref{REPE_CMPSx})
\AndENRU SCASx (\myref{REPNE_SCASx}).

Les instructions préfixées par REPE/REPNE sont sensibles au flag DF, qui est utilisé pour définir la direction.

\end{description}

\subsubsection{Instructions les plus fréquemment utilisées}

Celles-ci peuvent être mémorisées en premier.

\begin{description}
% in order to keep them easily sorted...
\myindex{x86!\Instructions!ADC}
\myindex{x86!\Flags!CF}
  \item[ADC] (\emph{add with carry})
  ajoute des valeurs, \glslink{increment}{incrémente} le résultat si le flag CF est
  mis. ADC est souvent utilisé pour ajouter des grandes valeurs, par exemple, pour
  ajouter deux valeurs 64-bit dans un environnement 32-bit en utilisant deux
  instructions, ADD et ADC. Par exemple:

\lstinputlisting[style=customasmx86]{appendix/x86/instructions/ADC_example_FR.lst}

Un autre exemple: \myref{sec:64bit_in_32_env}.

\input{appendix/x86/instructions/ADD}
\input{appendix/x86/instructions/AND}
\input{appendix/x86/instructions/CALL}
\input{appendix/x86/instructions/CMP}
\input{appendix/x86/instructions/DEC_FR}
\input{appendix/x86/instructions/IMUL}
\input{appendix/x86/instructions/INC_FR}
\input{appendix/x86/instructions/JCXZ}
\input{appendix/x86/instructions/JMP}
\item[Jcc] (où cc ---  condition code)

\myindex{x86!\Instructions!JAE}
\myindex{x86!\Instructions!JA}
\myindex{x86!\Instructions!JBE}
\myindex{x86!\Instructions!JB}
\myindex{x86!\Instructions!JC}
\myindex{x86!\Instructions!JE}
\myindex{x86!\Instructions!JGE}
\myindex{x86!\Instructions!JG}
\myindex{x86!\Instructions!JLE}
\myindex{x86!\Instructions!JL}
\myindex{x86!\Instructions!JNAE}
\myindex{x86!\Instructions!JNA}
\myindex{x86!\Instructions!JNBE}
\myindex{x86!\Instructions!JNB}
\myindex{x86!\Instructions!JNC}
\myindex{x86!\Instructions!JNE}
\myindex{x86!\Instructions!JNGE}
\myindex{x86!\Instructions!JNG}
\myindex{x86!\Instructions!JNLE}
\myindex{x86!\Instructions!JNL}
\myindex{x86!\Instructions!JNO}
\myindex{x86!\Instructions!JNS}
\myindex{x86!\Instructions!JNZ}
\myindex{x86!\Instructions!JO}
\myindex{x86!\Instructions!JPO}
\myindex{x86!\Instructions!JP}
\myindex{x86!\Instructions!JS}
\myindex{x86!\Instructions!JZ}

Beaucoup de ces instructions ont des synonymes (notés avec AKA), qui ont été ajoutés
par commodité. Ils sont codés avec le même opcode.
L'opcode a un \glslink{jump offset}{offset de saut}.

\label{Jcc}
\begin{description}
\item[JAE] \ac{AKA} JNC: saut si supérieur ou égal (non signé): C=0
\item[JA] \ac{AKA} JNBE: saut si supérieur (non signé): CF=0 et ZF=0
\item[JBE] saut si inférieur ou égal (non signé): CF=1 ou ZF=1
\item[JB] \ac{AKA} JC: saut si inférieur(non signé): CF=1
\item[JC] \ac{AKA} JB: saut si CF=1
\item[JE] \ac{AKA} JZ: saut si égal ou zéro: ZF=1
\item[JGE] saut si supérieur ou égal (signé): SF=OF
\item[JG] saut si supérieur (signé): ZF=0 et SF=OF
\item[JLE] saut si inférieur ou égal (signé): ZF=1 ou SF$\neq$OF
\item[JL] saut si inférieur (signé): SF$\neq$OF
\item[JNAE] \ac{AKA} JC: saut si non supérieur ou égal (non signé): CF=1
\item[JNA] saut si non supérieur (non signé): CF=1 \AndENRU ZF=1
\item[JNBE] saut si non inférieur ou égal (non signé): CF=0 \AndENRU ZF=0
\item[JNB] \ac{AKA} JNC: saut si non inférieur (non signé): CF=0
\item[JNC] \ac{AKA} JAE: saut si CF=0, synonyme de JNB.
\item[JNE] \ac{AKA} JNZ: saut si non égal ou non zéro: ZF=0
\item[JNGE] saut si non supérieur ou égal (signé): SF$\neq$OF
\item[JNG] saut si non supérieur (signé): ZF=1 ou SF$\neq$OF
\item[JNLE] saut si non inférieur ou égal (signé): ZF=0 \AndENRU SF=OF
\item[JNL] saut si non inférieur (signé): SF=OF
\item[JNO] saut si non débordement: OF=0
\item[JNS] saut si le flag SF vaut zéro
\item[JNZ] \ac{AKA} JNE: saut si non égal ou non zéro: ZF=0
\item[JO] saut si débordement: OF=1
\item[JPO] saut si le flag PF vaut 0 (Jump Parity Odd)
\item[JP] \ac{AKA} \ac{JPE}: saut si le flag PF est mis
\item[JS] saut si le flag SF est mis
\item[JZ] \ac{AKA} JE: saut si égal ou zéro: ZF=1
\end{description}


\input{appendix/x86/instructions/LAHF}
\myindex{x86!\Instructions!LEAVE}
\label{x86_ins:LEAVE}
\item[LEAVE] \RU{аналог команд \TT{MOV ESP, EBP} и \TT{POP EBP} --- 
то есть возврат \glslink{stack pointer}{указателя стека} и регистра \EBP в первоначальное состояние.}%
\EN{equivalent of the \TT{MOV ESP, EBP} and \TT{POP EBP} instruction
pair --- in other words, this instruction sets the \gls{stack pointer} (\ESP) back and restores
the \EBP register to its initial state.}%
\FR{équivalente à la paire d'instructions  \TT{MOV ESP, EBP} et \TT{POP EBP}  --- autrement dit,
cette instruction remet le \glslink{stack pointer}{pointeur de pile} et restaure le registre
\EBP à l'état initial.}


\myindex{x86!\Instructions!LEA}
\item[LEA] (\emph{Load Effective Address}) forme une adresse

\label{sec:LEA}

\newcommand{\URLAM}{\href{http://go.yurichev.com/17109}{Wikipédia}}

Cette instruction n'a pas été conçue pour sommer des valeurs et/ou les multiplier,
mais pour former une adresse, e.g., pour calculer l'adresse d'un élément d'un tableau
en ajoutant l'adresse du tableau, l'index de l'élément multiplié par la taille de
l'élément\footnote{Voir aussi: \URLAM}.
\par
Donc, la différence entre \MOV et \LEA est que \MOV forme une adresse mémoire et
charge une valeur depuis la mémoire ou l'y stocke, alors que \LEA forme simplement une adresse.
\par
Mais néanmoins, elle peut être utilisée pour tout autre calcul.
\par
\LEA est pratique car le calcul qu'elle effectue n'altère pas les flags du \ac{CPU}.
Ceci peut être très important pour les processeurs \ac{OOE} (afin de créer moins de dépendances).
À part ça, au moins à partir du Pentium, l'instruction \LEA est exécutée en 1 cycle.

\begin{lstlisting}[style=customc]
int f(int a, int b)
{
	return a*8+b;
};
\end{lstlisting}

\begin{lstlisting}[caption=MSVC 2010 \Optimizing,style=customasmx86]
_a$ = 8		; size = 4
_b$ = 12	; size = 4
_f	PROC
	mov	eax, DWORD PTR _b$[esp-4]
	mov	ecx, DWORD PTR _a$[esp-4]
	lea	eax, DWORD PTR [eax+ecx*8]
	ret	0
_f	ENDP
\end{lstlisting}

\myindex{Intel C++}
Intel C++ utilise encore plus LEA:

\begin{lstlisting}[style=customc]
int f1(int a)
{
	return a*13;
};
\end{lstlisting}

\begin{lstlisting}[caption=Intel C++ 2011,style=customasmx86]
_f1	PROC NEAR
        mov       ecx, DWORD PTR [4+esp]      ; ecx = a
	lea       edx, DWORD PTR [ecx+ecx*8]  ; edx = a*9
	lea       eax, DWORD PTR [edx+ecx*4]  ; eax = a*9 + a*4 = a*13
        ret
\end{lstlisting}

Ces deux instructions sont plus rapide qu'un IMUL.


\input{appendix/x86/instructions/MOVSB_W_D_Q_FR}
\myindex{x86!\Instructions!MOVSX}
  \item[MOVSX] \RU{загрузить с расширением знака}\EN{load with sign extension}\FR{charger avec extension du signe} %
  \RU{см. также}\EN{see also}\FR{voir aussi}: (\myref{MOVSX})

\input{appendix/x86/instructions/MOVZX}
\input{appendix/x86/instructions/MOV_FR}
\input{appendix/x86/instructions/MUL}
\input{appendix/x86/instructions/NEG}
\input{appendix/x86/instructions/NOP_FR}
\input{appendix/x86/instructions/NOT}
\input{appendix/x86/instructions/OR}
\input{appendix/x86/instructions/POP}
\input{appendix/x86/instructions/PUSH}
\myindex{x86!\Instructions!RET}
\myindex{MS-DOS}
\item[RET] Revient d'une sous-routine: \TT{POP tmp; JMP tmp}.

En fait, RET
est une macro du langage d'assemblage, sous les environnements Windows et *NIX, elle
est traduite en
RETN (\q{return near})
ou, du temps de MS-DOS, où la mémoire était adressée différemment
(\myref{8086_memory_model}), en RETF (\q{return far}).

\TT{RET} peut avoir un opérande.
Alors il fonctionne comme ceci: \\
\TT{POP tmp; ADD ESP op1; JMP tmp}.
\TT{RET} avec un opérande termine en général les fonctions avec la convention d'appel
\emph{stdcall}, voir aussi: \myref{sec:stdcall}.


\input{appendix/x86/instructions/SAHF}
\myindex{x86!\Instructions!SBB}
\myindex{x86!\Flags!CF}
  \item[SBB] (\emph{subtraction with borrow}) 
  \RU{вычесть одно значение из другого, \glslink{decrement}{декремент} результата если флаг CF выставлен.
  SBB часто используется для вычитания больших значений, например, для вычитания двух 64-битных
  значений в 32-битной среде используя инструкции SUB и SBB, например:}
  \EN{subtract values, \gls{decrement} the result if the CF flag is set.
  SBB is often used for subtraction of large values, for example,
  to subtract two 64-bit values in 32-bit environment using two SUB and SBB instructions. For example:}
  \FR{soustrait les valeurs, \glslink{decrement}{décrémente} le résultat si le flag
  CF est mis. SBB est souvent utilisé pour la soustraction de grandes valeurs, par
  exemple:}

\EN{\lstinputlisting[style=customasmx86]{appendix/x86/instructions/SBB_example_EN.lst}}
\RU{\lstinputlisting[style=customasmx86]{appendix/x86/instructions/SBB_example_RU.lst}}
\FR{\lstinputlisting[style=customasmx86]{appendix/x86/instructions/SBB_example_FR.lst}}

\RU{Еще один пример}\EN{One more example}\FR{Un autre exemple}: \myref{sec:64bit_in_32_env}.

\input{appendix/x86/instructions/SCASB_W_D_Q_FR}
\input{appendix/x86/instructions/SHx}
\input{appendix/x86/instructions/SHRD}
\input{appendix/x86/instructions/STOSB_W_D_Q_FR}
\myindex{x86!\Instructions!SUB}
  \item[SUB] \RU{вычесть одно значение из другого. 
  часто встречающийся вариант \TT{SUB reg,reg} означает обнуление \emph{reg}.}
  \EN{subtract values. 
  A frequently occurring pattern is \TT{SUB reg,reg}, which implies zeroing of \emph{reg}.}
  \FR{soustrait des valeurs.
  Une utilisation fréquente est \TT{SUB reg,reg}, qui met \emph{reg} à zéro.}

\input{appendix/x86/instructions/TEST}
\myindex{x86!\Instructions!XOR}
  \item[XOR] op1, op2: \ac{XOR} \RU{значений}\EN{values}\FR{valeurs}. $op1=op1\oplus{}op2$.
  \RU{Часто встречающийся вариант \TT{XOR reg,reg} означает обнуление регистра \emph{reg}.}
  \EN{A frequently occurring pattern is \TT{XOR reg,reg}, which implies zeroing of \emph{reg}.}
  \FR{Un schéma récurrent est \TT{XOR reg,reg}, qui met \emph{reg} à zéro.}
  \EN{See also}\RU{См.также}\FR{Voir aussi}: \myref{XOR_property}.


\end{description}

\subsubsection{Instructions les moins fréquemment utilisées}

\begin{description}
\myindex{x86!\Instructions!BSF}
  \item[BSF] \emph{bit scan forward}, \RU{см. также}\EN{see also}\FR{voir aussi}: \myref{instruction_BSF}

\myindex{x86!\Instructions!BSR}
  \item[BSR] \emph{bit scan reverse}

\myindex{x86!\Instructions!BSWAP}
  \item[BSWAP] \emph{(byte swap)}, \RU{смена \glslink{endianness}{порядка байт} в значении}\EN{change value \gls{endianness}}%
  \FR{change le \glslink{endianness}{boutisme} de la valeur}.

\input{appendix/x86/instructions/BTC}
\input{appendix/x86/instructions/BTR}
\input{appendix/x86/instructions/BTS}
\input{appendix/x86/instructions/BT}
\myindex{x86!\Instructions!CBW}
\myindex{x86!\Instructions!CWD}
\myindex{x86!\Instructions!CDQ}
\myindex{x86!\Instructions!CWDE}
\myindex{x86!\Instructions!CDQE}
\label{ins:CBW_CWD_etc}
\item[CBW/CWD/CWDE/CDQ/CDQE]

Étendre le signe de la valeur:

\begin{description}
\item[CBW] Convertit l'octet dans AL en un mot dans AX
\item[CWD] Convertit le mot dans AX en double-mot dans DX:AX
\item[CWDE] Convertit le mot dans AX en double-mot dans EAX
\item[CDQ] Convertit le double-mot dans EAX en quadruple-mot dans EDX:EAX
\item[CDQE] (x64) Convertit le double-mot dans EAX en quadruple-mot dans RAX
\end{description}

Cette instruction examine le signe de la valeur, l'étend à la partie haute de la
valeur nouvellement construite. Voir aussi: \myref{subsec:sign_extending_32_to_64}.

\newcommand{\StephenMorse}{[Stephen P. Morse, \emph{The 8086 Primer}, (1980)]\footnote{\AlsoAvailableAs \url{https://archive.org/details/The8086Primer}}}

Il est intéressant de savoir que ces instructions furent initialement appelées \TT{SEX}
(\emph{Sign EXtend}), comme l'écrit Stephen P. Morse (un des concepteurs du CPU 8086)
dans \StephenMorse:

\begin{framed}
\begin{quotation}
The process of stretching numbers by extending the sign bit is called sign extension.
The 8086 provides instructions (Fig. 3.29) to facilitate the task of sign extension.
These instructions were initially named SEX (sign extend) but were later renamed to the more
conservative CBW (convert byte to word) and CWD (convert word to double word).
\end{quotation}
\end{framed}

\myindex{x86!\Instructions!CLD}
\myindex{x86!\Flags!DF}
  \item[CLD] \RU{сбросить флаг DF}\EN{clear DF flag}\FR{efface le flag DF}.

\myindex{x86!\Instructions!CLI}
\myindex{x86!\Flags!IF}
  \item[CLI] (M) \RU{сбросить флаг IF}\EN{clear IF flag}\FR{efface le flag IF}.

\input{appendix/x86/instructions/CMC}
\input{appendix/x86/instructions/CMOVcc_FR}
\myindex{\CStandardLibrary!memcmp()}
\myindex{x86!\Instructions!CMPSB}
\myindex{x86!\Instructions!CMPSW}
\myindex{x86!\Instructions!CMPSD}
\myindex{x86!\Instructions!CMPSQ}
\item[CMPSB/CMPSW/CMPSD/CMPSQ] (M) compare un octet/
mot de 16-bit/
mot de 32-bit/
mot de 64-bit
à partir de l'adresse qui se trouve dans SI/ESI/RSI avec la variable à l'adresse
stockée dans DI/EDI/RDI.

\label{REPE_CMPSx}
Avec le préfixe REP, elle est répétée en boucle, le compteur est stocké dans le registre
CX/ECX/RCX, le processus se répétera jusqu'à ce que le flag ZF soit zéro (i.e., jusqu'à
ce que les valeurs soient égales l'une à l'autre, d'où le \q{E} dans REPE).

Ca fonctionne comme memcmp() en C.

Exemple tiré du noyau de Windows NT (\ac{WRK} v1.2):

\lstinputlisting[caption=base\textbackslash{}ntos\textbackslash{}rtl\textbackslash{}i386\textbackslash{}movemem.asm,style=customasmx86]{appendix/x86/instructions/RtlCompareMemory_WRK12.asm}

N.B.: cette fonction utilise une comparaison 32-bit (CMPSD) si la taille du bloc
est un multiple de 4, ou sinon une comparaison par octet (CMPSB).


\input{appendix/x86/instructions/CPUID}
\input{appendix/x86/instructions/DIV}
\input{appendix/x86/instructions/IDIV}
\myindex{x86!\Instructions!INT}
\myindex{MS-DOS}

\item[INT] (M): \INS{INT x} est similaire à \INS{PUSHF; CALL dword ptr [x*4]}
en environnement 16-bit.
  Elle était énormément utilisée dans MS-DOS, fonctionnant comme un vecteur syscall.
  Les registres AX/BX/CX/DX/SI/DI étaient remplis avec les arguments et le flux sautait
  à l'adresse dans la table des vecteurs d'interruption (Interrupt Vector Table,
  située au début de l'espace d'adressage).
  Elle était répandue car INT a un opcode court (2 octets) et le programme qui a
  besoin d'un service MS-DOS ne doit pas déterminer l'adresse du point d'entrée de
  ce service.
\myindex{x86!\Instructions!IRET}
  Le gestionnaire d'interruption renvoie le contrôle du flux à l'appelant en utilisant
  l'instruction IRET.

  Le numéro d'interruption les plus utilisé était 0x21, servant une grande partie
  de on \ac{API}.
  Voir aussi: [Ralf Brown \emph{Ralf Brown's Interrupt List}],
  pour les listes d'interruption plus exhaustives et d'autres informations sur MS-DOS.

\myindex{x86!\Instructions!SYSENTER}
\myindex{x86!\Instructions!SYSCALL}
  Durant l'ère post-MS-DOS, cette instruction était toujours utilisée comme un syscall
  à la fois dans Linux et Windows (\myref{syscalls}), mais fût remplacée plus tard
  par les instructions SYSENTER ou SYSCALL.

\item[INT 3] (M): cette instruction est proche de
\INS{INT}, elle a son propre opcode d'1 octet (\GTT{0xCC}),
et est très utilisée pour le déboggage.
Souvent, les débogueurs écrivent simplement l'octet \GTT{0xCC} à l'adresse du point
d'arrêt à mettre, et lorsqu'une exception est levée, l'octet original est restauré
et l'instruction originale à cette adresse est ré-exécutée. \\
Depuis \gls{Windows NT}, une exception \GTT{EXCEPTION\_BREAKPOINT} est déclenchée
lorsque le \ac{CPU} exécute cette instruction.
Cet évènement de débogage peut être intercepté et géré par un débogueur hôte, si
il y en a un de chargé.
S'il n'y en a pas de charger, Windows propose de lancer un des débogueurs enregistré
dans le système.
Si \ac{MSVS} est installé, son débogueur peut être chargé et connecté au processus.
Afin de protéger contre le \gls{reverse engineering}, de nombreuses méthodes anti-débogage
vérifient l'intégrité du code chargé.

\ac{MSVC} possède une \glslink{compiler intrinsic}{fonction intrinsèque} pour l'instruction:
\GTT{\_\_debugbreak()}\footnote{\href{http://go.yurichev.com/17226}{MSDN}}.


Il y a aussi une fonction win32 dans kernel32.dll appelée
\GTT{DebugBreak()}\footnote{\href{http://go.yurichev.com/17227}{MSDN}},
qui exécute aussi \GTT{INT 3}.


\myindex{x86!\Instructions!IN}
\myindex{MS-DOS}
  \item[IN] (M) \RU{получить данные из порта}\EN{input data from port}\FR{lire des données depuis le port}.
	  \RU{Эту инструкцию обычно можно найти в драйверах OS либо в старом коде для MS-DOS,
	  например}%
	  \EN{The instruction usually can be seen in OS drivers or in old MS-DOS code,
	  for example}%
	  \FR{On trouve cette instruction dans les drivers de l'OS ou dans de l'ancien
	  code MS-DOS, par exemple} (\myref{IN_example}).

\input{appendix/x86/instructions/IRET}
\input{appendix/x86/instructions/LOOP_FR}
\input{appendix/x86/instructions/OUT}
\input{appendix/x86/instructions/POPA}
\input{appendix/x86/instructions/POPCNT}
\input{appendix/x86/instructions/POPF}
\input{appendix/x86/instructions/PUSHA}
\input{appendix/x86/instructions/PUSHF}
\input{appendix/x86/instructions/RCx}
\myindex{x86!\Instructions!ROL}
\myindex{x86!\Instructions!ROR}
\label{ROL_ROR}
\item[ROL/ROR] (M) décalage cyclique

ROL: rotation à gauche:

\input{rotate_left}

ROR: rotation à droite:

\input{rotate_right}

En dépit du fait que presque presque tous les \ac{CPU}s aient ces instructions, il n'y
a pas d'opération correspondante en \CCpp, donc les compilateurs de ces \ac{PL}s
ne génèrent en général pas ces instructions.

Par commodité pour le programmeur, au moins \ac{MSVC} fourni les pseudo-fonctions
(fonctions intrinsèques du compilateur)
\emph{\_rotl()} \AndENRU \emph{\_rotr()}\FNMSDNROTxURL{},
qui sont traduites directement par le compilateur en ces instructions.


\myindex{x86!\Instructions!SAL}
  \item[SAL] \RU{Арифметический сдвиг влево}\EN{Arithmetic shift left}\FR{Décalage
  arithmétique à gauche}, \RU{синонимично}\EN{synonymous to}\FR{synonyme de} \TT{SHL}

\input{appendix/x86/instructions/SAR}
\input{appendix/x86/instructions/SETcc}
\input{appendix/x86/instructions/STC}
\input{appendix/x86/instructions/STD_FR}
\input{appendix/x86/instructions/STI}
\input{appendix/x86/instructions/SYSCALL}
\input{appendix/x86/instructions/SYSENTER}
\input{appendix/x86/instructions/UD2}
\input{appendix/x86/instructions/XCHG}
\end{description}

\subsubsection{Instructions FPU}

Le suffixe \TT{-R} dans le mnémonique signifie en général que les opérandes sont
inversés, le suffixe \TT{-P} implique qu'un élément est supprimé de la pile après
l'exécution de l'instruction, le suffixe \TT{-PP} implique que deux éléments sont
supprimés.

Les instructions \TT{-P} sont souvent utiles lorsque nous n'avons plus besoin que
la valeur soit présente dans la pile FPU après l'opération.

\begin{description}
\input{appendix/x86/instructions/FABS}
\input{appendix/x86/instructions/FADD} % + FADDP
\input{appendix/x86/instructions/FCHS}
\input{appendix/x86/instructions/FCOM_FR} % + FCOMP + FCOMPP
\input{appendix/x86/instructions/FDIVR} % + FDIVRP
\input{appendix/x86/instructions/FDIV} % + FDIVP
\input{appendix/x86/instructions/FILD}
\myindex{x86!\Instructions!FIST}
\myindex{x86!\Instructions!FISTP}
  \item[FIST] op: convertit la valeur dans ST(0) en un entier dans op
  \item[FISTP] op: convertit la valeur dans ST(0) en un entier dans op;
  supprime un élément de la pile
 % + FISTP
\input{appendix/x86/instructions/FLD1}
\input{appendix/x86/instructions/FLDCW}
\input{appendix/x86/instructions/FLDZ}
\input{appendix/x86/instructions/FLD}
\input{appendix/x86/instructions/FMUL} % + FMULP
\input{appendix/x86/instructions/FSINCOS}
\input{appendix/x86/instructions/FSQRT}
\input{appendix/x86/instructions/FSTCW_FR} % + FNSTCW
\input{appendix/x86/instructions/FSTSW_FR} % + FNSTSW
\input{appendix/x86/instructions/FST}
\input{appendix/x86/instructions/FSUBR} % + FSUBRP
\input{appendix/x86/instructions/FSUB} % + FSUBP
\input{appendix/x86/instructions/FUCOM_FR} % + FUCOMP + FUCOMPP
\input{appendix/x86/instructions/FXCH}
\end{description}

%\subsubsection{\RU{SIMD-инструкции}\EN{SIMD instructions}}

% TODO

%\begin{description}
%\input{appendix/x86/instructions/DIVSD}
%\input{appendix/x86/instructions/MOVDQA}
%\input{appendix/x86/instructions/MOVDQU}
%\input{appendix/x86/instructions/PADDD}
%\input{appendix/x86/instructions/PCMPEQB}
%\input{appendix/x86/instructions/PLMULHW}
%\input{appendix/x86/instructions/PLMULLD}
%\input{appendix/x86/instructions/PMOVMSKB}
%\input{appendix/x86/instructions/PXOR}
%\end{description}

% SHLD !
% SHRD !
% BSWAP !
% CMPXCHG
% XADD !
% CMPXCHG8B
% RDTSC !
% PAUSE!

% xsave
% fnclex, fnsave
% movsxd, movaps, wait, sfence, lfence, pushfq
% prefetchw
% REP RETN
% REP BSF
% movnti, movntdq, rdmsr, wrmsr
% ldmxcsr, stmxcsr, invlpg
% swapgs
% movq, movd
% mulsd
% POR
% IRETQ
% pslldq
% psrldq
% cqo, fxrstor, comisd, xrstor, wbinvd, movntq
% fprem
% addsb, subsd, frndint

% rare:
%\item[ENTER]
%\item[LES]
% LDS
% XLAT

\subsubsection{Instructions ayant un opcode affichable en ASCII\EN{Instructions having printable ASCII opcode}}

(En mode 32-bit).

\label{printable_x86_opcodes}
\myindex{Shellcode}
Elles peuvent être utilisées pour la création de shellcode.
Voir aussi: \myref{subsec:EICAR}.

% FIXME: start at 0x20...
\begin{center}
\begin{longtable}{ | l | l | l | }
\hline
\HeaderColor caractère ASCII\EN{ character} & 
\HeaderColor code hexadécimal\EN{hexadecimal code} & 
\HeaderColor instruction x86\EN{ instruction} \\
\hline
0	 &30	 &XOR \\
1	 &31	 &XOR \\
2	 &32	 &XOR \\
3	 &33	 &XOR \\
4	 &34	 &XOR \\
5	 &35	 &XOR \\
7	 &37	 &AAA \\
8	 &38	 &CMP \\
9	 &39	 &CMP \\
:	 &3a	 &CMP \\
;	 &3b	 &CMP \\
<	 &3c	 &CMP \\
=	 &3d	 &CMP \\
?	 &3f	 &AAS \\
@	 &40	 &INC \\
A	 &41	 &INC \\
B	 &42	 &INC \\
C	 &43	 &INC \\
D	 &44	 &INC \\
E	 &45	 &INC \\
F	 &46	 &INC \\
G	 &47	 &INC \\
H	 &48	 &DEC \\
I	 &49	 &DEC \\
J	 &4a	 &DEC \\
K	 &4b	 &DEC \\
L	 &4c	 &DEC \\
M	 &4d	 &DEC \\
N	 &4e	 &DEC \\
O	 &4f	 &DEC \\
P	 &50	 &PUSH \\
Q	 &51	 &PUSH \\
R	 &52	 &PUSH \\
S	 &53	 &PUSH \\
T	 &54	 &PUSH \\
U	 &55	 &PUSH \\
V	 &56	 &PUSH \\
W	 &57	 &PUSH \\
X	 &58	 &POP \\
Y	 &59	 &POP \\
Z	 &5a	 &POP \\
\lbrack{}	 &5b	 &POP \\
\textbackslash{}	 &5c	 &POP \\
\rbrack{}	 &5d	 &POP \\
\verb|^|	 &5e	 &POP \\
\_	 &5f	 &POP \\
\verb|`|	 &60	 &PUSHA \\
a	 &61	 &POPA \\
h	 &68	 &PUSH\\
i	 &69	 &IMUL\\
j	 &6a	 &PUSH\\
k	 &6b	 &IMUL\\
p	 &70	 &JO\\
q	 &71	 &JNO\\
r	 &72	 &JB\\
s	 &73	 &JAE\\
t	 &74	 &JE\\
u	 &75	 &JNE\\
v	 &76	 &JBE\\
w	 &77	 &JA\\
x	 &78	 &JS\\
y	 &79	 &JNS\\
z	 &7a	 &JP\\
\hline
\end{longtable}
\end{center}

Also: % TBT
\begin{center}
\begin{longtable}{ | l | l | l | }
\hline
\HeaderColor caractère ASCII\EN{ character} & 
\HeaderColor code hexadécimal\EN{hexadecimal code} & 
\HeaderColor instruction x86\EN{ instruction} \\
\hline
f	 &66	 &en mode 32-bit, change pour une\EN{(in 32-bit mode) switch to}\\
   & & taille d'opérande de 16-bit\EN{16-bit operand size} \\
g	 &67	 &en mode 32-bit, change pour une\EN{(in 32-bit mode) switch to}\\
   & & taille d'adresse 16-bit\EN{16-bit address size} \\
\hline
\end{longtable}
\end{center}

\myindex{x86!\Instructions!AAA}
\myindex{x86!\Instructions!AAS}
\myindex{x86!\Instructions!CMP}
\myindex{x86!\Instructions!DEC}
\myindex{x86!\Instructions!IMUL}
\myindex{x86!\Instructions!INC}
\myindex{x86!\Instructions!JA}
\myindex{x86!\Instructions!JAE}
\myindex{x86!\Instructions!JB}
\myindex{x86!\Instructions!JBE}
\myindex{x86!\Instructions!JE}
\myindex{x86!\Instructions!JNE}
\myindex{x86!\Instructions!JNO}
\myindex{x86!\Instructions!JNS}
\myindex{x86!\Instructions!JO}
\myindex{x86!\Instructions!JP}
\myindex{x86!\Instructions!JS}
\myindex{x86!\Instructions!POP}
\myindex{x86!\Instructions!POPA}
\myindex{x86!\Instructions!PUSH}
\myindex{x86!\Instructions!PUSHA}
\myindex{x86!\Instructions!XOR}

En résumé\EN{In summary}:
AAA, AAS, CMP, DEC, IMUL, INC, JA, JAE, JB, JBE, JE, JNE, JNO, JNS, JO, JP, JS, POP, POPA, PUSH, PUSHA, 
XOR.

 % subsection
\subsection{npad}
\label{sec:npad}

\RU{Это макрос в ассемблере, для выравнивания некоторой метки по некоторой границе.}
\EN{It is an assembly language macro for aligning labels on a specific boundary.}
\DE{Dies ist ein Assembler-Makro um Labels an bestimmten Grenzen auszurichten.}
\FR{C'est une macro du langage d'assemblage pour aligner les labels sur une limite
spécifique.}

\RU{Это нужно для тех \emph{нагруженных} меток, куда чаще всего передается управление, например, 
начало тела цикла. 
Для того чтобы процессор мог эффективнее вытягивать данные или код из памяти, через шину с памятью, 
кэширование, итд.}
\EN{That's often needed for the busy labels to where the control flow is often passed, e.g., loop body starts.
So the CPU can load the data or code from the memory effectively, through the memory bus, cache lines, etc.}
\DE{Dies ist oft nützlich Labels, die oft Ziel einer Kotrollstruktur sind, wie Schleifenköpfe.
Somit kann die CPU Daten oder Code sehr effizient vom Speicher durch den Bus, den Cache, usw. laden.}
\FR{C'est souvent nécessaire pour des labels très utilisés, comme par exemple le
début d'un corps de boucle. Ainsi, le CPU peut charger les données ou le code depuis
la mémoire efficacement, à travers le bus mémoire, les caches, etc.}

\RU{Взято из}\EN{Taken from}\DE{Entnommen von}\FR{Pris de} \TT{listing.inc} (MSVC):

\myindex{x86!\Instructions!NOP}
\RU{Это, кстати, любопытный пример различных вариантов \NOP{}-ов. 
Все эти инструкции не дают никакого эффекта, но отличаются разной длиной.}
\EN{By the way, it is a curious example of the different \NOP variations.
All these instructions have no effects whatsoever, but have a different size.}
\DE{Dies ist übrigens ein Beispiel für die unterschiedlichen \NOP-Variationen.
Keine dieser Anweisungen hat eine Auswirkung, aber alle haben eine unterschiedliche Größe.}
\FR{À propos, c'est un exemple curieux des différentes variations de \NOP. Toutes
ces instructions n'ont pas d'effet, mais ont une taille différente.}

\RU{Цель в том, чтобы была только одна инструкция, а не набор NOP-ов, 
считается что так лучше для производительности CPU.}
\EN{Having a single idle instruction instead of couple of NOP-s,
is accepted to be better for CPU performance.}
\DE{Eine einzelne Idle-Anweisung anstatt mehrerer NOPs hat positive Auswirkungen
auf die CPU-Performance.}
\FR{Avoir une seule instruction sans effet au lieu de plusieurs est accepté comme
étant meilleur pour la performance du CPU.}

\begin{lstlisting}[style=customasmx86]
;; LISTING.INC
;;
;; This file contains assembler macros and is included by the files created
;; with the -FA compiler switch to be assembled by MASM (Microsoft Macro
;; Assembler).
;;
;; Copyright (c) 1993-2003, Microsoft Corporation. All rights reserved.

;; non destructive nops
npad macro size
if size eq 1
  nop
else
 if size eq 2
   mov edi, edi
 else
  if size eq 3
    ; lea ecx, [ecx+00]
    DB 8DH, 49H, 00H
  else
   if size eq 4
     ; lea esp, [esp+00]
     DB 8DH, 64H, 24H, 00H
   else
    if size eq 5
      add eax, DWORD PTR 0
    else
     if size eq 6
       ; lea ebx, [ebx+00000000]
       DB 8DH, 9BH, 00H, 00H, 00H, 00H
     else
      if size eq 7
	; lea esp, [esp+00000000]
	DB 8DH, 0A4H, 24H, 00H, 00H, 00H, 00H 
      else
       if size eq 8
        ; jmp .+8; .npad 6
	DB 0EBH, 06H, 8DH, 9BH, 00H, 00H, 00H, 00H
       else
        if size eq 9
         ; jmp .+9; .npad 7
         DB 0EBH, 07H, 8DH, 0A4H, 24H, 00H, 00H, 00H, 00H
        else
         if size eq 10
          ; jmp .+A; .npad 7; .npad 1
          DB 0EBH, 08H, 8DH, 0A4H, 24H, 00H, 00H, 00H, 00H, 90H
         else
          if size eq 11
           ; jmp .+B; .npad 7; .npad 2
           DB 0EBH, 09H, 8DH, 0A4H, 24H, 00H, 00H, 00H, 00H, 8BH, 0FFH
          else
           if size eq 12
            ; jmp .+C; .npad 7; .npad 3
            DB 0EBH, 0AH, 8DH, 0A4H, 24H, 00H, 00H, 00H, 00H, 8DH, 49H, 00H
           else
            if size eq 13
             ; jmp .+D; .npad 7; .npad 4
             DB 0EBH, 0BH, 8DH, 0A4H, 24H, 00H, 00H, 00H, 00H, 8DH, 64H, 24H, 00H
            else
             if size eq 14
              ; jmp .+E; .npad 7; .npad 5
              DB 0EBH, 0CH, 8DH, 0A4H, 24H, 00H, 00H, 00H, 00H, 05H, 00H, 00H, 00H, 00H
             else
              if size eq 15
               ; jmp .+F; .npad 7; .npad 6
               DB 0EBH, 0DH, 8DH, 0A4H, 24H, 00H, 00H, 00H, 00H, 8DH, 9BH, 00H, 00H, 00H, 00H
              else
	       %out error: unsupported npad size
               .err
              endif
             endif
            endif
           endif
          endif
         endif
        endif
       endif
      endif
     endif
    endif
   endif
  endif
 endif
endif
endm
\end{lstlisting}
 % subsection

