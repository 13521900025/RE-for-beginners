\subsection{Instructions}


Il il y a un suffixe \emph{-S}  pour certaines instructions en ARM, indiquant que
l'instruction met les flags en fonction du résultat.
Les instructions qui n'ont pas ce suffixe ne modifient pas les flags.
\myindex{ARM!\Instructions!ADD}
\myindex{ARM!\Instructions!ADDS}
\myindex{ARM!\Instructions!CMP}
Par exemple \TT{ADD} contrairement à \TT{ADDS}
ajoute deux nombres, mais les flags sont inchangés.
De telles instructions sont pratiques à utiliser entre \CMP où les flags sont mis et, e.g.
les sauts conditionnels, où les flags sont utilisés.
Elles sont aussi meilleures en termes d'analyse de dépendance de données (car moins
de registres sont modifiés pendant leurs exécution).

% ADD
% ADDAL
% ADDCC
% ADDS
% ADR
% ADREQ
% ADRGT
% ADRHI
% ADRNE
% ASRS
% B
% BCS
% BEQ
% BGE
% BIC
% BL
% BLE
% BLEQ
% BLGT
% BLHI
% BLS
% BLT
% BLX
% BNE
% BX
% CMP
% IDIV
% IT
% LDMCSFD
% LDMEA
% LDMED
% LDMFA
% LDMFD
% LDMGEFD
% LDR.W
% LDR
% LDRB.W
% LDRB
% LDRSB
% LSL.W
% LSL
% LSLS
% MLA
% MOV
% MOVT.W
% MOVT
% MOVW
% MULS
% MVNS
% ORR
% POP
% PUSH
% RSB
% SMMUL
% STMEA
% STMED
% STMFA
% STMFD
% STMIA
% STMIB
% STR
% SUB
% SUBEQ
% SXTB
% TEST
% TST
% VADD
% VDIV
% VLDR
% VMOV
% VMOVGT
% VMRS
% VMUL
%\myindex{ARM!Optional operators!ASR
%\myindex{ARM!Optional operators!LSL
%\myindex{ARM!Optional operators!LSR
%\myindex{ARM!Optional operators!ROR
%\myindex{ARM!Optional operators!RRX

% AArch64
% RET is BR X30 or BR LR but with additional hint to CPU

\subsubsection{Table des codes conditionnels}

% TODO rework this!
\small
\begin{center}
\begin{tabular}{ | l | l | l | }
\hline
\HeaderColor Code & 
\HeaderColor Description & 
\HeaderColor Flags \\
\hline
EQ & Égal & Z == 1 \\
\hline
NE & Non égal & Z == 0 \\
\hline
CS \ac{AKA} HS (Higher or Same) & Retenue mise/ Non-signé, Plus grand que, égal & C == 1 \\
\hline
CC \ac{AKA} LO (LOwer) & Retenue à zéro / non-signé, moins que & C == 0 \\
\hline
MI & moins, négatif / moins que & N == 1 \\
\hline
PL & plus, positif ou zéro / Plus grnad que, égal & N == 0 \\
\hline
VS & débordement & V == 1 \\
\hline
VC & Pas de débordement & V == 0 \\
\hline
HI & non signé supérieur / plus grand que & C == 1 \AndENRU \\
 & & Z == 0 \\
\hline
LS & non signé inférieur ou égal / inférieur ou égal & C == 0 \OrENRU \\
 & & Z == 1 \\
\hline
GE & signé supérieur ou égal / supérieur ou égal & N == V \\
\hline
LT & signé plus petit que / plus petit que & N != V \\
\hline
GT & signé plus grand que / plus grnad que & Z == 0 \AndENRU \\
 & & N == V \\
\hline
LE & signé inférieur ou égal / moins que, égal & Z == 1 \OrENRU \\
 & & N != V \\
\hline
None / AL & toujours & n'importe lequel \\
\hline
\end{tabular}
\end{center}
\normalsize
