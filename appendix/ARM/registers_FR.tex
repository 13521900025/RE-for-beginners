\subsection{ARM 32-bit (AArch32)}

\subsubsection{Registres d'usage général}

\begin{itemize}
\myindex{ARM!\Registers!R0}
	\item R0 --- le résultat d'une fonction est en général renvoyé dans R0
	\item R1...R12 --- \ac{GPR}s
	\item R13 --- \ac{AKA} SP (\glslink{stack pointer}{pointeur de pile})
\myindex{ARM!\Registers!Link Register}
	\item R14 --- \ac{AKA} LR (\gls{link register})
	\item R15 --- \ac{AKA} PC (program counter)
\end{itemize}

\myindex{ARM!\Registers!scratch registers}
\Reg{0}-\Reg{3} sont aussi appelés \q{registres scratch}: les arguments de la fonctions sont
d'habitude passés par eux, et leurs valeurs n'ont pas besoin d'être restaurées en sortant de la
fonction.

\subsubsection{Current Program Status Register (CPSR)}

\begin{center}
\begin{tabular}{ | l | l | }
\hline
\headercolor\ Bit &
\headercolor\ Description \\
\hline
0..4           & M --- processor mode \\
\hline
5              & T --- Thumb state \\
\hline
6              & F --- FIQ disable \\
\hline
7              & I --- IRQ disable \\
\hline
8              & A --- imprecise data abort disable \\
\hline
9              & E --- data endianness \\
\hline
10..15, 25, 26 & IT --- if-then state \\
\hline
16..19         & GE --- greater-than-or-equal-to \\
\hline
20..23         & DNM --- do not modify \\
\hline
24             & J --- Java state \\
\hline
27             & Q --- sticky overflow \\
\hline
28             & V --- overflow \\
\hline
29             & C --- carry/borrow/extend \\
\hline
\myindex{ARM!\Registers!Z}
30             & Z --- zero bit \\
\hline
31             & N --- negative/less than \\
\hline
\end{tabular}
\end{center}

% TODO
% \myindex{ARM!\Registers!APSR}
% \subsubsection{Application Program Status Register (APSR)}

% TODO
% \myindex{ARM!\Registers!FPSCR}
% \subsubsection{Floating-Point Status and Control Register (FPPSR)}
% http://infocenter.arm.com/help/index.jsp?topic=/com.arm.doc.ddi0344b/Chdfafia.html

\subsubsection{Registres VFP (virgule flottante) et registres NEON}
\label{ARM_VFP_registers}

% http://infocenter.arm.com/help/index.jsp?topic=/com.arm.doc.dht0002a/ch01s03s02.html

\myindex{ARM!D-\registers{}}
\myindex{ARM!S-\registers{}}
\begin{center}
\begin{tabular}{ | l | l | l | l | }
\hline
0..31\textsuperscript{bits} & 32..64 & 65..96 & 97..127 \\
\hline
\multicolumn{4}{ | c | }{Q0\textsuperscript{128 bits}} \\
\hline
\multicolumn{2}{ | c | }{D0\textsuperscript{64 bits}} & \multicolumn{2}{ c | }{D1} \\
\hline
S0\textsuperscript{32 bits} & S1 & S2 & S3 \\
\hline
\end{tabular}
\end{center}

Les registres-S sont 32-bit, utilisés pour le stockage de nombre en simple précision.
Les registres-D sont 64-bit, utilisés pour le stockage de nombre en double précision.

Les registres-D et -S partagent le même espace physique dans le CPU---il est possible d'accéder
un registre-D via les registres-S (mais c'est insensé).

De même, les registres \gls{NEON} sont des 128-bit et partagent le même espace physique dans le CPU
avec les autres registres en virgule flottante.

En VFP les registres-S sont présents: S0..S31.

En VFPv2 16 registres-D ont été ajoutés, qui occupent en fait le même espace que S0..S31.

En VFPv3 (\gls{NEON} ou \q{SIMD avancé}) il y a 16 registres-D de plus, D0..D31, mais les registres
D16..D31 ne partagent pas l'espace avec aucun autre registres-S.

En \gls{NEON} ou \q{SIMD avancé} 16 autres registres-Q 128-bit ont été ajoutés,
qui partagent le même espace que D0..D31.

\subsection{ARM 64-bit (AArch64)}

\subsubsection{Registres d'usage général}
\label{ARM64_GPRs}

Le nombre de registres a été doublé depuis AArch32.

\begin{itemize}
\myindex{ARM!\Registers!X0}
	\item X0 --- le résultat d'une fonction est en général renvoyé dans X0
	\item X0...X7 --- Les arguments de fonction sont passés ici
	\item X8
	\item X9...X15 --- sont des registres temporaires, la fonction appelée peut les utiliser sans en
	restaurer le contenu.
	\item X16
	\item X17
	\item X18
	\item X19...X29 --- la fonction appelée peut les utiliser mais doit restaurer leurs valeurs à sa
	sortie.
	\item X29 --- utilisé comme \ac{FP} (au moins dans GCC)
	\item X30 --- \q{Procedure Link Register} \ac{AKA} \ac{LR} (\gls{link register}).
	\item X31 --- ce registre contient toujours zéro \ac{AKA} XZR ou ZR \q{Zero Register}.
	Sa partie 32-bit est appelée WZR.
	\item \ac{SP}, n'est plus un registre d'usage général.
\end{itemize}

Voir aussi: \ARMPCS.

La partie 32-bit de chaque registre-X est aussi accessible par les registres-W (W0, W1, etc.).

\begin{center}
\begin{tabular}{ | l | l | }
\hline
\RU{Старшие 32 бита}\EN{High 32-bit part}\ES{Parte alta de 32 bits}\PTBRph{}\PLph{}\ITph{}\DE{Oberer 32-Bit-Teil}\THAph{}\NLph{}\FR{Partie 32 bits haute} & \RU{младшие 32 бита}\EN{low 32-bit part}\ES{parte baja de 32 bits}\PTBRph{}\PL{Starsze 32 bity}\ITph{}\DE{Unterer 32-Bit-Teil}\THAph{}\NLph{}\FR{Partie 32 bits basse} \\
\hline
\multicolumn{2}{ | c | }{X0} \\
\hline
\multicolumn{1}{ | c | }{} & \multicolumn{1}{ c | }{W0} \\
\hline
\end{tabular}
\end{center}

