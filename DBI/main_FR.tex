\chapter{Dynamic binary instrumentation}

Les outils \ac{DBI} peuvent être vus comme des débogueurs très avancés et rapide.

% sections
\section{Utiliser PIN DBI pour intercepter les XOR}

\newcommand{\GitHubPinXORURL}{https://github.com/DennisYurichev/RE-for-beginners/tree/master/DBI/XOR/files}

PIN d'Intel est un outil \ac{DBI}.
Cela signifie qu'il prend un binaire compilé et y insère vos instructions, où vous
voulez.

Essayons d'intercepter toutes les instructions XOR.
Elles sont utilisées intensément en cryptographie, et nous pouvons essayer de lancer l'archiveur
WinRAR en mode chiffrement avec l'espoir que des instructions sont effectivement utilisées durant
le chiffrement.

Voici le code source de mon outil PIN: \url{\GitHubPinXORURL/XOR_ins.cpp}.

Le code est presque auto-documenté: il scanne le fichier exécutable en entrée à la recherche
des instructions XOR/PXOR et insère un appel à notre fonction avant chaque.
La fonction log\_info() vérifie d'abord si les opérandes sont différents (puisque
l'instruction XOR est souvent utilisée pour effacer simplement un registre, comme
XOR EAX, EAX), et si ils sont différents, il incrémente un compteur à cette EIP/RIP,
afin que les statistiques soient collectées.

J'ai préparé deux fichiers pour tester: test1.bin (30720 octets) et test2.bin (5547752 octets),
je vais les compresser avec RAR avec un mot de passe et voir les différences dans
les statistiques.

Vous devez aussi désactiver \ac{ASLR}
\footnote{\url{https://stackoverflow.com/q/9560993}},
afin que l'outil PIN rapporte les mêmes RIPs que dans l'exécutable RAR.

Maintenant, lançons-le:

\begin{lstlisting}
c:\pin-3.2-81205-msvc-windows\pin.exe -t XOR_ins.dll -- rar a -pLongPassword tmp.rar test1.bin
c:\pin-3.2-81205-msvc-windows\pin.exe -t XOR_ins.dll -- rar a -pLongPassword tmp.rar test2.bin
\end{lstlisting}

Maintenant voici les statistiques pour test1.bin: \\
\url{\GitHubPinXORURL/XOR_ins.out.test1}.
... et pour test2.bin: \\
\url{\GitHubPinXORURL/XOR_ins.out.test2}.
Jusqu'ici, vous pouvez ignorer toutes les adresses autres que ip=0x1400xxxxx, qui
sont dans d'autres DLLs.

Maintenant, regardons la différence: \url{\GitHubPinXORURL/XOR_ins.diff}.

Certaines instructions XOR sont exécutées plus souvent pour test2.bin (qui est plus
gros) que pour test1.bin (qui est plus petit).
Donc elles sont clairement liées à la taille du fichier!

Le premier bloc de différence est:

\begin{lstlisting}
< ip=0x140017b21 count=0xd84
< ip=0x140017b48 count=0x81f
< ip=0x140017b59 count=0x858
< ip=0x140017b6a count=0xc13
< ip=0x140017b7b count=0xefc
< ip=0x140017b8a count=0xefd
< ip=0x140017b92 count=0xb86
< ip=0x140017ba1 count=0xf01
---
> ip=0x140017b21 count=0x9eab5
> ip=0x140017b48 count=0x79863
> ip=0x140017b59 count=0x862e8
> ip=0x140017b6a count=0x99495
> ip=0x140017b7b count=0xa891c
> ip=0x140017b8a count=0xa89f4
> ip=0x140017b92 count=0x8ed72
> ip=0x140017ba1 count=0xa8a8a
\end{lstlisting}

C'est en effet une sorte de boucle à l'intérieur de RAR.EXE:

\begin{lstlisting}
.text:0000000140017B21 loc_140017B21:
.text:0000000140017B21                 xor     r11d, [rbx]
.text:0000000140017B24                 mov     r9d, [rbx+4]
.text:0000000140017B28                 add     rbx, 8
.text:0000000140017B2C                 mov     eax, r9d
.text:0000000140017B2F                 shr     eax, 18h
.text:0000000140017B32                 movzx   edx, al
.text:0000000140017B35                 mov     eax, r9d
.text:0000000140017B38                 shr     eax, 10h
.text:0000000140017B3B                 movzx   ecx, al
.text:0000000140017B3E                 mov     eax, r9d
.text:0000000140017B41                 shr     eax, 8
.text:0000000140017B44                 mov     r8d, [rsi+rdx*4]
.text:0000000140017B48                 xor     r8d, [rsi+rcx*4+400h]
.text:0000000140017B50                 movzx   ecx, al
.text:0000000140017B53                 mov     eax, r11d
.text:0000000140017B56                 shr     eax, 18h
.text:0000000140017B59                 xor     r8d, [rsi+rcx*4+800h]
.text:0000000140017B61                 movzx   ecx, al
.text:0000000140017B64                 mov     eax, r11d
.text:0000000140017B67                 shr     eax, 10h
.text:0000000140017B6A                 xor     r8d, [rsi+rcx*4+1000h]
.text:0000000140017B72                 movzx   ecx, al
.text:0000000140017B75                 mov     eax, r11d
.text:0000000140017B78                 shr     eax, 8
.text:0000000140017B7B                 xor     r8d, [rsi+rcx*4+1400h]
.text:0000000140017B83                 movzx   ecx, al
.text:0000000140017B86                 movzx   eax, r9b
.text:0000000140017B8A                 xor     r8d, [rsi+rcx*4+1800h]
.text:0000000140017B92                 xor     r8d, [rsi+rax*4+0C00h]
.text:0000000140017B9A                 movzx   eax, r11b
.text:0000000140017B9E                 mov     r11d, r8d
.text:0000000140017BA1                 xor     r11d, [rsi+rax*4+1C00h]
.text:0000000140017BA9                 sub     rdi, 1
.text:0000000140017BAD                 jnz     loc_140017B21
\end{lstlisting}

Que fait-elle? Aucune idée à ce stade.

La suivante:

\begin{lstlisting}
< ip=0x14002c4f1 count=0x4fce
---
> ip=0x14002c4f1 count=0x4463be
\end{lstlisting}

0x4fce est 20430, qui est proche de la taille de test1.bin (30720 octets).
0x4463be est 4481982, qui est proche de la taille de test2.bin (5547752 octets).
Par égal, mais proche.

Ceci est un morceau de code avec cette instruction XOR:

\begin{lstlisting}
.text:000000014002C4EA loc_14002C4EA:
.text:000000014002C4EA                 movzx   eax, byte ptr [r8]
.text:000000014002C4EE                 shl     ecx, 5
.text:000000014002C4F1                 xor     ecx, eax
.text:000000014002C4F3                 and     ecx, 7FFFh
.text:000000014002C4F9                 cmp     [r11+rcx*4], esi
.text:000000014002C4FD                 jb      short loc_14002C507
.text:000000014002C4FF                 cmp     [r11+rcx*4], r10d
.text:000000014002C503                 ja      short loc_14002C507
.text:000000014002C505                 inc     ebx
\end{lstlisting}

Le corps de la boucle peut être écrit comme:

\begin{lstlisting}
state = input_byte ^ (state<<5) & 0x7FFF}.
\end{lstlisting}

\emph{state} est ensuite utilisé comme un index dans une table. Est-ce une sorte de \ac{CRC}?
Je ne sais pas, mais ça pourrait être une routine effectuant une somme de contrôle.
Ou peut-être une routine \ac{CRC} optimisée?
Une idée?

Le bloc suivant:

\begin{lstlisting}
< ip=0x14004104a count=0x367
< ip=0x140041057 count=0x367
---
> ip=0x14004104a count=0x24193
> ip=0x140041057 count=0x24193
\end{lstlisting}

\begin{lstlisting}
.text:0000000140041039 loc_140041039:
.text:0000000140041039                 mov     rax, r10
.text:000000014004103C                 add     r10, 10h
.text:0000000140041040                 cmp     byte ptr [rcx+1], 0
.text:0000000140041044                 movdqu  xmm0, xmmword ptr [rax]
.text:0000000140041048                 jz      short loc_14004104E
.text:000000014004104A                 pxor    xmm0, xmm1
.text:000000014004104E
.text:000000014004104E loc_14004104E:
.text:000000014004104E                 movdqu  xmm1, xmmword ptr [rcx+18h]
.text:0000000140041053                 movsxd  r8, dword ptr [rcx+4]
.text:0000000140041057                 pxor    xmm1, xmm0
.text:000000014004105B                 cmp     r8d, 1
.text:000000014004105F                 jle     short loc_14004107C
.text:0000000140041061                 lea     rdx, [rcx+28h]
.text:0000000140041065                 lea     r9d, [r8-1]
.text:0000000140041069
.text:0000000140041069 loc_140041069:
.text:0000000140041069                 movdqu  xmm0, xmmword ptr [rdx]
.text:000000014004106D                 lea     rdx, [rdx+10h]
.text:0000000140041071                 aesenc  xmm1, xmm0
.text:0000000140041076                 sub     r9, 1
.text:000000014004107A                 jnz     short loc_140041069
.text:000000014004107C
\end{lstlisting}

Ce morceau possède les instructions PXOR et AESENC (la dernière est une instruction
de chiffrement \ac{AES}).
Donc oui, nous avons trouvé une fonction de chiffrement, RAR utilise \ac{AES}.

Il y a ensuite un autre gros bloc d'instructions XOR presque contigus:

\begin{lstlisting}
< ip=0x140043e10 count=0x23006
---
> ip=0x140043e10 count=0x23004
499c510
< ip=0x140043e56 count=0x22ffd
---
> ip=0x140043e56 count=0x23002
\end{lstlisting}

Mais le compteur n'est pas très différent pendant la compression/chiffrement de test1.bin/test2.bin.
Qu'y a-t-il à ces adresses?

\begin{lstlisting}
.text:0000000140043E07                 xor     ecx, r9d
.text:0000000140043E0A                 mov     r11d, eax
.text:0000000140043E0D                 and     ecx, r10d
.text:0000000140043E10                 xor     ecx, r8d
.text:0000000140043E13                 rol     eax, 8
.text:0000000140043E16                 and     eax, esi
.text:0000000140043E18                 ror     r11d, 8
.text:0000000140043E1C                 add     edx, 5A827999h
.text:0000000140043E22                 ror     r10d, 2
.text:0000000140043E26                 add     r8d, 5A827999h
.text:0000000140043E2D                 and     r11d, r12d
.text:0000000140043E30                 or      r11d, eax
.text:0000000140043E33                 mov     eax, ebx
\end{lstlisting}

Googlons la constante 5A827999h... ceci ressemble à du SHA-1! mais pourquoi RAR utiliserait-il
SHA-1 pendant le chiffrement?

Voici la réponse:

\begin{lstlisting}
In comparison, WinRAR uses its own key derivation scheme that requires (password length * 2 + 11)*4096 SHA-1 transformations. That’s why it takes longer to brute-force attack encrypted WinRAR archives.
\end{lstlisting}
( \url{http://www.tomshardware.com/reviews/password-recovery-gpu,2945-8.html} )

C'est la génération de la clef: le mot de passe entré est hashé plusieurs fois et
le hash est utilisé comme clef \ac{AES}.
C'est pourquoi nous voyons que le comptage de l'instruction XOR est presque inchangé
lorsque nous passons au fichier de test plus gros.

C'est tout ce qu'il faut faire, ça m'a pris quelques heures d'écrire cet outil et
d'obtenir au moins 3 éléments: 1) c'est probablement une somme de contrôle; 2) chiffrement
\ac{AES}; 3) calcul de somme SHA-1.
La première fonction est encore un mystère pour moi.

Cependant, ceci est impressionnant, car j'ai ne me suis pas plongé dans le code de
RAR (qui est propriétaire, bien sûr). Je n'ai même pas jeté un coup d'\oe{}il dans
le code source de UnRAR (qui est disponible).

Les fichiers, incluant les fichiers de test et l'exécutable RAR que j'ai utilisé (win64, 5.40): \\
\url{\GitHubPinXORURL}.


% TODO: \section{Cracker Minesweeper avec PIN}

\newcommand{\GitHubMinesweeperURL}{https://github.com/DennisYurichev/RE-for-beginners/tree/master/DBI/minesweeper}

Dans ce livre, j'ai expliqué comment cracker Minesweeper pour Windows XP: \myref{minesweeper_winxp}.

Le Minesweeper de Windows Vista et 7 est différent: il a probablement été (r)écrit
en C++, et l'information de la case n'est maintenant plus stockée dans un tableau
global, mais plutôt dans des blocs du heap alloués par malloc.

Ceci est un cas où nous pouvons essayer l'outil PIN DBI.

\subsection{Intercepter tous les appels à rand()}

Tout d'abord, puisque Minesweeper dispose les mines aléatoirement, il doit appeler
rand() ou une fonction similaire.
Essayons d'intercepter tous les appels à rand(): \url{\GitHubMinesweeperURL/minesweeper1.cpp}.

Nous pouvons maintenant le lancer:

\begin{lstlisting}
c:\pin-3.2-81205-msvc-windows\pin.exe -t minesweeper1.dll -- C:\PATH\TO\MineSweeper.exe
\end{lstlisting}

Durant le démarrage, PIN cherche tous les appels à la fonction rand() et ajoute un
hook juste après chaque appel.
Le hook est la fonction RandAfter() que nous avons défini: elle logue la valeur et l'adresse
de retour.
Voici un log que j'ai obtenu en lançant la configuration 9*9 standard (10 mines):
 \url{\GitHubMinesweeperURL/minesweeper1.out.10mines}.
La fonction rand() a été appelée de nombreuses fois depuis différents endroits, mais
a été appelée depuis 0x10002770d exactement 10 fois.
J'ai changé la configuration de Minesweeper à 16*16 (40 mines) et rand() a été appelée
40 fois depuis 0x10002770d.
Donc oui, c'est ce que l'on cherche.
Lorsque je charge minesweeper.exe (depuis Windows 7) dans IDA et une fois que le
PDB est récupéré depuis le site web de Microsoft, la fonction qui appelle rand()
en 0x10002770d est appelée Board::placeMines().

\subsection{Remplacer les appels à rand() par notre function}

Essayons maintenant de remplacer la fonction rand() avec notre version, qui renvoie
toujours zéro: \url{\GitHubMinesweeperURL/minesweeper2.cpp}.
Durant le démarrage, PIN remplace tous les appels à la fonction rand() par des appels
à notre fonction, qui écrit dans le log et renvoie zéro.
Ok, je l'ai lancé et ai cliqué sur la case la plus en haut à gauche.

\begin{figure}[H]
\centering
\myincludegraphicsSmall{DBI/minesweeper/minesweeper1.png}
%\caption{TODO}
\end{figure}

Oui, contrairement à Minesweeper de Windows XP, les mines sont placées aléatoirement
\emph{après} que l'utilisateur ai cliqué sur une case, afin de garantir qu'il n'y
a pas de mine sur la première case cliquée par l'utilisateur.
Donc Minesweeper a placé les mines dans des cases autres que celle la plus en
haut à gauche (où j'ai cliqué).

Maintenant j'ai cliqué sur la case la plus en haut à droite:

\begin{figure}[H]
\centering
\myincludegraphicsSmall{DBI/minesweeper/minesweeper2.png}
%\caption{TODO}
\end{figure}

Ceci peut-être une sorte de blague? Je ne sais pas.

J'ai cliqué sur la 5ème case (droite du milieu) de la 1ère ligne:

\begin{figure}[H]
\centering
\myincludegraphicsSmall{DBI/minesweeper/minesweeper3.png}
%\caption{TODO}
\end{figure}

C'est bien, car Minesweeper peut effectuer un placement correct même avec un \ac{PRNG}
aussi mauvais!

\subsection{Regarder comment les mines sont placées}

Comment pouvons-nous obtenir des informations sur où les mines sont placées?
Le résultat de rand() semble être inutile: elle renvoie zéro à chaque fois, mais
Minesweeper a réussi à placer les mines dans des cases différentes, quoique, alignées.

Ce Minesweeper est aussi écrit dans la tradition C++, donc il n'a pas de tableau
global.

Mettons-nous dans la peau du programmeur.
Il doit y avoir une boucle comme:

\begin{lstlisting}
for (int i; i<mines_total; i++)
{
	// get coordinates using rand()
	// put a cell: in other words, modify a block allocated in heap
};
\end{lstlisting}

Comment pouvons-nous obtenir des information sur le bloc de qui est modifié à la
2nde étape?
Ce que nous devons faire:
1) suivre toutes les allocations dans la heap en interceptant malloc()/realloc()/free().
2) suivre toutes les écritures en mémoire (lent).
3) suivre les appels à rand().

Maintenant l'algorithme:
1) suivre tous les blocs du heap qui sont modifiés entre le 1er et le 2nd appel à
rand() depuis 0x10002770d;
2) à chaque fois qu'un bloc du heap est libéré, afficher son contenu.

Suivre toutes les écritures en mémoire est lent, mais après le 2nd appel à rand(),
nous n'avons plus besoin de les suivre (puisque nous avons déjà obtenu une liste
de blocs intéressants à ce point), donc nous arrêtons.

Maintenant le code: \url{\GitHubMinesweeperURL/minesweeper3.cpp}.

Il s'avère que seulement 4 blocs de heap sont modifiés entre les deux premiers appels
à rand(), voici à quoi ils ressemblent:

\begin{lstlisting}
free(0x20aa6360)
free(): we have this block in our records, size=0x28
0x20AA6360: 36 00 00 00 4E 00 00 00-2D 00 00 00 29 00 00 00 "6...N...-...)..."
0x20AA6370: 06 00 00 00 37 00 00 00-35 00 00 00 19 00 00 00 "....7...5......."
0x20AA6380: 46 00 00 00 0B 00 00 00-                        "F.......        "

...

free(0x20af9d10)
free(): we have this block in our records, size=0x18
0x20AF9D10: 0A 00 00 00 0A 00 00 00-0A 00 00 00 00 00 00 00 "................"
0x20AF9D20: 60 63 AA 20 00 00 00 00-                        "`c. ....        "

...

free(0x20b28b20)
free(): we have this block in our records, size=0x140
0x20B28B20: 02 00 00 00 03 00 00 00-04 00 00 00 05 00 00 00 "................"
0x20B28B30: 07 00 00 00 08 00 00 00-0C 00 00 00 0D 00 00 00 "................"
0x20B28B40: 0E 00 00 00 0F 00 00 00-10 00 00 00 11 00 00 00 "................"
0x20B28B50: 12 00 00 00 13 00 00 00-14 00 00 00 15 00 00 00 "................"
0x20B28B60: 16 00 00 00 17 00 00 00-18 00 00 00 1A 00 00 00 "................"
0x20B28B70: 1B 00 00 00 1C 00 00 00-1D 00 00 00 1E 00 00 00 "................"
0x20B28B80: 1F 00 00 00 20 00 00 00-21 00 00 00 22 00 00 00 ".... ...!..."..."
0x20B28B90: 23 00 00 00 24 00 00 00-25 00 00 00 26 00 00 00 "#...\$...%...&..."
0x20B28BA0: 27 00 00 00 28 00 00 00-2A 00 00 00 2B 00 00 00 "'...(...*...+..."
0x20B28BB0: 2C 00 00 00 2E 00 00 00-2F 00 00 00 30 00 00 00 ",......./...0..."
0x20B28BC0: 31 00 00 00 32 00 00 00-33 00 00 00 34 00 00 00 "1...2...3...4..."
0x20B28BD0: 38 00 00 00 39 00 00 00-3A 00 00 00 3B 00 00 00 "8...9...:...;..."
0x20B28BE0: 3C 00 00 00 3D 00 00 00-3E 00 00 00 3F 00 00 00 "<...=...>...?..."
0x20B28BF0: 40 00 00 00 41 00 00 00-42 00 00 00 43 00 00 00 "@...A...B...C..."
0x20B28C00: 44 00 00 00 45 00 00 00-47 00 00 00 48 00 00 00 "D...E...G...H..."
0x20B28C10: 49 00 00 00 4A 00 00 00-4B 00 00 00 4C 00 00 00 "I...J...K...L..."
0x20B28C20: 4D 00 00 00 4F 00 00 00-50 00 00 00 50 00 00 00 "M...O...P...P..."
0x20B28C30: 50 00 00 00 50 00 00 00-50 00 00 00 50 00 00 00 "P...P...P...P..."
0x20B28C40: 50 00 00 00 50 00 00 00-50 00 00 00 50 00 00 00 "P...P...P...P..."
0x20B28C50: 50 00 00 00 00 00 00 00-00 00 00 00 00 00 00 00 "P..............."

...

free(0x20af9cf0)
free(): we have this block in our records, size=0x18
0x20AF9CF0: 43 00 00 00 50 00 00 00-10 00 00 00 20 00 74 00 "C...P....... .t."
0x20AF9D00: 20 8B B2 20 00 00 00 00-                        " .. ....        "
\end{lstlisting}

Nous voyons facilement que les plus gros blocs (avec une taille de 0x28 et 0x140) sont
juste des tableaux de valeurs jusqu'à $\approx$ 0x50.
Attendez... 0x50 est 80 en représentation décimale. et 9*9=81 (configuration standard
de Minesweeper).

Après une rapide investigation, j'ai trouvé que chaque élément 32-bit est en fait
les coordonnées d'une case.
Une case est représentée en utilisant un seul nombre, c'est un nombre dans un tableau-2D.
Ligne et colonne de chaque mine sont décodées comme ceci: \emph{row=n / WIDTH; col=n \% HEIGHT;}

Lorsque j'ai essayé de décoder ces deux blocs les plus gros, j'ai obtenu ces
cartes de case:

\begin{lstlisting}
try_to_dump_cells(). unique elements=0xa
......*..
..*......
.......*.
.........
.....*...
*.......*
**.......
.......*.
......*..

...

try_to_dump_cells(). unique elements=0x44
*.****.**
...******
*******.*
*********
*****.***
.*******.
..*******
*******.*
******.**
\end{lstlisting}

Il semble que le premier bloc soit juste une liste des mines placées, tandis que le
second bloc est une liste des cases libres, mais le second semble quelque peu désynchroniser
du premier, et une version inversée du premier ne coïncide que partiellement.
Néanmoins, la première carte est correcte - nous pouvons jeter un coup d'\oe{}il
dans le fichier de log alors que Minesweeper est encore chargé et presque toutes
les cases sont cachées, et cliquer tranquillement sur les cases marquées d'un point
ici.

Il semble donc que lorsque l'utilisateur clique pour l première fois quelque part,
Minesweeper place les 10 mines, puis détruit le bloc avec leurs liste (peut-être
copie-t-il toutes les données dans un autre bloc avant?), donc nous pouvons les voir
lors de l'appel à free().

Un autre fait: la méthode Array<NodeType>::Add(NodeType) modifie les blocs que nous
avons observé, et est appelée depuis de nombreux endroits, Board::placeMines() incluse.
Mais c'est cool: je ne suis jamais allé dans les détails, tout a été résolu simplement
en utilisant PIN.

Les fichiers: \url{\GitHubMinesweeperURL}.

\subsection{Exercice}

Essayez de comprendre comment le résultat de rand() est converti en coordonnée(s).
Pour blaguer, faite que rand() renvoie des résultats tels que les mines
soient placées en formant un symbole ou une figure.

% TODO: \section{Compiler Pin}

Compiler Pin pour Windows peut s'avérer délicat.
Ceci est ma recette qui fonctionne.

\begin{itemize}

\item Décompacter le dernier Pin, disons, \verb|C:\pin-3.7\|

\item Installer le dernier Cygwin, dans, disons, \verb|c:\cygwin64|

\item Installer MSVC 2015 ou plus récent.

\item Ouvrir le fichier \verb|C:\pin-3.7\source\tools\Config\makefile.default.rules|, remplacer \verb|mkdir -p $@| par \verb|/bin/mkdir -p $@|

\item (Si nécessaire) dans \verb|C:\pin-3.7\source\tools\SimpleExamples\makefile.rules|, ajouter votre pintool à la liste \verb|TEST_TOOL_ROOTS|.

\item Ouvrir "VS2015 x86 Native Tools Command Prompt". Taper:

\begin{lstlisting}
cd c:\pin-3.7\source\tools\SimpleExamples
c:\cygwin64\bin\make all TARGET=ia32
\end{lstlisting}

Maintenant les outils pintools sont dans \verb|c:\pin-3.7\source\tools\SimpleExamples\obj-ia32|

\item Pour winx64, utiliser "x64 Native Tools Command Prompt" et lancer:

\begin{lstlisting}
c:\cygwin64\bin\make all TARGET=intel64
\end{lstlisting}

\item Lancer pintool:

\begin{lstlisting}
c:\pin-3.7\pin.exe -t C:\pin-3.7\source\tools\SimpleExamples\obj-ia32\XOR_ins.dll -- program.exe arguments
\end{lstlisting}

\end{itemize}


\section{Pourquoi ``instrumentation''?}

Peut-être que c'est un terme de profilage de code.
Il y a au moins deux méthodes:
1) "échantillonnage": vous rentrez dans le code se déroulant autant de fois que possible
(des centaines par seconde), et regardez, où en est l'exécution à ce moment;
2) "instrumentation": le code compilé est intercalé avec de l'autre code, qui peut
incrémenter des compteurs, etc.

Peut-être que les outils \ac{DBI} ont hérités du terme?

